\chapter{Conclusion}

A large number of experiments have been performed with the \gls{caes} following the roadmap towards the holy grail of single-shot, ultrafast \gls{cdi} of single molecules which are described in a this thesis and those from previous students~\cite{sheludko_shaped_2010,bell_cold_2011,saliba_cold_2011,mcculloch_generation_2013,taylor_rydberg_2013,tielen_development_2015,murphy_measurement_2017,speirs_electron_2017}.
The research detailed in this thesis decribes the last set of research performed with this iteration of the \gls{caes} and has paved the way for the next iteration.

In Chater~\ref{chapter:polspec} laser frequency stablisation using \gls{ps} was demonstrated with impressive linewidth reduction, narrowing \gls{ecdl} \gls{rms} linewidth to \unit[600]{Hz} and that of cateye lasers to \unit[360]{Hz}, significantly lower than previously shown with this technique and previously only achieved with high-finesse optical cavities.
The long-term frequency drift was only \unit[51]{kHz}, easily sufficient for most laser cooling experiments and again much lower than previously reported.
Even greater improvements to laser frequency linewidth and drift is expected if the noise in the system can be reduced.
\Gls{ps} could also prove useful as an absolute reference for high-performance frequency stabilisation using an optical cavity for high-lying Rydberg excitation.

The \gls{caes} was originally intended to be a high-brightness electron source with similar characteristics to photocathode sources.
The finely controlled ionisation method allows for lower source temperatures and thus higher potential brightness compared to those achieable with thermionic sources.
The low emittance of the \gls{caes} has been demostrated multiple times along with the \unit[10]{K} source temperature, which is much lower than the temperatures typical of electrons originating from solid photocathodes.

\Glspl{caes} are progressing well along the path towards structural determination with single-shot ultrafast \gls{cdi} but there is still a long way to go.
The \gls{caes} described in Chapter~\ref{chapter:setup} was capable of single-shot diffraction through monocrystalline gold and ultrafast diffraction (see Chapter~\ref{chapter:diffraction}) through gold but due to the beam current contraints by this implementation it was not possible to achieve both single-shot and ultrafast imaging.
In Section~\ref{section:pulse_vs_continuous} some strategies for increasing the beam current were investigated; continuous source operations, rubidium oven temperature, and laser intensity.
These investigations indicated that the main limiting factor with this system was the intensity of the blue ionisation laser system as it was far from saturating the atoms available.
A blue laser with enough power to easily saturate the ionisation process would go a long way to providing the flux for \gls{cdi} and single-shot, ultrafast diffraction.

The beam quality from the source was also investigated and improved with the characterisation and correction of astigmatism described in Section~\ref{section:quadrupole}.
Improvements to beam quality allow for the maximisation of signal and have similar effect to improving the beam current.
There remain many issues with the quality of the beam from the \gls{caes} that could be fixed given years of work and resources.
Fortunately this work has already been done within the mature field of electron and ion microscopy and `all' that is required to piggyback off this is to acquire an existing microscope column and set it up with a \gls{caes} as a source.

Another issue with this \gls{caes} was the instabilities in the electron trajectories.
Electron beam drift was a major issue complicating the measurement presented in this thesis and the registration technique described in Section~\ref{section:registration} was essential to working around the beam drift and extracting signal from the multi-shot diffraction and emittance measurements.
The beam drift is attributed to a number of things such as the high-power switches, step-down transformers and the steel of the vacuum system itself all of which are exacerbated by the long distance from the source to the detector.
If the \gls{caes} was to be redesigned from scratch then there are a large number of details that could be designed differently to minimise or elimate the effects of drift, such as a shorter propagation distance, mu-metal vacuum components, minimising the volume of steel near the exmperiment and ensuring that high-power devices were shielded or far away.

Another of the interesting features of a \gls{caes} is the beam shaping capabilites which are especially interesting when considering shace charge.
Space charge degrades the quality, or emittance, of charged particle beams but uniform ellipsoidal profile bunches have linear interactions which can be undone, thus preserving the beam emittance.
In Chapter~\ref{chapter:brightness} a new emittance measurement technique is discussed which uses streaked pepperpots to determine the time-resolved emittance and brightness of a beam.
The technique was shown to match up well with theoretical predictions and simulations despite some experimental constraints limiting the implementation.
This technique should prove useful when quantifying the efficacy of techniques for the reduction of beam degradation due to space charge.
The technique is generally applicable to charged particle beams and could prove useful in contexts other than \glspl{caes}.

Next iteration of CAES