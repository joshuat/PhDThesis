\chapter{Conclusion}

A large number of experiments have been performed with the \gls{caeis} following the road map towards the holy grail of single-shot, ultrafast \gls{cdi} of single molecules which are described in a this thesis and those from previous students~\cite{sheludko_shaped_2010,bell_cold_2011,saliba_cold_2011,mcculloch_generation_2013,taylor_rydberg_2013,tielen_development_2015,murphy_measurement_2017,speirs_electron_2017}.
The research detailed in this thesis describes the last set of research performed with this iteration of the \gls{caeis} and has paved the way for the next iteration.

In Chapter~\ref{chapter:polspec} laser frequency stabilisation using \gls{ps} was demonstrated with impressive linewidth reduction, narrowing \gls{ecdl} \gls{rms} linewidth to \unit[600]{Hz} and that of cat-eye lasers to \unit[360]{Hz}, about two orders of magnitude lower than previously shown with this technique and previously only achieved with high-finesse optical cavities.
The long-term frequency drift was only \unit[51]{kHz}, easily sufficient for most laser cooling experiments and again much lower than previously reported.
Even greater improvements to laser frequency linewidth and drift is expected if the noise in the system can be reduced.
\Gls{ps} could also prove useful as an absolute reference for high-performance frequency stabilisation using an optical cavity for high-lying Rydberg excitation.

The \gls{caeis} was originally intended to be a high-brightness electron and ion source with similar characteristics to photocathode sources.
The finely controlled ionisation method allows for lower source temperatures and thus higher potential brightness compared to those achievable with thermionic sources.
The low emittance of the \gls{caeis} has been demonstrated multiple times along with the \unit[10]{K} source temperature, which is much lower than the temperatures typical of electrons originating from solid photocathodes.

\Glspl{caes} are progressing well along the path towards structural determination with single-shot ultrafast \gls{cdi} but there is still a long way to go.
The \gls{caeis} described in Chapter~\ref{chapter:setup} was capable of single-shot diffraction through monocrystalline gold and ultrafast diffraction (see Chapter~\ref{chapter:diffraction}) through gold but due to the beam current constraints by this implementation it was not possible to achieve both single-shot and ultrafast imaging.
In Section~\ref{section:pulse_vs_continuous} some strategies for increasing the beam current were investigated; continuous source operations, rubidium oven temperature, and laser intensity.
These investigations indicated that the main limiting factor with this system was the intensity of the blue ionisation laser system as it was far from saturating the atoms available.
A blue laser with enough power to easily saturate the ionisation process would go a long way to providing the flux for \gls{cdi} and single-shot, ultrafast diffraction.

The beam quality from the source was also investigated and improved with the characterisation and correction of astigmatism described in Section~\ref{section:quadrupole}.
Improvements to beam quality allow for the maximisation of signal and have similar effect to improving the beam current.
There remain many issues with the quality of the beam from the \gls{caes} that could be fixed given years of work and resources.
Fortunately this work has already been done within the mature field of electron and ion microscopy and `all' that is required to piggyback off this is to acquire an existing microscope column and set it up with a \gls{caeis} as a source.

Another issue with this \gls{caes} was the instabilities in the electron trajectories which were investigated in Section~\ref{section:stability}.
Electron beam drift was a significant issue complicating the measurements presented in this thesis and the registration technique described in Section~\ref{section:registration} was essential to working around the beam drift and extracting signal from the multi-shot diffraction and emittance measurements.
The beam drift had previously been attributed to the fast switching of the high-current magnetic coils for the magneto-optic trapping however the exploration of the performance of continuous operations revealed that the absence of the coils had little effect on the observed beam drift.
The beam drift can be attributed to a number of other effects such as the high-power switches, step-down transformers and the steel of the vacuum system itself all of which are exacerbated by the long distance from the source to the detector.
The sub-optimal design of the apparatus is partially the result of the general purpose design that allows for a wide range of investigations from atom-laser interactions and high-precision spectroscopy to electron and ion beam experiments.
Many lessons have been learnt during the years of design and operation of this \gls{caeis} and if it was to be redesigned from scratch then there are a large number of details that could be improved to minimise or eliminate beam drift, such as a shorter propagation distance, mu-metal vacuum components, minimising the volume of steel near the experiment and ensuring that high-power devices were shielded or far away.

Another of the interesting features of a \gls{caeis} is the beam shaping capabilities which are especially interesting when considering space charge.
Space charge degrades the brightness of charged particle beams but uniform ellipsoidal profile bunches are unique in that they preserve beam brightness.
The beam shaping capabilities of \glspl{caeis} allows the production of ellipsoidal bunches if the shaping is implemented with three dimensions of control, rather than the two-dimensional proof-of-concept implementation this apparatus used.
In Chapter~\ref{chapter:brightness} a new emittance measurement technique is discussed and demonstrated which uses streaked pepperpots to determine the time-resolved brightness of a beam.
The accuracy of the technique was demonstrated with comparison to theoretical predictions and simulations over two dramatically different timescales.
This technique should prove useful when quantifying the efficacy of techniques for the reduction of beam degradation due to space charge.
While the implementation of this technique was somewhat constrained by the geometry and current of the source the technique is generally applicable to charged particle beams and could prove useful in contexts other than \glspl{caeis}.

It had been hoped that the \gls{caes} could be used to demonstrate \gls{cdi} with relatively simple samples however the combination of beam drift and the low current achievable made this impossible.
If the beam trajectories had been stable then \gls{cdi} could have been performed with extremely long measurements that averaged many thousands of single-shots.
The the beam current had been larger then there would have been enough signal in a single-shot that the registration algorithms would have been able to compensate for the drift and average out multiple shots.
Unfortunately with the apparatus as it was the effort and cost required to achieve either more stability or current would not have been justified.
A more feasible option is to await the next generation of \gls{caes} with its design informed by the experience gained over the lifetime of the apparatus used in the research described in this thesis.

The \gls{caeis} used in for this research has reached the end of its useful life.
The apparatus has been very productive during its lifetime reaching a number of milestones along the road to molecular imaging, low-temperature electron and ion production~\cite{saliba_spatial_2012,mcculloch_high-coherence_2013,mcculloch_field_2017,speirs_identification_2017}, arbitrary beam shaping~\cite{mcculloch_arbitrarily_2011}, space-charge observation and manipulation~\cite{murphy_detailed_2014,murphy_increasing_2015,thompson_suppression_2016}, and demonstrations of ultrafast diffraction~\cite{speirs_single-shot_2015}.

It still remains to be seen if electrons generated from the photoionisation of laser cooled atoms can be used to perform ultrafast, single-shot coherent diffractive imaging of arbitrary molecules, let alone if such a source would be competitive with its rivals such as \glspl{xfel} or photocathode electron sources.
Fortunately, due to the high brightness of the source, the technique also shows promise as an ion source for potential use with ion microscopes and focused ion beam milling.

The lessons learnt during the cause of these investigations have informed the design of the next iteration of \gls{caeis} at the University of Melbourne with the aim of creating a more reliable, higher-current source that takes advantage of a modern electron or ion microscope column.
The first portion of this source has already been constructed and used to examine electron-ion coincidence measurements to permitting heralded ions~\cite{mcculloch_heralded_2018}.