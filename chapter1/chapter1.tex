\chapter{Polarisation Spectroscopy}
\pagenumbering{arabic}
\setcounter{page}{1}

\section{Literature Review}

Laser frequency stabilisation is an essential tool for atomic physics experiments, without it experiments such as \gls{bec} and atomic clocks would not be possible.
There are a large number of techniques available for laser frequency stabilisation with numerous advantages and disadvantages among them.

\Gls{ps} is one such technique that will be discussed in detail here.
\Gls{ps} was first described by Wieman and H\"anch in 1976 as: ``a sensitive method of Doppler-free spectroscopy, monitoring the nonlinear interaction of two monochromatic laser beams in an absorbing gas via changes in light polarisation."\cite{wieman_doppler-free_1976}
A detailed discussion of physics of \gls{ps} can be found in section \ref{section:pol_spec_theory}.

\subsection{Pol Spec origin}
\subsection{Pol Spec developements}
\section{Comparison with other techniques}
\subsection{Sat Abs}
\subsection{PDH}
\subsection{other modulation based techniques}
\section{Theory}\label{section:pol_spec_theory}
\subsection{Basic Theory}
\subsection{Fast Theory}
\subsection{OBEs (timescale of state evolution, step in simulating)}
\section{Developements}
\subsection{Balanced Polarimeter}
\subsection{Beam splitter to co-propagate}
\subsection{High bandwidth feedback}
\subsection{Lincoln's magic detectors?}
\section{Experimental Setup (with details - fibres, calcite, etc.)}
\section{Measurement Techniques}
\subsection{Self-heterodyne}
\subsection{Heterodyne}
\subsection{Noise measurements}
\subsection{Side of peak (include basic cavity theory here?)}
\subsection{Integration}
\subsection{Drift}
\subsection{Bandwidth Stuff}
