\chapter{Polarisation Spectroscopy}

\section{The Plan}
So here's the first go at a plan:
\begin{enumerate}
    \item Literature Review
    \begin{itemize}
        \item Pol Spec origin
        \item Pol Spec developements
    \end{itemize}
    \item Comparison with other techniques
    \begin{itemize}
        \item Sat Abs
        \item PDH
        \item other modulation based techniques
    \end{itemize}
    \item Theory
    \begin{itemize}
        \item Basic Theory
        \item Fast Theory
        \item OBEs (timescale of state evolution, step in simulating)
    \end{itemize}
    \item Developements
    \begin{itemize}
        \item Balanced Polarimeter
        \item Beam splitter to co-propagate
        \item High bandwidth feedback
        \item Lincoln's magic detectors?
    \end{itemize}
    \item Experimental Setup (with details - fibres, calcite, etc.)
    \item Measurement Techniques
    \begin{itemize}
        \item Self-heterodyne
        \item Heterodyne
        \item Noise measurements
        \item Side of peak (include basic cavity theory here?)
        \item Integration
        \item Drift
        \item Bandwidth Stuff
    \end{itemize}
\end{enumerate}

\section{Introduction}

Laser frequency stabilisation is an essential tool for atomic physics experiments, without it experiments such as \gls{bec} and atomic clocks would not be possible. There are a large number of techniques available for laser frequency stabilisation with numerous advantages and disadvantages among them.

\Gls{ps} is one such technique that will be discussed in detail here.  \Gls{ps} was first described by Wieman and H\"anch in 1976\cite{wieman_doppler-free_1976}
