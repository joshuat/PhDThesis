\chapter*{Abstract}
\addcontentsline{toc}{chapter}{Abstract}

The understanding of atomic structures and processes is continually improving with the great technological development in imaging techniques.
Ultrafast electron and X-ray techniques are able to perform measurements at atomic lengths and timescales and both these techniques require the generation of high-brightness ultrashort-duration electron bunches.
Electron imaging techniques directly use these short bright bunches and in \glspl{xfel} the electron bunches are used to generate short bright bunches of X-rays.

It is hoped that \glspl{caeis} will also be able to produce ultrashort high-brightness electron bunches that are brighter than conventional sources.
\Glspl{caeis} generate electrons via near-threshold ionisation from an ultracold atomic gas and have been shown to create electrons bunches with temperature as low as \unit[10]{K}.
Conventional photocathode sources have temperatures of thousands of Kelvin and, as brightness is proportional to the temperature of the source, \glspl{caeis} have the potential to produce much brighter electron bunches.
\Gls{caeis} are also capable of producing extremely cold ion bunches and show great promise as an ion source for ion milling and microscopy.

This thesis describes a number of developments involved with the \gls{caeis} project at the University of Melbourne, in particular pushing the boundaries of laser frequency stabilisation to allow for precise selection of atomic states for cooling and ionisation, and a new technique for measuring the brightness of charged particle bunches.

Laser frequency stabilisation is an essential component of the \gls{caeis} and many other applications including metrology, spectroscopy and laser cooling.
Polarisation spectroscopy is a commonly used technique for laser frequency stabilisation but the full measurement and control bandwidth has not previously been demonstrated.
Here it is shown that the bandwidth is sufficient to not only stablise the frequency of the laser, but also to reduce the laser linewidth to much less than \unit[1]{kHz}, two order, two orders of magnitude better than previously reported.
This demonstration provides a new approch for precisely accessing the high-lying Rydberg-levels of atoms, if used in conjunction with cavity based frequency locking methods, allowing for a greater exploration of the ionisation methods involved in a \gls{caeis}.

Brightness is the most comprehensive figure of merit for charged particle beams and a new technique for measuring the brightness with sub-nanosecond time-resolution is presented.
The technique achieves time-resolved brightness measurements by streaking one-dimensional pepperpot measurements across the detector.
Time-resolved brightness measurements have the potential to reveal information related to the ionisation processes used in \glspl{caeis} and can show the efficacy of techniques used to counter the effects of space charge in the beams produced from a \gls{caeis}.

The performance of the \gls{caeis} apparatus operating in its normal pulsed mode is compared to continuous operation with emphasis on the beam current and electron trajectory stability.
The beam quality was also improved by identifying an astigmatism in the beam and correcting it with a 3D printed quadrupole lens.
Virtually all the beam measurements presented here utilise image processing techniques that allow for multi-shot averaging despite instabilities in the electron beam trajectories.

This iteration of a \glspl{caeis} was able to produce ultrashort-duration electron bunches and these have been used to demonstrate ultrafast electron diffraction from thin gold foils.
This is an important step along the path to being able to perform ultrafast single-shot \gls{cdi} and the next iteration of the \gls{caeis} should have sufficient current to demonstrate diffraction that is both single-shot and ultrafast.