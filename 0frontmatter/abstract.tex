\chapter*{Abstract}
\addcontentsline{toc}{chapter}{Abstract}

The understanding of atomic structures and processes has been greatly enhanced by the developments in imaging techniques.
Ultrafast electron and X-ray techniques are able to perform measurements at atomic lengths and timescales both these techniques require the generation of high-brightness ultrashort-duration electron bunches.
Electron techniques directly use this short bright bunches and electron bunches are used to generate short bright X-ray bunches in \glspl{xfel}.

It is hoped that \glspl{caeis} will also be able to produce ultrashort high-brightness electron bunches that are brighter than current sources due to the unique method of generating electrons used by these sources.
\Glspl{caeis} generate electrons via near-threshold ionisation from an ultracold atom gas and has been shown to create electrons bunches with temperature as low as \unit[10]{K}.
Conventional photocathode sources have temperatures of thousands of Kelvin and thus \glspl{caeis} have the potential to produce brighter electron bunches.
\Gls{caeis} are also capable of producing extremely cold ion bunches and shows great promise as an ion source for ion milling and microscopy.

This thesis describes a number of developments involved with the \gls{caeis} project, in particular pushing the boundaries of laser frequency stabilisation, a new technique for measuring the brightness of charged particle bunches.
A number of small developments and investigations are presented along with the newest diffraction measurements.

Laser frequency stabilisation is an essential component of the \gls{caeis} and many other applications including metrology, spectroscopy and laser cooling.
The high-bandwidth of the laser stabilisation technique polarisation spectroscopy is utilised, along with high-bandwidth feedback, to narrow laser linewidth to less than \unit[1]{kHz}, which is two orders of magnitude better than previously reported.

Brightness is the most comprehensive figure of merit for charged particle beams and a new technique for measuring the brightness with time-resolution is presented.
The technique achieve time-resolution by streaking one-dimensional pepperpot measurements across the detector.
Time-resolved brightness measurements have the potential to reveal information related to the ionisation processes used in \glspl{caeis} and can show the efficacy of techniques used to counter the effects of space charge in the beam.

The performance of the \gls{caeis} apparatus operating in pulsed mode is compared to continuous operation with a focus on the beam current and electron trajectory stability.
The beam quality was also improved by identifying an astigmatism in the beam and correcting it with a 3D printed quadrupole lens.
Virtually all the beam measurements presented here utilise the image processing techniques that allow for multi-shot averaging despite the inconvenient instabilities in the electron beam trajectories.

This iteration of a \glspl{caeis} was able to produce ultrashort-duration electron bunches and these have been used to demonstrate ultrafast electron diffraction from thin gold foils.
This is an important step along the path to being able to perform ultrafast single-shot \gls{cdi} and next iteration of the \gls{caeis} should have sufficient current to demonstrate \emph{single-shot} ultrafast diffraction and then move on to \gls{cdi}.