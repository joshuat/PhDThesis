\part{Coherent Diffractive Imaging with a Cold-Atom Electron Source}

Chapters should be split in to separate files when I'm satisfied with the vague plan.

\chapter{Introduction}

\section{Diffractive Imaging}

\subsection{Crystallography etc}

\section{Coherent Diffractive Imaging}

\section{Cold-Atom Electron Sources}




\chapter{Cold-Atom Electron Source}

\section{Properties}

\section{How it works}

\section{Current Limitations}

\section{Pulsed vs Continuous}

\subsection{Oven Temperature to Electron Count}

\section{Stability}

\section{Source Characterisation}

\subsection{Emittance Measurements}

\subsection{Streaked emittance}

\subsection{Coherence}

\subsection{Noise characterisation}

\section{Future Ideal source}


\chapter{Ultrafast Diffractive Imaging}

\section{Theory}

\section{Why ultrafast?}

\section{How does CAES do it?}

\section{Results}

\subsection{Gold}

\subsection{Aluminium}

\subsection{Graphene}

\subsection{Other Stuff}

\section{Why don't all our samples work?}


\chapter{Coherent Diffractive Imaging}

\section{Theory}

\section{Geometry}

\section{Phase Retrieval}

\section{Simulation}

\section{Samples}

\subsection{Fabrication}

\section{Hurray it worked}

\section{Why flux was a problem and is now the limiting factor}



\chapter{Conclusion}

We were able to demonstrate CDI.

Where do we go from here?

\section{New source}

\section{What can the old source investigate?}


