\chapter{Cold-Atom Electron Source}\label{chapter:setup}

The \gls{caeis} at the University of Melbourne is a source of low temperature electrons or rubidium ions with promising potential as a alternative charge particle source.
The \gls{caeis} works by carefully ionising rubidium atoms trapped in a \gls{mot} in order to generate a low temperature plasma which can then be accelerated to form a particle beam.
The apparatus described here is reaching the end of it's useful life, greater understanding of the limitations imposed by this implementation of a \gls{caeis} and the already impressive developments achieved with this source have paved the way for the next generation of \gls{caeis} which is not compatible with the apparatus.
Numerous doctoral students have worked on this system and the design and construction details can be found in their theses~\cite{theses}.

This chapter...

\section{Properties}

Cold

High coherence

Low emittance

Shaping

Reversal of space charge expansion

Ions or electrons

Applicable to any atom that can be trapped.


\section{How it works}

\label{section:two_stage_ionisation}
\label{section:excess_energy}
\label{section:pulse_blaster}

\section{Current Limitations}

\section{Pulsed vs Continuous}

\subsection{Oven Temperature to Electron Count}

\section{Stability}\label{section:stability}

\section{Source Characterisation}

\subsection{Astigmatism}

\subsubsection{Quadrupole Correction}\label{section:quadrupole}

\subsection{Emittance Measurements}

\subsection{Streaked emittance}

\subsection{Coherence}

\subsection{Noise characterisation}

\section{Future Ideal source}