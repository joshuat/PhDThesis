\chapter{Cold-Atom Electron and Ion Source}\label{chapter:setup}

The \gls{caeis} at the University of Melbourne is a source of low temperature electrons or rubidium ions with promising potential as a alternative charge particle source.
The \gls{caeis} works by carefully ionising rubidium atoms trapped in a \gls{mot} in order to generate a low temperature plasma which can then be accelerated to form a particle beam.
The apparatus described here is reaching the end of it's useful life, greater understanding of the limitations imposed by this implementation of a \gls{caeis} and the already impressive developments achieved with this source have paved the way for the next generation of \gls{caeis} which is not compatible with the apparatus.
Numerous doctoral students have worked on this system and the design and construction details can be found in their theses~\cite{sheludko_shaped_2010,bell_cold_2011,saliba_cold_2011,mcculloch_generation_2013,murphy_measurement_2017,speirs_electron_2017}.

This chapter provides an overview of the \gls{caeis} and some of the investigations into source stability, source current and beam optimisation.

\section{Description}
In order to generate electrons and ions the University of Melbourne \gls{caeis} began by trapping and cooling atoms in a \gls{mot} that was loaded from a Zeeman slower.
The optical and magnetic trapping fields were then extinguised and the ground-state atoms carefully ionised using a combination of a red excitation laser and a blue ionisation laser.
The charged particles are accelerated by a static electric produced by the accelerators electrodes, one polarity accelerating ions towards the detector and the other electrons.

When the source is acting as an electron source the it is essential to turn off the \gls{mot} and Zeeman slower magnetic fields before ionisation due to the significant deviation that the fields cause to the electron trajectories.
Ion trajectories are not adversly affected if the fields are left on due to the much higher ion mass.

\subsection{Rubidium Oven}
The source begins with an effusive Rubidium oven with a long heated collimation tube.
Typically effusive ovens are wasteful with large numbers of atoms lost into a large solid angles however this experimented makes use of a long heated collimation tube to collect and re-emit atoms that were initially emitted at high angles.
These atoms are re-emitted back to the reservoir or into the collimated atom beam leaving the collimation tube.
The rubidium reservoir was typically heated to \unit[80]{$^\circ$C} and the collimation tube to \unit[120]{$^\circ$C}.
A brief schematic of the oven can be found to the left in Figure~\ref{figure:zeemanoven} and more detail on the design, operation, and performance of the oven can be found in References~\cite{bell_slow_2010} and \cite{bell_cold_2011}.

\begin{figure}
    \center
    \includegraphics[width=145mm]{part2/Figs/ZeemanOven.pdf}
    \caption{A schematic of the rubidium atom source. Atomic vapour from the oven is directed into the Zeeman slower where a laser detuned from the atomic resonance (shown in red), in combination with a tapered magnetic coil (blue and green), with a magnetic field as shown, slows the thermal atoms.}
    \label{figure:zeemanoven}
\end{figure}

\subsection{Zeeman Slower}
Following the oven is the tapered pitch Zeeman slower which slows atoms down so that they can be captured by the trapping fields of \gls{mot}~\cite{bell_slow_2010}.
Zeeman slowers operate by using a laser red-detuned from resonance to slow the atoms down however as the atoms slow the conditions for resonance change due to the changing Doppler shift.
The solution to this quandrary used here is a tapered magnetic coil to apply a magnetic field to shift the atomic resonance such that a particular velocity class of atoms remains resonant with the light field, and thus is slowed, along the length of the Zeeman slower.
Atoms leaving the Zeeman slower typically had velocities around \unit[35]{m/s}, well within the capture velocity of the \gls{mot}.
A schematic of the Zeeman slower with along with the magnetic field produced by the tapered coil is shown in Figure~\ref{figure:zeemanoven}.

When extracting electrons from the \gls{mot} the magnetic coil must be turned off to prevent disruptions to the electron trajectory.

\subsection{Magneto-Optic Trap}
\Glspl{mot} use a combination of magnetic and light fields to trap and cool atoms to $\muup$K temperatures.
In the \gls{caeis} the \gls{mot} was formed from six counter propagating \unit[780]{nm} lasers in a a retro-reflective quasi-mirror \gls{mot} configuration~\cite{hanssen_using_2006,mcculloch_generation_2013} as shown in Figure~\ref{figure:mot} to allow for the accelerator structure consisting of transparent and reflective electrodes.
Both accellerator electrodes were approximately \unit[11]{cm} in diameter and \unit[4]{mm} thick with the transmissive plate being \gls{ar} coated at \unit[780]{nm} and coated with indium-tin-oxide which is also transmisive to \unit[780]{nm} to 96\%.
The second reflective electrode was coposed of copper coated with polished gold.s
The magnetic component of the \gls{mot} was formed from two magnetic coils in an anti-Helmholtz configuration providing a zero-field region in the centre of the trap.

The \gls{mot} trapping lasers were detuned \unit[$-10$]{MHz} from the 5S$_{1/2}$(F=3) $\rightarrow$ 5P$_{3/2}$(F$^\prime$=4) Rb85 transition and were mixed with on resonant 5S$_{1/2}$(F=2) $\rightarrow$ 5P$_{3/2}$(F$^\prime$=3) light to pump atoms that fell into the dark F=2 state.

{\color{red}Reference to section showing laser setup.}

\begin{figure}
    \center
    \includegraphics[width=145mm]{part2/Figs/MOTdiagram.pdf}
    \caption{A diagram of the magneto-optic trap, ionisation lasers and accelerator structures.}
    \label{figure:mot}
\end{figure}

\subsection{Ionisation}
Rubidium has a ground state ionisation threshold of \unit[4.18]{eV} which can be generated using a one blue and one red photon.
The \gls{caeis} has two options for the generation of blue light and two for red.
Red light could be generated by light from a \gls{cw} diode laser applified by a \gls{ta} and locked to the Rb85 5S$_{1/2}$(F=3) $\rightarrow$ 5P$_{3/2}$(F$^\prime$=4) cycling transition or it could be generated with a mode-locked Ti:sapphire amplified pulsed laser with a wavelength range of \unit[770]{nm} to \unit[830]{nm} and a minimum pulse length of \unit[35]{fs}.
Blue light could be generated with a tunable dye laser that produced \unit[460 to 490]{nm} light with a \gls{fwhm} duration of \unit[5]{nm} or with a \gls{cw} laser generated by a high-power tunable frequency-double diode laser.

There were a number of pathways available to ionise the atoms trapped in the \gls{mot} each with different bunch temperatures and duration\cite{speirs_identification_2017,speirs_electron_2017}.

Sequential ionisation utilised a single red photon to excite from the ground state to an intermediate excited state followed by a single blue photon to transition to a field-ionising state or to the ionisation continuum.
The bunch duration was determined by the shortest of; the duration of the laser pulse driving the transition from the exited state to the ionising state, the lifetime of the intermediate state, or by the depletion time of the intermediate state.

Multiphoton excitation happens when the laser intensities are high enough for nonlinear optical transitions to occur which was the case when the pulse lasers were tightly focused into the atom cloud.
In multiphoton excitation two or more photons are absorbed without transitioning via a real intermediate state.
If $n$ is the number of photon absorbed for the atom to reach its final ionised state then the transition rate is proportional to the $n$th power of the optical intensity~\cite{joachain_atoms_2011}.
Due to the short lifetime of the virtual intermediate states the bunch duration is determined by the duration of the laser pulses.
Multiphoton excitation can occur with just with photons of one colour or with two colour of photons.

Reonsance-enhanced multiphoton excitation occurs when a there is a combination of sequential excitation and multiphon excitation.
Here a number of photons are absorbed to excite the atom to a real intermediate state followed by more photon being absorbed to transition to the final state.
As less photons are required for each transision the overall transition rate can be much higher when compared to plain multiphoton excitation.

\begin{figure}
    \center
    \includegraphics{part2/Figs/ionisationmodes.pdf}
    \caption{A number of photoexcitation pathways were possible in the presence of high intensity illumination by the red a blue lasers in the \gls{caes} such as sequential excitation (SE), multiphoton excitation (MPE), resonance-enhanced multiphoton excitation (REMPE), and two-colour multiphoton excitation (TCMPE). TCMPE is the only pathway that produces bunches that are cold and ultrashort. The images show the transverse momentum distributions for the detected bunches.}
    \label{figure:ionisation_modes}
\end{figure}

The temperature of particles generated from the \gls{caes} depends primarily on the excess ionisation energy given to the atoms by the absorbed photons.
Due to the complex orbits of electrons in high-lying states the relationship between absorbed photon energy and temperature is complex~\cite{mcculloch_high-coherence_2013} but it is generally true that with greater photon energy comes greater source temperature.
The classical ionisation threshold is lower in the presence of electric field, such as the accelerating field in the \gls{caeis}, due to the Stark-shift and the excess energy of an ion or electron relative to the classical ionisation threshold is:
\begin{equation}\label{equation:ionisation_energy_stark}
\Delta E = -E_I + 2\sqrt{ke^3F} + \sum_{i=0}^{n}{\frac{hc}{\lambda_i}},
\end{equation}
where $E_I=4.18\,meV$ is the ground state ionisation energy of rubidium-85, $k$ is the Coulomb constant, e is the elementary charge, $F$ is the strength of the electric field, $h$ is the Plank constant, and $c$ is speed of light.
The second term represent the Stark-shift of the classical ionisation threshold, and the last term is the sum of the energy of the $n$ photons involved in the ionisation, with wavelength $\lambda_i$.
Equation~\ref{equation:ionisation_energy_stark} assumes that rubidium is hydrogen-like which is a good approximation as long as $E_I \gg 2\sqrt{ke^3F}$.

Sequential excitation and two-colour multiphoton excitation are the only ionisation processing that producce cold electrons as the excess ionisation energy is minised by tuning the photon energy.
This can be seen in Figure~\ref{figure:ionisation_modes} in the momentum distributions.

Due to the tunability of the lasers involved in generating cold electrons the it was possible to directly ionise regardless of the electric field which resulted in short duration bunches, field-ionise such that the majority of the bunch had a short duration but some particles continued to tunnel out over a longer duration, or to excite the atoms such that tunneling through the Stark-shifted potential was the only route to ionisation resulting in lower bunch current and long duration.
The short bunch duration depended on the duration of the laser pulses, generally either the ultrafast bunches with less than \unit[320]{ps} duration using the red femtosecond pulse laser, or \unit[5]{ns} long bunches with \gls{cw} red and the \unit[5]{ns} pulsed blue laser.
Long duration bunch length depended on the lifetime of the atomic states involved and tended to be of order \unit[10]{$\muup$s}.

\subsubsection{Beam Shaping}

\subsubsection{Ultrafast Laser}

\subsection{Accelerator}

\subsection{Electron Optics}

\subsection{Sample Management}

\subsubsection{Sample Holder}

\subsubsection{Samples}

\subsection{Detector}

\section{Properties}

The primary advantage of \glspl{caeis} over alternative sources is the low transverse temperature of the particles produced.
Electrons produced from the sources can have temperatures as low as \unit[10]{K} which is extrememly low when compared to other electron sources such as photocathode sources (\unit[$10^3$-$10^4$]{K}~\cite{claessens_ultracold_2005}).

Cold

High coherence

Low emittance

Shaping

Reversal of space charge expansion

Ions or electrons

Applicable to any atom that can be trapped.


\section{How it works}

\label{section:two_stage_ionisation}
\label{section:excess_energy}
\label{section:pulse_blaster}

\section{Current Limitations}

\section{Pulsed vs Continuous}

\subsection{Oven Temperature to Electron Count}

\section{Stability}\label{section:stability}

\section{Source Characterisation}

\subsection{Astigmatism}

\subsubsection{Quadrupole Correction}\label{section:quadrupole}

\subsection{Coherence}

\subsection{Noise characterisation}

\section{Future Ideal source}