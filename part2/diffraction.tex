\chapter{Ultrafast Electron Diffractive Imaging}\label{chapter:diffraction}

One of the central motivations for \gls{caeis} is their potential use as an electron source for use with ultrafast electron diffractive imaging~\cite{speirs_single-shot_2015,van_der_geer_ultracold_2009}.
If care is taken with the ionization pathways the \glspl{caes} are able to produce ultrafast bunches of cold electrons~\cite{speirs_single-shot_2015,speirs_identification_2017,speirs_electron_2017} and with improvements to the bunch current the next generation of these sources will have the potential to be able to realise single-shot ultrafast diffractive imaging of small biomolecules~\cite{mcculloch_cold_2016}.

Previous results have shown that diffraction from large crystalline samples is possible using traditional crystallographic techniques~\cite{speirs_single-shot_2015} and this chapter describes extensions to the results shown in References~\cite{speirs_single-shot_2015} and \cite{speirs_electron_2017}, namely achieving \emph{single-shot} ultrafast diffractive imaging and the demonstration of diffraction from aluminium by using a additional voltage bias applied to the sample holder to achieve the required beam energy.

\section{Crystallography}

Crystallography refers the the science of diffractive imaging from crystals and has been studied for over \unit[100]{years}, being the subject of the 1915 Nobel prize in physics~\cite{bragg_structure_1913}.
Crystallography has been an highly productive science having produced results such as the structure of DNA~\cite{franklin_structure_1953,dennis_eternal_2003}, and structural determination of biomolecules~\cite{longchamp_how_2015} {\color{red} more references}.
Crystallographic techniques have been under constant development and refinement since their inception.
The majority of crystallography to date has been performed using X-rays however relatively recent developements have utilised electrons as electron diffraction techniques are more versatile than those of X-rays~\cite{cowley_electron_1992}.

Due to the mature understanding of crystallographic techniques, crystallography is ideal for the first steps in demonstrating the capabilities of \glspl{caes}.
While the \gls{caes} is able to operate in \gls{cw} mode the performance with this apparatus has been optimised for pulse mode (see Section~\ref{section:pulse_vs_continuous}) and thus the results described in this chapter have been taken with pulse bunches of electrons.

\subsection{Theory}

\section{Why ultrafast?}

\section{How does CAES do it?}

\section{Sample Bias}\label{section:sample_bias}

\section{Results}

\subsection{Gold}

\subsection{Aluminium}

\subsection{Graphene}

\subsection{Other Stuff}

\section{Why don't all our samples work?}