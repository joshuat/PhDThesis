\chapter{Ultrafast Electron Diffractive Imaging}\label{chapter:diffraction}

One of the central motivations for \gls{caeis} is their potential use as an electron source for use with ultrafast electron diffractive imaging~\cite{speirs_single-shot_2015,van_der_geer_ultracold_2009}.
If care is taken with the ionization pathways the \glspl{caes} are able to produce ultrafast bunches of cold electrons~\cite{speirs_single-shot_2015,speirs_identification_2017,speirs_electron_2017} and with improvements to the bunch current the next generation of these sources will have the potential to be able to realise single-shot ultrafast diffractive imaging of small biomolecules~\cite{mcculloch_cold_2016}.

Previous results have shown that diffraction from large crystalline samples is possible using traditional crystallographic techniques~\cite{speirs_single-shot_2015} and this chapter describes extensions to the results shown in References~\cite{speirs_single-shot_2015} and \cite{speirs_electron_2017}, namely achieving \emph{single-shot} ultrafast diffractive imaging and the demonstration of diffraction from aluminium by using a additional voltage bias applied to the sample holder to achieve the required beam energy.

\section{Crystallography}

Crystallography refers the the science of diffractive imaging from crystals and has been studied for over \unit[100]{years}, being the subject of the 1915 Nobel prize in physics~\cite{bragg_structure_1913}.
Crystallography has been an highly productive science having produced results such as the structure of DNA~\cite{franklin_structure_1953,dennis_eternal_2003}, and structural determination of biomolecules~\cite{longchamp_how_2015} {\color{red} more references}.
Crystallographic techniques have been under constant development and refinement since their inception.
The majority of crystallography to date has been performed using X-rays however relatively recent developements have utilised electrons as electron diffraction techniques are more versatile than those of X-rays~\cite{cowley_electron_1992}.

Due to the mature understanding of crystallographic techniques, crystallography is ideal for the first steps in demonstrating the capabilities of \glspl{caes}.
While the \gls{caes} is able to operate in \gls{cw} mode the performance with this apparatus has been optimised for pulse mode (see Section~\ref{section:pulse_vs_continuous}) and thus the results described in this chapter have been taken with pulse bunches of electrons.

\subsection{Theory}

Crystalline sturctures consist of repeated sub-structures, unit cells, each with identical arrangements of atoms.
A perfect infinite cystal can be described as a lattice of unit cells where a set of basis vectors, $\mathbf{\hat{a}}$, $\mathbf{\hat{b}}$ and $\mathbf{\hat{c}}$, can be used to describe translations,
\begin{equation}
\mathbf{t} = u\mathbf{\hat{a}} + v\mathbf{\hat{b}} + w\mathbf{\hat{c}},
\end{equation}
where the integer coordiantes $u$, $v$ and $w$ describe the number of unit cells along each basis vector.
A primitive unit cell may contain one or more atoms and symmetry is sometimes more apparent if the unit cell consists of multiple primitive unit cells.

When a wave interacting with the crystal lattice interacts with a specific atom in every unit cell then the reflected waves will be in phase, thus creating the maxima of the diffraction pattern.
These planes in the crystal that these atoms line up on can eb described by points in the reciprocal lattice.
The reciprocal lattice is described by the reciprocal lattice basis vectors, $\mathbf{\hat{a}^*}$, $\mathbf{\hat{b}^*}$ and $\mathbf{\hat{c}^*}$, where
\begin{align}
\mathbf{\hat{a}^*}&=\frac{2\pi\:\mathbf{\hat{b}}\times\mathbf{\hat{c}}}{\mathbf{\hat{a}}\cdot (\mathbf{\hat{b}} \times \mathbf{\hat{c}})}  &  \mathbf{\hat{b}^*}&=\frac{2\pi\:\mathbf{\hat{c}}\times\mathbf{\hat{a}}}{\mathbf{\hat{a}}\cdot (\mathbf{\hat{b}} \times \mathbf{\hat{c}})}  &  \mathbf{\hat{c}^*}&=\frac{2\pi\:\mathbf{\hat{a}}\times\mathbf{\hat{b}}}{\mathbf{\hat{a}}\cdot (\mathbf{\hat{b}} \times \mathbf{\hat{c}})}.
\end{align}
The $2\pi$ originates from the cenvention chosen for the wave-vector, $|\mathbf{k}|=2\pi/\lambda$. The reciprocal lattice vector is thus,
\begin{equation}
\mathbf{g} = h\mathbf{\hat{a}^*} + \mathbf{\hat{b}^*} + l\mathbf{\hat{c}^*},
\end{equation}
where the integers $h$, $k$ and $l$ are known as the Miller indices.

Scattering events can be described by, $\mathbf{q}$, the scattering vector, which is the difference between the initial and final wavevectors, $\mathbf{k_0} and \mathbf{k}$;
\begin{equation}
\mathbf{q} = \mathbf{k} - \mathbf{k_0}.
\end{equation}
In the case of elastically scatter waves, $|\mathbf{k}| = |\mathbf{k_0}|$, constructive interference occurs only where $\mathbf{q}$ is equal to a reciprocal lattice vector, so that the scattering condition for crystals is
\begin{equation}
\mathbf{q} = \mathbf{g}.
\end{equation}
The realationship between the scattering vector and the scattering angle is
\begin{equation}\label{equation:diffraction_angle}
|\mathbf{q}| = 2|\mathbf{k_0}|\sin(\theta)
\end{equation}
where $\theta$ is the half angle between the initial and final wavevectors.
Due to the wavevector convention used the relation between real and reciprocal space distances is
\begin{equation}\label{equation:diffraction_distance}
|\mathbf{g}| = \frac{2\pi}{d_{hkl}},
\end{equation}
where $d_{hkl}$ is the distance between the set of planes described by the subscripted Miller indices.

If we combine Equations~\ref{equation:diffraction_angle} and \ref{equation:diffraction_distance} then we get the Bragg condition,
\begin{equation}
2d_{hkl}\sin\theta = n\lambda,
\end{equation}
where $n$ is the diffraction order.

The relative intensity of a wave with scattering vector $\mathbf{q}$ is given by
\begin{equation}
I(\mathbf{q}) = |\tilde{V}(\mathbf{q})|^2,
\end{equation}
where $\tilde{V}(\mathbf{q}$ is the Fourier transform of the crystal potential evaluated at $\mathbf{q}$ and we approximate to single-scattering events only.

The Fourier transform of the crystal potential, for an infinite crystal, can be written in terms of the structure factors, $V_g$,
\begin{equation}
\tilde{V}(\mathbf{q}) = (2\pi)^3 \sum_g V_g \delta(\mathbf{q} - \mathbf{g}),
\end{equation}
where the sum takes the scattering crontribution from all the reciprocal lattice poitns in to account, and the scattering condition is implemented with the Dirac delta.
The crystal basis affects the calculation of the structure factors for each reciprocal lattice point.
For a particular reciprocal lattice point, the structure factor is calculated by taking the position $\mathbf{x_j}$ and scatterring factors $\tilde{V}_j$ for each atom, $j$, in the cyrstal basis:
\begin{equation}
V_g = \frac{1}{V_{cell}}\sum\tilde{V}_j(\mathbf{g})e^{-i\mathbf{g}\cdot\mathbf{x_j}}.
\end{equation}


\section{Why ultrafast?}

\section{How does CAES do it?}

\section{Sample Bias}\label{section:sample_bias}

\section{Results}

\subsection{Gold}

\subsection{Aluminium}

\subsection{Graphene}

\subsection{Other Stuff}

\section{Why don't all our samples work?}