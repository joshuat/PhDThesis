\chapter{Time-Resolved Brightness Measurements}\label{chapter:brightness}

The most comprehensive figure of merit for charged particle beams is brightness, which incorporates beam current and emittance~\cite{reiser_theory_2008}.
Brightness measurements can also serve as a powerful diagnostic of the processes involved in beam creation and manipulation~\cite{mills_brightness_1980,qiang_three-dimensional_2006,bazarov_maximum_2009}.
Time-resolved brightness measurements have the potential to reveal information related to effects such as electron diffusion time, image-charge formation, and space-charge interactions.
This chapter presents a technique for measuring time-resolved brightness by combining pepperpot emittance measurements with beam streaking, and demonstrates the method with electrons generated from the \gls{caeis} as shown in Figure~\ref{figure:emittance_schematic_render}.
This work has been published as Reference~\cite{torrance_time-resolved_2018}.

\begin{figure}
\center
\includegraphics[width=\linewidth]{part2/Figs/render.pdf}
\caption[The experimental setup for the streaked brightness measurements.]{The experimental setup for time-resolved brightness measurements by streaking one-dimensional pepperpots with a \gls{caes}. Inset are shown the one- and two-dimensional pepperpot masks.}
\label{figure:emittance_schematic_render}
\end{figure}

The pepperpot method measures the transverse emittance of a charged particle beam using an array of apertures or slits to divide the beam into smaller beamlets that are detected after propagation~\cite{collins_extraction_1964,van_steenbergen_evaluation_1967,wang_beam_1991,brunetti_low_2010}.
The divergence, and thus the emittance, can be determined by measuring the position and size of each beamlet, giving an approximation of the transverse phase space of the beam~\cite{zhang_emittance_1996,wang_beam_1991}.
The pepperpot method has been used on both electron and ion beams, and with electron energies up to 500\,MeV~\cite{delerue_single-shot_2010}.

Time-resolved emittance measurements have been performed previously using time-gated detectors to study the feasibility and performance of laser-generated ion sources~\cite{collier_cern_1996,yoshida_simple_2001}, and with scanning slits and the time-resolved signal from a Faraday cup with the aim of minimising the emittance of an electron storage ring~\cite{walter_time_2003}.
These methods require many shots to accumulate the temporal profile, and consequently only measure the average beam behaviour and are fundamentally incapable of observing shot-to-shot variation in bunch dynamics.
An alternative method to determine the temporal emittance profile of a bunch is to apply a time-varying deflection to the beamlets formed by a one-dimensional pepperpot mask (a line of apertures) to `streak' the measurement across the detector.
Streaking of diffracted electron bunches has been used to observe non-repeatable dynamical processes~\cite{li_note:_2010, mukojima_streak-camera_2015}, and to characterise the electron pulse itself~\cite{ratner_time-resolved_2015}.
With a sufficiently intense source, streak measurements could be performed in a single shot, providing information on shot-to-shot behaviour to elucidate information unavailable to multiple shot averages, such as transient and stochastic effects.

There are a number of important reasons for interest in time-resolved brightness measurements with a \gls{caeis} as low emittance and high brightness are important to all applications for both electrons and ions.
Knowledge of source brightness is of particular interest in a number of areas such as the advantage of knowing beam coherence for \gls{ued}, the ability to observe the performance of techniques to counter space-charge expansion, and the potential to provide information on the ionisation processes in a \gls{caes}.
Beam emittance is also a vital factor when considering ion beam milling and ion microscope applications as the emittance limits the resolution achievable~\cite{scipioni_helium_2008,khizroev_focused-ion-beam-based_2004}.

The \gls{caeis} is a promising candidate for \gls{ued} imaging and structure determination due to its high brightness and coherence, with the potential to enable measurement of atomic-level sub-picosecond structural dynamics~\cite{siwick_atomic-level_2003,putkunz_atom-scale_2012,sparkes_high-coherence_2014}.
\gls{ued} is often performed with photocathode electron sources which have sufficient current and bunch duration for \gls{ued} but limited coherence~\cite{van_oudheusden_electron_2007,tokita_single-shot_2009,sciaini_femtosecond_2011}.
A \gls{caes} has the necessary coherence for \gls{ued} of nanocrystals and large molecules but not yet sufficient current~\cite{saliba_spatial_2012,mcculloch_high-coherence_2013}.
In the context of (ultrafast) diffractive imaging, time-resolved knowledge of the source coherence can be incorporated into the image reconstruction to improve the process and, as the coherence can be calculated from the emittance, the technique presented here could prove to be useful for time-resolved \gls{ued}~\cite{flewett_extracting_2009,abbey_lensless_2011}.

Electron bunches from the \gls{caes} can be produced with durations over a range of timescales, from femtoseconds to microseconds.
With the use of femtosecond duration pulse lasers, a \gls{caes} can produce ultrashort electron bunches, which are dense enough to experience space-charge expansion and the related loss of beam quality~\cite{luiten_how_2004,van_oudheusden_compression_2010,thompson_suppression_2016}.
One of the obstacles to single-shot, \gls{ued} is degradation of beam quality due to space-charge expansion and streaked pepperpot measurements could be used to observe the performance of techniques designed to counter space-charge expansion~\cite{thompson_suppression_2016}.

The production of electrons from photoionised cold atoms is a complex process, and the temporal profile of pulses from these sources has been characterised under a variety of ionisation conditions~\cite{speirs_identification_2017}.
The same complex ionisation processes that result in variable pulse duration are also likely to affect the transverse velocity of the emitted electrons, and thus the emittance, as a function of time.
The streaked pepperpot method presented allows the transverse velocity spread of the liberated electrons to be measured as a function of time, which can illuminate the underlying atomic ionisation processes and provide a diagnostic which could allow optimisation of electron bunch brightness in the future.

This chapter presents a simple method of measuring time-resolved emittance that is applicable to a wide range of charged particle sources and has the potential to be performed with a single-shot measurement.
The streaked-pepperpot technique was developed to observe time-varying effects during the photoionisation and extraction of electrons from the \gls{caes}.
The brightness of a \gls{caeis} is important to all of it's potential applications and the time-resolution provided by the technique described in this chapter will be a useful tool to examine the behaviour of the source and hopefully provide avenues for improvement.

This chapter examines time-resolved emittance measurements through the streaking of pepperpots; including the theory involved, various technical considerations, example measurements and implementation of this technique with the \gls{caeis} at the University of Melbourne.

\section{Brightness and Emittance}

A given ensemble of particles can be described by its density in six-dimensional phase space, $(x, p_x, y, p_x, z, p_z)$ where $(x, y, z)$ are the positions and $(p_x, p_y, p_z)$ are the momenta of each particle.
The extent of the beam in phase space is called the \emph{emittance}, $\epsilon$, of the beam.
Each Cartesian direction is examined separately, $(\epsilon_x, x, p_x)$, $(\epsilon_y, y, p_y)$ and $(\epsilon_z, z, p_z)$ where $z$ is the optic axis of the beam.

Typically the gradients of trajectories in $x$-$z$ and $y$-$z$ are measured rather than the momenta.
These gradients are referred to as the divergence and are defined as
\begin{equation}\label{equation:divergence}
x^\prime \equiv \frac{dx}{dz} = \frac{v_x}{v_z}.
\end{equation}
The space of $(x, x^\prime)$ is referred to as trace-space.
An example of the trace-space occupied by a beam is shown in Figure~\ref{figure:emittance_example}.
The \emph{emittance} can be defined as
\begin{equation}
\epsilon_x \equiv \frac{A^x}{\pi}
\end{equation}
where $A^x$ is the area occupied by the beam in trace space.

\begin{figure}
\center
\includegraphics{part2/Figs/EmittanceExample.pdf}
\caption[Trace-space for expanding, collimated and shrinking beam.]{An example of the trace-space occupied by an expanding, collimated and shrinking beam. The dotted line indicates the area occupied by the beam, the emittance.}
\label{figure:emittance_example}
\end{figure}

The \emph{\gls{rms} emittance} is a more practical measure and is defined as
\begin{equation}\label{emittance}
\bar{\epsilon}_x \equiv \sqrt{\langle x^2\rangle \langle x^{\prime 2}\rangle - \langle x x^\prime\rangle^2} \simeq \frac{\epsilon_x}{4}.
\end{equation}

The emittance of a beam represents the `focusability' of the beam.
A low emittance beam can be focused to a smaller waist than a high emittance beam, in fact a beam with zero emittance could be focused to a point whereas any beam with non-zero emittance has a finite spot size, this is shown in Figure~\ref{figure:focusability}.
This makes emittance an important quantity to consider in almost all beam applications but is particularly important to charged beams due to complications produced by space-charge repulsion.

\begin{figure}
\center
\includegraphics{part2/Figs/EmittanceFocasability.pdf}
\caption[Comparison of zero and non-zero emittance.]{On the left is a particle beam with zero emittance, and thus can be focused to a beam waist of zero width. On the right is a beam with non-zero emittance and thus a non-zero beam waist. Below each figure are plots of the phase space at difference points in the beam where the dashed line is indicative of the phase-space area or emittance.}
\label{figure:focusability}
\end{figure}

The emittance of a beam is not the only factor that determines the quality of the beam.
Emittance could be made arbitrarily small through the use of collimating slits however the reduced particle could would reduce the usefulness of that beam for most applications.
A more comprehensive figure of merit is the number of particles with a given emittance, known as \emph{brightness}~\cite{reiser_theory_2008};
\begin{equation}
B = \frac{I}{8\pi^2\bar{\epsilon}_x\bar{\epsilon}_y},
\end{equation}
where $I$ is the current of the beam and $\bar{\epsilon}_{x,y}$ is the \gls{rms} emittance as described above.
Brightness provides a better metric of a beam's quality with the additional requirement of knowing the beam current.

When comparing the emittance and brightness of particle beams from different sources it is often useful to use the \emph{normalised} emittance and brightness
\begin{align}
\bar{\epsilon}_{n,x} = \beta\bar{\epsilon}_x,\\
B_n = \frac{I}{8\pi^2\bar{\epsilon}_{n,x}\bar{\epsilon}_{n,y}}
\end{align}
where $\beta=v/c\approx v_z/c$.


\subsection{Emittance with a CAEIS}
\label{section:excess_energy_emittance}

For a \gls{caes} the normalised \gls{rms} emittance at the source is given by~\cite{mcculloch_high-coherence_2013}
\begin{equation}\label{equation:excess_energy_emittance}
\bar{\epsilon}_{n,x} = \sigma_x \sqrt{\frac{k_B T}{m c^2}}
\end{equation}
where $\sigma_x$ is the \gls{rms} beam radius, $k_B$ is Boltzmann's constant, $T$ is the temperature of the source, $m$ is the particle mass and $c$ is the speed of light.
Equation~\ref{equation:excess_energy_emittance} highlights one of the advantages of a \gls{caeis} as the low source temperature allows for low emittance bunches compared to the common thermionic sources.

The temperature of the \gls{caeis} source can be calculated from the wavelength of the ionisation lasers, $\lambda_{red}$ and $\lambda_{blue}$,
\begin{align}
 E_{red} &= \frac{hc}{\lambda_{red}} \\
 E_{blue}&= \frac{hc}{\lambda_{blue}},
\end{align}
where $h$ is Plank's constant.
We can define the total ionisation energy to be
\begin{align}
E_{total} = E_{red} + E_{blue}.
\end{align}
The ionisation energy for Rubidium 85, $E_I$ is \unit[4.18]{eV}~\cite{steck_rubidium_2008} so we can determine the excess energy of ionisation
\begin{equation}
E_{excess} = E_{total} - E_I
\end{equation}

The excess energy can be related to the temperatures of the ionised particles via
\begin{equation}\label{equation:excess_energy_temperature}
T = \frac{E_{excess}}{k_B}.
\end{equation}
Combining Equations \ref{equation:excess_energy_emittance} and \ref{equation:excess_energy_temperature} allows for the calculation of the expected emittance of bunches generated from the \gls{caeis} for above-threshold ionisation.
The emittance of bunches generated from below-threshold pathways, such as Rydberg-excitation with field-ionisation, cannot be calculated using Equation~\ref{equation:excess_energy_emittance}.

\section{Measurement}

Directly calculating the emittance of an ensemble with Equation~\ref{emittance} requires full knowledge of the position and momenta of the particles which is difficult as beam monitors tend to only measure the transverse positions of particles.
There are a number of methods to practically calculate the emittance of a particle beam, namely pepperpots, the multiple profile method methods and the lens method.
In the \gls{caeis} the lens method is not practical as the lenses in the system are not well characterised and multi-profile measurements are tedious due to the required manual z-translation of the detector and bellows.
The pepperpot method however is eminently achievable with this system although the details of the geometry are not ideal with this iteration of a \gls{caeis}.

\subsection{Pepperpots}

The pepperpot method uses a beam mask, consisting of an array of apertures, to separate the beam into a number of `beamlets' which are then propagated and detected downstream.
By examining the size of the beamlets the divergence of the beam can be estimated and thus the emittance of the beam can be calculated.
Ideally the extent of the array should be larger than the size of the beam and the holes as small as is practical while maintaining sufficient flux and ensuring the spots on the detector do not overlap.
The name refers to the similarity of the simplest beam mask to the perforated lid of a container for pepper.

A useful derivation of this technique is presented in Reference~\cite{zhang_emittance_1996} with the one dimensional result:

\begin{dmath}\label{equation:pepperpot}
\epsilon_x^2 = \left\langle x^2\right\rangle \left\langle x^{\prime2}\right\rangle - \left\langle xx^\prime\right\rangle^2\allowbreak
\approx \frac{1}{N^2} \left\{\left[\sum_{j=1}^p{n_j\left(x_{sj}-\bar{x}\right)^2}\right] \left[ \sum_{j=1}^p{\left[n_j\sigma_{x_j^\prime}^2 + n_j\left(\bar{x_j^\prime}-\bar{x^\prime}\right)^2\right]}\right] - \left[ \sum_{j=1}^p{n_jx_{sj}\bar{x_j^\prime}-N\bar{x}\bar{x^\prime}}\right]^2\right\}
\end{dmath}
where;
\begin{itemize}
    \item $N$ is the total number of particles after the beam mask,
    \item $p$ is the total number of holes in the $x$ direction,
    \item $n_j$ is the number of particles passing through the $j$-th hole and hitting the detector,
    \item $x_{sj}$ is position of the $j$-th hole,
    \item $\bar{x}$ is the mean position of the holes,
    \item $\sigma_{x_j^\prime}$ is the \gls{rms} divergence of the $j$-th beamlet,
    \item $\bar{x_j^\prime}$ is the mean divergence of the $j$-th beamlet, and
    \item $\bar{x^\prime}$ is the mean divergence of all beamlets.
\end{itemize}

An example pepperpot mask and detected beamlets are shown in Figure~\ref{figure:pepperpot_example}.
For two-dimensional pepperpots this equation can be implemented by appropriately rotating the detected beamlets and then performing row and column sums of the pixels followed by application of Equation~\ref{equation:pepperpot} for $x$ and $y$.
For one-dimensional streaked pepperpot measurements the image should be rotated such that the streaks are horizontal, or vertical, and then each column, or row, can be used to determine the emittance at that point in the streak.

\begin{figure}
    \centering
    \begin{subfigure}{0.49\linewidth}
    \centering
    \includegraphics[width=\linewidth]{part2/Figs/example_pepperpot_mask.jpg}
    \caption{}
    \label{figure:pepperpot_mask}
    \end{subfigure}
    \begin{subfigure}{0.49\linewidth}
    \centering
    \includegraphics[width=\linewidth]{part2/Figs/example_pepperpot_detector_linear.jpeg}
    % 486.971 pepperpot from 2017.06.06
    \caption{}
    \label{figure:pepperpot_image}
    \end{subfigure}
    
    \begin{subfigure}{0.49\linewidth}
    \centering
    \includegraphics[width=\linewidth]{part2/Figs/example_pepperpot_1d.jpg}
    \caption{}
    \label{figure:1d_pepperpot}
    \end{subfigure}
    \begin{subfigure}{0.49\linewidth}
    \centering
    \includegraphics[width=\linewidth]{part2/Figs/example_streaked_pepperpot.png}
    \caption{}
    \label{figure:streaked_1d_pepperpot}
    \end{subfigure}
    % Manual labels.....
    \caption[One- and two-dimensional pepperpot masks.]{(a) a pepperpot mask cut into a thing copper disk and, (b), the corresponding set of beamlets on the detector. (c) a one-dimensional pepperpot mask (held in place by a copper ring) and the corresponding streaked set of beamlets on the detector, (d). The beam images are log scaled.}
    \label{figure:pepperpot_example}
\end{figure}

\subsubsection{Temporal Resolution with Streaking}
A one-dimensional pepperpot consisting of a line of apertures provides a prime target for streak measurements.
In this case streaking is performed with a time varying electric field which deflects the charged particles across the detector.
For a pulsed charged particle source, such as the \gls{caeis}, the streak can be performed over the duration of the bunch or over a portion of the bunch if higher resolution is required.
There are some considerations for how fast an electric field can be swept, which will not be examined in detail here, so extremely short bunches will require more sophisticated streaking systems, such as a \gls{rf} cavity or photoactivated switching which are able to provide temporal resolution as low as \unit[100]{fs}~\cite{li_note:_2010,kassier_compact_2010,van_rens_theory_2018}.
An example of a streaked pepperpot measurement is shown in Figure~\ref{figure:streaked_1d_pepperpot}.

\section{Experimental Setup}

A number of modifications were made to the \gls{caeis} to prepare it for the streaked emittance measurements and there were a number of restrictions on various parameters due to the precise setup of the apparatus.
The \gls{caeis} was operated in electron mode for these measurements due to the available magnetic optics and the larger emittance expected from electrons as the source temperature is higher for electrons, a minimum temperature of approximately \unit[10]{K} for electrons and \unit[100]{$\muup$K} for ions~\cite{saliba_spatial_2012}.
Using electrons also allows control of the bunch emittance as the excess energy can be controlled with the blue ionisation laser wavelength, with ions the change in emittance would be negligible.

See Chapter~\ref{chapter:setup} for a more detailed description of the \gls{caeis} and Figure~\ref{figure:emittance_schematic} for a schematic of the setup used for the measurements in this chapter.

\begin{figure}
\center
\includegraphics{part2/Figs/EmittanceApparatusSchematic.pdf}
\caption[Brightness measurement apparatus with dimensions.]{A schematic of the experimental apparatus with relevant dimensions. The blue region indicates the electron beam envelope. Note that components are not necessarily orientate realistically in this schematic (particularly the deflectors).}
\label{figure:emittance_schematic}
\end{figure}

\subsection{Beam Optics}
The quadrupole electron optics discussed in Section \ref{section:quadrupole} were use to reduce the astigmatism present in the electron beam as discussed in the aforementioned section.
The two-dimensional pepperpot provide another avenue to monitor the astigmatism of the beam as with an astigmatic beam the spacing of the beamlets on the detector are not the same along each axis as shown in Figure~\ref{figure:astigmatic_pepperpot}.
When focusing was required the Einzel lens was used.

\begin{figure}
    \center
    \includegraphics[width=0.5\linewidth]{part2/Figs/example_astigmatic_pepperpot.jpeg}
    \caption[Pepperpot measurements with and without beam astigmatism correction.]{An example of a pepperpot without the beam correction from the quadrupole lens. It is readily apparent that spacing of the beamlets is not the same for the $x$ and $y$ axes. This image is log scaled.}
    \label{figure:astigmatic_pepperpot}
    % From 2017.06.01
\end{figure}

A number of permanent magnets were used to steer the beam through the various apertures and onto the detector and care was taken to attempt to keep the beam on the central axis of the apparatus.
This was achieved by manually adjusting the positions of the magnets, which were mounted on posts external to the vacuum system, to ensure that the beam was passing through the apertures in the system and by scanning the Einzel lens to verify that the beam was passing through its centre.
Due to the unstable nature of the apparatus these magnets needed to be adjusted on a day-to-day basis to maintain the beam path.

The Einzel lens was used to focus the beam through a focus to the approximate size of the pepperpot mask extent.
This was done to increase the divergence of the beamlets, making measurement easier, and to increase the electron flux transmitted through the pepperpot mask.
The Einzel lens voltage required for this was approximately \unit[4.8][kV] and varied slightly with beam emittance, i.e. high emittance beams result in a larger beam size and thus required adjusting the Einzel lens voltage by \unit[100]{V} or so.

\subsection{Beam Energy}
In many emittance measurements the beam energy would not be a free parameter as emittance would be used as a diagnostic for a specific system or scenario, however in this experiment we are able to vary the energy of our beam from a few hundred eV to around \unit[10]{keV}.
There were a number of considerations in determining the optimal beam energy to use for pepperpot measurements such as measurement resolution, beam current and beam stability.

The resolution of a pepperpot measurement is greater when the size of the beamlets on the detector is larger and, as shown in Equation~\ref{equation:divergence} the divergence is greater for slower beams.
Unfortunately the slower the beam the more fragile the alignment of the beam becomes with the \gls{caes}.
It was impossible to align a \unit[500][eV] beam unobstructed through the system to the detector and even had it been possible the stability of the beam path would have been difficult to work with.
A relatively high energy beam of \unit[8]{keV} was used as this provided a robust beam alignment that was resistant to the transient changes in the magnetic environment while providing sufficient measurement resolution.

\subsection{Bellows}

The bellows provide an additional mechanism to control the resolution of the measurement by adjusting the propagation distance of the pepperpot beamlets and thus their size on the detector.
Once the strength of the Einzel lens had been set as described above the bellows were positioned such that the length of the streaks covered the majority of the detector which gave a propagation distance from the pepperpots of \unit[475]{mm}.

\subsection{Streaking}
The streaking was performed using a pair of deflectors located after the pepperpot sample and orientated such that the one-dimensional pepperpots could be streaked across the detector.
For these measurements one deflector was grounded and the other given a time-varying voltage which was supplied in one of two ways:
\begin{itemize}
\item A fast ramp using a bipolar push-pull solid-state switch with a fixed transition time of \unit[10]{ns} and
\item A slower ramp using an amplified signal generator with a minimum transition of approximately \unit[10]{$\muup$s}.
\end{itemize}
An example of the voltage ramps is shown in Figure~\ref{figure:deflector_voltages}. The slow ramp was performed by amplifying the output of signal generator\footnote{Rigol DG4162, able to operate up to \unit[160]{MHz}} which was highly configurable and thus capable of a wide range of pulse lengths, limited by the speed of the amplifier to a transition time minimum of approximately \unit[10]{$\muup$s} in duration as shown in Figure~\ref{figure:deflector_voltages}.
More detail on the design of the streaking electronics can be found in Reference~\cite{speirs_electron_2017}.

\begin{figure}
    \center
    %% Creator: Matplotlib, PGF backend
%%
%% To include the figure in your LaTeX document, write
%%   \input{<filename>.pgf}
%%
%% Make sure the required packages are loaded in your preamble
%%   \usepackage{pgf}
%%
%% Figures using additional raster images can only be included by \input if
%% they are in the same directory as the main LaTeX file. For loading figures
%% from other directories you can use the `import` package
%%   \usepackage{import}
%% and then include the figures with
%%   \import{<path to file>}{<filename>.pgf}
%%
%% Matplotlib used the following preamble
%%
\begingroup%
\makeatletter%
\begin{pgfpicture}%
\pgfpathrectangle{\pgfpointorigin}{\pgfqpoint{4.969839in}{2.484920in}}%
\pgfusepath{use as bounding box, clip}%
\begin{pgfscope}%
\pgfsetbuttcap%
\pgfsetmiterjoin%
\definecolor{currentfill}{rgb}{1.000000,1.000000,1.000000}%
\pgfsetfillcolor{currentfill}%
\pgfsetlinewidth{0.000000pt}%
\definecolor{currentstroke}{rgb}{1.000000,1.000000,1.000000}%
\pgfsetstrokecolor{currentstroke}%
\pgfsetdash{}{0pt}%
\pgfpathmoveto{\pgfqpoint{0.000000in}{0.000000in}}%
\pgfpathlineto{\pgfqpoint{4.969839in}{0.000000in}}%
\pgfpathlineto{\pgfqpoint{4.969839in}{2.484920in}}%
\pgfpathlineto{\pgfqpoint{0.000000in}{2.484920in}}%
\pgfpathclose%
\pgfusepath{fill}%
\end{pgfscope}%
\begin{pgfscope}%
\pgfsetbuttcap%
\pgfsetmiterjoin%
\definecolor{currentfill}{rgb}{1.000000,1.000000,1.000000}%
\pgfsetfillcolor{currentfill}%
\pgfsetlinewidth{0.000000pt}%
\definecolor{currentstroke}{rgb}{0.000000,0.000000,0.000000}%
\pgfsetstrokecolor{currentstroke}%
\pgfsetstrokeopacity{0.000000}%
\pgfsetdash{}{0pt}%
\pgfpathmoveto{\pgfqpoint{0.721094in}{0.587500in}}%
\pgfpathlineto{\pgfqpoint{2.725935in}{0.587500in}}%
\pgfpathlineto{\pgfqpoint{2.725935in}{2.334920in}}%
\pgfpathlineto{\pgfqpoint{0.721094in}{2.334920in}}%
\pgfpathclose%
\pgfusepath{fill}%
\end{pgfscope}%
\begin{pgfscope}%
\pgfpathrectangle{\pgfqpoint{0.721094in}{0.587500in}}{\pgfqpoint{2.004842in}{1.747420in}} %
\pgfusepath{clip}%
\pgfsetrectcap%
\pgfsetroundjoin%
\pgfsetlinewidth{1.003750pt}%
\definecolor{currentstroke}{rgb}{0.000000,0.000000,0.000000}%
\pgfsetstrokecolor{currentstroke}%
\pgfsetdash{}{0pt}%
\pgfpathmoveto{\pgfqpoint{0.752119in}{0.689689in}}%
\pgfpathlineto{\pgfqpoint{0.757480in}{0.690761in}}%
\pgfpathlineto{\pgfqpoint{0.760218in}{0.690603in}}%
\pgfpathlineto{\pgfqpoint{0.764927in}{0.688590in}}%
\pgfpathlineto{\pgfqpoint{0.767711in}{0.688285in}}%
\pgfpathlineto{\pgfqpoint{0.769612in}{0.687979in}}%
\pgfpathlineto{\pgfqpoint{0.769658in}{0.688031in}}%
\pgfpathlineto{\pgfqpoint{0.771778in}{0.688795in}}%
\pgfpathlineto{\pgfqpoint{0.773462in}{0.689748in}}%
\pgfpathlineto{\pgfqpoint{0.775650in}{0.689571in}}%
\pgfpathlineto{\pgfqpoint{0.782329in}{0.687657in}}%
\pgfpathlineto{\pgfqpoint{0.785697in}{0.690137in}}%
\pgfpathlineto{\pgfqpoint{0.786258in}{0.689211in}}%
\pgfpathlineto{\pgfqpoint{0.788412in}{0.687701in}}%
\pgfpathlineto{\pgfqpoint{0.791001in}{0.688400in}}%
\pgfpathlineto{\pgfqpoint{0.792490in}{0.689234in}}%
\pgfpathlineto{\pgfqpoint{0.792846in}{0.688784in}}%
\pgfpathlineto{\pgfqpoint{0.794209in}{0.687583in}}%
\pgfpathlineto{\pgfqpoint{0.794633in}{0.688067in}}%
\pgfpathlineto{\pgfqpoint{0.797107in}{0.690902in}}%
\pgfpathlineto{\pgfqpoint{0.797520in}{0.690576in}}%
\pgfpathlineto{\pgfqpoint{0.799422in}{0.690379in}}%
\pgfpathlineto{\pgfqpoint{0.801128in}{0.690280in}}%
\pgfpathlineto{\pgfqpoint{0.805310in}{0.687071in}}%
\pgfpathlineto{\pgfqpoint{0.806570in}{0.687647in}}%
\pgfpathlineto{\pgfqpoint{0.808713in}{0.688506in}}%
\pgfpathlineto{\pgfqpoint{0.812126in}{0.688352in}}%
\pgfpathlineto{\pgfqpoint{0.815311in}{0.690456in}}%
\pgfpathlineto{\pgfqpoint{0.816846in}{0.689229in}}%
\pgfpathlineto{\pgfqpoint{0.819436in}{0.687802in}}%
\pgfpathlineto{\pgfqpoint{0.820890in}{0.689018in}}%
\pgfpathlineto{\pgfqpoint{0.822872in}{0.689749in}}%
\pgfpathlineto{\pgfqpoint{0.825691in}{0.689476in}}%
\pgfpathlineto{\pgfqpoint{0.833664in}{0.689271in}}%
\pgfpathlineto{\pgfqpoint{0.838407in}{0.691744in}}%
\pgfpathlineto{\pgfqpoint{0.843345in}{0.689341in}}%
\pgfpathlineto{\pgfqpoint{0.847160in}{0.690175in}}%
\pgfpathlineto{\pgfqpoint{0.848752in}{0.689144in}}%
\pgfpathlineto{\pgfqpoint{0.850826in}{0.688263in}}%
\pgfpathlineto{\pgfqpoint{0.850929in}{0.688360in}}%
\pgfpathlineto{\pgfqpoint{0.855030in}{0.690806in}}%
\pgfpathlineto{\pgfqpoint{0.856348in}{0.689937in}}%
\pgfpathlineto{\pgfqpoint{0.858123in}{0.688527in}}%
\pgfpathlineto{\pgfqpoint{0.858467in}{0.688856in}}%
\pgfpathlineto{\pgfqpoint{0.859807in}{0.689598in}}%
\pgfpathlineto{\pgfqpoint{0.860174in}{0.689129in}}%
\pgfpathlineto{\pgfqpoint{0.861595in}{0.687703in}}%
\pgfpathlineto{\pgfqpoint{0.862053in}{0.688181in}}%
\pgfpathlineto{\pgfqpoint{0.864092in}{0.689079in}}%
\pgfpathlineto{\pgfqpoint{0.866143in}{0.689228in}}%
\pgfpathlineto{\pgfqpoint{0.870725in}{0.688176in}}%
\pgfpathlineto{\pgfqpoint{0.872226in}{0.688538in}}%
\pgfpathlineto{\pgfqpoint{0.873795in}{0.689880in}}%
\pgfpathlineto{\pgfqpoint{0.874219in}{0.689457in}}%
\pgfpathlineto{\pgfqpoint{0.876465in}{0.688192in}}%
\pgfpathlineto{\pgfqpoint{0.876591in}{0.688305in}}%
\pgfpathlineto{\pgfqpoint{0.879088in}{0.689202in}}%
\pgfpathlineto{\pgfqpoint{0.881723in}{0.689223in}}%
\pgfpathlineto{\pgfqpoint{0.885423in}{0.690128in}}%
\pgfpathlineto{\pgfqpoint{0.887887in}{0.688763in}}%
\pgfpathlineto{\pgfqpoint{0.891232in}{0.690848in}}%
\pgfpathlineto{\pgfqpoint{0.891679in}{0.690076in}}%
\pgfpathlineto{\pgfqpoint{0.893248in}{0.688605in}}%
\pgfpathlineto{\pgfqpoint{0.893569in}{0.688832in}}%
\pgfpathlineto{\pgfqpoint{0.896021in}{0.689453in}}%
\pgfpathlineto{\pgfqpoint{0.898999in}{0.688604in}}%
\pgfpathlineto{\pgfqpoint{0.900362in}{0.688582in}}%
\pgfpathlineto{\pgfqpoint{0.900523in}{0.688825in}}%
\pgfpathlineto{\pgfqpoint{0.901943in}{0.693489in}}%
\pgfpathlineto{\pgfqpoint{0.915863in}{0.747785in}}%
\pgfpathlineto{\pgfqpoint{0.926299in}{0.790127in}}%
\pgfpathlineto{\pgfqpoint{0.929186in}{0.800019in}}%
\pgfpathlineto{\pgfqpoint{0.931237in}{0.806783in}}%
\pgfpathlineto{\pgfqpoint{0.939027in}{0.838366in}}%
\pgfpathlineto{\pgfqpoint{0.979960in}{0.994691in}}%
\pgfpathlineto{\pgfqpoint{0.989114in}{1.027640in}}%
\pgfpathlineto{\pgfqpoint{1.005004in}{1.075562in}}%
\pgfpathlineto{\pgfqpoint{1.007169in}{1.083562in}}%
\pgfpathlineto{\pgfqpoint{1.014352in}{1.107776in}}%
\pgfpathlineto{\pgfqpoint{1.017594in}{1.120146in}}%
\pgfpathlineto{\pgfqpoint{1.028798in}{1.159890in}}%
\pgfpathlineto{\pgfqpoint{1.037138in}{1.188890in}}%
\pgfpathlineto{\pgfqpoint{1.046177in}{1.222626in}}%
\pgfpathlineto{\pgfqpoint{1.048893in}{1.231334in}}%
\pgfpathlineto{\pgfqpoint{1.057519in}{1.262665in}}%
\pgfpathlineto{\pgfqpoint{1.070843in}{1.306886in}}%
\pgfpathlineto{\pgfqpoint{1.072836in}{1.313390in}}%
\pgfpathlineto{\pgfqpoint{1.079584in}{1.339286in}}%
\pgfpathlineto{\pgfqpoint{1.111627in}{1.450967in}}%
\pgfpathlineto{\pgfqpoint{1.114617in}{1.460058in}}%
\pgfpathlineto{\pgfqpoint{1.122041in}{1.482669in}}%
\pgfpathlineto{\pgfqpoint{1.128651in}{1.506518in}}%
\pgfpathlineto{\pgfqpoint{1.202922in}{1.733628in}}%
\pgfpathlineto{\pgfqpoint{1.212820in}{1.757921in}}%
\pgfpathlineto{\pgfqpoint{1.217276in}{1.766719in}}%
\pgfpathlineto{\pgfqpoint{1.224001in}{1.771119in}}%
\pgfpathlineto{\pgfqpoint{1.234232in}{1.771856in}}%
\pgfpathlineto{\pgfqpoint{1.238138in}{1.768261in}}%
\pgfpathlineto{\pgfqpoint{1.242618in}{1.764064in}}%
\pgfpathlineto{\pgfqpoint{1.244680in}{1.763060in}}%
\pgfpathlineto{\pgfqpoint{1.246387in}{1.762043in}}%
\pgfpathlineto{\pgfqpoint{1.260008in}{1.747365in}}%
\pgfpathlineto{\pgfqpoint{1.261360in}{1.746180in}}%
\pgfpathlineto{\pgfqpoint{1.266366in}{1.738435in}}%
\pgfpathlineto{\pgfqpoint{1.272083in}{1.730845in}}%
\pgfpathlineto{\pgfqpoint{1.274535in}{1.728151in}}%
\pgfpathlineto{\pgfqpoint{1.279174in}{1.725207in}}%
\pgfpathlineto{\pgfqpoint{1.284616in}{1.719093in}}%
\pgfpathlineto{\pgfqpoint{1.292670in}{1.707405in}}%
\pgfpathlineto{\pgfqpoint{1.295557in}{1.702949in}}%
\pgfpathlineto{\pgfqpoint{1.298684in}{1.699296in}}%
\pgfpathlineto{\pgfqpoint{1.300437in}{1.697488in}}%
\pgfpathlineto{\pgfqpoint{1.303381in}{1.694250in}}%
\pgfpathlineto{\pgfqpoint{1.308594in}{1.686672in}}%
\pgfpathlineto{\pgfqpoint{1.355748in}{1.622068in}}%
\pgfpathlineto{\pgfqpoint{1.358715in}{1.618138in}}%
\pgfpathlineto{\pgfqpoint{1.367456in}{1.608102in}}%
\pgfpathlineto{\pgfqpoint{1.372348in}{1.599734in}}%
\pgfpathlineto{\pgfqpoint{1.374525in}{1.597638in}}%
\pgfpathlineto{\pgfqpoint{1.378282in}{1.593686in}}%
\pgfpathlineto{\pgfqpoint{1.380734in}{1.591312in}}%
\pgfpathlineto{\pgfqpoint{1.384480in}{1.584916in}}%
\pgfpathlineto{\pgfqpoint{1.386462in}{1.583687in}}%
\pgfpathlineto{\pgfqpoint{1.391617in}{1.578113in}}%
\pgfpathlineto{\pgfqpoint{1.392648in}{1.576727in}}%
\pgfpathlineto{\pgfqpoint{1.397609in}{1.567902in}}%
\pgfpathlineto{\pgfqpoint{1.410268in}{1.554287in}}%
\pgfpathlineto{\pgfqpoint{1.414931in}{1.547250in}}%
\pgfpathlineto{\pgfqpoint{1.418265in}{1.543926in}}%
\pgfpathlineto{\pgfqpoint{1.423786in}{1.536573in}}%
\pgfpathlineto{\pgfqpoint{1.426009in}{1.531899in}}%
\pgfpathlineto{\pgfqpoint{1.426582in}{1.532275in}}%
\pgfpathlineto{\pgfqpoint{1.428002in}{1.531914in}}%
\pgfpathlineto{\pgfqpoint{1.440948in}{1.514196in}}%
\pgfpathlineto{\pgfqpoint{1.443067in}{1.512428in}}%
\pgfpathlineto{\pgfqpoint{1.444190in}{1.511622in}}%
\pgfpathlineto{\pgfqpoint{1.450583in}{1.502342in}}%
\pgfpathlineto{\pgfqpoint{1.451659in}{1.500734in}}%
\pgfpathlineto{\pgfqpoint{1.455646in}{1.494135in}}%
\pgfpathlineto{\pgfqpoint{1.458121in}{1.492225in}}%
\pgfpathlineto{\pgfqpoint{1.460847in}{1.487228in}}%
\pgfpathlineto{\pgfqpoint{1.464536in}{1.482005in}}%
\pgfpathlineto{\pgfqpoint{1.466839in}{1.479624in}}%
\pgfpathlineto{\pgfqpoint{1.468420in}{1.477973in}}%
\pgfpathlineto{\pgfqpoint{1.477379in}{1.464049in}}%
\pgfpathlineto{\pgfqpoint{1.481709in}{1.457092in}}%
\pgfpathlineto{\pgfqpoint{1.483359in}{1.456732in}}%
\pgfpathlineto{\pgfqpoint{1.484768in}{1.456067in}}%
\pgfpathlineto{\pgfqpoint{1.493624in}{1.442885in}}%
\pgfpathlineto{\pgfqpoint{1.502812in}{1.428717in}}%
\pgfpathlineto{\pgfqpoint{1.505229in}{1.426211in}}%
\pgfpathlineto{\pgfqpoint{1.506913in}{1.424677in}}%
\pgfpathlineto{\pgfqpoint{1.513157in}{1.414533in}}%
\pgfpathlineto{\pgfqpoint{1.514142in}{1.415512in}}%
\pgfpathlineto{\pgfqpoint{1.515058in}{1.414712in}}%
\pgfpathlineto{\pgfqpoint{1.520065in}{1.406549in}}%
\pgfpathlineto{\pgfqpoint{1.521829in}{1.404004in}}%
\pgfpathlineto{\pgfqpoint{1.527603in}{1.396204in}}%
\pgfpathlineto{\pgfqpoint{1.531189in}{1.391776in}}%
\pgfpathlineto{\pgfqpoint{1.533835in}{1.388304in}}%
\pgfpathlineto{\pgfqpoint{1.536734in}{1.384778in}}%
\pgfpathlineto{\pgfqpoint{1.543264in}{1.375524in}}%
\pgfpathlineto{\pgfqpoint{1.547468in}{1.369448in}}%
\pgfpathlineto{\pgfqpoint{1.549118in}{1.364838in}}%
\pgfpathlineto{\pgfqpoint{1.550378in}{1.362683in}}%
\pgfpathlineto{\pgfqpoint{1.550813in}{1.362972in}}%
\pgfpathlineto{\pgfqpoint{1.551890in}{1.363081in}}%
\pgfpathlineto{\pgfqpoint{1.552165in}{1.362678in}}%
\pgfpathlineto{\pgfqpoint{1.556816in}{1.356719in}}%
\pgfpathlineto{\pgfqpoint{1.558065in}{1.355117in}}%
\pgfpathlineto{\pgfqpoint{1.562029in}{1.348977in}}%
\pgfpathlineto{\pgfqpoint{1.563690in}{1.345756in}}%
\pgfpathlineto{\pgfqpoint{1.566440in}{1.341526in}}%
\pgfpathlineto{\pgfqpoint{1.575410in}{1.328043in}}%
\pgfpathlineto{\pgfqpoint{1.576762in}{1.326294in}}%
\pgfpathlineto{\pgfqpoint{1.577105in}{1.326471in}}%
\pgfpathlineto{\pgfqpoint{1.578618in}{1.326166in}}%
\pgfpathlineto{\pgfqpoint{1.578640in}{1.326129in}}%
\pgfpathlineto{\pgfqpoint{1.586866in}{1.314489in}}%
\pgfpathlineto{\pgfqpoint{1.588550in}{1.312895in}}%
\pgfpathlineto{\pgfqpoint{1.593602in}{1.305780in}}%
\pgfpathlineto{\pgfqpoint{1.598460in}{1.298882in}}%
\pgfpathlineto{\pgfqpoint{1.602882in}{1.293602in}}%
\pgfpathlineto{\pgfqpoint{1.604841in}{1.291861in}}%
\pgfpathlineto{\pgfqpoint{1.609985in}{1.283114in}}%
\pgfpathlineto{\pgfqpoint{1.612127in}{1.279752in}}%
\pgfpathlineto{\pgfqpoint{1.618348in}{1.273377in}}%
\pgfpathlineto{\pgfqpoint{1.619677in}{1.272361in}}%
\pgfpathlineto{\pgfqpoint{1.631900in}{1.253417in}}%
\pgfpathlineto{\pgfqpoint{1.633012in}{1.251913in}}%
\pgfpathlineto{\pgfqpoint{1.639885in}{1.240300in}}%
\pgfpathlineto{\pgfqpoint{1.644182in}{1.236609in}}%
\pgfpathlineto{\pgfqpoint{1.645465in}{1.235862in}}%
\pgfpathlineto{\pgfqpoint{1.649268in}{1.227735in}}%
\pgfpathlineto{\pgfqpoint{1.651342in}{1.225425in}}%
\pgfpathlineto{\pgfqpoint{1.654160in}{1.223106in}}%
\pgfpathlineto{\pgfqpoint{1.685206in}{1.179584in}}%
\pgfpathlineto{\pgfqpoint{1.688884in}{1.175108in}}%
\pgfpathlineto{\pgfqpoint{1.693088in}{1.169662in}}%
\pgfpathlineto{\pgfqpoint{1.695013in}{1.168246in}}%
\pgfpathlineto{\pgfqpoint{1.698014in}{1.161756in}}%
\pgfpathlineto{\pgfqpoint{1.701199in}{1.157167in}}%
\pgfpathlineto{\pgfqpoint{1.703617in}{1.154726in}}%
\pgfpathlineto{\pgfqpoint{1.705713in}{1.153520in}}%
\pgfpathlineto{\pgfqpoint{1.708394in}{1.148573in}}%
\pgfpathlineto{\pgfqpoint{1.711510in}{1.145589in}}%
\pgfpathlineto{\pgfqpoint{1.713068in}{1.143322in}}%
\pgfpathlineto{\pgfqpoint{1.717135in}{1.136976in}}%
\pgfpathlineto{\pgfqpoint{1.721832in}{1.131429in}}%
\pgfpathlineto{\pgfqpoint{1.724318in}{1.127537in}}%
\pgfpathlineto{\pgfqpoint{1.728270in}{1.121913in}}%
\pgfpathlineto{\pgfqpoint{1.731971in}{1.115799in}}%
\pgfpathlineto{\pgfqpoint{1.736358in}{1.110947in}}%
\pgfpathlineto{\pgfqpoint{1.740242in}{1.106808in}}%
\pgfpathlineto{\pgfqpoint{1.744447in}{1.102299in}}%
\pgfpathlineto{\pgfqpoint{1.749201in}{1.095011in}}%
\pgfpathlineto{\pgfqpoint{1.751859in}{1.091012in}}%
\pgfpathlineto{\pgfqpoint{1.755238in}{1.088229in}}%
\pgfpathlineto{\pgfqpoint{1.756899in}{1.087371in}}%
\pgfpathlineto{\pgfqpoint{1.763578in}{1.079622in}}%
\pgfpathlineto{\pgfqpoint{1.769811in}{1.073292in}}%
\pgfpathlineto{\pgfqpoint{1.793376in}{1.049202in}}%
\pgfpathlineto{\pgfqpoint{1.794522in}{1.047958in}}%
\pgfpathlineto{\pgfqpoint{1.798291in}{1.043058in}}%
\pgfpathlineto{\pgfqpoint{1.809713in}{1.030218in}}%
\pgfpathlineto{\pgfqpoint{1.811706in}{1.028828in}}%
\pgfpathlineto{\pgfqpoint{1.818328in}{1.020790in}}%
\pgfpathlineto{\pgfqpoint{1.819531in}{1.019518in}}%
\pgfpathlineto{\pgfqpoint{1.823254in}{1.014603in}}%
\pgfpathlineto{\pgfqpoint{1.831250in}{1.004983in}}%
\pgfpathlineto{\pgfqpoint{1.836417in}{0.997117in}}%
\pgfpathlineto{\pgfqpoint{1.850004in}{0.982241in}}%
\pgfpathlineto{\pgfqpoint{1.854839in}{0.976714in}}%
\pgfpathlineto{\pgfqpoint{1.857669in}{0.973873in}}%
\pgfpathlineto{\pgfqpoint{1.863729in}{0.964990in}}%
\pgfpathlineto{\pgfqpoint{1.865585in}{0.963988in}}%
\pgfpathlineto{\pgfqpoint{1.872814in}{0.954955in}}%
\pgfpathlineto{\pgfqpoint{1.877236in}{0.949667in}}%
\pgfpathlineto{\pgfqpoint{1.879080in}{0.948432in}}%
\pgfpathlineto{\pgfqpoint{1.887111in}{0.938698in}}%
\pgfpathlineto{\pgfqpoint{1.888784in}{0.936210in}}%
\pgfpathlineto{\pgfqpoint{1.889185in}{0.936451in}}%
\pgfpathlineto{\pgfqpoint{1.890170in}{0.936276in}}%
\pgfpathlineto{\pgfqpoint{1.890365in}{0.935926in}}%
\pgfpathlineto{\pgfqpoint{1.893721in}{0.931469in}}%
\pgfpathlineto{\pgfqpoint{1.897949in}{0.925893in}}%
\pgfpathlineto{\pgfqpoint{1.900733in}{0.921804in}}%
\pgfpathlineto{\pgfqpoint{1.902302in}{0.921065in}}%
\pgfpathlineto{\pgfqpoint{1.910677in}{0.910874in}}%
\pgfpathlineto{\pgfqpoint{1.912945in}{0.909223in}}%
\pgfpathlineto{\pgfqpoint{1.925363in}{0.897617in}}%
\pgfpathlineto{\pgfqpoint{1.926395in}{0.896294in}}%
\pgfpathlineto{\pgfqpoint{1.929430in}{0.892285in}}%
\pgfpathlineto{\pgfqpoint{1.932008in}{0.889435in}}%
\pgfpathlineto{\pgfqpoint{1.940039in}{0.878864in}}%
\pgfpathlineto{\pgfqpoint{1.943098in}{0.874784in}}%
\pgfpathlineto{\pgfqpoint{1.947142in}{0.872461in}}%
\pgfpathlineto{\pgfqpoint{1.953603in}{0.863985in}}%
\pgfpathlineto{\pgfqpoint{1.956307in}{0.860312in}}%
\pgfpathlineto{\pgfqpoint{1.958495in}{0.858394in}}%
\pgfpathlineto{\pgfqpoint{1.963559in}{0.855485in}}%
\pgfpathlineto{\pgfqpoint{1.970719in}{0.846461in}}%
\pgfpathlineto{\pgfqpoint{1.974877in}{0.844077in}}%
\pgfpathlineto{\pgfqpoint{1.976974in}{0.839271in}}%
\pgfpathlineto{\pgfqpoint{1.981843in}{0.833745in}}%
\pgfpathlineto{\pgfqpoint{1.983699in}{0.833196in}}%
\pgfpathlineto{\pgfqpoint{1.984844in}{0.832196in}}%
\pgfpathlineto{\pgfqpoint{1.987697in}{0.828102in}}%
\pgfpathlineto{\pgfqpoint{1.988121in}{0.828367in}}%
\pgfpathlineto{\pgfqpoint{1.989690in}{0.828023in}}%
\pgfpathlineto{\pgfqpoint{2.003472in}{0.814832in}}%
\pgfpathlineto{\pgfqpoint{2.004652in}{0.814157in}}%
\pgfpathlineto{\pgfqpoint{2.009338in}{0.808426in}}%
\pgfpathlineto{\pgfqpoint{2.012568in}{0.804914in}}%
\pgfpathlineto{\pgfqpoint{2.019958in}{0.797686in}}%
\pgfpathlineto{\pgfqpoint{2.027565in}{0.789133in}}%
\pgfpathlineto{\pgfqpoint{2.030406in}{0.785580in}}%
\pgfpathlineto{\pgfqpoint{2.032720in}{0.785694in}}%
\pgfpathlineto{\pgfqpoint{2.034083in}{0.784279in}}%
\pgfpathlineto{\pgfqpoint{2.037486in}{0.780792in}}%
\pgfpathlineto{\pgfqpoint{2.050202in}{0.767572in}}%
\pgfpathlineto{\pgfqpoint{2.052448in}{0.765540in}}%
\pgfpathlineto{\pgfqpoint{2.054292in}{0.765537in}}%
\pgfpathlineto{\pgfqpoint{2.055701in}{0.765471in}}%
\pgfpathlineto{\pgfqpoint{2.055804in}{0.765330in}}%
\pgfpathlineto{\pgfqpoint{2.060650in}{0.760263in}}%
\pgfpathlineto{\pgfqpoint{2.064511in}{0.755092in}}%
\pgfpathlineto{\pgfqpoint{2.067455in}{0.751407in}}%
\pgfpathlineto{\pgfqpoint{2.069758in}{0.749642in}}%
\pgfpathlineto{\pgfqpoint{2.072210in}{0.745326in}}%
\pgfpathlineto{\pgfqpoint{2.072484in}{0.745475in}}%
\pgfpathlineto{\pgfqpoint{2.074295in}{0.745372in}}%
\pgfpathlineto{\pgfqpoint{2.079599in}{0.739779in}}%
\pgfpathlineto{\pgfqpoint{2.082325in}{0.738292in}}%
\pgfpathlineto{\pgfqpoint{2.085178in}{0.734046in}}%
\pgfpathlineto{\pgfqpoint{2.087137in}{0.733061in}}%
\pgfpathlineto{\pgfqpoint{2.090585in}{0.730257in}}%
\pgfpathlineto{\pgfqpoint{2.093037in}{0.728641in}}%
\pgfpathlineto{\pgfqpoint{2.097024in}{0.722535in}}%
\pgfpathlineto{\pgfqpoint{2.098685in}{0.721657in}}%
\pgfpathlineto{\pgfqpoint{2.102660in}{0.717993in}}%
\pgfpathlineto{\pgfqpoint{2.106544in}{0.716109in}}%
\pgfpathlineto{\pgfqpoint{2.109110in}{0.713745in}}%
\pgfpathlineto{\pgfqpoint{2.112409in}{0.711950in}}%
\pgfpathlineto{\pgfqpoint{2.116030in}{0.707073in}}%
\pgfpathlineto{\pgfqpoint{2.125240in}{0.700933in}}%
\pgfpathlineto{\pgfqpoint{2.126902in}{0.700136in}}%
\pgfpathlineto{\pgfqpoint{2.130167in}{0.697592in}}%
\pgfpathlineto{\pgfqpoint{2.132824in}{0.696747in}}%
\pgfpathlineto{\pgfqpoint{2.135677in}{0.698244in}}%
\pgfpathlineto{\pgfqpoint{2.137407in}{0.696712in}}%
\pgfpathlineto{\pgfqpoint{2.139389in}{0.695736in}}%
\pgfpathlineto{\pgfqpoint{2.142780in}{0.697860in}}%
\pgfpathlineto{\pgfqpoint{2.144475in}{0.697698in}}%
\pgfpathlineto{\pgfqpoint{2.146503in}{0.695086in}}%
\pgfpathlineto{\pgfqpoint{2.148267in}{0.695009in}}%
\pgfpathlineto{\pgfqpoint{2.151521in}{0.695448in}}%
\pgfpathlineto{\pgfqpoint{2.153262in}{0.695518in}}%
\pgfpathlineto{\pgfqpoint{2.154969in}{0.696497in}}%
\pgfpathlineto{\pgfqpoint{2.155279in}{0.696153in}}%
\pgfpathlineto{\pgfqpoint{2.157994in}{0.694366in}}%
\pgfpathlineto{\pgfqpoint{2.160285in}{0.694535in}}%
\pgfpathlineto{\pgfqpoint{2.163504in}{0.693671in}}%
\pgfpathlineto{\pgfqpoint{2.165131in}{0.693375in}}%
\pgfpathlineto{\pgfqpoint{2.167159in}{0.694042in}}%
\pgfpathlineto{\pgfqpoint{2.168396in}{0.693098in}}%
\pgfpathlineto{\pgfqpoint{2.169874in}{0.691844in}}%
\pgfpathlineto{\pgfqpoint{2.170263in}{0.692230in}}%
\pgfpathlineto{\pgfqpoint{2.172681in}{0.693541in}}%
\pgfpathlineto{\pgfqpoint{2.188536in}{0.692735in}}%
\pgfpathlineto{\pgfqpoint{2.191251in}{0.692822in}}%
\pgfpathlineto{\pgfqpoint{2.195078in}{0.692076in}}%
\pgfpathlineto{\pgfqpoint{2.197346in}{0.692417in}}%
\pgfpathlineto{\pgfqpoint{2.199236in}{0.692364in}}%
\pgfpathlineto{\pgfqpoint{2.201276in}{0.692521in}}%
\pgfpathlineto{\pgfqpoint{2.206591in}{0.692259in}}%
\pgfpathlineto{\pgfqpoint{2.208619in}{0.692770in}}%
\pgfpathlineto{\pgfqpoint{2.210990in}{0.692380in}}%
\pgfpathlineto{\pgfqpoint{2.212755in}{0.691495in}}%
\pgfpathlineto{\pgfqpoint{2.214485in}{0.691268in}}%
\pgfpathlineto{\pgfqpoint{2.216226in}{0.691275in}}%
\pgfpathlineto{\pgfqpoint{2.219709in}{0.689883in}}%
\pgfpathlineto{\pgfqpoint{2.222435in}{0.691214in}}%
\pgfpathlineto{\pgfqpoint{2.222641in}{0.690951in}}%
\pgfpathlineto{\pgfqpoint{2.224429in}{0.689850in}}%
\pgfpathlineto{\pgfqpoint{2.224669in}{0.690050in}}%
\pgfpathlineto{\pgfqpoint{2.227888in}{0.691483in}}%
\pgfpathlineto{\pgfqpoint{2.230260in}{0.691514in}}%
\pgfpathlineto{\pgfqpoint{2.232837in}{0.692075in}}%
\pgfpathlineto{\pgfqpoint{2.235931in}{0.690224in}}%
\pgfpathlineto{\pgfqpoint{2.240410in}{0.691372in}}%
\pgfpathlineto{\pgfqpoint{2.242048in}{0.691700in}}%
\pgfpathlineto{\pgfqpoint{2.245371in}{0.692473in}}%
\pgfpathlineto{\pgfqpoint{2.247948in}{0.689839in}}%
\pgfpathlineto{\pgfqpoint{2.248899in}{0.689902in}}%
\pgfpathlineto{\pgfqpoint{2.249151in}{0.690189in}}%
\pgfpathlineto{\pgfqpoint{2.251488in}{0.691313in}}%
\pgfpathlineto{\pgfqpoint{2.256758in}{0.690372in}}%
\pgfpathlineto{\pgfqpoint{2.263552in}{0.690809in}}%
\pgfpathlineto{\pgfqpoint{2.265144in}{0.689243in}}%
\pgfpathlineto{\pgfqpoint{2.265568in}{0.689709in}}%
\pgfpathlineto{\pgfqpoint{2.267550in}{0.692078in}}%
\pgfpathlineto{\pgfqpoint{2.268065in}{0.691491in}}%
\pgfpathlineto{\pgfqpoint{2.270276in}{0.690192in}}%
\pgfpathlineto{\pgfqpoint{2.272636in}{0.690032in}}%
\pgfpathlineto{\pgfqpoint{2.275478in}{0.689331in}}%
\pgfpathlineto{\pgfqpoint{2.277803in}{0.689685in}}%
\pgfpathlineto{\pgfqpoint{2.279900in}{0.689921in}}%
\pgfpathlineto{\pgfqpoint{2.282512in}{0.691185in}}%
\pgfpathlineto{\pgfqpoint{2.284459in}{0.691182in}}%
\pgfpathlineto{\pgfqpoint{2.287507in}{0.692643in}}%
\pgfpathlineto{\pgfqpoint{2.290520in}{0.691386in}}%
\pgfpathlineto{\pgfqpoint{2.294025in}{0.691068in}}%
\pgfpathlineto{\pgfqpoint{2.296546in}{0.691464in}}%
\pgfpathlineto{\pgfqpoint{2.301082in}{0.691073in}}%
\pgfpathlineto{\pgfqpoint{2.303064in}{0.692527in}}%
\pgfpathlineto{\pgfqpoint{2.303557in}{0.691819in}}%
\pgfpathlineto{\pgfqpoint{2.306123in}{0.689794in}}%
\pgfpathlineto{\pgfqpoint{2.308701in}{0.690177in}}%
\pgfpathlineto{\pgfqpoint{2.311106in}{0.690653in}}%
\pgfpathlineto{\pgfqpoint{2.312951in}{0.691425in}}%
\pgfpathlineto{\pgfqpoint{2.314383in}{0.691050in}}%
\pgfpathlineto{\pgfqpoint{2.317774in}{0.688109in}}%
\pgfpathlineto{\pgfqpoint{2.319584in}{0.688754in}}%
\pgfpathlineto{\pgfqpoint{2.323743in}{0.690604in}}%
\pgfpathlineto{\pgfqpoint{2.328680in}{0.689248in}}%
\pgfpathlineto{\pgfqpoint{2.330685in}{0.689275in}}%
\pgfpathlineto{\pgfqpoint{2.332358in}{0.689434in}}%
\pgfpathlineto{\pgfqpoint{2.332427in}{0.689355in}}%
\pgfpathlineto{\pgfqpoint{2.334660in}{0.688427in}}%
\pgfpathlineto{\pgfqpoint{2.337799in}{0.688686in}}%
\pgfpathlineto{\pgfqpoint{2.339232in}{0.687632in}}%
\pgfpathlineto{\pgfqpoint{2.339587in}{0.687939in}}%
\pgfpathlineto{\pgfqpoint{2.342749in}{0.689664in}}%
\pgfpathlineto{\pgfqpoint{2.348144in}{0.688554in}}%
\pgfpathlineto{\pgfqpoint{2.349840in}{0.688320in}}%
\pgfpathlineto{\pgfqpoint{2.351387in}{0.688627in}}%
\pgfpathlineto{\pgfqpoint{2.351581in}{0.688369in}}%
\pgfpathlineto{\pgfqpoint{2.353185in}{0.686809in}}%
\pgfpathlineto{\pgfqpoint{2.353632in}{0.687304in}}%
\pgfpathlineto{\pgfqpoint{2.355820in}{0.688481in}}%
\pgfpathlineto{\pgfqpoint{2.358936in}{0.688832in}}%
\pgfpathlineto{\pgfqpoint{2.361800in}{0.689089in}}%
\pgfpathlineto{\pgfqpoint{2.364092in}{0.688730in}}%
\pgfpathlineto{\pgfqpoint{2.369281in}{0.688892in}}%
\pgfpathlineto{\pgfqpoint{2.371091in}{0.688853in}}%
\pgfpathlineto{\pgfqpoint{2.373314in}{0.689668in}}%
\pgfpathlineto{\pgfqpoint{2.375158in}{0.689819in}}%
\pgfpathlineto{\pgfqpoint{2.376625in}{0.689731in}}%
\pgfpathlineto{\pgfqpoint{2.376705in}{0.689627in}}%
\pgfpathlineto{\pgfqpoint{2.378469in}{0.689041in}}%
\pgfpathlineto{\pgfqpoint{2.378504in}{0.689067in}}%
\pgfpathlineto{\pgfqpoint{2.381081in}{0.689639in}}%
\pgfpathlineto{\pgfqpoint{2.388986in}{0.689492in}}%
\pgfpathlineto{\pgfqpoint{2.390292in}{0.690029in}}%
\pgfpathlineto{\pgfqpoint{2.390555in}{0.689840in}}%
\pgfpathlineto{\pgfqpoint{2.391804in}{0.690166in}}%
\pgfpathlineto{\pgfqpoint{2.391873in}{0.690268in}}%
\pgfpathlineto{\pgfqpoint{2.393786in}{0.691912in}}%
\pgfpathlineto{\pgfqpoint{2.394107in}{0.691656in}}%
\pgfpathlineto{\pgfqpoint{2.395871in}{0.691568in}}%
\pgfpathlineto{\pgfqpoint{2.397452in}{0.691248in}}%
\pgfpathlineto{\pgfqpoint{2.399514in}{0.691095in}}%
\pgfpathlineto{\pgfqpoint{2.401485in}{0.691420in}}%
\pgfpathlineto{\pgfqpoint{2.404429in}{0.690676in}}%
\pgfpathlineto{\pgfqpoint{2.406869in}{0.691549in}}%
\pgfpathlineto{\pgfqpoint{2.410214in}{0.691076in}}%
\pgfpathlineto{\pgfqpoint{2.414087in}{0.692509in}}%
\pgfpathlineto{\pgfqpoint{2.415633in}{0.692848in}}%
\pgfpathlineto{\pgfqpoint{2.418073in}{0.693651in}}%
\pgfpathlineto{\pgfqpoint{2.420926in}{0.692858in}}%
\pgfpathlineto{\pgfqpoint{2.422908in}{0.693551in}}%
\pgfpathlineto{\pgfqpoint{2.423011in}{0.693451in}}%
\pgfpathlineto{\pgfqpoint{2.425474in}{0.692524in}}%
\pgfpathlineto{\pgfqpoint{2.433493in}{0.695117in}}%
\pgfpathlineto{\pgfqpoint{2.437343in}{0.694132in}}%
\pgfpathlineto{\pgfqpoint{2.440264in}{0.694979in}}%
\pgfpathlineto{\pgfqpoint{2.441604in}{0.695468in}}%
\pgfpathlineto{\pgfqpoint{2.443048in}{0.696947in}}%
\pgfpathlineto{\pgfqpoint{2.443506in}{0.696469in}}%
\pgfpathlineto{\pgfqpoint{2.446634in}{0.694074in}}%
\pgfpathlineto{\pgfqpoint{2.449647in}{0.693489in}}%
\pgfpathlineto{\pgfqpoint{2.451102in}{0.693736in}}%
\pgfpathlineto{\pgfqpoint{2.454034in}{0.695314in}}%
\pgfpathlineto{\pgfqpoint{2.458640in}{0.696374in}}%
\pgfpathlineto{\pgfqpoint{2.462936in}{0.695832in}}%
\pgfpathlineto{\pgfqpoint{2.464735in}{0.696674in}}%
\pgfpathlineto{\pgfqpoint{2.466499in}{0.696328in}}%
\pgfpathlineto{\pgfqpoint{2.468240in}{0.696347in}}%
\pgfpathlineto{\pgfqpoint{2.468263in}{0.696378in}}%
\pgfpathlineto{\pgfqpoint{2.470543in}{0.700690in}}%
\pgfpathlineto{\pgfqpoint{2.470543in}{0.700690in}}%
\pgfusepath{stroke}%
\end{pgfscope}%
\begin{pgfscope}%
\pgfpathrectangle{\pgfqpoint{0.721094in}{0.587500in}}{\pgfqpoint{2.004842in}{1.747420in}} %
\pgfusepath{clip}%
\pgfsetbuttcap%
\pgfsetroundjoin%
\pgfsetlinewidth{1.003750pt}%
\definecolor{currentstroke}{rgb}{0.000000,0.000000,0.000000}%
\pgfsetstrokecolor{currentstroke}%
\pgfsetdash{{1.000000pt}{3.000000pt}}{0.000000pt}%
\pgfpathmoveto{\pgfqpoint{1.293906in}{0.587500in}}%
\pgfpathlineto{\pgfqpoint{1.293906in}{2.334920in}}%
\pgfusepath{stroke}%
\end{pgfscope}%
\begin{pgfscope}%
\pgfsetrectcap%
\pgfsetmiterjoin%
\pgfsetlinewidth{1.003750pt}%
\definecolor{currentstroke}{rgb}{0.000000,0.000000,0.000000}%
\pgfsetstrokecolor{currentstroke}%
\pgfsetdash{}{0pt}%
\pgfpathmoveto{\pgfqpoint{2.725935in}{0.587500in}}%
\pgfpathlineto{\pgfqpoint{2.725935in}{2.334920in}}%
\pgfusepath{stroke}%
\end{pgfscope}%
\begin{pgfscope}%
\pgfsetrectcap%
\pgfsetmiterjoin%
\pgfsetlinewidth{1.003750pt}%
\definecolor{currentstroke}{rgb}{0.000000,0.000000,0.000000}%
\pgfsetstrokecolor{currentstroke}%
\pgfsetdash{}{0pt}%
\pgfpathmoveto{\pgfqpoint{0.721094in}{0.587500in}}%
\pgfpathlineto{\pgfqpoint{0.721094in}{2.334920in}}%
\pgfusepath{stroke}%
\end{pgfscope}%
\begin{pgfscope}%
\pgfsetrectcap%
\pgfsetmiterjoin%
\pgfsetlinewidth{1.003750pt}%
\definecolor{currentstroke}{rgb}{0.000000,0.000000,0.000000}%
\pgfsetstrokecolor{currentstroke}%
\pgfsetdash{}{0pt}%
\pgfpathmoveto{\pgfqpoint{0.721094in}{0.587500in}}%
\pgfpathlineto{\pgfqpoint{2.725935in}{0.587500in}}%
\pgfusepath{stroke}%
\end{pgfscope}%
\begin{pgfscope}%
\pgfsetrectcap%
\pgfsetmiterjoin%
\pgfsetlinewidth{1.003750pt}%
\definecolor{currentstroke}{rgb}{0.000000,0.000000,0.000000}%
\pgfsetstrokecolor{currentstroke}%
\pgfsetdash{}{0pt}%
\pgfpathmoveto{\pgfqpoint{0.721094in}{2.334920in}}%
\pgfpathlineto{\pgfqpoint{2.725935in}{2.334920in}}%
\pgfusepath{stroke}%
\end{pgfscope}%
\begin{pgfscope}%
\pgfsetbuttcap%
\pgfsetroundjoin%
\definecolor{currentfill}{rgb}{0.000000,0.000000,0.000000}%
\pgfsetfillcolor{currentfill}%
\pgfsetlinewidth{0.501875pt}%
\definecolor{currentstroke}{rgb}{0.000000,0.000000,0.000000}%
\pgfsetstrokecolor{currentstroke}%
\pgfsetdash{}{0pt}%
\pgfsys@defobject{currentmarker}{\pgfqpoint{0.000000in}{0.000000in}}{\pgfqpoint{0.000000in}{0.055556in}}{%
\pgfpathmoveto{\pgfqpoint{0.000000in}{0.000000in}}%
\pgfpathlineto{\pgfqpoint{0.000000in}{0.055556in}}%
\pgfusepath{stroke,fill}%
}%
\begin{pgfscope}%
\pgfsys@transformshift{0.721094in}{0.587500in}%
\pgfsys@useobject{currentmarker}{}%
\end{pgfscope}%
\end{pgfscope}%
\begin{pgfscope}%
\pgfsetbuttcap%
\pgfsetroundjoin%
\definecolor{currentfill}{rgb}{0.000000,0.000000,0.000000}%
\pgfsetfillcolor{currentfill}%
\pgfsetlinewidth{0.501875pt}%
\definecolor{currentstroke}{rgb}{0.000000,0.000000,0.000000}%
\pgfsetstrokecolor{currentstroke}%
\pgfsetdash{}{0pt}%
\pgfsys@defobject{currentmarker}{\pgfqpoint{0.000000in}{-0.055556in}}{\pgfqpoint{0.000000in}{0.000000in}}{%
\pgfpathmoveto{\pgfqpoint{0.000000in}{0.000000in}}%
\pgfpathlineto{\pgfqpoint{0.000000in}{-0.055556in}}%
\pgfusepath{stroke,fill}%
}%
\begin{pgfscope}%
\pgfsys@transformshift{0.721094in}{2.334920in}%
\pgfsys@useobject{currentmarker}{}%
\end{pgfscope}%
\end{pgfscope}%
\begin{pgfscope}%
\pgftext[x=0.721094in,y=0.531944in,,top]{\rmfamily\fontsize{10.000000}{12.000000}\selectfont -10}%
\end{pgfscope}%
\begin{pgfscope}%
\pgfsetbuttcap%
\pgfsetroundjoin%
\definecolor{currentfill}{rgb}{0.000000,0.000000,0.000000}%
\pgfsetfillcolor{currentfill}%
\pgfsetlinewidth{0.501875pt}%
\definecolor{currentstroke}{rgb}{0.000000,0.000000,0.000000}%
\pgfsetstrokecolor{currentstroke}%
\pgfsetdash{}{0pt}%
\pgfsys@defobject{currentmarker}{\pgfqpoint{0.000000in}{0.000000in}}{\pgfqpoint{0.000000in}{0.055556in}}{%
\pgfpathmoveto{\pgfqpoint{0.000000in}{0.000000in}}%
\pgfpathlineto{\pgfqpoint{0.000000in}{0.055556in}}%
\pgfusepath{stroke,fill}%
}%
\begin{pgfscope}%
\pgfsys@transformshift{1.007500in}{0.587500in}%
\pgfsys@useobject{currentmarker}{}%
\end{pgfscope}%
\end{pgfscope}%
\begin{pgfscope}%
\pgfsetbuttcap%
\pgfsetroundjoin%
\definecolor{currentfill}{rgb}{0.000000,0.000000,0.000000}%
\pgfsetfillcolor{currentfill}%
\pgfsetlinewidth{0.501875pt}%
\definecolor{currentstroke}{rgb}{0.000000,0.000000,0.000000}%
\pgfsetstrokecolor{currentstroke}%
\pgfsetdash{}{0pt}%
\pgfsys@defobject{currentmarker}{\pgfqpoint{0.000000in}{-0.055556in}}{\pgfqpoint{0.000000in}{0.000000in}}{%
\pgfpathmoveto{\pgfqpoint{0.000000in}{0.000000in}}%
\pgfpathlineto{\pgfqpoint{0.000000in}{-0.055556in}}%
\pgfusepath{stroke,fill}%
}%
\begin{pgfscope}%
\pgfsys@transformshift{1.007500in}{2.334920in}%
\pgfsys@useobject{currentmarker}{}%
\end{pgfscope}%
\end{pgfscope}%
\begin{pgfscope}%
\pgftext[x=1.007500in,y=0.531944in,,top]{\rmfamily\fontsize{10.000000}{12.000000}\selectfont -5}%
\end{pgfscope}%
\begin{pgfscope}%
\pgfsetbuttcap%
\pgfsetroundjoin%
\definecolor{currentfill}{rgb}{0.000000,0.000000,0.000000}%
\pgfsetfillcolor{currentfill}%
\pgfsetlinewidth{0.501875pt}%
\definecolor{currentstroke}{rgb}{0.000000,0.000000,0.000000}%
\pgfsetstrokecolor{currentstroke}%
\pgfsetdash{}{0pt}%
\pgfsys@defobject{currentmarker}{\pgfqpoint{0.000000in}{0.000000in}}{\pgfqpoint{0.000000in}{0.055556in}}{%
\pgfpathmoveto{\pgfqpoint{0.000000in}{0.000000in}}%
\pgfpathlineto{\pgfqpoint{0.000000in}{0.055556in}}%
\pgfusepath{stroke,fill}%
}%
\begin{pgfscope}%
\pgfsys@transformshift{1.293906in}{0.587500in}%
\pgfsys@useobject{currentmarker}{}%
\end{pgfscope}%
\end{pgfscope}%
\begin{pgfscope}%
\pgfsetbuttcap%
\pgfsetroundjoin%
\definecolor{currentfill}{rgb}{0.000000,0.000000,0.000000}%
\pgfsetfillcolor{currentfill}%
\pgfsetlinewidth{0.501875pt}%
\definecolor{currentstroke}{rgb}{0.000000,0.000000,0.000000}%
\pgfsetstrokecolor{currentstroke}%
\pgfsetdash{}{0pt}%
\pgfsys@defobject{currentmarker}{\pgfqpoint{0.000000in}{-0.055556in}}{\pgfqpoint{0.000000in}{0.000000in}}{%
\pgfpathmoveto{\pgfqpoint{0.000000in}{0.000000in}}%
\pgfpathlineto{\pgfqpoint{0.000000in}{-0.055556in}}%
\pgfusepath{stroke,fill}%
}%
\begin{pgfscope}%
\pgfsys@transformshift{1.293906in}{2.334920in}%
\pgfsys@useobject{currentmarker}{}%
\end{pgfscope}%
\end{pgfscope}%
\begin{pgfscope}%
\pgftext[x=1.293906in,y=0.531944in,,top]{\rmfamily\fontsize{10.000000}{12.000000}\selectfont 0}%
\end{pgfscope}%
\begin{pgfscope}%
\pgfsetbuttcap%
\pgfsetroundjoin%
\definecolor{currentfill}{rgb}{0.000000,0.000000,0.000000}%
\pgfsetfillcolor{currentfill}%
\pgfsetlinewidth{0.501875pt}%
\definecolor{currentstroke}{rgb}{0.000000,0.000000,0.000000}%
\pgfsetstrokecolor{currentstroke}%
\pgfsetdash{}{0pt}%
\pgfsys@defobject{currentmarker}{\pgfqpoint{0.000000in}{0.000000in}}{\pgfqpoint{0.000000in}{0.055556in}}{%
\pgfpathmoveto{\pgfqpoint{0.000000in}{0.000000in}}%
\pgfpathlineto{\pgfqpoint{0.000000in}{0.055556in}}%
\pgfusepath{stroke,fill}%
}%
\begin{pgfscope}%
\pgfsys@transformshift{1.580312in}{0.587500in}%
\pgfsys@useobject{currentmarker}{}%
\end{pgfscope}%
\end{pgfscope}%
\begin{pgfscope}%
\pgfsetbuttcap%
\pgfsetroundjoin%
\definecolor{currentfill}{rgb}{0.000000,0.000000,0.000000}%
\pgfsetfillcolor{currentfill}%
\pgfsetlinewidth{0.501875pt}%
\definecolor{currentstroke}{rgb}{0.000000,0.000000,0.000000}%
\pgfsetstrokecolor{currentstroke}%
\pgfsetdash{}{0pt}%
\pgfsys@defobject{currentmarker}{\pgfqpoint{0.000000in}{-0.055556in}}{\pgfqpoint{0.000000in}{0.000000in}}{%
\pgfpathmoveto{\pgfqpoint{0.000000in}{0.000000in}}%
\pgfpathlineto{\pgfqpoint{0.000000in}{-0.055556in}}%
\pgfusepath{stroke,fill}%
}%
\begin{pgfscope}%
\pgfsys@transformshift{1.580312in}{2.334920in}%
\pgfsys@useobject{currentmarker}{}%
\end{pgfscope}%
\end{pgfscope}%
\begin{pgfscope}%
\pgftext[x=1.580312in,y=0.531944in,,top]{\rmfamily\fontsize{10.000000}{12.000000}\selectfont 5}%
\end{pgfscope}%
\begin{pgfscope}%
\pgfsetbuttcap%
\pgfsetroundjoin%
\definecolor{currentfill}{rgb}{0.000000,0.000000,0.000000}%
\pgfsetfillcolor{currentfill}%
\pgfsetlinewidth{0.501875pt}%
\definecolor{currentstroke}{rgb}{0.000000,0.000000,0.000000}%
\pgfsetstrokecolor{currentstroke}%
\pgfsetdash{}{0pt}%
\pgfsys@defobject{currentmarker}{\pgfqpoint{0.000000in}{0.000000in}}{\pgfqpoint{0.000000in}{0.055556in}}{%
\pgfpathmoveto{\pgfqpoint{0.000000in}{0.000000in}}%
\pgfpathlineto{\pgfqpoint{0.000000in}{0.055556in}}%
\pgfusepath{stroke,fill}%
}%
\begin{pgfscope}%
\pgfsys@transformshift{1.866718in}{0.587500in}%
\pgfsys@useobject{currentmarker}{}%
\end{pgfscope}%
\end{pgfscope}%
\begin{pgfscope}%
\pgfsetbuttcap%
\pgfsetroundjoin%
\definecolor{currentfill}{rgb}{0.000000,0.000000,0.000000}%
\pgfsetfillcolor{currentfill}%
\pgfsetlinewidth{0.501875pt}%
\definecolor{currentstroke}{rgb}{0.000000,0.000000,0.000000}%
\pgfsetstrokecolor{currentstroke}%
\pgfsetdash{}{0pt}%
\pgfsys@defobject{currentmarker}{\pgfqpoint{0.000000in}{-0.055556in}}{\pgfqpoint{0.000000in}{0.000000in}}{%
\pgfpathmoveto{\pgfqpoint{0.000000in}{0.000000in}}%
\pgfpathlineto{\pgfqpoint{0.000000in}{-0.055556in}}%
\pgfusepath{stroke,fill}%
}%
\begin{pgfscope}%
\pgfsys@transformshift{1.866718in}{2.334920in}%
\pgfsys@useobject{currentmarker}{}%
\end{pgfscope}%
\end{pgfscope}%
\begin{pgfscope}%
\pgftext[x=1.866718in,y=0.531944in,,top]{\rmfamily\fontsize{10.000000}{12.000000}\selectfont 10}%
\end{pgfscope}%
\begin{pgfscope}%
\pgfsetbuttcap%
\pgfsetroundjoin%
\definecolor{currentfill}{rgb}{0.000000,0.000000,0.000000}%
\pgfsetfillcolor{currentfill}%
\pgfsetlinewidth{0.501875pt}%
\definecolor{currentstroke}{rgb}{0.000000,0.000000,0.000000}%
\pgfsetstrokecolor{currentstroke}%
\pgfsetdash{}{0pt}%
\pgfsys@defobject{currentmarker}{\pgfqpoint{0.000000in}{0.000000in}}{\pgfqpoint{0.000000in}{0.055556in}}{%
\pgfpathmoveto{\pgfqpoint{0.000000in}{0.000000in}}%
\pgfpathlineto{\pgfqpoint{0.000000in}{0.055556in}}%
\pgfusepath{stroke,fill}%
}%
\begin{pgfscope}%
\pgfsys@transformshift{2.153123in}{0.587500in}%
\pgfsys@useobject{currentmarker}{}%
\end{pgfscope}%
\end{pgfscope}%
\begin{pgfscope}%
\pgfsetbuttcap%
\pgfsetroundjoin%
\definecolor{currentfill}{rgb}{0.000000,0.000000,0.000000}%
\pgfsetfillcolor{currentfill}%
\pgfsetlinewidth{0.501875pt}%
\definecolor{currentstroke}{rgb}{0.000000,0.000000,0.000000}%
\pgfsetstrokecolor{currentstroke}%
\pgfsetdash{}{0pt}%
\pgfsys@defobject{currentmarker}{\pgfqpoint{0.000000in}{-0.055556in}}{\pgfqpoint{0.000000in}{0.000000in}}{%
\pgfpathmoveto{\pgfqpoint{0.000000in}{0.000000in}}%
\pgfpathlineto{\pgfqpoint{0.000000in}{-0.055556in}}%
\pgfusepath{stroke,fill}%
}%
\begin{pgfscope}%
\pgfsys@transformshift{2.153123in}{2.334920in}%
\pgfsys@useobject{currentmarker}{}%
\end{pgfscope}%
\end{pgfscope}%
\begin{pgfscope}%
\pgftext[x=2.153123in,y=0.531944in,,top]{\rmfamily\fontsize{10.000000}{12.000000}\selectfont 15}%
\end{pgfscope}%
\begin{pgfscope}%
\pgfsetbuttcap%
\pgfsetroundjoin%
\definecolor{currentfill}{rgb}{0.000000,0.000000,0.000000}%
\pgfsetfillcolor{currentfill}%
\pgfsetlinewidth{0.501875pt}%
\definecolor{currentstroke}{rgb}{0.000000,0.000000,0.000000}%
\pgfsetstrokecolor{currentstroke}%
\pgfsetdash{}{0pt}%
\pgfsys@defobject{currentmarker}{\pgfqpoint{0.000000in}{0.000000in}}{\pgfqpoint{0.000000in}{0.055556in}}{%
\pgfpathmoveto{\pgfqpoint{0.000000in}{0.000000in}}%
\pgfpathlineto{\pgfqpoint{0.000000in}{0.055556in}}%
\pgfusepath{stroke,fill}%
}%
\begin{pgfscope}%
\pgfsys@transformshift{2.439529in}{0.587500in}%
\pgfsys@useobject{currentmarker}{}%
\end{pgfscope}%
\end{pgfscope}%
\begin{pgfscope}%
\pgfsetbuttcap%
\pgfsetroundjoin%
\definecolor{currentfill}{rgb}{0.000000,0.000000,0.000000}%
\pgfsetfillcolor{currentfill}%
\pgfsetlinewidth{0.501875pt}%
\definecolor{currentstroke}{rgb}{0.000000,0.000000,0.000000}%
\pgfsetstrokecolor{currentstroke}%
\pgfsetdash{}{0pt}%
\pgfsys@defobject{currentmarker}{\pgfqpoint{0.000000in}{-0.055556in}}{\pgfqpoint{0.000000in}{0.000000in}}{%
\pgfpathmoveto{\pgfqpoint{0.000000in}{0.000000in}}%
\pgfpathlineto{\pgfqpoint{0.000000in}{-0.055556in}}%
\pgfusepath{stroke,fill}%
}%
\begin{pgfscope}%
\pgfsys@transformshift{2.439529in}{2.334920in}%
\pgfsys@useobject{currentmarker}{}%
\end{pgfscope}%
\end{pgfscope}%
\begin{pgfscope}%
\pgftext[x=2.439529in,y=0.531944in,,top]{\rmfamily\fontsize{10.000000}{12.000000}\selectfont 20}%
\end{pgfscope}%
\begin{pgfscope}%
\pgfsetbuttcap%
\pgfsetroundjoin%
\definecolor{currentfill}{rgb}{0.000000,0.000000,0.000000}%
\pgfsetfillcolor{currentfill}%
\pgfsetlinewidth{0.501875pt}%
\definecolor{currentstroke}{rgb}{0.000000,0.000000,0.000000}%
\pgfsetstrokecolor{currentstroke}%
\pgfsetdash{}{0pt}%
\pgfsys@defobject{currentmarker}{\pgfqpoint{0.000000in}{0.000000in}}{\pgfqpoint{0.000000in}{0.055556in}}{%
\pgfpathmoveto{\pgfqpoint{0.000000in}{0.000000in}}%
\pgfpathlineto{\pgfqpoint{0.000000in}{0.055556in}}%
\pgfusepath{stroke,fill}%
}%
\begin{pgfscope}%
\pgfsys@transformshift{2.725935in}{0.587500in}%
\pgfsys@useobject{currentmarker}{}%
\end{pgfscope}%
\end{pgfscope}%
\begin{pgfscope}%
\pgfsetbuttcap%
\pgfsetroundjoin%
\definecolor{currentfill}{rgb}{0.000000,0.000000,0.000000}%
\pgfsetfillcolor{currentfill}%
\pgfsetlinewidth{0.501875pt}%
\definecolor{currentstroke}{rgb}{0.000000,0.000000,0.000000}%
\pgfsetstrokecolor{currentstroke}%
\pgfsetdash{}{0pt}%
\pgfsys@defobject{currentmarker}{\pgfqpoint{0.000000in}{-0.055556in}}{\pgfqpoint{0.000000in}{0.000000in}}{%
\pgfpathmoveto{\pgfqpoint{0.000000in}{0.000000in}}%
\pgfpathlineto{\pgfqpoint{0.000000in}{-0.055556in}}%
\pgfusepath{stroke,fill}%
}%
\begin{pgfscope}%
\pgfsys@transformshift{2.725935in}{2.334920in}%
\pgfsys@useobject{currentmarker}{}%
\end{pgfscope}%
\end{pgfscope}%
\begin{pgfscope}%
\pgftext[x=2.725935in,y=0.531944in,,top]{\rmfamily\fontsize{10.000000}{12.000000}\selectfont 25}%
\end{pgfscope}%
\begin{pgfscope}%
\pgftext[x=1.723515in,y=0.339043in,,top]{\rmfamily\fontsize{10.000000}{12.000000}\selectfont Time (\(\displaystyle \mu s\))}%
\end{pgfscope}%
\begin{pgfscope}%
\pgfsetbuttcap%
\pgfsetroundjoin%
\definecolor{currentfill}{rgb}{0.000000,0.000000,0.000000}%
\pgfsetfillcolor{currentfill}%
\pgfsetlinewidth{0.501875pt}%
\definecolor{currentstroke}{rgb}{0.000000,0.000000,0.000000}%
\pgfsetstrokecolor{currentstroke}%
\pgfsetdash{}{0pt}%
\pgfsys@defobject{currentmarker}{\pgfqpoint{0.000000in}{0.000000in}}{\pgfqpoint{0.055556in}{0.000000in}}{%
\pgfpathmoveto{\pgfqpoint{0.000000in}{0.000000in}}%
\pgfpathlineto{\pgfqpoint{0.055556in}{0.000000in}}%
\pgfusepath{stroke,fill}%
}%
\begin{pgfscope}%
\pgfsys@transformshift{0.721094in}{0.684579in}%
\pgfsys@useobject{currentmarker}{}%
\end{pgfscope}%
\end{pgfscope}%
\begin{pgfscope}%
\pgfsetbuttcap%
\pgfsetroundjoin%
\definecolor{currentfill}{rgb}{0.000000,0.000000,0.000000}%
\pgfsetfillcolor{currentfill}%
\pgfsetlinewidth{0.501875pt}%
\definecolor{currentstroke}{rgb}{0.000000,0.000000,0.000000}%
\pgfsetstrokecolor{currentstroke}%
\pgfsetdash{}{0pt}%
\pgfsys@defobject{currentmarker}{\pgfqpoint{-0.055556in}{0.000000in}}{\pgfqpoint{0.000000in}{0.000000in}}{%
\pgfpathmoveto{\pgfqpoint{0.000000in}{0.000000in}}%
\pgfpathlineto{\pgfqpoint{-0.055556in}{0.000000in}}%
\pgfusepath{stroke,fill}%
}%
\begin{pgfscope}%
\pgfsys@transformshift{2.725935in}{0.684579in}%
\pgfsys@useobject{currentmarker}{}%
\end{pgfscope}%
\end{pgfscope}%
\begin{pgfscope}%
\pgftext[x=0.665538in,y=0.684579in,right,]{\rmfamily\fontsize{10.000000}{12.000000}\selectfont -100}%
\end{pgfscope}%
\begin{pgfscope}%
\pgfsetbuttcap%
\pgfsetroundjoin%
\definecolor{currentfill}{rgb}{0.000000,0.000000,0.000000}%
\pgfsetfillcolor{currentfill}%
\pgfsetlinewidth{0.501875pt}%
\definecolor{currentstroke}{rgb}{0.000000,0.000000,0.000000}%
\pgfsetstrokecolor{currentstroke}%
\pgfsetdash{}{0pt}%
\pgfsys@defobject{currentmarker}{\pgfqpoint{0.000000in}{0.000000in}}{\pgfqpoint{0.055556in}{0.000000in}}{%
\pgfpathmoveto{\pgfqpoint{0.000000in}{0.000000in}}%
\pgfpathlineto{\pgfqpoint{0.055556in}{0.000000in}}%
\pgfusepath{stroke,fill}%
}%
\begin{pgfscope}%
\pgfsys@transformshift{0.721094in}{1.072894in}%
\pgfsys@useobject{currentmarker}{}%
\end{pgfscope}%
\end{pgfscope}%
\begin{pgfscope}%
\pgfsetbuttcap%
\pgfsetroundjoin%
\definecolor{currentfill}{rgb}{0.000000,0.000000,0.000000}%
\pgfsetfillcolor{currentfill}%
\pgfsetlinewidth{0.501875pt}%
\definecolor{currentstroke}{rgb}{0.000000,0.000000,0.000000}%
\pgfsetstrokecolor{currentstroke}%
\pgfsetdash{}{0pt}%
\pgfsys@defobject{currentmarker}{\pgfqpoint{-0.055556in}{0.000000in}}{\pgfqpoint{0.000000in}{0.000000in}}{%
\pgfpathmoveto{\pgfqpoint{0.000000in}{0.000000in}}%
\pgfpathlineto{\pgfqpoint{-0.055556in}{0.000000in}}%
\pgfusepath{stroke,fill}%
}%
\begin{pgfscope}%
\pgfsys@transformshift{2.725935in}{1.072894in}%
\pgfsys@useobject{currentmarker}{}%
\end{pgfscope}%
\end{pgfscope}%
\begin{pgfscope}%
\pgftext[x=0.665538in,y=1.072894in,right,]{\rmfamily\fontsize{10.000000}{12.000000}\selectfont 0}%
\end{pgfscope}%
\begin{pgfscope}%
\pgfsetbuttcap%
\pgfsetroundjoin%
\definecolor{currentfill}{rgb}{0.000000,0.000000,0.000000}%
\pgfsetfillcolor{currentfill}%
\pgfsetlinewidth{0.501875pt}%
\definecolor{currentstroke}{rgb}{0.000000,0.000000,0.000000}%
\pgfsetstrokecolor{currentstroke}%
\pgfsetdash{}{0pt}%
\pgfsys@defobject{currentmarker}{\pgfqpoint{0.000000in}{0.000000in}}{\pgfqpoint{0.055556in}{0.000000in}}{%
\pgfpathmoveto{\pgfqpoint{0.000000in}{0.000000in}}%
\pgfpathlineto{\pgfqpoint{0.055556in}{0.000000in}}%
\pgfusepath{stroke,fill}%
}%
\begin{pgfscope}%
\pgfsys@transformshift{0.721094in}{1.461210in}%
\pgfsys@useobject{currentmarker}{}%
\end{pgfscope}%
\end{pgfscope}%
\begin{pgfscope}%
\pgfsetbuttcap%
\pgfsetroundjoin%
\definecolor{currentfill}{rgb}{0.000000,0.000000,0.000000}%
\pgfsetfillcolor{currentfill}%
\pgfsetlinewidth{0.501875pt}%
\definecolor{currentstroke}{rgb}{0.000000,0.000000,0.000000}%
\pgfsetstrokecolor{currentstroke}%
\pgfsetdash{}{0pt}%
\pgfsys@defobject{currentmarker}{\pgfqpoint{-0.055556in}{0.000000in}}{\pgfqpoint{0.000000in}{0.000000in}}{%
\pgfpathmoveto{\pgfqpoint{0.000000in}{0.000000in}}%
\pgfpathlineto{\pgfqpoint{-0.055556in}{0.000000in}}%
\pgfusepath{stroke,fill}%
}%
\begin{pgfscope}%
\pgfsys@transformshift{2.725935in}{1.461210in}%
\pgfsys@useobject{currentmarker}{}%
\end{pgfscope}%
\end{pgfscope}%
\begin{pgfscope}%
\pgftext[x=0.665538in,y=1.461210in,right,]{\rmfamily\fontsize{10.000000}{12.000000}\selectfont 100}%
\end{pgfscope}%
\begin{pgfscope}%
\pgfsetbuttcap%
\pgfsetroundjoin%
\definecolor{currentfill}{rgb}{0.000000,0.000000,0.000000}%
\pgfsetfillcolor{currentfill}%
\pgfsetlinewidth{0.501875pt}%
\definecolor{currentstroke}{rgb}{0.000000,0.000000,0.000000}%
\pgfsetstrokecolor{currentstroke}%
\pgfsetdash{}{0pt}%
\pgfsys@defobject{currentmarker}{\pgfqpoint{0.000000in}{0.000000in}}{\pgfqpoint{0.055556in}{0.000000in}}{%
\pgfpathmoveto{\pgfqpoint{0.000000in}{0.000000in}}%
\pgfpathlineto{\pgfqpoint{0.055556in}{0.000000in}}%
\pgfusepath{stroke,fill}%
}%
\begin{pgfscope}%
\pgfsys@transformshift{0.721094in}{1.849525in}%
\pgfsys@useobject{currentmarker}{}%
\end{pgfscope}%
\end{pgfscope}%
\begin{pgfscope}%
\pgfsetbuttcap%
\pgfsetroundjoin%
\definecolor{currentfill}{rgb}{0.000000,0.000000,0.000000}%
\pgfsetfillcolor{currentfill}%
\pgfsetlinewidth{0.501875pt}%
\definecolor{currentstroke}{rgb}{0.000000,0.000000,0.000000}%
\pgfsetstrokecolor{currentstroke}%
\pgfsetdash{}{0pt}%
\pgfsys@defobject{currentmarker}{\pgfqpoint{-0.055556in}{0.000000in}}{\pgfqpoint{0.000000in}{0.000000in}}{%
\pgfpathmoveto{\pgfqpoint{0.000000in}{0.000000in}}%
\pgfpathlineto{\pgfqpoint{-0.055556in}{0.000000in}}%
\pgfusepath{stroke,fill}%
}%
\begin{pgfscope}%
\pgfsys@transformshift{2.725935in}{1.849525in}%
\pgfsys@useobject{currentmarker}{}%
\end{pgfscope}%
\end{pgfscope}%
\begin{pgfscope}%
\pgftext[x=0.665538in,y=1.849525in,right,]{\rmfamily\fontsize{10.000000}{12.000000}\selectfont 200}%
\end{pgfscope}%
\begin{pgfscope}%
\pgfsetbuttcap%
\pgfsetroundjoin%
\definecolor{currentfill}{rgb}{0.000000,0.000000,0.000000}%
\pgfsetfillcolor{currentfill}%
\pgfsetlinewidth{0.501875pt}%
\definecolor{currentstroke}{rgb}{0.000000,0.000000,0.000000}%
\pgfsetstrokecolor{currentstroke}%
\pgfsetdash{}{0pt}%
\pgfsys@defobject{currentmarker}{\pgfqpoint{0.000000in}{0.000000in}}{\pgfqpoint{0.055556in}{0.000000in}}{%
\pgfpathmoveto{\pgfqpoint{0.000000in}{0.000000in}}%
\pgfpathlineto{\pgfqpoint{0.055556in}{0.000000in}}%
\pgfusepath{stroke,fill}%
}%
\begin{pgfscope}%
\pgfsys@transformshift{0.721094in}{2.237841in}%
\pgfsys@useobject{currentmarker}{}%
\end{pgfscope}%
\end{pgfscope}%
\begin{pgfscope}%
\pgfsetbuttcap%
\pgfsetroundjoin%
\definecolor{currentfill}{rgb}{0.000000,0.000000,0.000000}%
\pgfsetfillcolor{currentfill}%
\pgfsetlinewidth{0.501875pt}%
\definecolor{currentstroke}{rgb}{0.000000,0.000000,0.000000}%
\pgfsetstrokecolor{currentstroke}%
\pgfsetdash{}{0pt}%
\pgfsys@defobject{currentmarker}{\pgfqpoint{-0.055556in}{0.000000in}}{\pgfqpoint{0.000000in}{0.000000in}}{%
\pgfpathmoveto{\pgfqpoint{0.000000in}{0.000000in}}%
\pgfpathlineto{\pgfqpoint{-0.055556in}{0.000000in}}%
\pgfusepath{stroke,fill}%
}%
\begin{pgfscope}%
\pgfsys@transformshift{2.725935in}{2.237841in}%
\pgfsys@useobject{currentmarker}{}%
\end{pgfscope}%
\end{pgfscope}%
\begin{pgfscope}%
\pgftext[x=0.665538in,y=2.237841in,right,]{\rmfamily\fontsize{10.000000}{12.000000}\selectfont 300}%
\end{pgfscope}%
\begin{pgfscope}%
\pgftext[x=0.341464in,y=1.461210in,,bottom,rotate=90.000000]{\rmfamily\fontsize{10.000000}{12.000000}\selectfont Deflector Voltage(V)}%
\end{pgfscope}%
\begin{pgfscope}%
\pgfsetbuttcap%
\pgfsetmiterjoin%
\definecolor{currentfill}{rgb}{1.000000,1.000000,1.000000}%
\pgfsetfillcolor{currentfill}%
\pgfsetlinewidth{0.000000pt}%
\definecolor{currentstroke}{rgb}{0.000000,0.000000,0.000000}%
\pgfsetstrokecolor{currentstroke}%
\pgfsetstrokeopacity{0.000000}%
\pgfsetdash{}{0pt}%
\pgfpathmoveto{\pgfqpoint{2.725935in}{0.587500in}}%
\pgfpathlineto{\pgfqpoint{4.730777in}{0.587500in}}%
\pgfpathlineto{\pgfqpoint{4.730777in}{2.334920in}}%
\pgfpathlineto{\pgfqpoint{2.725935in}{2.334920in}}%
\pgfpathclose%
\pgfusepath{fill}%
\end{pgfscope}%
\begin{pgfscope}%
\pgfpathrectangle{\pgfqpoint{2.725935in}{0.587500in}}{\pgfqpoint{2.004842in}{1.747420in}} %
\pgfusepath{clip}%
\pgfsetrectcap%
\pgfsetroundjoin%
\pgfsetlinewidth{1.003750pt}%
\definecolor{currentstroke}{rgb}{0.000000,0.000000,0.000000}%
\pgfsetstrokecolor{currentstroke}%
\pgfsetdash{}{0pt}%
\pgfpathmoveto{\pgfqpoint{2.503176in}{2.212685in}}%
\pgfpathmoveto{\pgfqpoint{2.715935in}{2.216781in}}%
\pgfpathlineto{\pgfqpoint{2.717025in}{2.216777in}}%
\pgfpathlineto{\pgfqpoint{2.734846in}{2.216685in}}%
\pgfpathlineto{\pgfqpoint{2.752667in}{2.216608in}}%
\pgfpathlineto{\pgfqpoint{2.770488in}{2.216576in}}%
\pgfpathlineto{\pgfqpoint{2.788309in}{2.216614in}}%
\pgfpathlineto{\pgfqpoint{2.806129in}{2.216745in}}%
\pgfpathlineto{\pgfqpoint{2.823950in}{2.216986in}}%
\pgfpathlineto{\pgfqpoint{2.841771in}{2.217352in}}%
\pgfpathlineto{\pgfqpoint{2.859592in}{2.217847in}}%
\pgfpathlineto{\pgfqpoint{2.877413in}{2.218465in}}%
\pgfpathlineto{\pgfqpoint{2.895234in}{2.219191in}}%
\pgfpathlineto{\pgfqpoint{2.913054in}{2.220002in}}%
\pgfpathlineto{\pgfqpoint{2.930875in}{2.220863in}}%
\pgfpathlineto{\pgfqpoint{2.948696in}{2.221734in}}%
\pgfpathlineto{\pgfqpoint{2.966517in}{2.222566in}}%
\pgfpathlineto{\pgfqpoint{2.984338in}{2.223304in}}%
\pgfpathlineto{\pgfqpoint{3.002158in}{2.223889in}}%
\pgfpathlineto{\pgfqpoint{3.019979in}{2.224255in}}%
\pgfpathlineto{\pgfqpoint{3.037800in}{2.224338in}}%
\pgfpathlineto{\pgfqpoint{3.055621in}{2.224069in}}%
\pgfpathlineto{\pgfqpoint{3.073442in}{2.223384in}}%
\pgfpathlineto{\pgfqpoint{3.091263in}{2.222214in}}%
\pgfpathlineto{\pgfqpoint{3.109083in}{2.220491in}}%
\pgfpathlineto{\pgfqpoint{3.126904in}{2.218129in}}%
\pgfpathlineto{\pgfqpoint{3.144725in}{2.215024in}}%
\pgfpathlineto{\pgfqpoint{3.162546in}{2.211040in}}%
\pgfpathlineto{\pgfqpoint{3.180367in}{2.206000in}}%
\pgfpathlineto{\pgfqpoint{3.198187in}{2.199687in}}%
\pgfpathlineto{\pgfqpoint{3.216008in}{2.191847in}}%
\pgfpathlineto{\pgfqpoint{3.233829in}{2.182199in}}%
\pgfpathlineto{\pgfqpoint{3.251650in}{2.170447in}}%
\pgfpathlineto{\pgfqpoint{3.269471in}{2.156300in}}%
\pgfpathlineto{\pgfqpoint{3.287291in}{2.139479in}}%
\pgfpathlineto{\pgfqpoint{3.305112in}{2.119734in}}%
\pgfpathlineto{\pgfqpoint{3.322933in}{2.096849in}}%
\pgfpathlineto{\pgfqpoint{3.340754in}{2.070648in}}%
\pgfpathlineto{\pgfqpoint{3.358575in}{2.041002in}}%
\pgfpathlineto{\pgfqpoint{3.376396in}{2.007830in}}%
\pgfpathlineto{\pgfqpoint{3.394216in}{1.971113in}}%
\pgfpathlineto{\pgfqpoint{3.412037in}{1.930894in}}%
\pgfpathlineto{\pgfqpoint{3.429858in}{1.887291in}}%
\pgfpathlineto{\pgfqpoint{3.447679in}{1.840497in}}%
\pgfpathlineto{\pgfqpoint{3.465500in}{1.790779in}}%
\pgfpathlineto{\pgfqpoint{3.483320in}{1.738469in}}%
\pgfpathlineto{\pgfqpoint{3.501141in}{1.683959in}}%
\pgfpathlineto{\pgfqpoint{3.518962in}{1.627685in}}%
\pgfpathlineto{\pgfqpoint{3.536783in}{1.570113in}}%
\pgfpathlineto{\pgfqpoint{3.554604in}{1.511728in}}%
\pgfpathlineto{\pgfqpoint{3.572425in}{1.453019in}}%
\pgfpathlineto{\pgfqpoint{3.590245in}{1.394470in}}%
\pgfpathlineto{\pgfqpoint{3.608066in}{1.336541in}}%
\pgfpathlineto{\pgfqpoint{3.625887in}{1.279663in}}%
\pgfpathlineto{\pgfqpoint{3.643708in}{1.224225in}}%
\pgfpathlineto{\pgfqpoint{3.661529in}{1.170563in}}%
\pgfpathlineto{\pgfqpoint{3.679349in}{1.118962in}}%
\pgfpathlineto{\pgfqpoint{3.697170in}{1.069652in}}%
\pgfpathlineto{\pgfqpoint{3.714991in}{1.022813in}}%
\pgfpathlineto{\pgfqpoint{3.732812in}{0.978579in}}%
\pgfpathlineto{\pgfqpoint{3.750633in}{0.937049in}}%
\pgfpathlineto{\pgfqpoint{3.768453in}{0.898291in}}%
\pgfpathlineto{\pgfqpoint{3.786274in}{0.862349in}}%
\pgfpathlineto{\pgfqpoint{3.804095in}{0.829247in}}%
\pgfpathlineto{\pgfqpoint{3.821916in}{0.798995in}}%
\pgfpathlineto{\pgfqpoint{3.839737in}{0.771587in}}%
\pgfpathlineto{\pgfqpoint{3.857558in}{0.747010in}}%
\pgfpathlineto{\pgfqpoint{3.875378in}{0.725237in}}%
\pgfpathlineto{\pgfqpoint{3.893199in}{0.706240in}}%
\pgfpathlineto{\pgfqpoint{3.911020in}{0.689975in}}%
\pgfpathlineto{\pgfqpoint{3.928841in}{0.676389in}}%
\pgfpathlineto{\pgfqpoint{3.946662in}{0.665411in}}%
\pgfpathlineto{\pgfqpoint{3.964482in}{0.656951in}}%
\pgfpathlineto{\pgfqpoint{3.982303in}{0.650891in}}%
\pgfpathlineto{\pgfqpoint{4.000124in}{0.647083in}}%
\pgfpathlineto{\pgfqpoint{4.017945in}{0.645357in}}%
\pgfpathlineto{\pgfqpoint{4.035766in}{0.645517in}}%
\pgfpathlineto{\pgfqpoint{4.053586in}{0.647358in}}%
\pgfpathlineto{\pgfqpoint{4.071407in}{0.650667in}}%
\pgfpathlineto{\pgfqpoint{4.089228in}{0.655240in}}%
\pgfpathlineto{\pgfqpoint{4.107049in}{0.660890in}}%
\pgfpathlineto{\pgfqpoint{4.124870in}{0.667448in}}%
\pgfpathlineto{\pgfqpoint{4.142691in}{0.674769in}}%
\pgfpathlineto{\pgfqpoint{4.160511in}{0.682729in}}%
\pgfpathlineto{\pgfqpoint{4.178332in}{0.691219in}}%
\pgfpathlineto{\pgfqpoint{4.196153in}{0.700142in}}%
\pgfpathlineto{\pgfqpoint{4.213974in}{0.709411in}}%
\pgfpathlineto{\pgfqpoint{4.231795in}{0.718949in}}%
\pgfpathlineto{\pgfqpoint{4.249615in}{0.728688in}}%
\pgfpathlineto{\pgfqpoint{4.267436in}{0.738576in}}%
\pgfpathlineto{\pgfqpoint{4.285257in}{0.748582in}}%
\pgfpathlineto{\pgfqpoint{4.303078in}{0.758699in}}%
\pgfpathlineto{\pgfqpoint{4.320899in}{0.768942in}}%
\pgfpathlineto{\pgfqpoint{4.338720in}{0.779354in}}%
\pgfpathlineto{\pgfqpoint{4.356540in}{0.789989in}}%
\pgfpathlineto{\pgfqpoint{4.374361in}{0.800905in}}%
\pgfpathlineto{\pgfqpoint{4.392182in}{0.812159in}}%
\pgfpathlineto{\pgfqpoint{4.410003in}{0.823787in}}%
\pgfpathlineto{\pgfqpoint{4.427824in}{0.835807in}}%
\pgfpathlineto{\pgfqpoint{4.445644in}{0.848206in}}%
\pgfpathlineto{\pgfqpoint{4.463465in}{0.860937in}}%
\pgfpathlineto{\pgfqpoint{4.481286in}{0.873923in}}%
\pgfpathlineto{\pgfqpoint{4.499107in}{0.887055in}}%
\pgfpathlineto{\pgfqpoint{4.516928in}{0.900193in}}%
\pgfpathlineto{\pgfqpoint{4.534748in}{0.913170in}}%
\pgfpathlineto{\pgfqpoint{4.552569in}{0.925797in}}%
\pgfpathlineto{\pgfqpoint{4.570390in}{0.937869in}}%
\pgfpathlineto{\pgfqpoint{4.588211in}{0.949175in}}%
\pgfpathlineto{\pgfqpoint{4.606032in}{0.959511in}}%
\pgfpathlineto{\pgfqpoint{4.623853in}{0.968690in}}%
\pgfpathlineto{\pgfqpoint{4.641673in}{0.976553in}}%
\pgfpathlineto{\pgfqpoint{4.659494in}{0.982974in}}%
\pgfpathlineto{\pgfqpoint{4.677315in}{0.987858in}}%
\pgfpathlineto{\pgfqpoint{4.695136in}{0.991144in}}%
\pgfpathlineto{\pgfqpoint{4.712957in}{0.992794in}}%
\pgfusepath{stroke}%
\end{pgfscope}%
\begin{pgfscope}%
\pgfpathrectangle{\pgfqpoint{2.725935in}{0.587500in}}{\pgfqpoint{2.004842in}{1.747420in}} %
\pgfusepath{clip}%
\pgfsetbuttcap%
\pgfsetroundjoin%
\pgfsetlinewidth{1.003750pt}%
\definecolor{currentstroke}{rgb}{0.000000,0.000000,0.000000}%
\pgfsetstrokecolor{currentstroke}%
\pgfsetdash{{1.000000pt}{3.000000pt}}{0.000000pt}%
\pgfpathmoveto{\pgfqpoint{3.394216in}{0.587500in}}%
\pgfpathlineto{\pgfqpoint{3.394216in}{2.334920in}}%
\pgfusepath{stroke}%
\end{pgfscope}%
\begin{pgfscope}%
\pgfsetrectcap%
\pgfsetmiterjoin%
\pgfsetlinewidth{1.003750pt}%
\definecolor{currentstroke}{rgb}{0.000000,0.000000,0.000000}%
\pgfsetstrokecolor{currentstroke}%
\pgfsetdash{}{0pt}%
\pgfpathmoveto{\pgfqpoint{4.730777in}{0.587500in}}%
\pgfpathlineto{\pgfqpoint{4.730777in}{2.334920in}}%
\pgfusepath{stroke}%
\end{pgfscope}%
\begin{pgfscope}%
\pgfsetrectcap%
\pgfsetmiterjoin%
\pgfsetlinewidth{1.003750pt}%
\definecolor{currentstroke}{rgb}{0.000000,0.000000,0.000000}%
\pgfsetstrokecolor{currentstroke}%
\pgfsetdash{}{0pt}%
\pgfpathmoveto{\pgfqpoint{2.725935in}{0.587500in}}%
\pgfpathlineto{\pgfqpoint{2.725935in}{2.334920in}}%
\pgfusepath{stroke}%
\end{pgfscope}%
\begin{pgfscope}%
\pgfsetrectcap%
\pgfsetmiterjoin%
\pgfsetlinewidth{1.003750pt}%
\definecolor{currentstroke}{rgb}{0.000000,0.000000,0.000000}%
\pgfsetstrokecolor{currentstroke}%
\pgfsetdash{}{0pt}%
\pgfpathmoveto{\pgfqpoint{2.725935in}{0.587500in}}%
\pgfpathlineto{\pgfqpoint{4.730777in}{0.587500in}}%
\pgfusepath{stroke}%
\end{pgfscope}%
\begin{pgfscope}%
\pgfsetrectcap%
\pgfsetmiterjoin%
\pgfsetlinewidth{1.003750pt}%
\definecolor{currentstroke}{rgb}{0.000000,0.000000,0.000000}%
\pgfsetstrokecolor{currentstroke}%
\pgfsetdash{}{0pt}%
\pgfpathmoveto{\pgfqpoint{2.725935in}{2.334920in}}%
\pgfpathlineto{\pgfqpoint{4.730777in}{2.334920in}}%
\pgfusepath{stroke}%
\end{pgfscope}%
\begin{pgfscope}%
\pgfsetbuttcap%
\pgfsetroundjoin%
\definecolor{currentfill}{rgb}{0.000000,0.000000,0.000000}%
\pgfsetfillcolor{currentfill}%
\pgfsetlinewidth{0.501875pt}%
\definecolor{currentstroke}{rgb}{0.000000,0.000000,0.000000}%
\pgfsetstrokecolor{currentstroke}%
\pgfsetdash{}{0pt}%
\pgfsys@defobject{currentmarker}{\pgfqpoint{0.000000in}{0.000000in}}{\pgfqpoint{0.000000in}{0.055556in}}{%
\pgfpathmoveto{\pgfqpoint{0.000000in}{0.000000in}}%
\pgfpathlineto{\pgfqpoint{0.000000in}{0.055556in}}%
\pgfusepath{stroke,fill}%
}%
\begin{pgfscope}%
\pgfsys@transformshift{2.725935in}{0.587500in}%
\pgfsys@useobject{currentmarker}{}%
\end{pgfscope}%
\end{pgfscope}%
\begin{pgfscope}%
\pgfsetbuttcap%
\pgfsetroundjoin%
\definecolor{currentfill}{rgb}{0.000000,0.000000,0.000000}%
\pgfsetfillcolor{currentfill}%
\pgfsetlinewidth{0.501875pt}%
\definecolor{currentstroke}{rgb}{0.000000,0.000000,0.000000}%
\pgfsetstrokecolor{currentstroke}%
\pgfsetdash{}{0pt}%
\pgfsys@defobject{currentmarker}{\pgfqpoint{0.000000in}{-0.055556in}}{\pgfqpoint{0.000000in}{0.000000in}}{%
\pgfpathmoveto{\pgfqpoint{0.000000in}{0.000000in}}%
\pgfpathlineto{\pgfqpoint{0.000000in}{-0.055556in}}%
\pgfusepath{stroke,fill}%
}%
\begin{pgfscope}%
\pgfsys@transformshift{2.725935in}{2.334920in}%
\pgfsys@useobject{currentmarker}{}%
\end{pgfscope}%
\end{pgfscope}%
\begin{pgfscope}%
\pgftext[x=2.725935in,y=0.531944in,,top]{\rmfamily\fontsize{10.000000}{12.000000}\selectfont -15}%
\end{pgfscope}%
\begin{pgfscope}%
\pgfsetbuttcap%
\pgfsetroundjoin%
\definecolor{currentfill}{rgb}{0.000000,0.000000,0.000000}%
\pgfsetfillcolor{currentfill}%
\pgfsetlinewidth{0.501875pt}%
\definecolor{currentstroke}{rgb}{0.000000,0.000000,0.000000}%
\pgfsetstrokecolor{currentstroke}%
\pgfsetdash{}{0pt}%
\pgfsys@defobject{currentmarker}{\pgfqpoint{0.000000in}{0.000000in}}{\pgfqpoint{0.000000in}{0.055556in}}{%
\pgfpathmoveto{\pgfqpoint{0.000000in}{0.000000in}}%
\pgfpathlineto{\pgfqpoint{0.000000in}{0.055556in}}%
\pgfusepath{stroke,fill}%
}%
\begin{pgfscope}%
\pgfsys@transformshift{2.948696in}{0.587500in}%
\pgfsys@useobject{currentmarker}{}%
\end{pgfscope}%
\end{pgfscope}%
\begin{pgfscope}%
\pgfsetbuttcap%
\pgfsetroundjoin%
\definecolor{currentfill}{rgb}{0.000000,0.000000,0.000000}%
\pgfsetfillcolor{currentfill}%
\pgfsetlinewidth{0.501875pt}%
\definecolor{currentstroke}{rgb}{0.000000,0.000000,0.000000}%
\pgfsetstrokecolor{currentstroke}%
\pgfsetdash{}{0pt}%
\pgfsys@defobject{currentmarker}{\pgfqpoint{0.000000in}{-0.055556in}}{\pgfqpoint{0.000000in}{0.000000in}}{%
\pgfpathmoveto{\pgfqpoint{0.000000in}{0.000000in}}%
\pgfpathlineto{\pgfqpoint{0.000000in}{-0.055556in}}%
\pgfusepath{stroke,fill}%
}%
\begin{pgfscope}%
\pgfsys@transformshift{2.948696in}{2.334920in}%
\pgfsys@useobject{currentmarker}{}%
\end{pgfscope}%
\end{pgfscope}%
\begin{pgfscope}%
\pgftext[x=2.948696in,y=0.531944in,,top]{\rmfamily\fontsize{10.000000}{12.000000}\selectfont -10}%
\end{pgfscope}%
\begin{pgfscope}%
\pgfsetbuttcap%
\pgfsetroundjoin%
\definecolor{currentfill}{rgb}{0.000000,0.000000,0.000000}%
\pgfsetfillcolor{currentfill}%
\pgfsetlinewidth{0.501875pt}%
\definecolor{currentstroke}{rgb}{0.000000,0.000000,0.000000}%
\pgfsetstrokecolor{currentstroke}%
\pgfsetdash{}{0pt}%
\pgfsys@defobject{currentmarker}{\pgfqpoint{0.000000in}{0.000000in}}{\pgfqpoint{0.000000in}{0.055556in}}{%
\pgfpathmoveto{\pgfqpoint{0.000000in}{0.000000in}}%
\pgfpathlineto{\pgfqpoint{0.000000in}{0.055556in}}%
\pgfusepath{stroke,fill}%
}%
\begin{pgfscope}%
\pgfsys@transformshift{3.171456in}{0.587500in}%
\pgfsys@useobject{currentmarker}{}%
\end{pgfscope}%
\end{pgfscope}%
\begin{pgfscope}%
\pgfsetbuttcap%
\pgfsetroundjoin%
\definecolor{currentfill}{rgb}{0.000000,0.000000,0.000000}%
\pgfsetfillcolor{currentfill}%
\pgfsetlinewidth{0.501875pt}%
\definecolor{currentstroke}{rgb}{0.000000,0.000000,0.000000}%
\pgfsetstrokecolor{currentstroke}%
\pgfsetdash{}{0pt}%
\pgfsys@defobject{currentmarker}{\pgfqpoint{0.000000in}{-0.055556in}}{\pgfqpoint{0.000000in}{0.000000in}}{%
\pgfpathmoveto{\pgfqpoint{0.000000in}{0.000000in}}%
\pgfpathlineto{\pgfqpoint{0.000000in}{-0.055556in}}%
\pgfusepath{stroke,fill}%
}%
\begin{pgfscope}%
\pgfsys@transformshift{3.171456in}{2.334920in}%
\pgfsys@useobject{currentmarker}{}%
\end{pgfscope}%
\end{pgfscope}%
\begin{pgfscope}%
\pgftext[x=3.171456in,y=0.531944in,,top]{\rmfamily\fontsize{10.000000}{12.000000}\selectfont -5}%
\end{pgfscope}%
\begin{pgfscope}%
\pgfsetbuttcap%
\pgfsetroundjoin%
\definecolor{currentfill}{rgb}{0.000000,0.000000,0.000000}%
\pgfsetfillcolor{currentfill}%
\pgfsetlinewidth{0.501875pt}%
\definecolor{currentstroke}{rgb}{0.000000,0.000000,0.000000}%
\pgfsetstrokecolor{currentstroke}%
\pgfsetdash{}{0pt}%
\pgfsys@defobject{currentmarker}{\pgfqpoint{0.000000in}{0.000000in}}{\pgfqpoint{0.000000in}{0.055556in}}{%
\pgfpathmoveto{\pgfqpoint{0.000000in}{0.000000in}}%
\pgfpathlineto{\pgfqpoint{0.000000in}{0.055556in}}%
\pgfusepath{stroke,fill}%
}%
\begin{pgfscope}%
\pgfsys@transformshift{3.394216in}{0.587500in}%
\pgfsys@useobject{currentmarker}{}%
\end{pgfscope}%
\end{pgfscope}%
\begin{pgfscope}%
\pgfsetbuttcap%
\pgfsetroundjoin%
\definecolor{currentfill}{rgb}{0.000000,0.000000,0.000000}%
\pgfsetfillcolor{currentfill}%
\pgfsetlinewidth{0.501875pt}%
\definecolor{currentstroke}{rgb}{0.000000,0.000000,0.000000}%
\pgfsetstrokecolor{currentstroke}%
\pgfsetdash{}{0pt}%
\pgfsys@defobject{currentmarker}{\pgfqpoint{0.000000in}{-0.055556in}}{\pgfqpoint{0.000000in}{0.000000in}}{%
\pgfpathmoveto{\pgfqpoint{0.000000in}{0.000000in}}%
\pgfpathlineto{\pgfqpoint{0.000000in}{-0.055556in}}%
\pgfusepath{stroke,fill}%
}%
\begin{pgfscope}%
\pgfsys@transformshift{3.394216in}{2.334920in}%
\pgfsys@useobject{currentmarker}{}%
\end{pgfscope}%
\end{pgfscope}%
\begin{pgfscope}%
\pgftext[x=3.394216in,y=0.531944in,,top]{\rmfamily\fontsize{10.000000}{12.000000}\selectfont 0}%
\end{pgfscope}%
\begin{pgfscope}%
\pgfsetbuttcap%
\pgfsetroundjoin%
\definecolor{currentfill}{rgb}{0.000000,0.000000,0.000000}%
\pgfsetfillcolor{currentfill}%
\pgfsetlinewidth{0.501875pt}%
\definecolor{currentstroke}{rgb}{0.000000,0.000000,0.000000}%
\pgfsetstrokecolor{currentstroke}%
\pgfsetdash{}{0pt}%
\pgfsys@defobject{currentmarker}{\pgfqpoint{0.000000in}{0.000000in}}{\pgfqpoint{0.000000in}{0.055556in}}{%
\pgfpathmoveto{\pgfqpoint{0.000000in}{0.000000in}}%
\pgfpathlineto{\pgfqpoint{0.000000in}{0.055556in}}%
\pgfusepath{stroke,fill}%
}%
\begin{pgfscope}%
\pgfsys@transformshift{3.616976in}{0.587500in}%
\pgfsys@useobject{currentmarker}{}%
\end{pgfscope}%
\end{pgfscope}%
\begin{pgfscope}%
\pgfsetbuttcap%
\pgfsetroundjoin%
\definecolor{currentfill}{rgb}{0.000000,0.000000,0.000000}%
\pgfsetfillcolor{currentfill}%
\pgfsetlinewidth{0.501875pt}%
\definecolor{currentstroke}{rgb}{0.000000,0.000000,0.000000}%
\pgfsetstrokecolor{currentstroke}%
\pgfsetdash{}{0pt}%
\pgfsys@defobject{currentmarker}{\pgfqpoint{0.000000in}{-0.055556in}}{\pgfqpoint{0.000000in}{0.000000in}}{%
\pgfpathmoveto{\pgfqpoint{0.000000in}{0.000000in}}%
\pgfpathlineto{\pgfqpoint{0.000000in}{-0.055556in}}%
\pgfusepath{stroke,fill}%
}%
\begin{pgfscope}%
\pgfsys@transformshift{3.616976in}{2.334920in}%
\pgfsys@useobject{currentmarker}{}%
\end{pgfscope}%
\end{pgfscope}%
\begin{pgfscope}%
\pgftext[x=3.616976in,y=0.531944in,,top]{\rmfamily\fontsize{10.000000}{12.000000}\selectfont 5}%
\end{pgfscope}%
\begin{pgfscope}%
\pgfsetbuttcap%
\pgfsetroundjoin%
\definecolor{currentfill}{rgb}{0.000000,0.000000,0.000000}%
\pgfsetfillcolor{currentfill}%
\pgfsetlinewidth{0.501875pt}%
\definecolor{currentstroke}{rgb}{0.000000,0.000000,0.000000}%
\pgfsetstrokecolor{currentstroke}%
\pgfsetdash{}{0pt}%
\pgfsys@defobject{currentmarker}{\pgfqpoint{0.000000in}{0.000000in}}{\pgfqpoint{0.000000in}{0.055556in}}{%
\pgfpathmoveto{\pgfqpoint{0.000000in}{0.000000in}}%
\pgfpathlineto{\pgfqpoint{0.000000in}{0.055556in}}%
\pgfusepath{stroke,fill}%
}%
\begin{pgfscope}%
\pgfsys@transformshift{3.839736in}{0.587500in}%
\pgfsys@useobject{currentmarker}{}%
\end{pgfscope}%
\end{pgfscope}%
\begin{pgfscope}%
\pgfsetbuttcap%
\pgfsetroundjoin%
\definecolor{currentfill}{rgb}{0.000000,0.000000,0.000000}%
\pgfsetfillcolor{currentfill}%
\pgfsetlinewidth{0.501875pt}%
\definecolor{currentstroke}{rgb}{0.000000,0.000000,0.000000}%
\pgfsetstrokecolor{currentstroke}%
\pgfsetdash{}{0pt}%
\pgfsys@defobject{currentmarker}{\pgfqpoint{0.000000in}{-0.055556in}}{\pgfqpoint{0.000000in}{0.000000in}}{%
\pgfpathmoveto{\pgfqpoint{0.000000in}{0.000000in}}%
\pgfpathlineto{\pgfqpoint{0.000000in}{-0.055556in}}%
\pgfusepath{stroke,fill}%
}%
\begin{pgfscope}%
\pgfsys@transformshift{3.839736in}{2.334920in}%
\pgfsys@useobject{currentmarker}{}%
\end{pgfscope}%
\end{pgfscope}%
\begin{pgfscope}%
\pgftext[x=3.839736in,y=0.531944in,,top]{\rmfamily\fontsize{10.000000}{12.000000}\selectfont 10}%
\end{pgfscope}%
\begin{pgfscope}%
\pgfsetbuttcap%
\pgfsetroundjoin%
\definecolor{currentfill}{rgb}{0.000000,0.000000,0.000000}%
\pgfsetfillcolor{currentfill}%
\pgfsetlinewidth{0.501875pt}%
\definecolor{currentstroke}{rgb}{0.000000,0.000000,0.000000}%
\pgfsetstrokecolor{currentstroke}%
\pgfsetdash{}{0pt}%
\pgfsys@defobject{currentmarker}{\pgfqpoint{0.000000in}{0.000000in}}{\pgfqpoint{0.000000in}{0.055556in}}{%
\pgfpathmoveto{\pgfqpoint{0.000000in}{0.000000in}}%
\pgfpathlineto{\pgfqpoint{0.000000in}{0.055556in}}%
\pgfusepath{stroke,fill}%
}%
\begin{pgfscope}%
\pgfsys@transformshift{4.062496in}{0.587500in}%
\pgfsys@useobject{currentmarker}{}%
\end{pgfscope}%
\end{pgfscope}%
\begin{pgfscope}%
\pgfsetbuttcap%
\pgfsetroundjoin%
\definecolor{currentfill}{rgb}{0.000000,0.000000,0.000000}%
\pgfsetfillcolor{currentfill}%
\pgfsetlinewidth{0.501875pt}%
\definecolor{currentstroke}{rgb}{0.000000,0.000000,0.000000}%
\pgfsetstrokecolor{currentstroke}%
\pgfsetdash{}{0pt}%
\pgfsys@defobject{currentmarker}{\pgfqpoint{0.000000in}{-0.055556in}}{\pgfqpoint{0.000000in}{0.000000in}}{%
\pgfpathmoveto{\pgfqpoint{0.000000in}{0.000000in}}%
\pgfpathlineto{\pgfqpoint{0.000000in}{-0.055556in}}%
\pgfusepath{stroke,fill}%
}%
\begin{pgfscope}%
\pgfsys@transformshift{4.062496in}{2.334920in}%
\pgfsys@useobject{currentmarker}{}%
\end{pgfscope}%
\end{pgfscope}%
\begin{pgfscope}%
\pgftext[x=4.062496in,y=0.531944in,,top]{\rmfamily\fontsize{10.000000}{12.000000}\selectfont 15}%
\end{pgfscope}%
\begin{pgfscope}%
\pgfsetbuttcap%
\pgfsetroundjoin%
\definecolor{currentfill}{rgb}{0.000000,0.000000,0.000000}%
\pgfsetfillcolor{currentfill}%
\pgfsetlinewidth{0.501875pt}%
\definecolor{currentstroke}{rgb}{0.000000,0.000000,0.000000}%
\pgfsetstrokecolor{currentstroke}%
\pgfsetdash{}{0pt}%
\pgfsys@defobject{currentmarker}{\pgfqpoint{0.000000in}{0.000000in}}{\pgfqpoint{0.000000in}{0.055556in}}{%
\pgfpathmoveto{\pgfqpoint{0.000000in}{0.000000in}}%
\pgfpathlineto{\pgfqpoint{0.000000in}{0.055556in}}%
\pgfusepath{stroke,fill}%
}%
\begin{pgfscope}%
\pgfsys@transformshift{4.285257in}{0.587500in}%
\pgfsys@useobject{currentmarker}{}%
\end{pgfscope}%
\end{pgfscope}%
\begin{pgfscope}%
\pgfsetbuttcap%
\pgfsetroundjoin%
\definecolor{currentfill}{rgb}{0.000000,0.000000,0.000000}%
\pgfsetfillcolor{currentfill}%
\pgfsetlinewidth{0.501875pt}%
\definecolor{currentstroke}{rgb}{0.000000,0.000000,0.000000}%
\pgfsetstrokecolor{currentstroke}%
\pgfsetdash{}{0pt}%
\pgfsys@defobject{currentmarker}{\pgfqpoint{0.000000in}{-0.055556in}}{\pgfqpoint{0.000000in}{0.000000in}}{%
\pgfpathmoveto{\pgfqpoint{0.000000in}{0.000000in}}%
\pgfpathlineto{\pgfqpoint{0.000000in}{-0.055556in}}%
\pgfusepath{stroke,fill}%
}%
\begin{pgfscope}%
\pgfsys@transformshift{4.285257in}{2.334920in}%
\pgfsys@useobject{currentmarker}{}%
\end{pgfscope}%
\end{pgfscope}%
\begin{pgfscope}%
\pgftext[x=4.285257in,y=0.531944in,,top]{\rmfamily\fontsize{10.000000}{12.000000}\selectfont 20}%
\end{pgfscope}%
\begin{pgfscope}%
\pgfsetbuttcap%
\pgfsetroundjoin%
\definecolor{currentfill}{rgb}{0.000000,0.000000,0.000000}%
\pgfsetfillcolor{currentfill}%
\pgfsetlinewidth{0.501875pt}%
\definecolor{currentstroke}{rgb}{0.000000,0.000000,0.000000}%
\pgfsetstrokecolor{currentstroke}%
\pgfsetdash{}{0pt}%
\pgfsys@defobject{currentmarker}{\pgfqpoint{0.000000in}{0.000000in}}{\pgfqpoint{0.000000in}{0.055556in}}{%
\pgfpathmoveto{\pgfqpoint{0.000000in}{0.000000in}}%
\pgfpathlineto{\pgfqpoint{0.000000in}{0.055556in}}%
\pgfusepath{stroke,fill}%
}%
\begin{pgfscope}%
\pgfsys@transformshift{4.508017in}{0.587500in}%
\pgfsys@useobject{currentmarker}{}%
\end{pgfscope}%
\end{pgfscope}%
\begin{pgfscope}%
\pgfsetbuttcap%
\pgfsetroundjoin%
\definecolor{currentfill}{rgb}{0.000000,0.000000,0.000000}%
\pgfsetfillcolor{currentfill}%
\pgfsetlinewidth{0.501875pt}%
\definecolor{currentstroke}{rgb}{0.000000,0.000000,0.000000}%
\pgfsetstrokecolor{currentstroke}%
\pgfsetdash{}{0pt}%
\pgfsys@defobject{currentmarker}{\pgfqpoint{0.000000in}{-0.055556in}}{\pgfqpoint{0.000000in}{0.000000in}}{%
\pgfpathmoveto{\pgfqpoint{0.000000in}{0.000000in}}%
\pgfpathlineto{\pgfqpoint{0.000000in}{-0.055556in}}%
\pgfusepath{stroke,fill}%
}%
\begin{pgfscope}%
\pgfsys@transformshift{4.508017in}{2.334920in}%
\pgfsys@useobject{currentmarker}{}%
\end{pgfscope}%
\end{pgfscope}%
\begin{pgfscope}%
\pgftext[x=4.508017in,y=0.531944in,,top]{\rmfamily\fontsize{10.000000}{12.000000}\selectfont 25}%
\end{pgfscope}%
\begin{pgfscope}%
\pgfsetbuttcap%
\pgfsetroundjoin%
\definecolor{currentfill}{rgb}{0.000000,0.000000,0.000000}%
\pgfsetfillcolor{currentfill}%
\pgfsetlinewidth{0.501875pt}%
\definecolor{currentstroke}{rgb}{0.000000,0.000000,0.000000}%
\pgfsetstrokecolor{currentstroke}%
\pgfsetdash{}{0pt}%
\pgfsys@defobject{currentmarker}{\pgfqpoint{0.000000in}{0.000000in}}{\pgfqpoint{0.000000in}{0.055556in}}{%
\pgfpathmoveto{\pgfqpoint{0.000000in}{0.000000in}}%
\pgfpathlineto{\pgfqpoint{0.000000in}{0.055556in}}%
\pgfusepath{stroke,fill}%
}%
\begin{pgfscope}%
\pgfsys@transformshift{4.730777in}{0.587500in}%
\pgfsys@useobject{currentmarker}{}%
\end{pgfscope}%
\end{pgfscope}%
\begin{pgfscope}%
\pgfsetbuttcap%
\pgfsetroundjoin%
\definecolor{currentfill}{rgb}{0.000000,0.000000,0.000000}%
\pgfsetfillcolor{currentfill}%
\pgfsetlinewidth{0.501875pt}%
\definecolor{currentstroke}{rgb}{0.000000,0.000000,0.000000}%
\pgfsetstrokecolor{currentstroke}%
\pgfsetdash{}{0pt}%
\pgfsys@defobject{currentmarker}{\pgfqpoint{0.000000in}{-0.055556in}}{\pgfqpoint{0.000000in}{0.000000in}}{%
\pgfpathmoveto{\pgfqpoint{0.000000in}{0.000000in}}%
\pgfpathlineto{\pgfqpoint{0.000000in}{-0.055556in}}%
\pgfusepath{stroke,fill}%
}%
\begin{pgfscope}%
\pgfsys@transformshift{4.730777in}{2.334920in}%
\pgfsys@useobject{currentmarker}{}%
\end{pgfscope}%
\end{pgfscope}%
\begin{pgfscope}%
\pgftext[x=4.730777in,y=0.531944in,,top]{\rmfamily\fontsize{10.000000}{12.000000}\selectfont 30}%
\end{pgfscope}%
\begin{pgfscope}%
\pgftext[x=3.728356in,y=0.339043in,,top]{\rmfamily\fontsize{10.000000}{12.000000}\selectfont Time (\(\displaystyle ns\))}%
\end{pgfscope}%
\begin{pgfscope}%
\pgfsetbuttcap%
\pgfsetroundjoin%
\definecolor{currentfill}{rgb}{0.000000,0.000000,0.000000}%
\pgfsetfillcolor{currentfill}%
\pgfsetlinewidth{0.501875pt}%
\definecolor{currentstroke}{rgb}{0.000000,0.000000,0.000000}%
\pgfsetstrokecolor{currentstroke}%
\pgfsetdash{}{0pt}%
\pgfsys@defobject{currentmarker}{\pgfqpoint{0.000000in}{0.000000in}}{\pgfqpoint{0.055556in}{0.000000in}}{%
\pgfpathmoveto{\pgfqpoint{0.000000in}{0.000000in}}%
\pgfpathlineto{\pgfqpoint{0.055556in}{0.000000in}}%
\pgfusepath{stroke,fill}%
}%
\begin{pgfscope}%
\pgfsys@transformshift{2.725935in}{0.684579in}%
\pgfsys@useobject{currentmarker}{}%
\end{pgfscope}%
\end{pgfscope}%
\begin{pgfscope}%
\pgfsetbuttcap%
\pgfsetroundjoin%
\definecolor{currentfill}{rgb}{0.000000,0.000000,0.000000}%
\pgfsetfillcolor{currentfill}%
\pgfsetlinewidth{0.501875pt}%
\definecolor{currentstroke}{rgb}{0.000000,0.000000,0.000000}%
\pgfsetstrokecolor{currentstroke}%
\pgfsetdash{}{0pt}%
\pgfsys@defobject{currentmarker}{\pgfqpoint{-0.055556in}{0.000000in}}{\pgfqpoint{0.000000in}{0.000000in}}{%
\pgfpathmoveto{\pgfqpoint{0.000000in}{0.000000in}}%
\pgfpathlineto{\pgfqpoint{-0.055556in}{0.000000in}}%
\pgfusepath{stroke,fill}%
}%
\begin{pgfscope}%
\pgfsys@transformshift{4.730777in}{0.684579in}%
\pgfsys@useobject{currentmarker}{}%
\end{pgfscope}%
\end{pgfscope}%
\begin{pgfscope}%
\pgfsetbuttcap%
\pgfsetroundjoin%
\definecolor{currentfill}{rgb}{0.000000,0.000000,0.000000}%
\pgfsetfillcolor{currentfill}%
\pgfsetlinewidth{0.501875pt}%
\definecolor{currentstroke}{rgb}{0.000000,0.000000,0.000000}%
\pgfsetstrokecolor{currentstroke}%
\pgfsetdash{}{0pt}%
\pgfsys@defobject{currentmarker}{\pgfqpoint{0.000000in}{0.000000in}}{\pgfqpoint{0.055556in}{0.000000in}}{%
\pgfpathmoveto{\pgfqpoint{0.000000in}{0.000000in}}%
\pgfpathlineto{\pgfqpoint{0.055556in}{0.000000in}}%
\pgfusepath{stroke,fill}%
}%
\begin{pgfscope}%
\pgfsys@transformshift{2.725935in}{1.072894in}%
\pgfsys@useobject{currentmarker}{}%
\end{pgfscope}%
\end{pgfscope}%
\begin{pgfscope}%
\pgfsetbuttcap%
\pgfsetroundjoin%
\definecolor{currentfill}{rgb}{0.000000,0.000000,0.000000}%
\pgfsetfillcolor{currentfill}%
\pgfsetlinewidth{0.501875pt}%
\definecolor{currentstroke}{rgb}{0.000000,0.000000,0.000000}%
\pgfsetstrokecolor{currentstroke}%
\pgfsetdash{}{0pt}%
\pgfsys@defobject{currentmarker}{\pgfqpoint{-0.055556in}{0.000000in}}{\pgfqpoint{0.000000in}{0.000000in}}{%
\pgfpathmoveto{\pgfqpoint{0.000000in}{0.000000in}}%
\pgfpathlineto{\pgfqpoint{-0.055556in}{0.000000in}}%
\pgfusepath{stroke,fill}%
}%
\begin{pgfscope}%
\pgfsys@transformshift{4.730777in}{1.072894in}%
\pgfsys@useobject{currentmarker}{}%
\end{pgfscope}%
\end{pgfscope}%
\begin{pgfscope}%
\pgfsetbuttcap%
\pgfsetroundjoin%
\definecolor{currentfill}{rgb}{0.000000,0.000000,0.000000}%
\pgfsetfillcolor{currentfill}%
\pgfsetlinewidth{0.501875pt}%
\definecolor{currentstroke}{rgb}{0.000000,0.000000,0.000000}%
\pgfsetstrokecolor{currentstroke}%
\pgfsetdash{}{0pt}%
\pgfsys@defobject{currentmarker}{\pgfqpoint{0.000000in}{0.000000in}}{\pgfqpoint{0.055556in}{0.000000in}}{%
\pgfpathmoveto{\pgfqpoint{0.000000in}{0.000000in}}%
\pgfpathlineto{\pgfqpoint{0.055556in}{0.000000in}}%
\pgfusepath{stroke,fill}%
}%
\begin{pgfscope}%
\pgfsys@transformshift{2.725935in}{1.461210in}%
\pgfsys@useobject{currentmarker}{}%
\end{pgfscope}%
\end{pgfscope}%
\begin{pgfscope}%
\pgfsetbuttcap%
\pgfsetroundjoin%
\definecolor{currentfill}{rgb}{0.000000,0.000000,0.000000}%
\pgfsetfillcolor{currentfill}%
\pgfsetlinewidth{0.501875pt}%
\definecolor{currentstroke}{rgb}{0.000000,0.000000,0.000000}%
\pgfsetstrokecolor{currentstroke}%
\pgfsetdash{}{0pt}%
\pgfsys@defobject{currentmarker}{\pgfqpoint{-0.055556in}{0.000000in}}{\pgfqpoint{0.000000in}{0.000000in}}{%
\pgfpathmoveto{\pgfqpoint{0.000000in}{0.000000in}}%
\pgfpathlineto{\pgfqpoint{-0.055556in}{0.000000in}}%
\pgfusepath{stroke,fill}%
}%
\begin{pgfscope}%
\pgfsys@transformshift{4.730777in}{1.461210in}%
\pgfsys@useobject{currentmarker}{}%
\end{pgfscope}%
\end{pgfscope}%
\begin{pgfscope}%
\pgfsetbuttcap%
\pgfsetroundjoin%
\definecolor{currentfill}{rgb}{0.000000,0.000000,0.000000}%
\pgfsetfillcolor{currentfill}%
\pgfsetlinewidth{0.501875pt}%
\definecolor{currentstroke}{rgb}{0.000000,0.000000,0.000000}%
\pgfsetstrokecolor{currentstroke}%
\pgfsetdash{}{0pt}%
\pgfsys@defobject{currentmarker}{\pgfqpoint{0.000000in}{0.000000in}}{\pgfqpoint{0.055556in}{0.000000in}}{%
\pgfpathmoveto{\pgfqpoint{0.000000in}{0.000000in}}%
\pgfpathlineto{\pgfqpoint{0.055556in}{0.000000in}}%
\pgfusepath{stroke,fill}%
}%
\begin{pgfscope}%
\pgfsys@transformshift{2.725935in}{1.849525in}%
\pgfsys@useobject{currentmarker}{}%
\end{pgfscope}%
\end{pgfscope}%
\begin{pgfscope}%
\pgfsetbuttcap%
\pgfsetroundjoin%
\definecolor{currentfill}{rgb}{0.000000,0.000000,0.000000}%
\pgfsetfillcolor{currentfill}%
\pgfsetlinewidth{0.501875pt}%
\definecolor{currentstroke}{rgb}{0.000000,0.000000,0.000000}%
\pgfsetstrokecolor{currentstroke}%
\pgfsetdash{}{0pt}%
\pgfsys@defobject{currentmarker}{\pgfqpoint{-0.055556in}{0.000000in}}{\pgfqpoint{0.000000in}{0.000000in}}{%
\pgfpathmoveto{\pgfqpoint{0.000000in}{0.000000in}}%
\pgfpathlineto{\pgfqpoint{-0.055556in}{0.000000in}}%
\pgfusepath{stroke,fill}%
}%
\begin{pgfscope}%
\pgfsys@transformshift{4.730777in}{1.849525in}%
\pgfsys@useobject{currentmarker}{}%
\end{pgfscope}%
\end{pgfscope}%
\begin{pgfscope}%
\pgfsetbuttcap%
\pgfsetroundjoin%
\definecolor{currentfill}{rgb}{0.000000,0.000000,0.000000}%
\pgfsetfillcolor{currentfill}%
\pgfsetlinewidth{0.501875pt}%
\definecolor{currentstroke}{rgb}{0.000000,0.000000,0.000000}%
\pgfsetstrokecolor{currentstroke}%
\pgfsetdash{}{0pt}%
\pgfsys@defobject{currentmarker}{\pgfqpoint{0.000000in}{0.000000in}}{\pgfqpoint{0.055556in}{0.000000in}}{%
\pgfpathmoveto{\pgfqpoint{0.000000in}{0.000000in}}%
\pgfpathlineto{\pgfqpoint{0.055556in}{0.000000in}}%
\pgfusepath{stroke,fill}%
}%
\begin{pgfscope}%
\pgfsys@transformshift{2.725935in}{2.237841in}%
\pgfsys@useobject{currentmarker}{}%
\end{pgfscope}%
\end{pgfscope}%
\begin{pgfscope}%
\pgfsetbuttcap%
\pgfsetroundjoin%
\definecolor{currentfill}{rgb}{0.000000,0.000000,0.000000}%
\pgfsetfillcolor{currentfill}%
\pgfsetlinewidth{0.501875pt}%
\definecolor{currentstroke}{rgb}{0.000000,0.000000,0.000000}%
\pgfsetstrokecolor{currentstroke}%
\pgfsetdash{}{0pt}%
\pgfsys@defobject{currentmarker}{\pgfqpoint{-0.055556in}{0.000000in}}{\pgfqpoint{0.000000in}{0.000000in}}{%
\pgfpathmoveto{\pgfqpoint{0.000000in}{0.000000in}}%
\pgfpathlineto{\pgfqpoint{-0.055556in}{0.000000in}}%
\pgfusepath{stroke,fill}%
}%
\begin{pgfscope}%
\pgfsys@transformshift{4.730777in}{2.237841in}%
\pgfsys@useobject{currentmarker}{}%
\end{pgfscope}%
\end{pgfscope}%
\end{pgfpicture}%
\makeatother%
\endgroup%

    \caption[Streak deflector voltages.]{The voltages applied to the deflectors for the long duration (left) and short duration (right) bunches. The zero on the time axis refers to the start of the electron bunches as shown in Figure~\ref{figure:streaks}.}
    \label{figure:deflector_voltages}
    % Code and data located in 2017.07.25, plot2error.py
\end{figure}

Synchronisation with the experimental cycle was easily achieved using triggers from the pulse blaster, see Section~\ref{section:pulse_blaster}, and, while the pulse blaster did not have sufficient temporal resolution to correctly trigger the fast ramp precisely, adjusting cable lengths for the trigger signal by a couple of meters provided adequate control.

An interesting effect that limited the potential length of the usable portion of fast streaks was `ringing' in the signal, this is shown in Figure~\ref{figure:ringing}.
Due to the ringing the range of the usable voltage sweep is somewhat truncated as any bunch with a duration longer than the sweep time will oscillate about the end point of the streak.
Fortunately the majority of the streak is usable and bunches with durations shorted than the voltage sweep time are not affected.

\begin{figure}
    \center
    %% Creator: Matplotlib, PGF backend
%%
%% To include the figure in your LaTeX document, write
%%   \input{<filename>.pgf}
%%
%% Make sure the required packages are loaded in your preamble
%%   \usepackage{pgf}
%%
%% Figures using additional raster images can only be included by \input if
%% they are in the same directory as the main LaTeX file. For loading figures
%% from other directories you can use the `import` package
%%   \usepackage{import}
%% and then include the figures with
%%   \import{<path to file>}{<filename>.pgf}
%%
%% Matplotlib used the following preamble
%%
\begingroup%
\makeatletter%
\begin{pgfpicture}%
\pgfpathrectangle{\pgfpointorigin}{\pgfqpoint{5.710000in}{1.903333in}}%
\pgfusepath{use as bounding box, clip}%
\begin{pgfscope}%
\pgfsetbuttcap%
\pgfsetmiterjoin%
\definecolor{currentfill}{rgb}{1.000000,1.000000,1.000000}%
\pgfsetfillcolor{currentfill}%
\pgfsetlinewidth{0.000000pt}%
\definecolor{currentstroke}{rgb}{1.000000,1.000000,1.000000}%
\pgfsetstrokecolor{currentstroke}%
\pgfsetdash{}{0pt}%
\pgfpathmoveto{\pgfqpoint{0.000000in}{0.000000in}}%
\pgfpathlineto{\pgfqpoint{5.710000in}{0.000000in}}%
\pgfpathlineto{\pgfqpoint{5.710000in}{1.903333in}}%
\pgfpathlineto{\pgfqpoint{0.000000in}{1.903333in}}%
\pgfpathclose%
\pgfusepath{fill}%
\end{pgfscope}%
\begin{pgfscope}%
\pgfsetbuttcap%
\pgfsetmiterjoin%
\definecolor{currentfill}{rgb}{1.000000,1.000000,1.000000}%
\pgfsetfillcolor{currentfill}%
\pgfsetlinewidth{0.000000pt}%
\definecolor{currentstroke}{rgb}{0.000000,0.000000,0.000000}%
\pgfsetstrokecolor{currentstroke}%
\pgfsetstrokeopacity{0.000000}%
\pgfsetdash{}{0pt}%
\pgfpathmoveto{\pgfqpoint{3.140547in}{0.528906in}}%
\pgfpathlineto{\pgfqpoint{5.560000in}{0.528906in}}%
\pgfpathlineto{\pgfqpoint{5.560000in}{1.753333in}}%
\pgfpathlineto{\pgfqpoint{3.140547in}{1.753333in}}%
\pgfpathclose%
\pgfusepath{fill}%
\end{pgfscope}%
\begin{pgfscope}%
\pgfpathrectangle{\pgfqpoint{3.140547in}{0.528906in}}{\pgfqpoint{2.419453in}{1.224427in}} %
\pgfusepath{clip}%
\pgftext[at=\pgfqpoint{3.140547in}{0.528906in},left,bottom]{\pgfimage[interpolate=true,width=2.430000in,height=1.230000in]{VoltagesRinging-img0.png}}%
\end{pgfscope}%
\begin{pgfscope}%
\pgfsetrectcap%
\pgfsetmiterjoin%
\pgfsetlinewidth{1.003750pt}%
\definecolor{currentstroke}{rgb}{0.000000,0.000000,0.000000}%
\pgfsetstrokecolor{currentstroke}%
\pgfsetdash{}{0pt}%
\pgfpathmoveto{\pgfqpoint{5.560000in}{0.528906in}}%
\pgfpathlineto{\pgfqpoint{5.560000in}{1.753333in}}%
\pgfusepath{stroke}%
\end{pgfscope}%
\begin{pgfscope}%
\pgfsetrectcap%
\pgfsetmiterjoin%
\pgfsetlinewidth{1.003750pt}%
\definecolor{currentstroke}{rgb}{0.000000,0.000000,0.000000}%
\pgfsetstrokecolor{currentstroke}%
\pgfsetdash{}{0pt}%
\pgfpathmoveto{\pgfqpoint{3.140547in}{1.753333in}}%
\pgfpathlineto{\pgfqpoint{5.560000in}{1.753333in}}%
\pgfusepath{stroke}%
\end{pgfscope}%
\begin{pgfscope}%
\pgfsetrectcap%
\pgfsetmiterjoin%
\pgfsetlinewidth{1.003750pt}%
\definecolor{currentstroke}{rgb}{0.000000,0.000000,0.000000}%
\pgfsetstrokecolor{currentstroke}%
\pgfsetdash{}{0pt}%
\pgfpathmoveto{\pgfqpoint{3.140547in}{0.528906in}}%
\pgfpathlineto{\pgfqpoint{3.140547in}{1.753333in}}%
\pgfusepath{stroke}%
\end{pgfscope}%
\begin{pgfscope}%
\pgfsetrectcap%
\pgfsetmiterjoin%
\pgfsetlinewidth{1.003750pt}%
\definecolor{currentstroke}{rgb}{0.000000,0.000000,0.000000}%
\pgfsetstrokecolor{currentstroke}%
\pgfsetdash{}{0pt}%
\pgfpathmoveto{\pgfqpoint{3.140547in}{0.528906in}}%
\pgfpathlineto{\pgfqpoint{5.560000in}{0.528906in}}%
\pgfusepath{stroke}%
\end{pgfscope}%
\begin{pgfscope}%
\pgfsetbuttcap%
\pgfsetmiterjoin%
\definecolor{currentfill}{rgb}{1.000000,1.000000,1.000000}%
\pgfsetfillcolor{currentfill}%
\pgfsetlinewidth{0.000000pt}%
\definecolor{currentstroke}{rgb}{0.000000,0.000000,0.000000}%
\pgfsetstrokecolor{currentstroke}%
\pgfsetstrokeopacity{0.000000}%
\pgfsetdash{}{0pt}%
\pgfpathmoveto{\pgfqpoint{0.721094in}{0.528906in}}%
\pgfpathlineto{\pgfqpoint{3.140547in}{0.528906in}}%
\pgfpathlineto{\pgfqpoint{3.140547in}{1.753333in}}%
\pgfpathlineto{\pgfqpoint{0.721094in}{1.753333in}}%
\pgfpathclose%
\pgfusepath{fill}%
\end{pgfscope}%
\begin{pgfscope}%
\pgfpathrectangle{\pgfqpoint{0.721094in}{0.528906in}}{\pgfqpoint{2.419453in}{1.224427in}} %
\pgfusepath{clip}%
\pgfsetrectcap%
\pgfsetroundjoin%
\pgfsetlinewidth{1.003750pt}%
\definecolor{currentstroke}{rgb}{0.309804,0.478431,0.682353}%
\pgfsetstrokecolor{currentstroke}%
\pgfsetdash{}{0pt}%
\pgfpathmoveto{\pgfqpoint{0.721094in}{1.615394in}}%
\pgfpathlineto{\pgfqpoint{0.723179in}{1.613984in}}%
\pgfpathlineto{\pgfqpoint{0.728914in}{1.607964in}}%
\pgfpathlineto{\pgfqpoint{0.730999in}{1.609663in}}%
\pgfpathlineto{\pgfqpoint{0.736733in}{1.615661in}}%
\pgfpathlineto{\pgfqpoint{0.738819in}{1.614948in}}%
\pgfpathlineto{\pgfqpoint{0.743511in}{1.612346in}}%
\pgfpathlineto{\pgfqpoint{0.747160in}{1.613969in}}%
\pgfpathlineto{\pgfqpoint{0.756022in}{1.618328in}}%
\pgfpathlineto{\pgfqpoint{0.759150in}{1.617318in}}%
\pgfpathlineto{\pgfqpoint{0.769055in}{1.613297in}}%
\pgfpathlineto{\pgfqpoint{0.774790in}{1.617654in}}%
\pgfpathlineto{\pgfqpoint{0.779482in}{1.618282in}}%
\pgfpathlineto{\pgfqpoint{0.785738in}{1.621773in}}%
\pgfpathlineto{\pgfqpoint{0.787302in}{1.618460in}}%
\pgfpathlineto{\pgfqpoint{0.789387in}{1.606829in}}%
\pgfpathlineto{\pgfqpoint{0.791472in}{1.579468in}}%
\pgfpathlineto{\pgfqpoint{0.794079in}{1.506757in}}%
\pgfpathlineto{\pgfqpoint{0.797728in}{1.315969in}}%
\pgfpathlineto{\pgfqpoint{0.807633in}{0.723504in}}%
\pgfpathlineto{\pgfqpoint{0.811282in}{0.639261in}}%
\pgfpathlineto{\pgfqpoint{0.813368in}{0.626614in}}%
\pgfpathlineto{\pgfqpoint{0.814410in}{0.627876in}}%
\pgfpathlineto{\pgfqpoint{0.816496in}{0.640545in}}%
\pgfpathlineto{\pgfqpoint{0.821187in}{0.691712in}}%
\pgfpathlineto{\pgfqpoint{0.827965in}{0.787317in}}%
\pgfpathlineto{\pgfqpoint{0.832657in}{0.843605in}}%
\pgfpathlineto{\pgfqpoint{0.834742in}{0.850155in}}%
\pgfpathlineto{\pgfqpoint{0.835784in}{0.849856in}}%
\pgfpathlineto{\pgfqpoint{0.837870in}{0.844597in}}%
\pgfpathlineto{\pgfqpoint{0.848296in}{0.799932in}}%
\pgfpathlineto{\pgfqpoint{0.855073in}{0.775169in}}%
\pgfpathlineto{\pgfqpoint{0.857159in}{0.773329in}}%
\pgfpathlineto{\pgfqpoint{0.859765in}{0.775318in}}%
\pgfpathlineto{\pgfqpoint{0.863936in}{0.780807in}}%
\pgfpathlineto{\pgfqpoint{0.869149in}{0.789951in}}%
\pgfpathlineto{\pgfqpoint{0.882182in}{0.796257in}}%
\pgfpathlineto{\pgfqpoint{0.884267in}{0.794356in}}%
\pgfpathlineto{\pgfqpoint{0.890002in}{0.787186in}}%
\pgfpathlineto{\pgfqpoint{0.895736in}{0.787303in}}%
\pgfpathlineto{\pgfqpoint{0.900950in}{0.786844in}}%
\pgfpathlineto{\pgfqpoint{0.922845in}{0.778086in}}%
\pgfpathlineto{\pgfqpoint{0.927537in}{0.773182in}}%
\pgfpathlineto{\pgfqpoint{0.931708in}{0.770989in}}%
\pgfpathlineto{\pgfqpoint{0.935357in}{0.769774in}}%
\pgfpathlineto{\pgfqpoint{0.937963in}{0.768160in}}%
\pgfpathlineto{\pgfqpoint{0.942655in}{0.764469in}}%
\pgfpathlineto{\pgfqpoint{0.950997in}{0.764088in}}%
\pgfpathlineto{\pgfqpoint{0.959338in}{0.756722in}}%
\pgfpathlineto{\pgfqpoint{0.962987in}{0.754002in}}%
\pgfpathlineto{\pgfqpoint{0.966636in}{0.752148in}}%
\pgfpathlineto{\pgfqpoint{0.973413in}{0.750946in}}%
\pgfpathlineto{\pgfqpoint{0.977063in}{0.749884in}}%
\pgfpathlineto{\pgfqpoint{0.981233in}{0.749499in}}%
\pgfpathlineto{\pgfqpoint{0.991660in}{0.741853in}}%
\pgfpathlineto{\pgfqpoint{0.996352in}{0.740535in}}%
\pgfpathlineto{\pgfqpoint{1.011470in}{0.732501in}}%
\pgfpathlineto{\pgfqpoint{1.018247in}{0.732132in}}%
\pgfpathlineto{\pgfqpoint{1.022939in}{0.728830in}}%
\pgfpathlineto{\pgfqpoint{1.027109in}{0.727991in}}%
\pgfpathlineto{\pgfqpoint{1.032844in}{0.724910in}}%
\pgfpathlineto{\pgfqpoint{1.037015in}{0.721905in}}%
\pgfpathlineto{\pgfqpoint{1.041185in}{0.719960in}}%
\pgfpathlineto{\pgfqpoint{1.051612in}{0.717170in}}%
\pgfpathlineto{\pgfqpoint{1.055782in}{0.714872in}}%
\pgfpathlineto{\pgfqpoint{1.059431in}{0.713315in}}%
\pgfpathlineto{\pgfqpoint{1.064645in}{0.709956in}}%
\pgfpathlineto{\pgfqpoint{1.072464in}{0.710269in}}%
\pgfpathlineto{\pgfqpoint{1.075592in}{0.710404in}}%
\pgfpathlineto{\pgfqpoint{1.079242in}{0.710550in}}%
\pgfpathlineto{\pgfqpoint{1.094881in}{0.702606in}}%
\pgfpathlineto{\pgfqpoint{1.098531in}{0.704362in}}%
\pgfpathlineto{\pgfqpoint{1.101658in}{0.704550in}}%
\pgfpathlineto{\pgfqpoint{1.105308in}{0.704748in}}%
\pgfpathlineto{\pgfqpoint{1.108436in}{0.704812in}}%
\pgfpathlineto{\pgfqpoint{1.112606in}{0.703927in}}%
\pgfpathlineto{\pgfqpoint{1.115734in}{0.703312in}}%
\pgfpathlineto{\pgfqpoint{1.122511in}{0.700093in}}%
\pgfpathlineto{\pgfqpoint{1.134502in}{0.699550in}}%
\pgfpathlineto{\pgfqpoint{1.141800in}{0.698466in}}%
\pgfpathlineto{\pgfqpoint{1.144928in}{0.700110in}}%
\pgfpathlineto{\pgfqpoint{1.148577in}{0.701256in}}%
\pgfpathlineto{\pgfqpoint{1.152227in}{0.699854in}}%
\pgfpathlineto{\pgfqpoint{1.156397in}{0.698009in}}%
\pgfpathlineto{\pgfqpoint{1.163174in}{0.699734in}}%
\pgfpathlineto{\pgfqpoint{1.167345in}{0.697552in}}%
\pgfpathlineto{\pgfqpoint{1.177250in}{0.697836in}}%
\pgfpathlineto{\pgfqpoint{1.184027in}{0.700445in}}%
\pgfpathlineto{\pgfqpoint{1.188719in}{0.703541in}}%
\pgfpathlineto{\pgfqpoint{1.192368in}{0.701865in}}%
\pgfpathlineto{\pgfqpoint{1.201231in}{0.698689in}}%
\pgfpathlineto{\pgfqpoint{1.204359in}{0.701537in}}%
\pgfpathlineto{\pgfqpoint{1.208529in}{0.704944in}}%
\pgfpathlineto{\pgfqpoint{1.212700in}{0.704827in}}%
\pgfpathlineto{\pgfqpoint{1.217913in}{0.705849in}}%
\pgfpathlineto{\pgfqpoint{1.224690in}{0.708497in}}%
\pgfpathlineto{\pgfqpoint{1.231468in}{0.709852in}}%
\pgfpathlineto{\pgfqpoint{1.237202in}{0.712844in}}%
\pgfpathlineto{\pgfqpoint{1.242415in}{0.711140in}}%
\pgfpathlineto{\pgfqpoint{1.248150in}{0.712876in}}%
\pgfpathlineto{\pgfqpoint{1.252320in}{0.713804in}}%
\pgfpathlineto{\pgfqpoint{1.257534in}{0.714684in}}%
\pgfpathlineto{\pgfqpoint{1.267439in}{0.714520in}}%
\pgfpathlineto{\pgfqpoint{1.273173in}{0.717400in}}%
\pgfpathlineto{\pgfqpoint{1.278908in}{0.721169in}}%
\pgfpathlineto{\pgfqpoint{1.282557in}{0.721246in}}%
\pgfpathlineto{\pgfqpoint{1.292983in}{0.727658in}}%
\pgfpathlineto{\pgfqpoint{1.297675in}{0.729145in}}%
\pgfpathlineto{\pgfqpoint{1.300282in}{0.730430in}}%
\pgfpathlineto{\pgfqpoint{1.304974in}{0.733862in}}%
\pgfpathlineto{\pgfqpoint{1.311230in}{0.733826in}}%
\pgfpathlineto{\pgfqpoint{1.315400in}{0.736166in}}%
\pgfpathlineto{\pgfqpoint{1.329476in}{0.735600in}}%
\pgfpathlineto{\pgfqpoint{1.337296in}{0.737897in}}%
\pgfpathlineto{\pgfqpoint{1.342509in}{0.737792in}}%
\pgfpathlineto{\pgfqpoint{1.346680in}{0.741978in}}%
\pgfpathlineto{\pgfqpoint{1.350850in}{0.746556in}}%
\pgfpathlineto{\pgfqpoint{1.353457in}{0.745446in}}%
\pgfpathlineto{\pgfqpoint{1.356585in}{0.744513in}}%
\pgfpathlineto{\pgfqpoint{1.365968in}{0.744148in}}%
\pgfpathlineto{\pgfqpoint{1.368054in}{0.743766in}}%
\pgfpathlineto{\pgfqpoint{1.370660in}{0.746624in}}%
\pgfpathlineto{\pgfqpoint{1.376395in}{0.753449in}}%
\pgfpathlineto{\pgfqpoint{1.379001in}{0.753423in}}%
\pgfpathlineto{\pgfqpoint{1.385779in}{0.751418in}}%
\pgfpathlineto{\pgfqpoint{1.389428in}{0.752623in}}%
\pgfpathlineto{\pgfqpoint{1.393599in}{0.753758in}}%
\pgfpathlineto{\pgfqpoint{1.398812in}{0.752825in}}%
\pgfpathlineto{\pgfqpoint{1.410281in}{0.757575in}}%
\pgfpathlineto{\pgfqpoint{1.413930in}{0.756940in}}%
\pgfpathlineto{\pgfqpoint{1.418622in}{0.755897in}}%
\pgfpathlineto{\pgfqpoint{1.426442in}{0.757943in}}%
\pgfpathlineto{\pgfqpoint{1.430612in}{0.756274in}}%
\pgfpathlineto{\pgfqpoint{1.436347in}{0.756821in}}%
\pgfpathlineto{\pgfqpoint{1.438432in}{0.757542in}}%
\pgfpathlineto{\pgfqpoint{1.443645in}{0.761145in}}%
\pgfpathlineto{\pgfqpoint{1.449380in}{0.757781in}}%
\pgfpathlineto{\pgfqpoint{1.458242in}{0.760657in}}%
\pgfpathlineto{\pgfqpoint{1.461370in}{0.759755in}}%
\pgfpathlineto{\pgfqpoint{1.468669in}{0.755952in}}%
\pgfpathlineto{\pgfqpoint{1.472839in}{0.757532in}}%
\pgfpathlineto{\pgfqpoint{1.477010in}{0.759043in}}%
\pgfpathlineto{\pgfqpoint{1.483787in}{0.758416in}}%
\pgfpathlineto{\pgfqpoint{1.494214in}{0.757885in}}%
\pgfpathlineto{\pgfqpoint{1.499427in}{0.754920in}}%
\pgfpathlineto{\pgfqpoint{1.502555in}{0.755732in}}%
\pgfpathlineto{\pgfqpoint{1.509332in}{0.759555in}}%
\pgfpathlineto{\pgfqpoint{1.520280in}{0.759610in}}%
\pgfpathlineto{\pgfqpoint{1.524972in}{0.760559in}}%
\pgfpathlineto{\pgfqpoint{1.527578in}{0.758204in}}%
\pgfpathlineto{\pgfqpoint{1.532791in}{0.752545in}}%
\pgfpathlineto{\pgfqpoint{1.534877in}{0.753003in}}%
\pgfpathlineto{\pgfqpoint{1.539047in}{0.755086in}}%
\pgfpathlineto{\pgfqpoint{1.542175in}{0.752727in}}%
\pgfpathlineto{\pgfqpoint{1.548431in}{0.748378in}}%
\pgfpathlineto{\pgfqpoint{1.559379in}{0.744975in}}%
\pgfpathlineto{\pgfqpoint{1.564071in}{0.745674in}}%
\pgfpathlineto{\pgfqpoint{1.570326in}{0.743773in}}%
\pgfpathlineto{\pgfqpoint{1.573454in}{0.747871in}}%
\pgfpathlineto{\pgfqpoint{1.576061in}{0.749986in}}%
\pgfpathlineto{\pgfqpoint{1.578146in}{0.748295in}}%
\pgfpathlineto{\pgfqpoint{1.585966in}{0.739199in}}%
\pgfpathlineto{\pgfqpoint{1.590137in}{0.737213in}}%
\pgfpathlineto{\pgfqpoint{1.596914in}{0.735644in}}%
\pgfpathlineto{\pgfqpoint{1.600563in}{0.735085in}}%
\pgfpathlineto{\pgfqpoint{1.604212in}{0.734895in}}%
\pgfpathlineto{\pgfqpoint{1.615681in}{0.730406in}}%
\pgfpathlineto{\pgfqpoint{1.622459in}{0.730348in}}%
\pgfpathlineto{\pgfqpoint{1.627151in}{0.726858in}}%
\pgfpathlineto{\pgfqpoint{1.631321in}{0.726704in}}%
\pgfpathlineto{\pgfqpoint{1.635492in}{0.729391in}}%
\pgfpathlineto{\pgfqpoint{1.637577in}{0.729785in}}%
\pgfpathlineto{\pgfqpoint{1.639662in}{0.727577in}}%
\pgfpathlineto{\pgfqpoint{1.643833in}{0.722988in}}%
\pgfpathlineto{\pgfqpoint{1.649567in}{0.721797in}}%
\pgfpathlineto{\pgfqpoint{1.655302in}{0.723248in}}%
\pgfpathlineto{\pgfqpoint{1.659472in}{0.722127in}}%
\pgfpathlineto{\pgfqpoint{1.663122in}{0.720913in}}%
\pgfpathlineto{\pgfqpoint{1.668856in}{0.718781in}}%
\pgfpathlineto{\pgfqpoint{1.694922in}{0.717204in}}%
\pgfpathlineto{\pgfqpoint{1.700136in}{0.712540in}}%
\pgfpathlineto{\pgfqpoint{1.704827in}{0.711724in}}%
\pgfpathlineto{\pgfqpoint{1.707955in}{0.710692in}}%
\pgfpathlineto{\pgfqpoint{1.713690in}{0.711205in}}%
\pgfpathlineto{\pgfqpoint{1.718903in}{0.708260in}}%
\pgfpathlineto{\pgfqpoint{1.725159in}{0.708426in}}%
\pgfpathlineto{\pgfqpoint{1.737671in}{0.708849in}}%
\pgfpathlineto{\pgfqpoint{1.743405in}{0.709487in}}%
\pgfpathlineto{\pgfqpoint{1.747054in}{0.705569in}}%
\pgfpathlineto{\pgfqpoint{1.750182in}{0.704053in}}%
\pgfpathlineto{\pgfqpoint{1.769993in}{0.704228in}}%
\pgfpathlineto{\pgfqpoint{1.774685in}{0.705344in}}%
\pgfpathlineto{\pgfqpoint{1.780940in}{0.703875in}}%
\pgfpathlineto{\pgfqpoint{1.787196in}{0.702097in}}%
\pgfpathlineto{\pgfqpoint{1.790845in}{0.700338in}}%
\pgfpathlineto{\pgfqpoint{1.792931in}{0.702697in}}%
\pgfpathlineto{\pgfqpoint{1.796580in}{0.706849in}}%
\pgfpathlineto{\pgfqpoint{1.810134in}{0.707248in}}%
\pgfpathlineto{\pgfqpoint{1.813262in}{0.705316in}}%
\pgfpathlineto{\pgfqpoint{1.816390in}{0.703942in}}%
\pgfpathlineto{\pgfqpoint{1.818997in}{0.706376in}}%
\pgfpathlineto{\pgfqpoint{1.822646in}{0.708905in}}%
\pgfpathlineto{\pgfqpoint{1.828381in}{0.709389in}}%
\pgfpathlineto{\pgfqpoint{1.833073in}{0.709299in}}%
\pgfpathlineto{\pgfqpoint{1.844020in}{0.712285in}}%
\pgfpathlineto{\pgfqpoint{1.849234in}{0.710958in}}%
\pgfpathlineto{\pgfqpoint{1.860703in}{0.711159in}}%
\pgfpathlineto{\pgfqpoint{1.863831in}{0.712214in}}%
\pgfpathlineto{\pgfqpoint{1.873736in}{0.719033in}}%
\pgfpathlineto{\pgfqpoint{1.876864in}{0.716962in}}%
\pgfpathlineto{\pgfqpoint{1.880513in}{0.715463in}}%
\pgfpathlineto{\pgfqpoint{1.893025in}{0.717409in}}%
\pgfpathlineto{\pgfqpoint{1.902930in}{0.723292in}}%
\pgfpathlineto{\pgfqpoint{1.913356in}{0.724346in}}%
\pgfpathlineto{\pgfqpoint{1.920133in}{0.724279in}}%
\pgfpathlineto{\pgfqpoint{1.923261in}{0.725946in}}%
\pgfpathlineto{\pgfqpoint{1.928474in}{0.729708in}}%
\pgfpathlineto{\pgfqpoint{1.933166in}{0.728813in}}%
\pgfpathlineto{\pgfqpoint{1.938901in}{0.730687in}}%
\pgfpathlineto{\pgfqpoint{1.946721in}{0.731015in}}%
\pgfpathlineto{\pgfqpoint{1.950891in}{0.731992in}}%
\pgfpathlineto{\pgfqpoint{1.954540in}{0.733365in}}%
\pgfpathlineto{\pgfqpoint{1.960275in}{0.735289in}}%
\pgfpathlineto{\pgfqpoint{1.966531in}{0.734832in}}%
\pgfpathlineto{\pgfqpoint{1.970701in}{0.736126in}}%
\pgfpathlineto{\pgfqpoint{1.973308in}{0.736113in}}%
\pgfpathlineto{\pgfqpoint{1.977479in}{0.735089in}}%
\pgfpathlineto{\pgfqpoint{1.983734in}{0.738223in}}%
\pgfpathlineto{\pgfqpoint{1.986862in}{0.738510in}}%
\pgfpathlineto{\pgfqpoint{1.991554in}{0.742624in}}%
\pgfpathlineto{\pgfqpoint{1.994682in}{0.744439in}}%
\pgfpathlineto{\pgfqpoint{1.996767in}{0.743124in}}%
\pgfpathlineto{\pgfqpoint{2.000938in}{0.740062in}}%
\pgfpathlineto{\pgfqpoint{2.006673in}{0.742937in}}%
\pgfpathlineto{\pgfqpoint{2.016056in}{0.746697in}}%
\pgfpathlineto{\pgfqpoint{2.023876in}{0.744635in}}%
\pgfpathlineto{\pgfqpoint{2.028568in}{0.743283in}}%
\pgfpathlineto{\pgfqpoint{2.031696in}{0.746010in}}%
\pgfpathlineto{\pgfqpoint{2.034824in}{0.747486in}}%
\pgfpathlineto{\pgfqpoint{2.040037in}{0.747306in}}%
\pgfpathlineto{\pgfqpoint{2.045250in}{0.750451in}}%
\pgfpathlineto{\pgfqpoint{2.049942in}{0.749027in}}%
\pgfpathlineto{\pgfqpoint{2.054113in}{0.748478in}}%
\pgfpathlineto{\pgfqpoint{2.064539in}{0.751759in}}%
\pgfpathlineto{\pgfqpoint{2.073402in}{0.754256in}}%
\pgfpathlineto{\pgfqpoint{2.076530in}{0.752424in}}%
\pgfpathlineto{\pgfqpoint{2.082786in}{0.749341in}}%
\pgfpathlineto{\pgfqpoint{2.085392in}{0.749230in}}%
\pgfpathlineto{\pgfqpoint{2.092691in}{0.752141in}}%
\pgfpathlineto{\pgfqpoint{2.097383in}{0.749600in}}%
\pgfpathlineto{\pgfqpoint{2.101032in}{0.748933in}}%
\pgfpathlineto{\pgfqpoint{2.106766in}{0.746060in}}%
\pgfpathlineto{\pgfqpoint{2.116150in}{0.745970in}}%
\pgfpathlineto{\pgfqpoint{2.119799in}{0.742863in}}%
\pgfpathlineto{\pgfqpoint{2.121885in}{0.744165in}}%
\pgfpathlineto{\pgfqpoint{2.125013in}{0.746344in}}%
\pgfpathlineto{\pgfqpoint{2.131790in}{0.746931in}}%
\pgfpathlineto{\pgfqpoint{2.135439in}{0.748347in}}%
\pgfpathlineto{\pgfqpoint{2.142216in}{0.747239in}}%
\pgfpathlineto{\pgfqpoint{2.145865in}{0.748790in}}%
\pgfpathlineto{\pgfqpoint{2.148472in}{0.747423in}}%
\pgfpathlineto{\pgfqpoint{2.154207in}{0.743760in}}%
\pgfpathlineto{\pgfqpoint{2.157335in}{0.745526in}}%
\pgfpathlineto{\pgfqpoint{2.159941in}{0.745783in}}%
\pgfpathlineto{\pgfqpoint{2.164633in}{0.744667in}}%
\pgfpathlineto{\pgfqpoint{2.168804in}{0.745325in}}%
\pgfpathlineto{\pgfqpoint{2.171932in}{0.742709in}}%
\pgfpathlineto{\pgfqpoint{2.175059in}{0.740641in}}%
\pgfpathlineto{\pgfqpoint{2.177666in}{0.741954in}}%
\pgfpathlineto{\pgfqpoint{2.180794in}{0.743131in}}%
\pgfpathlineto{\pgfqpoint{2.193306in}{0.739631in}}%
\pgfpathlineto{\pgfqpoint{2.202689in}{0.735379in}}%
\pgfpathlineto{\pgfqpoint{2.208945in}{0.737322in}}%
\pgfpathlineto{\pgfqpoint{2.213116in}{0.737704in}}%
\pgfpathlineto{\pgfqpoint{2.215723in}{0.735543in}}%
\pgfpathlineto{\pgfqpoint{2.220936in}{0.729808in}}%
\pgfpathlineto{\pgfqpoint{2.223542in}{0.730611in}}%
\pgfpathlineto{\pgfqpoint{2.230841in}{0.733994in}}%
\pgfpathlineto{\pgfqpoint{2.238139in}{0.731110in}}%
\pgfpathlineto{\pgfqpoint{2.242831in}{0.730524in}}%
\pgfpathlineto{\pgfqpoint{2.248044in}{0.728827in}}%
\pgfpathlineto{\pgfqpoint{2.252736in}{0.729241in}}%
\pgfpathlineto{\pgfqpoint{2.258992in}{0.726611in}}%
\pgfpathlineto{\pgfqpoint{2.264727in}{0.728383in}}%
\pgfpathlineto{\pgfqpoint{2.267855in}{0.726351in}}%
\pgfpathlineto{\pgfqpoint{2.273589in}{0.721907in}}%
\pgfpathlineto{\pgfqpoint{2.276717in}{0.722936in}}%
\pgfpathlineto{\pgfqpoint{2.280888in}{0.724220in}}%
\pgfpathlineto{\pgfqpoint{2.293921in}{0.722194in}}%
\pgfpathlineto{\pgfqpoint{2.299134in}{0.719530in}}%
\pgfpathlineto{\pgfqpoint{2.302262in}{0.719530in}}%
\pgfpathlineto{\pgfqpoint{2.307996in}{0.716792in}}%
\pgfpathlineto{\pgfqpoint{2.311124in}{0.717308in}}%
\pgfpathlineto{\pgfqpoint{2.316859in}{0.715117in}}%
\pgfpathlineto{\pgfqpoint{2.324157in}{0.719644in}}%
\pgfpathlineto{\pgfqpoint{2.326764in}{0.716948in}}%
\pgfpathlineto{\pgfqpoint{2.330935in}{0.712459in}}%
\pgfpathlineto{\pgfqpoint{2.336669in}{0.712599in}}%
\pgfpathlineto{\pgfqpoint{2.347617in}{0.713935in}}%
\pgfpathlineto{\pgfqpoint{2.353873in}{0.710772in}}%
\pgfpathlineto{\pgfqpoint{2.367948in}{0.710777in}}%
\pgfpathlineto{\pgfqpoint{2.370555in}{0.711045in}}%
\pgfpathlineto{\pgfqpoint{2.377332in}{0.713781in}}%
\pgfpathlineto{\pgfqpoint{2.382024in}{0.713644in}}%
\pgfpathlineto{\pgfqpoint{2.385673in}{0.712947in}}%
\pgfpathlineto{\pgfqpoint{2.391408in}{0.710450in}}%
\pgfpathlineto{\pgfqpoint{2.398185in}{0.710794in}}%
\pgfpathlineto{\pgfqpoint{2.412261in}{0.711471in}}%
\pgfpathlineto{\pgfqpoint{2.417995in}{0.709639in}}%
\pgfpathlineto{\pgfqpoint{2.420602in}{0.712899in}}%
\pgfpathlineto{\pgfqpoint{2.424251in}{0.717138in}}%
\pgfpathlineto{\pgfqpoint{2.426336in}{0.716111in}}%
\pgfpathlineto{\pgfqpoint{2.431028in}{0.712754in}}%
\pgfpathlineto{\pgfqpoint{2.435199in}{0.713331in}}%
\pgfpathlineto{\pgfqpoint{2.441455in}{0.715120in}}%
\pgfpathlineto{\pgfqpoint{2.451881in}{0.712612in}}%
\pgfpathlineto{\pgfqpoint{2.456052in}{0.711943in}}%
\pgfpathlineto{\pgfqpoint{2.459180in}{0.712961in}}%
\pgfpathlineto{\pgfqpoint{2.463872in}{0.715157in}}%
\pgfpathlineto{\pgfqpoint{2.467521in}{0.716205in}}%
\pgfpathlineto{\pgfqpoint{2.470649in}{0.716839in}}%
\pgfpathlineto{\pgfqpoint{2.480554in}{0.712587in}}%
\pgfpathlineto{\pgfqpoint{2.491502in}{0.716726in}}%
\pgfpathlineto{\pgfqpoint{2.495151in}{0.719348in}}%
\pgfpathlineto{\pgfqpoint{2.516525in}{0.722529in}}%
\pgfpathlineto{\pgfqpoint{2.526430in}{0.720728in}}%
\pgfpathlineto{\pgfqpoint{2.531643in}{0.721993in}}%
\pgfpathlineto{\pgfqpoint{2.534771in}{0.722581in}}%
\pgfpathlineto{\pgfqpoint{2.539463in}{0.724112in}}%
\pgfpathlineto{\pgfqpoint{2.543634in}{0.724129in}}%
\pgfpathlineto{\pgfqpoint{2.548326in}{0.726652in}}%
\pgfpathlineto{\pgfqpoint{2.555103in}{0.724925in}}%
\pgfpathlineto{\pgfqpoint{2.563444in}{0.730805in}}%
\pgfpathlineto{\pgfqpoint{2.568657in}{0.731788in}}%
\pgfpathlineto{\pgfqpoint{2.574392in}{0.731765in}}%
\pgfpathlineto{\pgfqpoint{2.597330in}{0.733121in}}%
\pgfpathlineto{\pgfqpoint{2.608278in}{0.733154in}}%
\pgfpathlineto{\pgfqpoint{2.611406in}{0.735364in}}%
\pgfpathlineto{\pgfqpoint{2.615576in}{0.738444in}}%
\pgfpathlineto{\pgfqpoint{2.618183in}{0.737587in}}%
\pgfpathlineto{\pgfqpoint{2.621832in}{0.736163in}}%
\pgfpathlineto{\pgfqpoint{2.628609in}{0.737825in}}%
\pgfpathlineto{\pgfqpoint{2.631737in}{0.738437in}}%
\pgfpathlineto{\pgfqpoint{2.636429in}{0.739936in}}%
\pgfpathlineto{\pgfqpoint{2.640600in}{0.740246in}}%
\pgfpathlineto{\pgfqpoint{2.646855in}{0.743809in}}%
\pgfpathlineto{\pgfqpoint{2.650505in}{0.742010in}}%
\pgfpathlineto{\pgfqpoint{2.656239in}{0.738159in}}%
\pgfpathlineto{\pgfqpoint{2.659888in}{0.739852in}}%
\pgfpathlineto{\pgfqpoint{2.669272in}{0.742397in}}%
\pgfpathlineto{\pgfqpoint{2.675528in}{0.741543in}}%
\pgfpathlineto{\pgfqpoint{2.680741in}{0.743056in}}%
\pgfpathlineto{\pgfqpoint{2.686476in}{0.741465in}}%
\pgfpathlineto{\pgfqpoint{2.689082in}{0.741009in}}%
\pgfpathlineto{\pgfqpoint{2.692732in}{0.739599in}}%
\pgfpathlineto{\pgfqpoint{2.701594in}{0.743111in}}%
\pgfpathlineto{\pgfqpoint{2.705765in}{0.742751in}}%
\pgfpathlineto{\pgfqpoint{2.708893in}{0.743575in}}%
\pgfpathlineto{\pgfqpoint{2.712542in}{0.744875in}}%
\pgfpathlineto{\pgfqpoint{2.720883in}{0.742688in}}%
\pgfpathlineto{\pgfqpoint{2.724011in}{0.741258in}}%
\pgfpathlineto{\pgfqpoint{2.729224in}{0.740375in}}%
\pgfpathlineto{\pgfqpoint{2.732352in}{0.740032in}}%
\pgfpathlineto{\pgfqpoint{2.734959in}{0.742927in}}%
\pgfpathlineto{\pgfqpoint{2.738608in}{0.746453in}}%
\pgfpathlineto{\pgfqpoint{2.740693in}{0.745486in}}%
\pgfpathlineto{\pgfqpoint{2.745385in}{0.742156in}}%
\pgfpathlineto{\pgfqpoint{2.756333in}{0.742272in}}%
\pgfpathlineto{\pgfqpoint{2.764153in}{0.746385in}}%
\pgfpathlineto{\pgfqpoint{2.771451in}{0.740037in}}%
\pgfpathlineto{\pgfqpoint{2.774058in}{0.741483in}}%
\pgfpathlineto{\pgfqpoint{2.777186in}{0.743316in}}%
\pgfpathlineto{\pgfqpoint{2.779271in}{0.741692in}}%
\pgfpathlineto{\pgfqpoint{2.782399in}{0.739579in}}%
\pgfpathlineto{\pgfqpoint{2.787612in}{0.740893in}}%
\pgfpathlineto{\pgfqpoint{2.791783in}{0.742210in}}%
\pgfpathlineto{\pgfqpoint{2.798039in}{0.740609in}}%
\pgfpathlineto{\pgfqpoint{2.802209in}{0.740692in}}%
\pgfpathlineto{\pgfqpoint{2.808465in}{0.738961in}}%
\pgfpathlineto{\pgfqpoint{2.812636in}{0.736990in}}%
\pgfpathlineto{\pgfqpoint{2.818370in}{0.738208in}}%
\pgfpathlineto{\pgfqpoint{2.821498in}{0.737675in}}%
\pgfpathlineto{\pgfqpoint{2.826190in}{0.737835in}}%
\pgfpathlineto{\pgfqpoint{2.835574in}{0.735502in}}%
\pgfpathlineto{\pgfqpoint{2.839744in}{0.734851in}}%
\pgfpathlineto{\pgfqpoint{2.844958in}{0.734630in}}%
\pgfpathlineto{\pgfqpoint{2.850171in}{0.734503in}}%
\pgfpathlineto{\pgfqpoint{2.859033in}{0.730738in}}%
\pgfpathlineto{\pgfqpoint{2.863204in}{0.730791in}}%
\pgfpathlineto{\pgfqpoint{2.878322in}{0.731763in}}%
\pgfpathlineto{\pgfqpoint{2.881450in}{0.730214in}}%
\pgfpathlineto{\pgfqpoint{2.885099in}{0.728153in}}%
\pgfpathlineto{\pgfqpoint{2.889791in}{0.729106in}}%
\pgfpathlineto{\pgfqpoint{2.893962in}{0.723799in}}%
\pgfpathlineto{\pgfqpoint{2.897090in}{0.721821in}}%
\pgfpathlineto{\pgfqpoint{2.916379in}{0.724693in}}%
\pgfpathlineto{\pgfqpoint{2.919507in}{0.724333in}}%
\pgfpathlineto{\pgfqpoint{2.925241in}{0.722292in}}%
\pgfpathlineto{\pgfqpoint{2.931497in}{0.723493in}}%
\pgfpathlineto{\pgfqpoint{2.940359in}{0.717959in}}%
\pgfpathlineto{\pgfqpoint{2.943487in}{0.718067in}}%
\pgfpathlineto{\pgfqpoint{2.947658in}{0.718687in}}%
\pgfpathlineto{\pgfqpoint{2.954435in}{0.716260in}}%
\pgfpathlineto{\pgfqpoint{2.964340in}{0.718414in}}%
\pgfpathlineto{\pgfqpoint{2.973203in}{0.717084in}}%
\pgfpathlineto{\pgfqpoint{2.983629in}{0.717050in}}%
\pgfpathlineto{\pgfqpoint{2.986757in}{0.715545in}}%
\pgfpathlineto{\pgfqpoint{2.996662in}{0.715510in}}%
\pgfpathlineto{\pgfqpoint{3.012302in}{0.711595in}}%
\pgfpathlineto{\pgfqpoint{3.015951in}{0.709337in}}%
\pgfpathlineto{\pgfqpoint{3.018558in}{0.710843in}}%
\pgfpathlineto{\pgfqpoint{3.024292in}{0.715528in}}%
\pgfpathlineto{\pgfqpoint{3.027420in}{0.712961in}}%
\pgfpathlineto{\pgfqpoint{3.030027in}{0.711893in}}%
\pgfpathlineto{\pgfqpoint{3.055050in}{0.716476in}}%
\pgfpathlineto{\pgfqpoint{3.059742in}{0.713938in}}%
\pgfpathlineto{\pgfqpoint{3.063913in}{0.714329in}}%
\pgfpathlineto{\pgfqpoint{3.069126in}{0.711783in}}%
\pgfpathlineto{\pgfqpoint{3.076424in}{0.714906in}}%
\pgfpathlineto{\pgfqpoint{3.083723in}{0.715186in}}%
\pgfpathlineto{\pgfqpoint{3.089457in}{0.715475in}}%
\pgfpathlineto{\pgfqpoint{3.094149in}{0.716042in}}%
\pgfpathlineto{\pgfqpoint{3.100405in}{0.718786in}}%
\pgfpathlineto{\pgfqpoint{3.117087in}{0.718908in}}%
\pgfpathlineto{\pgfqpoint{3.121258in}{0.717577in}}%
\pgfpathlineto{\pgfqpoint{3.136898in}{0.725724in}}%
\pgfpathlineto{\pgfqpoint{3.140547in}{0.722059in}}%
\pgfpathlineto{\pgfqpoint{3.140547in}{0.722059in}}%
\pgfusepath{stroke}%
\end{pgfscope}%
\begin{pgfscope}%
\pgfsetrectcap%
\pgfsetmiterjoin%
\pgfsetlinewidth{1.003750pt}%
\definecolor{currentstroke}{rgb}{0.000000,0.000000,0.000000}%
\pgfsetstrokecolor{currentstroke}%
\pgfsetdash{}{0pt}%
\pgfpathmoveto{\pgfqpoint{3.140547in}{0.528906in}}%
\pgfpathlineto{\pgfqpoint{3.140547in}{1.753333in}}%
\pgfusepath{stroke}%
\end{pgfscope}%
\begin{pgfscope}%
\pgfsetrectcap%
\pgfsetmiterjoin%
\pgfsetlinewidth{1.003750pt}%
\definecolor{currentstroke}{rgb}{0.000000,0.000000,0.000000}%
\pgfsetstrokecolor{currentstroke}%
\pgfsetdash{}{0pt}%
\pgfpathmoveto{\pgfqpoint{0.721094in}{1.753333in}}%
\pgfpathlineto{\pgfqpoint{3.140547in}{1.753333in}}%
\pgfusepath{stroke}%
\end{pgfscope}%
\begin{pgfscope}%
\pgfsetrectcap%
\pgfsetmiterjoin%
\pgfsetlinewidth{1.003750pt}%
\definecolor{currentstroke}{rgb}{0.000000,0.000000,0.000000}%
\pgfsetstrokecolor{currentstroke}%
\pgfsetdash{}{0pt}%
\pgfpathmoveto{\pgfqpoint{0.721094in}{0.528906in}}%
\pgfpathlineto{\pgfqpoint{0.721094in}{1.753333in}}%
\pgfusepath{stroke}%
\end{pgfscope}%
\begin{pgfscope}%
\pgfsetrectcap%
\pgfsetmiterjoin%
\pgfsetlinewidth{1.003750pt}%
\definecolor{currentstroke}{rgb}{0.000000,0.000000,0.000000}%
\pgfsetstrokecolor{currentstroke}%
\pgfsetdash{}{0pt}%
\pgfpathmoveto{\pgfqpoint{0.721094in}{0.528906in}}%
\pgfpathlineto{\pgfqpoint{3.140547in}{0.528906in}}%
\pgfusepath{stroke}%
\end{pgfscope}%
\begin{pgfscope}%
\pgfsetbuttcap%
\pgfsetroundjoin%
\definecolor{currentfill}{rgb}{0.000000,0.000000,0.000000}%
\pgfsetfillcolor{currentfill}%
\pgfsetlinewidth{0.501875pt}%
\definecolor{currentstroke}{rgb}{0.000000,0.000000,0.000000}%
\pgfsetstrokecolor{currentstroke}%
\pgfsetdash{}{0pt}%
\pgfsys@defobject{currentmarker}{\pgfqpoint{0.000000in}{0.000000in}}{\pgfqpoint{0.000000in}{0.055556in}}{%
\pgfpathmoveto{\pgfqpoint{0.000000in}{0.000000in}}%
\pgfpathlineto{\pgfqpoint{0.000000in}{0.055556in}}%
\pgfusepath{stroke,fill}%
}%
\begin{pgfscope}%
\pgfsys@transformshift{0.795121in}{0.528906in}%
\pgfsys@useobject{currentmarker}{}%
\end{pgfscope}%
\end{pgfscope}%
\begin{pgfscope}%
\pgfsetbuttcap%
\pgfsetroundjoin%
\definecolor{currentfill}{rgb}{0.000000,0.000000,0.000000}%
\pgfsetfillcolor{currentfill}%
\pgfsetlinewidth{0.501875pt}%
\definecolor{currentstroke}{rgb}{0.000000,0.000000,0.000000}%
\pgfsetstrokecolor{currentstroke}%
\pgfsetdash{}{0pt}%
\pgfsys@defobject{currentmarker}{\pgfqpoint{0.000000in}{-0.055556in}}{\pgfqpoint{0.000000in}{0.000000in}}{%
\pgfpathmoveto{\pgfqpoint{0.000000in}{0.000000in}}%
\pgfpathlineto{\pgfqpoint{0.000000in}{-0.055556in}}%
\pgfusepath{stroke,fill}%
}%
\begin{pgfscope}%
\pgfsys@transformshift{0.795121in}{1.753333in}%
\pgfsys@useobject{currentmarker}{}%
\end{pgfscope}%
\end{pgfscope}%
\begin{pgfscope}%
\pgftext[x=0.795121in,y=0.473351in,,top]{\rmfamily\fontsize{10.000000}{12.000000}\selectfont 0}%
\end{pgfscope}%
\begin{pgfscope}%
\pgfsetbuttcap%
\pgfsetroundjoin%
\definecolor{currentfill}{rgb}{0.000000,0.000000,0.000000}%
\pgfsetfillcolor{currentfill}%
\pgfsetlinewidth{0.501875pt}%
\definecolor{currentstroke}{rgb}{0.000000,0.000000,0.000000}%
\pgfsetstrokecolor{currentstroke}%
\pgfsetdash{}{0pt}%
\pgfsys@defobject{currentmarker}{\pgfqpoint{0.000000in}{0.000000in}}{\pgfqpoint{0.000000in}{0.055556in}}{%
\pgfpathmoveto{\pgfqpoint{0.000000in}{0.000000in}}%
\pgfpathlineto{\pgfqpoint{0.000000in}{0.055556in}}%
\pgfusepath{stroke,fill}%
}%
\begin{pgfscope}%
\pgfsys@transformshift{1.446773in}{0.528906in}%
\pgfsys@useobject{currentmarker}{}%
\end{pgfscope}%
\end{pgfscope}%
\begin{pgfscope}%
\pgfsetbuttcap%
\pgfsetroundjoin%
\definecolor{currentfill}{rgb}{0.000000,0.000000,0.000000}%
\pgfsetfillcolor{currentfill}%
\pgfsetlinewidth{0.501875pt}%
\definecolor{currentstroke}{rgb}{0.000000,0.000000,0.000000}%
\pgfsetstrokecolor{currentstroke}%
\pgfsetdash{}{0pt}%
\pgfsys@defobject{currentmarker}{\pgfqpoint{0.000000in}{-0.055556in}}{\pgfqpoint{0.000000in}{0.000000in}}{%
\pgfpathmoveto{\pgfqpoint{0.000000in}{0.000000in}}%
\pgfpathlineto{\pgfqpoint{0.000000in}{-0.055556in}}%
\pgfusepath{stroke,fill}%
}%
\begin{pgfscope}%
\pgfsys@transformshift{1.446773in}{1.753333in}%
\pgfsys@useobject{currentmarker}{}%
\end{pgfscope}%
\end{pgfscope}%
\begin{pgfscope}%
\pgftext[x=1.446773in,y=0.473351in,,top]{\rmfamily\fontsize{10.000000}{12.000000}\selectfont 500}%
\end{pgfscope}%
\begin{pgfscope}%
\pgfsetbuttcap%
\pgfsetroundjoin%
\definecolor{currentfill}{rgb}{0.000000,0.000000,0.000000}%
\pgfsetfillcolor{currentfill}%
\pgfsetlinewidth{0.501875pt}%
\definecolor{currentstroke}{rgb}{0.000000,0.000000,0.000000}%
\pgfsetstrokecolor{currentstroke}%
\pgfsetdash{}{0pt}%
\pgfsys@defobject{currentmarker}{\pgfqpoint{0.000000in}{0.000000in}}{\pgfqpoint{0.000000in}{0.055556in}}{%
\pgfpathmoveto{\pgfqpoint{0.000000in}{0.000000in}}%
\pgfpathlineto{\pgfqpoint{0.000000in}{0.055556in}}%
\pgfusepath{stroke,fill}%
}%
\begin{pgfscope}%
\pgfsys@transformshift{2.098425in}{0.528906in}%
\pgfsys@useobject{currentmarker}{}%
\end{pgfscope}%
\end{pgfscope}%
\begin{pgfscope}%
\pgfsetbuttcap%
\pgfsetroundjoin%
\definecolor{currentfill}{rgb}{0.000000,0.000000,0.000000}%
\pgfsetfillcolor{currentfill}%
\pgfsetlinewidth{0.501875pt}%
\definecolor{currentstroke}{rgb}{0.000000,0.000000,0.000000}%
\pgfsetstrokecolor{currentstroke}%
\pgfsetdash{}{0pt}%
\pgfsys@defobject{currentmarker}{\pgfqpoint{0.000000in}{-0.055556in}}{\pgfqpoint{0.000000in}{0.000000in}}{%
\pgfpathmoveto{\pgfqpoint{0.000000in}{0.000000in}}%
\pgfpathlineto{\pgfqpoint{0.000000in}{-0.055556in}}%
\pgfusepath{stroke,fill}%
}%
\begin{pgfscope}%
\pgfsys@transformshift{2.098425in}{1.753333in}%
\pgfsys@useobject{currentmarker}{}%
\end{pgfscope}%
\end{pgfscope}%
\begin{pgfscope}%
\pgftext[x=2.098425in,y=0.473351in,,top]{\rmfamily\fontsize{10.000000}{12.000000}\selectfont 1000}%
\end{pgfscope}%
\begin{pgfscope}%
\pgfsetbuttcap%
\pgfsetroundjoin%
\definecolor{currentfill}{rgb}{0.000000,0.000000,0.000000}%
\pgfsetfillcolor{currentfill}%
\pgfsetlinewidth{0.501875pt}%
\definecolor{currentstroke}{rgb}{0.000000,0.000000,0.000000}%
\pgfsetstrokecolor{currentstroke}%
\pgfsetdash{}{0pt}%
\pgfsys@defobject{currentmarker}{\pgfqpoint{0.000000in}{0.000000in}}{\pgfqpoint{0.000000in}{0.055556in}}{%
\pgfpathmoveto{\pgfqpoint{0.000000in}{0.000000in}}%
\pgfpathlineto{\pgfqpoint{0.000000in}{0.055556in}}%
\pgfusepath{stroke,fill}%
}%
\begin{pgfscope}%
\pgfsys@transformshift{2.750077in}{0.528906in}%
\pgfsys@useobject{currentmarker}{}%
\end{pgfscope}%
\end{pgfscope}%
\begin{pgfscope}%
\pgfsetbuttcap%
\pgfsetroundjoin%
\definecolor{currentfill}{rgb}{0.000000,0.000000,0.000000}%
\pgfsetfillcolor{currentfill}%
\pgfsetlinewidth{0.501875pt}%
\definecolor{currentstroke}{rgb}{0.000000,0.000000,0.000000}%
\pgfsetstrokecolor{currentstroke}%
\pgfsetdash{}{0pt}%
\pgfsys@defobject{currentmarker}{\pgfqpoint{0.000000in}{-0.055556in}}{\pgfqpoint{0.000000in}{0.000000in}}{%
\pgfpathmoveto{\pgfqpoint{0.000000in}{0.000000in}}%
\pgfpathlineto{\pgfqpoint{0.000000in}{-0.055556in}}%
\pgfusepath{stroke,fill}%
}%
\begin{pgfscope}%
\pgfsys@transformshift{2.750077in}{1.753333in}%
\pgfsys@useobject{currentmarker}{}%
\end{pgfscope}%
\end{pgfscope}%
\begin{pgfscope}%
\pgftext[x=2.750077in,y=0.473351in,,top]{\rmfamily\fontsize{10.000000}{12.000000}\selectfont 1500}%
\end{pgfscope}%
\begin{pgfscope}%
\pgftext[x=1.930820in,y=0.280450in,,top]{\rmfamily\fontsize{10.000000}{12.000000}\selectfont Streak Time (ns)}%
\end{pgfscope}%
\begin{pgfscope}%
\pgfsetbuttcap%
\pgfsetroundjoin%
\definecolor{currentfill}{rgb}{0.000000,0.000000,0.000000}%
\pgfsetfillcolor{currentfill}%
\pgfsetlinewidth{0.501875pt}%
\definecolor{currentstroke}{rgb}{0.000000,0.000000,0.000000}%
\pgfsetstrokecolor{currentstroke}%
\pgfsetdash{}{0pt}%
\pgfsys@defobject{currentmarker}{\pgfqpoint{0.000000in}{0.000000in}}{\pgfqpoint{0.055556in}{0.000000in}}{%
\pgfpathmoveto{\pgfqpoint{0.000000in}{0.000000in}}%
\pgfpathlineto{\pgfqpoint{0.055556in}{0.000000in}}%
\pgfusepath{stroke,fill}%
}%
\begin{pgfscope}%
\pgfsys@transformshift{0.721094in}{0.651349in}%
\pgfsys@useobject{currentmarker}{}%
\end{pgfscope}%
\end{pgfscope}%
\begin{pgfscope}%
\pgfsetbuttcap%
\pgfsetroundjoin%
\definecolor{currentfill}{rgb}{0.000000,0.000000,0.000000}%
\pgfsetfillcolor{currentfill}%
\pgfsetlinewidth{0.501875pt}%
\definecolor{currentstroke}{rgb}{0.000000,0.000000,0.000000}%
\pgfsetstrokecolor{currentstroke}%
\pgfsetdash{}{0pt}%
\pgfsys@defobject{currentmarker}{\pgfqpoint{-0.055556in}{0.000000in}}{\pgfqpoint{0.000000in}{0.000000in}}{%
\pgfpathmoveto{\pgfqpoint{0.000000in}{0.000000in}}%
\pgfpathlineto{\pgfqpoint{-0.055556in}{0.000000in}}%
\pgfusepath{stroke,fill}%
}%
\begin{pgfscope}%
\pgfsys@transformshift{3.140547in}{0.651349in}%
\pgfsys@useobject{currentmarker}{}%
\end{pgfscope}%
\end{pgfscope}%
\begin{pgfscope}%
\pgftext[x=0.665538in,y=0.651349in,right,]{\rmfamily\fontsize{10.000000}{12.000000}\selectfont -100}%
\end{pgfscope}%
\begin{pgfscope}%
\pgfsetbuttcap%
\pgfsetroundjoin%
\definecolor{currentfill}{rgb}{0.000000,0.000000,0.000000}%
\pgfsetfillcolor{currentfill}%
\pgfsetlinewidth{0.501875pt}%
\definecolor{currentstroke}{rgb}{0.000000,0.000000,0.000000}%
\pgfsetstrokecolor{currentstroke}%
\pgfsetdash{}{0pt}%
\pgfsys@defobject{currentmarker}{\pgfqpoint{0.000000in}{0.000000in}}{\pgfqpoint{0.055556in}{0.000000in}}{%
\pgfpathmoveto{\pgfqpoint{0.000000in}{0.000000in}}%
\pgfpathlineto{\pgfqpoint{0.055556in}{0.000000in}}%
\pgfusepath{stroke,fill}%
}%
\begin{pgfscope}%
\pgfsys@transformshift{0.721094in}{0.896234in}%
\pgfsys@useobject{currentmarker}{}%
\end{pgfscope}%
\end{pgfscope}%
\begin{pgfscope}%
\pgfsetbuttcap%
\pgfsetroundjoin%
\definecolor{currentfill}{rgb}{0.000000,0.000000,0.000000}%
\pgfsetfillcolor{currentfill}%
\pgfsetlinewidth{0.501875pt}%
\definecolor{currentstroke}{rgb}{0.000000,0.000000,0.000000}%
\pgfsetstrokecolor{currentstroke}%
\pgfsetdash{}{0pt}%
\pgfsys@defobject{currentmarker}{\pgfqpoint{-0.055556in}{0.000000in}}{\pgfqpoint{0.000000in}{0.000000in}}{%
\pgfpathmoveto{\pgfqpoint{0.000000in}{0.000000in}}%
\pgfpathlineto{\pgfqpoint{-0.055556in}{0.000000in}}%
\pgfusepath{stroke,fill}%
}%
\begin{pgfscope}%
\pgfsys@transformshift{3.140547in}{0.896234in}%
\pgfsys@useobject{currentmarker}{}%
\end{pgfscope}%
\end{pgfscope}%
\begin{pgfscope}%
\pgftext[x=0.665538in,y=0.896234in,right,]{\rmfamily\fontsize{10.000000}{12.000000}\selectfont 0}%
\end{pgfscope}%
\begin{pgfscope}%
\pgfsetbuttcap%
\pgfsetroundjoin%
\definecolor{currentfill}{rgb}{0.000000,0.000000,0.000000}%
\pgfsetfillcolor{currentfill}%
\pgfsetlinewidth{0.501875pt}%
\definecolor{currentstroke}{rgb}{0.000000,0.000000,0.000000}%
\pgfsetstrokecolor{currentstroke}%
\pgfsetdash{}{0pt}%
\pgfsys@defobject{currentmarker}{\pgfqpoint{0.000000in}{0.000000in}}{\pgfqpoint{0.055556in}{0.000000in}}{%
\pgfpathmoveto{\pgfqpoint{0.000000in}{0.000000in}}%
\pgfpathlineto{\pgfqpoint{0.055556in}{0.000000in}}%
\pgfusepath{stroke,fill}%
}%
\begin{pgfscope}%
\pgfsys@transformshift{0.721094in}{1.141120in}%
\pgfsys@useobject{currentmarker}{}%
\end{pgfscope}%
\end{pgfscope}%
\begin{pgfscope}%
\pgfsetbuttcap%
\pgfsetroundjoin%
\definecolor{currentfill}{rgb}{0.000000,0.000000,0.000000}%
\pgfsetfillcolor{currentfill}%
\pgfsetlinewidth{0.501875pt}%
\definecolor{currentstroke}{rgb}{0.000000,0.000000,0.000000}%
\pgfsetstrokecolor{currentstroke}%
\pgfsetdash{}{0pt}%
\pgfsys@defobject{currentmarker}{\pgfqpoint{-0.055556in}{0.000000in}}{\pgfqpoint{0.000000in}{0.000000in}}{%
\pgfpathmoveto{\pgfqpoint{0.000000in}{0.000000in}}%
\pgfpathlineto{\pgfqpoint{-0.055556in}{0.000000in}}%
\pgfusepath{stroke,fill}%
}%
\begin{pgfscope}%
\pgfsys@transformshift{3.140547in}{1.141120in}%
\pgfsys@useobject{currentmarker}{}%
\end{pgfscope}%
\end{pgfscope}%
\begin{pgfscope}%
\pgftext[x=0.665538in,y=1.141120in,right,]{\rmfamily\fontsize{10.000000}{12.000000}\selectfont 100}%
\end{pgfscope}%
\begin{pgfscope}%
\pgfsetbuttcap%
\pgfsetroundjoin%
\definecolor{currentfill}{rgb}{0.000000,0.000000,0.000000}%
\pgfsetfillcolor{currentfill}%
\pgfsetlinewidth{0.501875pt}%
\definecolor{currentstroke}{rgb}{0.000000,0.000000,0.000000}%
\pgfsetstrokecolor{currentstroke}%
\pgfsetdash{}{0pt}%
\pgfsys@defobject{currentmarker}{\pgfqpoint{0.000000in}{0.000000in}}{\pgfqpoint{0.055556in}{0.000000in}}{%
\pgfpathmoveto{\pgfqpoint{0.000000in}{0.000000in}}%
\pgfpathlineto{\pgfqpoint{0.055556in}{0.000000in}}%
\pgfusepath{stroke,fill}%
}%
\begin{pgfscope}%
\pgfsys@transformshift{0.721094in}{1.386005in}%
\pgfsys@useobject{currentmarker}{}%
\end{pgfscope}%
\end{pgfscope}%
\begin{pgfscope}%
\pgfsetbuttcap%
\pgfsetroundjoin%
\definecolor{currentfill}{rgb}{0.000000,0.000000,0.000000}%
\pgfsetfillcolor{currentfill}%
\pgfsetlinewidth{0.501875pt}%
\definecolor{currentstroke}{rgb}{0.000000,0.000000,0.000000}%
\pgfsetstrokecolor{currentstroke}%
\pgfsetdash{}{0pt}%
\pgfsys@defobject{currentmarker}{\pgfqpoint{-0.055556in}{0.000000in}}{\pgfqpoint{0.000000in}{0.000000in}}{%
\pgfpathmoveto{\pgfqpoint{0.000000in}{0.000000in}}%
\pgfpathlineto{\pgfqpoint{-0.055556in}{0.000000in}}%
\pgfusepath{stroke,fill}%
}%
\begin{pgfscope}%
\pgfsys@transformshift{3.140547in}{1.386005in}%
\pgfsys@useobject{currentmarker}{}%
\end{pgfscope}%
\end{pgfscope}%
\begin{pgfscope}%
\pgftext[x=0.665538in,y=1.386005in,right,]{\rmfamily\fontsize{10.000000}{12.000000}\selectfont 200}%
\end{pgfscope}%
\begin{pgfscope}%
\pgfsetbuttcap%
\pgfsetroundjoin%
\definecolor{currentfill}{rgb}{0.000000,0.000000,0.000000}%
\pgfsetfillcolor{currentfill}%
\pgfsetlinewidth{0.501875pt}%
\definecolor{currentstroke}{rgb}{0.000000,0.000000,0.000000}%
\pgfsetstrokecolor{currentstroke}%
\pgfsetdash{}{0pt}%
\pgfsys@defobject{currentmarker}{\pgfqpoint{0.000000in}{0.000000in}}{\pgfqpoint{0.055556in}{0.000000in}}{%
\pgfpathmoveto{\pgfqpoint{0.000000in}{0.000000in}}%
\pgfpathlineto{\pgfqpoint{0.055556in}{0.000000in}}%
\pgfusepath{stroke,fill}%
}%
\begin{pgfscope}%
\pgfsys@transformshift{0.721094in}{1.630891in}%
\pgfsys@useobject{currentmarker}{}%
\end{pgfscope}%
\end{pgfscope}%
\begin{pgfscope}%
\pgfsetbuttcap%
\pgfsetroundjoin%
\definecolor{currentfill}{rgb}{0.000000,0.000000,0.000000}%
\pgfsetfillcolor{currentfill}%
\pgfsetlinewidth{0.501875pt}%
\definecolor{currentstroke}{rgb}{0.000000,0.000000,0.000000}%
\pgfsetstrokecolor{currentstroke}%
\pgfsetdash{}{0pt}%
\pgfsys@defobject{currentmarker}{\pgfqpoint{-0.055556in}{0.000000in}}{\pgfqpoint{0.000000in}{0.000000in}}{%
\pgfpathmoveto{\pgfqpoint{0.000000in}{0.000000in}}%
\pgfpathlineto{\pgfqpoint{-0.055556in}{0.000000in}}%
\pgfusepath{stroke,fill}%
}%
\begin{pgfscope}%
\pgfsys@transformshift{3.140547in}{1.630891in}%
\pgfsys@useobject{currentmarker}{}%
\end{pgfscope}%
\end{pgfscope}%
\begin{pgfscope}%
\pgftext[x=0.665538in,y=1.630891in,right,]{\rmfamily\fontsize{10.000000}{12.000000}\selectfont 300}%
\end{pgfscope}%
\begin{pgfscope}%
\pgftext[x=0.341464in,y=1.141120in,,bottom,rotate=90.000000]{\rmfamily\fontsize{10.000000}{12.000000}\selectfont Deflector Voltage (V)}%
\end{pgfscope}%
\end{pgfpicture}%
\makeatother%
\endgroup%

    \caption[Streak deflector voltage oscillations.]{An example of ringing in the fast deflector voltage sweep. On the left is the Voltage on the detector from the start of the electron bunch, notice the slow damping oscillation after the initial \unit[10]{ns} sweep. On the right is an example of a streaked emittance measurement for a relatively long bunch length, notice the oscillations corresponding to the `ringing' present in the deflector voltage located on the right side of the streak.}
    \label{figure:ringing}
    % Fast streak from 2017.06.11 Set 12, code in 2017.07.25, plot2.py
\end{figure}

\subsubsection{Calibration}
Calibration of the streak timing to the streak image was achieved with a calibration image, the voltage profile of the deflector during the streaking and assuming the streak position-time relation was linear.
The calibration image was taken by averaging a number of images with static deflector voltages at the maximum and minimum deflector positions in order to determine the pixel-voltage gradient, $\frac{dpixel}{dV}$.
The averaging of images at the two deflector positions was not required but is convenient for analysis.
Any transformations applied to the streaked pepperpot images, such as rotation and deskewing, needed to be applied to the calibration image before measuring $dpixel$.
An example calibration image is shown in Figure~\ref{figure:example_calibration}.
The deflector voltage profile was measured by a \unit[1]{GHz} oscilloscope\footnote{Tektronix DPO4104B-L Digital Phosphor Oscilloscope} with a probe attached to the vacuum feedthrough in order to minimise lag when comparing the deflector voltage to other signals, such as timing triggers.
The voltage gradient, $\frac{dV}{dt}$, was measured by a linear fit to the streak region of the deflector voltage profile.
It was then simple to calculate the  deflection gradient for the image
\begin{equation}
\frac{dpixel}{dt} = \frac{dpixel}{dV} \frac{dV}{dt}.
\end{equation}

\begin{figure}
    \center
    \includegraphics[width=0.5\linewidth]{part2/Figs/example_calibration.jpeg}
    \caption[Streaked pepperpot calibration image.]{An example of an image used to calibration the timing of the streaked pepperpots.
    This image is composed of the average of several images where the deflectors were at the minimum and maximum voltage.}
    % Set 9 2017.06.11
    \label{figure:example_calibration}
\end{figure}

The oscilloscope was also connected to a photodiode that monitors the intensity of the pulsed ionisation laser and this was used to determine the time during the voltage sweep that corresponds to the start of the streak as the streak begins when the blue laser is activated.
The beam current of the short duration streaks correlates strongly with the intensity of the blue laser as shown in Figure~\ref{figure:blue_streak} which implies that the atom cloud is not saturated by the blue laser, thus suggesting that electron flux could be increased with a higher intensity laser.

\begin{figure}
    \center
    %% Creator: Matplotlib, PGF backend
%%
%% To include the figure in your LaTeX document, write
%%   \input{<filename>.pgf}
%%
%% Make sure the required packages are loaded in your preamble
%%   \usepackage{pgf}
%%
%% Figures using additional raster images can only be included by \input if
%% they are in the same directory as the main LaTeX file. For loading figures
%% from other directories you can use the `import` package
%%   \usepackage{import}
%% and then include the figures with
%%   \import{<path to file>}{<filename>.pgf}
%%
%% Matplotlib used the following preamble
%%
\begingroup%
\makeatletter%
\begin{pgfpicture}%
\pgfpathrectangle{\pgfpointorigin}{\pgfqpoint{4.910600in}{5.710000in}}%
\pgfusepath{use as bounding box, clip}%
\begin{pgfscope}%
\pgfsetbuttcap%
\pgfsetmiterjoin%
\definecolor{currentfill}{rgb}{1.000000,1.000000,1.000000}%
\pgfsetfillcolor{currentfill}%
\pgfsetlinewidth{0.000000pt}%
\definecolor{currentstroke}{rgb}{1.000000,1.000000,1.000000}%
\pgfsetstrokecolor{currentstroke}%
\pgfsetdash{}{0pt}%
\pgfpathmoveto{\pgfqpoint{0.000000in}{0.000000in}}%
\pgfpathlineto{\pgfqpoint{4.910600in}{0.000000in}}%
\pgfpathlineto{\pgfqpoint{4.910600in}{5.710000in}}%
\pgfpathlineto{\pgfqpoint{0.000000in}{5.710000in}}%
\pgfpathclose%
\pgfusepath{fill}%
\end{pgfscope}%
\begin{pgfscope}%
\pgfsetbuttcap%
\pgfsetmiterjoin%
\definecolor{currentfill}{rgb}{1.000000,1.000000,1.000000}%
\pgfsetfillcolor{currentfill}%
\pgfsetlinewidth{0.000000pt}%
\definecolor{currentstroke}{rgb}{0.000000,0.000000,0.000000}%
\pgfsetstrokecolor{currentstroke}%
\pgfsetstrokeopacity{0.000000}%
\pgfsetdash{}{0pt}%
\pgfpathmoveto{\pgfqpoint{0.397389in}{3.891140in}}%
\pgfpathlineto{\pgfqpoint{4.730600in}{3.891140in}}%
\pgfpathlineto{\pgfqpoint{4.730600in}{5.525824in}}%
\pgfpathlineto{\pgfqpoint{0.397389in}{5.525824in}}%
\pgfpathclose%
\pgfusepath{fill}%
\end{pgfscope}%
\begin{pgfscope}%
\pgfpathrectangle{\pgfqpoint{0.397389in}{3.891140in}}{\pgfqpoint{4.333211in}{1.634684in}} %
\pgfusepath{clip}%
\pgftext[at=\pgfqpoint{0.397389in}{3.891140in},left,bottom]{\pgfimage[interpolate=true,width=4.340000in,height=1.650000in]{blue_electrons-img0.png}}%
\end{pgfscope}%
\begin{pgfscope}%
\pgfsetrectcap%
\pgfsetmiterjoin%
\pgfsetlinewidth{1.003750pt}%
\definecolor{currentstroke}{rgb}{0.000000,0.000000,0.000000}%
\pgfsetstrokecolor{currentstroke}%
\pgfsetdash{}{0pt}%
\pgfpathmoveto{\pgfqpoint{4.730600in}{3.891140in}}%
\pgfpathlineto{\pgfqpoint{4.730600in}{5.525824in}}%
\pgfusepath{stroke}%
\end{pgfscope}%
\begin{pgfscope}%
\pgfsetrectcap%
\pgfsetmiterjoin%
\pgfsetlinewidth{1.003750pt}%
\definecolor{currentstroke}{rgb}{0.000000,0.000000,0.000000}%
\pgfsetstrokecolor{currentstroke}%
\pgfsetdash{}{0pt}%
\pgfpathmoveto{\pgfqpoint{0.397389in}{3.891140in}}%
\pgfpathlineto{\pgfqpoint{4.730600in}{3.891140in}}%
\pgfusepath{stroke}%
\end{pgfscope}%
\begin{pgfscope}%
\pgfsetrectcap%
\pgfsetmiterjoin%
\pgfsetlinewidth{1.003750pt}%
\definecolor{currentstroke}{rgb}{0.000000,0.000000,0.000000}%
\pgfsetstrokecolor{currentstroke}%
\pgfsetdash{}{0pt}%
\pgfpathmoveto{\pgfqpoint{0.397389in}{5.525824in}}%
\pgfpathlineto{\pgfqpoint{4.730600in}{5.525824in}}%
\pgfusepath{stroke}%
\end{pgfscope}%
\begin{pgfscope}%
\pgfsetrectcap%
\pgfsetmiterjoin%
\pgfsetlinewidth{1.003750pt}%
\definecolor{currentstroke}{rgb}{0.000000,0.000000,0.000000}%
\pgfsetstrokecolor{currentstroke}%
\pgfsetdash{}{0pt}%
\pgfpathmoveto{\pgfqpoint{0.397389in}{3.891140in}}%
\pgfpathlineto{\pgfqpoint{0.397389in}{5.525824in}}%
\pgfusepath{stroke}%
\end{pgfscope}%
\begin{pgfscope}%
\pgftext[x=0.327945in,y=4.708482in,,bottom,rotate=90.000000]{\fontsize{12.000000}{14.400000}\selectfont Electron Streak}%
\end{pgfscope}%
\begin{pgfscope}%
\pgfsetbuttcap%
\pgfsetmiterjoin%
\definecolor{currentfill}{rgb}{1.000000,1.000000,1.000000}%
\pgfsetfillcolor{currentfill}%
\pgfsetlinewidth{0.000000pt}%
\definecolor{currentstroke}{rgb}{0.000000,0.000000,0.000000}%
\pgfsetstrokecolor{currentstroke}%
\pgfsetstrokeopacity{0.000000}%
\pgfsetdash{}{0pt}%
\pgfpathmoveto{\pgfqpoint{0.397389in}{2.243927in}}%
\pgfpathlineto{\pgfqpoint{4.730600in}{2.243927in}}%
\pgfpathlineto{\pgfqpoint{4.730600in}{3.886963in}}%
\pgfpathlineto{\pgfqpoint{0.397389in}{3.886963in}}%
\pgfpathclose%
\pgfusepath{fill}%
\end{pgfscope}%
\begin{pgfscope}%
\pgfpathrectangle{\pgfqpoint{0.397389in}{2.243927in}}{\pgfqpoint{4.333211in}{1.643037in}} %
\pgfusepath{clip}%
\pgfsetrectcap%
\pgfsetroundjoin%
\pgfsetlinewidth{1.003750pt}%
\definecolor{currentstroke}{rgb}{0.309804,0.478431,0.682353}%
\pgfsetstrokecolor{currentstroke}%
\pgfsetdash{}{0pt}%
\pgfpathmoveto{\pgfqpoint{0.397389in}{2.420810in}}%
\pgfpathlineto{\pgfqpoint{0.436427in}{2.423302in}}%
\pgfpathlineto{\pgfqpoint{0.449440in}{2.423375in}}%
\pgfpathlineto{\pgfqpoint{0.462453in}{2.421564in}}%
\pgfpathlineto{\pgfqpoint{0.488478in}{2.422204in}}%
\pgfpathlineto{\pgfqpoint{0.501491in}{2.421096in}}%
\pgfpathlineto{\pgfqpoint{0.514503in}{2.421534in}}%
\pgfpathlineto{\pgfqpoint{0.540529in}{2.424061in}}%
\pgfpathlineto{\pgfqpoint{0.579566in}{2.428608in}}%
\pgfpathlineto{\pgfqpoint{0.592579in}{2.430516in}}%
\pgfpathlineto{\pgfqpoint{0.618604in}{2.432283in}}%
\pgfpathlineto{\pgfqpoint{0.670655in}{2.439753in}}%
\pgfpathlineto{\pgfqpoint{0.683668in}{2.444052in}}%
\pgfpathlineto{\pgfqpoint{0.748731in}{2.454413in}}%
\pgfpathlineto{\pgfqpoint{0.761743in}{2.457015in}}%
\pgfpathlineto{\pgfqpoint{0.774756in}{2.461572in}}%
\pgfpathlineto{\pgfqpoint{0.839819in}{2.493958in}}%
\pgfpathlineto{\pgfqpoint{0.852832in}{2.501366in}}%
\pgfpathlineto{\pgfqpoint{0.891870in}{2.534086in}}%
\pgfpathlineto{\pgfqpoint{0.904883in}{2.548993in}}%
\pgfpathlineto{\pgfqpoint{0.930908in}{2.588643in}}%
\pgfpathlineto{\pgfqpoint{0.956933in}{2.640202in}}%
\pgfpathlineto{\pgfqpoint{0.969946in}{2.670408in}}%
\pgfpathlineto{\pgfqpoint{0.995971in}{2.746365in}}%
\pgfpathlineto{\pgfqpoint{1.008984in}{2.787652in}}%
\pgfpathlineto{\pgfqpoint{1.035009in}{2.885488in}}%
\pgfpathlineto{\pgfqpoint{1.061034in}{2.995291in}}%
\pgfpathlineto{\pgfqpoint{1.100072in}{3.175136in}}%
\pgfpathlineto{\pgfqpoint{1.126098in}{3.291371in}}%
\pgfpathlineto{\pgfqpoint{1.152123in}{3.392549in}}%
\pgfpathlineto{\pgfqpoint{1.165135in}{3.435634in}}%
\pgfpathlineto{\pgfqpoint{1.178148in}{3.470467in}}%
\pgfpathlineto{\pgfqpoint{1.191161in}{3.498552in}}%
\pgfpathlineto{\pgfqpoint{1.204173in}{3.521071in}}%
\pgfpathlineto{\pgfqpoint{1.217186in}{3.539265in}}%
\pgfpathlineto{\pgfqpoint{1.230199in}{3.549672in}}%
\pgfpathlineto{\pgfqpoint{1.243211in}{3.553789in}}%
\pgfpathlineto{\pgfqpoint{1.256224in}{3.551466in}}%
\pgfpathlineto{\pgfqpoint{1.269237in}{3.543101in}}%
\pgfpathlineto{\pgfqpoint{1.295262in}{3.512219in}}%
\pgfpathlineto{\pgfqpoint{1.308275in}{3.493630in}}%
\pgfpathlineto{\pgfqpoint{1.321287in}{3.470576in}}%
\pgfpathlineto{\pgfqpoint{1.360325in}{3.409562in}}%
\pgfpathlineto{\pgfqpoint{1.399363in}{3.364105in}}%
\pgfpathlineto{\pgfqpoint{1.412376in}{3.356905in}}%
\pgfpathlineto{\pgfqpoint{1.425388in}{3.354458in}}%
\pgfpathlineto{\pgfqpoint{1.438401in}{3.356052in}}%
\pgfpathlineto{\pgfqpoint{1.451414in}{3.362740in}}%
\pgfpathlineto{\pgfqpoint{1.464426in}{3.374414in}}%
\pgfpathlineto{\pgfqpoint{1.477439in}{3.391860in}}%
\pgfpathlineto{\pgfqpoint{1.490452in}{3.414610in}}%
\pgfpathlineto{\pgfqpoint{1.503464in}{3.440875in}}%
\pgfpathlineto{\pgfqpoint{1.529490in}{3.501582in}}%
\pgfpathlineto{\pgfqpoint{1.555515in}{3.572957in}}%
\pgfpathlineto{\pgfqpoint{1.581540in}{3.636788in}}%
\pgfpathlineto{\pgfqpoint{1.594553in}{3.662483in}}%
\pgfpathlineto{\pgfqpoint{1.607565in}{3.682530in}}%
\pgfpathlineto{\pgfqpoint{1.620578in}{3.698908in}}%
\pgfpathlineto{\pgfqpoint{1.633591in}{3.707589in}}%
\pgfpathlineto{\pgfqpoint{1.646603in}{3.711483in}}%
\pgfpathlineto{\pgfqpoint{1.672629in}{3.709470in}}%
\pgfpathlineto{\pgfqpoint{1.698654in}{3.698935in}}%
\pgfpathlineto{\pgfqpoint{1.711667in}{3.690381in}}%
\pgfpathlineto{\pgfqpoint{1.776730in}{3.633204in}}%
\pgfpathlineto{\pgfqpoint{1.802755in}{3.607371in}}%
\pgfpathlineto{\pgfqpoint{1.815768in}{3.591676in}}%
\pgfpathlineto{\pgfqpoint{1.828780in}{3.578961in}}%
\pgfpathlineto{\pgfqpoint{1.867818in}{3.532185in}}%
\pgfpathlineto{\pgfqpoint{1.919869in}{3.463458in}}%
\pgfpathlineto{\pgfqpoint{1.932881in}{3.440985in}}%
\pgfpathlineto{\pgfqpoint{1.971919in}{3.387972in}}%
\pgfpathlineto{\pgfqpoint{1.984932in}{3.374011in}}%
\pgfpathlineto{\pgfqpoint{1.997945in}{3.355045in}}%
\pgfpathlineto{\pgfqpoint{2.023970in}{3.323067in}}%
\pgfpathlineto{\pgfqpoint{2.049995in}{3.296708in}}%
\pgfpathlineto{\pgfqpoint{2.076021in}{3.273554in}}%
\pgfpathlineto{\pgfqpoint{2.115059in}{3.242323in}}%
\pgfpathlineto{\pgfqpoint{2.128071in}{3.235475in}}%
\pgfpathlineto{\pgfqpoint{2.167109in}{3.208643in}}%
\pgfpathlineto{\pgfqpoint{2.193134in}{3.196382in}}%
\pgfpathlineto{\pgfqpoint{2.219160in}{3.186999in}}%
\pgfpathlineto{\pgfqpoint{2.245185in}{3.176019in}}%
\pgfpathlineto{\pgfqpoint{2.284223in}{3.167198in}}%
\pgfpathlineto{\pgfqpoint{2.323261in}{3.157570in}}%
\pgfpathlineto{\pgfqpoint{2.336273in}{3.156346in}}%
\pgfpathlineto{\pgfqpoint{2.362299in}{3.148278in}}%
\pgfpathlineto{\pgfqpoint{2.375311in}{3.148788in}}%
\pgfpathlineto{\pgfqpoint{2.388324in}{3.147905in}}%
\pgfpathlineto{\pgfqpoint{2.401337in}{3.145445in}}%
\pgfpathlineto{\pgfqpoint{2.414349in}{3.148025in}}%
\pgfpathlineto{\pgfqpoint{2.427362in}{3.148571in}}%
\pgfpathlineto{\pgfqpoint{2.453387in}{3.146322in}}%
\pgfpathlineto{\pgfqpoint{2.466400in}{3.140771in}}%
\pgfpathlineto{\pgfqpoint{2.492425in}{3.133938in}}%
\pgfpathlineto{\pgfqpoint{2.531463in}{3.115181in}}%
\pgfpathlineto{\pgfqpoint{2.557488in}{3.100491in}}%
\pgfpathlineto{\pgfqpoint{2.570501in}{3.090801in}}%
\pgfpathlineto{\pgfqpoint{2.596526in}{3.075805in}}%
\pgfpathlineto{\pgfqpoint{2.609539in}{3.070012in}}%
\pgfpathlineto{\pgfqpoint{2.635564in}{3.049332in}}%
\pgfpathlineto{\pgfqpoint{2.648577in}{3.039038in}}%
\pgfpathlineto{\pgfqpoint{2.661590in}{3.031395in}}%
\pgfpathlineto{\pgfqpoint{2.674602in}{3.022418in}}%
\pgfpathlineto{\pgfqpoint{2.687615in}{3.015389in}}%
\pgfpathlineto{\pgfqpoint{2.700627in}{3.010377in}}%
\pgfpathlineto{\pgfqpoint{2.713640in}{3.004003in}}%
\pgfpathlineto{\pgfqpoint{2.726653in}{2.995968in}}%
\pgfpathlineto{\pgfqpoint{2.778703in}{2.948544in}}%
\pgfpathlineto{\pgfqpoint{2.804729in}{2.918287in}}%
\pgfpathlineto{\pgfqpoint{2.843767in}{2.879036in}}%
\pgfpathlineto{\pgfqpoint{2.856779in}{2.868829in}}%
\pgfpathlineto{\pgfqpoint{2.869792in}{2.860171in}}%
\pgfpathlineto{\pgfqpoint{2.882805in}{2.849004in}}%
\pgfpathlineto{\pgfqpoint{2.895817in}{2.839721in}}%
\pgfpathlineto{\pgfqpoint{2.908830in}{2.828433in}}%
\pgfpathlineto{\pgfqpoint{2.921842in}{2.819246in}}%
\pgfpathlineto{\pgfqpoint{2.947868in}{2.795917in}}%
\pgfpathlineto{\pgfqpoint{2.973893in}{2.774924in}}%
\pgfpathlineto{\pgfqpoint{3.012931in}{2.737396in}}%
\pgfpathlineto{\pgfqpoint{3.064982in}{2.696014in}}%
\pgfpathlineto{\pgfqpoint{3.077994in}{2.688553in}}%
\pgfpathlineto{\pgfqpoint{3.104019in}{2.675943in}}%
\pgfpathlineto{\pgfqpoint{3.130045in}{2.659492in}}%
\pgfpathlineto{\pgfqpoint{3.143057in}{2.647639in}}%
\pgfpathlineto{\pgfqpoint{3.182095in}{2.615775in}}%
\pgfpathlineto{\pgfqpoint{3.208121in}{2.590824in}}%
\pgfpathlineto{\pgfqpoint{3.247159in}{2.553485in}}%
\pgfpathlineto{\pgfqpoint{3.299209in}{2.510918in}}%
\pgfpathlineto{\pgfqpoint{3.325234in}{2.493310in}}%
\pgfpathlineto{\pgfqpoint{3.338247in}{2.485405in}}%
\pgfpathlineto{\pgfqpoint{3.351260in}{2.476032in}}%
\pgfpathlineto{\pgfqpoint{3.429336in}{2.438493in}}%
\pgfpathlineto{\pgfqpoint{3.481386in}{2.419615in}}%
\pgfpathlineto{\pgfqpoint{3.494399in}{2.415889in}}%
\pgfpathlineto{\pgfqpoint{3.520424in}{2.410353in}}%
\pgfpathlineto{\pgfqpoint{3.533437in}{2.406669in}}%
\pgfpathlineto{\pgfqpoint{3.546449in}{2.404221in}}%
\pgfpathlineto{\pgfqpoint{3.559462in}{2.400019in}}%
\pgfpathlineto{\pgfqpoint{3.572475in}{2.397716in}}%
\pgfpathlineto{\pgfqpoint{3.598500in}{2.390980in}}%
\pgfpathlineto{\pgfqpoint{3.624525in}{2.386090in}}%
\pgfpathlineto{\pgfqpoint{3.637538in}{2.385430in}}%
\pgfpathlineto{\pgfqpoint{3.650551in}{2.382940in}}%
\pgfpathlineto{\pgfqpoint{3.676576in}{2.381647in}}%
\pgfpathlineto{\pgfqpoint{3.689588in}{2.379917in}}%
\pgfpathlineto{\pgfqpoint{3.715614in}{2.379282in}}%
\pgfpathlineto{\pgfqpoint{3.728626in}{2.380375in}}%
\pgfpathlineto{\pgfqpoint{3.741639in}{2.379926in}}%
\pgfpathlineto{\pgfqpoint{3.754652in}{2.377614in}}%
\pgfpathlineto{\pgfqpoint{3.806702in}{2.375697in}}%
\pgfpathlineto{\pgfqpoint{3.819715in}{2.375148in}}%
\pgfpathlineto{\pgfqpoint{3.832728in}{2.376141in}}%
\pgfpathlineto{\pgfqpoint{3.871765in}{2.372461in}}%
\pgfpathlineto{\pgfqpoint{3.884778in}{2.372889in}}%
\pgfpathlineto{\pgfqpoint{3.897791in}{2.371691in}}%
\pgfpathlineto{\pgfqpoint{3.936829in}{2.371478in}}%
\pgfpathlineto{\pgfqpoint{3.949841in}{2.370404in}}%
\pgfpathlineto{\pgfqpoint{3.962854in}{2.368088in}}%
\pgfpathlineto{\pgfqpoint{4.066955in}{2.363567in}}%
\pgfpathlineto{\pgfqpoint{4.223107in}{2.363203in}}%
\pgfpathlineto{\pgfqpoint{4.236120in}{2.362104in}}%
\pgfpathlineto{\pgfqpoint{4.275157in}{2.362512in}}%
\pgfpathlineto{\pgfqpoint{4.314195in}{2.361534in}}%
\pgfpathlineto{\pgfqpoint{4.340221in}{2.362382in}}%
\pgfpathlineto{\pgfqpoint{4.379259in}{2.364750in}}%
\pgfpathlineto{\pgfqpoint{4.444322in}{2.362690in}}%
\pgfpathlineto{\pgfqpoint{4.457334in}{2.364071in}}%
\pgfpathlineto{\pgfqpoint{4.470347in}{2.362744in}}%
\pgfpathlineto{\pgfqpoint{4.483360in}{2.363365in}}%
\pgfpathlineto{\pgfqpoint{4.496372in}{2.362176in}}%
\pgfpathlineto{\pgfqpoint{4.509385in}{2.362914in}}%
\pgfpathlineto{\pgfqpoint{4.535410in}{2.361130in}}%
\pgfpathlineto{\pgfqpoint{4.548423in}{2.362561in}}%
\pgfpathlineto{\pgfqpoint{4.561436in}{2.361999in}}%
\pgfpathlineto{\pgfqpoint{4.574448in}{2.360287in}}%
\pgfpathlineto{\pgfqpoint{4.600474in}{2.358817in}}%
\pgfpathlineto{\pgfqpoint{4.626499in}{2.355703in}}%
\pgfpathlineto{\pgfqpoint{4.691562in}{2.353942in}}%
\pgfpathlineto{\pgfqpoint{4.730600in}{2.351637in}}%
\pgfpathlineto{\pgfqpoint{4.730600in}{2.351637in}}%
\pgfusepath{stroke}%
\end{pgfscope}%
\begin{pgfscope}%
\pgfsetrectcap%
\pgfsetmiterjoin%
\pgfsetlinewidth{1.003750pt}%
\definecolor{currentstroke}{rgb}{0.000000,0.000000,0.000000}%
\pgfsetstrokecolor{currentstroke}%
\pgfsetdash{}{0pt}%
\pgfpathmoveto{\pgfqpoint{4.730600in}{2.243927in}}%
\pgfpathlineto{\pgfqpoint{4.730600in}{3.886963in}}%
\pgfusepath{stroke}%
\end{pgfscope}%
\begin{pgfscope}%
\pgfsetrectcap%
\pgfsetmiterjoin%
\pgfsetlinewidth{1.003750pt}%
\definecolor{currentstroke}{rgb}{0.000000,0.000000,0.000000}%
\pgfsetstrokecolor{currentstroke}%
\pgfsetdash{}{0pt}%
\pgfpathmoveto{\pgfqpoint{0.397389in}{2.243927in}}%
\pgfpathlineto{\pgfqpoint{4.730600in}{2.243927in}}%
\pgfusepath{stroke}%
\end{pgfscope}%
\begin{pgfscope}%
\pgfsetrectcap%
\pgfsetmiterjoin%
\pgfsetlinewidth{1.003750pt}%
\definecolor{currentstroke}{rgb}{0.000000,0.000000,0.000000}%
\pgfsetstrokecolor{currentstroke}%
\pgfsetdash{}{0pt}%
\pgfpathmoveto{\pgfqpoint{0.397389in}{3.886963in}}%
\pgfpathlineto{\pgfqpoint{4.730600in}{3.886963in}}%
\pgfusepath{stroke}%
\end{pgfscope}%
\begin{pgfscope}%
\pgfsetrectcap%
\pgfsetmiterjoin%
\pgfsetlinewidth{1.003750pt}%
\definecolor{currentstroke}{rgb}{0.000000,0.000000,0.000000}%
\pgfsetstrokecolor{currentstroke}%
\pgfsetdash{}{0pt}%
\pgfpathmoveto{\pgfqpoint{0.397389in}{2.243927in}}%
\pgfpathlineto{\pgfqpoint{0.397389in}{3.886963in}}%
\pgfusepath{stroke}%
\end{pgfscope}%
\begin{pgfscope}%
\pgfsetbuttcap%
\pgfsetroundjoin%
\definecolor{currentfill}{rgb}{0.000000,0.000000,0.000000}%
\pgfsetfillcolor{currentfill}%
\pgfsetlinewidth{0.501875pt}%
\definecolor{currentstroke}{rgb}{0.000000,0.000000,0.000000}%
\pgfsetstrokecolor{currentstroke}%
\pgfsetdash{}{0pt}%
\pgfsys@defobject{currentmarker}{\pgfqpoint{0.000000in}{0.000000in}}{\pgfqpoint{0.000000in}{0.055556in}}{%
\pgfpathmoveto{\pgfqpoint{0.000000in}{0.000000in}}%
\pgfpathlineto{\pgfqpoint{0.000000in}{0.055556in}}%
\pgfusepath{stroke,fill}%
}%
\begin{pgfscope}%
\pgfsys@transformshift{0.476092in}{2.243927in}%
\pgfsys@useobject{currentmarker}{}%
\end{pgfscope}%
\end{pgfscope}%
\begin{pgfscope}%
\pgfsetbuttcap%
\pgfsetroundjoin%
\definecolor{currentfill}{rgb}{0.000000,0.000000,0.000000}%
\pgfsetfillcolor{currentfill}%
\pgfsetlinewidth{0.501875pt}%
\definecolor{currentstroke}{rgb}{0.000000,0.000000,0.000000}%
\pgfsetstrokecolor{currentstroke}%
\pgfsetdash{}{0pt}%
\pgfsys@defobject{currentmarker}{\pgfqpoint{0.000000in}{-0.055556in}}{\pgfqpoint{0.000000in}{0.000000in}}{%
\pgfpathmoveto{\pgfqpoint{0.000000in}{0.000000in}}%
\pgfpathlineto{\pgfqpoint{0.000000in}{-0.055556in}}%
\pgfusepath{stroke,fill}%
}%
\begin{pgfscope}%
\pgfsys@transformshift{0.476092in}{3.886963in}%
\pgfsys@useobject{currentmarker}{}%
\end{pgfscope}%
\end{pgfscope}%
\begin{pgfscope}%
\pgfsetbuttcap%
\pgfsetroundjoin%
\definecolor{currentfill}{rgb}{0.000000,0.000000,0.000000}%
\pgfsetfillcolor{currentfill}%
\pgfsetlinewidth{0.501875pt}%
\definecolor{currentstroke}{rgb}{0.000000,0.000000,0.000000}%
\pgfsetstrokecolor{currentstroke}%
\pgfsetdash{}{0pt}%
\pgfsys@defobject{currentmarker}{\pgfqpoint{0.000000in}{0.000000in}}{\pgfqpoint{0.000000in}{0.055556in}}{%
\pgfpathmoveto{\pgfqpoint{0.000000in}{0.000000in}}%
\pgfpathlineto{\pgfqpoint{0.000000in}{0.055556in}}%
\pgfusepath{stroke,fill}%
}%
\begin{pgfscope}%
\pgfsys@transformshift{1.243211in}{2.243927in}%
\pgfsys@useobject{currentmarker}{}%
\end{pgfscope}%
\end{pgfscope}%
\begin{pgfscope}%
\pgfsetbuttcap%
\pgfsetroundjoin%
\definecolor{currentfill}{rgb}{0.000000,0.000000,0.000000}%
\pgfsetfillcolor{currentfill}%
\pgfsetlinewidth{0.501875pt}%
\definecolor{currentstroke}{rgb}{0.000000,0.000000,0.000000}%
\pgfsetstrokecolor{currentstroke}%
\pgfsetdash{}{0pt}%
\pgfsys@defobject{currentmarker}{\pgfqpoint{0.000000in}{-0.055556in}}{\pgfqpoint{0.000000in}{0.000000in}}{%
\pgfpathmoveto{\pgfqpoint{0.000000in}{0.000000in}}%
\pgfpathlineto{\pgfqpoint{0.000000in}{-0.055556in}}%
\pgfusepath{stroke,fill}%
}%
\begin{pgfscope}%
\pgfsys@transformshift{1.243211in}{3.886963in}%
\pgfsys@useobject{currentmarker}{}%
\end{pgfscope}%
\end{pgfscope}%
\begin{pgfscope}%
\pgfsetbuttcap%
\pgfsetroundjoin%
\definecolor{currentfill}{rgb}{0.000000,0.000000,0.000000}%
\pgfsetfillcolor{currentfill}%
\pgfsetlinewidth{0.501875pt}%
\definecolor{currentstroke}{rgb}{0.000000,0.000000,0.000000}%
\pgfsetstrokecolor{currentstroke}%
\pgfsetdash{}{0pt}%
\pgfsys@defobject{currentmarker}{\pgfqpoint{0.000000in}{0.000000in}}{\pgfqpoint{0.000000in}{0.055556in}}{%
\pgfpathmoveto{\pgfqpoint{0.000000in}{0.000000in}}%
\pgfpathlineto{\pgfqpoint{0.000000in}{0.055556in}}%
\pgfusepath{stroke,fill}%
}%
\begin{pgfscope}%
\pgfsys@transformshift{2.010331in}{2.243927in}%
\pgfsys@useobject{currentmarker}{}%
\end{pgfscope}%
\end{pgfscope}%
\begin{pgfscope}%
\pgfsetbuttcap%
\pgfsetroundjoin%
\definecolor{currentfill}{rgb}{0.000000,0.000000,0.000000}%
\pgfsetfillcolor{currentfill}%
\pgfsetlinewidth{0.501875pt}%
\definecolor{currentstroke}{rgb}{0.000000,0.000000,0.000000}%
\pgfsetstrokecolor{currentstroke}%
\pgfsetdash{}{0pt}%
\pgfsys@defobject{currentmarker}{\pgfqpoint{0.000000in}{-0.055556in}}{\pgfqpoint{0.000000in}{0.000000in}}{%
\pgfpathmoveto{\pgfqpoint{0.000000in}{0.000000in}}%
\pgfpathlineto{\pgfqpoint{0.000000in}{-0.055556in}}%
\pgfusepath{stroke,fill}%
}%
\begin{pgfscope}%
\pgfsys@transformshift{2.010331in}{3.886963in}%
\pgfsys@useobject{currentmarker}{}%
\end{pgfscope}%
\end{pgfscope}%
\begin{pgfscope}%
\pgfsetbuttcap%
\pgfsetroundjoin%
\definecolor{currentfill}{rgb}{0.000000,0.000000,0.000000}%
\pgfsetfillcolor{currentfill}%
\pgfsetlinewidth{0.501875pt}%
\definecolor{currentstroke}{rgb}{0.000000,0.000000,0.000000}%
\pgfsetstrokecolor{currentstroke}%
\pgfsetdash{}{0pt}%
\pgfsys@defobject{currentmarker}{\pgfqpoint{0.000000in}{0.000000in}}{\pgfqpoint{0.000000in}{0.055556in}}{%
\pgfpathmoveto{\pgfqpoint{0.000000in}{0.000000in}}%
\pgfpathlineto{\pgfqpoint{0.000000in}{0.055556in}}%
\pgfusepath{stroke,fill}%
}%
\begin{pgfscope}%
\pgfsys@transformshift{2.777450in}{2.243927in}%
\pgfsys@useobject{currentmarker}{}%
\end{pgfscope}%
\end{pgfscope}%
\begin{pgfscope}%
\pgfsetbuttcap%
\pgfsetroundjoin%
\definecolor{currentfill}{rgb}{0.000000,0.000000,0.000000}%
\pgfsetfillcolor{currentfill}%
\pgfsetlinewidth{0.501875pt}%
\definecolor{currentstroke}{rgb}{0.000000,0.000000,0.000000}%
\pgfsetstrokecolor{currentstroke}%
\pgfsetdash{}{0pt}%
\pgfsys@defobject{currentmarker}{\pgfqpoint{0.000000in}{-0.055556in}}{\pgfqpoint{0.000000in}{0.000000in}}{%
\pgfpathmoveto{\pgfqpoint{0.000000in}{0.000000in}}%
\pgfpathlineto{\pgfqpoint{0.000000in}{-0.055556in}}%
\pgfusepath{stroke,fill}%
}%
\begin{pgfscope}%
\pgfsys@transformshift{2.777450in}{3.886963in}%
\pgfsys@useobject{currentmarker}{}%
\end{pgfscope}%
\end{pgfscope}%
\begin{pgfscope}%
\pgfsetbuttcap%
\pgfsetroundjoin%
\definecolor{currentfill}{rgb}{0.000000,0.000000,0.000000}%
\pgfsetfillcolor{currentfill}%
\pgfsetlinewidth{0.501875pt}%
\definecolor{currentstroke}{rgb}{0.000000,0.000000,0.000000}%
\pgfsetstrokecolor{currentstroke}%
\pgfsetdash{}{0pt}%
\pgfsys@defobject{currentmarker}{\pgfqpoint{0.000000in}{0.000000in}}{\pgfqpoint{0.000000in}{0.055556in}}{%
\pgfpathmoveto{\pgfqpoint{0.000000in}{0.000000in}}%
\pgfpathlineto{\pgfqpoint{0.000000in}{0.055556in}}%
\pgfusepath{stroke,fill}%
}%
\begin{pgfscope}%
\pgfsys@transformshift{3.544570in}{2.243927in}%
\pgfsys@useobject{currentmarker}{}%
\end{pgfscope}%
\end{pgfscope}%
\begin{pgfscope}%
\pgfsetbuttcap%
\pgfsetroundjoin%
\definecolor{currentfill}{rgb}{0.000000,0.000000,0.000000}%
\pgfsetfillcolor{currentfill}%
\pgfsetlinewidth{0.501875pt}%
\definecolor{currentstroke}{rgb}{0.000000,0.000000,0.000000}%
\pgfsetstrokecolor{currentstroke}%
\pgfsetdash{}{0pt}%
\pgfsys@defobject{currentmarker}{\pgfqpoint{0.000000in}{-0.055556in}}{\pgfqpoint{0.000000in}{0.000000in}}{%
\pgfpathmoveto{\pgfqpoint{0.000000in}{0.000000in}}%
\pgfpathlineto{\pgfqpoint{0.000000in}{-0.055556in}}%
\pgfusepath{stroke,fill}%
}%
\begin{pgfscope}%
\pgfsys@transformshift{3.544570in}{3.886963in}%
\pgfsys@useobject{currentmarker}{}%
\end{pgfscope}%
\end{pgfscope}%
\begin{pgfscope}%
\pgfsetbuttcap%
\pgfsetroundjoin%
\definecolor{currentfill}{rgb}{0.000000,0.000000,0.000000}%
\pgfsetfillcolor{currentfill}%
\pgfsetlinewidth{0.501875pt}%
\definecolor{currentstroke}{rgb}{0.000000,0.000000,0.000000}%
\pgfsetstrokecolor{currentstroke}%
\pgfsetdash{}{0pt}%
\pgfsys@defobject{currentmarker}{\pgfqpoint{0.000000in}{0.000000in}}{\pgfqpoint{0.000000in}{0.055556in}}{%
\pgfpathmoveto{\pgfqpoint{0.000000in}{0.000000in}}%
\pgfpathlineto{\pgfqpoint{0.000000in}{0.055556in}}%
\pgfusepath{stroke,fill}%
}%
\begin{pgfscope}%
\pgfsys@transformshift{4.311689in}{2.243927in}%
\pgfsys@useobject{currentmarker}{}%
\end{pgfscope}%
\end{pgfscope}%
\begin{pgfscope}%
\pgfsetbuttcap%
\pgfsetroundjoin%
\definecolor{currentfill}{rgb}{0.000000,0.000000,0.000000}%
\pgfsetfillcolor{currentfill}%
\pgfsetlinewidth{0.501875pt}%
\definecolor{currentstroke}{rgb}{0.000000,0.000000,0.000000}%
\pgfsetstrokecolor{currentstroke}%
\pgfsetdash{}{0pt}%
\pgfsys@defobject{currentmarker}{\pgfqpoint{0.000000in}{-0.055556in}}{\pgfqpoint{0.000000in}{0.000000in}}{%
\pgfpathmoveto{\pgfqpoint{0.000000in}{0.000000in}}%
\pgfpathlineto{\pgfqpoint{0.000000in}{-0.055556in}}%
\pgfusepath{stroke,fill}%
}%
\begin{pgfscope}%
\pgfsys@transformshift{4.311689in}{3.886963in}%
\pgfsys@useobject{currentmarker}{}%
\end{pgfscope}%
\end{pgfscope}%
\begin{pgfscope}%
\pgftext[x=0.327945in,y=3.065445in,,bottom,rotate=90.000000]{\fontsize{12.000000}{14.400000}\selectfont Blue Laser}%
\end{pgfscope}%
\begin{pgfscope}%
\pgfsetbuttcap%
\pgfsetmiterjoin%
\definecolor{currentfill}{rgb}{1.000000,1.000000,1.000000}%
\pgfsetfillcolor{currentfill}%
\pgfsetlinewidth{0.000000pt}%
\definecolor{currentstroke}{rgb}{0.000000,0.000000,0.000000}%
\pgfsetstrokecolor{currentstroke}%
\pgfsetstrokeopacity{0.000000}%
\pgfsetdash{}{0pt}%
\pgfpathmoveto{\pgfqpoint{0.397389in}{0.600890in}}%
\pgfpathlineto{\pgfqpoint{4.730600in}{0.600890in}}%
\pgfpathlineto{\pgfqpoint{4.730600in}{2.243927in}}%
\pgfpathlineto{\pgfqpoint{0.397389in}{2.243927in}}%
\pgfpathclose%
\pgfusepath{fill}%
\end{pgfscope}%
\begin{pgfscope}%
\pgfpathrectangle{\pgfqpoint{0.397389in}{0.600890in}}{\pgfqpoint{4.333211in}{1.643037in}} %
\pgfusepath{clip}%
\pgfsetrectcap%
\pgfsetroundjoin%
\pgfsetlinewidth{1.003750pt}%
\definecolor{currentstroke}{rgb}{0.301961,0.607843,0.301961}%
\pgfsetstrokecolor{currentstroke}%
\pgfsetdash{}{0pt}%
\pgfpathmoveto{\pgfqpoint{0.396983in}{0.647443in}}%
\pgfpathlineto{\pgfqpoint{0.408969in}{0.652646in}}%
\pgfpathlineto{\pgfqpoint{0.437736in}{0.670171in}}%
\pgfpathlineto{\pgfqpoint{0.454517in}{0.682220in}}%
\pgfpathlineto{\pgfqpoint{0.495270in}{0.706592in}}%
\pgfpathlineto{\pgfqpoint{0.514448in}{0.720832in}}%
\pgfpathlineto{\pgfqpoint{0.533626in}{0.739179in}}%
\pgfpathlineto{\pgfqpoint{0.581571in}{0.781624in}}%
\pgfpathlineto{\pgfqpoint{0.591160in}{0.790661in}}%
\pgfpathlineto{\pgfqpoint{0.624721in}{0.826534in}}%
\pgfpathlineto{\pgfqpoint{0.658283in}{0.867062in}}%
\pgfpathlineto{\pgfqpoint{0.670269in}{0.882123in}}%
\pgfpathlineto{\pgfqpoint{0.684652in}{0.900470in}}%
\pgfpathlineto{\pgfqpoint{0.703830in}{0.923747in}}%
\pgfpathlineto{\pgfqpoint{0.746981in}{0.984265in}}%
\pgfpathlineto{\pgfqpoint{0.761364in}{1.006994in}}%
\pgfpathlineto{\pgfqpoint{0.782940in}{1.040950in}}%
\pgfpathlineto{\pgfqpoint{0.799720in}{1.071346in}}%
\pgfpathlineto{\pgfqpoint{0.823693in}{1.114339in}}%
\pgfpathlineto{\pgfqpoint{0.847665in}{1.161166in}}%
\pgfpathlineto{\pgfqpoint{0.857254in}{1.181156in}}%
\pgfpathlineto{\pgfqpoint{0.905199in}{1.304384in}}%
\pgfpathlineto{\pgfqpoint{0.919583in}{1.342447in}}%
\pgfpathlineto{\pgfqpoint{0.941158in}{1.401870in}}%
\pgfpathlineto{\pgfqpoint{0.962733in}{1.462663in}}%
\pgfpathlineto{\pgfqpoint{0.972322in}{1.491416in}}%
\pgfpathlineto{\pgfqpoint{1.017870in}{1.645861in}}%
\pgfpathlineto{\pgfqpoint{1.037048in}{1.709940in}}%
\pgfpathlineto{\pgfqpoint{1.080198in}{1.849324in}}%
\pgfpathlineto{\pgfqpoint{1.108965in}{1.934488in}}%
\pgfpathlineto{\pgfqpoint{1.137732in}{2.007603in}}%
\pgfpathlineto{\pgfqpoint{1.152116in}{2.037726in}}%
\pgfpathlineto{\pgfqpoint{1.178486in}{2.089481in}}%
\pgfpathlineto{\pgfqpoint{1.190472in}{2.107007in}}%
\pgfpathlineto{\pgfqpoint{1.202458in}{2.122068in}}%
\pgfpathlineto{\pgfqpoint{1.212047in}{2.131926in}}%
\pgfpathlineto{\pgfqpoint{1.219239in}{2.136582in}}%
\pgfpathlineto{\pgfqpoint{1.228828in}{2.139868in}}%
\pgfpathlineto{\pgfqpoint{1.238417in}{2.142332in}}%
\pgfpathlineto{\pgfqpoint{1.250403in}{2.142880in}}%
\pgfpathlineto{\pgfqpoint{1.259992in}{2.140689in}}%
\pgfpathlineto{\pgfqpoint{1.291156in}{2.125628in}}%
\pgfpathlineto{\pgfqpoint{1.298348in}{2.119604in}}%
\pgfpathlineto{\pgfqpoint{1.317526in}{2.100709in}}%
\pgfpathlineto{\pgfqpoint{1.329512in}{2.087291in}}%
\pgfpathlineto{\pgfqpoint{1.346293in}{2.064836in}}%
\pgfpathlineto{\pgfqpoint{1.439786in}{1.927642in}}%
\pgfpathlineto{\pgfqpoint{1.451772in}{1.914224in}}%
\pgfpathlineto{\pgfqpoint{1.466155in}{1.899437in}}%
\pgfpathlineto{\pgfqpoint{1.480539in}{1.885745in}}%
\pgfpathlineto{\pgfqpoint{1.502114in}{1.869588in}}%
\pgfpathlineto{\pgfqpoint{1.516498in}{1.862468in}}%
\pgfpathlineto{\pgfqpoint{1.533278in}{1.859730in}}%
\pgfpathlineto{\pgfqpoint{1.550059in}{1.860825in}}%
\pgfpathlineto{\pgfqpoint{1.564443in}{1.863290in}}%
\pgfpathlineto{\pgfqpoint{1.600401in}{1.876708in}}%
\pgfpathlineto{\pgfqpoint{1.621977in}{1.892591in}}%
\pgfpathlineto{\pgfqpoint{1.744236in}{1.989530in}}%
\pgfpathlineto{\pgfqpoint{1.756222in}{1.994733in}}%
\pgfpathlineto{\pgfqpoint{1.773003in}{2.001305in}}%
\pgfpathlineto{\pgfqpoint{1.782592in}{2.003769in}}%
\pgfpathlineto{\pgfqpoint{1.813756in}{2.006234in}}%
\pgfpathlineto{\pgfqpoint{1.828140in}{2.001853in}}%
\pgfpathlineto{\pgfqpoint{1.844921in}{1.993090in}}%
\pgfpathlineto{\pgfqpoint{1.859304in}{1.982410in}}%
\pgfpathlineto{\pgfqpoint{1.880879in}{1.961598in}}%
\pgfpathlineto{\pgfqpoint{1.890468in}{1.952014in}}%
\pgfpathlineto{\pgfqpoint{1.919235in}{1.914224in}}%
\pgfpathlineto{\pgfqpoint{1.933619in}{1.892865in}}%
\pgfpathlineto{\pgfqpoint{1.962386in}{1.844943in}}%
\pgfpathlineto{\pgfqpoint{1.979167in}{1.813999in}}%
\pgfpathlineto{\pgfqpoint{2.010331in}{1.754849in}}%
\pgfpathlineto{\pgfqpoint{2.063070in}{1.648874in}}%
\pgfpathlineto{\pgfqpoint{2.087043in}{1.608619in}}%
\pgfpathlineto{\pgfqpoint{2.111015in}{1.569734in}}%
\pgfpathlineto{\pgfqpoint{2.151768in}{1.506477in}}%
\pgfpathlineto{\pgfqpoint{2.187727in}{1.462937in}}%
\pgfpathlineto{\pgfqpoint{2.202111in}{1.448423in}}%
\pgfpathlineto{\pgfqpoint{2.216494in}{1.434731in}}%
\pgfpathlineto{\pgfqpoint{2.245261in}{1.413919in}}%
\pgfpathlineto{\pgfqpoint{2.254850in}{1.407621in}}%
\pgfpathlineto{\pgfqpoint{2.281220in}{1.391738in}}%
\pgfpathlineto{\pgfqpoint{2.290809in}{1.385440in}}%
\pgfpathlineto{\pgfqpoint{2.302795in}{1.379689in}}%
\pgfpathlineto{\pgfqpoint{2.333959in}{1.366545in}}%
\pgfpathlineto{\pgfqpoint{2.357932in}{1.357508in}}%
\pgfpathlineto{\pgfqpoint{2.372315in}{1.350389in}}%
\pgfpathlineto{\pgfqpoint{2.393890in}{1.339709in}}%
\pgfpathlineto{\pgfqpoint{2.405877in}{1.332589in}}%
\pgfpathlineto{\pgfqpoint{2.427452in}{1.321909in}}%
\pgfpathlineto{\pgfqpoint{2.463411in}{1.296168in}}%
\pgfpathlineto{\pgfqpoint{2.475397in}{1.285215in}}%
\pgfpathlineto{\pgfqpoint{2.523342in}{1.242770in}}%
\pgfpathlineto{\pgfqpoint{2.540123in}{1.230173in}}%
\pgfpathlineto{\pgfqpoint{2.585670in}{1.192657in}}%
\pgfpathlineto{\pgfqpoint{2.604848in}{1.174310in}}%
\pgfpathlineto{\pgfqpoint{2.676766in}{1.109958in}}%
\pgfpathlineto{\pgfqpoint{2.722314in}{1.074632in}}%
\pgfpathlineto{\pgfqpoint{2.743889in}{1.063405in}}%
\pgfpathlineto{\pgfqpoint{2.767861in}{1.051356in}}%
\pgfpathlineto{\pgfqpoint{2.779847in}{1.043415in}}%
\pgfpathlineto{\pgfqpoint{2.796628in}{1.031366in}}%
\pgfpathlineto{\pgfqpoint{2.832587in}{1.011923in}}%
\pgfpathlineto{\pgfqpoint{2.971627in}{0.963180in}}%
\pgfpathlineto{\pgfqpoint{2.990805in}{0.955512in}}%
\pgfpathlineto{\pgfqpoint{3.045942in}{0.937713in}}%
\pgfpathlineto{\pgfqpoint{3.057928in}{0.933879in}}%
\pgfpathlineto{\pgfqpoint{3.077106in}{0.926211in}}%
\pgfpathlineto{\pgfqpoint{3.086695in}{0.920734in}}%
\pgfpathlineto{\pgfqpoint{3.108270in}{0.907043in}}%
\pgfpathlineto{\pgfqpoint{3.139435in}{0.890338in}}%
\pgfpathlineto{\pgfqpoint{3.153818in}{0.884862in}}%
\pgfpathlineto{\pgfqpoint{3.170599in}{0.878016in}}%
\pgfpathlineto{\pgfqpoint{3.213749in}{0.859394in}}%
\pgfpathlineto{\pgfqpoint{3.242516in}{0.848441in}}%
\pgfpathlineto{\pgfqpoint{3.283270in}{0.830915in}}%
\pgfpathlineto{\pgfqpoint{3.307242in}{0.823795in}}%
\pgfpathlineto{\pgfqpoint{3.362379in}{0.810103in}}%
\pgfpathlineto{\pgfqpoint{3.381557in}{0.804627in}}%
\pgfpathlineto{\pgfqpoint{3.479844in}{0.783267in}}%
\pgfpathlineto{\pgfqpoint{3.518200in}{0.777790in}}%
\pgfpathlineto{\pgfqpoint{3.542172in}{0.771218in}}%
\pgfpathlineto{\pgfqpoint{3.587720in}{0.756705in}}%
\pgfpathlineto{\pgfqpoint{3.623679in}{0.747394in}}%
\pgfpathlineto{\pgfqpoint{3.638062in}{0.743560in}}%
\pgfpathlineto{\pgfqpoint{3.676418in}{0.733702in}}%
\pgfpathlineto{\pgfqpoint{3.736350in}{0.714260in}}%
\pgfpathlineto{\pgfqpoint{3.808267in}{0.698925in}}%
\pgfpathlineto{\pgfqpoint{3.825048in}{0.695912in}}%
\pgfpathlineto{\pgfqpoint{3.872993in}{0.686054in}}%
\pgfpathlineto{\pgfqpoint{3.918540in}{0.681947in}}%
\pgfpathlineto{\pgfqpoint{3.935321in}{0.680851in}}%
\pgfpathlineto{\pgfqpoint{3.990458in}{0.675648in}}%
\pgfpathlineto{\pgfqpoint{4.038403in}{0.674553in}}%
\pgfpathlineto{\pgfqpoint{4.057581in}{0.671267in}}%
\pgfpathlineto{\pgfqpoint{4.105526in}{0.666885in}}%
\pgfpathlineto{\pgfqpoint{4.129498in}{0.665790in}}%
\pgfpathlineto{\pgfqpoint{4.189429in}{0.658122in}}%
\pgfpathlineto{\pgfqpoint{4.213402in}{0.651003in}}%
\pgfpathlineto{\pgfqpoint{4.258950in}{0.641692in}}%
\pgfpathlineto{\pgfqpoint{4.282922in}{0.637858in}}%
\pgfpathlineto{\pgfqpoint{4.302100in}{0.632382in}}%
\pgfpathlineto{\pgfqpoint{4.347648in}{0.617868in}}%
\pgfpathlineto{\pgfqpoint{4.376415in}{0.612391in}}%
\pgfpathlineto{\pgfqpoint{4.393196in}{0.609379in}}%
\pgfpathlineto{\pgfqpoint{4.412374in}{0.606093in}}%
\pgfpathlineto{\pgfqpoint{4.465113in}{0.601164in}}%
\pgfpathlineto{\pgfqpoint{4.498674in}{0.603902in}}%
\pgfpathlineto{\pgfqpoint{4.513058in}{0.604998in}}%
\pgfpathlineto{\pgfqpoint{4.546619in}{0.607736in}}%
\pgfpathlineto{\pgfqpoint{4.565797in}{0.608831in}}%
\pgfpathlineto{\pgfqpoint{4.580181in}{0.609379in}}%
\pgfpathlineto{\pgfqpoint{4.644907in}{0.610474in}}%
\pgfpathlineto{\pgfqpoint{4.659290in}{0.613213in}}%
\pgfpathlineto{\pgfqpoint{4.673674in}{0.617047in}}%
\pgfpathlineto{\pgfqpoint{4.700043in}{0.621976in}}%
\pgfpathlineto{\pgfqpoint{4.726413in}{0.623893in}}%
\pgfpathlineto{\pgfqpoint{4.731208in}{0.624166in}}%
\pgfpathlineto{\pgfqpoint{4.731208in}{0.624166in}}%
\pgfusepath{stroke}%
\end{pgfscope}%
\begin{pgfscope}%
\pgfsetrectcap%
\pgfsetmiterjoin%
\pgfsetlinewidth{1.003750pt}%
\definecolor{currentstroke}{rgb}{0.000000,0.000000,0.000000}%
\pgfsetstrokecolor{currentstroke}%
\pgfsetdash{}{0pt}%
\pgfpathmoveto{\pgfqpoint{4.730600in}{0.600890in}}%
\pgfpathlineto{\pgfqpoint{4.730600in}{2.243927in}}%
\pgfusepath{stroke}%
\end{pgfscope}%
\begin{pgfscope}%
\pgfsetrectcap%
\pgfsetmiterjoin%
\pgfsetlinewidth{1.003750pt}%
\definecolor{currentstroke}{rgb}{0.000000,0.000000,0.000000}%
\pgfsetstrokecolor{currentstroke}%
\pgfsetdash{}{0pt}%
\pgfpathmoveto{\pgfqpoint{0.397389in}{0.600890in}}%
\pgfpathlineto{\pgfqpoint{4.730600in}{0.600890in}}%
\pgfusepath{stroke}%
\end{pgfscope}%
\begin{pgfscope}%
\pgfsetrectcap%
\pgfsetmiterjoin%
\pgfsetlinewidth{1.003750pt}%
\definecolor{currentstroke}{rgb}{0.000000,0.000000,0.000000}%
\pgfsetstrokecolor{currentstroke}%
\pgfsetdash{}{0pt}%
\pgfpathmoveto{\pgfqpoint{0.397389in}{2.243927in}}%
\pgfpathlineto{\pgfqpoint{4.730600in}{2.243927in}}%
\pgfusepath{stroke}%
\end{pgfscope}%
\begin{pgfscope}%
\pgfsetrectcap%
\pgfsetmiterjoin%
\pgfsetlinewidth{1.003750pt}%
\definecolor{currentstroke}{rgb}{0.000000,0.000000,0.000000}%
\pgfsetstrokecolor{currentstroke}%
\pgfsetdash{}{0pt}%
\pgfpathmoveto{\pgfqpoint{0.397389in}{0.600890in}}%
\pgfpathlineto{\pgfqpoint{0.397389in}{2.243927in}}%
\pgfusepath{stroke}%
\end{pgfscope}%
\begin{pgfscope}%
\pgfsetbuttcap%
\pgfsetroundjoin%
\definecolor{currentfill}{rgb}{0.000000,0.000000,0.000000}%
\pgfsetfillcolor{currentfill}%
\pgfsetlinewidth{0.501875pt}%
\definecolor{currentstroke}{rgb}{0.000000,0.000000,0.000000}%
\pgfsetstrokecolor{currentstroke}%
\pgfsetdash{}{0pt}%
\pgfsys@defobject{currentmarker}{\pgfqpoint{0.000000in}{0.000000in}}{\pgfqpoint{0.000000in}{0.055556in}}{%
\pgfpathmoveto{\pgfqpoint{0.000000in}{0.000000in}}%
\pgfpathlineto{\pgfqpoint{0.000000in}{0.055556in}}%
\pgfusepath{stroke,fill}%
}%
\begin{pgfscope}%
\pgfsys@transformshift{0.476092in}{0.600890in}%
\pgfsys@useobject{currentmarker}{}%
\end{pgfscope}%
\end{pgfscope}%
\begin{pgfscope}%
\pgfsetbuttcap%
\pgfsetroundjoin%
\definecolor{currentfill}{rgb}{0.000000,0.000000,0.000000}%
\pgfsetfillcolor{currentfill}%
\pgfsetlinewidth{0.501875pt}%
\definecolor{currentstroke}{rgb}{0.000000,0.000000,0.000000}%
\pgfsetstrokecolor{currentstroke}%
\pgfsetdash{}{0pt}%
\pgfsys@defobject{currentmarker}{\pgfqpoint{0.000000in}{-0.055556in}}{\pgfqpoint{0.000000in}{0.000000in}}{%
\pgfpathmoveto{\pgfqpoint{0.000000in}{0.000000in}}%
\pgfpathlineto{\pgfqpoint{0.000000in}{-0.055556in}}%
\pgfusepath{stroke,fill}%
}%
\begin{pgfscope}%
\pgfsys@transformshift{0.476092in}{2.243927in}%
\pgfsys@useobject{currentmarker}{}%
\end{pgfscope}%
\end{pgfscope}%
\begin{pgfscope}%
\pgftext[x=0.476092in,y=0.545334in,,top]{\fontsize{12.000000}{14.400000}\selectfont \(\displaystyle -2\)}%
\end{pgfscope}%
\begin{pgfscope}%
\pgfsetbuttcap%
\pgfsetroundjoin%
\definecolor{currentfill}{rgb}{0.000000,0.000000,0.000000}%
\pgfsetfillcolor{currentfill}%
\pgfsetlinewidth{0.501875pt}%
\definecolor{currentstroke}{rgb}{0.000000,0.000000,0.000000}%
\pgfsetstrokecolor{currentstroke}%
\pgfsetdash{}{0pt}%
\pgfsys@defobject{currentmarker}{\pgfqpoint{0.000000in}{0.000000in}}{\pgfqpoint{0.000000in}{0.055556in}}{%
\pgfpathmoveto{\pgfqpoint{0.000000in}{0.000000in}}%
\pgfpathlineto{\pgfqpoint{0.000000in}{0.055556in}}%
\pgfusepath{stroke,fill}%
}%
\begin{pgfscope}%
\pgfsys@transformshift{1.243211in}{0.600890in}%
\pgfsys@useobject{currentmarker}{}%
\end{pgfscope}%
\end{pgfscope}%
\begin{pgfscope}%
\pgfsetbuttcap%
\pgfsetroundjoin%
\definecolor{currentfill}{rgb}{0.000000,0.000000,0.000000}%
\pgfsetfillcolor{currentfill}%
\pgfsetlinewidth{0.501875pt}%
\definecolor{currentstroke}{rgb}{0.000000,0.000000,0.000000}%
\pgfsetstrokecolor{currentstroke}%
\pgfsetdash{}{0pt}%
\pgfsys@defobject{currentmarker}{\pgfqpoint{0.000000in}{-0.055556in}}{\pgfqpoint{0.000000in}{0.000000in}}{%
\pgfpathmoveto{\pgfqpoint{0.000000in}{0.000000in}}%
\pgfpathlineto{\pgfqpoint{0.000000in}{-0.055556in}}%
\pgfusepath{stroke,fill}%
}%
\begin{pgfscope}%
\pgfsys@transformshift{1.243211in}{2.243927in}%
\pgfsys@useobject{currentmarker}{}%
\end{pgfscope}%
\end{pgfscope}%
\begin{pgfscope}%
\pgftext[x=1.243211in,y=0.545334in,,top]{\fontsize{12.000000}{14.400000}\selectfont \(\displaystyle 0\)}%
\end{pgfscope}%
\begin{pgfscope}%
\pgfsetbuttcap%
\pgfsetroundjoin%
\definecolor{currentfill}{rgb}{0.000000,0.000000,0.000000}%
\pgfsetfillcolor{currentfill}%
\pgfsetlinewidth{0.501875pt}%
\definecolor{currentstroke}{rgb}{0.000000,0.000000,0.000000}%
\pgfsetstrokecolor{currentstroke}%
\pgfsetdash{}{0pt}%
\pgfsys@defobject{currentmarker}{\pgfqpoint{0.000000in}{0.000000in}}{\pgfqpoint{0.000000in}{0.055556in}}{%
\pgfpathmoveto{\pgfqpoint{0.000000in}{0.000000in}}%
\pgfpathlineto{\pgfqpoint{0.000000in}{0.055556in}}%
\pgfusepath{stroke,fill}%
}%
\begin{pgfscope}%
\pgfsys@transformshift{2.010331in}{0.600890in}%
\pgfsys@useobject{currentmarker}{}%
\end{pgfscope}%
\end{pgfscope}%
\begin{pgfscope}%
\pgfsetbuttcap%
\pgfsetroundjoin%
\definecolor{currentfill}{rgb}{0.000000,0.000000,0.000000}%
\pgfsetfillcolor{currentfill}%
\pgfsetlinewidth{0.501875pt}%
\definecolor{currentstroke}{rgb}{0.000000,0.000000,0.000000}%
\pgfsetstrokecolor{currentstroke}%
\pgfsetdash{}{0pt}%
\pgfsys@defobject{currentmarker}{\pgfqpoint{0.000000in}{-0.055556in}}{\pgfqpoint{0.000000in}{0.000000in}}{%
\pgfpathmoveto{\pgfqpoint{0.000000in}{0.000000in}}%
\pgfpathlineto{\pgfqpoint{0.000000in}{-0.055556in}}%
\pgfusepath{stroke,fill}%
}%
\begin{pgfscope}%
\pgfsys@transformshift{2.010331in}{2.243927in}%
\pgfsys@useobject{currentmarker}{}%
\end{pgfscope}%
\end{pgfscope}%
\begin{pgfscope}%
\pgftext[x=2.010331in,y=0.545334in,,top]{\fontsize{12.000000}{14.400000}\selectfont \(\displaystyle 2\)}%
\end{pgfscope}%
\begin{pgfscope}%
\pgfsetbuttcap%
\pgfsetroundjoin%
\definecolor{currentfill}{rgb}{0.000000,0.000000,0.000000}%
\pgfsetfillcolor{currentfill}%
\pgfsetlinewidth{0.501875pt}%
\definecolor{currentstroke}{rgb}{0.000000,0.000000,0.000000}%
\pgfsetstrokecolor{currentstroke}%
\pgfsetdash{}{0pt}%
\pgfsys@defobject{currentmarker}{\pgfqpoint{0.000000in}{0.000000in}}{\pgfqpoint{0.000000in}{0.055556in}}{%
\pgfpathmoveto{\pgfqpoint{0.000000in}{0.000000in}}%
\pgfpathlineto{\pgfqpoint{0.000000in}{0.055556in}}%
\pgfusepath{stroke,fill}%
}%
\begin{pgfscope}%
\pgfsys@transformshift{2.777450in}{0.600890in}%
\pgfsys@useobject{currentmarker}{}%
\end{pgfscope}%
\end{pgfscope}%
\begin{pgfscope}%
\pgfsetbuttcap%
\pgfsetroundjoin%
\definecolor{currentfill}{rgb}{0.000000,0.000000,0.000000}%
\pgfsetfillcolor{currentfill}%
\pgfsetlinewidth{0.501875pt}%
\definecolor{currentstroke}{rgb}{0.000000,0.000000,0.000000}%
\pgfsetstrokecolor{currentstroke}%
\pgfsetdash{}{0pt}%
\pgfsys@defobject{currentmarker}{\pgfqpoint{0.000000in}{-0.055556in}}{\pgfqpoint{0.000000in}{0.000000in}}{%
\pgfpathmoveto{\pgfqpoint{0.000000in}{0.000000in}}%
\pgfpathlineto{\pgfqpoint{0.000000in}{-0.055556in}}%
\pgfusepath{stroke,fill}%
}%
\begin{pgfscope}%
\pgfsys@transformshift{2.777450in}{2.243927in}%
\pgfsys@useobject{currentmarker}{}%
\end{pgfscope}%
\end{pgfscope}%
\begin{pgfscope}%
\pgftext[x=2.777450in,y=0.545334in,,top]{\fontsize{12.000000}{14.400000}\selectfont \(\displaystyle 4\)}%
\end{pgfscope}%
\begin{pgfscope}%
\pgfsetbuttcap%
\pgfsetroundjoin%
\definecolor{currentfill}{rgb}{0.000000,0.000000,0.000000}%
\pgfsetfillcolor{currentfill}%
\pgfsetlinewidth{0.501875pt}%
\definecolor{currentstroke}{rgb}{0.000000,0.000000,0.000000}%
\pgfsetstrokecolor{currentstroke}%
\pgfsetdash{}{0pt}%
\pgfsys@defobject{currentmarker}{\pgfqpoint{0.000000in}{0.000000in}}{\pgfqpoint{0.000000in}{0.055556in}}{%
\pgfpathmoveto{\pgfqpoint{0.000000in}{0.000000in}}%
\pgfpathlineto{\pgfqpoint{0.000000in}{0.055556in}}%
\pgfusepath{stroke,fill}%
}%
\begin{pgfscope}%
\pgfsys@transformshift{3.544570in}{0.600890in}%
\pgfsys@useobject{currentmarker}{}%
\end{pgfscope}%
\end{pgfscope}%
\begin{pgfscope}%
\pgfsetbuttcap%
\pgfsetroundjoin%
\definecolor{currentfill}{rgb}{0.000000,0.000000,0.000000}%
\pgfsetfillcolor{currentfill}%
\pgfsetlinewidth{0.501875pt}%
\definecolor{currentstroke}{rgb}{0.000000,0.000000,0.000000}%
\pgfsetstrokecolor{currentstroke}%
\pgfsetdash{}{0pt}%
\pgfsys@defobject{currentmarker}{\pgfqpoint{0.000000in}{-0.055556in}}{\pgfqpoint{0.000000in}{0.000000in}}{%
\pgfpathmoveto{\pgfqpoint{0.000000in}{0.000000in}}%
\pgfpathlineto{\pgfqpoint{0.000000in}{-0.055556in}}%
\pgfusepath{stroke,fill}%
}%
\begin{pgfscope}%
\pgfsys@transformshift{3.544570in}{2.243927in}%
\pgfsys@useobject{currentmarker}{}%
\end{pgfscope}%
\end{pgfscope}%
\begin{pgfscope}%
\pgftext[x=3.544570in,y=0.545334in,,top]{\fontsize{12.000000}{14.400000}\selectfont \(\displaystyle 6\)}%
\end{pgfscope}%
\begin{pgfscope}%
\pgfsetbuttcap%
\pgfsetroundjoin%
\definecolor{currentfill}{rgb}{0.000000,0.000000,0.000000}%
\pgfsetfillcolor{currentfill}%
\pgfsetlinewidth{0.501875pt}%
\definecolor{currentstroke}{rgb}{0.000000,0.000000,0.000000}%
\pgfsetstrokecolor{currentstroke}%
\pgfsetdash{}{0pt}%
\pgfsys@defobject{currentmarker}{\pgfqpoint{0.000000in}{0.000000in}}{\pgfqpoint{0.000000in}{0.055556in}}{%
\pgfpathmoveto{\pgfqpoint{0.000000in}{0.000000in}}%
\pgfpathlineto{\pgfqpoint{0.000000in}{0.055556in}}%
\pgfusepath{stroke,fill}%
}%
\begin{pgfscope}%
\pgfsys@transformshift{4.311689in}{0.600890in}%
\pgfsys@useobject{currentmarker}{}%
\end{pgfscope}%
\end{pgfscope}%
\begin{pgfscope}%
\pgfsetbuttcap%
\pgfsetroundjoin%
\definecolor{currentfill}{rgb}{0.000000,0.000000,0.000000}%
\pgfsetfillcolor{currentfill}%
\pgfsetlinewidth{0.501875pt}%
\definecolor{currentstroke}{rgb}{0.000000,0.000000,0.000000}%
\pgfsetstrokecolor{currentstroke}%
\pgfsetdash{}{0pt}%
\pgfsys@defobject{currentmarker}{\pgfqpoint{0.000000in}{-0.055556in}}{\pgfqpoint{0.000000in}{0.000000in}}{%
\pgfpathmoveto{\pgfqpoint{0.000000in}{0.000000in}}%
\pgfpathlineto{\pgfqpoint{0.000000in}{-0.055556in}}%
\pgfusepath{stroke,fill}%
}%
\begin{pgfscope}%
\pgfsys@transformshift{4.311689in}{2.243927in}%
\pgfsys@useobject{currentmarker}{}%
\end{pgfscope}%
\end{pgfscope}%
\begin{pgfscope}%
\pgftext[x=4.311689in,y=0.545334in,,top]{\fontsize{12.000000}{14.400000}\selectfont \(\displaystyle 8\)}%
\end{pgfscope}%
\begin{pgfscope}%
\pgftext[x=2.563995in,y=0.327743in,,top]{\fontsize{12.000000}{14.400000}\selectfont Time (ns)}%
\end{pgfscope}%
\begin{pgfscope}%
\pgftext[x=0.327945in,y=1.422408in,,bottom,rotate=90.000000]{\fontsize{12.000000}{14.400000}\selectfont Electrons}%
\end{pgfscope}%
\end{pgfpicture}%
\makeatother%
\endgroup%

    \caption[Comparison between time-resolved electron count and ionisation laser power.]{A comparison of the relative intensities of an unobstructed electron bunch focused and streaked across the detector and the blue laser pulse used to ionise the atoms.
    The top portion shows the electron bunch focused on the detector and streaked, the middle portion shows the intensity of the blue laser, and the bottom portion shows the electron count calculated by sum of each column of the streak image.}
    % 2017.04.12 - ThesisFigure.py
    \label{figure:blue_streak}
\end{figure}

\subsection{Pepperpots}
There were a number of iterations on the precise configuration of the one- and two-dimensional pepperpot masks used in these experiments.
There are a number of considerations when designing pepperpot masks:
\begin{itemize}
    \item{\emph{Aperture size:} Smaller apertures provide a better resolution to the emittance measurement but reduce the signal as fewer particles pass through the hole.
    Aperture size affects the size of the beamlets on the detector, which affects the beamlet overlap. Smaller apertures also allow for a smaller aperture spacing as when the aperture diameter and aperture spacing are similar the pepperpot can become fragile. The aperture size also affects the minimum resolvable emittance with smaller aperture able to resolve lower emittance.}
    \item{\emph{Aperture spacing:} Aperture spacing, also known as pitch, determines how well the full beam is sampled.
    Smaller pitch allows for better sampling of the full beam however pitch should be large enough that the beamlet overlap on the detector is manageable.
    If the pitch is too small then the pepperpot can also become fragile.}
    \item{\emph{Extent:} Ideally a the total extent of the pepperpot should be much larger than the largest beam for a particular apparatus.
    Unfortunately, due to the sample mounting arrangement the size of samples was limited to \unit[3]{mm} diameter and only a \unit[2]{mm} diameter portion of the sample was accessible to the beam. A view of the sample holder without the `lid' is shown in Figure~\ref{figure:sample_holder_pepperpots}.}
\end{itemize}

A number of restrictions aided the selection of parameters for the pepperpots used, the maximum pepperpot extent and need to focus the beam to the pepperpot size to maximise flux dictated the approximate size of the beamlets on the detector and thus the lowest feasibly pitch in order to minimise the overlap of beamlets on the detector.
The pepperpots were cut\footnote{The pepperpots were cut using a Oxford Laser Systems Alpha 532 laser micromachining system.} from \unit[25]{$\muup$m} thick copper pinholes\footnote{Gilder Grids GA50-C3, \unit[50]{$\muup$m} aperture.} with \unit[50]{$\muup$m} diameter holes cut in the centre.
The central pinholes were used as the central aperture for the pepperpot arrays and since \unit[50]{$\muup$m} apertures provide adequate flux the rest of the apertures were also cut to the same diameter to simplify analysis while providing acceptable measurement resolution.
A pitch of \unit[200]{$\muup$m} was chosen to provide acceptable sampling of the beam while minimising the overlap of beamlets on the detector.
Given the \unit[2]{mm} diameter available this pitch allowed for one dimensional pepperpots with 7 apertures and two dimensional pepperpots with 7$\times$7 apertures.
The pepperpots used are shown in Figures~\ref{figure:pepperpot_mask} and \ref{figure:1d_pepperpot}.

\begin{figure}
    \center
    \includegraphics[width=0.49\linewidth]{part2/Figs/sample_holder.jpg}
    \caption[Sample holder.]{One of two sample holders that are used to load up to eight samples into the vacuum system. Clockwise from the top: five \unit[50]{$\muup$m} apertures with \unit[300]{$\muup$m} pitch, 7 by 7 pepperpot of \unit[50]{$\muup$m} apertures with \unit[200]{$\muup$m} pitch, 5 by 5 pepperpot of \unit[50]{$\muup$m} apertures with \unit[300]{$\muup$m} pitch, thin gold sample. Each sample is \unit[3]{mm} in diameter. A `lid' with appropriate apertures is placed over the samples to hold them in place allowing approximately \unit[2]{mm} diameter of each sample accessible to the beam.}
    \label{figure:sample_holder_pepperpots}
\end{figure}

\subsection{Laser Parameters}
The two-colour ionisation scheme required a `excitation' and `ionisation' laser, see Section~\ref{section:two_stage_ionisation}, a \gls{cw} red and pulsed blue laser respectively.
These lasers have a number of parameters that affect the emittance measurements.

\subsubsection{Excitation Laser}
The excitation laser was a \unit[780]{nm} laser, frequency locked to the Rb85 D2 F=3 to F=4 transition and was used in the first step of ionisation to excite ground state atoms that were trapped in the \gls{mot}.
The spatial profile of this laser is highly controllable with a \gls{slm} allowing for arbitrary profiles to be mapped to the electron or ion bunch~\cite{mcculloch_arbitrarily_2011}.
The highest electron flux is possible by simply saturating the \gls{mot} however control over the electron beam profile and size is usually desirable.
One commonly used and simple distribution is a flat-top distribution as it can provide even electron intensity across the sample under examination.
High emittance beams are unable to maintain the profile given by the excitation laser and will trend towards a Gaussian profile as they propagate, see Figure~\ref{figure:flat_top}.

\begin{figure}
    \center
    %% Creator: Matplotlib, PGF backend
%%
%% To include the figure in your LaTeX document, write
%%   \input{<filename>.pgf}
%%
%% Make sure the required packages are loaded in your preamble
%%   \usepackage{pgf}
%%
%% Figures using additional raster images can only be included by \input if
%% they are in the same directory as the main LaTeX file. For loading figures
%% from other directories you can use the `import` package
%%   \usepackage{import}
%% and then include the figures with
%%   \import{<path to file>}{<filename>.pgf}
%%
%% Matplotlib used the following preamble
%%
\begingroup%
\makeatletter%
\begin{pgfpicture}%
\pgfpathrectangle{\pgfpointorigin}{\pgfqpoint{5.710000in}{4.301533in}}%
\pgfusepath{use as bounding box, clip}%
\begin{pgfscope}%
\pgfsetbuttcap%
\pgfsetmiterjoin%
\definecolor{currentfill}{rgb}{1.000000,1.000000,1.000000}%
\pgfsetfillcolor{currentfill}%
\pgfsetlinewidth{0.000000pt}%
\definecolor{currentstroke}{rgb}{1.000000,1.000000,1.000000}%
\pgfsetstrokecolor{currentstroke}%
\pgfsetdash{}{0pt}%
\pgfpathmoveto{\pgfqpoint{0.000000in}{0.000000in}}%
\pgfpathlineto{\pgfqpoint{5.710000in}{0.000000in}}%
\pgfpathlineto{\pgfqpoint{5.710000in}{4.301533in}}%
\pgfpathlineto{\pgfqpoint{0.000000in}{4.301533in}}%
\pgfpathclose%
\pgfusepath{fill}%
\end{pgfscope}%
\begin{pgfscope}%
\pgfsetbuttcap%
\pgfsetmiterjoin%
\definecolor{currentfill}{rgb}{1.000000,1.000000,1.000000}%
\pgfsetfillcolor{currentfill}%
\pgfsetlinewidth{0.000000pt}%
\definecolor{currentstroke}{rgb}{0.000000,0.000000,0.000000}%
\pgfsetstrokecolor{currentstroke}%
\pgfsetstrokeopacity{0.000000}%
\pgfsetdash{}{0pt}%
\pgfpathmoveto{\pgfqpoint{0.344444in}{2.313483in}}%
\pgfpathlineto{\pgfqpoint{2.082963in}{2.313483in}}%
\pgfpathlineto{\pgfqpoint{2.082963in}{4.052001in}}%
\pgfpathlineto{\pgfqpoint{0.344444in}{4.052001in}}%
\pgfpathclose%
\pgfusepath{fill}%
\end{pgfscope}%
\begin{pgfscope}%
\pgfpathrectangle{\pgfqpoint{0.344444in}{2.313483in}}{\pgfqpoint{1.738519in}{1.738519in}} %
\pgfusepath{clip}%
\pgftext[at=\pgfqpoint{0.344444in}{2.313483in},left,bottom]{\pgfimage[interpolate=true,width=1.750000in,height=1.750000in]{flattops_emittance-img0.png}}%
\end{pgfscope}%
\begin{pgfscope}%
\pgfpathrectangle{\pgfqpoint{0.344444in}{2.313483in}}{\pgfqpoint{1.738519in}{1.738519in}} %
\pgfusepath{clip}%
\pgfsetbuttcap%
\pgfsetroundjoin%
\pgfsetlinewidth{1.003750pt}%
\definecolor{currentstroke}{rgb}{1.000000,1.000000,1.000000}%
\pgfsetstrokecolor{currentstroke}%
\pgfsetdash{{6.000000pt}{6.000000pt}}{0.000000pt}%
\pgfpathmoveto{\pgfqpoint{0.344444in}{3.179844in}}%
\pgfpathlineto{\pgfqpoint{2.082963in}{3.179844in}}%
\pgfusepath{stroke}%
\end{pgfscope}%
\begin{pgfscope}%
\pgfsetrectcap%
\pgfsetmiterjoin%
\pgfsetlinewidth{1.003750pt}%
\definecolor{currentstroke}{rgb}{0.000000,0.000000,0.000000}%
\pgfsetstrokecolor{currentstroke}%
\pgfsetdash{}{0pt}%
\pgfpathmoveto{\pgfqpoint{0.344444in}{4.052001in}}%
\pgfpathlineto{\pgfqpoint{2.082963in}{4.052001in}}%
\pgfusepath{stroke}%
\end{pgfscope}%
\begin{pgfscope}%
\pgfsetrectcap%
\pgfsetmiterjoin%
\pgfsetlinewidth{1.003750pt}%
\definecolor{currentstroke}{rgb}{0.000000,0.000000,0.000000}%
\pgfsetstrokecolor{currentstroke}%
\pgfsetdash{}{0pt}%
\pgfpathmoveto{\pgfqpoint{0.344444in}{2.313483in}}%
\pgfpathlineto{\pgfqpoint{0.344444in}{4.052001in}}%
\pgfusepath{stroke}%
\end{pgfscope}%
\begin{pgfscope}%
\pgfsetrectcap%
\pgfsetmiterjoin%
\pgfsetlinewidth{1.003750pt}%
\definecolor{currentstroke}{rgb}{0.000000,0.000000,0.000000}%
\pgfsetstrokecolor{currentstroke}%
\pgfsetdash{}{0pt}%
\pgfpathmoveto{\pgfqpoint{0.344444in}{2.313483in}}%
\pgfpathlineto{\pgfqpoint{2.082963in}{2.313483in}}%
\pgfusepath{stroke}%
\end{pgfscope}%
\begin{pgfscope}%
\pgfsetrectcap%
\pgfsetmiterjoin%
\pgfsetlinewidth{1.003750pt}%
\definecolor{currentstroke}{rgb}{0.000000,0.000000,0.000000}%
\pgfsetstrokecolor{currentstroke}%
\pgfsetdash{}{0pt}%
\pgfpathmoveto{\pgfqpoint{2.082963in}{2.313483in}}%
\pgfpathlineto{\pgfqpoint{2.082963in}{4.052001in}}%
\pgfusepath{stroke}%
\end{pgfscope}%
\begin{pgfscope}%
\pgftext[x=1.213704in,y=4.121446in,,base]{\rmfamily\fontsize{12.000000}{14.400000}\selectfont -42.0\(\displaystyle \,\)meV}%
\end{pgfscope}%
\begin{pgfscope}%
\pgfsetbuttcap%
\pgfsetmiterjoin%
\definecolor{currentfill}{rgb}{1.000000,1.000000,1.000000}%
\pgfsetfillcolor{currentfill}%
\pgfsetlinewidth{0.000000pt}%
\definecolor{currentstroke}{rgb}{0.000000,0.000000,0.000000}%
\pgfsetstrokecolor{currentstroke}%
\pgfsetstrokeopacity{0.000000}%
\pgfsetdash{}{0pt}%
\pgfpathmoveto{\pgfqpoint{0.344444in}{0.528906in}}%
\pgfpathlineto{\pgfqpoint{2.082963in}{0.528906in}}%
\pgfpathlineto{\pgfqpoint{2.082963in}{2.298130in}}%
\pgfpathlineto{\pgfqpoint{0.344444in}{2.298130in}}%
\pgfpathclose%
\pgfusepath{fill}%
\end{pgfscope}%
\begin{pgfscope}%
\pgfpathrectangle{\pgfqpoint{0.344444in}{0.528906in}}{\pgfqpoint{1.738519in}{1.769224in}} %
\pgfusepath{clip}%
\pgfsetrectcap%
\pgfsetroundjoin%
\pgfsetlinewidth{1.003750pt}%
\definecolor{currentstroke}{rgb}{0.309804,0.478431,0.682353}%
\pgfsetstrokecolor{currentstroke}%
\pgfsetdash{}{0pt}%
\pgfpathmoveto{\pgfqpoint{0.548454in}{0.528906in}}%
\pgfpathlineto{\pgfqpoint{0.552889in}{0.533554in}}%
\pgfpathlineto{\pgfqpoint{0.557324in}{0.533836in}}%
\pgfpathlineto{\pgfqpoint{0.561759in}{0.536504in}}%
\pgfpathlineto{\pgfqpoint{0.570629in}{0.537677in}}%
\pgfpathlineto{\pgfqpoint{0.575064in}{0.535328in}}%
\pgfpathlineto{\pgfqpoint{0.579499in}{0.534252in}}%
\pgfpathlineto{\pgfqpoint{0.592804in}{0.540686in}}%
\pgfpathlineto{\pgfqpoint{0.597239in}{0.536114in}}%
\pgfpathlineto{\pgfqpoint{0.601674in}{0.536883in}}%
\pgfpathlineto{\pgfqpoint{0.606109in}{0.539996in}}%
\pgfpathlineto{\pgfqpoint{0.610544in}{0.538258in}}%
\pgfpathlineto{\pgfqpoint{0.619414in}{0.539461in}}%
\pgfpathlineto{\pgfqpoint{0.623849in}{0.545196in}}%
\pgfpathlineto{\pgfqpoint{0.628284in}{0.547227in}}%
\pgfpathlineto{\pgfqpoint{0.641589in}{0.548122in}}%
\pgfpathlineto{\pgfqpoint{0.646024in}{0.557476in}}%
\pgfpathlineto{\pgfqpoint{0.650459in}{0.556931in}}%
\pgfpathlineto{\pgfqpoint{0.663764in}{0.545651in}}%
\pgfpathlineto{\pgfqpoint{0.668199in}{0.546282in}}%
\pgfpathlineto{\pgfqpoint{0.672634in}{0.549522in}}%
\pgfpathlineto{\pgfqpoint{0.677069in}{0.548602in}}%
\pgfpathlineto{\pgfqpoint{0.685939in}{0.562588in}}%
\pgfpathlineto{\pgfqpoint{0.694809in}{0.564962in}}%
\pgfpathlineto{\pgfqpoint{0.712549in}{0.580311in}}%
\pgfpathlineto{\pgfqpoint{0.716984in}{0.581485in}}%
\pgfpathlineto{\pgfqpoint{0.725854in}{0.573942in}}%
\pgfpathlineto{\pgfqpoint{0.734724in}{0.567367in}}%
\pgfpathlineto{\pgfqpoint{0.743594in}{0.574467in}}%
\pgfpathlineto{\pgfqpoint{0.752464in}{0.578663in}}%
\pgfpathlineto{\pgfqpoint{0.765769in}{0.593475in}}%
\pgfpathlineto{\pgfqpoint{0.774639in}{0.587626in}}%
\pgfpathlineto{\pgfqpoint{0.787944in}{0.602622in}}%
\pgfpathlineto{\pgfqpoint{0.801249in}{0.620761in}}%
\pgfpathlineto{\pgfqpoint{0.805684in}{0.636240in}}%
\pgfpathlineto{\pgfqpoint{0.810119in}{0.643149in}}%
\pgfpathlineto{\pgfqpoint{0.818989in}{0.661469in}}%
\pgfpathlineto{\pgfqpoint{0.827859in}{0.673765in}}%
\pgfpathlineto{\pgfqpoint{0.832294in}{0.681999in}}%
\pgfpathlineto{\pgfqpoint{0.841164in}{0.706374in}}%
\pgfpathlineto{\pgfqpoint{0.845599in}{0.721586in}}%
\pgfpathlineto{\pgfqpoint{0.854469in}{0.737340in}}%
\pgfpathlineto{\pgfqpoint{0.858904in}{0.749076in}}%
\pgfpathlineto{\pgfqpoint{0.863339in}{0.769561in}}%
\pgfpathlineto{\pgfqpoint{0.867774in}{0.777734in}}%
\pgfpathlineto{\pgfqpoint{0.885514in}{0.899942in}}%
\pgfpathlineto{\pgfqpoint{0.889949in}{0.916907in}}%
\pgfpathlineto{\pgfqpoint{0.903254in}{1.027988in}}%
\pgfpathlineto{\pgfqpoint{0.912124in}{1.165294in}}%
\pgfpathlineto{\pgfqpoint{0.925429in}{1.314570in}}%
\pgfpathlineto{\pgfqpoint{0.934299in}{1.444410in}}%
\pgfpathlineto{\pgfqpoint{0.960909in}{1.775648in}}%
\pgfpathlineto{\pgfqpoint{0.969779in}{1.835196in}}%
\pgfpathlineto{\pgfqpoint{0.974214in}{1.867277in}}%
\pgfpathlineto{\pgfqpoint{0.978649in}{1.878871in}}%
\pgfpathlineto{\pgfqpoint{0.983084in}{1.896056in}}%
\pgfpathlineto{\pgfqpoint{0.991954in}{1.995545in}}%
\pgfpathlineto{\pgfqpoint{0.996389in}{2.033174in}}%
\pgfpathlineto{\pgfqpoint{1.000824in}{2.024836in}}%
\pgfpathlineto{\pgfqpoint{1.005259in}{2.013753in}}%
\pgfpathlineto{\pgfqpoint{1.009694in}{2.017897in}}%
\pgfpathlineto{\pgfqpoint{1.014129in}{2.066395in}}%
\pgfpathlineto{\pgfqpoint{1.018564in}{2.088728in}}%
\pgfpathlineto{\pgfqpoint{1.022999in}{2.122734in}}%
\pgfpathlineto{\pgfqpoint{1.027434in}{2.128927in}}%
\pgfpathlineto{\pgfqpoint{1.031869in}{2.111144in}}%
\pgfpathlineto{\pgfqpoint{1.036304in}{2.105857in}}%
\pgfpathlineto{\pgfqpoint{1.040739in}{2.104194in}}%
\pgfpathlineto{\pgfqpoint{1.049609in}{2.165419in}}%
\pgfpathlineto{\pgfqpoint{1.054044in}{2.176166in}}%
\pgfpathlineto{\pgfqpoint{1.058479in}{2.162533in}}%
\pgfpathlineto{\pgfqpoint{1.062914in}{2.154452in}}%
\pgfpathlineto{\pgfqpoint{1.067349in}{2.156928in}}%
\pgfpathlineto{\pgfqpoint{1.071784in}{2.162204in}}%
\pgfpathlineto{\pgfqpoint{1.076219in}{2.184277in}}%
\pgfpathlineto{\pgfqpoint{1.080654in}{2.191096in}}%
\pgfpathlineto{\pgfqpoint{1.085089in}{2.163031in}}%
\pgfpathlineto{\pgfqpoint{1.089524in}{2.157762in}}%
\pgfpathlineto{\pgfqpoint{1.093959in}{2.165795in}}%
\pgfpathlineto{\pgfqpoint{1.098394in}{2.177782in}}%
\pgfpathlineto{\pgfqpoint{1.102829in}{2.178694in}}%
\pgfpathlineto{\pgfqpoint{1.107264in}{2.181721in}}%
\pgfpathlineto{\pgfqpoint{1.111699in}{2.169696in}}%
\pgfpathlineto{\pgfqpoint{1.120569in}{2.095751in}}%
\pgfpathlineto{\pgfqpoint{1.125004in}{2.077058in}}%
\pgfpathlineto{\pgfqpoint{1.129439in}{2.082405in}}%
\pgfpathlineto{\pgfqpoint{1.133874in}{2.115723in}}%
\pgfpathlineto{\pgfqpoint{1.138309in}{2.126944in}}%
\pgfpathlineto{\pgfqpoint{1.142744in}{2.121590in}}%
\pgfpathlineto{\pgfqpoint{1.147179in}{2.110296in}}%
\pgfpathlineto{\pgfqpoint{1.151614in}{2.095571in}}%
\pgfpathlineto{\pgfqpoint{1.156049in}{2.097762in}}%
\pgfpathlineto{\pgfqpoint{1.169354in}{2.093174in}}%
\pgfpathlineto{\pgfqpoint{1.173789in}{2.071650in}}%
\pgfpathlineto{\pgfqpoint{1.182659in}{2.018304in}}%
\pgfpathlineto{\pgfqpoint{1.187094in}{2.022916in}}%
\pgfpathlineto{\pgfqpoint{1.191529in}{2.061376in}}%
\pgfpathlineto{\pgfqpoint{1.195964in}{2.062878in}}%
\pgfpathlineto{\pgfqpoint{1.204834in}{2.051515in}}%
\pgfpathlineto{\pgfqpoint{1.209269in}{2.031225in}}%
\pgfpathlineto{\pgfqpoint{1.213704in}{2.032011in}}%
\pgfpathlineto{\pgfqpoint{1.218139in}{2.064403in}}%
\pgfpathlineto{\pgfqpoint{1.227009in}{2.110361in}}%
\pgfpathlineto{\pgfqpoint{1.231444in}{2.117540in}}%
\pgfpathlineto{\pgfqpoint{1.240314in}{2.057915in}}%
\pgfpathlineto{\pgfqpoint{1.244749in}{2.076094in}}%
\pgfpathlineto{\pgfqpoint{1.249184in}{2.123289in}}%
\pgfpathlineto{\pgfqpoint{1.253619in}{2.124280in}}%
\pgfpathlineto{\pgfqpoint{1.258054in}{2.117004in}}%
\pgfpathlineto{\pgfqpoint{1.266924in}{2.076071in}}%
\pgfpathlineto{\pgfqpoint{1.271359in}{2.093355in}}%
\pgfpathlineto{\pgfqpoint{1.275794in}{2.149568in}}%
\pgfpathlineto{\pgfqpoint{1.280229in}{2.173950in}}%
\pgfpathlineto{\pgfqpoint{1.284664in}{2.183126in}}%
\pgfpathlineto{\pgfqpoint{1.289099in}{2.170925in}}%
\pgfpathlineto{\pgfqpoint{1.293534in}{2.133447in}}%
\pgfpathlineto{\pgfqpoint{1.297969in}{2.140297in}}%
\pgfpathlineto{\pgfqpoint{1.302404in}{2.158230in}}%
\pgfpathlineto{\pgfqpoint{1.306839in}{2.192069in}}%
\pgfpathlineto{\pgfqpoint{1.315709in}{2.210618in}}%
\pgfpathlineto{\pgfqpoint{1.320144in}{2.198444in}}%
\pgfpathlineto{\pgfqpoint{1.324579in}{2.171601in}}%
\pgfpathlineto{\pgfqpoint{1.329014in}{2.179477in}}%
\pgfpathlineto{\pgfqpoint{1.337884in}{2.213881in}}%
\pgfpathlineto{\pgfqpoint{1.342319in}{2.203182in}}%
\pgfpathlineto{\pgfqpoint{1.355624in}{2.142320in}}%
\pgfpathlineto{\pgfqpoint{1.360059in}{2.162022in}}%
\pgfpathlineto{\pgfqpoint{1.364494in}{2.173806in}}%
\pgfpathlineto{\pgfqpoint{1.368929in}{2.202371in}}%
\pgfpathlineto{\pgfqpoint{1.373364in}{2.191003in}}%
\pgfpathlineto{\pgfqpoint{1.377799in}{2.155151in}}%
\pgfpathlineto{\pgfqpoint{1.382234in}{2.131069in}}%
\pgfpathlineto{\pgfqpoint{1.386669in}{2.144175in}}%
\pgfpathlineto{\pgfqpoint{1.391104in}{2.175426in}}%
\pgfpathlineto{\pgfqpoint{1.395539in}{2.175458in}}%
\pgfpathlineto{\pgfqpoint{1.408844in}{2.100592in}}%
\pgfpathlineto{\pgfqpoint{1.413279in}{2.088977in}}%
\pgfpathlineto{\pgfqpoint{1.417714in}{2.098148in}}%
\pgfpathlineto{\pgfqpoint{1.422149in}{2.097860in}}%
\pgfpathlineto{\pgfqpoint{1.426584in}{2.050440in}}%
\pgfpathlineto{\pgfqpoint{1.431019in}{2.025320in}}%
\pgfpathlineto{\pgfqpoint{1.435454in}{1.978909in}}%
\pgfpathlineto{\pgfqpoint{1.448759in}{1.909175in}}%
\pgfpathlineto{\pgfqpoint{1.453193in}{1.872184in}}%
\pgfpathlineto{\pgfqpoint{1.462063in}{1.823466in}}%
\pgfpathlineto{\pgfqpoint{1.466498in}{1.752877in}}%
\pgfpathlineto{\pgfqpoint{1.470933in}{1.700121in}}%
\pgfpathlineto{\pgfqpoint{1.475368in}{1.674704in}}%
\pgfpathlineto{\pgfqpoint{1.484238in}{1.595780in}}%
\pgfpathlineto{\pgfqpoint{1.501978in}{1.371753in}}%
\pgfpathlineto{\pgfqpoint{1.510848in}{1.305171in}}%
\pgfpathlineto{\pgfqpoint{1.515283in}{1.255704in}}%
\pgfpathlineto{\pgfqpoint{1.524153in}{1.117114in}}%
\pgfpathlineto{\pgfqpoint{1.528588in}{1.072745in}}%
\pgfpathlineto{\pgfqpoint{1.541893in}{0.995896in}}%
\pgfpathlineto{\pgfqpoint{1.546328in}{0.969754in}}%
\pgfpathlineto{\pgfqpoint{1.550763in}{0.961544in}}%
\pgfpathlineto{\pgfqpoint{1.555198in}{0.958844in}}%
\pgfpathlineto{\pgfqpoint{1.564068in}{0.959427in}}%
\pgfpathlineto{\pgfqpoint{1.581808in}{0.896625in}}%
\pgfpathlineto{\pgfqpoint{1.590678in}{0.872929in}}%
\pgfpathlineto{\pgfqpoint{1.599548in}{0.826063in}}%
\pgfpathlineto{\pgfqpoint{1.603983in}{0.792994in}}%
\pgfpathlineto{\pgfqpoint{1.617288in}{0.733911in}}%
\pgfpathlineto{\pgfqpoint{1.626158in}{0.720527in}}%
\pgfpathlineto{\pgfqpoint{1.630593in}{0.709343in}}%
\pgfpathlineto{\pgfqpoint{1.635028in}{0.692188in}}%
\pgfpathlineto{\pgfqpoint{1.648333in}{0.657140in}}%
\pgfpathlineto{\pgfqpoint{1.652768in}{0.662336in}}%
\pgfpathlineto{\pgfqpoint{1.657203in}{0.661157in}}%
\pgfpathlineto{\pgfqpoint{1.661638in}{0.662074in}}%
\pgfpathlineto{\pgfqpoint{1.670508in}{0.641313in}}%
\pgfpathlineto{\pgfqpoint{1.679378in}{0.637475in}}%
\pgfpathlineto{\pgfqpoint{1.688248in}{0.638295in}}%
\pgfpathlineto{\pgfqpoint{1.692683in}{0.645075in}}%
\pgfpathlineto{\pgfqpoint{1.697118in}{0.637880in}}%
\pgfpathlineto{\pgfqpoint{1.701553in}{0.641279in}}%
\pgfpathlineto{\pgfqpoint{1.705988in}{0.637541in}}%
\pgfpathlineto{\pgfqpoint{1.710423in}{0.642227in}}%
\pgfpathlineto{\pgfqpoint{1.714858in}{0.642190in}}%
\pgfpathlineto{\pgfqpoint{1.719293in}{0.640376in}}%
\pgfpathlineto{\pgfqpoint{1.723728in}{0.632655in}}%
\pgfpathlineto{\pgfqpoint{1.728163in}{0.622226in}}%
\pgfpathlineto{\pgfqpoint{1.732598in}{0.620279in}}%
\pgfpathlineto{\pgfqpoint{1.737033in}{0.622260in}}%
\pgfpathlineto{\pgfqpoint{1.741468in}{0.621093in}}%
\pgfpathlineto{\pgfqpoint{1.750338in}{0.614419in}}%
\pgfpathlineto{\pgfqpoint{1.754773in}{0.615245in}}%
\pgfpathlineto{\pgfqpoint{1.759208in}{0.612564in}}%
\pgfpathlineto{\pgfqpoint{1.763643in}{0.611813in}}%
\pgfpathlineto{\pgfqpoint{1.768078in}{0.612488in}}%
\pgfpathlineto{\pgfqpoint{1.776948in}{0.619127in}}%
\pgfpathlineto{\pgfqpoint{1.781383in}{0.622605in}}%
\pgfpathlineto{\pgfqpoint{1.790253in}{0.612137in}}%
\pgfpathlineto{\pgfqpoint{1.794688in}{0.615217in}}%
\pgfpathlineto{\pgfqpoint{1.799123in}{0.620545in}}%
\pgfpathlineto{\pgfqpoint{1.803558in}{0.610906in}}%
\pgfpathlineto{\pgfqpoint{1.807993in}{0.611681in}}%
\pgfpathlineto{\pgfqpoint{1.812428in}{0.602116in}}%
\pgfpathlineto{\pgfqpoint{1.816863in}{0.596717in}}%
\pgfpathlineto{\pgfqpoint{1.821298in}{0.597597in}}%
\pgfpathlineto{\pgfqpoint{1.830168in}{0.602991in}}%
\pgfpathlineto{\pgfqpoint{1.839038in}{0.590150in}}%
\pgfpathlineto{\pgfqpoint{1.843473in}{0.584606in}}%
\pgfpathlineto{\pgfqpoint{1.852343in}{0.587402in}}%
\pgfpathlineto{\pgfqpoint{1.856778in}{0.578737in}}%
\pgfpathlineto{\pgfqpoint{1.861213in}{0.580871in}}%
\pgfpathlineto{\pgfqpoint{1.865648in}{0.587963in}}%
\pgfpathlineto{\pgfqpoint{1.870083in}{0.584104in}}%
\pgfpathlineto{\pgfqpoint{1.874518in}{0.584852in}}%
\pgfpathlineto{\pgfqpoint{1.874518in}{0.584852in}}%
\pgfusepath{stroke}%
\end{pgfscope}%
\begin{pgfscope}%
\pgfsetrectcap%
\pgfsetmiterjoin%
\pgfsetlinewidth{1.003750pt}%
\definecolor{currentstroke}{rgb}{0.000000,0.000000,0.000000}%
\pgfsetstrokecolor{currentstroke}%
\pgfsetdash{}{0pt}%
\pgfpathmoveto{\pgfqpoint{0.344444in}{2.298130in}}%
\pgfpathlineto{\pgfqpoint{2.082963in}{2.298130in}}%
\pgfusepath{stroke}%
\end{pgfscope}%
\begin{pgfscope}%
\pgfsetrectcap%
\pgfsetmiterjoin%
\pgfsetlinewidth{1.003750pt}%
\definecolor{currentstroke}{rgb}{0.000000,0.000000,0.000000}%
\pgfsetstrokecolor{currentstroke}%
\pgfsetdash{}{0pt}%
\pgfpathmoveto{\pgfqpoint{0.344444in}{0.528906in}}%
\pgfpathlineto{\pgfqpoint{0.344444in}{2.298130in}}%
\pgfusepath{stroke}%
\end{pgfscope}%
\begin{pgfscope}%
\pgfsetrectcap%
\pgfsetmiterjoin%
\pgfsetlinewidth{1.003750pt}%
\definecolor{currentstroke}{rgb}{0.000000,0.000000,0.000000}%
\pgfsetstrokecolor{currentstroke}%
\pgfsetdash{}{0pt}%
\pgfpathmoveto{\pgfqpoint{0.344444in}{0.528906in}}%
\pgfpathlineto{\pgfqpoint{2.082963in}{0.528906in}}%
\pgfusepath{stroke}%
\end{pgfscope}%
\begin{pgfscope}%
\pgfsetrectcap%
\pgfsetmiterjoin%
\pgfsetlinewidth{1.003750pt}%
\definecolor{currentstroke}{rgb}{0.000000,0.000000,0.000000}%
\pgfsetstrokecolor{currentstroke}%
\pgfsetdash{}{0pt}%
\pgfpathmoveto{\pgfqpoint{2.082963in}{0.528906in}}%
\pgfpathlineto{\pgfqpoint{2.082963in}{2.298130in}}%
\pgfusepath{stroke}%
\end{pgfscope}%
\begin{pgfscope}%
\pgfsetbuttcap%
\pgfsetroundjoin%
\definecolor{currentfill}{rgb}{0.000000,0.000000,0.000000}%
\pgfsetfillcolor{currentfill}%
\pgfsetlinewidth{0.501875pt}%
\definecolor{currentstroke}{rgb}{0.000000,0.000000,0.000000}%
\pgfsetstrokecolor{currentstroke}%
\pgfsetdash{}{0pt}%
\pgfsys@defobject{currentmarker}{\pgfqpoint{0.000000in}{0.000000in}}{\pgfqpoint{0.000000in}{0.055556in}}{%
\pgfpathmoveto{\pgfqpoint{0.000000in}{0.000000in}}%
\pgfpathlineto{\pgfqpoint{0.000000in}{0.055556in}}%
\pgfusepath{stroke,fill}%
}%
\begin{pgfscope}%
\pgfsys@transformshift{0.670417in}{0.528906in}%
\pgfsys@useobject{currentmarker}{}%
\end{pgfscope}%
\end{pgfscope}%
\begin{pgfscope}%
\pgfsetbuttcap%
\pgfsetroundjoin%
\definecolor{currentfill}{rgb}{0.000000,0.000000,0.000000}%
\pgfsetfillcolor{currentfill}%
\pgfsetlinewidth{0.501875pt}%
\definecolor{currentstroke}{rgb}{0.000000,0.000000,0.000000}%
\pgfsetstrokecolor{currentstroke}%
\pgfsetdash{}{0pt}%
\pgfsys@defobject{currentmarker}{\pgfqpoint{0.000000in}{-0.055556in}}{\pgfqpoint{0.000000in}{0.000000in}}{%
\pgfpathmoveto{\pgfqpoint{0.000000in}{0.000000in}}%
\pgfpathlineto{\pgfqpoint{0.000000in}{-0.055556in}}%
\pgfusepath{stroke,fill}%
}%
\begin{pgfscope}%
\pgfsys@transformshift{0.670417in}{2.298130in}%
\pgfsys@useobject{currentmarker}{}%
\end{pgfscope}%
\end{pgfscope}%
\begin{pgfscope}%
\pgftext[x=0.670417in,y=0.473351in,,top]{\rmfamily\fontsize{10.000000}{12.000000}\selectfont -5}%
\end{pgfscope}%
\begin{pgfscope}%
\pgfsetbuttcap%
\pgfsetroundjoin%
\definecolor{currentfill}{rgb}{0.000000,0.000000,0.000000}%
\pgfsetfillcolor{currentfill}%
\pgfsetlinewidth{0.501875pt}%
\definecolor{currentstroke}{rgb}{0.000000,0.000000,0.000000}%
\pgfsetstrokecolor{currentstroke}%
\pgfsetdash{}{0pt}%
\pgfsys@defobject{currentmarker}{\pgfqpoint{0.000000in}{0.000000in}}{\pgfqpoint{0.000000in}{0.055556in}}{%
\pgfpathmoveto{\pgfqpoint{0.000000in}{0.000000in}}%
\pgfpathlineto{\pgfqpoint{0.000000in}{0.055556in}}%
\pgfusepath{stroke,fill}%
}%
\begin{pgfscope}%
\pgfsys@transformshift{1.213704in}{0.528906in}%
\pgfsys@useobject{currentmarker}{}%
\end{pgfscope}%
\end{pgfscope}%
\begin{pgfscope}%
\pgfsetbuttcap%
\pgfsetroundjoin%
\definecolor{currentfill}{rgb}{0.000000,0.000000,0.000000}%
\pgfsetfillcolor{currentfill}%
\pgfsetlinewidth{0.501875pt}%
\definecolor{currentstroke}{rgb}{0.000000,0.000000,0.000000}%
\pgfsetstrokecolor{currentstroke}%
\pgfsetdash{}{0pt}%
\pgfsys@defobject{currentmarker}{\pgfqpoint{0.000000in}{-0.055556in}}{\pgfqpoint{0.000000in}{0.000000in}}{%
\pgfpathmoveto{\pgfqpoint{0.000000in}{0.000000in}}%
\pgfpathlineto{\pgfqpoint{0.000000in}{-0.055556in}}%
\pgfusepath{stroke,fill}%
}%
\begin{pgfscope}%
\pgfsys@transformshift{1.213704in}{2.298130in}%
\pgfsys@useobject{currentmarker}{}%
\end{pgfscope}%
\end{pgfscope}%
\begin{pgfscope}%
\pgftext[x=1.213704in,y=0.473351in,,top]{\rmfamily\fontsize{10.000000}{12.000000}\selectfont 0}%
\end{pgfscope}%
\begin{pgfscope}%
\pgfsetbuttcap%
\pgfsetroundjoin%
\definecolor{currentfill}{rgb}{0.000000,0.000000,0.000000}%
\pgfsetfillcolor{currentfill}%
\pgfsetlinewidth{0.501875pt}%
\definecolor{currentstroke}{rgb}{0.000000,0.000000,0.000000}%
\pgfsetstrokecolor{currentstroke}%
\pgfsetdash{}{0pt}%
\pgfsys@defobject{currentmarker}{\pgfqpoint{0.000000in}{0.000000in}}{\pgfqpoint{0.000000in}{0.055556in}}{%
\pgfpathmoveto{\pgfqpoint{0.000000in}{0.000000in}}%
\pgfpathlineto{\pgfqpoint{0.000000in}{0.055556in}}%
\pgfusepath{stroke,fill}%
}%
\begin{pgfscope}%
\pgfsys@transformshift{1.756991in}{0.528906in}%
\pgfsys@useobject{currentmarker}{}%
\end{pgfscope}%
\end{pgfscope}%
\begin{pgfscope}%
\pgfsetbuttcap%
\pgfsetroundjoin%
\definecolor{currentfill}{rgb}{0.000000,0.000000,0.000000}%
\pgfsetfillcolor{currentfill}%
\pgfsetlinewidth{0.501875pt}%
\definecolor{currentstroke}{rgb}{0.000000,0.000000,0.000000}%
\pgfsetstrokecolor{currentstroke}%
\pgfsetdash{}{0pt}%
\pgfsys@defobject{currentmarker}{\pgfqpoint{0.000000in}{-0.055556in}}{\pgfqpoint{0.000000in}{0.000000in}}{%
\pgfpathmoveto{\pgfqpoint{0.000000in}{0.000000in}}%
\pgfpathlineto{\pgfqpoint{0.000000in}{-0.055556in}}%
\pgfusepath{stroke,fill}%
}%
\begin{pgfscope}%
\pgfsys@transformshift{1.756991in}{2.298130in}%
\pgfsys@useobject{currentmarker}{}%
\end{pgfscope}%
\end{pgfscope}%
\begin{pgfscope}%
\pgftext[x=1.756991in,y=0.473351in,,top]{\rmfamily\fontsize{10.000000}{12.000000}\selectfont 5}%
\end{pgfscope}%
\begin{pgfscope}%
\pgftext[x=0.275000in,y=1.413518in,,bottom,rotate=90.000000]{\rmfamily\fontsize{10.000000}{12.000000}\selectfont Normalised Intensity}%
\end{pgfscope}%
\begin{pgfscope}%
\pgfsetbuttcap%
\pgfsetmiterjoin%
\definecolor{currentfill}{rgb}{1.000000,1.000000,1.000000}%
\pgfsetfillcolor{currentfill}%
\pgfsetlinewidth{0.000000pt}%
\definecolor{currentstroke}{rgb}{0.000000,0.000000,0.000000}%
\pgfsetstrokecolor{currentstroke}%
\pgfsetstrokeopacity{0.000000}%
\pgfsetdash{}{0pt}%
\pgfpathmoveto{\pgfqpoint{2.082963in}{2.313483in}}%
\pgfpathlineto{\pgfqpoint{3.821481in}{2.313483in}}%
\pgfpathlineto{\pgfqpoint{3.821481in}{4.052001in}}%
\pgfpathlineto{\pgfqpoint{2.082963in}{4.052001in}}%
\pgfpathclose%
\pgfusepath{fill}%
\end{pgfscope}%
\begin{pgfscope}%
\pgfpathrectangle{\pgfqpoint{2.082963in}{2.313483in}}{\pgfqpoint{1.738519in}{1.738519in}} %
\pgfusepath{clip}%
\pgftext[at=\pgfqpoint{2.082963in}{2.313483in},left,bottom]{\pgfimage[interpolate=true,width=1.750000in,height=1.750000in]{flattops_emittance-img1.png}}%
\end{pgfscope}%
\begin{pgfscope}%
\pgfpathrectangle{\pgfqpoint{2.082963in}{2.313483in}}{\pgfqpoint{1.738519in}{1.738519in}} %
\pgfusepath{clip}%
\pgfsetbuttcap%
\pgfsetroundjoin%
\pgfsetlinewidth{1.003750pt}%
\definecolor{currentstroke}{rgb}{1.000000,1.000000,1.000000}%
\pgfsetstrokecolor{currentstroke}%
\pgfsetdash{{6.000000pt}{6.000000pt}}{0.000000pt}%
\pgfpathmoveto{\pgfqpoint{2.082963in}{3.179844in}}%
\pgfpathlineto{\pgfqpoint{3.821481in}{3.179844in}}%
\pgfusepath{stroke}%
\end{pgfscope}%
\begin{pgfscope}%
\pgfsetrectcap%
\pgfsetmiterjoin%
\pgfsetlinewidth{1.003750pt}%
\definecolor{currentstroke}{rgb}{0.000000,0.000000,0.000000}%
\pgfsetstrokecolor{currentstroke}%
\pgfsetdash{}{0pt}%
\pgfpathmoveto{\pgfqpoint{2.082963in}{4.052001in}}%
\pgfpathlineto{\pgfqpoint{3.821481in}{4.052001in}}%
\pgfusepath{stroke}%
\end{pgfscope}%
\begin{pgfscope}%
\pgfsetrectcap%
\pgfsetmiterjoin%
\pgfsetlinewidth{1.003750pt}%
\definecolor{currentstroke}{rgb}{0.000000,0.000000,0.000000}%
\pgfsetstrokecolor{currentstroke}%
\pgfsetdash{}{0pt}%
\pgfpathmoveto{\pgfqpoint{2.082963in}{2.313483in}}%
\pgfpathlineto{\pgfqpoint{2.082963in}{4.052001in}}%
\pgfusepath{stroke}%
\end{pgfscope}%
\begin{pgfscope}%
\pgfsetrectcap%
\pgfsetmiterjoin%
\pgfsetlinewidth{1.003750pt}%
\definecolor{currentstroke}{rgb}{0.000000,0.000000,0.000000}%
\pgfsetstrokecolor{currentstroke}%
\pgfsetdash{}{0pt}%
\pgfpathmoveto{\pgfqpoint{2.082963in}{2.313483in}}%
\pgfpathlineto{\pgfqpoint{3.821481in}{2.313483in}}%
\pgfusepath{stroke}%
\end{pgfscope}%
\begin{pgfscope}%
\pgfsetrectcap%
\pgfsetmiterjoin%
\pgfsetlinewidth{1.003750pt}%
\definecolor{currentstroke}{rgb}{0.000000,0.000000,0.000000}%
\pgfsetstrokecolor{currentstroke}%
\pgfsetdash{}{0pt}%
\pgfpathmoveto{\pgfqpoint{3.821481in}{2.313483in}}%
\pgfpathlineto{\pgfqpoint{3.821481in}{4.052001in}}%
\pgfusepath{stroke}%
\end{pgfscope}%
\begin{pgfscope}%
\pgftext[x=2.952222in,y=4.121446in,,base]{\rmfamily\fontsize{12.000000}{14.400000}\selectfont 5.5\(\displaystyle \,\)meV}%
\end{pgfscope}%
\begin{pgfscope}%
\pgfsetbuttcap%
\pgfsetmiterjoin%
\definecolor{currentfill}{rgb}{1.000000,1.000000,1.000000}%
\pgfsetfillcolor{currentfill}%
\pgfsetlinewidth{0.000000pt}%
\definecolor{currentstroke}{rgb}{0.000000,0.000000,0.000000}%
\pgfsetstrokecolor{currentstroke}%
\pgfsetstrokeopacity{0.000000}%
\pgfsetdash{}{0pt}%
\pgfpathmoveto{\pgfqpoint{2.082963in}{0.528906in}}%
\pgfpathlineto{\pgfqpoint{3.821481in}{0.528906in}}%
\pgfpathlineto{\pgfqpoint{3.821481in}{2.298130in}}%
\pgfpathlineto{\pgfqpoint{2.082963in}{2.298130in}}%
\pgfpathclose%
\pgfusepath{fill}%
\end{pgfscope}%
\begin{pgfscope}%
\pgfpathrectangle{\pgfqpoint{2.082963in}{0.528906in}}{\pgfqpoint{1.738519in}{1.769224in}} %
\pgfusepath{clip}%
\pgfsetrectcap%
\pgfsetroundjoin%
\pgfsetlinewidth{1.003750pt}%
\definecolor{currentstroke}{rgb}{0.960784,0.682353,0.125490}%
\pgfsetstrokecolor{currentstroke}%
\pgfsetdash{}{0pt}%
\pgfpathmoveto{\pgfqpoint{2.286973in}{0.534057in}}%
\pgfpathlineto{\pgfqpoint{2.291408in}{0.534692in}}%
\pgfpathlineto{\pgfqpoint{2.300278in}{0.532455in}}%
\pgfpathlineto{\pgfqpoint{2.304713in}{0.528906in}}%
\pgfpathlineto{\pgfqpoint{2.309148in}{0.528976in}}%
\pgfpathlineto{\pgfqpoint{2.313583in}{0.532921in}}%
\pgfpathlineto{\pgfqpoint{2.322453in}{0.550042in}}%
\pgfpathlineto{\pgfqpoint{2.335758in}{0.543593in}}%
\pgfpathlineto{\pgfqpoint{2.340193in}{0.540890in}}%
\pgfpathlineto{\pgfqpoint{2.349063in}{0.550318in}}%
\pgfpathlineto{\pgfqpoint{2.353498in}{0.553548in}}%
\pgfpathlineto{\pgfqpoint{2.362368in}{0.568835in}}%
\pgfpathlineto{\pgfqpoint{2.366803in}{0.573564in}}%
\pgfpathlineto{\pgfqpoint{2.371238in}{0.574651in}}%
\pgfpathlineto{\pgfqpoint{2.375673in}{0.578609in}}%
\pgfpathlineto{\pgfqpoint{2.380108in}{0.575794in}}%
\pgfpathlineto{\pgfqpoint{2.388978in}{0.586610in}}%
\pgfpathlineto{\pgfqpoint{2.393413in}{0.591543in}}%
\pgfpathlineto{\pgfqpoint{2.397848in}{0.590131in}}%
\pgfpathlineto{\pgfqpoint{2.406718in}{0.593083in}}%
\pgfpathlineto{\pgfqpoint{2.411153in}{0.602689in}}%
\pgfpathlineto{\pgfqpoint{2.415588in}{0.609091in}}%
\pgfpathlineto{\pgfqpoint{2.420023in}{0.621615in}}%
\pgfpathlineto{\pgfqpoint{2.428893in}{0.619043in}}%
\pgfpathlineto{\pgfqpoint{2.437763in}{0.642151in}}%
\pgfpathlineto{\pgfqpoint{2.442198in}{0.655067in}}%
\pgfpathlineto{\pgfqpoint{2.446633in}{0.663900in}}%
\pgfpathlineto{\pgfqpoint{2.451068in}{0.676019in}}%
\pgfpathlineto{\pgfqpoint{2.464373in}{0.740461in}}%
\pgfpathlineto{\pgfqpoint{2.468808in}{0.749190in}}%
\pgfpathlineto{\pgfqpoint{2.477678in}{0.770817in}}%
\pgfpathlineto{\pgfqpoint{2.482113in}{0.780587in}}%
\pgfpathlineto{\pgfqpoint{2.486548in}{0.793634in}}%
\pgfpathlineto{\pgfqpoint{2.499853in}{0.852777in}}%
\pgfpathlineto{\pgfqpoint{2.504288in}{0.868956in}}%
\pgfpathlineto{\pgfqpoint{2.508723in}{0.897694in}}%
\pgfpathlineto{\pgfqpoint{2.513158in}{0.911781in}}%
\pgfpathlineto{\pgfqpoint{2.517593in}{0.932942in}}%
\pgfpathlineto{\pgfqpoint{2.535333in}{1.050383in}}%
\pgfpathlineto{\pgfqpoint{2.539768in}{1.060044in}}%
\pgfpathlineto{\pgfqpoint{2.544203in}{1.080086in}}%
\pgfpathlineto{\pgfqpoint{2.548638in}{1.109538in}}%
\pgfpathlineto{\pgfqpoint{2.557508in}{1.198797in}}%
\pgfpathlineto{\pgfqpoint{2.561943in}{1.227295in}}%
\pgfpathlineto{\pgfqpoint{2.566378in}{1.243391in}}%
\pgfpathlineto{\pgfqpoint{2.570813in}{1.250407in}}%
\pgfpathlineto{\pgfqpoint{2.584118in}{1.341802in}}%
\pgfpathlineto{\pgfqpoint{2.601858in}{1.429280in}}%
\pgfpathlineto{\pgfqpoint{2.610728in}{1.483946in}}%
\pgfpathlineto{\pgfqpoint{2.615163in}{1.523218in}}%
\pgfpathlineto{\pgfqpoint{2.624033in}{1.570559in}}%
\pgfpathlineto{\pgfqpoint{2.628467in}{1.606933in}}%
\pgfpathlineto{\pgfqpoint{2.637337in}{1.654382in}}%
\pgfpathlineto{\pgfqpoint{2.641772in}{1.691386in}}%
\pgfpathlineto{\pgfqpoint{2.646207in}{1.708190in}}%
\pgfpathlineto{\pgfqpoint{2.650642in}{1.714558in}}%
\pgfpathlineto{\pgfqpoint{2.655077in}{1.726983in}}%
\pgfpathlineto{\pgfqpoint{2.659512in}{1.753955in}}%
\pgfpathlineto{\pgfqpoint{2.663947in}{1.802237in}}%
\pgfpathlineto{\pgfqpoint{2.668382in}{1.831664in}}%
\pgfpathlineto{\pgfqpoint{2.672817in}{1.846193in}}%
\pgfpathlineto{\pgfqpoint{2.677252in}{1.829702in}}%
\pgfpathlineto{\pgfqpoint{2.681687in}{1.822367in}}%
\pgfpathlineto{\pgfqpoint{2.686122in}{1.812374in}}%
\pgfpathlineto{\pgfqpoint{2.694992in}{1.863895in}}%
\pgfpathlineto{\pgfqpoint{2.699427in}{1.901825in}}%
\pgfpathlineto{\pgfqpoint{2.703862in}{1.921468in}}%
\pgfpathlineto{\pgfqpoint{2.708297in}{1.920022in}}%
\pgfpathlineto{\pgfqpoint{2.712732in}{1.904699in}}%
\pgfpathlineto{\pgfqpoint{2.717167in}{1.873817in}}%
\pgfpathlineto{\pgfqpoint{2.721602in}{1.883256in}}%
\pgfpathlineto{\pgfqpoint{2.726037in}{1.899668in}}%
\pgfpathlineto{\pgfqpoint{2.730472in}{1.905616in}}%
\pgfpathlineto{\pgfqpoint{2.734907in}{1.903223in}}%
\pgfpathlineto{\pgfqpoint{2.743777in}{1.891293in}}%
\pgfpathlineto{\pgfqpoint{2.748212in}{1.898833in}}%
\pgfpathlineto{\pgfqpoint{2.752647in}{1.935014in}}%
\pgfpathlineto{\pgfqpoint{2.761517in}{1.975047in}}%
\pgfpathlineto{\pgfqpoint{2.765952in}{1.973808in}}%
\pgfpathlineto{\pgfqpoint{2.770387in}{1.968368in}}%
\pgfpathlineto{\pgfqpoint{2.774822in}{1.967907in}}%
\pgfpathlineto{\pgfqpoint{2.779257in}{1.965812in}}%
\pgfpathlineto{\pgfqpoint{2.783692in}{1.982040in}}%
\pgfpathlineto{\pgfqpoint{2.788127in}{2.010575in}}%
\pgfpathlineto{\pgfqpoint{2.792562in}{2.018467in}}%
\pgfpathlineto{\pgfqpoint{2.796997in}{2.010169in}}%
\pgfpathlineto{\pgfqpoint{2.801432in}{2.010010in}}%
\pgfpathlineto{\pgfqpoint{2.805867in}{2.020287in}}%
\pgfpathlineto{\pgfqpoint{2.810302in}{2.036125in}}%
\pgfpathlineto{\pgfqpoint{2.819172in}{2.098406in}}%
\pgfpathlineto{\pgfqpoint{2.823607in}{2.100388in}}%
\pgfpathlineto{\pgfqpoint{2.832477in}{2.075593in}}%
\pgfpathlineto{\pgfqpoint{2.836912in}{2.098410in}}%
\pgfpathlineto{\pgfqpoint{2.845782in}{2.153373in}}%
\pgfpathlineto{\pgfqpoint{2.850217in}{2.158757in}}%
\pgfpathlineto{\pgfqpoint{2.854652in}{2.166994in}}%
\pgfpathlineto{\pgfqpoint{2.859087in}{2.166381in}}%
\pgfpathlineto{\pgfqpoint{2.863522in}{2.175029in}}%
\pgfpathlineto{\pgfqpoint{2.867957in}{2.156021in}}%
\pgfpathlineto{\pgfqpoint{2.872392in}{2.164828in}}%
\pgfpathlineto{\pgfqpoint{2.876827in}{2.181119in}}%
\pgfpathlineto{\pgfqpoint{2.881262in}{2.185696in}}%
\pgfpathlineto{\pgfqpoint{2.885697in}{2.168545in}}%
\pgfpathlineto{\pgfqpoint{2.890132in}{2.159581in}}%
\pgfpathlineto{\pgfqpoint{2.894567in}{2.145660in}}%
\pgfpathlineto{\pgfqpoint{2.903437in}{2.182712in}}%
\pgfpathlineto{\pgfqpoint{2.907872in}{2.179755in}}%
\pgfpathlineto{\pgfqpoint{2.912307in}{2.154531in}}%
\pgfpathlineto{\pgfqpoint{2.916742in}{2.147595in}}%
\pgfpathlineto{\pgfqpoint{2.921177in}{2.156767in}}%
\pgfpathlineto{\pgfqpoint{2.925612in}{2.171223in}}%
\pgfpathlineto{\pgfqpoint{2.930047in}{2.203983in}}%
\pgfpathlineto{\pgfqpoint{2.934482in}{2.213881in}}%
\pgfpathlineto{\pgfqpoint{2.943352in}{2.173691in}}%
\pgfpathlineto{\pgfqpoint{2.947787in}{2.164649in}}%
\pgfpathlineto{\pgfqpoint{2.952222in}{2.184486in}}%
\pgfpathlineto{\pgfqpoint{2.961092in}{2.202197in}}%
\pgfpathlineto{\pgfqpoint{2.965527in}{2.195284in}}%
\pgfpathlineto{\pgfqpoint{2.969962in}{2.185666in}}%
\pgfpathlineto{\pgfqpoint{2.974397in}{2.178864in}}%
\pgfpathlineto{\pgfqpoint{2.983267in}{2.195860in}}%
\pgfpathlineto{\pgfqpoint{2.987702in}{2.195185in}}%
\pgfpathlineto{\pgfqpoint{2.992137in}{2.206092in}}%
\pgfpathlineto{\pgfqpoint{3.001007in}{2.147543in}}%
\pgfpathlineto{\pgfqpoint{3.005442in}{2.136505in}}%
\pgfpathlineto{\pgfqpoint{3.009877in}{2.138243in}}%
\pgfpathlineto{\pgfqpoint{3.014312in}{2.143640in}}%
\pgfpathlineto{\pgfqpoint{3.018747in}{2.145336in}}%
\pgfpathlineto{\pgfqpoint{3.023182in}{2.148415in}}%
\pgfpathlineto{\pgfqpoint{3.027617in}{2.140131in}}%
\pgfpathlineto{\pgfqpoint{3.032052in}{2.122359in}}%
\pgfpathlineto{\pgfqpoint{3.036487in}{2.126552in}}%
\pgfpathlineto{\pgfqpoint{3.040922in}{2.136288in}}%
\pgfpathlineto{\pgfqpoint{3.045357in}{2.153887in}}%
\pgfpathlineto{\pgfqpoint{3.049792in}{2.142971in}}%
\pgfpathlineto{\pgfqpoint{3.054227in}{2.150369in}}%
\pgfpathlineto{\pgfqpoint{3.063097in}{2.137084in}}%
\pgfpathlineto{\pgfqpoint{3.067532in}{2.123421in}}%
\pgfpathlineto{\pgfqpoint{3.071967in}{2.134016in}}%
\pgfpathlineto{\pgfqpoint{3.076402in}{2.140109in}}%
\pgfpathlineto{\pgfqpoint{3.080837in}{2.126678in}}%
\pgfpathlineto{\pgfqpoint{3.089707in}{2.062790in}}%
\pgfpathlineto{\pgfqpoint{3.094142in}{2.062457in}}%
\pgfpathlineto{\pgfqpoint{3.098577in}{2.057793in}}%
\pgfpathlineto{\pgfqpoint{3.111882in}{2.021629in}}%
\pgfpathlineto{\pgfqpoint{3.120752in}{1.975192in}}%
\pgfpathlineto{\pgfqpoint{3.125187in}{1.971218in}}%
\pgfpathlineto{\pgfqpoint{3.129622in}{1.964588in}}%
\pgfpathlineto{\pgfqpoint{3.138492in}{1.970169in}}%
\pgfpathlineto{\pgfqpoint{3.142927in}{1.955658in}}%
\pgfpathlineto{\pgfqpoint{3.147362in}{1.929941in}}%
\pgfpathlineto{\pgfqpoint{3.151797in}{1.929181in}}%
\pgfpathlineto{\pgfqpoint{3.156232in}{1.945927in}}%
\pgfpathlineto{\pgfqpoint{3.165102in}{1.900745in}}%
\pgfpathlineto{\pgfqpoint{3.178407in}{1.781858in}}%
\pgfpathlineto{\pgfqpoint{3.182842in}{1.761569in}}%
\pgfpathlineto{\pgfqpoint{3.187277in}{1.764738in}}%
\pgfpathlineto{\pgfqpoint{3.196147in}{1.744889in}}%
\pgfpathlineto{\pgfqpoint{3.209452in}{1.637360in}}%
\pgfpathlineto{\pgfqpoint{3.213887in}{1.610750in}}%
\pgfpathlineto{\pgfqpoint{3.218322in}{1.599364in}}%
\pgfpathlineto{\pgfqpoint{3.227192in}{1.550639in}}%
\pgfpathlineto{\pgfqpoint{3.231627in}{1.525057in}}%
\pgfpathlineto{\pgfqpoint{3.236062in}{1.486272in}}%
\pgfpathlineto{\pgfqpoint{3.240497in}{1.464622in}}%
\pgfpathlineto{\pgfqpoint{3.244932in}{1.457572in}}%
\pgfpathlineto{\pgfqpoint{3.249367in}{1.431932in}}%
\pgfpathlineto{\pgfqpoint{3.253802in}{1.413235in}}%
\pgfpathlineto{\pgfqpoint{3.275977in}{1.284999in}}%
\pgfpathlineto{\pgfqpoint{3.293717in}{1.142090in}}%
\pgfpathlineto{\pgfqpoint{3.307022in}{1.069406in}}%
\pgfpathlineto{\pgfqpoint{3.315892in}{0.998173in}}%
\pgfpathlineto{\pgfqpoint{3.320327in}{0.973555in}}%
\pgfpathlineto{\pgfqpoint{3.333632in}{0.920118in}}%
\pgfpathlineto{\pgfqpoint{3.342502in}{0.872466in}}%
\pgfpathlineto{\pgfqpoint{3.351372in}{0.844549in}}%
\pgfpathlineto{\pgfqpoint{3.360242in}{0.828771in}}%
\pgfpathlineto{\pgfqpoint{3.364677in}{0.815777in}}%
\pgfpathlineto{\pgfqpoint{3.373547in}{0.781131in}}%
\pgfpathlineto{\pgfqpoint{3.377982in}{0.757236in}}%
\pgfpathlineto{\pgfqpoint{3.382417in}{0.746027in}}%
\pgfpathlineto{\pgfqpoint{3.386852in}{0.728327in}}%
\pgfpathlineto{\pgfqpoint{3.391287in}{0.719283in}}%
\pgfpathlineto{\pgfqpoint{3.395722in}{0.716059in}}%
\pgfpathlineto{\pgfqpoint{3.404592in}{0.693921in}}%
\pgfpathlineto{\pgfqpoint{3.409027in}{0.689179in}}%
\pgfpathlineto{\pgfqpoint{3.413462in}{0.686536in}}%
\pgfpathlineto{\pgfqpoint{3.417897in}{0.675850in}}%
\pgfpathlineto{\pgfqpoint{3.422332in}{0.671106in}}%
\pgfpathlineto{\pgfqpoint{3.435637in}{0.648476in}}%
\pgfpathlineto{\pgfqpoint{3.440072in}{0.647367in}}%
\pgfpathlineto{\pgfqpoint{3.444507in}{0.642828in}}%
\pgfpathlineto{\pgfqpoint{3.448942in}{0.644009in}}%
\pgfpathlineto{\pgfqpoint{3.453377in}{0.638451in}}%
\pgfpathlineto{\pgfqpoint{3.457812in}{0.630255in}}%
\pgfpathlineto{\pgfqpoint{3.466682in}{0.621059in}}%
\pgfpathlineto{\pgfqpoint{3.471117in}{0.620891in}}%
\pgfpathlineto{\pgfqpoint{3.475552in}{0.622342in}}%
\pgfpathlineto{\pgfqpoint{3.479987in}{0.620827in}}%
\pgfpathlineto{\pgfqpoint{3.484422in}{0.630446in}}%
\pgfpathlineto{\pgfqpoint{3.488857in}{0.626744in}}%
\pgfpathlineto{\pgfqpoint{3.493292in}{0.619239in}}%
\pgfpathlineto{\pgfqpoint{3.497727in}{0.616070in}}%
\pgfpathlineto{\pgfqpoint{3.502162in}{0.610292in}}%
\pgfpathlineto{\pgfqpoint{3.506597in}{0.599694in}}%
\pgfpathlineto{\pgfqpoint{3.515467in}{0.595925in}}%
\pgfpathlineto{\pgfqpoint{3.519902in}{0.598350in}}%
\pgfpathlineto{\pgfqpoint{3.524337in}{0.588800in}}%
\pgfpathlineto{\pgfqpoint{3.528772in}{0.586742in}}%
\pgfpathlineto{\pgfqpoint{3.533207in}{0.586837in}}%
\pgfpathlineto{\pgfqpoint{3.537642in}{0.585638in}}%
\pgfpathlineto{\pgfqpoint{3.542077in}{0.578375in}}%
\pgfpathlineto{\pgfqpoint{3.546512in}{0.588633in}}%
\pgfpathlineto{\pgfqpoint{3.550947in}{0.590686in}}%
\pgfpathlineto{\pgfqpoint{3.555382in}{0.588949in}}%
\pgfpathlineto{\pgfqpoint{3.559817in}{0.590576in}}%
\pgfpathlineto{\pgfqpoint{3.564252in}{0.584407in}}%
\pgfpathlineto{\pgfqpoint{3.573122in}{0.578005in}}%
\pgfpathlineto{\pgfqpoint{3.577557in}{0.572941in}}%
\pgfpathlineto{\pgfqpoint{3.586427in}{0.580524in}}%
\pgfpathlineto{\pgfqpoint{3.590862in}{0.580667in}}%
\pgfpathlineto{\pgfqpoint{3.595297in}{0.584520in}}%
\pgfpathlineto{\pgfqpoint{3.599732in}{0.582132in}}%
\pgfpathlineto{\pgfqpoint{3.604167in}{0.581760in}}%
\pgfpathlineto{\pgfqpoint{3.608602in}{0.587348in}}%
\pgfpathlineto{\pgfqpoint{3.613037in}{0.587597in}}%
\pgfpathlineto{\pgfqpoint{3.613037in}{0.587597in}}%
\pgfusepath{stroke}%
\end{pgfscope}%
\begin{pgfscope}%
\pgfsetrectcap%
\pgfsetmiterjoin%
\pgfsetlinewidth{1.003750pt}%
\definecolor{currentstroke}{rgb}{0.000000,0.000000,0.000000}%
\pgfsetstrokecolor{currentstroke}%
\pgfsetdash{}{0pt}%
\pgfpathmoveto{\pgfqpoint{2.082963in}{2.298130in}}%
\pgfpathlineto{\pgfqpoint{3.821481in}{2.298130in}}%
\pgfusepath{stroke}%
\end{pgfscope}%
\begin{pgfscope}%
\pgfsetrectcap%
\pgfsetmiterjoin%
\pgfsetlinewidth{1.003750pt}%
\definecolor{currentstroke}{rgb}{0.000000,0.000000,0.000000}%
\pgfsetstrokecolor{currentstroke}%
\pgfsetdash{}{0pt}%
\pgfpathmoveto{\pgfqpoint{2.082963in}{0.528906in}}%
\pgfpathlineto{\pgfqpoint{2.082963in}{2.298130in}}%
\pgfusepath{stroke}%
\end{pgfscope}%
\begin{pgfscope}%
\pgfsetrectcap%
\pgfsetmiterjoin%
\pgfsetlinewidth{1.003750pt}%
\definecolor{currentstroke}{rgb}{0.000000,0.000000,0.000000}%
\pgfsetstrokecolor{currentstroke}%
\pgfsetdash{}{0pt}%
\pgfpathmoveto{\pgfqpoint{2.082963in}{0.528906in}}%
\pgfpathlineto{\pgfqpoint{3.821481in}{0.528906in}}%
\pgfusepath{stroke}%
\end{pgfscope}%
\begin{pgfscope}%
\pgfsetrectcap%
\pgfsetmiterjoin%
\pgfsetlinewidth{1.003750pt}%
\definecolor{currentstroke}{rgb}{0.000000,0.000000,0.000000}%
\pgfsetstrokecolor{currentstroke}%
\pgfsetdash{}{0pt}%
\pgfpathmoveto{\pgfqpoint{3.821481in}{0.528906in}}%
\pgfpathlineto{\pgfqpoint{3.821481in}{2.298130in}}%
\pgfusepath{stroke}%
\end{pgfscope}%
\begin{pgfscope}%
\pgfsetbuttcap%
\pgfsetroundjoin%
\definecolor{currentfill}{rgb}{0.000000,0.000000,0.000000}%
\pgfsetfillcolor{currentfill}%
\pgfsetlinewidth{0.501875pt}%
\definecolor{currentstroke}{rgb}{0.000000,0.000000,0.000000}%
\pgfsetstrokecolor{currentstroke}%
\pgfsetdash{}{0pt}%
\pgfsys@defobject{currentmarker}{\pgfqpoint{0.000000in}{0.000000in}}{\pgfqpoint{0.000000in}{0.055556in}}{%
\pgfpathmoveto{\pgfqpoint{0.000000in}{0.000000in}}%
\pgfpathlineto{\pgfqpoint{0.000000in}{0.055556in}}%
\pgfusepath{stroke,fill}%
}%
\begin{pgfscope}%
\pgfsys@transformshift{2.408935in}{0.528906in}%
\pgfsys@useobject{currentmarker}{}%
\end{pgfscope}%
\end{pgfscope}%
\begin{pgfscope}%
\pgfsetbuttcap%
\pgfsetroundjoin%
\definecolor{currentfill}{rgb}{0.000000,0.000000,0.000000}%
\pgfsetfillcolor{currentfill}%
\pgfsetlinewidth{0.501875pt}%
\definecolor{currentstroke}{rgb}{0.000000,0.000000,0.000000}%
\pgfsetstrokecolor{currentstroke}%
\pgfsetdash{}{0pt}%
\pgfsys@defobject{currentmarker}{\pgfqpoint{0.000000in}{-0.055556in}}{\pgfqpoint{0.000000in}{0.000000in}}{%
\pgfpathmoveto{\pgfqpoint{0.000000in}{0.000000in}}%
\pgfpathlineto{\pgfqpoint{0.000000in}{-0.055556in}}%
\pgfusepath{stroke,fill}%
}%
\begin{pgfscope}%
\pgfsys@transformshift{2.408935in}{2.298130in}%
\pgfsys@useobject{currentmarker}{}%
\end{pgfscope}%
\end{pgfscope}%
\begin{pgfscope}%
\pgftext[x=2.408935in,y=0.473351in,,top]{\rmfamily\fontsize{10.000000}{12.000000}\selectfont -5}%
\end{pgfscope}%
\begin{pgfscope}%
\pgfsetbuttcap%
\pgfsetroundjoin%
\definecolor{currentfill}{rgb}{0.000000,0.000000,0.000000}%
\pgfsetfillcolor{currentfill}%
\pgfsetlinewidth{0.501875pt}%
\definecolor{currentstroke}{rgb}{0.000000,0.000000,0.000000}%
\pgfsetstrokecolor{currentstroke}%
\pgfsetdash{}{0pt}%
\pgfsys@defobject{currentmarker}{\pgfqpoint{0.000000in}{0.000000in}}{\pgfqpoint{0.000000in}{0.055556in}}{%
\pgfpathmoveto{\pgfqpoint{0.000000in}{0.000000in}}%
\pgfpathlineto{\pgfqpoint{0.000000in}{0.055556in}}%
\pgfusepath{stroke,fill}%
}%
\begin{pgfscope}%
\pgfsys@transformshift{2.952222in}{0.528906in}%
\pgfsys@useobject{currentmarker}{}%
\end{pgfscope}%
\end{pgfscope}%
\begin{pgfscope}%
\pgfsetbuttcap%
\pgfsetroundjoin%
\definecolor{currentfill}{rgb}{0.000000,0.000000,0.000000}%
\pgfsetfillcolor{currentfill}%
\pgfsetlinewidth{0.501875pt}%
\definecolor{currentstroke}{rgb}{0.000000,0.000000,0.000000}%
\pgfsetstrokecolor{currentstroke}%
\pgfsetdash{}{0pt}%
\pgfsys@defobject{currentmarker}{\pgfqpoint{0.000000in}{-0.055556in}}{\pgfqpoint{0.000000in}{0.000000in}}{%
\pgfpathmoveto{\pgfqpoint{0.000000in}{0.000000in}}%
\pgfpathlineto{\pgfqpoint{0.000000in}{-0.055556in}}%
\pgfusepath{stroke,fill}%
}%
\begin{pgfscope}%
\pgfsys@transformshift{2.952222in}{2.298130in}%
\pgfsys@useobject{currentmarker}{}%
\end{pgfscope}%
\end{pgfscope}%
\begin{pgfscope}%
\pgftext[x=2.952222in,y=0.473351in,,top]{\rmfamily\fontsize{10.000000}{12.000000}\selectfont 0}%
\end{pgfscope}%
\begin{pgfscope}%
\pgfsetbuttcap%
\pgfsetroundjoin%
\definecolor{currentfill}{rgb}{0.000000,0.000000,0.000000}%
\pgfsetfillcolor{currentfill}%
\pgfsetlinewidth{0.501875pt}%
\definecolor{currentstroke}{rgb}{0.000000,0.000000,0.000000}%
\pgfsetstrokecolor{currentstroke}%
\pgfsetdash{}{0pt}%
\pgfsys@defobject{currentmarker}{\pgfqpoint{0.000000in}{0.000000in}}{\pgfqpoint{0.000000in}{0.055556in}}{%
\pgfpathmoveto{\pgfqpoint{0.000000in}{0.000000in}}%
\pgfpathlineto{\pgfqpoint{0.000000in}{0.055556in}}%
\pgfusepath{stroke,fill}%
}%
\begin{pgfscope}%
\pgfsys@transformshift{3.495509in}{0.528906in}%
\pgfsys@useobject{currentmarker}{}%
\end{pgfscope}%
\end{pgfscope}%
\begin{pgfscope}%
\pgfsetbuttcap%
\pgfsetroundjoin%
\definecolor{currentfill}{rgb}{0.000000,0.000000,0.000000}%
\pgfsetfillcolor{currentfill}%
\pgfsetlinewidth{0.501875pt}%
\definecolor{currentstroke}{rgb}{0.000000,0.000000,0.000000}%
\pgfsetstrokecolor{currentstroke}%
\pgfsetdash{}{0pt}%
\pgfsys@defobject{currentmarker}{\pgfqpoint{0.000000in}{-0.055556in}}{\pgfqpoint{0.000000in}{0.000000in}}{%
\pgfpathmoveto{\pgfqpoint{0.000000in}{0.000000in}}%
\pgfpathlineto{\pgfqpoint{0.000000in}{-0.055556in}}%
\pgfusepath{stroke,fill}%
}%
\begin{pgfscope}%
\pgfsys@transformshift{3.495509in}{2.298130in}%
\pgfsys@useobject{currentmarker}{}%
\end{pgfscope}%
\end{pgfscope}%
\begin{pgfscope}%
\pgftext[x=3.495509in,y=0.473351in,,top]{\rmfamily\fontsize{10.000000}{12.000000}\selectfont 5}%
\end{pgfscope}%
\begin{pgfscope}%
\pgftext[x=2.952222in,y=0.280450in,,top]{\rmfamily\fontsize{10.000000}{12.000000}\selectfont Position (mm)}%
\end{pgfscope}%
\begin{pgfscope}%
\pgfsetbuttcap%
\pgfsetmiterjoin%
\definecolor{currentfill}{rgb}{1.000000,1.000000,1.000000}%
\pgfsetfillcolor{currentfill}%
\pgfsetlinewidth{0.000000pt}%
\definecolor{currentstroke}{rgb}{0.000000,0.000000,0.000000}%
\pgfsetstrokecolor{currentstroke}%
\pgfsetstrokeopacity{0.000000}%
\pgfsetdash{}{0pt}%
\pgfpathmoveto{\pgfqpoint{3.821481in}{2.313483in}}%
\pgfpathlineto{\pgfqpoint{5.560000in}{2.313483in}}%
\pgfpathlineto{\pgfqpoint{5.560000in}{4.052001in}}%
\pgfpathlineto{\pgfqpoint{3.821481in}{4.052001in}}%
\pgfpathclose%
\pgfusepath{fill}%
\end{pgfscope}%
\begin{pgfscope}%
\pgfpathrectangle{\pgfqpoint{3.821481in}{2.313483in}}{\pgfqpoint{1.738519in}{1.738519in}} %
\pgfusepath{clip}%
\pgftext[at=\pgfqpoint{3.821481in}{2.313483in},left,bottom]{\pgfimage[interpolate=true,width=1.750000in,height=1.750000in]{flattops_emittance-img2.png}}%
\end{pgfscope}%
\begin{pgfscope}%
\pgfpathrectangle{\pgfqpoint{3.821481in}{2.313483in}}{\pgfqpoint{1.738519in}{1.738519in}} %
\pgfusepath{clip}%
\pgfsetbuttcap%
\pgfsetroundjoin%
\pgfsetlinewidth{1.003750pt}%
\definecolor{currentstroke}{rgb}{1.000000,1.000000,1.000000}%
\pgfsetstrokecolor{currentstroke}%
\pgfsetdash{{6.000000pt}{6.000000pt}}{0.000000pt}%
\pgfpathmoveto{\pgfqpoint{3.821481in}{3.179844in}}%
\pgfpathlineto{\pgfqpoint{5.560000in}{3.179844in}}%
\pgfusepath{stroke}%
\end{pgfscope}%
\begin{pgfscope}%
\pgfsetrectcap%
\pgfsetmiterjoin%
\pgfsetlinewidth{1.003750pt}%
\definecolor{currentstroke}{rgb}{0.000000,0.000000,0.000000}%
\pgfsetstrokecolor{currentstroke}%
\pgfsetdash{}{0pt}%
\pgfpathmoveto{\pgfqpoint{3.821481in}{4.052001in}}%
\pgfpathlineto{\pgfqpoint{5.560000in}{4.052001in}}%
\pgfusepath{stroke}%
\end{pgfscope}%
\begin{pgfscope}%
\pgfsetrectcap%
\pgfsetmiterjoin%
\pgfsetlinewidth{1.003750pt}%
\definecolor{currentstroke}{rgb}{0.000000,0.000000,0.000000}%
\pgfsetstrokecolor{currentstroke}%
\pgfsetdash{}{0pt}%
\pgfpathmoveto{\pgfqpoint{3.821481in}{2.313483in}}%
\pgfpathlineto{\pgfqpoint{3.821481in}{4.052001in}}%
\pgfusepath{stroke}%
\end{pgfscope}%
\begin{pgfscope}%
\pgfsetrectcap%
\pgfsetmiterjoin%
\pgfsetlinewidth{1.003750pt}%
\definecolor{currentstroke}{rgb}{0.000000,0.000000,0.000000}%
\pgfsetstrokecolor{currentstroke}%
\pgfsetdash{}{0pt}%
\pgfpathmoveto{\pgfqpoint{3.821481in}{2.313483in}}%
\pgfpathlineto{\pgfqpoint{5.560000in}{2.313483in}}%
\pgfusepath{stroke}%
\end{pgfscope}%
\begin{pgfscope}%
\pgfsetrectcap%
\pgfsetmiterjoin%
\pgfsetlinewidth{1.003750pt}%
\definecolor{currentstroke}{rgb}{0.000000,0.000000,0.000000}%
\pgfsetstrokecolor{currentstroke}%
\pgfsetdash{}{0pt}%
\pgfpathmoveto{\pgfqpoint{5.560000in}{2.313483in}}%
\pgfpathlineto{\pgfqpoint{5.560000in}{4.052001in}}%
\pgfusepath{stroke}%
\end{pgfscope}%
\begin{pgfscope}%
\pgftext[x=4.690741in,y=4.121446in,,base]{\rmfamily\fontsize{12.000000}{14.400000}\selectfont 22.5\(\displaystyle \,\)meV}%
\end{pgfscope}%
\begin{pgfscope}%
\pgfsetbuttcap%
\pgfsetmiterjoin%
\definecolor{currentfill}{rgb}{1.000000,1.000000,1.000000}%
\pgfsetfillcolor{currentfill}%
\pgfsetlinewidth{0.000000pt}%
\definecolor{currentstroke}{rgb}{0.000000,0.000000,0.000000}%
\pgfsetstrokecolor{currentstroke}%
\pgfsetstrokeopacity{0.000000}%
\pgfsetdash{}{0pt}%
\pgfpathmoveto{\pgfqpoint{3.821481in}{0.528906in}}%
\pgfpathlineto{\pgfqpoint{5.560000in}{0.528906in}}%
\pgfpathlineto{\pgfqpoint{5.560000in}{2.298130in}}%
\pgfpathlineto{\pgfqpoint{3.821481in}{2.298130in}}%
\pgfpathclose%
\pgfusepath{fill}%
\end{pgfscope}%
\begin{pgfscope}%
\pgfpathrectangle{\pgfqpoint{3.821481in}{0.528906in}}{\pgfqpoint{1.738519in}{1.769224in}} %
\pgfusepath{clip}%
\pgfsetrectcap%
\pgfsetroundjoin%
\pgfsetlinewidth{1.003750pt}%
\definecolor{currentstroke}{rgb}{1.000000,0.400000,0.200000}%
\pgfsetstrokecolor{currentstroke}%
\pgfsetdash{}{0pt}%
\pgfpathmoveto{\pgfqpoint{4.025491in}{0.540199in}}%
\pgfpathlineto{\pgfqpoint{4.029926in}{0.531500in}}%
\pgfpathlineto{\pgfqpoint{4.034361in}{0.528906in}}%
\pgfpathlineto{\pgfqpoint{4.038796in}{0.529892in}}%
\pgfpathlineto{\pgfqpoint{4.043231in}{0.532960in}}%
\pgfpathlineto{\pgfqpoint{4.056536in}{0.550484in}}%
\pgfpathlineto{\pgfqpoint{4.065406in}{0.548766in}}%
\pgfpathlineto{\pgfqpoint{4.069841in}{0.546560in}}%
\pgfpathlineto{\pgfqpoint{4.074276in}{0.550413in}}%
\pgfpathlineto{\pgfqpoint{4.078711in}{0.560245in}}%
\pgfpathlineto{\pgfqpoint{4.087581in}{0.574532in}}%
\pgfpathlineto{\pgfqpoint{4.092016in}{0.586686in}}%
\pgfpathlineto{\pgfqpoint{4.100886in}{0.599207in}}%
\pgfpathlineto{\pgfqpoint{4.105321in}{0.600013in}}%
\pgfpathlineto{\pgfqpoint{4.109756in}{0.606309in}}%
\pgfpathlineto{\pgfqpoint{4.114191in}{0.608686in}}%
\pgfpathlineto{\pgfqpoint{4.118626in}{0.620709in}}%
\pgfpathlineto{\pgfqpoint{4.123061in}{0.628661in}}%
\pgfpathlineto{\pgfqpoint{4.127496in}{0.631280in}}%
\pgfpathlineto{\pgfqpoint{4.136366in}{0.659977in}}%
\pgfpathlineto{\pgfqpoint{4.145236in}{0.693694in}}%
\pgfpathlineto{\pgfqpoint{4.149671in}{0.707058in}}%
\pgfpathlineto{\pgfqpoint{4.158541in}{0.716734in}}%
\pgfpathlineto{\pgfqpoint{4.162976in}{0.727957in}}%
\pgfpathlineto{\pgfqpoint{4.171846in}{0.762036in}}%
\pgfpathlineto{\pgfqpoint{4.176281in}{0.782874in}}%
\pgfpathlineto{\pgfqpoint{4.185151in}{0.803418in}}%
\pgfpathlineto{\pgfqpoint{4.189586in}{0.806140in}}%
\pgfpathlineto{\pgfqpoint{4.207326in}{0.895275in}}%
\pgfpathlineto{\pgfqpoint{4.220631in}{0.948460in}}%
\pgfpathlineto{\pgfqpoint{4.233936in}{1.004104in}}%
\pgfpathlineto{\pgfqpoint{4.238371in}{1.014395in}}%
\pgfpathlineto{\pgfqpoint{4.242806in}{1.030600in}}%
\pgfpathlineto{\pgfqpoint{4.247241in}{1.038382in}}%
\pgfpathlineto{\pgfqpoint{4.260546in}{1.100259in}}%
\pgfpathlineto{\pgfqpoint{4.264981in}{1.117762in}}%
\pgfpathlineto{\pgfqpoint{4.269416in}{1.145298in}}%
\pgfpathlineto{\pgfqpoint{4.273851in}{1.157933in}}%
\pgfpathlineto{\pgfqpoint{4.278286in}{1.177606in}}%
\pgfpathlineto{\pgfqpoint{4.282721in}{1.208050in}}%
\pgfpathlineto{\pgfqpoint{4.291591in}{1.232232in}}%
\pgfpathlineto{\pgfqpoint{4.296026in}{1.244963in}}%
\pgfpathlineto{\pgfqpoint{4.304896in}{1.307373in}}%
\pgfpathlineto{\pgfqpoint{4.318201in}{1.377959in}}%
\pgfpathlineto{\pgfqpoint{4.322636in}{1.378581in}}%
\pgfpathlineto{\pgfqpoint{4.327071in}{1.387238in}}%
\pgfpathlineto{\pgfqpoint{4.331506in}{1.405363in}}%
\pgfpathlineto{\pgfqpoint{4.335941in}{1.434850in}}%
\pgfpathlineto{\pgfqpoint{4.340376in}{1.456115in}}%
\pgfpathlineto{\pgfqpoint{4.344811in}{1.491612in}}%
\pgfpathlineto{\pgfqpoint{4.349246in}{1.517640in}}%
\pgfpathlineto{\pgfqpoint{4.353681in}{1.526033in}}%
\pgfpathlineto{\pgfqpoint{4.358116in}{1.528892in}}%
\pgfpathlineto{\pgfqpoint{4.366986in}{1.554219in}}%
\pgfpathlineto{\pgfqpoint{4.371421in}{1.581098in}}%
\pgfpathlineto{\pgfqpoint{4.375856in}{1.588009in}}%
\pgfpathlineto{\pgfqpoint{4.384726in}{1.580357in}}%
\pgfpathlineto{\pgfqpoint{4.393596in}{1.600690in}}%
\pgfpathlineto{\pgfqpoint{4.406901in}{1.710812in}}%
\pgfpathlineto{\pgfqpoint{4.415771in}{1.729206in}}%
\pgfpathlineto{\pgfqpoint{4.420206in}{1.748864in}}%
\pgfpathlineto{\pgfqpoint{4.424641in}{1.789402in}}%
\pgfpathlineto{\pgfqpoint{4.429076in}{1.814198in}}%
\pgfpathlineto{\pgfqpoint{4.442381in}{1.824796in}}%
\pgfpathlineto{\pgfqpoint{4.446816in}{1.815787in}}%
\pgfpathlineto{\pgfqpoint{4.451251in}{1.822924in}}%
\pgfpathlineto{\pgfqpoint{4.455686in}{1.838082in}}%
\pgfpathlineto{\pgfqpoint{4.460121in}{1.873518in}}%
\pgfpathlineto{\pgfqpoint{4.464556in}{1.869763in}}%
\pgfpathlineto{\pgfqpoint{4.468991in}{1.868388in}}%
\pgfpathlineto{\pgfqpoint{4.473426in}{1.845558in}}%
\pgfpathlineto{\pgfqpoint{4.477861in}{1.831547in}}%
\pgfpathlineto{\pgfqpoint{4.482296in}{1.854880in}}%
\pgfpathlineto{\pgfqpoint{4.495601in}{1.901859in}}%
\pgfpathlineto{\pgfqpoint{4.500036in}{1.892802in}}%
\pgfpathlineto{\pgfqpoint{4.504471in}{1.868983in}}%
\pgfpathlineto{\pgfqpoint{4.508906in}{1.867743in}}%
\pgfpathlineto{\pgfqpoint{4.513341in}{1.899462in}}%
\pgfpathlineto{\pgfqpoint{4.522211in}{1.942281in}}%
\pgfpathlineto{\pgfqpoint{4.526646in}{1.940600in}}%
\pgfpathlineto{\pgfqpoint{4.531081in}{1.937692in}}%
\pgfpathlineto{\pgfqpoint{4.535516in}{1.941566in}}%
\pgfpathlineto{\pgfqpoint{4.539951in}{1.947946in}}%
\pgfpathlineto{\pgfqpoint{4.548821in}{2.002372in}}%
\pgfpathlineto{\pgfqpoint{4.553256in}{1.996682in}}%
\pgfpathlineto{\pgfqpoint{4.557691in}{1.994949in}}%
\pgfpathlineto{\pgfqpoint{4.562126in}{1.990341in}}%
\pgfpathlineto{\pgfqpoint{4.566561in}{1.993589in}}%
\pgfpathlineto{\pgfqpoint{4.575431in}{2.077795in}}%
\pgfpathlineto{\pgfqpoint{4.579866in}{2.098728in}}%
\pgfpathlineto{\pgfqpoint{4.584301in}{2.106126in}}%
\pgfpathlineto{\pgfqpoint{4.588736in}{2.078238in}}%
\pgfpathlineto{\pgfqpoint{4.593171in}{2.090976in}}%
\pgfpathlineto{\pgfqpoint{4.597606in}{2.114867in}}%
\pgfpathlineto{\pgfqpoint{4.602041in}{2.156353in}}%
\pgfpathlineto{\pgfqpoint{4.606476in}{2.170751in}}%
\pgfpathlineto{\pgfqpoint{4.610911in}{2.156914in}}%
\pgfpathlineto{\pgfqpoint{4.615346in}{2.149992in}}%
\pgfpathlineto{\pgfqpoint{4.624216in}{2.179455in}}%
\pgfpathlineto{\pgfqpoint{4.628651in}{2.191479in}}%
\pgfpathlineto{\pgfqpoint{4.633086in}{2.208370in}}%
\pgfpathlineto{\pgfqpoint{4.641956in}{2.180785in}}%
\pgfpathlineto{\pgfqpoint{4.646391in}{2.148456in}}%
\pgfpathlineto{\pgfqpoint{4.650826in}{2.164187in}}%
\pgfpathlineto{\pgfqpoint{4.655261in}{2.173953in}}%
\pgfpathlineto{\pgfqpoint{4.659696in}{2.199446in}}%
\pgfpathlineto{\pgfqpoint{4.664131in}{2.213881in}}%
\pgfpathlineto{\pgfqpoint{4.668566in}{2.199755in}}%
\pgfpathlineto{\pgfqpoint{4.677436in}{2.145239in}}%
\pgfpathlineto{\pgfqpoint{4.681871in}{2.150391in}}%
\pgfpathlineto{\pgfqpoint{4.690741in}{2.195250in}}%
\pgfpathlineto{\pgfqpoint{4.699611in}{2.176615in}}%
\pgfpathlineto{\pgfqpoint{4.708481in}{2.131324in}}%
\pgfpathlineto{\pgfqpoint{4.712916in}{2.155225in}}%
\pgfpathlineto{\pgfqpoint{4.717351in}{2.192494in}}%
\pgfpathlineto{\pgfqpoint{4.721786in}{2.192214in}}%
\pgfpathlineto{\pgfqpoint{4.726221in}{2.157750in}}%
\pgfpathlineto{\pgfqpoint{4.735091in}{2.119182in}}%
\pgfpathlineto{\pgfqpoint{4.739526in}{2.114140in}}%
\pgfpathlineto{\pgfqpoint{4.743961in}{2.134099in}}%
\pgfpathlineto{\pgfqpoint{4.748396in}{2.125078in}}%
\pgfpathlineto{\pgfqpoint{4.752831in}{2.121729in}}%
\pgfpathlineto{\pgfqpoint{4.761701in}{2.074660in}}%
\pgfpathlineto{\pgfqpoint{4.766136in}{2.070710in}}%
\pgfpathlineto{\pgfqpoint{4.770571in}{2.074813in}}%
\pgfpathlineto{\pgfqpoint{4.775006in}{2.089977in}}%
\pgfpathlineto{\pgfqpoint{4.783876in}{2.069479in}}%
\pgfpathlineto{\pgfqpoint{4.792746in}{2.033217in}}%
\pgfpathlineto{\pgfqpoint{4.797181in}{2.043618in}}%
\pgfpathlineto{\pgfqpoint{4.801616in}{2.062075in}}%
\pgfpathlineto{\pgfqpoint{4.806051in}{2.072468in}}%
\pgfpathlineto{\pgfqpoint{4.814921in}{2.056966in}}%
\pgfpathlineto{\pgfqpoint{4.819356in}{2.030712in}}%
\pgfpathlineto{\pgfqpoint{4.823791in}{2.017500in}}%
\pgfpathlineto{\pgfqpoint{4.828226in}{2.018874in}}%
\pgfpathlineto{\pgfqpoint{4.832661in}{2.045560in}}%
\pgfpathlineto{\pgfqpoint{4.837096in}{2.054189in}}%
\pgfpathlineto{\pgfqpoint{4.841531in}{2.037310in}}%
\pgfpathlineto{\pgfqpoint{4.845966in}{1.996258in}}%
\pgfpathlineto{\pgfqpoint{4.850401in}{1.966418in}}%
\pgfpathlineto{\pgfqpoint{4.854836in}{1.961888in}}%
\pgfpathlineto{\pgfqpoint{4.859271in}{1.967754in}}%
\pgfpathlineto{\pgfqpoint{4.863706in}{1.953896in}}%
\pgfpathlineto{\pgfqpoint{4.868141in}{1.944686in}}%
\pgfpathlineto{\pgfqpoint{4.872576in}{1.920584in}}%
\pgfpathlineto{\pgfqpoint{4.881446in}{1.838659in}}%
\pgfpathlineto{\pgfqpoint{4.885881in}{1.860694in}}%
\pgfpathlineto{\pgfqpoint{4.894751in}{1.917434in}}%
\pgfpathlineto{\pgfqpoint{4.899186in}{1.910966in}}%
\pgfpathlineto{\pgfqpoint{4.903621in}{1.888929in}}%
\pgfpathlineto{\pgfqpoint{4.908056in}{1.848758in}}%
\pgfpathlineto{\pgfqpoint{4.912491in}{1.871791in}}%
\pgfpathlineto{\pgfqpoint{4.916926in}{1.862767in}}%
\pgfpathlineto{\pgfqpoint{4.921361in}{1.849628in}}%
\pgfpathlineto{\pgfqpoint{4.930231in}{1.808532in}}%
\pgfpathlineto{\pgfqpoint{4.934666in}{1.767169in}}%
\pgfpathlineto{\pgfqpoint{4.943536in}{1.713316in}}%
\pgfpathlineto{\pgfqpoint{4.947971in}{1.723915in}}%
\pgfpathlineto{\pgfqpoint{4.952406in}{1.710342in}}%
\pgfpathlineto{\pgfqpoint{4.956841in}{1.689786in}}%
\pgfpathlineto{\pgfqpoint{4.961276in}{1.658226in}}%
\pgfpathlineto{\pgfqpoint{4.965711in}{1.615935in}}%
\pgfpathlineto{\pgfqpoint{4.970146in}{1.595892in}}%
\pgfpathlineto{\pgfqpoint{4.974580in}{1.588521in}}%
\pgfpathlineto{\pgfqpoint{4.983450in}{1.566726in}}%
\pgfpathlineto{\pgfqpoint{4.987885in}{1.569161in}}%
\pgfpathlineto{\pgfqpoint{4.996755in}{1.516237in}}%
\pgfpathlineto{\pgfqpoint{5.001190in}{1.503732in}}%
\pgfpathlineto{\pgfqpoint{5.005625in}{1.502568in}}%
\pgfpathlineto{\pgfqpoint{5.010060in}{1.492444in}}%
\pgfpathlineto{\pgfqpoint{5.014495in}{1.477728in}}%
\pgfpathlineto{\pgfqpoint{5.018930in}{1.448895in}}%
\pgfpathlineto{\pgfqpoint{5.027800in}{1.373331in}}%
\pgfpathlineto{\pgfqpoint{5.036670in}{1.329176in}}%
\pgfpathlineto{\pgfqpoint{5.041105in}{1.306454in}}%
\pgfpathlineto{\pgfqpoint{5.045540in}{1.275923in}}%
\pgfpathlineto{\pgfqpoint{5.049975in}{1.258164in}}%
\pgfpathlineto{\pgfqpoint{5.058845in}{1.193744in}}%
\pgfpathlineto{\pgfqpoint{5.063280in}{1.177077in}}%
\pgfpathlineto{\pgfqpoint{5.072150in}{1.129363in}}%
\pgfpathlineto{\pgfqpoint{5.076585in}{1.114492in}}%
\pgfpathlineto{\pgfqpoint{5.081020in}{1.092330in}}%
\pgfpathlineto{\pgfqpoint{5.085455in}{1.098563in}}%
\pgfpathlineto{\pgfqpoint{5.089890in}{1.085801in}}%
\pgfpathlineto{\pgfqpoint{5.103195in}{0.985615in}}%
\pgfpathlineto{\pgfqpoint{5.107630in}{0.975555in}}%
\pgfpathlineto{\pgfqpoint{5.112065in}{0.958318in}}%
\pgfpathlineto{\pgfqpoint{5.120935in}{0.938344in}}%
\pgfpathlineto{\pgfqpoint{5.134240in}{0.873489in}}%
\pgfpathlineto{\pgfqpoint{5.143110in}{0.820713in}}%
\pgfpathlineto{\pgfqpoint{5.147545in}{0.819432in}}%
\pgfpathlineto{\pgfqpoint{5.151980in}{0.800522in}}%
\pgfpathlineto{\pgfqpoint{5.165285in}{0.756436in}}%
\pgfpathlineto{\pgfqpoint{5.169720in}{0.740557in}}%
\pgfpathlineto{\pgfqpoint{5.178590in}{0.724150in}}%
\pgfpathlineto{\pgfqpoint{5.191895in}{0.678330in}}%
\pgfpathlineto{\pgfqpoint{5.196330in}{0.672858in}}%
\pgfpathlineto{\pgfqpoint{5.200765in}{0.671892in}}%
\pgfpathlineto{\pgfqpoint{5.209635in}{0.675935in}}%
\pgfpathlineto{\pgfqpoint{5.214070in}{0.664344in}}%
\pgfpathlineto{\pgfqpoint{5.218505in}{0.659650in}}%
\pgfpathlineto{\pgfqpoint{5.227375in}{0.662608in}}%
\pgfpathlineto{\pgfqpoint{5.231810in}{0.660892in}}%
\pgfpathlineto{\pgfqpoint{5.236245in}{0.657221in}}%
\pgfpathlineto{\pgfqpoint{5.240680in}{0.645645in}}%
\pgfpathlineto{\pgfqpoint{5.245115in}{0.637366in}}%
\pgfpathlineto{\pgfqpoint{5.253985in}{0.613397in}}%
\pgfpathlineto{\pgfqpoint{5.258420in}{0.610056in}}%
\pgfpathlineto{\pgfqpoint{5.262855in}{0.610174in}}%
\pgfpathlineto{\pgfqpoint{5.271725in}{0.598254in}}%
\pgfpathlineto{\pgfqpoint{5.276160in}{0.594589in}}%
\pgfpathlineto{\pgfqpoint{5.280595in}{0.587946in}}%
\pgfpathlineto{\pgfqpoint{5.285030in}{0.587393in}}%
\pgfpathlineto{\pgfqpoint{5.289465in}{0.594429in}}%
\pgfpathlineto{\pgfqpoint{5.293900in}{0.603895in}}%
\pgfpathlineto{\pgfqpoint{5.298335in}{0.608359in}}%
\pgfpathlineto{\pgfqpoint{5.302770in}{0.605572in}}%
\pgfpathlineto{\pgfqpoint{5.307205in}{0.599267in}}%
\pgfpathlineto{\pgfqpoint{5.311640in}{0.589846in}}%
\pgfpathlineto{\pgfqpoint{5.316075in}{0.598722in}}%
\pgfpathlineto{\pgfqpoint{5.320510in}{0.611702in}}%
\pgfpathlineto{\pgfqpoint{5.324945in}{0.612725in}}%
\pgfpathlineto{\pgfqpoint{5.329380in}{0.608483in}}%
\pgfpathlineto{\pgfqpoint{5.333815in}{0.597365in}}%
\pgfpathlineto{\pgfqpoint{5.338250in}{0.589895in}}%
\pgfpathlineto{\pgfqpoint{5.351555in}{0.589351in}}%
\pgfpathlineto{\pgfqpoint{5.351555in}{0.589351in}}%
\pgfusepath{stroke}%
\end{pgfscope}%
\begin{pgfscope}%
\pgfsetrectcap%
\pgfsetmiterjoin%
\pgfsetlinewidth{1.003750pt}%
\definecolor{currentstroke}{rgb}{0.000000,0.000000,0.000000}%
\pgfsetstrokecolor{currentstroke}%
\pgfsetdash{}{0pt}%
\pgfpathmoveto{\pgfqpoint{3.821481in}{2.298130in}}%
\pgfpathlineto{\pgfqpoint{5.560000in}{2.298130in}}%
\pgfusepath{stroke}%
\end{pgfscope}%
\begin{pgfscope}%
\pgfsetrectcap%
\pgfsetmiterjoin%
\pgfsetlinewidth{1.003750pt}%
\definecolor{currentstroke}{rgb}{0.000000,0.000000,0.000000}%
\pgfsetstrokecolor{currentstroke}%
\pgfsetdash{}{0pt}%
\pgfpathmoveto{\pgfqpoint{3.821481in}{0.528906in}}%
\pgfpathlineto{\pgfqpoint{3.821481in}{2.298130in}}%
\pgfusepath{stroke}%
\end{pgfscope}%
\begin{pgfscope}%
\pgfsetrectcap%
\pgfsetmiterjoin%
\pgfsetlinewidth{1.003750pt}%
\definecolor{currentstroke}{rgb}{0.000000,0.000000,0.000000}%
\pgfsetstrokecolor{currentstroke}%
\pgfsetdash{}{0pt}%
\pgfpathmoveto{\pgfqpoint{3.821481in}{0.528906in}}%
\pgfpathlineto{\pgfqpoint{5.560000in}{0.528906in}}%
\pgfusepath{stroke}%
\end{pgfscope}%
\begin{pgfscope}%
\pgfsetrectcap%
\pgfsetmiterjoin%
\pgfsetlinewidth{1.003750pt}%
\definecolor{currentstroke}{rgb}{0.000000,0.000000,0.000000}%
\pgfsetstrokecolor{currentstroke}%
\pgfsetdash{}{0pt}%
\pgfpathmoveto{\pgfqpoint{5.560000in}{0.528906in}}%
\pgfpathlineto{\pgfqpoint{5.560000in}{2.298130in}}%
\pgfusepath{stroke}%
\end{pgfscope}%
\begin{pgfscope}%
\pgfsetbuttcap%
\pgfsetroundjoin%
\definecolor{currentfill}{rgb}{0.000000,0.000000,0.000000}%
\pgfsetfillcolor{currentfill}%
\pgfsetlinewidth{0.501875pt}%
\definecolor{currentstroke}{rgb}{0.000000,0.000000,0.000000}%
\pgfsetstrokecolor{currentstroke}%
\pgfsetdash{}{0pt}%
\pgfsys@defobject{currentmarker}{\pgfqpoint{0.000000in}{0.000000in}}{\pgfqpoint{0.000000in}{0.055556in}}{%
\pgfpathmoveto{\pgfqpoint{0.000000in}{0.000000in}}%
\pgfpathlineto{\pgfqpoint{0.000000in}{0.055556in}}%
\pgfusepath{stroke,fill}%
}%
\begin{pgfscope}%
\pgfsys@transformshift{4.147454in}{0.528906in}%
\pgfsys@useobject{currentmarker}{}%
\end{pgfscope}%
\end{pgfscope}%
\begin{pgfscope}%
\pgfsetbuttcap%
\pgfsetroundjoin%
\definecolor{currentfill}{rgb}{0.000000,0.000000,0.000000}%
\pgfsetfillcolor{currentfill}%
\pgfsetlinewidth{0.501875pt}%
\definecolor{currentstroke}{rgb}{0.000000,0.000000,0.000000}%
\pgfsetstrokecolor{currentstroke}%
\pgfsetdash{}{0pt}%
\pgfsys@defobject{currentmarker}{\pgfqpoint{0.000000in}{-0.055556in}}{\pgfqpoint{0.000000in}{0.000000in}}{%
\pgfpathmoveto{\pgfqpoint{0.000000in}{0.000000in}}%
\pgfpathlineto{\pgfqpoint{0.000000in}{-0.055556in}}%
\pgfusepath{stroke,fill}%
}%
\begin{pgfscope}%
\pgfsys@transformshift{4.147454in}{2.298130in}%
\pgfsys@useobject{currentmarker}{}%
\end{pgfscope}%
\end{pgfscope}%
\begin{pgfscope}%
\pgftext[x=4.147454in,y=0.473351in,,top]{\rmfamily\fontsize{10.000000}{12.000000}\selectfont -5}%
\end{pgfscope}%
\begin{pgfscope}%
\pgfsetbuttcap%
\pgfsetroundjoin%
\definecolor{currentfill}{rgb}{0.000000,0.000000,0.000000}%
\pgfsetfillcolor{currentfill}%
\pgfsetlinewidth{0.501875pt}%
\definecolor{currentstroke}{rgb}{0.000000,0.000000,0.000000}%
\pgfsetstrokecolor{currentstroke}%
\pgfsetdash{}{0pt}%
\pgfsys@defobject{currentmarker}{\pgfqpoint{0.000000in}{0.000000in}}{\pgfqpoint{0.000000in}{0.055556in}}{%
\pgfpathmoveto{\pgfqpoint{0.000000in}{0.000000in}}%
\pgfpathlineto{\pgfqpoint{0.000000in}{0.055556in}}%
\pgfusepath{stroke,fill}%
}%
\begin{pgfscope}%
\pgfsys@transformshift{4.690741in}{0.528906in}%
\pgfsys@useobject{currentmarker}{}%
\end{pgfscope}%
\end{pgfscope}%
\begin{pgfscope}%
\pgfsetbuttcap%
\pgfsetroundjoin%
\definecolor{currentfill}{rgb}{0.000000,0.000000,0.000000}%
\pgfsetfillcolor{currentfill}%
\pgfsetlinewidth{0.501875pt}%
\definecolor{currentstroke}{rgb}{0.000000,0.000000,0.000000}%
\pgfsetstrokecolor{currentstroke}%
\pgfsetdash{}{0pt}%
\pgfsys@defobject{currentmarker}{\pgfqpoint{0.000000in}{-0.055556in}}{\pgfqpoint{0.000000in}{0.000000in}}{%
\pgfpathmoveto{\pgfqpoint{0.000000in}{0.000000in}}%
\pgfpathlineto{\pgfqpoint{0.000000in}{-0.055556in}}%
\pgfusepath{stroke,fill}%
}%
\begin{pgfscope}%
\pgfsys@transformshift{4.690741in}{2.298130in}%
\pgfsys@useobject{currentmarker}{}%
\end{pgfscope}%
\end{pgfscope}%
\begin{pgfscope}%
\pgftext[x=4.690741in,y=0.473351in,,top]{\rmfamily\fontsize{10.000000}{12.000000}\selectfont 0}%
\end{pgfscope}%
\begin{pgfscope}%
\pgfsetbuttcap%
\pgfsetroundjoin%
\definecolor{currentfill}{rgb}{0.000000,0.000000,0.000000}%
\pgfsetfillcolor{currentfill}%
\pgfsetlinewidth{0.501875pt}%
\definecolor{currentstroke}{rgb}{0.000000,0.000000,0.000000}%
\pgfsetstrokecolor{currentstroke}%
\pgfsetdash{}{0pt}%
\pgfsys@defobject{currentmarker}{\pgfqpoint{0.000000in}{0.000000in}}{\pgfqpoint{0.000000in}{0.055556in}}{%
\pgfpathmoveto{\pgfqpoint{0.000000in}{0.000000in}}%
\pgfpathlineto{\pgfqpoint{0.000000in}{0.055556in}}%
\pgfusepath{stroke,fill}%
}%
\begin{pgfscope}%
\pgfsys@transformshift{5.234028in}{0.528906in}%
\pgfsys@useobject{currentmarker}{}%
\end{pgfscope}%
\end{pgfscope}%
\begin{pgfscope}%
\pgfsetbuttcap%
\pgfsetroundjoin%
\definecolor{currentfill}{rgb}{0.000000,0.000000,0.000000}%
\pgfsetfillcolor{currentfill}%
\pgfsetlinewidth{0.501875pt}%
\definecolor{currentstroke}{rgb}{0.000000,0.000000,0.000000}%
\pgfsetstrokecolor{currentstroke}%
\pgfsetdash{}{0pt}%
\pgfsys@defobject{currentmarker}{\pgfqpoint{0.000000in}{-0.055556in}}{\pgfqpoint{0.000000in}{0.000000in}}{%
\pgfpathmoveto{\pgfqpoint{0.000000in}{0.000000in}}%
\pgfpathlineto{\pgfqpoint{0.000000in}{-0.055556in}}%
\pgfusepath{stroke,fill}%
}%
\begin{pgfscope}%
\pgfsys@transformshift{5.234028in}{2.298130in}%
\pgfsys@useobject{currentmarker}{}%
\end{pgfscope}%
\end{pgfscope}%
\begin{pgfscope}%
\pgftext[x=5.234028in,y=0.473351in,,top]{\rmfamily\fontsize{10.000000}{12.000000}\selectfont 5}%
\end{pgfscope}%
\end{pgfpicture}%
\makeatother%
\endgroup%

    \caption[Blurring of flat-top electron profile with emittance.]{Unobstructed, unfocused electron beams generated with a flat-top profile on the excitation laser with varying ionisation laser wavelengths, and thus excess energies. The excess energy is listed above each beam profile. Below each profile is a trace of the central row of pixels (indicated by the dashed white line). The dotted lines on the lower figures indicate the profiles of the first and last profile for comparison.}
    \label{figure:flat_top}
    % Data and code at 2017.05.09
\end{figure}

If the electron beam has a simple, non-uniform spatial distribution, such as a Gaussian, then the specifics of the beam profile can be extrapolated from the pepperpot images allowing for further metrics to be calculated such as beam size and total electron count.
Using a Gaussian distribution also allows for the profile to be maintained with higher emittance beams.
An example is shown in Figure~\ref{figure:pepperpot_profile}, the rows and columns of the pepperpot image can be summed to generate a one-dimensional set of beamlets.
From the amplitude and standard deviation of each beamlet the total number of electrons in each beamlet can be calculated and this corresponds to the number of electrons that passed through the corresponding apertures.
The apertures sample the full beam and a Gaussian can be fitted to the beamlet electron counts to estimate the full beam profile and thus the beam size and full beam electron count can be determined.

\begin{figure}
    \center
    %% Creator: Matplotlib, PGF backend
%%
%% To include the figure in your LaTeX document, write
%%   \input{<filename>.pgf}
%%
%% Make sure the required packages are loaded in your preamble
%%   \usepackage{pgf}
%%
%% Figures using additional raster images can only be included by \input if
%% they are in the same directory as the main LaTeX file. For loading figures
%% from other directories you can use the `import` package
%%   \usepackage{import}
%% and then include the figures with
%%   \import{<path to file>}{<filename>.pgf}
%%
%% Matplotlib used the following preamble
%%
\begingroup%
\makeatletter%
\begin{pgfpicture}%
\pgfpathrectangle{\pgfpointorigin}{\pgfqpoint{5.700000in}{2.660000in}}%
\pgfusepath{use as bounding box, clip}%
\begin{pgfscope}%
\pgfsetbuttcap%
\pgfsetmiterjoin%
\definecolor{currentfill}{rgb}{1.000000,1.000000,1.000000}%
\pgfsetfillcolor{currentfill}%
\pgfsetlinewidth{0.000000pt}%
\definecolor{currentstroke}{rgb}{1.000000,1.000000,1.000000}%
\pgfsetstrokecolor{currentstroke}%
\pgfsetdash{}{0pt}%
\pgfpathmoveto{\pgfqpoint{0.000000in}{0.000000in}}%
\pgfpathlineto{\pgfqpoint{5.700000in}{0.000000in}}%
\pgfpathlineto{\pgfqpoint{5.700000in}{2.660000in}}%
\pgfpathlineto{\pgfqpoint{0.000000in}{2.660000in}}%
\pgfpathclose%
\pgfusepath{fill}%
\end{pgfscope}%
\begin{pgfscope}%
\pgfsetbuttcap%
\pgfsetmiterjoin%
\definecolor{currentfill}{rgb}{1.000000,1.000000,1.000000}%
\pgfsetfillcolor{currentfill}%
\pgfsetlinewidth{0.000000pt}%
\definecolor{currentstroke}{rgb}{0.000000,0.000000,0.000000}%
\pgfsetstrokecolor{currentstroke}%
\pgfsetstrokeopacity{0.000000}%
\pgfsetdash{}{0pt}%
\pgfpathmoveto{\pgfqpoint{0.175093in}{0.802500in}}%
\pgfpathlineto{\pgfqpoint{1.867593in}{0.802500in}}%
\pgfpathlineto{\pgfqpoint{1.867593in}{2.495000in}}%
\pgfpathlineto{\pgfqpoint{0.175093in}{2.495000in}}%
\pgfpathclose%
\pgfusepath{fill}%
\end{pgfscope}%
\begin{pgfscope}%
\pgfpathrectangle{\pgfqpoint{0.175093in}{0.802500in}}{\pgfqpoint{1.692500in}{1.692500in}} %
\pgfusepath{clip}%
\pgftext[at=\pgfqpoint{0.175093in}{0.802500in},left,bottom]{\pgfimage[interpolate=true,width=1.700000in,height=1.700000in]{beam_size-img0.png}}%
\end{pgfscope}%
\begin{pgfscope}%
\pgfsetrectcap%
\pgfsetmiterjoin%
\pgfsetlinewidth{1.003750pt}%
\definecolor{currentstroke}{rgb}{0.000000,0.000000,0.000000}%
\pgfsetstrokecolor{currentstroke}%
\pgfsetdash{}{0pt}%
\pgfpathmoveto{\pgfqpoint{0.175093in}{2.495000in}}%
\pgfpathlineto{\pgfqpoint{1.867593in}{2.495000in}}%
\pgfusepath{stroke}%
\end{pgfscope}%
\begin{pgfscope}%
\pgfsetrectcap%
\pgfsetmiterjoin%
\pgfsetlinewidth{1.003750pt}%
\definecolor{currentstroke}{rgb}{0.000000,0.000000,0.000000}%
\pgfsetstrokecolor{currentstroke}%
\pgfsetdash{}{0pt}%
\pgfpathmoveto{\pgfqpoint{1.867593in}{0.802500in}}%
\pgfpathlineto{\pgfqpoint{1.867593in}{2.495000in}}%
\pgfusepath{stroke}%
\end{pgfscope}%
\begin{pgfscope}%
\pgfsetrectcap%
\pgfsetmiterjoin%
\pgfsetlinewidth{1.003750pt}%
\definecolor{currentstroke}{rgb}{0.000000,0.000000,0.000000}%
\pgfsetstrokecolor{currentstroke}%
\pgfsetdash{}{0pt}%
\pgfpathmoveto{\pgfqpoint{0.175093in}{0.802500in}}%
\pgfpathlineto{\pgfqpoint{0.175093in}{2.495000in}}%
\pgfusepath{stroke}%
\end{pgfscope}%
\begin{pgfscope}%
\pgfsetrectcap%
\pgfsetmiterjoin%
\pgfsetlinewidth{1.003750pt}%
\definecolor{currentstroke}{rgb}{0.000000,0.000000,0.000000}%
\pgfsetstrokecolor{currentstroke}%
\pgfsetdash{}{0pt}%
\pgfpathmoveto{\pgfqpoint{0.175093in}{0.802500in}}%
\pgfpathlineto{\pgfqpoint{1.867593in}{0.802500in}}%
\pgfusepath{stroke}%
\end{pgfscope}%
\begin{pgfscope}%
\pgfsetbuttcap%
\pgfsetmiterjoin%
\definecolor{currentfill}{rgb}{1.000000,1.000000,1.000000}%
\pgfsetfillcolor{currentfill}%
\pgfsetlinewidth{0.000000pt}%
\definecolor{currentstroke}{rgb}{0.000000,0.000000,0.000000}%
\pgfsetstrokecolor{currentstroke}%
\pgfsetstrokeopacity{0.000000}%
\pgfsetdash{}{0pt}%
\pgfpathmoveto{\pgfqpoint{1.877685in}{0.802500in}}%
\pgfpathlineto{\pgfqpoint{3.590370in}{0.802500in}}%
\pgfpathlineto{\pgfqpoint{3.590370in}{2.495000in}}%
\pgfpathlineto{\pgfqpoint{1.877685in}{2.495000in}}%
\pgfpathclose%
\pgfusepath{fill}%
\end{pgfscope}%
\begin{pgfscope}%
\pgfpathrectangle{\pgfqpoint{1.877685in}{0.802500in}}{\pgfqpoint{1.712685in}{1.692500in}} %
\pgfusepath{clip}%
\pgfsetrectcap%
\pgfsetroundjoin%
\pgfsetlinewidth{1.003750pt}%
\definecolor{currentstroke}{rgb}{0.309804,0.478431,0.682353}%
\pgfsetstrokecolor{currentstroke}%
\pgfsetdash{}{0pt}%
\pgfpathmoveto{\pgfqpoint{1.867685in}{0.803628in}}%
\pgfpathlineto{\pgfqpoint{1.872566in}{0.802693in}}%
\pgfpathlineto{\pgfqpoint{1.883158in}{0.804573in}}%
\pgfpathlineto{\pgfqpoint{1.900810in}{0.803198in}}%
\pgfpathlineto{\pgfqpoint{1.904341in}{0.804772in}}%
\pgfpathlineto{\pgfqpoint{1.921994in}{0.804178in}}%
\pgfpathlineto{\pgfqpoint{1.936116in}{0.805658in}}%
\pgfpathlineto{\pgfqpoint{1.946708in}{0.806308in}}%
\pgfpathlineto{\pgfqpoint{1.950239in}{0.804553in}}%
\pgfpathlineto{\pgfqpoint{1.960830in}{0.805880in}}%
\pgfpathlineto{\pgfqpoint{1.967892in}{0.805168in}}%
\pgfpathlineto{\pgfqpoint{1.971422in}{0.807580in}}%
\pgfpathlineto{\pgfqpoint{1.974953in}{0.807029in}}%
\pgfpathlineto{\pgfqpoint{1.978483in}{0.805200in}}%
\pgfpathlineto{\pgfqpoint{1.982014in}{0.806274in}}%
\pgfpathlineto{\pgfqpoint{1.985544in}{0.805370in}}%
\pgfpathlineto{\pgfqpoint{1.989075in}{0.806790in}}%
\pgfpathlineto{\pgfqpoint{1.996136in}{0.805623in}}%
\pgfpathlineto{\pgfqpoint{2.006728in}{0.807267in}}%
\pgfpathlineto{\pgfqpoint{2.010259in}{0.806493in}}%
\pgfpathlineto{\pgfqpoint{2.013789in}{0.806999in}}%
\pgfpathlineto{\pgfqpoint{2.017320in}{0.805694in}}%
\pgfpathlineto{\pgfqpoint{2.024381in}{0.807548in}}%
\pgfpathlineto{\pgfqpoint{2.031442in}{0.808682in}}%
\pgfpathlineto{\pgfqpoint{2.038503in}{0.808958in}}%
\pgfpathlineto{\pgfqpoint{2.056156in}{0.809362in}}%
\pgfpathlineto{\pgfqpoint{2.063217in}{0.811313in}}%
\pgfpathlineto{\pgfqpoint{2.066748in}{0.811931in}}%
\pgfpathlineto{\pgfqpoint{2.070278in}{0.810458in}}%
\pgfpathlineto{\pgfqpoint{2.077340in}{0.812448in}}%
\pgfpathlineto{\pgfqpoint{2.084401in}{0.811681in}}%
\pgfpathlineto{\pgfqpoint{2.094992in}{0.813607in}}%
\pgfpathlineto{\pgfqpoint{2.098523in}{0.812697in}}%
\pgfpathlineto{\pgfqpoint{2.105584in}{0.813514in}}%
\pgfpathlineto{\pgfqpoint{2.109115in}{0.813857in}}%
\pgfpathlineto{\pgfqpoint{2.112645in}{0.812572in}}%
\pgfpathlineto{\pgfqpoint{2.116176in}{0.814585in}}%
\pgfpathlineto{\pgfqpoint{2.123237in}{0.812172in}}%
\pgfpathlineto{\pgfqpoint{2.126768in}{0.810953in}}%
\pgfpathlineto{\pgfqpoint{2.137359in}{0.814903in}}%
\pgfpathlineto{\pgfqpoint{2.140890in}{0.815649in}}%
\pgfpathlineto{\pgfqpoint{2.144421in}{0.815229in}}%
\pgfpathlineto{\pgfqpoint{2.151482in}{0.817793in}}%
\pgfpathlineto{\pgfqpoint{2.155012in}{0.817422in}}%
\pgfpathlineto{\pgfqpoint{2.158543in}{0.819142in}}%
\pgfpathlineto{\pgfqpoint{2.169135in}{0.817212in}}%
\pgfpathlineto{\pgfqpoint{2.172665in}{0.818835in}}%
\pgfpathlineto{\pgfqpoint{2.176196in}{0.817437in}}%
\pgfpathlineto{\pgfqpoint{2.200910in}{0.823816in}}%
\pgfpathlineto{\pgfqpoint{2.204441in}{0.825745in}}%
\pgfpathlineto{\pgfqpoint{2.207971in}{0.826040in}}%
\pgfpathlineto{\pgfqpoint{2.215032in}{0.828508in}}%
\pgfpathlineto{\pgfqpoint{2.218563in}{0.829925in}}%
\pgfpathlineto{\pgfqpoint{2.225624in}{0.830221in}}%
\pgfpathlineto{\pgfqpoint{2.236216in}{0.835626in}}%
\pgfpathlineto{\pgfqpoint{2.243277in}{0.840370in}}%
\pgfpathlineto{\pgfqpoint{2.253869in}{0.842871in}}%
\pgfpathlineto{\pgfqpoint{2.260930in}{0.844445in}}%
\pgfpathlineto{\pgfqpoint{2.267991in}{0.846283in}}%
\pgfpathlineto{\pgfqpoint{2.271522in}{0.846389in}}%
\pgfpathlineto{\pgfqpoint{2.275052in}{0.847737in}}%
\pgfpathlineto{\pgfqpoint{2.278583in}{0.847449in}}%
\pgfpathlineto{\pgfqpoint{2.303297in}{0.857919in}}%
\pgfpathlineto{\pgfqpoint{2.310358in}{0.856979in}}%
\pgfpathlineto{\pgfqpoint{2.313889in}{0.859430in}}%
\pgfpathlineto{\pgfqpoint{2.317419in}{0.858389in}}%
\pgfpathlineto{\pgfqpoint{2.320950in}{0.859750in}}%
\pgfpathlineto{\pgfqpoint{2.324480in}{0.857329in}}%
\pgfpathlineto{\pgfqpoint{2.331541in}{0.859046in}}%
\pgfpathlineto{\pgfqpoint{2.345664in}{0.857604in}}%
\pgfpathlineto{\pgfqpoint{2.349194in}{0.859332in}}%
\pgfpathlineto{\pgfqpoint{2.356256in}{0.858206in}}%
\pgfpathlineto{\pgfqpoint{2.359786in}{0.858173in}}%
\pgfpathlineto{\pgfqpoint{2.370378in}{0.855523in}}%
\pgfpathlineto{\pgfqpoint{2.380970in}{0.856338in}}%
\pgfpathlineto{\pgfqpoint{2.388031in}{0.860106in}}%
\pgfpathlineto{\pgfqpoint{2.391561in}{0.859614in}}%
\pgfpathlineto{\pgfqpoint{2.402153in}{0.870009in}}%
\pgfpathlineto{\pgfqpoint{2.409214in}{0.880269in}}%
\pgfpathlineto{\pgfqpoint{2.423337in}{0.911362in}}%
\pgfpathlineto{\pgfqpoint{2.433928in}{0.949747in}}%
\pgfpathlineto{\pgfqpoint{2.440989in}{0.980572in}}%
\pgfpathlineto{\pgfqpoint{2.465704in}{1.107698in}}%
\pgfpathlineto{\pgfqpoint{2.472765in}{1.131457in}}%
\pgfpathlineto{\pgfqpoint{2.476295in}{1.138752in}}%
\pgfpathlineto{\pgfqpoint{2.479826in}{1.138233in}}%
\pgfpathlineto{\pgfqpoint{2.483356in}{1.136283in}}%
\pgfpathlineto{\pgfqpoint{2.490418in}{1.121082in}}%
\pgfpathlineto{\pgfqpoint{2.501009in}{1.076735in}}%
\pgfpathlineto{\pgfqpoint{2.511601in}{1.009544in}}%
\pgfpathlineto{\pgfqpoint{2.522193in}{0.957132in}}%
\pgfpathlineto{\pgfqpoint{2.529254in}{0.939341in}}%
\pgfpathlineto{\pgfqpoint{2.532785in}{0.935512in}}%
\pgfpathlineto{\pgfqpoint{2.539846in}{0.939631in}}%
\pgfpathlineto{\pgfqpoint{2.543376in}{0.944119in}}%
\pgfpathlineto{\pgfqpoint{2.546907in}{0.950651in}}%
\pgfpathlineto{\pgfqpoint{2.553968in}{0.970608in}}%
\pgfpathlineto{\pgfqpoint{2.561029in}{0.997705in}}%
\pgfpathlineto{\pgfqpoint{2.568090in}{1.042221in}}%
\pgfpathlineto{\pgfqpoint{2.575152in}{1.104888in}}%
\pgfpathlineto{\pgfqpoint{2.582213in}{1.210414in}}%
\pgfpathlineto{\pgfqpoint{2.589274in}{1.367125in}}%
\pgfpathlineto{\pgfqpoint{2.606927in}{1.842863in}}%
\pgfpathlineto{\pgfqpoint{2.610457in}{1.894275in}}%
\pgfpathlineto{\pgfqpoint{2.613988in}{1.915297in}}%
\pgfpathlineto{\pgfqpoint{2.617519in}{1.898766in}}%
\pgfpathlineto{\pgfqpoint{2.621049in}{1.848269in}}%
\pgfpathlineto{\pgfqpoint{2.628110in}{1.666940in}}%
\pgfpathlineto{\pgfqpoint{2.642233in}{1.222222in}}%
\pgfpathlineto{\pgfqpoint{2.645763in}{1.151701in}}%
\pgfpathlineto{\pgfqpoint{2.652824in}{1.071550in}}%
\pgfpathlineto{\pgfqpoint{2.656355in}{1.045320in}}%
\pgfpathlineto{\pgfqpoint{2.663416in}{1.015437in}}%
\pgfpathlineto{\pgfqpoint{2.666947in}{1.007920in}}%
\pgfpathlineto{\pgfqpoint{2.670477in}{1.004413in}}%
\pgfpathlineto{\pgfqpoint{2.674008in}{1.006288in}}%
\pgfpathlineto{\pgfqpoint{2.681069in}{1.018029in}}%
\pgfpathlineto{\pgfqpoint{2.684600in}{1.027766in}}%
\pgfpathlineto{\pgfqpoint{2.691661in}{1.061319in}}%
\pgfpathlineto{\pgfqpoint{2.698722in}{1.113496in}}%
\pgfpathlineto{\pgfqpoint{2.705783in}{1.202685in}}%
\pgfpathlineto{\pgfqpoint{2.709314in}{1.274084in}}%
\pgfpathlineto{\pgfqpoint{2.712844in}{1.388337in}}%
\pgfpathlineto{\pgfqpoint{2.719905in}{1.774665in}}%
\pgfpathlineto{\pgfqpoint{2.726967in}{2.230381in}}%
\pgfpathlineto{\pgfqpoint{2.730497in}{2.382441in}}%
\pgfpathlineto{\pgfqpoint{2.734028in}{2.461119in}}%
\pgfpathlineto{\pgfqpoint{2.737558in}{2.455017in}}%
\pgfpathlineto{\pgfqpoint{2.741089in}{2.362135in}}%
\pgfpathlineto{\pgfqpoint{2.744620in}{2.182697in}}%
\pgfpathlineto{\pgfqpoint{2.755211in}{1.451758in}}%
\pgfpathlineto{\pgfqpoint{2.758742in}{1.298750in}}%
\pgfpathlineto{\pgfqpoint{2.762272in}{1.201681in}}%
\pgfpathlineto{\pgfqpoint{2.765803in}{1.140470in}}%
\pgfpathlineto{\pgfqpoint{2.772864in}{1.066605in}}%
\pgfpathlineto{\pgfqpoint{2.779925in}{1.021375in}}%
\pgfpathlineto{\pgfqpoint{2.783456in}{1.007529in}}%
\pgfpathlineto{\pgfqpoint{2.786987in}{0.998815in}}%
\pgfpathlineto{\pgfqpoint{2.790517in}{0.994637in}}%
\pgfpathlineto{\pgfqpoint{2.797578in}{0.995777in}}%
\pgfpathlineto{\pgfqpoint{2.801109in}{1.003760in}}%
\pgfpathlineto{\pgfqpoint{2.808170in}{1.034256in}}%
\pgfpathlineto{\pgfqpoint{2.815231in}{1.078990in}}%
\pgfpathlineto{\pgfqpoint{2.822292in}{1.153086in}}%
\pgfpathlineto{\pgfqpoint{2.825823in}{1.214391in}}%
\pgfpathlineto{\pgfqpoint{2.829353in}{1.308214in}}%
\pgfpathlineto{\pgfqpoint{2.836415in}{1.620222in}}%
\pgfpathlineto{\pgfqpoint{2.843476in}{1.995830in}}%
\pgfpathlineto{\pgfqpoint{2.847006in}{2.124262in}}%
\pgfpathlineto{\pgfqpoint{2.850537in}{2.192227in}}%
\pgfpathlineto{\pgfqpoint{2.854068in}{2.197489in}}%
\pgfpathlineto{\pgfqpoint{2.857598in}{2.141017in}}%
\pgfpathlineto{\pgfqpoint{2.861129in}{2.030737in}}%
\pgfpathlineto{\pgfqpoint{2.878782in}{1.256960in}}%
\pgfpathlineto{\pgfqpoint{2.885843in}{1.117539in}}%
\pgfpathlineto{\pgfqpoint{2.892904in}{1.045265in}}%
\pgfpathlineto{\pgfqpoint{2.899965in}{0.998486in}}%
\pgfpathlineto{\pgfqpoint{2.907026in}{0.967333in}}%
\pgfpathlineto{\pgfqpoint{2.910557in}{0.957365in}}%
\pgfpathlineto{\pgfqpoint{2.921149in}{0.945253in}}%
\pgfpathlineto{\pgfqpoint{2.924679in}{0.945991in}}%
\pgfpathlineto{\pgfqpoint{2.928210in}{0.949295in}}%
\pgfpathlineto{\pgfqpoint{2.931740in}{0.954773in}}%
\pgfpathlineto{\pgfqpoint{2.935271in}{0.964456in}}%
\pgfpathlineto{\pgfqpoint{2.938802in}{0.980927in}}%
\pgfpathlineto{\pgfqpoint{2.945863in}{1.036867in}}%
\pgfpathlineto{\pgfqpoint{2.956454in}{1.144137in}}%
\pgfpathlineto{\pgfqpoint{2.963516in}{1.212821in}}%
\pgfpathlineto{\pgfqpoint{2.967046in}{1.235810in}}%
\pgfpathlineto{\pgfqpoint{2.970577in}{1.249735in}}%
\pgfpathlineto{\pgfqpoint{2.974107in}{1.253863in}}%
\pgfpathlineto{\pgfqpoint{2.977638in}{1.248784in}}%
\pgfpathlineto{\pgfqpoint{2.981168in}{1.232598in}}%
\pgfpathlineto{\pgfqpoint{2.988230in}{1.172429in}}%
\pgfpathlineto{\pgfqpoint{3.002352in}{1.029097in}}%
\pgfpathlineto{\pgfqpoint{3.009413in}{0.971077in}}%
\pgfpathlineto{\pgfqpoint{3.016474in}{0.929509in}}%
\pgfpathlineto{\pgfqpoint{3.023535in}{0.903687in}}%
\pgfpathlineto{\pgfqpoint{3.030597in}{0.884037in}}%
\pgfpathlineto{\pgfqpoint{3.034127in}{0.875771in}}%
\pgfpathlineto{\pgfqpoint{3.041188in}{0.867344in}}%
\pgfpathlineto{\pgfqpoint{3.044719in}{0.863711in}}%
\pgfpathlineto{\pgfqpoint{3.055311in}{0.860045in}}%
\pgfpathlineto{\pgfqpoint{3.058841in}{0.860810in}}%
\pgfpathlineto{\pgfqpoint{3.065902in}{0.866602in}}%
\pgfpathlineto{\pgfqpoint{3.069433in}{0.872919in}}%
\pgfpathlineto{\pgfqpoint{3.080025in}{0.881132in}}%
\pgfpathlineto{\pgfqpoint{3.087086in}{0.884590in}}%
\pgfpathlineto{\pgfqpoint{3.097678in}{0.890880in}}%
\pgfpathlineto{\pgfqpoint{3.108269in}{0.890364in}}%
\pgfpathlineto{\pgfqpoint{3.129453in}{0.874718in}}%
\pgfpathlineto{\pgfqpoint{3.132984in}{0.873402in}}%
\pgfpathlineto{\pgfqpoint{3.147106in}{0.862513in}}%
\pgfpathlineto{\pgfqpoint{3.150636in}{0.858256in}}%
\pgfpathlineto{\pgfqpoint{3.157698in}{0.847790in}}%
\pgfpathlineto{\pgfqpoint{3.168289in}{0.841738in}}%
\pgfpathlineto{\pgfqpoint{3.171820in}{0.837603in}}%
\pgfpathlineto{\pgfqpoint{3.175350in}{0.835914in}}%
\pgfpathlineto{\pgfqpoint{3.182412in}{0.830772in}}%
\pgfpathlineto{\pgfqpoint{3.185942in}{0.828594in}}%
\pgfpathlineto{\pgfqpoint{3.193003in}{0.826948in}}%
\pgfpathlineto{\pgfqpoint{3.196534in}{0.823169in}}%
\pgfpathlineto{\pgfqpoint{3.203595in}{0.822734in}}%
\pgfpathlineto{\pgfqpoint{3.210656in}{0.819082in}}%
\pgfpathlineto{\pgfqpoint{3.214187in}{0.818575in}}%
\pgfpathlineto{\pgfqpoint{3.217717in}{0.816661in}}%
\pgfpathlineto{\pgfqpoint{3.224779in}{0.816974in}}%
\pgfpathlineto{\pgfqpoint{3.228309in}{0.815732in}}%
\pgfpathlineto{\pgfqpoint{3.231840in}{0.816759in}}%
\pgfpathlineto{\pgfqpoint{3.235370in}{0.814554in}}%
\pgfpathlineto{\pgfqpoint{3.238901in}{0.815215in}}%
\pgfpathlineto{\pgfqpoint{3.242432in}{0.812732in}}%
\pgfpathlineto{\pgfqpoint{3.249493in}{0.812083in}}%
\pgfpathlineto{\pgfqpoint{3.256554in}{0.811093in}}%
\pgfpathlineto{\pgfqpoint{3.263615in}{0.811000in}}%
\pgfpathlineto{\pgfqpoint{3.267146in}{0.811716in}}%
\pgfpathlineto{\pgfqpoint{3.281268in}{0.808692in}}%
\pgfpathlineto{\pgfqpoint{3.291860in}{0.808896in}}%
\pgfpathlineto{\pgfqpoint{3.295390in}{0.808094in}}%
\pgfpathlineto{\pgfqpoint{3.298921in}{0.808677in}}%
\pgfpathlineto{\pgfqpoint{3.305982in}{0.806736in}}%
\pgfpathlineto{\pgfqpoint{3.313043in}{0.809426in}}%
\pgfpathlineto{\pgfqpoint{3.316574in}{0.808610in}}%
\pgfpathlineto{\pgfqpoint{3.320104in}{0.809120in}}%
\pgfpathlineto{\pgfqpoint{3.327166in}{0.807020in}}%
\pgfpathlineto{\pgfqpoint{3.334227in}{0.806817in}}%
\pgfpathlineto{\pgfqpoint{3.348349in}{0.806423in}}%
\pgfpathlineto{\pgfqpoint{3.351880in}{0.805538in}}%
\pgfpathlineto{\pgfqpoint{3.358941in}{0.805973in}}%
\pgfpathlineto{\pgfqpoint{3.362471in}{0.804112in}}%
\pgfpathlineto{\pgfqpoint{3.369532in}{0.806172in}}%
\pgfpathlineto{\pgfqpoint{3.418961in}{0.804692in}}%
\pgfpathlineto{\pgfqpoint{3.429552in}{0.803996in}}%
\pgfpathlineto{\pgfqpoint{3.443675in}{0.802544in}}%
\pgfpathlineto{\pgfqpoint{3.447205in}{0.803440in}}%
\pgfpathlineto{\pgfqpoint{3.450736in}{0.801995in}}%
\pgfpathlineto{\pgfqpoint{3.461328in}{0.803210in}}%
\pgfpathlineto{\pgfqpoint{3.464858in}{0.801447in}}%
\pgfpathlineto{\pgfqpoint{3.468389in}{0.803380in}}%
\pgfpathlineto{\pgfqpoint{3.478981in}{0.802883in}}%
\pgfpathlineto{\pgfqpoint{3.486042in}{0.800459in}}%
\pgfpathlineto{\pgfqpoint{3.489572in}{0.801667in}}%
\pgfpathlineto{\pgfqpoint{3.510756in}{0.801697in}}%
\pgfpathlineto{\pgfqpoint{3.514286in}{0.799964in}}%
\pgfpathlineto{\pgfqpoint{3.521347in}{0.801694in}}%
\pgfpathlineto{\pgfqpoint{3.524878in}{0.800774in}}%
\pgfpathlineto{\pgfqpoint{3.531939in}{0.801229in}}%
\pgfpathlineto{\pgfqpoint{3.549592in}{0.802240in}}%
\pgfpathlineto{\pgfqpoint{3.556653in}{0.801140in}}%
\pgfpathlineto{\pgfqpoint{3.560184in}{0.799926in}}%
\pgfpathlineto{\pgfqpoint{3.570776in}{0.801110in}}%
\pgfpathlineto{\pgfqpoint{3.574306in}{0.800103in}}%
\pgfpathlineto{\pgfqpoint{3.581367in}{0.802340in}}%
\pgfpathlineto{\pgfqpoint{3.584898in}{0.802237in}}%
\pgfpathlineto{\pgfqpoint{3.591959in}{0.799565in}}%
\pgfpathlineto{\pgfqpoint{3.600370in}{0.799745in}}%
\pgfpathlineto{\pgfqpoint{3.600370in}{0.799745in}}%
\pgfusepath{stroke}%
\end{pgfscope}%
\begin{pgfscope}%
\pgfsetrectcap%
\pgfsetmiterjoin%
\pgfsetlinewidth{1.003750pt}%
\definecolor{currentstroke}{rgb}{0.000000,0.000000,0.000000}%
\pgfsetstrokecolor{currentstroke}%
\pgfsetdash{}{0pt}%
\pgfpathmoveto{\pgfqpoint{1.877685in}{2.495000in}}%
\pgfpathlineto{\pgfqpoint{3.590370in}{2.495000in}}%
\pgfusepath{stroke}%
\end{pgfscope}%
\begin{pgfscope}%
\pgfsetrectcap%
\pgfsetmiterjoin%
\pgfsetlinewidth{1.003750pt}%
\definecolor{currentstroke}{rgb}{0.000000,0.000000,0.000000}%
\pgfsetstrokecolor{currentstroke}%
\pgfsetdash{}{0pt}%
\pgfpathmoveto{\pgfqpoint{3.590370in}{0.802500in}}%
\pgfpathlineto{\pgfqpoint{3.590370in}{2.495000in}}%
\pgfusepath{stroke}%
\end{pgfscope}%
\begin{pgfscope}%
\pgfsetrectcap%
\pgfsetmiterjoin%
\pgfsetlinewidth{1.003750pt}%
\definecolor{currentstroke}{rgb}{0.000000,0.000000,0.000000}%
\pgfsetstrokecolor{currentstroke}%
\pgfsetdash{}{0pt}%
\pgfpathmoveto{\pgfqpoint{1.877685in}{0.802500in}}%
\pgfpathlineto{\pgfqpoint{1.877685in}{2.495000in}}%
\pgfusepath{stroke}%
\end{pgfscope}%
\begin{pgfscope}%
\pgfsetrectcap%
\pgfsetmiterjoin%
\pgfsetlinewidth{1.003750pt}%
\definecolor{currentstroke}{rgb}{0.000000,0.000000,0.000000}%
\pgfsetstrokecolor{currentstroke}%
\pgfsetdash{}{0pt}%
\pgfpathmoveto{\pgfqpoint{1.877685in}{0.802500in}}%
\pgfpathlineto{\pgfqpoint{3.590370in}{0.802500in}}%
\pgfusepath{stroke}%
\end{pgfscope}%
\begin{pgfscope}%
\pgfsetbuttcap%
\pgfsetroundjoin%
\definecolor{currentfill}{rgb}{0.000000,0.000000,0.000000}%
\pgfsetfillcolor{currentfill}%
\pgfsetlinewidth{0.501875pt}%
\definecolor{currentstroke}{rgb}{0.000000,0.000000,0.000000}%
\pgfsetstrokecolor{currentstroke}%
\pgfsetdash{}{0pt}%
\pgfsys@defobject{currentmarker}{\pgfqpoint{0.000000in}{0.000000in}}{\pgfqpoint{0.000000in}{0.055556in}}{%
\pgfpathmoveto{\pgfqpoint{0.000000in}{0.000000in}}%
\pgfpathlineto{\pgfqpoint{0.000000in}{0.055556in}}%
\pgfusepath{stroke,fill}%
}%
\begin{pgfscope}%
\pgfsys@transformshift{2.301532in}{0.802500in}%
\pgfsys@useobject{currentmarker}{}%
\end{pgfscope}%
\end{pgfscope}%
\begin{pgfscope}%
\pgfsetbuttcap%
\pgfsetroundjoin%
\definecolor{currentfill}{rgb}{0.000000,0.000000,0.000000}%
\pgfsetfillcolor{currentfill}%
\pgfsetlinewidth{0.501875pt}%
\definecolor{currentstroke}{rgb}{0.000000,0.000000,0.000000}%
\pgfsetstrokecolor{currentstroke}%
\pgfsetdash{}{0pt}%
\pgfsys@defobject{currentmarker}{\pgfqpoint{0.000000in}{-0.055556in}}{\pgfqpoint{0.000000in}{0.000000in}}{%
\pgfpathmoveto{\pgfqpoint{0.000000in}{0.000000in}}%
\pgfpathlineto{\pgfqpoint{0.000000in}{-0.055556in}}%
\pgfusepath{stroke,fill}%
}%
\begin{pgfscope}%
\pgfsys@transformshift{2.301532in}{2.495000in}%
\pgfsys@useobject{currentmarker}{}%
\end{pgfscope}%
\end{pgfscope}%
\begin{pgfscope}%
\pgftext[x=2.301532in,y=0.746944in,,top]{\rmfamily\fontsize{11.000000}{13.200000}\selectfont -5}%
\end{pgfscope}%
\begin{pgfscope}%
\pgfsetbuttcap%
\pgfsetroundjoin%
\definecolor{currentfill}{rgb}{0.000000,0.000000,0.000000}%
\pgfsetfillcolor{currentfill}%
\pgfsetlinewidth{0.501875pt}%
\definecolor{currentstroke}{rgb}{0.000000,0.000000,0.000000}%
\pgfsetstrokecolor{currentstroke}%
\pgfsetdash{}{0pt}%
\pgfsys@defobject{currentmarker}{\pgfqpoint{0.000000in}{0.000000in}}{\pgfqpoint{0.000000in}{0.055556in}}{%
\pgfpathmoveto{\pgfqpoint{0.000000in}{0.000000in}}%
\pgfpathlineto{\pgfqpoint{0.000000in}{0.055556in}}%
\pgfusepath{stroke,fill}%
}%
\begin{pgfscope}%
\pgfsys@transformshift{2.734028in}{0.802500in}%
\pgfsys@useobject{currentmarker}{}%
\end{pgfscope}%
\end{pgfscope}%
\begin{pgfscope}%
\pgfsetbuttcap%
\pgfsetroundjoin%
\definecolor{currentfill}{rgb}{0.000000,0.000000,0.000000}%
\pgfsetfillcolor{currentfill}%
\pgfsetlinewidth{0.501875pt}%
\definecolor{currentstroke}{rgb}{0.000000,0.000000,0.000000}%
\pgfsetstrokecolor{currentstroke}%
\pgfsetdash{}{0pt}%
\pgfsys@defobject{currentmarker}{\pgfqpoint{0.000000in}{-0.055556in}}{\pgfqpoint{0.000000in}{0.000000in}}{%
\pgfpathmoveto{\pgfqpoint{0.000000in}{0.000000in}}%
\pgfpathlineto{\pgfqpoint{0.000000in}{-0.055556in}}%
\pgfusepath{stroke,fill}%
}%
\begin{pgfscope}%
\pgfsys@transformshift{2.734028in}{2.495000in}%
\pgfsys@useobject{currentmarker}{}%
\end{pgfscope}%
\end{pgfscope}%
\begin{pgfscope}%
\pgftext[x=2.734028in,y=0.746944in,,top]{\rmfamily\fontsize{11.000000}{13.200000}\selectfont 0}%
\end{pgfscope}%
\begin{pgfscope}%
\pgfsetbuttcap%
\pgfsetroundjoin%
\definecolor{currentfill}{rgb}{0.000000,0.000000,0.000000}%
\pgfsetfillcolor{currentfill}%
\pgfsetlinewidth{0.501875pt}%
\definecolor{currentstroke}{rgb}{0.000000,0.000000,0.000000}%
\pgfsetstrokecolor{currentstroke}%
\pgfsetdash{}{0pt}%
\pgfsys@defobject{currentmarker}{\pgfqpoint{0.000000in}{0.000000in}}{\pgfqpoint{0.000000in}{0.055556in}}{%
\pgfpathmoveto{\pgfqpoint{0.000000in}{0.000000in}}%
\pgfpathlineto{\pgfqpoint{0.000000in}{0.055556in}}%
\pgfusepath{stroke,fill}%
}%
\begin{pgfscope}%
\pgfsys@transformshift{3.166524in}{0.802500in}%
\pgfsys@useobject{currentmarker}{}%
\end{pgfscope}%
\end{pgfscope}%
\begin{pgfscope}%
\pgfsetbuttcap%
\pgfsetroundjoin%
\definecolor{currentfill}{rgb}{0.000000,0.000000,0.000000}%
\pgfsetfillcolor{currentfill}%
\pgfsetlinewidth{0.501875pt}%
\definecolor{currentstroke}{rgb}{0.000000,0.000000,0.000000}%
\pgfsetstrokecolor{currentstroke}%
\pgfsetdash{}{0pt}%
\pgfsys@defobject{currentmarker}{\pgfqpoint{0.000000in}{-0.055556in}}{\pgfqpoint{0.000000in}{0.000000in}}{%
\pgfpathmoveto{\pgfqpoint{0.000000in}{0.000000in}}%
\pgfpathlineto{\pgfqpoint{0.000000in}{-0.055556in}}%
\pgfusepath{stroke,fill}%
}%
\begin{pgfscope}%
\pgfsys@transformshift{3.166524in}{2.495000in}%
\pgfsys@useobject{currentmarker}{}%
\end{pgfscope}%
\end{pgfscope}%
\begin{pgfscope}%
\pgftext[x=3.166524in,y=0.746944in,,top]{\rmfamily\fontsize{11.000000}{13.200000}\selectfont 5}%
\end{pgfscope}%
\begin{pgfscope}%
\pgftext[x=2.152520in,y=0.445150in,left,base]{\rmfamily\fontsize{11.000000}{13.200000}\selectfont Detector Position}%
\end{pgfscope}%
\begin{pgfscope}%
\pgftext[x=2.548148in,y=0.278704in,left,base]{\rmfamily\fontsize{11.000000}{13.200000}\selectfont (mm)}%
\end{pgfscope}%
\begin{pgfscope}%
\pgfsetbuttcap%
\pgfsetmiterjoin%
\definecolor{currentfill}{rgb}{1.000000,1.000000,1.000000}%
\pgfsetfillcolor{currentfill}%
\pgfsetlinewidth{0.000000pt}%
\definecolor{currentstroke}{rgb}{0.000000,0.000000,0.000000}%
\pgfsetstrokecolor{currentstroke}%
\pgfsetstrokeopacity{0.000000}%
\pgfsetdash{}{0pt}%
\pgfpathmoveto{\pgfqpoint{3.590370in}{0.802500in}}%
\pgfpathlineto{\pgfqpoint{5.303056in}{0.802500in}}%
\pgfpathlineto{\pgfqpoint{5.303056in}{2.495000in}}%
\pgfpathlineto{\pgfqpoint{3.590370in}{2.495000in}}%
\pgfpathclose%
\pgfusepath{fill}%
\end{pgfscope}%
\begin{pgfscope}%
\pgfpathrectangle{\pgfqpoint{3.590370in}{0.802500in}}{\pgfqpoint{1.712685in}{1.692500in}} %
\pgfusepath{clip}%
\pgfsetrectcap%
\pgfsetroundjoin%
\pgfsetlinewidth{1.003750pt}%
\definecolor{currentstroke}{rgb}{0.000000,0.000000,0.000000}%
\pgfsetstrokecolor{currentstroke}%
\pgfsetdash{}{0pt}%
\pgfpathmoveto{\pgfqpoint{3.590370in}{0.802500in}}%
\pgfpathlineto{\pgfqpoint{3.890005in}{0.802500in}}%
\pgfpathlineto{\pgfqpoint{3.891118in}{2.291918in}}%
\pgfpathlineto{\pgfqpoint{3.975639in}{2.291918in}}%
\pgfpathlineto{\pgfqpoint{3.976752in}{0.802500in}}%
\pgfpathlineto{\pgfqpoint{4.061273in}{0.802500in}}%
\pgfpathlineto{\pgfqpoint{4.062386in}{2.291918in}}%
\pgfpathlineto{\pgfqpoint{4.146907in}{2.291918in}}%
\pgfpathlineto{\pgfqpoint{4.148021in}{0.802500in}}%
\pgfpathlineto{\pgfqpoint{4.232542in}{0.802500in}}%
\pgfpathlineto{\pgfqpoint{4.233655in}{2.291918in}}%
\pgfpathlineto{\pgfqpoint{4.318176in}{2.291918in}}%
\pgfpathlineto{\pgfqpoint{4.319289in}{0.802500in}}%
\pgfpathlineto{\pgfqpoint{4.403810in}{0.802500in}}%
\pgfpathlineto{\pgfqpoint{4.404923in}{2.291918in}}%
\pgfpathlineto{\pgfqpoint{4.489444in}{2.291918in}}%
\pgfpathlineto{\pgfqpoint{4.490558in}{0.802500in}}%
\pgfpathlineto{\pgfqpoint{4.575079in}{0.802500in}}%
\pgfpathlineto{\pgfqpoint{4.576192in}{2.291918in}}%
\pgfpathlineto{\pgfqpoint{4.660713in}{2.291918in}}%
\pgfpathlineto{\pgfqpoint{4.661826in}{0.802500in}}%
\pgfpathlineto{\pgfqpoint{4.746347in}{0.802500in}}%
\pgfpathlineto{\pgfqpoint{4.747460in}{2.291918in}}%
\pgfpathlineto{\pgfqpoint{4.831981in}{2.291918in}}%
\pgfpathlineto{\pgfqpoint{4.833095in}{0.802500in}}%
\pgfpathlineto{\pgfqpoint{4.917616in}{0.802500in}}%
\pgfpathlineto{\pgfqpoint{4.918729in}{2.291918in}}%
\pgfpathlineto{\pgfqpoint{5.003250in}{2.291918in}}%
\pgfpathlineto{\pgfqpoint{5.004363in}{0.802500in}}%
\pgfpathlineto{\pgfqpoint{5.302970in}{0.802500in}}%
\pgfpathlineto{\pgfqpoint{5.302970in}{0.802500in}}%
\pgfusepath{stroke}%
\end{pgfscope}%
\begin{pgfscope}%
\pgfpathrectangle{\pgfqpoint{3.590370in}{0.802500in}}{\pgfqpoint{1.712685in}{1.692500in}} %
\pgfusepath{clip}%
\pgfsetbuttcap%
\pgfsetroundjoin%
\definecolor{currentfill}{rgb}{0.309804,0.478431,0.682353}%
\pgfsetfillcolor{currentfill}%
\pgfsetlinewidth{0.501875pt}%
\definecolor{currentstroke}{rgb}{0.309804,0.478431,0.682353}%
\pgfsetstrokecolor{currentstroke}%
\pgfsetdash{}{0pt}%
\pgfsys@defobject{currentmarker}{\pgfqpoint{-0.041667in}{-0.041667in}}{\pgfqpoint{0.041667in}{0.041667in}}{%
\pgfpathmoveto{\pgfqpoint{-0.041667in}{-0.041667in}}%
\pgfpathlineto{\pgfqpoint{0.041667in}{0.041667in}}%
\pgfpathmoveto{\pgfqpoint{-0.041667in}{0.041667in}}%
\pgfpathlineto{\pgfqpoint{0.041667in}{-0.041667in}}%
\pgfusepath{stroke,fill}%
}%
\begin{pgfscope}%
\pgfsys@transformshift{3.932907in}{0.903623in}%
\pgfsys@useobject{currentmarker}{}%
\end{pgfscope}%
\begin{pgfscope}%
\pgfsys@transformshift{4.104176in}{1.202603in}%
\pgfsys@useobject{currentmarker}{}%
\end{pgfscope}%
\begin{pgfscope}%
\pgfsys@transformshift{4.275444in}{2.061265in}%
\pgfsys@useobject{currentmarker}{}%
\end{pgfscope}%
\begin{pgfscope}%
\pgfsys@transformshift{4.446713in}{2.291918in}%
\pgfsys@useobject{currentmarker}{}%
\end{pgfscope}%
\begin{pgfscope}%
\pgfsys@transformshift{4.617981in}{2.036437in}%
\pgfsys@useobject{currentmarker}{}%
\end{pgfscope}%
\begin{pgfscope}%
\pgfsys@transformshift{4.789250in}{1.264041in}%
\pgfsys@useobject{currentmarker}{}%
\end{pgfscope}%
\begin{pgfscope}%
\pgfsys@transformshift{4.960519in}{0.973105in}%
\pgfsys@useobject{currentmarker}{}%
\end{pgfscope}%
\end{pgfscope}%
\begin{pgfscope}%
\pgfpathrectangle{\pgfqpoint{3.590370in}{0.802500in}}{\pgfqpoint{1.712685in}{1.692500in}} %
\pgfusepath{clip}%
\pgfsetbuttcap%
\pgfsetroundjoin%
\pgfsetlinewidth{1.003750pt}%
\definecolor{currentstroke}{rgb}{1.000000,0.400000,0.200000}%
\pgfsetstrokecolor{currentstroke}%
\pgfsetdash{{1.000000pt}{3.000000pt}}{0.000000pt}%
\pgfpathmoveto{\pgfqpoint{3.590370in}{0.804030in}}%
\pgfpathlineto{\pgfqpoint{3.664872in}{0.806416in}}%
\pgfpathlineto{\pgfqpoint{3.713512in}{0.810104in}}%
\pgfpathlineto{\pgfqpoint{3.750849in}{0.815094in}}%
\pgfpathlineto{\pgfqpoint{3.781849in}{0.821428in}}%
\pgfpathlineto{\pgfqpoint{3.808738in}{0.829154in}}%
\pgfpathlineto{\pgfqpoint{3.832887in}{0.838386in}}%
\pgfpathlineto{\pgfqpoint{3.855066in}{0.849239in}}%
\pgfpathlineto{\pgfqpoint{3.875789in}{0.861848in}}%
\pgfpathlineto{\pgfqpoint{3.895571in}{0.876492in}}%
\pgfpathlineto{\pgfqpoint{3.914667in}{0.893408in}}%
\pgfpathlineto{\pgfqpoint{3.933421in}{0.913027in}}%
\pgfpathlineto{\pgfqpoint{3.951918in}{0.935632in}}%
\pgfpathlineto{\pgfqpoint{3.970415in}{0.961797in}}%
\pgfpathlineto{\pgfqpoint{3.989083in}{0.992130in}}%
\pgfpathlineto{\pgfqpoint{4.008009in}{1.027208in}}%
\pgfpathlineto{\pgfqpoint{4.027362in}{1.067857in}}%
\pgfpathlineto{\pgfqpoint{4.047315in}{1.115049in}}%
\pgfpathlineto{\pgfqpoint{4.068038in}{1.169894in}}%
\pgfpathlineto{\pgfqpoint{4.089875in}{1.234139in}}%
\pgfpathlineto{\pgfqpoint{4.113168in}{1.309781in}}%
\pgfpathlineto{\pgfqpoint{4.138515in}{1.399880in}}%
\pgfpathlineto{\pgfqpoint{4.167374in}{1.510998in}}%
\pgfpathlineto{\pgfqpoint{4.203940in}{1.661290in}}%
\pgfpathlineto{\pgfqpoint{4.290088in}{2.018845in}}%
\pgfpathlineto{\pgfqpoint{4.315693in}{2.113556in}}%
\pgfpathlineto{\pgfqpoint{4.336930in}{2.183938in}}%
\pgfpathlineto{\pgfqpoint{4.355256in}{2.237307in}}%
\pgfpathlineto{\pgfqpoint{4.371526in}{2.278126in}}%
\pgfpathlineto{\pgfqpoint{4.386084in}{2.308886in}}%
\pgfpathlineto{\pgfqpoint{4.399186in}{2.331592in}}%
\pgfpathlineto{\pgfqpoint{4.411003in}{2.347828in}}%
\pgfpathlineto{\pgfqpoint{4.421708in}{2.358943in}}%
\pgfpathlineto{\pgfqpoint{4.431556in}{2.366087in}}%
\pgfpathlineto{\pgfqpoint{4.440633in}{2.370017in}}%
\pgfpathlineto{\pgfqpoint{4.449282in}{2.371374in}}%
\pgfpathlineto{\pgfqpoint{4.457760in}{2.370438in}}%
\pgfpathlineto{\pgfqpoint{4.466323in}{2.367218in}}%
\pgfpathlineto{\pgfqpoint{4.475315in}{2.361393in}}%
\pgfpathlineto{\pgfqpoint{4.484991in}{2.352358in}}%
\pgfpathlineto{\pgfqpoint{4.495439in}{2.339442in}}%
\pgfpathlineto{\pgfqpoint{4.506914in}{2.321571in}}%
\pgfpathlineto{\pgfqpoint{4.519502in}{2.297689in}}%
\pgfpathlineto{\pgfqpoint{4.533460in}{2.266238in}}%
\pgfpathlineto{\pgfqpoint{4.548960in}{2.225605in}}%
\pgfpathlineto{\pgfqpoint{4.566344in}{2.173550in}}%
\pgfpathlineto{\pgfqpoint{4.586211in}{2.106742in}}%
\pgfpathlineto{\pgfqpoint{4.609589in}{2.019973in}}%
\pgfpathlineto{\pgfqpoint{4.639390in}{1.900234in}}%
\pgfpathlineto{\pgfqpoint{4.700019in}{1.644537in}}%
\pgfpathlineto{\pgfqpoint{4.740182in}{1.481012in}}%
\pgfpathlineto{\pgfqpoint{4.769982in}{1.368712in}}%
\pgfpathlineto{\pgfqpoint{4.796015in}{1.279004in}}%
\pgfpathlineto{\pgfqpoint{4.819821in}{1.204706in}}%
\pgfpathlineto{\pgfqpoint{4.842172in}{1.142042in}}%
\pgfpathlineto{\pgfqpoint{4.863495in}{1.088732in}}%
\pgfpathlineto{\pgfqpoint{4.883961in}{1.043409in}}%
\pgfpathlineto{\pgfqpoint{4.903829in}{1.004686in}}%
\pgfpathlineto{\pgfqpoint{4.923268in}{0.971555in}}%
\pgfpathlineto{\pgfqpoint{4.942450in}{0.943164in}}%
\pgfpathlineto{\pgfqpoint{4.961546in}{0.918807in}}%
\pgfpathlineto{\pgfqpoint{4.980728in}{0.897905in}}%
\pgfpathlineto{\pgfqpoint{5.000082in}{0.880064in}}%
\pgfpathlineto{\pgfqpoint{5.019949in}{0.864752in}}%
\pgfpathlineto{\pgfqpoint{5.040587in}{0.851654in}}%
\pgfpathlineto{\pgfqpoint{5.062252in}{0.840540in}}%
\pgfpathlineto{\pgfqpoint{5.085459in}{0.831144in}}%
\pgfpathlineto{\pgfqpoint{5.110892in}{0.823275in}}%
\pgfpathlineto{\pgfqpoint{5.139408in}{0.816827in}}%
\pgfpathlineto{\pgfqpoint{5.172463in}{0.811699in}}%
\pgfpathlineto{\pgfqpoint{5.212455in}{0.807829in}}%
\pgfpathlineto{\pgfqpoint{5.264520in}{0.805149in}}%
\pgfpathlineto{\pgfqpoint{5.302970in}{0.804150in}}%
\pgfpathlineto{\pgfqpoint{5.302970in}{0.804150in}}%
\pgfusepath{stroke}%
\end{pgfscope}%
\begin{pgfscope}%
\pgfsetrectcap%
\pgfsetmiterjoin%
\pgfsetlinewidth{1.003750pt}%
\definecolor{currentstroke}{rgb}{0.000000,0.000000,0.000000}%
\pgfsetstrokecolor{currentstroke}%
\pgfsetdash{}{0pt}%
\pgfpathmoveto{\pgfqpoint{3.590370in}{2.495000in}}%
\pgfpathlineto{\pgfqpoint{5.303056in}{2.495000in}}%
\pgfusepath{stroke}%
\end{pgfscope}%
\begin{pgfscope}%
\pgfsetrectcap%
\pgfsetmiterjoin%
\pgfsetlinewidth{1.003750pt}%
\definecolor{currentstroke}{rgb}{0.000000,0.000000,0.000000}%
\pgfsetstrokecolor{currentstroke}%
\pgfsetdash{}{0pt}%
\pgfpathmoveto{\pgfqpoint{5.303056in}{0.802500in}}%
\pgfpathlineto{\pgfqpoint{5.303056in}{2.495000in}}%
\pgfusepath{stroke}%
\end{pgfscope}%
\begin{pgfscope}%
\pgfsetrectcap%
\pgfsetmiterjoin%
\pgfsetlinewidth{1.003750pt}%
\definecolor{currentstroke}{rgb}{0.000000,0.000000,0.000000}%
\pgfsetstrokecolor{currentstroke}%
\pgfsetdash{}{0pt}%
\pgfpathmoveto{\pgfqpoint{3.590370in}{0.802500in}}%
\pgfpathlineto{\pgfqpoint{3.590370in}{2.495000in}}%
\pgfusepath{stroke}%
\end{pgfscope}%
\begin{pgfscope}%
\pgfsetrectcap%
\pgfsetmiterjoin%
\pgfsetlinewidth{1.003750pt}%
\definecolor{currentstroke}{rgb}{0.000000,0.000000,0.000000}%
\pgfsetstrokecolor{currentstroke}%
\pgfsetdash{}{0pt}%
\pgfpathmoveto{\pgfqpoint{3.590370in}{0.802500in}}%
\pgfpathlineto{\pgfqpoint{5.303056in}{0.802500in}}%
\pgfusepath{stroke}%
\end{pgfscope}%
\begin{pgfscope}%
\pgfsetbuttcap%
\pgfsetroundjoin%
\definecolor{currentfill}{rgb}{0.000000,0.000000,0.000000}%
\pgfsetfillcolor{currentfill}%
\pgfsetlinewidth{0.501875pt}%
\definecolor{currentstroke}{rgb}{0.000000,0.000000,0.000000}%
\pgfsetstrokecolor{currentstroke}%
\pgfsetdash{}{0pt}%
\pgfsys@defobject{currentmarker}{\pgfqpoint{0.000000in}{0.000000in}}{\pgfqpoint{0.000000in}{0.055556in}}{%
\pgfpathmoveto{\pgfqpoint{0.000000in}{0.000000in}}%
\pgfpathlineto{\pgfqpoint{0.000000in}{0.055556in}}%
\pgfusepath{stroke,fill}%
}%
\begin{pgfscope}%
\pgfsys@transformshift{3.590370in}{0.802500in}%
\pgfsys@useobject{currentmarker}{}%
\end{pgfscope}%
\end{pgfscope}%
\begin{pgfscope}%
\pgfsetbuttcap%
\pgfsetroundjoin%
\definecolor{currentfill}{rgb}{0.000000,0.000000,0.000000}%
\pgfsetfillcolor{currentfill}%
\pgfsetlinewidth{0.501875pt}%
\definecolor{currentstroke}{rgb}{0.000000,0.000000,0.000000}%
\pgfsetstrokecolor{currentstroke}%
\pgfsetdash{}{0pt}%
\pgfsys@defobject{currentmarker}{\pgfqpoint{0.000000in}{-0.055556in}}{\pgfqpoint{0.000000in}{0.000000in}}{%
\pgfpathmoveto{\pgfqpoint{0.000000in}{0.000000in}}%
\pgfpathlineto{\pgfqpoint{0.000000in}{-0.055556in}}%
\pgfusepath{stroke,fill}%
}%
\begin{pgfscope}%
\pgfsys@transformshift{3.590370in}{2.495000in}%
\pgfsys@useobject{currentmarker}{}%
\end{pgfscope}%
\end{pgfscope}%
\begin{pgfscope}%
\pgftext[x=3.590370in,y=0.746944in,,top]{\rmfamily\fontsize{11.000000}{13.200000}\selectfont -1.0}%
\end{pgfscope}%
\begin{pgfscope}%
\pgfsetbuttcap%
\pgfsetroundjoin%
\definecolor{currentfill}{rgb}{0.000000,0.000000,0.000000}%
\pgfsetfillcolor{currentfill}%
\pgfsetlinewidth{0.501875pt}%
\definecolor{currentstroke}{rgb}{0.000000,0.000000,0.000000}%
\pgfsetstrokecolor{currentstroke}%
\pgfsetdash{}{0pt}%
\pgfsys@defobject{currentmarker}{\pgfqpoint{0.000000in}{0.000000in}}{\pgfqpoint{0.000000in}{0.055556in}}{%
\pgfpathmoveto{\pgfqpoint{0.000000in}{0.000000in}}%
\pgfpathlineto{\pgfqpoint{0.000000in}{0.055556in}}%
\pgfusepath{stroke,fill}%
}%
\begin{pgfscope}%
\pgfsys@transformshift{4.018542in}{0.802500in}%
\pgfsys@useobject{currentmarker}{}%
\end{pgfscope}%
\end{pgfscope}%
\begin{pgfscope}%
\pgfsetbuttcap%
\pgfsetroundjoin%
\definecolor{currentfill}{rgb}{0.000000,0.000000,0.000000}%
\pgfsetfillcolor{currentfill}%
\pgfsetlinewidth{0.501875pt}%
\definecolor{currentstroke}{rgb}{0.000000,0.000000,0.000000}%
\pgfsetstrokecolor{currentstroke}%
\pgfsetdash{}{0pt}%
\pgfsys@defobject{currentmarker}{\pgfqpoint{0.000000in}{-0.055556in}}{\pgfqpoint{0.000000in}{0.000000in}}{%
\pgfpathmoveto{\pgfqpoint{0.000000in}{0.000000in}}%
\pgfpathlineto{\pgfqpoint{0.000000in}{-0.055556in}}%
\pgfusepath{stroke,fill}%
}%
\begin{pgfscope}%
\pgfsys@transformshift{4.018542in}{2.495000in}%
\pgfsys@useobject{currentmarker}{}%
\end{pgfscope}%
\end{pgfscope}%
\begin{pgfscope}%
\pgftext[x=4.018542in,y=0.746944in,,top]{\rmfamily\fontsize{11.000000}{13.200000}\selectfont -0.5}%
\end{pgfscope}%
\begin{pgfscope}%
\pgfsetbuttcap%
\pgfsetroundjoin%
\definecolor{currentfill}{rgb}{0.000000,0.000000,0.000000}%
\pgfsetfillcolor{currentfill}%
\pgfsetlinewidth{0.501875pt}%
\definecolor{currentstroke}{rgb}{0.000000,0.000000,0.000000}%
\pgfsetstrokecolor{currentstroke}%
\pgfsetdash{}{0pt}%
\pgfsys@defobject{currentmarker}{\pgfqpoint{0.000000in}{0.000000in}}{\pgfqpoint{0.000000in}{0.055556in}}{%
\pgfpathmoveto{\pgfqpoint{0.000000in}{0.000000in}}%
\pgfpathlineto{\pgfqpoint{0.000000in}{0.055556in}}%
\pgfusepath{stroke,fill}%
}%
\begin{pgfscope}%
\pgfsys@transformshift{4.446713in}{0.802500in}%
\pgfsys@useobject{currentmarker}{}%
\end{pgfscope}%
\end{pgfscope}%
\begin{pgfscope}%
\pgfsetbuttcap%
\pgfsetroundjoin%
\definecolor{currentfill}{rgb}{0.000000,0.000000,0.000000}%
\pgfsetfillcolor{currentfill}%
\pgfsetlinewidth{0.501875pt}%
\definecolor{currentstroke}{rgb}{0.000000,0.000000,0.000000}%
\pgfsetstrokecolor{currentstroke}%
\pgfsetdash{}{0pt}%
\pgfsys@defobject{currentmarker}{\pgfqpoint{0.000000in}{-0.055556in}}{\pgfqpoint{0.000000in}{0.000000in}}{%
\pgfpathmoveto{\pgfqpoint{0.000000in}{0.000000in}}%
\pgfpathlineto{\pgfqpoint{0.000000in}{-0.055556in}}%
\pgfusepath{stroke,fill}%
}%
\begin{pgfscope}%
\pgfsys@transformshift{4.446713in}{2.495000in}%
\pgfsys@useobject{currentmarker}{}%
\end{pgfscope}%
\end{pgfscope}%
\begin{pgfscope}%
\pgftext[x=4.446713in,y=0.746944in,,top]{\rmfamily\fontsize{11.000000}{13.200000}\selectfont 0.0}%
\end{pgfscope}%
\begin{pgfscope}%
\pgfsetbuttcap%
\pgfsetroundjoin%
\definecolor{currentfill}{rgb}{0.000000,0.000000,0.000000}%
\pgfsetfillcolor{currentfill}%
\pgfsetlinewidth{0.501875pt}%
\definecolor{currentstroke}{rgb}{0.000000,0.000000,0.000000}%
\pgfsetstrokecolor{currentstroke}%
\pgfsetdash{}{0pt}%
\pgfsys@defobject{currentmarker}{\pgfqpoint{0.000000in}{0.000000in}}{\pgfqpoint{0.000000in}{0.055556in}}{%
\pgfpathmoveto{\pgfqpoint{0.000000in}{0.000000in}}%
\pgfpathlineto{\pgfqpoint{0.000000in}{0.055556in}}%
\pgfusepath{stroke,fill}%
}%
\begin{pgfscope}%
\pgfsys@transformshift{4.874884in}{0.802500in}%
\pgfsys@useobject{currentmarker}{}%
\end{pgfscope}%
\end{pgfscope}%
\begin{pgfscope}%
\pgfsetbuttcap%
\pgfsetroundjoin%
\definecolor{currentfill}{rgb}{0.000000,0.000000,0.000000}%
\pgfsetfillcolor{currentfill}%
\pgfsetlinewidth{0.501875pt}%
\definecolor{currentstroke}{rgb}{0.000000,0.000000,0.000000}%
\pgfsetstrokecolor{currentstroke}%
\pgfsetdash{}{0pt}%
\pgfsys@defobject{currentmarker}{\pgfqpoint{0.000000in}{-0.055556in}}{\pgfqpoint{0.000000in}{0.000000in}}{%
\pgfpathmoveto{\pgfqpoint{0.000000in}{0.000000in}}%
\pgfpathlineto{\pgfqpoint{0.000000in}{-0.055556in}}%
\pgfusepath{stroke,fill}%
}%
\begin{pgfscope}%
\pgfsys@transformshift{4.874884in}{2.495000in}%
\pgfsys@useobject{currentmarker}{}%
\end{pgfscope}%
\end{pgfscope}%
\begin{pgfscope}%
\pgftext[x=4.874884in,y=0.746944in,,top]{\rmfamily\fontsize{11.000000}{13.200000}\selectfont 0.5}%
\end{pgfscope}%
\begin{pgfscope}%
\pgfsetbuttcap%
\pgfsetroundjoin%
\definecolor{currentfill}{rgb}{0.000000,0.000000,0.000000}%
\pgfsetfillcolor{currentfill}%
\pgfsetlinewidth{0.501875pt}%
\definecolor{currentstroke}{rgb}{0.000000,0.000000,0.000000}%
\pgfsetstrokecolor{currentstroke}%
\pgfsetdash{}{0pt}%
\pgfsys@defobject{currentmarker}{\pgfqpoint{0.000000in}{0.000000in}}{\pgfqpoint{0.000000in}{0.055556in}}{%
\pgfpathmoveto{\pgfqpoint{0.000000in}{0.000000in}}%
\pgfpathlineto{\pgfqpoint{0.000000in}{0.055556in}}%
\pgfusepath{stroke,fill}%
}%
\begin{pgfscope}%
\pgfsys@transformshift{5.303056in}{0.802500in}%
\pgfsys@useobject{currentmarker}{}%
\end{pgfscope}%
\end{pgfscope}%
\begin{pgfscope}%
\pgfsetbuttcap%
\pgfsetroundjoin%
\definecolor{currentfill}{rgb}{0.000000,0.000000,0.000000}%
\pgfsetfillcolor{currentfill}%
\pgfsetlinewidth{0.501875pt}%
\definecolor{currentstroke}{rgb}{0.000000,0.000000,0.000000}%
\pgfsetstrokecolor{currentstroke}%
\pgfsetdash{}{0pt}%
\pgfsys@defobject{currentmarker}{\pgfqpoint{0.000000in}{-0.055556in}}{\pgfqpoint{0.000000in}{0.000000in}}{%
\pgfpathmoveto{\pgfqpoint{0.000000in}{0.000000in}}%
\pgfpathlineto{\pgfqpoint{0.000000in}{-0.055556in}}%
\pgfusepath{stroke,fill}%
}%
\begin{pgfscope}%
\pgfsys@transformshift{5.303056in}{2.495000in}%
\pgfsys@useobject{currentmarker}{}%
\end{pgfscope}%
\end{pgfscope}%
\begin{pgfscope}%
\pgftext[x=5.303056in,y=0.746944in,,top]{\rmfamily\fontsize{11.000000}{13.200000}\selectfont 1.0}%
\end{pgfscope}%
\begin{pgfscope}%
\pgftext[x=3.806061in,y=0.445150in,left,base]{\rmfamily\fontsize{11.000000}{13.200000}\selectfont Pepperpot Position}%
\end{pgfscope}%
\begin{pgfscope}%
\pgftext[x=4.260833in,y=0.278704in,left,base]{\rmfamily\fontsize{11.000000}{13.200000}\selectfont (mm)}%
\end{pgfscope}%
\begin{pgfscope}%
\pgftext[x=5.372500in,y=1.648750in,,top,rotate=90.000000]{\rmfamily\fontsize{11.000000}{13.200000}\selectfont Electron Count}%
\end{pgfscope}%
\end{pgfpicture}%
\makeatother%
\endgroup%

    \caption[Determining the beam profile from a pepperpot measurement.]{On the left is a pepperpot image from a below-threshold low-emittance bunch (blue is \unit[486.9]{nm}). In the centre is a column sum of the pixels in the first image for which the amplitude and standard deviation are calculated for each of the seven peaks. In the right image the black line indicates the shape of the pepperpot in one-dimension, the blue crosses indicate the total number of electrons in each of the seven, column-summed, beamlets and the red dotted line indicates a Gaussian fitted to the beamlet electron counts and represents the full profile of the electron bunch incident on the pepperpots.}
    \label{figure:pepperpot_profile}
    % Data and code at 2017.06.06
\end{figure}

\subsubsection{Ionisation Laser}
The ionisation laser was a \unit[457-493]{nm}, \unit[5]{ns} pulse laser\footnote{Sirah dye laser system CSTR-D-3000.} which ionises the atoms excited by the red excitation laser.
The wavelength can be tuned to select various ionisation pathways such as above-threshold ionisation or field ionisation.
The pathway selected affects the duration of the bunch produced, above threshold ionisation results in short bunches with durations the same as the duration of the ionisation laser and below threshold ionisation can result in much longer bunches with durations of \unit[10s]{$\muup$s} due to electrons tunnelling out after the end of the ionisation pulse.

Another useful impact of the ionisation lasers wavelength is on the excess energy of the electrons and thus the emittance of the electron bunches.
This is discussed in Sections~\ref{section:two_stage_ionisation} and \ref{section:excess_energy_emittance}.
Equation~\ref{equation:excess_energy_emittance} gives a useful comparison and allows for verification of the accuracy of the emittance measurements.

\section{Simulations}\label{section:pepperpot_simulations}

Due to the experimental constraints involving the low beam flux, pepperpot mask size and beamlet overlap it was not possible to engineer an idea setup of beam and pepperpot parameters.
The non-ideal experimental parameters cause some distortions to the emittance measurements so simulations were performed to explore the effects and verify the validity of the corrections.
These simulation also proved to be a useful confirmation of the analysis procedure used on the experimental data.

The simulations were performed by a simple homemade particle tracker, able to operate in one- and two-dimensions, create arbitrary bunch profiles with arbitrary emittances, propagate bunches and apply arbitrary masks to the bunch.

For these simulations a Gaussian bunch profile was used and the emittance of the bunch was set to a range of values similar to those achievable with the actual apparatus.
The path of the bunch was designed to replicate that of the \gls{caeis} by propagating the bunch to a `lens', then to a pepperpot mask and finally to a `detector'.
The `lens' was implemented by applying a radially dependent velocity change to the particles in order to focus the beam such that the beam size was approximately the same as the size of the pepperpot mask, approximately \unit[2]{mm}.

\subsection{Pepperpot Aperture Size}

The first set of simulations investigated the effects of aperture size on the emittance measurement.
The procedure outline in Reference~\cite{zhang_emittance_1996}, which formed the foundation of the analysis used here, assumes that the size of the apertures used is negligible compared to the divergence of the particles.
Due to the low electron flux it is not feasible to use pepperpots with negligibly small holes so it is necessary to understand and correct for finite aperture size.

The profile of a beamlet is a convolution of the aperture profile and the beamlet profile from an infinitesimally small aperture.
Thus, when the aperture size is large compared to the effects of the beam divergence, it will be difficult to determine the divergence whereas if the divergence affect is similar or large compared to the aperture size then the divergence will be easier to determine.
This results in an overestimation of beam divergence when the divergence and aperture size are similar and a minimum measurable divergence, and thus emittance, if the divergence is much smaller than the aperture size.
An example of a resolution limit for this experiment is shown in Figure~\ref{figure:wavelength_sim}.

The correction for finite apertures, where the aperture is not too large compared to the effect of the divergence, is to deconvolve the aperture size from the beamlet profile which is most easily implemented by fitting the width of a Gaussian convolved with the aperture size to the known beamlet size, taking magnification into account.
Reapplying the magnification to the result of the fit supplies the profile of beamlets had the apertures been infinitesimally small.

The results of the analysis of simulations that vary the size of the apertures is shown in Figure~\ref{figure:aperture_size_sim}, the improvement in the accuracy of the analysis when the aperture size correction is applied is evident.

\begin{figure}
\begin{subfigure}[b]{\textwidth}
    \center
    %% Creator: Matplotlib, PGF backend
%%
%% To include the figure in your LaTeX document, write
%%   \input{<filename>.pgf}
%%
%% Make sure the required packages are loaded in your preamble
%%   \usepackage{pgf}
%%
%% Figures using additional raster images can only be included by \input if
%% they are in the same directory as the main LaTeX file. For loading figures
%% from other directories you can use the `import` package
%%   \usepackage{import}
%% and then include the figures with
%%   \import{<path to file>}{<filename>.pgf}
%%
%% Matplotlib used the following preamble
%%
\begingroup%
\makeatletter%
\begin{pgfpicture}%
\pgfpathrectangle{\pgfpointorigin}{\pgfqpoint{4.282500in}{1.903333in}}%
\pgfusepath{use as bounding box, clip}%
\begin{pgfscope}%
\pgfsetbuttcap%
\pgfsetmiterjoin%
\definecolor{currentfill}{rgb}{1.000000,1.000000,1.000000}%
\pgfsetfillcolor{currentfill}%
\pgfsetlinewidth{0.000000pt}%
\definecolor{currentstroke}{rgb}{1.000000,1.000000,1.000000}%
\pgfsetstrokecolor{currentstroke}%
\pgfsetdash{}{0pt}%
\pgfpathmoveto{\pgfqpoint{0.000000in}{0.000000in}}%
\pgfpathlineto{\pgfqpoint{4.282500in}{0.000000in}}%
\pgfpathlineto{\pgfqpoint{4.282500in}{1.903333in}}%
\pgfpathlineto{\pgfqpoint{0.000000in}{1.903333in}}%
\pgfpathclose%
\pgfusepath{fill}%
\end{pgfscope}%
\begin{pgfscope}%
\pgfsetbuttcap%
\pgfsetmiterjoin%
\definecolor{currentfill}{rgb}{1.000000,1.000000,1.000000}%
\pgfsetfillcolor{currentfill}%
\pgfsetlinewidth{0.000000pt}%
\definecolor{currentstroke}{rgb}{0.000000,0.000000,0.000000}%
\pgfsetstrokecolor{currentstroke}%
\pgfsetstrokeopacity{0.000000}%
\pgfsetdash{}{0pt}%
\pgfpathmoveto{\pgfqpoint{0.671094in}{0.532031in}}%
\pgfpathlineto{\pgfqpoint{4.132500in}{0.532031in}}%
\pgfpathlineto{\pgfqpoint{4.132500in}{1.690833in}}%
\pgfpathlineto{\pgfqpoint{0.671094in}{1.690833in}}%
\pgfpathclose%
\pgfusepath{fill}%
\end{pgfscope}%
\begin{pgfscope}%
\pgfpathrectangle{\pgfqpoint{0.671094in}{0.532031in}}{\pgfqpoint{3.461406in}{1.158802in}} %
\pgfusepath{clip}%
\pgfsetrectcap%
\pgfsetroundjoin%
\pgfsetlinewidth{1.003750pt}%
\definecolor{currentstroke}{rgb}{0.309804,0.478431,0.682353}%
\pgfsetstrokecolor{currentstroke}%
\pgfsetdash{}{0pt}%
\pgfpathmoveto{\pgfqpoint{4.132500in}{1.572808in}}%
\pgfpathlineto{\pgfqpoint{4.079544in}{1.563053in}}%
\pgfpathlineto{\pgfqpoint{4.026639in}{1.555222in}}%
\pgfpathlineto{\pgfqpoint{3.973784in}{1.549966in}}%
\pgfpathlineto{\pgfqpoint{3.920980in}{1.544591in}}%
\pgfpathlineto{\pgfqpoint{3.868226in}{1.536748in}}%
\pgfpathlineto{\pgfqpoint{3.815522in}{1.528705in}}%
\pgfpathlineto{\pgfqpoint{3.762869in}{1.523376in}}%
\pgfpathlineto{\pgfqpoint{3.710265in}{1.515443in}}%
\pgfpathlineto{\pgfqpoint{3.657712in}{1.509306in}}%
\pgfpathlineto{\pgfqpoint{3.605208in}{1.500841in}}%
\pgfpathlineto{\pgfqpoint{3.552755in}{1.491874in}}%
\pgfpathlineto{\pgfqpoint{3.500352in}{1.485271in}}%
\pgfpathlineto{\pgfqpoint{3.447998in}{1.478126in}}%
\pgfpathlineto{\pgfqpoint{3.395694in}{1.471138in}}%
\pgfpathlineto{\pgfqpoint{3.343439in}{1.462425in}}%
\pgfpathlineto{\pgfqpoint{3.291234in}{1.455190in}}%
\pgfpathlineto{\pgfqpoint{3.239079in}{1.448289in}}%
\pgfpathlineto{\pgfqpoint{3.186973in}{1.441197in}}%
\pgfpathlineto{\pgfqpoint{3.134916in}{1.431944in}}%
\pgfpathlineto{\pgfqpoint{3.082909in}{1.424049in}}%
\pgfpathlineto{\pgfqpoint{3.030951in}{1.416029in}}%
\pgfpathlineto{\pgfqpoint{2.979042in}{1.408000in}}%
\pgfpathlineto{\pgfqpoint{2.927182in}{1.400593in}}%
\pgfpathlineto{\pgfqpoint{2.875371in}{1.390134in}}%
\pgfpathlineto{\pgfqpoint{2.823609in}{1.382241in}}%
\pgfpathlineto{\pgfqpoint{2.771896in}{1.373495in}}%
\pgfpathlineto{\pgfqpoint{2.720232in}{1.365232in}}%
\pgfpathlineto{\pgfqpoint{2.668616in}{1.356978in}}%
\pgfpathlineto{\pgfqpoint{2.617050in}{1.346558in}}%
\pgfpathlineto{\pgfqpoint{2.565531in}{1.338581in}}%
\pgfpathlineto{\pgfqpoint{2.514061in}{1.330259in}}%
\pgfpathlineto{\pgfqpoint{2.462640in}{1.320986in}}%
\pgfpathlineto{\pgfqpoint{2.411267in}{1.313835in}}%
\pgfpathlineto{\pgfqpoint{2.359943in}{1.305337in}}%
\pgfpathlineto{\pgfqpoint{2.308666in}{1.296492in}}%
\pgfpathlineto{\pgfqpoint{2.257438in}{1.286557in}}%
\pgfpathlineto{\pgfqpoint{2.206258in}{1.273871in}}%
\pgfpathlineto{\pgfqpoint{2.155126in}{1.263242in}}%
\pgfpathlineto{\pgfqpoint{2.104042in}{1.257911in}}%
\pgfpathlineto{\pgfqpoint{2.053006in}{1.248037in}}%
\pgfpathlineto{\pgfqpoint{2.002017in}{1.234588in}}%
\pgfpathlineto{\pgfqpoint{1.951077in}{1.227283in}}%
\pgfpathlineto{\pgfqpoint{1.900184in}{1.214456in}}%
\pgfpathlineto{\pgfqpoint{1.849339in}{1.202873in}}%
\pgfpathlineto{\pgfqpoint{1.798541in}{1.192909in}}%
\pgfpathlineto{\pgfqpoint{1.747791in}{1.183135in}}%
\pgfpathlineto{\pgfqpoint{1.697088in}{1.172500in}}%
\pgfpathlineto{\pgfqpoint{1.646433in}{1.162431in}}%
\pgfpathlineto{\pgfqpoint{1.595825in}{1.148248in}}%
\pgfpathlineto{\pgfqpoint{1.545265in}{1.135589in}}%
\pgfpathlineto{\pgfqpoint{1.494751in}{1.123289in}}%
\pgfpathlineto{\pgfqpoint{1.444284in}{1.120770in}}%
\pgfpathlineto{\pgfqpoint{1.393865in}{1.098463in}}%
\pgfpathlineto{\pgfqpoint{1.343493in}{1.098684in}}%
\pgfpathlineto{\pgfqpoint{1.293167in}{1.085858in}}%
\pgfpathlineto{\pgfqpoint{1.242889in}{1.079579in}}%
\pgfpathlineto{\pgfqpoint{1.192657in}{1.058107in}}%
\pgfpathlineto{\pgfqpoint{1.142471in}{1.039402in}}%
\pgfpathlineto{\pgfqpoint{1.092333in}{1.025191in}}%
\pgfpathlineto{\pgfqpoint{1.042241in}{1.021890in}}%
\pgfpathlineto{\pgfqpoint{0.992196in}{0.991500in}}%
\pgfpathlineto{\pgfqpoint{0.942197in}{0.970899in}}%
\pgfpathlineto{\pgfqpoint{0.892244in}{0.954807in}}%
\pgfpathlineto{\pgfqpoint{0.842338in}{0.961244in}}%
\pgfpathlineto{\pgfqpoint{0.792478in}{0.954895in}}%
\pgfpathlineto{\pgfqpoint{0.742664in}{0.950153in}}%
\pgfpathlineto{\pgfqpoint{0.692897in}{0.964516in}}%
\pgfpathlineto{\pgfqpoint{0.661094in}{0.959698in}}%
\pgfusepath{stroke}%
\end{pgfscope}%
\begin{pgfscope}%
\pgfpathrectangle{\pgfqpoint{0.671094in}{0.532031in}}{\pgfqpoint{3.461406in}{1.158802in}} %
\pgfusepath{clip}%
\pgfsetrectcap%
\pgfsetroundjoin%
\pgfsetlinewidth{1.003750pt}%
\definecolor{currentstroke}{rgb}{0.301961,0.607843,0.301961}%
\pgfsetstrokecolor{currentstroke}%
\pgfsetdash{}{0pt}%
\pgfpathmoveto{\pgfqpoint{4.132500in}{1.494160in}}%
\pgfpathlineto{\pgfqpoint{4.079544in}{1.483861in}}%
\pgfpathlineto{\pgfqpoint{4.026639in}{1.474145in}}%
\pgfpathlineto{\pgfqpoint{3.973784in}{1.468049in}}%
\pgfpathlineto{\pgfqpoint{3.920980in}{1.463665in}}%
\pgfpathlineto{\pgfqpoint{3.868226in}{1.453152in}}%
\pgfpathlineto{\pgfqpoint{3.815522in}{1.443988in}}%
\pgfpathlineto{\pgfqpoint{3.762869in}{1.439574in}}%
\pgfpathlineto{\pgfqpoint{3.710265in}{1.428902in}}%
\pgfpathlineto{\pgfqpoint{3.657712in}{1.421549in}}%
\pgfpathlineto{\pgfqpoint{3.605208in}{1.413720in}}%
\pgfpathlineto{\pgfqpoint{3.552755in}{1.403557in}}%
\pgfpathlineto{\pgfqpoint{3.500352in}{1.395136in}}%
\pgfpathlineto{\pgfqpoint{3.447998in}{1.387057in}}%
\pgfpathlineto{\pgfqpoint{3.395694in}{1.378390in}}%
\pgfpathlineto{\pgfqpoint{3.343439in}{1.368992in}}%
\pgfpathlineto{\pgfqpoint{3.291234in}{1.357909in}}%
\pgfpathlineto{\pgfqpoint{3.239079in}{1.351490in}}%
\pgfpathlineto{\pgfqpoint{3.186973in}{1.344455in}}%
\pgfpathlineto{\pgfqpoint{3.134916in}{1.330445in}}%
\pgfpathlineto{\pgfqpoint{3.082909in}{1.323622in}}%
\pgfpathlineto{\pgfqpoint{3.030951in}{1.313931in}}%
\pgfpathlineto{\pgfqpoint{2.979042in}{1.303277in}}%
\pgfpathlineto{\pgfqpoint{2.927182in}{1.295522in}}%
\pgfpathlineto{\pgfqpoint{2.875371in}{1.281822in}}%
\pgfpathlineto{\pgfqpoint{2.823609in}{1.270976in}}%
\pgfpathlineto{\pgfqpoint{2.771896in}{1.262346in}}%
\pgfpathlineto{\pgfqpoint{2.720232in}{1.250860in}}%
\pgfpathlineto{\pgfqpoint{2.668616in}{1.240517in}}%
\pgfpathlineto{\pgfqpoint{2.617050in}{1.227736in}}%
\pgfpathlineto{\pgfqpoint{2.565531in}{1.216407in}}%
\pgfpathlineto{\pgfqpoint{2.514061in}{1.207491in}}%
\pgfpathlineto{\pgfqpoint{2.462640in}{1.196444in}}%
\pgfpathlineto{\pgfqpoint{2.411267in}{1.186887in}}%
\pgfpathlineto{\pgfqpoint{2.359943in}{1.173410in}}%
\pgfpathlineto{\pgfqpoint{2.308666in}{1.163831in}}%
\pgfpathlineto{\pgfqpoint{2.257438in}{1.151419in}}%
\pgfpathlineto{\pgfqpoint{2.206258in}{1.133922in}}%
\pgfpathlineto{\pgfqpoint{2.155126in}{1.120916in}}%
\pgfpathlineto{\pgfqpoint{2.104042in}{1.112123in}}%
\pgfpathlineto{\pgfqpoint{2.053006in}{1.097154in}}%
\pgfpathlineto{\pgfqpoint{2.002017in}{1.077250in}}%
\pgfpathlineto{\pgfqpoint{1.951077in}{1.066018in}}%
\pgfpathlineto{\pgfqpoint{1.900184in}{1.047788in}}%
\pgfpathlineto{\pgfqpoint{1.849339in}{1.033769in}}%
\pgfpathlineto{\pgfqpoint{1.798541in}{1.018070in}}%
\pgfpathlineto{\pgfqpoint{1.747791in}{1.001612in}}%
\pgfpathlineto{\pgfqpoint{1.697088in}{0.985102in}}%
\pgfpathlineto{\pgfqpoint{1.646433in}{0.972684in}}%
\pgfpathlineto{\pgfqpoint{1.595825in}{0.949582in}}%
\pgfpathlineto{\pgfqpoint{1.545265in}{0.933265in}}%
\pgfpathlineto{\pgfqpoint{1.494751in}{0.915381in}}%
\pgfpathlineto{\pgfqpoint{1.444284in}{0.913523in}}%
\pgfpathlineto{\pgfqpoint{1.393865in}{0.888073in}}%
\pgfpathlineto{\pgfqpoint{1.343493in}{0.890129in}}%
\pgfpathlineto{\pgfqpoint{1.293167in}{0.878809in}}%
\pgfpathlineto{\pgfqpoint{1.242889in}{0.873216in}}%
\pgfpathlineto{\pgfqpoint{1.192657in}{0.858760in}}%
\pgfpathlineto{\pgfqpoint{1.142471in}{0.849374in}}%
\pgfpathlineto{\pgfqpoint{1.092333in}{0.843929in}}%
\pgfpathlineto{\pgfqpoint{1.042241in}{0.842974in}}%
\pgfpathlineto{\pgfqpoint{0.992196in}{0.834707in}}%
\pgfpathlineto{\pgfqpoint{0.942197in}{0.832260in}}%
\pgfpathlineto{\pgfqpoint{0.892244in}{0.829757in}}%
\pgfpathlineto{\pgfqpoint{0.842338in}{0.830128in}}%
\pgfpathlineto{\pgfqpoint{0.792478in}{0.830892in}}%
\pgfpathlineto{\pgfqpoint{0.742664in}{0.829833in}}%
\pgfpathlineto{\pgfqpoint{0.692897in}{0.831558in}}%
\pgfpathlineto{\pgfqpoint{0.661094in}{0.830740in}}%
\pgfusepath{stroke}%
\end{pgfscope}%
\begin{pgfscope}%
\pgfpathrectangle{\pgfqpoint{0.671094in}{0.532031in}}{\pgfqpoint{3.461406in}{1.158802in}} %
\pgfusepath{clip}%
\pgfsetbuttcap%
\pgfsetroundjoin%
\pgfsetlinewidth{1.003750pt}%
\definecolor{currentstroke}{rgb}{1.000000,0.400000,0.200000}%
\pgfsetstrokecolor{currentstroke}%
\pgfsetdash{{1.000000pt}{3.000000pt}}{0.000000pt}%
\pgfpathmoveto{\pgfqpoint{4.132500in}{1.495633in}}%
\pgfpathlineto{\pgfqpoint{3.825954in}{1.449115in}}%
\pgfpathlineto{\pgfqpoint{3.541336in}{1.403705in}}%
\pgfpathlineto{\pgfqpoint{3.277803in}{1.359440in}}%
\pgfpathlineto{\pgfqpoint{3.034036in}{1.316273in}}%
\pgfpathlineto{\pgfqpoint{2.808756in}{1.274151in}}%
\pgfpathlineto{\pgfqpoint{2.601742in}{1.233217in}}%
\pgfpathlineto{\pgfqpoint{2.411267in}{1.193319in}}%
\pgfpathlineto{\pgfqpoint{2.236656in}{1.154501in}}%
\pgfpathlineto{\pgfqpoint{2.077255in}{1.116819in}}%
\pgfpathlineto{\pgfqpoint{1.931923in}{1.080210in}}%
\pgfpathlineto{\pgfqpoint{1.800049in}{1.044734in}}%
\pgfpathlineto{\pgfqpoint{1.680533in}{1.010315in}}%
\pgfpathlineto{\pgfqpoint{1.572792in}{0.977010in}}%
\pgfpathlineto{\pgfqpoint{1.475758in}{0.944722in}}%
\pgfpathlineto{\pgfqpoint{1.388876in}{0.913500in}}%
\pgfpathlineto{\pgfqpoint{1.311598in}{0.883405in}}%
\pgfpathlineto{\pgfqpoint{1.242889in}{0.854296in}}%
\pgfpathlineto{\pgfqpoint{1.182218in}{0.826207in}}%
\pgfpathlineto{\pgfqpoint{1.129064in}{0.799182in}}%
\pgfpathlineto{\pgfqpoint{1.082906in}{0.773271in}}%
\pgfpathlineto{\pgfqpoint{1.042737in}{0.748208in}}%
\pgfpathlineto{\pgfqpoint{1.008542in}{0.724317in}}%
\pgfpathlineto{\pgfqpoint{0.979320in}{0.701247in}}%
\pgfpathlineto{\pgfqpoint{0.954568in}{0.678898in}}%
\pgfpathlineto{\pgfqpoint{0.934280in}{0.657649in}}%
\pgfpathlineto{\pgfqpoint{0.917956in}{0.637488in}}%
\pgfpathlineto{\pgfqpoint{0.904604in}{0.617533in}}%
\pgfpathlineto{\pgfqpoint{0.894222in}{0.597966in}}%
\pgfpathlineto{\pgfqpoint{0.886807in}{0.579267in}}%
\pgfpathlineto{\pgfqpoint{0.881864in}{0.560677in}}%
\pgfpathlineto{\pgfqpoint{0.879393in}{0.542770in}}%
\pgfpathlineto{\pgfqpoint{0.877910in}{0.532031in}}%
\pgfpathlineto{\pgfqpoint{0.661094in}{0.532031in}}%
\pgfpathlineto{\pgfqpoint{0.661094in}{0.532031in}}%
\pgfusepath{stroke}%
\end{pgfscope}%
\begin{pgfscope}%
\pgfsetrectcap%
\pgfsetmiterjoin%
\pgfsetlinewidth{1.003750pt}%
\definecolor{currentstroke}{rgb}{0.000000,0.000000,0.000000}%
\pgfsetstrokecolor{currentstroke}%
\pgfsetdash{}{0pt}%
\pgfpathmoveto{\pgfqpoint{0.671094in}{0.532031in}}%
\pgfpathlineto{\pgfqpoint{0.671094in}{1.690833in}}%
\pgfusepath{stroke}%
\end{pgfscope}%
\begin{pgfscope}%
\pgfsetrectcap%
\pgfsetmiterjoin%
\pgfsetlinewidth{1.003750pt}%
\definecolor{currentstroke}{rgb}{0.000000,0.000000,0.000000}%
\pgfsetstrokecolor{currentstroke}%
\pgfsetdash{}{0pt}%
\pgfpathmoveto{\pgfqpoint{0.671094in}{0.532031in}}%
\pgfpathlineto{\pgfqpoint{4.132500in}{0.532031in}}%
\pgfusepath{stroke}%
\end{pgfscope}%
\begin{pgfscope}%
\pgfsetrectcap%
\pgfsetmiterjoin%
\pgfsetlinewidth{1.003750pt}%
\definecolor{currentstroke}{rgb}{0.000000,0.000000,0.000000}%
\pgfsetstrokecolor{currentstroke}%
\pgfsetdash{}{0pt}%
\pgfpathmoveto{\pgfqpoint{0.671094in}{1.690833in}}%
\pgfpathlineto{\pgfqpoint{4.132500in}{1.690833in}}%
\pgfusepath{stroke}%
\end{pgfscope}%
\begin{pgfscope}%
\pgfsetrectcap%
\pgfsetmiterjoin%
\pgfsetlinewidth{1.003750pt}%
\definecolor{currentstroke}{rgb}{0.000000,0.000000,0.000000}%
\pgfsetstrokecolor{currentstroke}%
\pgfsetdash{}{0pt}%
\pgfpathmoveto{\pgfqpoint{4.132500in}{0.532031in}}%
\pgfpathlineto{\pgfqpoint{4.132500in}{1.690833in}}%
\pgfusepath{stroke}%
\end{pgfscope}%
\begin{pgfscope}%
\pgfsetbuttcap%
\pgfsetroundjoin%
\definecolor{currentfill}{rgb}{0.000000,0.000000,0.000000}%
\pgfsetfillcolor{currentfill}%
\pgfsetlinewidth{0.501875pt}%
\definecolor{currentstroke}{rgb}{0.000000,0.000000,0.000000}%
\pgfsetstrokecolor{currentstroke}%
\pgfsetdash{}{0pt}%
\pgfsys@defobject{currentmarker}{\pgfqpoint{0.000000in}{0.000000in}}{\pgfqpoint{0.000000in}{0.055556in}}{%
\pgfpathmoveto{\pgfqpoint{0.000000in}{0.000000in}}%
\pgfpathlineto{\pgfqpoint{0.000000in}{0.055556in}}%
\pgfusepath{stroke,fill}%
}%
\begin{pgfscope}%
\pgfsys@transformshift{0.878988in}{0.532031in}%
\pgfsys@useobject{currentmarker}{}%
\end{pgfscope}%
\end{pgfscope}%
\begin{pgfscope}%
\pgfsetbuttcap%
\pgfsetroundjoin%
\definecolor{currentfill}{rgb}{0.000000,0.000000,0.000000}%
\pgfsetfillcolor{currentfill}%
\pgfsetlinewidth{0.501875pt}%
\definecolor{currentstroke}{rgb}{0.000000,0.000000,0.000000}%
\pgfsetstrokecolor{currentstroke}%
\pgfsetdash{}{0pt}%
\pgfsys@defobject{currentmarker}{\pgfqpoint{0.000000in}{-0.055556in}}{\pgfqpoint{0.000000in}{0.000000in}}{%
\pgfpathmoveto{\pgfqpoint{0.000000in}{0.000000in}}%
\pgfpathlineto{\pgfqpoint{0.000000in}{-0.055556in}}%
\pgfusepath{stroke,fill}%
}%
\begin{pgfscope}%
\pgfsys@transformshift{0.878988in}{1.690833in}%
\pgfsys@useobject{currentmarker}{}%
\end{pgfscope}%
\end{pgfscope}%
\begin{pgfscope}%
\pgftext[x=0.878988in,y=0.476476in,,top]{\rmfamily\fontsize{10.000000}{12.000000}\selectfont 0}%
\end{pgfscope}%
\begin{pgfscope}%
\pgfsetbuttcap%
\pgfsetroundjoin%
\definecolor{currentfill}{rgb}{0.000000,0.000000,0.000000}%
\pgfsetfillcolor{currentfill}%
\pgfsetlinewidth{0.501875pt}%
\definecolor{currentstroke}{rgb}{0.000000,0.000000,0.000000}%
\pgfsetstrokecolor{currentstroke}%
\pgfsetdash{}{0pt}%
\pgfsys@defobject{currentmarker}{\pgfqpoint{0.000000in}{0.000000in}}{\pgfqpoint{0.000000in}{0.055556in}}{%
\pgfpathmoveto{\pgfqpoint{0.000000in}{0.000000in}}%
\pgfpathlineto{\pgfqpoint{0.000000in}{0.055556in}}%
\pgfusepath{stroke,fill}%
}%
\begin{pgfscope}%
\pgfsys@transformshift{1.294778in}{0.532031in}%
\pgfsys@useobject{currentmarker}{}%
\end{pgfscope}%
\end{pgfscope}%
\begin{pgfscope}%
\pgfsetbuttcap%
\pgfsetroundjoin%
\definecolor{currentfill}{rgb}{0.000000,0.000000,0.000000}%
\pgfsetfillcolor{currentfill}%
\pgfsetlinewidth{0.501875pt}%
\definecolor{currentstroke}{rgb}{0.000000,0.000000,0.000000}%
\pgfsetstrokecolor{currentstroke}%
\pgfsetdash{}{0pt}%
\pgfsys@defobject{currentmarker}{\pgfqpoint{0.000000in}{-0.055556in}}{\pgfqpoint{0.000000in}{0.000000in}}{%
\pgfpathmoveto{\pgfqpoint{0.000000in}{0.000000in}}%
\pgfpathlineto{\pgfqpoint{0.000000in}{-0.055556in}}%
\pgfusepath{stroke,fill}%
}%
\begin{pgfscope}%
\pgfsys@transformshift{1.294778in}{1.690833in}%
\pgfsys@useobject{currentmarker}{}%
\end{pgfscope}%
\end{pgfscope}%
\begin{pgfscope}%
\pgftext[x=1.294778in,y=0.476476in,,top]{\rmfamily\fontsize{10.000000}{12.000000}\selectfont 10}%
\end{pgfscope}%
\begin{pgfscope}%
\pgfsetbuttcap%
\pgfsetroundjoin%
\definecolor{currentfill}{rgb}{0.000000,0.000000,0.000000}%
\pgfsetfillcolor{currentfill}%
\pgfsetlinewidth{0.501875pt}%
\definecolor{currentstroke}{rgb}{0.000000,0.000000,0.000000}%
\pgfsetstrokecolor{currentstroke}%
\pgfsetdash{}{0pt}%
\pgfsys@defobject{currentmarker}{\pgfqpoint{0.000000in}{0.000000in}}{\pgfqpoint{0.000000in}{0.055556in}}{%
\pgfpathmoveto{\pgfqpoint{0.000000in}{0.000000in}}%
\pgfpathlineto{\pgfqpoint{0.000000in}{0.055556in}}%
\pgfusepath{stroke,fill}%
}%
\begin{pgfscope}%
\pgfsys@transformshift{1.710567in}{0.532031in}%
\pgfsys@useobject{currentmarker}{}%
\end{pgfscope}%
\end{pgfscope}%
\begin{pgfscope}%
\pgfsetbuttcap%
\pgfsetroundjoin%
\definecolor{currentfill}{rgb}{0.000000,0.000000,0.000000}%
\pgfsetfillcolor{currentfill}%
\pgfsetlinewidth{0.501875pt}%
\definecolor{currentstroke}{rgb}{0.000000,0.000000,0.000000}%
\pgfsetstrokecolor{currentstroke}%
\pgfsetdash{}{0pt}%
\pgfsys@defobject{currentmarker}{\pgfqpoint{0.000000in}{-0.055556in}}{\pgfqpoint{0.000000in}{0.000000in}}{%
\pgfpathmoveto{\pgfqpoint{0.000000in}{0.000000in}}%
\pgfpathlineto{\pgfqpoint{0.000000in}{-0.055556in}}%
\pgfusepath{stroke,fill}%
}%
\begin{pgfscope}%
\pgfsys@transformshift{1.710567in}{1.690833in}%
\pgfsys@useobject{currentmarker}{}%
\end{pgfscope}%
\end{pgfscope}%
\begin{pgfscope}%
\pgftext[x=1.710567in,y=0.476476in,,top]{\rmfamily\fontsize{10.000000}{12.000000}\selectfont 20}%
\end{pgfscope}%
\begin{pgfscope}%
\pgfsetbuttcap%
\pgfsetroundjoin%
\definecolor{currentfill}{rgb}{0.000000,0.000000,0.000000}%
\pgfsetfillcolor{currentfill}%
\pgfsetlinewidth{0.501875pt}%
\definecolor{currentstroke}{rgb}{0.000000,0.000000,0.000000}%
\pgfsetstrokecolor{currentstroke}%
\pgfsetdash{}{0pt}%
\pgfsys@defobject{currentmarker}{\pgfqpoint{0.000000in}{0.000000in}}{\pgfqpoint{0.000000in}{0.055556in}}{%
\pgfpathmoveto{\pgfqpoint{0.000000in}{0.000000in}}%
\pgfpathlineto{\pgfqpoint{0.000000in}{0.055556in}}%
\pgfusepath{stroke,fill}%
}%
\begin{pgfscope}%
\pgfsys@transformshift{2.126357in}{0.532031in}%
\pgfsys@useobject{currentmarker}{}%
\end{pgfscope}%
\end{pgfscope}%
\begin{pgfscope}%
\pgfsetbuttcap%
\pgfsetroundjoin%
\definecolor{currentfill}{rgb}{0.000000,0.000000,0.000000}%
\pgfsetfillcolor{currentfill}%
\pgfsetlinewidth{0.501875pt}%
\definecolor{currentstroke}{rgb}{0.000000,0.000000,0.000000}%
\pgfsetstrokecolor{currentstroke}%
\pgfsetdash{}{0pt}%
\pgfsys@defobject{currentmarker}{\pgfqpoint{0.000000in}{-0.055556in}}{\pgfqpoint{0.000000in}{0.000000in}}{%
\pgfpathmoveto{\pgfqpoint{0.000000in}{0.000000in}}%
\pgfpathlineto{\pgfqpoint{0.000000in}{-0.055556in}}%
\pgfusepath{stroke,fill}%
}%
\begin{pgfscope}%
\pgfsys@transformshift{2.126357in}{1.690833in}%
\pgfsys@useobject{currentmarker}{}%
\end{pgfscope}%
\end{pgfscope}%
\begin{pgfscope}%
\pgftext[x=2.126357in,y=0.476476in,,top]{\rmfamily\fontsize{10.000000}{12.000000}\selectfont 30}%
\end{pgfscope}%
\begin{pgfscope}%
\pgfsetbuttcap%
\pgfsetroundjoin%
\definecolor{currentfill}{rgb}{0.000000,0.000000,0.000000}%
\pgfsetfillcolor{currentfill}%
\pgfsetlinewidth{0.501875pt}%
\definecolor{currentstroke}{rgb}{0.000000,0.000000,0.000000}%
\pgfsetstrokecolor{currentstroke}%
\pgfsetdash{}{0pt}%
\pgfsys@defobject{currentmarker}{\pgfqpoint{0.000000in}{0.000000in}}{\pgfqpoint{0.000000in}{0.055556in}}{%
\pgfpathmoveto{\pgfqpoint{0.000000in}{0.000000in}}%
\pgfpathlineto{\pgfqpoint{0.000000in}{0.055556in}}%
\pgfusepath{stroke,fill}%
}%
\begin{pgfscope}%
\pgfsys@transformshift{2.542146in}{0.532031in}%
\pgfsys@useobject{currentmarker}{}%
\end{pgfscope}%
\end{pgfscope}%
\begin{pgfscope}%
\pgfsetbuttcap%
\pgfsetroundjoin%
\definecolor{currentfill}{rgb}{0.000000,0.000000,0.000000}%
\pgfsetfillcolor{currentfill}%
\pgfsetlinewidth{0.501875pt}%
\definecolor{currentstroke}{rgb}{0.000000,0.000000,0.000000}%
\pgfsetstrokecolor{currentstroke}%
\pgfsetdash{}{0pt}%
\pgfsys@defobject{currentmarker}{\pgfqpoint{0.000000in}{-0.055556in}}{\pgfqpoint{0.000000in}{0.000000in}}{%
\pgfpathmoveto{\pgfqpoint{0.000000in}{0.000000in}}%
\pgfpathlineto{\pgfqpoint{0.000000in}{-0.055556in}}%
\pgfusepath{stroke,fill}%
}%
\begin{pgfscope}%
\pgfsys@transformshift{2.542146in}{1.690833in}%
\pgfsys@useobject{currentmarker}{}%
\end{pgfscope}%
\end{pgfscope}%
\begin{pgfscope}%
\pgftext[x=2.542146in,y=0.476476in,,top]{\rmfamily\fontsize{10.000000}{12.000000}\selectfont 40}%
\end{pgfscope}%
\begin{pgfscope}%
\pgfsetbuttcap%
\pgfsetroundjoin%
\definecolor{currentfill}{rgb}{0.000000,0.000000,0.000000}%
\pgfsetfillcolor{currentfill}%
\pgfsetlinewidth{0.501875pt}%
\definecolor{currentstroke}{rgb}{0.000000,0.000000,0.000000}%
\pgfsetstrokecolor{currentstroke}%
\pgfsetdash{}{0pt}%
\pgfsys@defobject{currentmarker}{\pgfqpoint{0.000000in}{0.000000in}}{\pgfqpoint{0.000000in}{0.055556in}}{%
\pgfpathmoveto{\pgfqpoint{0.000000in}{0.000000in}}%
\pgfpathlineto{\pgfqpoint{0.000000in}{0.055556in}}%
\pgfusepath{stroke,fill}%
}%
\begin{pgfscope}%
\pgfsys@transformshift{2.957935in}{0.532031in}%
\pgfsys@useobject{currentmarker}{}%
\end{pgfscope}%
\end{pgfscope}%
\begin{pgfscope}%
\pgfsetbuttcap%
\pgfsetroundjoin%
\definecolor{currentfill}{rgb}{0.000000,0.000000,0.000000}%
\pgfsetfillcolor{currentfill}%
\pgfsetlinewidth{0.501875pt}%
\definecolor{currentstroke}{rgb}{0.000000,0.000000,0.000000}%
\pgfsetstrokecolor{currentstroke}%
\pgfsetdash{}{0pt}%
\pgfsys@defobject{currentmarker}{\pgfqpoint{0.000000in}{-0.055556in}}{\pgfqpoint{0.000000in}{0.000000in}}{%
\pgfpathmoveto{\pgfqpoint{0.000000in}{0.000000in}}%
\pgfpathlineto{\pgfqpoint{0.000000in}{-0.055556in}}%
\pgfusepath{stroke,fill}%
}%
\begin{pgfscope}%
\pgfsys@transformshift{2.957935in}{1.690833in}%
\pgfsys@useobject{currentmarker}{}%
\end{pgfscope}%
\end{pgfscope}%
\begin{pgfscope}%
\pgftext[x=2.957935in,y=0.476476in,,top]{\rmfamily\fontsize{10.000000}{12.000000}\selectfont 50}%
\end{pgfscope}%
\begin{pgfscope}%
\pgfsetbuttcap%
\pgfsetroundjoin%
\definecolor{currentfill}{rgb}{0.000000,0.000000,0.000000}%
\pgfsetfillcolor{currentfill}%
\pgfsetlinewidth{0.501875pt}%
\definecolor{currentstroke}{rgb}{0.000000,0.000000,0.000000}%
\pgfsetstrokecolor{currentstroke}%
\pgfsetdash{}{0pt}%
\pgfsys@defobject{currentmarker}{\pgfqpoint{0.000000in}{0.000000in}}{\pgfqpoint{0.000000in}{0.055556in}}{%
\pgfpathmoveto{\pgfqpoint{0.000000in}{0.000000in}}%
\pgfpathlineto{\pgfqpoint{0.000000in}{0.055556in}}%
\pgfusepath{stroke,fill}%
}%
\begin{pgfscope}%
\pgfsys@transformshift{3.373725in}{0.532031in}%
\pgfsys@useobject{currentmarker}{}%
\end{pgfscope}%
\end{pgfscope}%
\begin{pgfscope}%
\pgfsetbuttcap%
\pgfsetroundjoin%
\definecolor{currentfill}{rgb}{0.000000,0.000000,0.000000}%
\pgfsetfillcolor{currentfill}%
\pgfsetlinewidth{0.501875pt}%
\definecolor{currentstroke}{rgb}{0.000000,0.000000,0.000000}%
\pgfsetstrokecolor{currentstroke}%
\pgfsetdash{}{0pt}%
\pgfsys@defobject{currentmarker}{\pgfqpoint{0.000000in}{-0.055556in}}{\pgfqpoint{0.000000in}{0.000000in}}{%
\pgfpathmoveto{\pgfqpoint{0.000000in}{0.000000in}}%
\pgfpathlineto{\pgfqpoint{0.000000in}{-0.055556in}}%
\pgfusepath{stroke,fill}%
}%
\begin{pgfscope}%
\pgfsys@transformshift{3.373725in}{1.690833in}%
\pgfsys@useobject{currentmarker}{}%
\end{pgfscope}%
\end{pgfscope}%
\begin{pgfscope}%
\pgftext[x=3.373725in,y=0.476476in,,top]{\rmfamily\fontsize{10.000000}{12.000000}\selectfont 60}%
\end{pgfscope}%
\begin{pgfscope}%
\pgfsetbuttcap%
\pgfsetroundjoin%
\definecolor{currentfill}{rgb}{0.000000,0.000000,0.000000}%
\pgfsetfillcolor{currentfill}%
\pgfsetlinewidth{0.501875pt}%
\definecolor{currentstroke}{rgb}{0.000000,0.000000,0.000000}%
\pgfsetstrokecolor{currentstroke}%
\pgfsetdash{}{0pt}%
\pgfsys@defobject{currentmarker}{\pgfqpoint{0.000000in}{0.000000in}}{\pgfqpoint{0.000000in}{0.055556in}}{%
\pgfpathmoveto{\pgfqpoint{0.000000in}{0.000000in}}%
\pgfpathlineto{\pgfqpoint{0.000000in}{0.055556in}}%
\pgfusepath{stroke,fill}%
}%
\begin{pgfscope}%
\pgfsys@transformshift{3.789514in}{0.532031in}%
\pgfsys@useobject{currentmarker}{}%
\end{pgfscope}%
\end{pgfscope}%
\begin{pgfscope}%
\pgfsetbuttcap%
\pgfsetroundjoin%
\definecolor{currentfill}{rgb}{0.000000,0.000000,0.000000}%
\pgfsetfillcolor{currentfill}%
\pgfsetlinewidth{0.501875pt}%
\definecolor{currentstroke}{rgb}{0.000000,0.000000,0.000000}%
\pgfsetstrokecolor{currentstroke}%
\pgfsetdash{}{0pt}%
\pgfsys@defobject{currentmarker}{\pgfqpoint{0.000000in}{-0.055556in}}{\pgfqpoint{0.000000in}{0.000000in}}{%
\pgfpathmoveto{\pgfqpoint{0.000000in}{0.000000in}}%
\pgfpathlineto{\pgfqpoint{0.000000in}{-0.055556in}}%
\pgfusepath{stroke,fill}%
}%
\begin{pgfscope}%
\pgfsys@transformshift{3.789514in}{1.690833in}%
\pgfsys@useobject{currentmarker}{}%
\end{pgfscope}%
\end{pgfscope}%
\begin{pgfscope}%
\pgftext[x=3.789514in,y=0.476476in,,top]{\rmfamily\fontsize{10.000000}{12.000000}\selectfont 70}%
\end{pgfscope}%
\begin{pgfscope}%
\pgftext[x=2.401797in,y=0.283575in,,top]{\rmfamily\fontsize{10.000000}{12.000000}\selectfont Excess Energy (meV)}%
\end{pgfscope}%
\begin{pgfscope}%
\pgfsetbuttcap%
\pgfsetroundjoin%
\definecolor{currentfill}{rgb}{0.000000,0.000000,0.000000}%
\pgfsetfillcolor{currentfill}%
\pgfsetlinewidth{0.501875pt}%
\definecolor{currentstroke}{rgb}{0.000000,0.000000,0.000000}%
\pgfsetstrokecolor{currentstroke}%
\pgfsetdash{}{0pt}%
\pgfsys@defobject{currentmarker}{\pgfqpoint{0.000000in}{0.000000in}}{\pgfqpoint{0.055556in}{0.000000in}}{%
\pgfpathmoveto{\pgfqpoint{0.000000in}{0.000000in}}%
\pgfpathlineto{\pgfqpoint{0.055556in}{0.000000in}}%
\pgfusepath{stroke,fill}%
}%
\begin{pgfscope}%
\pgfsys@transformshift{0.671094in}{0.532031in}%
\pgfsys@useobject{currentmarker}{}%
\end{pgfscope}%
\end{pgfscope}%
\begin{pgfscope}%
\pgfsetbuttcap%
\pgfsetroundjoin%
\definecolor{currentfill}{rgb}{0.000000,0.000000,0.000000}%
\pgfsetfillcolor{currentfill}%
\pgfsetlinewidth{0.501875pt}%
\definecolor{currentstroke}{rgb}{0.000000,0.000000,0.000000}%
\pgfsetstrokecolor{currentstroke}%
\pgfsetdash{}{0pt}%
\pgfsys@defobject{currentmarker}{\pgfqpoint{-0.055556in}{0.000000in}}{\pgfqpoint{0.000000in}{0.000000in}}{%
\pgfpathmoveto{\pgfqpoint{0.000000in}{0.000000in}}%
\pgfpathlineto{\pgfqpoint{-0.055556in}{0.000000in}}%
\pgfusepath{stroke,fill}%
}%
\begin{pgfscope}%
\pgfsys@transformshift{4.132500in}{0.532031in}%
\pgfsys@useobject{currentmarker}{}%
\end{pgfscope}%
\end{pgfscope}%
\begin{pgfscope}%
\pgftext[x=0.615538in,y=0.532031in,right,]{\rmfamily\fontsize{10.000000}{12.000000}\selectfont 0}%
\end{pgfscope}%
\begin{pgfscope}%
\pgfsetbuttcap%
\pgfsetroundjoin%
\definecolor{currentfill}{rgb}{0.000000,0.000000,0.000000}%
\pgfsetfillcolor{currentfill}%
\pgfsetlinewidth{0.501875pt}%
\definecolor{currentstroke}{rgb}{0.000000,0.000000,0.000000}%
\pgfsetstrokecolor{currentstroke}%
\pgfsetdash{}{0pt}%
\pgfsys@defobject{currentmarker}{\pgfqpoint{0.000000in}{0.000000in}}{\pgfqpoint{0.055556in}{0.000000in}}{%
\pgfpathmoveto{\pgfqpoint{0.000000in}{0.000000in}}%
\pgfpathlineto{\pgfqpoint{0.055556in}{0.000000in}}%
\pgfusepath{stroke,fill}%
}%
\begin{pgfscope}%
\pgfsys@transformshift{0.671094in}{0.676882in}%
\pgfsys@useobject{currentmarker}{}%
\end{pgfscope}%
\end{pgfscope}%
\begin{pgfscope}%
\pgfsetbuttcap%
\pgfsetroundjoin%
\definecolor{currentfill}{rgb}{0.000000,0.000000,0.000000}%
\pgfsetfillcolor{currentfill}%
\pgfsetlinewidth{0.501875pt}%
\definecolor{currentstroke}{rgb}{0.000000,0.000000,0.000000}%
\pgfsetstrokecolor{currentstroke}%
\pgfsetdash{}{0pt}%
\pgfsys@defobject{currentmarker}{\pgfqpoint{-0.055556in}{0.000000in}}{\pgfqpoint{0.000000in}{0.000000in}}{%
\pgfpathmoveto{\pgfqpoint{0.000000in}{0.000000in}}%
\pgfpathlineto{\pgfqpoint{-0.055556in}{0.000000in}}%
\pgfusepath{stroke,fill}%
}%
\begin{pgfscope}%
\pgfsys@transformshift{4.132500in}{0.676882in}%
\pgfsys@useobject{currentmarker}{}%
\end{pgfscope}%
\end{pgfscope}%
\begin{pgfscope}%
\pgftext[x=0.615538in,y=0.676882in,right,]{\rmfamily\fontsize{10.000000}{12.000000}\selectfont 20}%
\end{pgfscope}%
\begin{pgfscope}%
\pgfsetbuttcap%
\pgfsetroundjoin%
\definecolor{currentfill}{rgb}{0.000000,0.000000,0.000000}%
\pgfsetfillcolor{currentfill}%
\pgfsetlinewidth{0.501875pt}%
\definecolor{currentstroke}{rgb}{0.000000,0.000000,0.000000}%
\pgfsetstrokecolor{currentstroke}%
\pgfsetdash{}{0pt}%
\pgfsys@defobject{currentmarker}{\pgfqpoint{0.000000in}{0.000000in}}{\pgfqpoint{0.055556in}{0.000000in}}{%
\pgfpathmoveto{\pgfqpoint{0.000000in}{0.000000in}}%
\pgfpathlineto{\pgfqpoint{0.055556in}{0.000000in}}%
\pgfusepath{stroke,fill}%
}%
\begin{pgfscope}%
\pgfsys@transformshift{0.671094in}{0.821732in}%
\pgfsys@useobject{currentmarker}{}%
\end{pgfscope}%
\end{pgfscope}%
\begin{pgfscope}%
\pgfsetbuttcap%
\pgfsetroundjoin%
\definecolor{currentfill}{rgb}{0.000000,0.000000,0.000000}%
\pgfsetfillcolor{currentfill}%
\pgfsetlinewidth{0.501875pt}%
\definecolor{currentstroke}{rgb}{0.000000,0.000000,0.000000}%
\pgfsetstrokecolor{currentstroke}%
\pgfsetdash{}{0pt}%
\pgfsys@defobject{currentmarker}{\pgfqpoint{-0.055556in}{0.000000in}}{\pgfqpoint{0.000000in}{0.000000in}}{%
\pgfpathmoveto{\pgfqpoint{0.000000in}{0.000000in}}%
\pgfpathlineto{\pgfqpoint{-0.055556in}{0.000000in}}%
\pgfusepath{stroke,fill}%
}%
\begin{pgfscope}%
\pgfsys@transformshift{4.132500in}{0.821732in}%
\pgfsys@useobject{currentmarker}{}%
\end{pgfscope}%
\end{pgfscope}%
\begin{pgfscope}%
\pgftext[x=0.615538in,y=0.821732in,right,]{\rmfamily\fontsize{10.000000}{12.000000}\selectfont 40}%
\end{pgfscope}%
\begin{pgfscope}%
\pgfsetbuttcap%
\pgfsetroundjoin%
\definecolor{currentfill}{rgb}{0.000000,0.000000,0.000000}%
\pgfsetfillcolor{currentfill}%
\pgfsetlinewidth{0.501875pt}%
\definecolor{currentstroke}{rgb}{0.000000,0.000000,0.000000}%
\pgfsetstrokecolor{currentstroke}%
\pgfsetdash{}{0pt}%
\pgfsys@defobject{currentmarker}{\pgfqpoint{0.000000in}{0.000000in}}{\pgfqpoint{0.055556in}{0.000000in}}{%
\pgfpathmoveto{\pgfqpoint{0.000000in}{0.000000in}}%
\pgfpathlineto{\pgfqpoint{0.055556in}{0.000000in}}%
\pgfusepath{stroke,fill}%
}%
\begin{pgfscope}%
\pgfsys@transformshift{0.671094in}{0.966582in}%
\pgfsys@useobject{currentmarker}{}%
\end{pgfscope}%
\end{pgfscope}%
\begin{pgfscope}%
\pgfsetbuttcap%
\pgfsetroundjoin%
\definecolor{currentfill}{rgb}{0.000000,0.000000,0.000000}%
\pgfsetfillcolor{currentfill}%
\pgfsetlinewidth{0.501875pt}%
\definecolor{currentstroke}{rgb}{0.000000,0.000000,0.000000}%
\pgfsetstrokecolor{currentstroke}%
\pgfsetdash{}{0pt}%
\pgfsys@defobject{currentmarker}{\pgfqpoint{-0.055556in}{0.000000in}}{\pgfqpoint{0.000000in}{0.000000in}}{%
\pgfpathmoveto{\pgfqpoint{0.000000in}{0.000000in}}%
\pgfpathlineto{\pgfqpoint{-0.055556in}{0.000000in}}%
\pgfusepath{stroke,fill}%
}%
\begin{pgfscope}%
\pgfsys@transformshift{4.132500in}{0.966582in}%
\pgfsys@useobject{currentmarker}{}%
\end{pgfscope}%
\end{pgfscope}%
\begin{pgfscope}%
\pgftext[x=0.615538in,y=0.966582in,right,]{\rmfamily\fontsize{10.000000}{12.000000}\selectfont 60}%
\end{pgfscope}%
\begin{pgfscope}%
\pgfsetbuttcap%
\pgfsetroundjoin%
\definecolor{currentfill}{rgb}{0.000000,0.000000,0.000000}%
\pgfsetfillcolor{currentfill}%
\pgfsetlinewidth{0.501875pt}%
\definecolor{currentstroke}{rgb}{0.000000,0.000000,0.000000}%
\pgfsetstrokecolor{currentstroke}%
\pgfsetdash{}{0pt}%
\pgfsys@defobject{currentmarker}{\pgfqpoint{0.000000in}{0.000000in}}{\pgfqpoint{0.055556in}{0.000000in}}{%
\pgfpathmoveto{\pgfqpoint{0.000000in}{0.000000in}}%
\pgfpathlineto{\pgfqpoint{0.055556in}{0.000000in}}%
\pgfusepath{stroke,fill}%
}%
\begin{pgfscope}%
\pgfsys@transformshift{0.671094in}{1.111432in}%
\pgfsys@useobject{currentmarker}{}%
\end{pgfscope}%
\end{pgfscope}%
\begin{pgfscope}%
\pgfsetbuttcap%
\pgfsetroundjoin%
\definecolor{currentfill}{rgb}{0.000000,0.000000,0.000000}%
\pgfsetfillcolor{currentfill}%
\pgfsetlinewidth{0.501875pt}%
\definecolor{currentstroke}{rgb}{0.000000,0.000000,0.000000}%
\pgfsetstrokecolor{currentstroke}%
\pgfsetdash{}{0pt}%
\pgfsys@defobject{currentmarker}{\pgfqpoint{-0.055556in}{0.000000in}}{\pgfqpoint{0.000000in}{0.000000in}}{%
\pgfpathmoveto{\pgfqpoint{0.000000in}{0.000000in}}%
\pgfpathlineto{\pgfqpoint{-0.055556in}{0.000000in}}%
\pgfusepath{stroke,fill}%
}%
\begin{pgfscope}%
\pgfsys@transformshift{4.132500in}{1.111432in}%
\pgfsys@useobject{currentmarker}{}%
\end{pgfscope}%
\end{pgfscope}%
\begin{pgfscope}%
\pgftext[x=0.615538in,y=1.111432in,right,]{\rmfamily\fontsize{10.000000}{12.000000}\selectfont 80}%
\end{pgfscope}%
\begin{pgfscope}%
\pgfsetbuttcap%
\pgfsetroundjoin%
\definecolor{currentfill}{rgb}{0.000000,0.000000,0.000000}%
\pgfsetfillcolor{currentfill}%
\pgfsetlinewidth{0.501875pt}%
\definecolor{currentstroke}{rgb}{0.000000,0.000000,0.000000}%
\pgfsetstrokecolor{currentstroke}%
\pgfsetdash{}{0pt}%
\pgfsys@defobject{currentmarker}{\pgfqpoint{0.000000in}{0.000000in}}{\pgfqpoint{0.055556in}{0.000000in}}{%
\pgfpathmoveto{\pgfqpoint{0.000000in}{0.000000in}}%
\pgfpathlineto{\pgfqpoint{0.055556in}{0.000000in}}%
\pgfusepath{stroke,fill}%
}%
\begin{pgfscope}%
\pgfsys@transformshift{0.671094in}{1.256283in}%
\pgfsys@useobject{currentmarker}{}%
\end{pgfscope}%
\end{pgfscope}%
\begin{pgfscope}%
\pgfsetbuttcap%
\pgfsetroundjoin%
\definecolor{currentfill}{rgb}{0.000000,0.000000,0.000000}%
\pgfsetfillcolor{currentfill}%
\pgfsetlinewidth{0.501875pt}%
\definecolor{currentstroke}{rgb}{0.000000,0.000000,0.000000}%
\pgfsetstrokecolor{currentstroke}%
\pgfsetdash{}{0pt}%
\pgfsys@defobject{currentmarker}{\pgfqpoint{-0.055556in}{0.000000in}}{\pgfqpoint{0.000000in}{0.000000in}}{%
\pgfpathmoveto{\pgfqpoint{0.000000in}{0.000000in}}%
\pgfpathlineto{\pgfqpoint{-0.055556in}{0.000000in}}%
\pgfusepath{stroke,fill}%
}%
\begin{pgfscope}%
\pgfsys@transformshift{4.132500in}{1.256283in}%
\pgfsys@useobject{currentmarker}{}%
\end{pgfscope}%
\end{pgfscope}%
\begin{pgfscope}%
\pgftext[x=0.615538in,y=1.256283in,right,]{\rmfamily\fontsize{10.000000}{12.000000}\selectfont 100}%
\end{pgfscope}%
\begin{pgfscope}%
\pgfsetbuttcap%
\pgfsetroundjoin%
\definecolor{currentfill}{rgb}{0.000000,0.000000,0.000000}%
\pgfsetfillcolor{currentfill}%
\pgfsetlinewidth{0.501875pt}%
\definecolor{currentstroke}{rgb}{0.000000,0.000000,0.000000}%
\pgfsetstrokecolor{currentstroke}%
\pgfsetdash{}{0pt}%
\pgfsys@defobject{currentmarker}{\pgfqpoint{0.000000in}{0.000000in}}{\pgfqpoint{0.055556in}{0.000000in}}{%
\pgfpathmoveto{\pgfqpoint{0.000000in}{0.000000in}}%
\pgfpathlineto{\pgfqpoint{0.055556in}{0.000000in}}%
\pgfusepath{stroke,fill}%
}%
\begin{pgfscope}%
\pgfsys@transformshift{0.671094in}{1.401133in}%
\pgfsys@useobject{currentmarker}{}%
\end{pgfscope}%
\end{pgfscope}%
\begin{pgfscope}%
\pgfsetbuttcap%
\pgfsetroundjoin%
\definecolor{currentfill}{rgb}{0.000000,0.000000,0.000000}%
\pgfsetfillcolor{currentfill}%
\pgfsetlinewidth{0.501875pt}%
\definecolor{currentstroke}{rgb}{0.000000,0.000000,0.000000}%
\pgfsetstrokecolor{currentstroke}%
\pgfsetdash{}{0pt}%
\pgfsys@defobject{currentmarker}{\pgfqpoint{-0.055556in}{0.000000in}}{\pgfqpoint{0.000000in}{0.000000in}}{%
\pgfpathmoveto{\pgfqpoint{0.000000in}{0.000000in}}%
\pgfpathlineto{\pgfqpoint{-0.055556in}{0.000000in}}%
\pgfusepath{stroke,fill}%
}%
\begin{pgfscope}%
\pgfsys@transformshift{4.132500in}{1.401133in}%
\pgfsys@useobject{currentmarker}{}%
\end{pgfscope}%
\end{pgfscope}%
\begin{pgfscope}%
\pgftext[x=0.615538in,y=1.401133in,right,]{\rmfamily\fontsize{10.000000}{12.000000}\selectfont 120}%
\end{pgfscope}%
\begin{pgfscope}%
\pgfsetbuttcap%
\pgfsetroundjoin%
\definecolor{currentfill}{rgb}{0.000000,0.000000,0.000000}%
\pgfsetfillcolor{currentfill}%
\pgfsetlinewidth{0.501875pt}%
\definecolor{currentstroke}{rgb}{0.000000,0.000000,0.000000}%
\pgfsetstrokecolor{currentstroke}%
\pgfsetdash{}{0pt}%
\pgfsys@defobject{currentmarker}{\pgfqpoint{0.000000in}{0.000000in}}{\pgfqpoint{0.055556in}{0.000000in}}{%
\pgfpathmoveto{\pgfqpoint{0.000000in}{0.000000in}}%
\pgfpathlineto{\pgfqpoint{0.055556in}{0.000000in}}%
\pgfusepath{stroke,fill}%
}%
\begin{pgfscope}%
\pgfsys@transformshift{0.671094in}{1.545983in}%
\pgfsys@useobject{currentmarker}{}%
\end{pgfscope}%
\end{pgfscope}%
\begin{pgfscope}%
\pgfsetbuttcap%
\pgfsetroundjoin%
\definecolor{currentfill}{rgb}{0.000000,0.000000,0.000000}%
\pgfsetfillcolor{currentfill}%
\pgfsetlinewidth{0.501875pt}%
\definecolor{currentstroke}{rgb}{0.000000,0.000000,0.000000}%
\pgfsetstrokecolor{currentstroke}%
\pgfsetdash{}{0pt}%
\pgfsys@defobject{currentmarker}{\pgfqpoint{-0.055556in}{0.000000in}}{\pgfqpoint{0.000000in}{0.000000in}}{%
\pgfpathmoveto{\pgfqpoint{0.000000in}{0.000000in}}%
\pgfpathlineto{\pgfqpoint{-0.055556in}{0.000000in}}%
\pgfusepath{stroke,fill}%
}%
\begin{pgfscope}%
\pgfsys@transformshift{4.132500in}{1.545983in}%
\pgfsys@useobject{currentmarker}{}%
\end{pgfscope}%
\end{pgfscope}%
\begin{pgfscope}%
\pgftext[x=0.615538in,y=1.545983in,right,]{\rmfamily\fontsize{10.000000}{12.000000}\selectfont 140}%
\end{pgfscope}%
\begin{pgfscope}%
\pgfsetbuttcap%
\pgfsetroundjoin%
\definecolor{currentfill}{rgb}{0.000000,0.000000,0.000000}%
\pgfsetfillcolor{currentfill}%
\pgfsetlinewidth{0.501875pt}%
\definecolor{currentstroke}{rgb}{0.000000,0.000000,0.000000}%
\pgfsetstrokecolor{currentstroke}%
\pgfsetdash{}{0pt}%
\pgfsys@defobject{currentmarker}{\pgfqpoint{0.000000in}{0.000000in}}{\pgfqpoint{0.055556in}{0.000000in}}{%
\pgfpathmoveto{\pgfqpoint{0.000000in}{0.000000in}}%
\pgfpathlineto{\pgfqpoint{0.055556in}{0.000000in}}%
\pgfusepath{stroke,fill}%
}%
\begin{pgfscope}%
\pgfsys@transformshift{0.671094in}{1.690833in}%
\pgfsys@useobject{currentmarker}{}%
\end{pgfscope}%
\end{pgfscope}%
\begin{pgfscope}%
\pgfsetbuttcap%
\pgfsetroundjoin%
\definecolor{currentfill}{rgb}{0.000000,0.000000,0.000000}%
\pgfsetfillcolor{currentfill}%
\pgfsetlinewidth{0.501875pt}%
\definecolor{currentstroke}{rgb}{0.000000,0.000000,0.000000}%
\pgfsetstrokecolor{currentstroke}%
\pgfsetdash{}{0pt}%
\pgfsys@defobject{currentmarker}{\pgfqpoint{-0.055556in}{0.000000in}}{\pgfqpoint{0.000000in}{0.000000in}}{%
\pgfpathmoveto{\pgfqpoint{0.000000in}{0.000000in}}%
\pgfpathlineto{\pgfqpoint{-0.055556in}{0.000000in}}%
\pgfusepath{stroke,fill}%
}%
\begin{pgfscope}%
\pgfsys@transformshift{4.132500in}{1.690833in}%
\pgfsys@useobject{currentmarker}{}%
\end{pgfscope}%
\end{pgfscope}%
\begin{pgfscope}%
\pgftext[x=0.615538in,y=1.690833in,right,]{\rmfamily\fontsize{10.000000}{12.000000}\selectfont 160}%
\end{pgfscope}%
\begin{pgfscope}%
\pgftext[x=0.337760in,y=1.111432in,,bottom,rotate=90.000000]{\rmfamily\fontsize{10.000000}{12.000000}\selectfont Emittance (nm rad)}%
\end{pgfscope}%
\end{pgfpicture}%
\makeatother%
\endgroup%

    \caption[Results of a simulation investigating the efficacy of the aperture size correction for various electron excess energies.]{The results of a simulation with a pepperpot mask with \unit[50]{$\muup$m} diameter apertures with an extent much larger than the beam size and negligible beam overlap. The blue line is the result of analysis of the simulated data without the aperture size correction and the green line indicates the analysis results with the aperture size correction. The resolution limit of the mask is apparent with excess energies less than \unit[10]{meV}.}
    \label{figure:wavelength_sim}
    % Data and code located in Code/Electrons/Simulation/1D Simulation SimScript1D.py - Wavelength Sweep
\end{subfigure}

\begin{subfigure}[b]{\textwidth}
    \center
    %% Creator: Matplotlib, PGF backend
%%
%% To include the figure in your LaTeX document, write
%%   \input{<filename>.pgf}
%%
%% Make sure the required packages are loaded in your preamble
%%   \usepackage{pgf}
%%
%% Figures using additional raster images can only be included by \input if
%% they are in the same directory as the main LaTeX file. For loading figures
%% from other directories you can use the `import` package
%%   \usepackage{import}
%% and then include the figures with
%%   \import{<path to file>}{<filename>.pgf}
%%
%% Matplotlib used the following preamble
%%
\begingroup%
\makeatletter%
\begin{pgfpicture}%
\pgfpathrectangle{\pgfpointorigin}{\pgfqpoint{5.424500in}{2.603760in}}%
\pgfusepath{use as bounding box, clip}%
\begin{pgfscope}%
\pgfsetbuttcap%
\pgfsetmiterjoin%
\definecolor{currentfill}{rgb}{1.000000,1.000000,1.000000}%
\pgfsetfillcolor{currentfill}%
\pgfsetlinewidth{0.000000pt}%
\definecolor{currentstroke}{rgb}{1.000000,1.000000,1.000000}%
\pgfsetstrokecolor{currentstroke}%
\pgfsetdash{}{0pt}%
\pgfpathmoveto{\pgfqpoint{0.000000in}{0.000000in}}%
\pgfpathlineto{\pgfqpoint{5.424500in}{0.000000in}}%
\pgfpathlineto{\pgfqpoint{5.424500in}{2.603760in}}%
\pgfpathlineto{\pgfqpoint{0.000000in}{2.603760in}}%
\pgfpathclose%
\pgfusepath{fill}%
\end{pgfscope}%
\begin{pgfscope}%
\pgfsetbuttcap%
\pgfsetmiterjoin%
\definecolor{currentfill}{rgb}{1.000000,1.000000,1.000000}%
\pgfsetfillcolor{currentfill}%
\pgfsetlinewidth{0.000000pt}%
\definecolor{currentstroke}{rgb}{0.000000,0.000000,0.000000}%
\pgfsetstrokecolor{currentstroke}%
\pgfsetstrokeopacity{0.000000}%
\pgfsetdash{}{0pt}%
\pgfpathmoveto{\pgfqpoint{0.671094in}{0.587500in}}%
\pgfpathlineto{\pgfqpoint{5.140906in}{0.587500in}}%
\pgfpathlineto{\pgfqpoint{5.140906in}{2.391260in}}%
\pgfpathlineto{\pgfqpoint{0.671094in}{2.391260in}}%
\pgfpathclose%
\pgfusepath{fill}%
\end{pgfscope}%
\begin{pgfscope}%
\pgfpathrectangle{\pgfqpoint{0.671094in}{0.587500in}}{\pgfqpoint{4.469813in}{1.803760in}} %
\pgfusepath{clip}%
\pgfsetrectcap%
\pgfsetroundjoin%
\pgfsetlinewidth{1.003750pt}%
\definecolor{currentstroke}{rgb}{0.000000,0.000000,1.000000}%
\pgfsetstrokecolor{currentstroke}%
\pgfsetdash{}{0pt}%
\pgfpathmoveto{\pgfqpoint{0.715792in}{0.789267in}}%
\pgfpathlineto{\pgfqpoint{0.760490in}{0.737965in}}%
\pgfpathlineto{\pgfqpoint{0.805188in}{0.778861in}}%
\pgfpathlineto{\pgfqpoint{0.849886in}{0.744250in}}%
\pgfpathlineto{\pgfqpoint{0.894584in}{0.794876in}}%
\pgfpathlineto{\pgfqpoint{0.939283in}{0.797677in}}%
\pgfpathlineto{\pgfqpoint{0.983981in}{0.784382in}}%
\pgfpathlineto{\pgfqpoint{1.028679in}{0.801245in}}%
\pgfpathlineto{\pgfqpoint{1.073377in}{0.780569in}}%
\pgfpathlineto{\pgfqpoint{1.118075in}{0.834314in}}%
\pgfpathlineto{\pgfqpoint{1.565056in}{0.842496in}}%
\pgfpathlineto{\pgfqpoint{2.012038in}{0.903476in}}%
\pgfpathlineto{\pgfqpoint{2.459019in}{1.032792in}}%
\pgfpathlineto{\pgfqpoint{2.906000in}{1.193503in}}%
\pgfpathlineto{\pgfqpoint{3.352981in}{1.352948in}}%
\pgfpathlineto{\pgfqpoint{3.799963in}{1.566966in}}%
\pgfpathlineto{\pgfqpoint{4.246944in}{1.785038in}}%
\pgfpathlineto{\pgfqpoint{4.693925in}{2.013143in}}%
\pgfpathlineto{\pgfqpoint{5.140906in}{2.317083in}}%
\pgfusepath{stroke}%
\end{pgfscope}%
\begin{pgfscope}%
\pgfpathrectangle{\pgfqpoint{0.671094in}{0.587500in}}{\pgfqpoint{4.469813in}{1.803760in}} %
\pgfusepath{clip}%
\pgfsetrectcap%
\pgfsetroundjoin%
\pgfsetlinewidth{1.003750pt}%
\definecolor{currentstroke}{rgb}{0.000000,0.500000,0.000000}%
\pgfsetstrokecolor{currentstroke}%
\pgfsetdash{}{0pt}%
\pgfpathmoveto{\pgfqpoint{0.715792in}{0.789225in}}%
\pgfpathlineto{\pgfqpoint{0.760490in}{0.737603in}}%
\pgfpathlineto{\pgfqpoint{0.805188in}{0.777544in}}%
\pgfpathlineto{\pgfqpoint{0.849886in}{0.741910in}}%
\pgfpathlineto{\pgfqpoint{0.894584in}{0.790800in}}%
\pgfpathlineto{\pgfqpoint{0.939283in}{0.792764in}}%
\pgfpathlineto{\pgfqpoint{0.983981in}{0.777037in}}%
\pgfpathlineto{\pgfqpoint{1.028679in}{0.791201in}}%
\pgfpathlineto{\pgfqpoint{1.073377in}{0.768069in}}%
\pgfpathlineto{\pgfqpoint{1.118075in}{0.818103in}}%
\pgfpathlineto{\pgfqpoint{1.565056in}{0.779882in}}%
\pgfpathlineto{\pgfqpoint{2.012038in}{0.762261in}}%
\pgfpathlineto{\pgfqpoint{2.459019in}{0.783345in}}%
\pgfpathlineto{\pgfqpoint{2.906000in}{0.797029in}}%
\pgfpathlineto{\pgfqpoint{3.352981in}{0.778169in}}%
\pgfpathlineto{\pgfqpoint{3.799963in}{0.792933in}}%
\pgfpathlineto{\pgfqpoint{4.246944in}{0.746404in}}%
\pgfpathlineto{\pgfqpoint{4.693925in}{0.690530in}}%
\pgfpathlineto{\pgfqpoint{5.140906in}{0.681919in}}%
\pgfusepath{stroke}%
\end{pgfscope}%
\begin{pgfscope}%
\pgfpathrectangle{\pgfqpoint{0.671094in}{0.587500in}}{\pgfqpoint{4.469813in}{1.803760in}} %
\pgfusepath{clip}%
\pgfsetbuttcap%
\pgfsetroundjoin%
\pgfsetlinewidth{1.003750pt}%
\definecolor{currentstroke}{rgb}{0.000000,0.000000,0.000000}%
\pgfsetstrokecolor{currentstroke}%
\pgfsetdash{{1.000000pt}{3.000000pt}}{0.000000pt}%
\pgfpathmoveto{\pgfqpoint{0.671094in}{0.736725in}}%
\pgfpathlineto{\pgfqpoint{5.140906in}{0.736725in}}%
\pgfusepath{stroke}%
\end{pgfscope}%
\begin{pgfscope}%
\pgfsetrectcap%
\pgfsetmiterjoin%
\pgfsetlinewidth{1.003750pt}%
\definecolor{currentstroke}{rgb}{0.000000,0.000000,0.000000}%
\pgfsetstrokecolor{currentstroke}%
\pgfsetdash{}{0pt}%
\pgfpathmoveto{\pgfqpoint{0.671094in}{2.391260in}}%
\pgfpathlineto{\pgfqpoint{5.140906in}{2.391260in}}%
\pgfusepath{stroke}%
\end{pgfscope}%
\begin{pgfscope}%
\pgfsetrectcap%
\pgfsetmiterjoin%
\pgfsetlinewidth{1.003750pt}%
\definecolor{currentstroke}{rgb}{0.000000,0.000000,0.000000}%
\pgfsetstrokecolor{currentstroke}%
\pgfsetdash{}{0pt}%
\pgfpathmoveto{\pgfqpoint{0.671094in}{0.587500in}}%
\pgfpathlineto{\pgfqpoint{5.140906in}{0.587500in}}%
\pgfusepath{stroke}%
\end{pgfscope}%
\begin{pgfscope}%
\pgfsetrectcap%
\pgfsetmiterjoin%
\pgfsetlinewidth{1.003750pt}%
\definecolor{currentstroke}{rgb}{0.000000,0.000000,0.000000}%
\pgfsetstrokecolor{currentstroke}%
\pgfsetdash{}{0pt}%
\pgfpathmoveto{\pgfqpoint{5.140906in}{0.587500in}}%
\pgfpathlineto{\pgfqpoint{5.140906in}{2.391260in}}%
\pgfusepath{stroke}%
\end{pgfscope}%
\begin{pgfscope}%
\pgfsetrectcap%
\pgfsetmiterjoin%
\pgfsetlinewidth{1.003750pt}%
\definecolor{currentstroke}{rgb}{0.000000,0.000000,0.000000}%
\pgfsetstrokecolor{currentstroke}%
\pgfsetdash{}{0pt}%
\pgfpathmoveto{\pgfqpoint{0.671094in}{0.587500in}}%
\pgfpathlineto{\pgfqpoint{0.671094in}{2.391260in}}%
\pgfusepath{stroke}%
\end{pgfscope}%
\begin{pgfscope}%
\pgfsetbuttcap%
\pgfsetroundjoin%
\definecolor{currentfill}{rgb}{0.000000,0.000000,0.000000}%
\pgfsetfillcolor{currentfill}%
\pgfsetlinewidth{0.501875pt}%
\definecolor{currentstroke}{rgb}{0.000000,0.000000,0.000000}%
\pgfsetstrokecolor{currentstroke}%
\pgfsetdash{}{0pt}%
\pgfsys@defobject{currentmarker}{\pgfqpoint{0.000000in}{0.000000in}}{\pgfqpoint{0.000000in}{0.055556in}}{%
\pgfpathmoveto{\pgfqpoint{0.000000in}{0.000000in}}%
\pgfpathlineto{\pgfqpoint{0.000000in}{0.055556in}}%
\pgfusepath{stroke,fill}%
}%
\begin{pgfscope}%
\pgfsys@transformshift{0.671094in}{0.587500in}%
\pgfsys@useobject{currentmarker}{}%
\end{pgfscope}%
\end{pgfscope}%
\begin{pgfscope}%
\pgfsetbuttcap%
\pgfsetroundjoin%
\definecolor{currentfill}{rgb}{0.000000,0.000000,0.000000}%
\pgfsetfillcolor{currentfill}%
\pgfsetlinewidth{0.501875pt}%
\definecolor{currentstroke}{rgb}{0.000000,0.000000,0.000000}%
\pgfsetstrokecolor{currentstroke}%
\pgfsetdash{}{0pt}%
\pgfsys@defobject{currentmarker}{\pgfqpoint{0.000000in}{-0.055556in}}{\pgfqpoint{0.000000in}{0.000000in}}{%
\pgfpathmoveto{\pgfqpoint{0.000000in}{0.000000in}}%
\pgfpathlineto{\pgfqpoint{0.000000in}{-0.055556in}}%
\pgfusepath{stroke,fill}%
}%
\begin{pgfscope}%
\pgfsys@transformshift{0.671094in}{2.391260in}%
\pgfsys@useobject{currentmarker}{}%
\end{pgfscope}%
\end{pgfscope}%
\begin{pgfscope}%
\pgftext[x=0.671094in,y=0.531944in,,top]{\rmfamily\fontsize{10.000000}{12.000000}\selectfont 0}%
\end{pgfscope}%
\begin{pgfscope}%
\pgfsetbuttcap%
\pgfsetroundjoin%
\definecolor{currentfill}{rgb}{0.000000,0.000000,0.000000}%
\pgfsetfillcolor{currentfill}%
\pgfsetlinewidth{0.501875pt}%
\definecolor{currentstroke}{rgb}{0.000000,0.000000,0.000000}%
\pgfsetstrokecolor{currentstroke}%
\pgfsetdash{}{0pt}%
\pgfsys@defobject{currentmarker}{\pgfqpoint{0.000000in}{0.000000in}}{\pgfqpoint{0.000000in}{0.055556in}}{%
\pgfpathmoveto{\pgfqpoint{0.000000in}{0.000000in}}%
\pgfpathlineto{\pgfqpoint{0.000000in}{0.055556in}}%
\pgfusepath{stroke,fill}%
}%
\begin{pgfscope}%
\pgfsys@transformshift{1.565056in}{0.587500in}%
\pgfsys@useobject{currentmarker}{}%
\end{pgfscope}%
\end{pgfscope}%
\begin{pgfscope}%
\pgfsetbuttcap%
\pgfsetroundjoin%
\definecolor{currentfill}{rgb}{0.000000,0.000000,0.000000}%
\pgfsetfillcolor{currentfill}%
\pgfsetlinewidth{0.501875pt}%
\definecolor{currentstroke}{rgb}{0.000000,0.000000,0.000000}%
\pgfsetstrokecolor{currentstroke}%
\pgfsetdash{}{0pt}%
\pgfsys@defobject{currentmarker}{\pgfqpoint{0.000000in}{-0.055556in}}{\pgfqpoint{0.000000in}{0.000000in}}{%
\pgfpathmoveto{\pgfqpoint{0.000000in}{0.000000in}}%
\pgfpathlineto{\pgfqpoint{0.000000in}{-0.055556in}}%
\pgfusepath{stroke,fill}%
}%
\begin{pgfscope}%
\pgfsys@transformshift{1.565056in}{2.391260in}%
\pgfsys@useobject{currentmarker}{}%
\end{pgfscope}%
\end{pgfscope}%
\begin{pgfscope}%
\pgftext[x=1.565056in,y=0.531944in,,top]{\rmfamily\fontsize{10.000000}{12.000000}\selectfont 20}%
\end{pgfscope}%
\begin{pgfscope}%
\pgfsetbuttcap%
\pgfsetroundjoin%
\definecolor{currentfill}{rgb}{0.000000,0.000000,0.000000}%
\pgfsetfillcolor{currentfill}%
\pgfsetlinewidth{0.501875pt}%
\definecolor{currentstroke}{rgb}{0.000000,0.000000,0.000000}%
\pgfsetstrokecolor{currentstroke}%
\pgfsetdash{}{0pt}%
\pgfsys@defobject{currentmarker}{\pgfqpoint{0.000000in}{0.000000in}}{\pgfqpoint{0.000000in}{0.055556in}}{%
\pgfpathmoveto{\pgfqpoint{0.000000in}{0.000000in}}%
\pgfpathlineto{\pgfqpoint{0.000000in}{0.055556in}}%
\pgfusepath{stroke,fill}%
}%
\begin{pgfscope}%
\pgfsys@transformshift{2.459019in}{0.587500in}%
\pgfsys@useobject{currentmarker}{}%
\end{pgfscope}%
\end{pgfscope}%
\begin{pgfscope}%
\pgfsetbuttcap%
\pgfsetroundjoin%
\definecolor{currentfill}{rgb}{0.000000,0.000000,0.000000}%
\pgfsetfillcolor{currentfill}%
\pgfsetlinewidth{0.501875pt}%
\definecolor{currentstroke}{rgb}{0.000000,0.000000,0.000000}%
\pgfsetstrokecolor{currentstroke}%
\pgfsetdash{}{0pt}%
\pgfsys@defobject{currentmarker}{\pgfqpoint{0.000000in}{-0.055556in}}{\pgfqpoint{0.000000in}{0.000000in}}{%
\pgfpathmoveto{\pgfqpoint{0.000000in}{0.000000in}}%
\pgfpathlineto{\pgfqpoint{0.000000in}{-0.055556in}}%
\pgfusepath{stroke,fill}%
}%
\begin{pgfscope}%
\pgfsys@transformshift{2.459019in}{2.391260in}%
\pgfsys@useobject{currentmarker}{}%
\end{pgfscope}%
\end{pgfscope}%
\begin{pgfscope}%
\pgftext[x=2.459019in,y=0.531944in,,top]{\rmfamily\fontsize{10.000000}{12.000000}\selectfont 40}%
\end{pgfscope}%
\begin{pgfscope}%
\pgfsetbuttcap%
\pgfsetroundjoin%
\definecolor{currentfill}{rgb}{0.000000,0.000000,0.000000}%
\pgfsetfillcolor{currentfill}%
\pgfsetlinewidth{0.501875pt}%
\definecolor{currentstroke}{rgb}{0.000000,0.000000,0.000000}%
\pgfsetstrokecolor{currentstroke}%
\pgfsetdash{}{0pt}%
\pgfsys@defobject{currentmarker}{\pgfqpoint{0.000000in}{0.000000in}}{\pgfqpoint{0.000000in}{0.055556in}}{%
\pgfpathmoveto{\pgfqpoint{0.000000in}{0.000000in}}%
\pgfpathlineto{\pgfqpoint{0.000000in}{0.055556in}}%
\pgfusepath{stroke,fill}%
}%
\begin{pgfscope}%
\pgfsys@transformshift{3.352981in}{0.587500in}%
\pgfsys@useobject{currentmarker}{}%
\end{pgfscope}%
\end{pgfscope}%
\begin{pgfscope}%
\pgfsetbuttcap%
\pgfsetroundjoin%
\definecolor{currentfill}{rgb}{0.000000,0.000000,0.000000}%
\pgfsetfillcolor{currentfill}%
\pgfsetlinewidth{0.501875pt}%
\definecolor{currentstroke}{rgb}{0.000000,0.000000,0.000000}%
\pgfsetstrokecolor{currentstroke}%
\pgfsetdash{}{0pt}%
\pgfsys@defobject{currentmarker}{\pgfqpoint{0.000000in}{-0.055556in}}{\pgfqpoint{0.000000in}{0.000000in}}{%
\pgfpathmoveto{\pgfqpoint{0.000000in}{0.000000in}}%
\pgfpathlineto{\pgfqpoint{0.000000in}{-0.055556in}}%
\pgfusepath{stroke,fill}%
}%
\begin{pgfscope}%
\pgfsys@transformshift{3.352981in}{2.391260in}%
\pgfsys@useobject{currentmarker}{}%
\end{pgfscope}%
\end{pgfscope}%
\begin{pgfscope}%
\pgftext[x=3.352981in,y=0.531944in,,top]{\rmfamily\fontsize{10.000000}{12.000000}\selectfont 60}%
\end{pgfscope}%
\begin{pgfscope}%
\pgfsetbuttcap%
\pgfsetroundjoin%
\definecolor{currentfill}{rgb}{0.000000,0.000000,0.000000}%
\pgfsetfillcolor{currentfill}%
\pgfsetlinewidth{0.501875pt}%
\definecolor{currentstroke}{rgb}{0.000000,0.000000,0.000000}%
\pgfsetstrokecolor{currentstroke}%
\pgfsetdash{}{0pt}%
\pgfsys@defobject{currentmarker}{\pgfqpoint{0.000000in}{0.000000in}}{\pgfqpoint{0.000000in}{0.055556in}}{%
\pgfpathmoveto{\pgfqpoint{0.000000in}{0.000000in}}%
\pgfpathlineto{\pgfqpoint{0.000000in}{0.055556in}}%
\pgfusepath{stroke,fill}%
}%
\begin{pgfscope}%
\pgfsys@transformshift{4.246944in}{0.587500in}%
\pgfsys@useobject{currentmarker}{}%
\end{pgfscope}%
\end{pgfscope}%
\begin{pgfscope}%
\pgfsetbuttcap%
\pgfsetroundjoin%
\definecolor{currentfill}{rgb}{0.000000,0.000000,0.000000}%
\pgfsetfillcolor{currentfill}%
\pgfsetlinewidth{0.501875pt}%
\definecolor{currentstroke}{rgb}{0.000000,0.000000,0.000000}%
\pgfsetstrokecolor{currentstroke}%
\pgfsetdash{}{0pt}%
\pgfsys@defobject{currentmarker}{\pgfqpoint{0.000000in}{-0.055556in}}{\pgfqpoint{0.000000in}{0.000000in}}{%
\pgfpathmoveto{\pgfqpoint{0.000000in}{0.000000in}}%
\pgfpathlineto{\pgfqpoint{0.000000in}{-0.055556in}}%
\pgfusepath{stroke,fill}%
}%
\begin{pgfscope}%
\pgfsys@transformshift{4.246944in}{2.391260in}%
\pgfsys@useobject{currentmarker}{}%
\end{pgfscope}%
\end{pgfscope}%
\begin{pgfscope}%
\pgftext[x=4.246944in,y=0.531944in,,top]{\rmfamily\fontsize{10.000000}{12.000000}\selectfont 80}%
\end{pgfscope}%
\begin{pgfscope}%
\pgfsetbuttcap%
\pgfsetroundjoin%
\definecolor{currentfill}{rgb}{0.000000,0.000000,0.000000}%
\pgfsetfillcolor{currentfill}%
\pgfsetlinewidth{0.501875pt}%
\definecolor{currentstroke}{rgb}{0.000000,0.000000,0.000000}%
\pgfsetstrokecolor{currentstroke}%
\pgfsetdash{}{0pt}%
\pgfsys@defobject{currentmarker}{\pgfqpoint{0.000000in}{0.000000in}}{\pgfqpoint{0.000000in}{0.055556in}}{%
\pgfpathmoveto{\pgfqpoint{0.000000in}{0.000000in}}%
\pgfpathlineto{\pgfqpoint{0.000000in}{0.055556in}}%
\pgfusepath{stroke,fill}%
}%
\begin{pgfscope}%
\pgfsys@transformshift{5.140906in}{0.587500in}%
\pgfsys@useobject{currentmarker}{}%
\end{pgfscope}%
\end{pgfscope}%
\begin{pgfscope}%
\pgfsetbuttcap%
\pgfsetroundjoin%
\definecolor{currentfill}{rgb}{0.000000,0.000000,0.000000}%
\pgfsetfillcolor{currentfill}%
\pgfsetlinewidth{0.501875pt}%
\definecolor{currentstroke}{rgb}{0.000000,0.000000,0.000000}%
\pgfsetstrokecolor{currentstroke}%
\pgfsetdash{}{0pt}%
\pgfsys@defobject{currentmarker}{\pgfqpoint{0.000000in}{-0.055556in}}{\pgfqpoint{0.000000in}{0.000000in}}{%
\pgfpathmoveto{\pgfqpoint{0.000000in}{0.000000in}}%
\pgfpathlineto{\pgfqpoint{0.000000in}{-0.055556in}}%
\pgfusepath{stroke,fill}%
}%
\begin{pgfscope}%
\pgfsys@transformshift{5.140906in}{2.391260in}%
\pgfsys@useobject{currentmarker}{}%
\end{pgfscope}%
\end{pgfscope}%
\begin{pgfscope}%
\pgftext[x=5.140906in,y=0.531944in,,top]{\rmfamily\fontsize{10.000000}{12.000000}\selectfont 100}%
\end{pgfscope}%
\begin{pgfscope}%
\pgftext[x=2.906000in,y=0.339043in,,top]{\rmfamily\fontsize{10.000000}{12.000000}\selectfont Aperture Size (μ\(\displaystyle \mu\)m)}%
\end{pgfscope}%
\begin{pgfscope}%
\pgfsetbuttcap%
\pgfsetroundjoin%
\definecolor{currentfill}{rgb}{0.000000,0.000000,0.000000}%
\pgfsetfillcolor{currentfill}%
\pgfsetlinewidth{0.501875pt}%
\definecolor{currentstroke}{rgb}{0.000000,0.000000,0.000000}%
\pgfsetstrokecolor{currentstroke}%
\pgfsetdash{}{0pt}%
\pgfsys@defobject{currentmarker}{\pgfqpoint{0.000000in}{0.000000in}}{\pgfqpoint{0.055556in}{0.000000in}}{%
\pgfpathmoveto{\pgfqpoint{0.000000in}{0.000000in}}%
\pgfpathlineto{\pgfqpoint{0.055556in}{0.000000in}}%
\pgfusepath{stroke,fill}%
}%
\begin{pgfscope}%
\pgfsys@transformshift{0.671094in}{0.587500in}%
\pgfsys@useobject{currentmarker}{}%
\end{pgfscope}%
\end{pgfscope}%
\begin{pgfscope}%
\pgfsetbuttcap%
\pgfsetroundjoin%
\definecolor{currentfill}{rgb}{0.000000,0.000000,0.000000}%
\pgfsetfillcolor{currentfill}%
\pgfsetlinewidth{0.501875pt}%
\definecolor{currentstroke}{rgb}{0.000000,0.000000,0.000000}%
\pgfsetstrokecolor{currentstroke}%
\pgfsetdash{}{0pt}%
\pgfsys@defobject{currentmarker}{\pgfqpoint{-0.055556in}{0.000000in}}{\pgfqpoint{0.000000in}{0.000000in}}{%
\pgfpathmoveto{\pgfqpoint{0.000000in}{0.000000in}}%
\pgfpathlineto{\pgfqpoint{-0.055556in}{0.000000in}}%
\pgfusepath{stroke,fill}%
}%
\begin{pgfscope}%
\pgfsys@transformshift{5.140906in}{0.587500in}%
\pgfsys@useobject{currentmarker}{}%
\end{pgfscope}%
\end{pgfscope}%
\begin{pgfscope}%
\pgftext[x=0.615538in,y=0.587500in,right,]{\rmfamily\fontsize{10.000000}{12.000000}\selectfont 115}%
\end{pgfscope}%
\begin{pgfscope}%
\pgfsetbuttcap%
\pgfsetroundjoin%
\definecolor{currentfill}{rgb}{0.000000,0.000000,0.000000}%
\pgfsetfillcolor{currentfill}%
\pgfsetlinewidth{0.501875pt}%
\definecolor{currentstroke}{rgb}{0.000000,0.000000,0.000000}%
\pgfsetstrokecolor{currentstroke}%
\pgfsetdash{}{0pt}%
\pgfsys@defobject{currentmarker}{\pgfqpoint{0.000000in}{0.000000in}}{\pgfqpoint{0.055556in}{0.000000in}}{%
\pgfpathmoveto{\pgfqpoint{0.000000in}{0.000000in}}%
\pgfpathlineto{\pgfqpoint{0.055556in}{0.000000in}}%
\pgfusepath{stroke,fill}%
}%
\begin{pgfscope}%
\pgfsys@transformshift{0.671094in}{0.787918in}%
\pgfsys@useobject{currentmarker}{}%
\end{pgfscope}%
\end{pgfscope}%
\begin{pgfscope}%
\pgfsetbuttcap%
\pgfsetroundjoin%
\definecolor{currentfill}{rgb}{0.000000,0.000000,0.000000}%
\pgfsetfillcolor{currentfill}%
\pgfsetlinewidth{0.501875pt}%
\definecolor{currentstroke}{rgb}{0.000000,0.000000,0.000000}%
\pgfsetstrokecolor{currentstroke}%
\pgfsetdash{}{0pt}%
\pgfsys@defobject{currentmarker}{\pgfqpoint{-0.055556in}{0.000000in}}{\pgfqpoint{0.000000in}{0.000000in}}{%
\pgfpathmoveto{\pgfqpoint{0.000000in}{0.000000in}}%
\pgfpathlineto{\pgfqpoint{-0.055556in}{0.000000in}}%
\pgfusepath{stroke,fill}%
}%
\begin{pgfscope}%
\pgfsys@transformshift{5.140906in}{0.787918in}%
\pgfsys@useobject{currentmarker}{}%
\end{pgfscope}%
\end{pgfscope}%
\begin{pgfscope}%
\pgftext[x=0.615538in,y=0.787918in,right,]{\rmfamily\fontsize{10.000000}{12.000000}\selectfont 120}%
\end{pgfscope}%
\begin{pgfscope}%
\pgfsetbuttcap%
\pgfsetroundjoin%
\definecolor{currentfill}{rgb}{0.000000,0.000000,0.000000}%
\pgfsetfillcolor{currentfill}%
\pgfsetlinewidth{0.501875pt}%
\definecolor{currentstroke}{rgb}{0.000000,0.000000,0.000000}%
\pgfsetstrokecolor{currentstroke}%
\pgfsetdash{}{0pt}%
\pgfsys@defobject{currentmarker}{\pgfqpoint{0.000000in}{0.000000in}}{\pgfqpoint{0.055556in}{0.000000in}}{%
\pgfpathmoveto{\pgfqpoint{0.000000in}{0.000000in}}%
\pgfpathlineto{\pgfqpoint{0.055556in}{0.000000in}}%
\pgfusepath{stroke,fill}%
}%
\begin{pgfscope}%
\pgfsys@transformshift{0.671094in}{0.988336in}%
\pgfsys@useobject{currentmarker}{}%
\end{pgfscope}%
\end{pgfscope}%
\begin{pgfscope}%
\pgfsetbuttcap%
\pgfsetroundjoin%
\definecolor{currentfill}{rgb}{0.000000,0.000000,0.000000}%
\pgfsetfillcolor{currentfill}%
\pgfsetlinewidth{0.501875pt}%
\definecolor{currentstroke}{rgb}{0.000000,0.000000,0.000000}%
\pgfsetstrokecolor{currentstroke}%
\pgfsetdash{}{0pt}%
\pgfsys@defobject{currentmarker}{\pgfqpoint{-0.055556in}{0.000000in}}{\pgfqpoint{0.000000in}{0.000000in}}{%
\pgfpathmoveto{\pgfqpoint{0.000000in}{0.000000in}}%
\pgfpathlineto{\pgfqpoint{-0.055556in}{0.000000in}}%
\pgfusepath{stroke,fill}%
}%
\begin{pgfscope}%
\pgfsys@transformshift{5.140906in}{0.988336in}%
\pgfsys@useobject{currentmarker}{}%
\end{pgfscope}%
\end{pgfscope}%
\begin{pgfscope}%
\pgftext[x=0.615538in,y=0.988336in,right,]{\rmfamily\fontsize{10.000000}{12.000000}\selectfont 125}%
\end{pgfscope}%
\begin{pgfscope}%
\pgfsetbuttcap%
\pgfsetroundjoin%
\definecolor{currentfill}{rgb}{0.000000,0.000000,0.000000}%
\pgfsetfillcolor{currentfill}%
\pgfsetlinewidth{0.501875pt}%
\definecolor{currentstroke}{rgb}{0.000000,0.000000,0.000000}%
\pgfsetstrokecolor{currentstroke}%
\pgfsetdash{}{0pt}%
\pgfsys@defobject{currentmarker}{\pgfqpoint{0.000000in}{0.000000in}}{\pgfqpoint{0.055556in}{0.000000in}}{%
\pgfpathmoveto{\pgfqpoint{0.000000in}{0.000000in}}%
\pgfpathlineto{\pgfqpoint{0.055556in}{0.000000in}}%
\pgfusepath{stroke,fill}%
}%
\begin{pgfscope}%
\pgfsys@transformshift{0.671094in}{1.188753in}%
\pgfsys@useobject{currentmarker}{}%
\end{pgfscope}%
\end{pgfscope}%
\begin{pgfscope}%
\pgfsetbuttcap%
\pgfsetroundjoin%
\definecolor{currentfill}{rgb}{0.000000,0.000000,0.000000}%
\pgfsetfillcolor{currentfill}%
\pgfsetlinewidth{0.501875pt}%
\definecolor{currentstroke}{rgb}{0.000000,0.000000,0.000000}%
\pgfsetstrokecolor{currentstroke}%
\pgfsetdash{}{0pt}%
\pgfsys@defobject{currentmarker}{\pgfqpoint{-0.055556in}{0.000000in}}{\pgfqpoint{0.000000in}{0.000000in}}{%
\pgfpathmoveto{\pgfqpoint{0.000000in}{0.000000in}}%
\pgfpathlineto{\pgfqpoint{-0.055556in}{0.000000in}}%
\pgfusepath{stroke,fill}%
}%
\begin{pgfscope}%
\pgfsys@transformshift{5.140906in}{1.188753in}%
\pgfsys@useobject{currentmarker}{}%
\end{pgfscope}%
\end{pgfscope}%
\begin{pgfscope}%
\pgftext[x=0.615538in,y=1.188753in,right,]{\rmfamily\fontsize{10.000000}{12.000000}\selectfont 130}%
\end{pgfscope}%
\begin{pgfscope}%
\pgfsetbuttcap%
\pgfsetroundjoin%
\definecolor{currentfill}{rgb}{0.000000,0.000000,0.000000}%
\pgfsetfillcolor{currentfill}%
\pgfsetlinewidth{0.501875pt}%
\definecolor{currentstroke}{rgb}{0.000000,0.000000,0.000000}%
\pgfsetstrokecolor{currentstroke}%
\pgfsetdash{}{0pt}%
\pgfsys@defobject{currentmarker}{\pgfqpoint{0.000000in}{0.000000in}}{\pgfqpoint{0.055556in}{0.000000in}}{%
\pgfpathmoveto{\pgfqpoint{0.000000in}{0.000000in}}%
\pgfpathlineto{\pgfqpoint{0.055556in}{0.000000in}}%
\pgfusepath{stroke,fill}%
}%
\begin{pgfscope}%
\pgfsys@transformshift{0.671094in}{1.389171in}%
\pgfsys@useobject{currentmarker}{}%
\end{pgfscope}%
\end{pgfscope}%
\begin{pgfscope}%
\pgfsetbuttcap%
\pgfsetroundjoin%
\definecolor{currentfill}{rgb}{0.000000,0.000000,0.000000}%
\pgfsetfillcolor{currentfill}%
\pgfsetlinewidth{0.501875pt}%
\definecolor{currentstroke}{rgb}{0.000000,0.000000,0.000000}%
\pgfsetstrokecolor{currentstroke}%
\pgfsetdash{}{0pt}%
\pgfsys@defobject{currentmarker}{\pgfqpoint{-0.055556in}{0.000000in}}{\pgfqpoint{0.000000in}{0.000000in}}{%
\pgfpathmoveto{\pgfqpoint{0.000000in}{0.000000in}}%
\pgfpathlineto{\pgfqpoint{-0.055556in}{0.000000in}}%
\pgfusepath{stroke,fill}%
}%
\begin{pgfscope}%
\pgfsys@transformshift{5.140906in}{1.389171in}%
\pgfsys@useobject{currentmarker}{}%
\end{pgfscope}%
\end{pgfscope}%
\begin{pgfscope}%
\pgftext[x=0.615538in,y=1.389171in,right,]{\rmfamily\fontsize{10.000000}{12.000000}\selectfont 135}%
\end{pgfscope}%
\begin{pgfscope}%
\pgfsetbuttcap%
\pgfsetroundjoin%
\definecolor{currentfill}{rgb}{0.000000,0.000000,0.000000}%
\pgfsetfillcolor{currentfill}%
\pgfsetlinewidth{0.501875pt}%
\definecolor{currentstroke}{rgb}{0.000000,0.000000,0.000000}%
\pgfsetstrokecolor{currentstroke}%
\pgfsetdash{}{0pt}%
\pgfsys@defobject{currentmarker}{\pgfqpoint{0.000000in}{0.000000in}}{\pgfqpoint{0.055556in}{0.000000in}}{%
\pgfpathmoveto{\pgfqpoint{0.000000in}{0.000000in}}%
\pgfpathlineto{\pgfqpoint{0.055556in}{0.000000in}}%
\pgfusepath{stroke,fill}%
}%
\begin{pgfscope}%
\pgfsys@transformshift{0.671094in}{1.589589in}%
\pgfsys@useobject{currentmarker}{}%
\end{pgfscope}%
\end{pgfscope}%
\begin{pgfscope}%
\pgfsetbuttcap%
\pgfsetroundjoin%
\definecolor{currentfill}{rgb}{0.000000,0.000000,0.000000}%
\pgfsetfillcolor{currentfill}%
\pgfsetlinewidth{0.501875pt}%
\definecolor{currentstroke}{rgb}{0.000000,0.000000,0.000000}%
\pgfsetstrokecolor{currentstroke}%
\pgfsetdash{}{0pt}%
\pgfsys@defobject{currentmarker}{\pgfqpoint{-0.055556in}{0.000000in}}{\pgfqpoint{0.000000in}{0.000000in}}{%
\pgfpathmoveto{\pgfqpoint{0.000000in}{0.000000in}}%
\pgfpathlineto{\pgfqpoint{-0.055556in}{0.000000in}}%
\pgfusepath{stroke,fill}%
}%
\begin{pgfscope}%
\pgfsys@transformshift{5.140906in}{1.589589in}%
\pgfsys@useobject{currentmarker}{}%
\end{pgfscope}%
\end{pgfscope}%
\begin{pgfscope}%
\pgftext[x=0.615538in,y=1.589589in,right,]{\rmfamily\fontsize{10.000000}{12.000000}\selectfont 140}%
\end{pgfscope}%
\begin{pgfscope}%
\pgfsetbuttcap%
\pgfsetroundjoin%
\definecolor{currentfill}{rgb}{0.000000,0.000000,0.000000}%
\pgfsetfillcolor{currentfill}%
\pgfsetlinewidth{0.501875pt}%
\definecolor{currentstroke}{rgb}{0.000000,0.000000,0.000000}%
\pgfsetstrokecolor{currentstroke}%
\pgfsetdash{}{0pt}%
\pgfsys@defobject{currentmarker}{\pgfqpoint{0.000000in}{0.000000in}}{\pgfqpoint{0.055556in}{0.000000in}}{%
\pgfpathmoveto{\pgfqpoint{0.000000in}{0.000000in}}%
\pgfpathlineto{\pgfqpoint{0.055556in}{0.000000in}}%
\pgfusepath{stroke,fill}%
}%
\begin{pgfscope}%
\pgfsys@transformshift{0.671094in}{1.790007in}%
\pgfsys@useobject{currentmarker}{}%
\end{pgfscope}%
\end{pgfscope}%
\begin{pgfscope}%
\pgfsetbuttcap%
\pgfsetroundjoin%
\definecolor{currentfill}{rgb}{0.000000,0.000000,0.000000}%
\pgfsetfillcolor{currentfill}%
\pgfsetlinewidth{0.501875pt}%
\definecolor{currentstroke}{rgb}{0.000000,0.000000,0.000000}%
\pgfsetstrokecolor{currentstroke}%
\pgfsetdash{}{0pt}%
\pgfsys@defobject{currentmarker}{\pgfqpoint{-0.055556in}{0.000000in}}{\pgfqpoint{0.000000in}{0.000000in}}{%
\pgfpathmoveto{\pgfqpoint{0.000000in}{0.000000in}}%
\pgfpathlineto{\pgfqpoint{-0.055556in}{0.000000in}}%
\pgfusepath{stroke,fill}%
}%
\begin{pgfscope}%
\pgfsys@transformshift{5.140906in}{1.790007in}%
\pgfsys@useobject{currentmarker}{}%
\end{pgfscope}%
\end{pgfscope}%
\begin{pgfscope}%
\pgftext[x=0.615538in,y=1.790007in,right,]{\rmfamily\fontsize{10.000000}{12.000000}\selectfont 145}%
\end{pgfscope}%
\begin{pgfscope}%
\pgfsetbuttcap%
\pgfsetroundjoin%
\definecolor{currentfill}{rgb}{0.000000,0.000000,0.000000}%
\pgfsetfillcolor{currentfill}%
\pgfsetlinewidth{0.501875pt}%
\definecolor{currentstroke}{rgb}{0.000000,0.000000,0.000000}%
\pgfsetstrokecolor{currentstroke}%
\pgfsetdash{}{0pt}%
\pgfsys@defobject{currentmarker}{\pgfqpoint{0.000000in}{0.000000in}}{\pgfqpoint{0.055556in}{0.000000in}}{%
\pgfpathmoveto{\pgfqpoint{0.000000in}{0.000000in}}%
\pgfpathlineto{\pgfqpoint{0.055556in}{0.000000in}}%
\pgfusepath{stroke,fill}%
}%
\begin{pgfscope}%
\pgfsys@transformshift{0.671094in}{1.990424in}%
\pgfsys@useobject{currentmarker}{}%
\end{pgfscope}%
\end{pgfscope}%
\begin{pgfscope}%
\pgfsetbuttcap%
\pgfsetroundjoin%
\definecolor{currentfill}{rgb}{0.000000,0.000000,0.000000}%
\pgfsetfillcolor{currentfill}%
\pgfsetlinewidth{0.501875pt}%
\definecolor{currentstroke}{rgb}{0.000000,0.000000,0.000000}%
\pgfsetstrokecolor{currentstroke}%
\pgfsetdash{}{0pt}%
\pgfsys@defobject{currentmarker}{\pgfqpoint{-0.055556in}{0.000000in}}{\pgfqpoint{0.000000in}{0.000000in}}{%
\pgfpathmoveto{\pgfqpoint{0.000000in}{0.000000in}}%
\pgfpathlineto{\pgfqpoint{-0.055556in}{0.000000in}}%
\pgfusepath{stroke,fill}%
}%
\begin{pgfscope}%
\pgfsys@transformshift{5.140906in}{1.990424in}%
\pgfsys@useobject{currentmarker}{}%
\end{pgfscope}%
\end{pgfscope}%
\begin{pgfscope}%
\pgftext[x=0.615538in,y=1.990424in,right,]{\rmfamily\fontsize{10.000000}{12.000000}\selectfont 150}%
\end{pgfscope}%
\begin{pgfscope}%
\pgfsetbuttcap%
\pgfsetroundjoin%
\definecolor{currentfill}{rgb}{0.000000,0.000000,0.000000}%
\pgfsetfillcolor{currentfill}%
\pgfsetlinewidth{0.501875pt}%
\definecolor{currentstroke}{rgb}{0.000000,0.000000,0.000000}%
\pgfsetstrokecolor{currentstroke}%
\pgfsetdash{}{0pt}%
\pgfsys@defobject{currentmarker}{\pgfqpoint{0.000000in}{0.000000in}}{\pgfqpoint{0.055556in}{0.000000in}}{%
\pgfpathmoveto{\pgfqpoint{0.000000in}{0.000000in}}%
\pgfpathlineto{\pgfqpoint{0.055556in}{0.000000in}}%
\pgfusepath{stroke,fill}%
}%
\begin{pgfscope}%
\pgfsys@transformshift{0.671094in}{2.190842in}%
\pgfsys@useobject{currentmarker}{}%
\end{pgfscope}%
\end{pgfscope}%
\begin{pgfscope}%
\pgfsetbuttcap%
\pgfsetroundjoin%
\definecolor{currentfill}{rgb}{0.000000,0.000000,0.000000}%
\pgfsetfillcolor{currentfill}%
\pgfsetlinewidth{0.501875pt}%
\definecolor{currentstroke}{rgb}{0.000000,0.000000,0.000000}%
\pgfsetstrokecolor{currentstroke}%
\pgfsetdash{}{0pt}%
\pgfsys@defobject{currentmarker}{\pgfqpoint{-0.055556in}{0.000000in}}{\pgfqpoint{0.000000in}{0.000000in}}{%
\pgfpathmoveto{\pgfqpoint{0.000000in}{0.000000in}}%
\pgfpathlineto{\pgfqpoint{-0.055556in}{0.000000in}}%
\pgfusepath{stroke,fill}%
}%
\begin{pgfscope}%
\pgfsys@transformshift{5.140906in}{2.190842in}%
\pgfsys@useobject{currentmarker}{}%
\end{pgfscope}%
\end{pgfscope}%
\begin{pgfscope}%
\pgftext[x=0.615538in,y=2.190842in,right,]{\rmfamily\fontsize{10.000000}{12.000000}\selectfont 155}%
\end{pgfscope}%
\begin{pgfscope}%
\pgfsetbuttcap%
\pgfsetroundjoin%
\definecolor{currentfill}{rgb}{0.000000,0.000000,0.000000}%
\pgfsetfillcolor{currentfill}%
\pgfsetlinewidth{0.501875pt}%
\definecolor{currentstroke}{rgb}{0.000000,0.000000,0.000000}%
\pgfsetstrokecolor{currentstroke}%
\pgfsetdash{}{0pt}%
\pgfsys@defobject{currentmarker}{\pgfqpoint{0.000000in}{0.000000in}}{\pgfqpoint{0.055556in}{0.000000in}}{%
\pgfpathmoveto{\pgfqpoint{0.000000in}{0.000000in}}%
\pgfpathlineto{\pgfqpoint{0.055556in}{0.000000in}}%
\pgfusepath{stroke,fill}%
}%
\begin{pgfscope}%
\pgfsys@transformshift{0.671094in}{2.391260in}%
\pgfsys@useobject{currentmarker}{}%
\end{pgfscope}%
\end{pgfscope}%
\begin{pgfscope}%
\pgfsetbuttcap%
\pgfsetroundjoin%
\definecolor{currentfill}{rgb}{0.000000,0.000000,0.000000}%
\pgfsetfillcolor{currentfill}%
\pgfsetlinewidth{0.501875pt}%
\definecolor{currentstroke}{rgb}{0.000000,0.000000,0.000000}%
\pgfsetstrokecolor{currentstroke}%
\pgfsetdash{}{0pt}%
\pgfsys@defobject{currentmarker}{\pgfqpoint{-0.055556in}{0.000000in}}{\pgfqpoint{0.000000in}{0.000000in}}{%
\pgfpathmoveto{\pgfqpoint{0.000000in}{0.000000in}}%
\pgfpathlineto{\pgfqpoint{-0.055556in}{0.000000in}}%
\pgfusepath{stroke,fill}%
}%
\begin{pgfscope}%
\pgfsys@transformshift{5.140906in}{2.391260in}%
\pgfsys@useobject{currentmarker}{}%
\end{pgfscope}%
\end{pgfscope}%
\begin{pgfscope}%
\pgftext[x=0.615538in,y=2.391260in,right,]{\rmfamily\fontsize{10.000000}{12.000000}\selectfont 160}%
\end{pgfscope}%
\begin{pgfscope}%
\pgftext[x=0.337760in,y=1.489380in,,bottom,rotate=90.000000]{\rmfamily\fontsize{10.000000}{12.000000}\selectfont Measured Emittance (nm rad)}%
\end{pgfscope}%
\end{pgfpicture}%
\makeatother%
\endgroup%

    \caption[Results of a simulation investigating the efficacy of the aperture size correction for various aperture sizes.]{The results of a simulation that varies the size of the apertures used in the pepperpot mask. The blue line indicates the results of analysis without the correction for aperture size and the green line indicates the results with the correction. The dotted black line is the true emittance of the bunch being simulated.}
    \label{figure:aperture_size_sim}
    % Data and code located in Code/Electrons/Simulation/1D Simulation SimScript1D.py
\end{subfigure}
    \caption[Results of a simulations investigating the efficacy of the aperture size correction for various excess energies and aperture sizes.]{}
\end{figure}

\subsection{Beamlet Overlap}
Once again, due to the experimental constraints it was difficult to produce ideal pepperpot measurements with one of the issues being the overlap of beamlets.
Most pepperpot analysis procedures assume that the beamlets are completely separated which was difficult to achieve with this apparatus while providing sufficient flux for streaked measurements and enough sampling of the beam to allow for estimation of the full-beam profile.
A na\"ive analysis of overlapped beamlets might just analyse the potion of the beamlet between the troughs in the signal (which would work fine for fully separated beamlets), however this results in an underestimate of the beamlet size due to the portion underneath the neighbouring beamlets.
A more sophisticated approach is to take the part of the beams that are overlapped into account in order to correctly determine the beamlet width.
The method used here is to fit the sum of N Gaussians, where N is the number of beamlets, to the data.

Shown in Figure~\ref{figure:overlap_example} is an example of overlapped beamlets, the improvement in the emittance measurement from the na\"ive approach to the fitting is clear in this example.
The results of the na\"ive approach can be used as a guess for the fitting of the more successful method.
Most of the measurements made during this project have a lower degree of overlap than that demonstrated in Figure~\ref{figure:overlap_example} and have a much smaller deviation from the correct emittance however the correction was still made and results in a small improvement in the accuracy of the analysis.

\begin{figure}
    \center
    %% Creator: Matplotlib, PGF backend
%%
%% To include the figure in your LaTeX document, write
%%   \input{<filename>.pgf}
%%
%% Make sure the required packages are loaded in your preamble
%%   \usepackage{pgf}
%%
%% Figures using additional raster images can only be included by \input if
%% they are in the same directory as the main LaTeX file. For loading figures
%% from other directories you can use the `import` package
%%   \usepackage{import}
%% and then include the figures with
%%   \import{<path to file>}{<filename>.pgf}
%%
%% Matplotlib used the following preamble
%%
\begingroup%
\makeatletter%
\begin{pgfpicture}%
\pgfpathrectangle{\pgfpointorigin}{\pgfqpoint{5.710000in}{1.903333in}}%
\pgfusepath{use as bounding box, clip}%
\begin{pgfscope}%
\pgfsetbuttcap%
\pgfsetmiterjoin%
\definecolor{currentfill}{rgb}{1.000000,1.000000,1.000000}%
\pgfsetfillcolor{currentfill}%
\pgfsetlinewidth{0.000000pt}%
\definecolor{currentstroke}{rgb}{1.000000,1.000000,1.000000}%
\pgfsetstrokecolor{currentstroke}%
\pgfsetdash{}{0pt}%
\pgfpathmoveto{\pgfqpoint{0.000000in}{0.000000in}}%
\pgfpathlineto{\pgfqpoint{5.710000in}{0.000000in}}%
\pgfpathlineto{\pgfqpoint{5.710000in}{1.903333in}}%
\pgfpathlineto{\pgfqpoint{0.000000in}{1.903333in}}%
\pgfpathclose%
\pgfusepath{fill}%
\end{pgfscope}%
\begin{pgfscope}%
\pgfsetbuttcap%
\pgfsetmiterjoin%
\definecolor{currentfill}{rgb}{1.000000,1.000000,1.000000}%
\pgfsetfillcolor{currentfill}%
\pgfsetlinewidth{0.000000pt}%
\definecolor{currentstroke}{rgb}{0.000000,0.000000,0.000000}%
\pgfsetstrokecolor{currentstroke}%
\pgfsetstrokeopacity{0.000000}%
\pgfsetdash{}{0pt}%
\pgfpathmoveto{\pgfqpoint{0.400000in}{0.529102in}}%
\pgfpathlineto{\pgfqpoint{2.120000in}{0.529102in}}%
\pgfpathlineto{\pgfqpoint{2.120000in}{1.753333in}}%
\pgfpathlineto{\pgfqpoint{0.400000in}{1.753333in}}%
\pgfpathclose%
\pgfusepath{fill}%
\end{pgfscope}%
\begin{pgfscope}%
\pgfpathrectangle{\pgfqpoint{0.400000in}{0.529102in}}{\pgfqpoint{1.720000in}{1.224232in}} %
\pgfusepath{clip}%
\pgfsetrectcap%
\pgfsetroundjoin%
\pgfsetlinewidth{1.003750pt}%
\definecolor{currentstroke}{rgb}{1.000000,0.400000,0.200000}%
\pgfsetstrokecolor{currentstroke}%
\pgfsetdash{}{0pt}%
\pgfpathmoveto{\pgfqpoint{0.400430in}{0.529102in}}%
\pgfpathlineto{\pgfqpoint{0.600050in}{0.530168in}}%
\pgfpathlineto{\pgfqpoint{0.621561in}{0.533303in}}%
\pgfpathlineto{\pgfqpoint{0.629305in}{0.536247in}}%
\pgfpathlineto{\pgfqpoint{0.643932in}{0.545082in}}%
\pgfpathlineto{\pgfqpoint{0.650815in}{0.552284in}}%
\pgfpathlineto{\pgfqpoint{0.658559in}{0.563857in}}%
\pgfpathlineto{\pgfqpoint{0.668884in}{0.583628in}}%
\pgfpathlineto{\pgfqpoint{0.680070in}{0.610129in}}%
\pgfpathlineto{\pgfqpoint{0.689535in}{0.635333in}}%
\pgfpathlineto{\pgfqpoint{0.721371in}{0.735971in}}%
\pgfpathlineto{\pgfqpoint{0.727394in}{0.747433in}}%
\pgfpathlineto{\pgfqpoint{0.742021in}{0.771269in}}%
\pgfpathlineto{\pgfqpoint{0.746323in}{0.771727in}}%
\pgfpathlineto{\pgfqpoint{0.754927in}{0.770369in}}%
\pgfpathlineto{\pgfqpoint{0.772996in}{0.759696in}}%
\pgfpathlineto{\pgfqpoint{0.784182in}{0.749365in}}%
\pgfpathlineto{\pgfqpoint{0.798809in}{0.749621in}}%
\pgfpathlineto{\pgfqpoint{0.802251in}{0.753707in}}%
\pgfpathlineto{\pgfqpoint{0.807414in}{0.764251in}}%
\pgfpathlineto{\pgfqpoint{0.816878in}{0.789774in}}%
\pgfpathlineto{\pgfqpoint{0.822901in}{0.812894in}}%
\pgfpathlineto{\pgfqpoint{0.830645in}{0.853347in}}%
\pgfpathlineto{\pgfqpoint{0.839250in}{0.895130in}}%
\pgfpathlineto{\pgfqpoint{0.859040in}{0.980250in}}%
\pgfpathlineto{\pgfqpoint{0.867644in}{1.003524in}}%
\pgfpathlineto{\pgfqpoint{0.871086in}{1.006578in}}%
\pgfpathlineto{\pgfqpoint{0.874527in}{1.005864in}}%
\pgfpathlineto{\pgfqpoint{0.879690in}{1.004443in}}%
\pgfpathlineto{\pgfqpoint{0.883992in}{1.003753in}}%
\pgfpathlineto{\pgfqpoint{0.886573in}{1.001183in}}%
\pgfpathlineto{\pgfqpoint{0.891736in}{0.991345in}}%
\pgfpathlineto{\pgfqpoint{0.898619in}{0.973794in}}%
\pgfpathlineto{\pgfqpoint{0.908944in}{0.945692in}}%
\pgfpathlineto{\pgfqpoint{0.916688in}{0.928962in}}%
\pgfpathlineto{\pgfqpoint{0.920990in}{0.923810in}}%
\pgfpathlineto{\pgfqpoint{0.923572in}{0.922783in}}%
\pgfpathlineto{\pgfqpoint{0.926153in}{0.924048in}}%
\pgfpathlineto{\pgfqpoint{0.929595in}{0.929752in}}%
\pgfpathlineto{\pgfqpoint{0.936478in}{0.948481in}}%
\pgfpathlineto{\pgfqpoint{0.942501in}{0.969460in}}%
\pgfpathlineto{\pgfqpoint{0.948524in}{1.000676in}}%
\pgfpathlineto{\pgfqpoint{0.964872in}{1.106201in}}%
\pgfpathlineto{\pgfqpoint{0.979500in}{1.221260in}}%
\pgfpathlineto{\pgfqpoint{0.988104in}{1.268824in}}%
\pgfpathlineto{\pgfqpoint{0.995848in}{1.295142in}}%
\pgfpathlineto{\pgfqpoint{1.000150in}{1.304429in}}%
\pgfpathlineto{\pgfqpoint{1.003592in}{1.306563in}}%
\pgfpathlineto{\pgfqpoint{1.007034in}{1.306176in}}%
\pgfpathlineto{\pgfqpoint{1.009615in}{1.304043in}}%
\pgfpathlineto{\pgfqpoint{1.013057in}{1.297699in}}%
\pgfpathlineto{\pgfqpoint{1.017359in}{1.284815in}}%
\pgfpathlineto{\pgfqpoint{1.023382in}{1.258018in}}%
\pgfpathlineto{\pgfqpoint{1.042311in}{1.167810in}}%
\pgfpathlineto{\pgfqpoint{1.051776in}{1.130574in}}%
\pgfpathlineto{\pgfqpoint{1.055218in}{1.122881in}}%
\pgfpathlineto{\pgfqpoint{1.057799in}{1.121208in}}%
\pgfpathlineto{\pgfqpoint{1.060380in}{1.122917in}}%
\pgfpathlineto{\pgfqpoint{1.063822in}{1.129750in}}%
\pgfpathlineto{\pgfqpoint{1.068984in}{1.147299in}}%
\pgfpathlineto{\pgfqpoint{1.075868in}{1.179452in}}%
\pgfpathlineto{\pgfqpoint{1.081891in}{1.217432in}}%
\pgfpathlineto{\pgfqpoint{1.090495in}{1.292792in}}%
\pgfpathlineto{\pgfqpoint{1.101681in}{1.402220in}}%
\pgfpathlineto{\pgfqpoint{1.112866in}{1.501067in}}%
\pgfpathlineto{\pgfqpoint{1.119750in}{1.545347in}}%
\pgfpathlineto{\pgfqpoint{1.124052in}{1.562277in}}%
\pgfpathlineto{\pgfqpoint{1.127494in}{1.568971in}}%
\pgfpathlineto{\pgfqpoint{1.131796in}{1.572392in}}%
\pgfpathlineto{\pgfqpoint{1.134377in}{1.572655in}}%
\pgfpathlineto{\pgfqpoint{1.136958in}{1.570079in}}%
\pgfpathlineto{\pgfqpoint{1.139540in}{1.563068in}}%
\pgfpathlineto{\pgfqpoint{1.143842in}{1.542158in}}%
\pgfpathlineto{\pgfqpoint{1.156748in}{1.462669in}}%
\pgfpathlineto{\pgfqpoint{1.179980in}{1.297022in}}%
\pgfpathlineto{\pgfqpoint{1.187724in}{1.268875in}}%
\pgfpathlineto{\pgfqpoint{1.191166in}{1.262710in}}%
\pgfpathlineto{\pgfqpoint{1.193747in}{1.262294in}}%
\pgfpathlineto{\pgfqpoint{1.196328in}{1.265170in}}%
\pgfpathlineto{\pgfqpoint{1.199770in}{1.273945in}}%
\pgfpathlineto{\pgfqpoint{1.204932in}{1.296256in}}%
\pgfpathlineto{\pgfqpoint{1.210955in}{1.332649in}}%
\pgfpathlineto{\pgfqpoint{1.219560in}{1.400519in}}%
\pgfpathlineto{\pgfqpoint{1.246233in}{1.632248in}}%
\pgfpathlineto{\pgfqpoint{1.252256in}{1.664763in}}%
\pgfpathlineto{\pgfqpoint{1.255698in}{1.673268in}}%
\pgfpathlineto{\pgfqpoint{1.259140in}{1.676411in}}%
\pgfpathlineto{\pgfqpoint{1.261721in}{1.675850in}}%
\pgfpathlineto{\pgfqpoint{1.265163in}{1.671521in}}%
\pgfpathlineto{\pgfqpoint{1.268604in}{1.663252in}}%
\pgfpathlineto{\pgfqpoint{1.272046in}{1.648344in}}%
\pgfpathlineto{\pgfqpoint{1.276348in}{1.618363in}}%
\pgfpathlineto{\pgfqpoint{1.292696in}{1.474052in}}%
\pgfpathlineto{\pgfqpoint{1.306463in}{1.346890in}}%
\pgfpathlineto{\pgfqpoint{1.312486in}{1.303026in}}%
\pgfpathlineto{\pgfqpoint{1.318509in}{1.275528in}}%
\pgfpathlineto{\pgfqpoint{1.322811in}{1.262721in}}%
\pgfpathlineto{\pgfqpoint{1.325393in}{1.260011in}}%
\pgfpathlineto{\pgfqpoint{1.327114in}{1.260555in}}%
\pgfpathlineto{\pgfqpoint{1.329695in}{1.264240in}}%
\pgfpathlineto{\pgfqpoint{1.334857in}{1.277525in}}%
\pgfpathlineto{\pgfqpoint{1.340020in}{1.296590in}}%
\pgfpathlineto{\pgfqpoint{1.345183in}{1.327509in}}%
\pgfpathlineto{\pgfqpoint{1.358089in}{1.421486in}}%
\pgfpathlineto{\pgfqpoint{1.368414in}{1.506247in}}%
\pgfpathlineto{\pgfqpoint{1.374437in}{1.535573in}}%
\pgfpathlineto{\pgfqpoint{1.384762in}{1.573694in}}%
\pgfpathlineto{\pgfqpoint{1.387344in}{1.575132in}}%
\pgfpathlineto{\pgfqpoint{1.389925in}{1.572256in}}%
\pgfpathlineto{\pgfqpoint{1.395088in}{1.561183in}}%
\pgfpathlineto{\pgfqpoint{1.398529in}{1.549725in}}%
\pgfpathlineto{\pgfqpoint{1.402831in}{1.527245in}}%
\pgfpathlineto{\pgfqpoint{1.409715in}{1.477946in}}%
\pgfpathlineto{\pgfqpoint{1.420900in}{1.375020in}}%
\pgfpathlineto{\pgfqpoint{1.443272in}{1.182097in}}%
\pgfpathlineto{\pgfqpoint{1.449295in}{1.153127in}}%
\pgfpathlineto{\pgfqpoint{1.454457in}{1.136709in}}%
\pgfpathlineto{\pgfqpoint{1.457899in}{1.130739in}}%
\pgfpathlineto{\pgfqpoint{1.460480in}{1.129112in}}%
\pgfpathlineto{\pgfqpoint{1.463062in}{1.129692in}}%
\pgfpathlineto{\pgfqpoint{1.466503in}{1.133542in}}%
\pgfpathlineto{\pgfqpoint{1.470805in}{1.142850in}}%
\pgfpathlineto{\pgfqpoint{1.477689in}{1.165037in}}%
\pgfpathlineto{\pgfqpoint{1.486293in}{1.201028in}}%
\pgfpathlineto{\pgfqpoint{1.503502in}{1.283651in}}%
\pgfpathlineto{\pgfqpoint{1.511246in}{1.307000in}}%
\pgfpathlineto{\pgfqpoint{1.513827in}{1.309624in}}%
\pgfpathlineto{\pgfqpoint{1.516408in}{1.309352in}}%
\pgfpathlineto{\pgfqpoint{1.518989in}{1.306820in}}%
\pgfpathlineto{\pgfqpoint{1.522431in}{1.299963in}}%
\pgfpathlineto{\pgfqpoint{1.527594in}{1.283251in}}%
\pgfpathlineto{\pgfqpoint{1.537059in}{1.244110in}}%
\pgfpathlineto{\pgfqpoint{1.543082in}{1.205119in}}%
\pgfpathlineto{\pgfqpoint{1.574917in}{0.978444in}}%
\pgfpathlineto{\pgfqpoint{1.581801in}{0.944943in}}%
\pgfpathlineto{\pgfqpoint{1.586103in}{0.932962in}}%
\pgfpathlineto{\pgfqpoint{1.590405in}{0.927790in}}%
\pgfpathlineto{\pgfqpoint{1.595568in}{0.925167in}}%
\pgfpathlineto{\pgfqpoint{1.599010in}{0.925326in}}%
\pgfpathlineto{\pgfqpoint{1.602451in}{0.928455in}}%
\pgfpathlineto{\pgfqpoint{1.606753in}{0.936444in}}%
\pgfpathlineto{\pgfqpoint{1.619660in}{0.967381in}}%
\pgfpathlineto{\pgfqpoint{1.629125in}{0.991913in}}%
\pgfpathlineto{\pgfqpoint{1.638589in}{1.008500in}}%
\pgfpathlineto{\pgfqpoint{1.642031in}{1.010599in}}%
\pgfpathlineto{\pgfqpoint{1.645473in}{1.009950in}}%
\pgfpathlineto{\pgfqpoint{1.649775in}{1.006462in}}%
\pgfpathlineto{\pgfqpoint{1.654077in}{0.999868in}}%
\pgfpathlineto{\pgfqpoint{1.659240in}{0.986623in}}%
\pgfpathlineto{\pgfqpoint{1.668704in}{0.952755in}}%
\pgfpathlineto{\pgfqpoint{1.675588in}{0.922720in}}%
\pgfpathlineto{\pgfqpoint{1.699680in}{0.806681in}}%
\pgfpathlineto{\pgfqpoint{1.707424in}{0.779339in}}%
\pgfpathlineto{\pgfqpoint{1.714307in}{0.762943in}}%
\pgfpathlineto{\pgfqpoint{1.721191in}{0.752433in}}%
\pgfpathlineto{\pgfqpoint{1.726353in}{0.747568in}}%
\pgfpathlineto{\pgfqpoint{1.731516in}{0.745269in}}%
\pgfpathlineto{\pgfqpoint{1.734957in}{0.745614in}}%
\pgfpathlineto{\pgfqpoint{1.738399in}{0.748127in}}%
\pgfpathlineto{\pgfqpoint{1.751306in}{0.761270in}}%
\pgfpathlineto{\pgfqpoint{1.757329in}{0.763765in}}%
\pgfpathlineto{\pgfqpoint{1.769375in}{0.771904in}}%
\pgfpathlineto{\pgfqpoint{1.778839in}{0.772554in}}%
\pgfpathlineto{\pgfqpoint{1.782281in}{0.770176in}}%
\pgfpathlineto{\pgfqpoint{1.786583in}{0.763585in}}%
\pgfpathlineto{\pgfqpoint{1.796908in}{0.741373in}}%
\pgfpathlineto{\pgfqpoint{1.803792in}{0.721182in}}%
\pgfpathlineto{\pgfqpoint{1.829605in}{0.637833in}}%
\pgfpathlineto{\pgfqpoint{1.842511in}{0.602359in}}%
\pgfpathlineto{\pgfqpoint{1.853697in}{0.577641in}}%
\pgfpathlineto{\pgfqpoint{1.862301in}{0.561980in}}%
\pgfpathlineto{\pgfqpoint{1.872626in}{0.548342in}}%
\pgfpathlineto{\pgfqpoint{1.880370in}{0.541695in}}%
\pgfpathlineto{\pgfqpoint{1.889835in}{0.536578in}}%
\pgfpathlineto{\pgfqpoint{1.900160in}{0.533092in}}%
\pgfpathlineto{\pgfqpoint{1.915648in}{0.530420in}}%
\pgfpathlineto{\pgfqpoint{1.937159in}{0.529341in}}%
\pgfpathlineto{\pgfqpoint{2.079130in}{0.529102in}}%
\pgfpathlineto{\pgfqpoint{2.119570in}{0.529102in}}%
\pgfpathlineto{\pgfqpoint{2.119570in}{0.529102in}}%
\pgfusepath{stroke}%
\end{pgfscope}%
\begin{pgfscope}%
\pgfsetrectcap%
\pgfsetmiterjoin%
\pgfsetlinewidth{1.003750pt}%
\definecolor{currentstroke}{rgb}{0.000000,0.000000,0.000000}%
\pgfsetstrokecolor{currentstroke}%
\pgfsetdash{}{0pt}%
\pgfpathmoveto{\pgfqpoint{2.120000in}{0.529102in}}%
\pgfpathlineto{\pgfqpoint{2.120000in}{1.753333in}}%
\pgfusepath{stroke}%
\end{pgfscope}%
\begin{pgfscope}%
\pgfsetrectcap%
\pgfsetmiterjoin%
\pgfsetlinewidth{1.003750pt}%
\definecolor{currentstroke}{rgb}{0.000000,0.000000,0.000000}%
\pgfsetstrokecolor{currentstroke}%
\pgfsetdash{}{0pt}%
\pgfpathmoveto{\pgfqpoint{0.400000in}{0.529102in}}%
\pgfpathlineto{\pgfqpoint{2.120000in}{0.529102in}}%
\pgfusepath{stroke}%
\end{pgfscope}%
\begin{pgfscope}%
\pgfsetrectcap%
\pgfsetmiterjoin%
\pgfsetlinewidth{1.003750pt}%
\definecolor{currentstroke}{rgb}{0.000000,0.000000,0.000000}%
\pgfsetstrokecolor{currentstroke}%
\pgfsetdash{}{0pt}%
\pgfpathmoveto{\pgfqpoint{0.400000in}{1.753333in}}%
\pgfpathlineto{\pgfqpoint{2.120000in}{1.753333in}}%
\pgfusepath{stroke}%
\end{pgfscope}%
\begin{pgfscope}%
\pgfsetrectcap%
\pgfsetmiterjoin%
\pgfsetlinewidth{1.003750pt}%
\definecolor{currentstroke}{rgb}{0.000000,0.000000,0.000000}%
\pgfsetstrokecolor{currentstroke}%
\pgfsetdash{}{0pt}%
\pgfpathmoveto{\pgfqpoint{0.400000in}{0.529102in}}%
\pgfpathlineto{\pgfqpoint{0.400000in}{1.753333in}}%
\pgfusepath{stroke}%
\end{pgfscope}%
\begin{pgfscope}%
\pgfsetbuttcap%
\pgfsetroundjoin%
\definecolor{currentfill}{rgb}{0.000000,0.000000,0.000000}%
\pgfsetfillcolor{currentfill}%
\pgfsetlinewidth{0.501875pt}%
\definecolor{currentstroke}{rgb}{0.000000,0.000000,0.000000}%
\pgfsetstrokecolor{currentstroke}%
\pgfsetdash{}{0pt}%
\pgfsys@defobject{currentmarker}{\pgfqpoint{0.000000in}{0.000000in}}{\pgfqpoint{0.000000in}{0.055556in}}{%
\pgfpathmoveto{\pgfqpoint{0.000000in}{0.000000in}}%
\pgfpathlineto{\pgfqpoint{0.000000in}{0.055556in}}%
\pgfusepath{stroke,fill}%
}%
\begin{pgfscope}%
\pgfsys@transformshift{0.830000in}{0.529102in}%
\pgfsys@useobject{currentmarker}{}%
\end{pgfscope}%
\end{pgfscope}%
\begin{pgfscope}%
\pgfsetbuttcap%
\pgfsetroundjoin%
\definecolor{currentfill}{rgb}{0.000000,0.000000,0.000000}%
\pgfsetfillcolor{currentfill}%
\pgfsetlinewidth{0.501875pt}%
\definecolor{currentstroke}{rgb}{0.000000,0.000000,0.000000}%
\pgfsetstrokecolor{currentstroke}%
\pgfsetdash{}{0pt}%
\pgfsys@defobject{currentmarker}{\pgfqpoint{0.000000in}{-0.055556in}}{\pgfqpoint{0.000000in}{0.000000in}}{%
\pgfpathmoveto{\pgfqpoint{0.000000in}{0.000000in}}%
\pgfpathlineto{\pgfqpoint{0.000000in}{-0.055556in}}%
\pgfusepath{stroke,fill}%
}%
\begin{pgfscope}%
\pgfsys@transformshift{0.830000in}{1.753333in}%
\pgfsys@useobject{currentmarker}{}%
\end{pgfscope}%
\end{pgfscope}%
\begin{pgfscope}%
\pgftext[x=0.830000in,y=0.473546in,,top]{\fontsize{10.000000}{12.000000}\selectfont -0.5}%
\end{pgfscope}%
\begin{pgfscope}%
\pgfsetbuttcap%
\pgfsetroundjoin%
\definecolor{currentfill}{rgb}{0.000000,0.000000,0.000000}%
\pgfsetfillcolor{currentfill}%
\pgfsetlinewidth{0.501875pt}%
\definecolor{currentstroke}{rgb}{0.000000,0.000000,0.000000}%
\pgfsetstrokecolor{currentstroke}%
\pgfsetdash{}{0pt}%
\pgfsys@defobject{currentmarker}{\pgfqpoint{0.000000in}{0.000000in}}{\pgfqpoint{0.000000in}{0.055556in}}{%
\pgfpathmoveto{\pgfqpoint{0.000000in}{0.000000in}}%
\pgfpathlineto{\pgfqpoint{0.000000in}{0.055556in}}%
\pgfusepath{stroke,fill}%
}%
\begin{pgfscope}%
\pgfsys@transformshift{1.260000in}{0.529102in}%
\pgfsys@useobject{currentmarker}{}%
\end{pgfscope}%
\end{pgfscope}%
\begin{pgfscope}%
\pgfsetbuttcap%
\pgfsetroundjoin%
\definecolor{currentfill}{rgb}{0.000000,0.000000,0.000000}%
\pgfsetfillcolor{currentfill}%
\pgfsetlinewidth{0.501875pt}%
\definecolor{currentstroke}{rgb}{0.000000,0.000000,0.000000}%
\pgfsetstrokecolor{currentstroke}%
\pgfsetdash{}{0pt}%
\pgfsys@defobject{currentmarker}{\pgfqpoint{0.000000in}{-0.055556in}}{\pgfqpoint{0.000000in}{0.000000in}}{%
\pgfpathmoveto{\pgfqpoint{0.000000in}{0.000000in}}%
\pgfpathlineto{\pgfqpoint{0.000000in}{-0.055556in}}%
\pgfusepath{stroke,fill}%
}%
\begin{pgfscope}%
\pgfsys@transformshift{1.260000in}{1.753333in}%
\pgfsys@useobject{currentmarker}{}%
\end{pgfscope}%
\end{pgfscope}%
\begin{pgfscope}%
\pgftext[x=1.260000in,y=0.473546in,,top]{\fontsize{10.000000}{12.000000}\selectfont 0.0}%
\end{pgfscope}%
\begin{pgfscope}%
\pgfsetbuttcap%
\pgfsetroundjoin%
\definecolor{currentfill}{rgb}{0.000000,0.000000,0.000000}%
\pgfsetfillcolor{currentfill}%
\pgfsetlinewidth{0.501875pt}%
\definecolor{currentstroke}{rgb}{0.000000,0.000000,0.000000}%
\pgfsetstrokecolor{currentstroke}%
\pgfsetdash{}{0pt}%
\pgfsys@defobject{currentmarker}{\pgfqpoint{0.000000in}{0.000000in}}{\pgfqpoint{0.000000in}{0.055556in}}{%
\pgfpathmoveto{\pgfqpoint{0.000000in}{0.000000in}}%
\pgfpathlineto{\pgfqpoint{0.000000in}{0.055556in}}%
\pgfusepath{stroke,fill}%
}%
\begin{pgfscope}%
\pgfsys@transformshift{1.690000in}{0.529102in}%
\pgfsys@useobject{currentmarker}{}%
\end{pgfscope}%
\end{pgfscope}%
\begin{pgfscope}%
\pgfsetbuttcap%
\pgfsetroundjoin%
\definecolor{currentfill}{rgb}{0.000000,0.000000,0.000000}%
\pgfsetfillcolor{currentfill}%
\pgfsetlinewidth{0.501875pt}%
\definecolor{currentstroke}{rgb}{0.000000,0.000000,0.000000}%
\pgfsetstrokecolor{currentstroke}%
\pgfsetdash{}{0pt}%
\pgfsys@defobject{currentmarker}{\pgfqpoint{0.000000in}{-0.055556in}}{\pgfqpoint{0.000000in}{0.000000in}}{%
\pgfpathmoveto{\pgfqpoint{0.000000in}{0.000000in}}%
\pgfpathlineto{\pgfqpoint{0.000000in}{-0.055556in}}%
\pgfusepath{stroke,fill}%
}%
\begin{pgfscope}%
\pgfsys@transformshift{1.690000in}{1.753333in}%
\pgfsys@useobject{currentmarker}{}%
\end{pgfscope}%
\end{pgfscope}%
\begin{pgfscope}%
\pgftext[x=1.690000in,y=0.473546in,,top]{\fontsize{10.000000}{12.000000}\selectfont 0.5}%
\end{pgfscope}%
\begin{pgfscope}%
\pgfsetbuttcap%
\pgfsetroundjoin%
\definecolor{currentfill}{rgb}{0.000000,0.000000,0.000000}%
\pgfsetfillcolor{currentfill}%
\pgfsetlinewidth{0.501875pt}%
\definecolor{currentstroke}{rgb}{0.000000,0.000000,0.000000}%
\pgfsetstrokecolor{currentstroke}%
\pgfsetdash{}{0pt}%
\pgfsys@defobject{currentmarker}{\pgfqpoint{0.000000in}{0.000000in}}{\pgfqpoint{0.055556in}{0.000000in}}{%
\pgfpathmoveto{\pgfqpoint{0.000000in}{0.000000in}}%
\pgfpathlineto{\pgfqpoint{0.055556in}{0.000000in}}%
\pgfusepath{stroke,fill}%
}%
\begin{pgfscope}%
\pgfsys@transformshift{0.400000in}{0.529102in}%
\pgfsys@useobject{currentmarker}{}%
\end{pgfscope}%
\end{pgfscope}%
\begin{pgfscope}%
\pgfsetbuttcap%
\pgfsetroundjoin%
\definecolor{currentfill}{rgb}{0.000000,0.000000,0.000000}%
\pgfsetfillcolor{currentfill}%
\pgfsetlinewidth{0.501875pt}%
\definecolor{currentstroke}{rgb}{0.000000,0.000000,0.000000}%
\pgfsetstrokecolor{currentstroke}%
\pgfsetdash{}{0pt}%
\pgfsys@defobject{currentmarker}{\pgfqpoint{-0.055556in}{0.000000in}}{\pgfqpoint{0.000000in}{0.000000in}}{%
\pgfpathmoveto{\pgfqpoint{0.000000in}{0.000000in}}%
\pgfpathlineto{\pgfqpoint{-0.055556in}{0.000000in}}%
\pgfusepath{stroke,fill}%
}%
\begin{pgfscope}%
\pgfsys@transformshift{2.120000in}{0.529102in}%
\pgfsys@useobject{currentmarker}{}%
\end{pgfscope}%
\end{pgfscope}%
\begin{pgfscope}%
\pgfsetbuttcap%
\pgfsetroundjoin%
\definecolor{currentfill}{rgb}{0.000000,0.000000,0.000000}%
\pgfsetfillcolor{currentfill}%
\pgfsetlinewidth{0.501875pt}%
\definecolor{currentstroke}{rgb}{0.000000,0.000000,0.000000}%
\pgfsetstrokecolor{currentstroke}%
\pgfsetdash{}{0pt}%
\pgfsys@defobject{currentmarker}{\pgfqpoint{0.000000in}{0.000000in}}{\pgfqpoint{0.055556in}{0.000000in}}{%
\pgfpathmoveto{\pgfqpoint{0.000000in}{0.000000in}}%
\pgfpathlineto{\pgfqpoint{0.055556in}{0.000000in}}%
\pgfusepath{stroke,fill}%
}%
\begin{pgfscope}%
\pgfsys@transformshift{0.400000in}{0.773948in}%
\pgfsys@useobject{currentmarker}{}%
\end{pgfscope}%
\end{pgfscope}%
\begin{pgfscope}%
\pgfsetbuttcap%
\pgfsetroundjoin%
\definecolor{currentfill}{rgb}{0.000000,0.000000,0.000000}%
\pgfsetfillcolor{currentfill}%
\pgfsetlinewidth{0.501875pt}%
\definecolor{currentstroke}{rgb}{0.000000,0.000000,0.000000}%
\pgfsetstrokecolor{currentstroke}%
\pgfsetdash{}{0pt}%
\pgfsys@defobject{currentmarker}{\pgfqpoint{-0.055556in}{0.000000in}}{\pgfqpoint{0.000000in}{0.000000in}}{%
\pgfpathmoveto{\pgfqpoint{0.000000in}{0.000000in}}%
\pgfpathlineto{\pgfqpoint{-0.055556in}{0.000000in}}%
\pgfusepath{stroke,fill}%
}%
\begin{pgfscope}%
\pgfsys@transformshift{2.120000in}{0.773948in}%
\pgfsys@useobject{currentmarker}{}%
\end{pgfscope}%
\end{pgfscope}%
\begin{pgfscope}%
\pgfsetbuttcap%
\pgfsetroundjoin%
\definecolor{currentfill}{rgb}{0.000000,0.000000,0.000000}%
\pgfsetfillcolor{currentfill}%
\pgfsetlinewidth{0.501875pt}%
\definecolor{currentstroke}{rgb}{0.000000,0.000000,0.000000}%
\pgfsetstrokecolor{currentstroke}%
\pgfsetdash{}{0pt}%
\pgfsys@defobject{currentmarker}{\pgfqpoint{0.000000in}{0.000000in}}{\pgfqpoint{0.055556in}{0.000000in}}{%
\pgfpathmoveto{\pgfqpoint{0.000000in}{0.000000in}}%
\pgfpathlineto{\pgfqpoint{0.055556in}{0.000000in}}%
\pgfusepath{stroke,fill}%
}%
\begin{pgfscope}%
\pgfsys@transformshift{0.400000in}{1.018794in}%
\pgfsys@useobject{currentmarker}{}%
\end{pgfscope}%
\end{pgfscope}%
\begin{pgfscope}%
\pgfsetbuttcap%
\pgfsetroundjoin%
\definecolor{currentfill}{rgb}{0.000000,0.000000,0.000000}%
\pgfsetfillcolor{currentfill}%
\pgfsetlinewidth{0.501875pt}%
\definecolor{currentstroke}{rgb}{0.000000,0.000000,0.000000}%
\pgfsetstrokecolor{currentstroke}%
\pgfsetdash{}{0pt}%
\pgfsys@defobject{currentmarker}{\pgfqpoint{-0.055556in}{0.000000in}}{\pgfqpoint{0.000000in}{0.000000in}}{%
\pgfpathmoveto{\pgfqpoint{0.000000in}{0.000000in}}%
\pgfpathlineto{\pgfqpoint{-0.055556in}{0.000000in}}%
\pgfusepath{stroke,fill}%
}%
\begin{pgfscope}%
\pgfsys@transformshift{2.120000in}{1.018794in}%
\pgfsys@useobject{currentmarker}{}%
\end{pgfscope}%
\end{pgfscope}%
\begin{pgfscope}%
\pgfsetbuttcap%
\pgfsetroundjoin%
\definecolor{currentfill}{rgb}{0.000000,0.000000,0.000000}%
\pgfsetfillcolor{currentfill}%
\pgfsetlinewidth{0.501875pt}%
\definecolor{currentstroke}{rgb}{0.000000,0.000000,0.000000}%
\pgfsetstrokecolor{currentstroke}%
\pgfsetdash{}{0pt}%
\pgfsys@defobject{currentmarker}{\pgfqpoint{0.000000in}{0.000000in}}{\pgfqpoint{0.055556in}{0.000000in}}{%
\pgfpathmoveto{\pgfqpoint{0.000000in}{0.000000in}}%
\pgfpathlineto{\pgfqpoint{0.055556in}{0.000000in}}%
\pgfusepath{stroke,fill}%
}%
\begin{pgfscope}%
\pgfsys@transformshift{0.400000in}{1.263641in}%
\pgfsys@useobject{currentmarker}{}%
\end{pgfscope}%
\end{pgfscope}%
\begin{pgfscope}%
\pgfsetbuttcap%
\pgfsetroundjoin%
\definecolor{currentfill}{rgb}{0.000000,0.000000,0.000000}%
\pgfsetfillcolor{currentfill}%
\pgfsetlinewidth{0.501875pt}%
\definecolor{currentstroke}{rgb}{0.000000,0.000000,0.000000}%
\pgfsetstrokecolor{currentstroke}%
\pgfsetdash{}{0pt}%
\pgfsys@defobject{currentmarker}{\pgfqpoint{-0.055556in}{0.000000in}}{\pgfqpoint{0.000000in}{0.000000in}}{%
\pgfpathmoveto{\pgfqpoint{0.000000in}{0.000000in}}%
\pgfpathlineto{\pgfqpoint{-0.055556in}{0.000000in}}%
\pgfusepath{stroke,fill}%
}%
\begin{pgfscope}%
\pgfsys@transformshift{2.120000in}{1.263641in}%
\pgfsys@useobject{currentmarker}{}%
\end{pgfscope}%
\end{pgfscope}%
\begin{pgfscope}%
\pgfsetbuttcap%
\pgfsetroundjoin%
\definecolor{currentfill}{rgb}{0.000000,0.000000,0.000000}%
\pgfsetfillcolor{currentfill}%
\pgfsetlinewidth{0.501875pt}%
\definecolor{currentstroke}{rgb}{0.000000,0.000000,0.000000}%
\pgfsetstrokecolor{currentstroke}%
\pgfsetdash{}{0pt}%
\pgfsys@defobject{currentmarker}{\pgfqpoint{0.000000in}{0.000000in}}{\pgfqpoint{0.055556in}{0.000000in}}{%
\pgfpathmoveto{\pgfqpoint{0.000000in}{0.000000in}}%
\pgfpathlineto{\pgfqpoint{0.055556in}{0.000000in}}%
\pgfusepath{stroke,fill}%
}%
\begin{pgfscope}%
\pgfsys@transformshift{0.400000in}{1.508487in}%
\pgfsys@useobject{currentmarker}{}%
\end{pgfscope}%
\end{pgfscope}%
\begin{pgfscope}%
\pgfsetbuttcap%
\pgfsetroundjoin%
\definecolor{currentfill}{rgb}{0.000000,0.000000,0.000000}%
\pgfsetfillcolor{currentfill}%
\pgfsetlinewidth{0.501875pt}%
\definecolor{currentstroke}{rgb}{0.000000,0.000000,0.000000}%
\pgfsetstrokecolor{currentstroke}%
\pgfsetdash{}{0pt}%
\pgfsys@defobject{currentmarker}{\pgfqpoint{-0.055556in}{0.000000in}}{\pgfqpoint{0.000000in}{0.000000in}}{%
\pgfpathmoveto{\pgfqpoint{0.000000in}{0.000000in}}%
\pgfpathlineto{\pgfqpoint{-0.055556in}{0.000000in}}%
\pgfusepath{stroke,fill}%
}%
\begin{pgfscope}%
\pgfsys@transformshift{2.120000in}{1.508487in}%
\pgfsys@useobject{currentmarker}{}%
\end{pgfscope}%
\end{pgfscope}%
\begin{pgfscope}%
\pgfsetbuttcap%
\pgfsetroundjoin%
\definecolor{currentfill}{rgb}{0.000000,0.000000,0.000000}%
\pgfsetfillcolor{currentfill}%
\pgfsetlinewidth{0.501875pt}%
\definecolor{currentstroke}{rgb}{0.000000,0.000000,0.000000}%
\pgfsetstrokecolor{currentstroke}%
\pgfsetdash{}{0pt}%
\pgfsys@defobject{currentmarker}{\pgfqpoint{0.000000in}{0.000000in}}{\pgfqpoint{0.055556in}{0.000000in}}{%
\pgfpathmoveto{\pgfqpoint{0.000000in}{0.000000in}}%
\pgfpathlineto{\pgfqpoint{0.055556in}{0.000000in}}%
\pgfusepath{stroke,fill}%
}%
\begin{pgfscope}%
\pgfsys@transformshift{0.400000in}{1.753333in}%
\pgfsys@useobject{currentmarker}{}%
\end{pgfscope}%
\end{pgfscope}%
\begin{pgfscope}%
\pgfsetbuttcap%
\pgfsetroundjoin%
\definecolor{currentfill}{rgb}{0.000000,0.000000,0.000000}%
\pgfsetfillcolor{currentfill}%
\pgfsetlinewidth{0.501875pt}%
\definecolor{currentstroke}{rgb}{0.000000,0.000000,0.000000}%
\pgfsetstrokecolor{currentstroke}%
\pgfsetdash{}{0pt}%
\pgfsys@defobject{currentmarker}{\pgfqpoint{-0.055556in}{0.000000in}}{\pgfqpoint{0.000000in}{0.000000in}}{%
\pgfpathmoveto{\pgfqpoint{0.000000in}{0.000000in}}%
\pgfpathlineto{\pgfqpoint{-0.055556in}{0.000000in}}%
\pgfusepath{stroke,fill}%
}%
\begin{pgfscope}%
\pgfsys@transformshift{2.120000in}{1.753333in}%
\pgfsys@useobject{currentmarker}{}%
\end{pgfscope}%
\end{pgfscope}%
\begin{pgfscope}%
\pgftext[x=0.275000in,y=1.141217in,,bottom,rotate=90.000000]{\fontsize{10.000000}{12.000000}\selectfont Particle Count}%
\end{pgfscope}%
\begin{pgfscope}%
\pgfsetbuttcap%
\pgfsetmiterjoin%
\definecolor{currentfill}{rgb}{1.000000,1.000000,1.000000}%
\pgfsetfillcolor{currentfill}%
\pgfsetlinewidth{0.000000pt}%
\definecolor{currentstroke}{rgb}{0.000000,0.000000,0.000000}%
\pgfsetstrokecolor{currentstroke}%
\pgfsetstrokeopacity{0.000000}%
\pgfsetdash{}{0pt}%
\pgfpathmoveto{\pgfqpoint{2.120000in}{0.529102in}}%
\pgfpathlineto{\pgfqpoint{3.840000in}{0.529102in}}%
\pgfpathlineto{\pgfqpoint{3.840000in}{1.753333in}}%
\pgfpathlineto{\pgfqpoint{2.120000in}{1.753333in}}%
\pgfpathclose%
\pgfusepath{fill}%
\end{pgfscope}%
\begin{pgfscope}%
\pgfpathrectangle{\pgfqpoint{2.120000in}{0.529102in}}{\pgfqpoint{1.720000in}{1.224232in}} %
\pgfusepath{clip}%
\pgfsetrectcap%
\pgfsetroundjoin%
\pgfsetlinewidth{1.003750pt}%
\definecolor{currentstroke}{rgb}{0.309804,0.478431,0.682353}%
\pgfsetstrokecolor{currentstroke}%
\pgfsetstrokeopacity{0.500000}%
\pgfsetdash{}{0pt}%
\pgfpathmoveto{\pgfqpoint{2.359646in}{0.531797in}}%
\pgfpathlineto{\pgfqpoint{2.361682in}{0.532329in}}%
\pgfpathlineto{\pgfqpoint{2.363717in}{0.532951in}}%
\pgfpathlineto{\pgfqpoint{2.365752in}{0.533676in}}%
\pgfpathlineto{\pgfqpoint{2.367787in}{0.534519in}}%
\pgfpathlineto{\pgfqpoint{2.369822in}{0.535492in}}%
\pgfpathlineto{\pgfqpoint{2.371857in}{0.536613in}}%
\pgfpathlineto{\pgfqpoint{2.373892in}{0.537898in}}%
\pgfpathlineto{\pgfqpoint{2.375927in}{0.539366in}}%
\pgfpathlineto{\pgfqpoint{2.377963in}{0.541034in}}%
\pgfpathlineto{\pgfqpoint{2.379998in}{0.542922in}}%
\pgfpathlineto{\pgfqpoint{2.382033in}{0.545051in}}%
\pgfpathlineto{\pgfqpoint{2.384068in}{0.547440in}}%
\pgfpathlineto{\pgfqpoint{2.386103in}{0.550110in}}%
\pgfpathlineto{\pgfqpoint{2.388138in}{0.553080in}}%
\pgfpathlineto{\pgfqpoint{2.390173in}{0.556369in}}%
\pgfpathlineto{\pgfqpoint{2.392209in}{0.559996in}}%
\pgfpathlineto{\pgfqpoint{2.394244in}{0.563977in}}%
\pgfpathlineto{\pgfqpoint{2.396279in}{0.568327in}}%
\pgfpathlineto{\pgfqpoint{2.398314in}{0.573058in}}%
\pgfpathlineto{\pgfqpoint{2.400349in}{0.578178in}}%
\pgfpathlineto{\pgfqpoint{2.402384in}{0.583695in}}%
\pgfpathlineto{\pgfqpoint{2.404419in}{0.589608in}}%
\pgfpathlineto{\pgfqpoint{2.406454in}{0.595917in}}%
\pgfpathlineto{\pgfqpoint{2.408490in}{0.602612in}}%
\pgfpathlineto{\pgfqpoint{2.410525in}{0.609682in}}%
\pgfpathlineto{\pgfqpoint{2.412560in}{0.617108in}}%
\pgfpathlineto{\pgfqpoint{2.414595in}{0.624866in}}%
\pgfpathlineto{\pgfqpoint{2.416630in}{0.632926in}}%
\pgfpathlineto{\pgfqpoint{2.418665in}{0.641251in}}%
\pgfpathlineto{\pgfqpoint{2.420700in}{0.649800in}}%
\pgfpathlineto{\pgfqpoint{2.422736in}{0.658524in}}%
\pgfpathlineto{\pgfqpoint{2.424771in}{0.667370in}}%
\pgfpathlineto{\pgfqpoint{2.426806in}{0.676280in}}%
\pgfpathlineto{\pgfqpoint{2.428841in}{0.685188in}}%
\pgfpathlineto{\pgfqpoint{2.430876in}{0.694030in}}%
\pgfpathlineto{\pgfqpoint{2.432911in}{0.702733in}}%
\pgfpathlineto{\pgfqpoint{2.434946in}{0.711225in}}%
\pgfpathlineto{\pgfqpoint{2.436981in}{0.719432in}}%
\pgfpathlineto{\pgfqpoint{2.439017in}{0.727279in}}%
\pgfpathlineto{\pgfqpoint{2.441052in}{0.734694in}}%
\pgfpathlineto{\pgfqpoint{2.443087in}{0.741604in}}%
\pgfpathlineto{\pgfqpoint{2.445122in}{0.747941in}}%
\pgfpathlineto{\pgfqpoint{2.447157in}{0.753640in}}%
\pgfpathlineto{\pgfqpoint{2.449192in}{0.758644in}}%
\pgfpathlineto{\pgfqpoint{2.451227in}{0.762898in}}%
\pgfpathlineto{\pgfqpoint{2.453263in}{0.766359in}}%
\pgfpathlineto{\pgfqpoint{2.455298in}{0.768988in}}%
\pgfpathlineto{\pgfqpoint{2.457333in}{0.770756in}}%
\pgfpathlineto{\pgfqpoint{2.459368in}{0.771646in}}%
\pgfpathlineto{\pgfqpoint{2.461403in}{0.771646in}}%
\pgfpathlineto{\pgfqpoint{2.463438in}{0.770756in}}%
\pgfpathlineto{\pgfqpoint{2.465473in}{0.768988in}}%
\pgfpathlineto{\pgfqpoint{2.467508in}{0.766359in}}%
\pgfpathlineto{\pgfqpoint{2.469544in}{0.762898in}}%
\pgfpathlineto{\pgfqpoint{2.471579in}{0.758644in}}%
\pgfpathlineto{\pgfqpoint{2.473614in}{0.753640in}}%
\pgfpathlineto{\pgfqpoint{2.475649in}{0.747941in}}%
\pgfpathlineto{\pgfqpoint{2.477684in}{0.741604in}}%
\pgfpathlineto{\pgfqpoint{2.479719in}{0.734694in}}%
\pgfpathlineto{\pgfqpoint{2.481754in}{0.727279in}}%
\pgfpathlineto{\pgfqpoint{2.483789in}{0.719432in}}%
\pgfpathlineto{\pgfqpoint{2.485825in}{0.711225in}}%
\pgfpathlineto{\pgfqpoint{2.487860in}{0.702733in}}%
\pgfpathlineto{\pgfqpoint{2.489895in}{0.694030in}}%
\pgfpathlineto{\pgfqpoint{2.491930in}{0.685188in}}%
\pgfpathlineto{\pgfqpoint{2.493965in}{0.676280in}}%
\pgfpathlineto{\pgfqpoint{2.496000in}{0.667370in}}%
\pgfpathlineto{\pgfqpoint{2.498035in}{0.658524in}}%
\pgfpathlineto{\pgfqpoint{2.500071in}{0.649800in}}%
\pgfpathlineto{\pgfqpoint{2.502106in}{0.641251in}}%
\pgfpathlineto{\pgfqpoint{2.504141in}{0.632926in}}%
\pgfpathlineto{\pgfqpoint{2.506176in}{0.624866in}}%
\pgfpathlineto{\pgfqpoint{2.508211in}{0.617108in}}%
\pgfpathlineto{\pgfqpoint{2.510246in}{0.609682in}}%
\pgfpathlineto{\pgfqpoint{2.512281in}{0.602612in}}%
\pgfpathlineto{\pgfqpoint{2.514316in}{0.595917in}}%
\pgfpathlineto{\pgfqpoint{2.516352in}{0.589608in}}%
\pgfpathlineto{\pgfqpoint{2.518387in}{0.583695in}}%
\pgfpathlineto{\pgfqpoint{2.520422in}{0.578178in}}%
\pgfpathlineto{\pgfqpoint{2.522457in}{0.573058in}}%
\pgfpathlineto{\pgfqpoint{2.524492in}{0.568327in}}%
\pgfpathlineto{\pgfqpoint{2.526527in}{0.563977in}}%
\pgfpathlineto{\pgfqpoint{2.528562in}{0.559996in}}%
\pgfpathlineto{\pgfqpoint{2.530598in}{0.556369in}}%
\pgfpathlineto{\pgfqpoint{2.532633in}{0.553080in}}%
\pgfpathlineto{\pgfqpoint{2.534668in}{0.550110in}}%
\pgfpathlineto{\pgfqpoint{2.536703in}{0.547440in}}%
\pgfpathlineto{\pgfqpoint{2.538738in}{0.545051in}}%
\pgfpathlineto{\pgfqpoint{2.540773in}{0.542922in}}%
\pgfpathlineto{\pgfqpoint{2.542808in}{0.541034in}}%
\pgfpathlineto{\pgfqpoint{2.544843in}{0.539366in}}%
\pgfpathlineto{\pgfqpoint{2.546879in}{0.537898in}}%
\pgfpathlineto{\pgfqpoint{2.548914in}{0.536613in}}%
\pgfpathlineto{\pgfqpoint{2.550949in}{0.535492in}}%
\pgfpathlineto{\pgfqpoint{2.552984in}{0.534519in}}%
\pgfpathlineto{\pgfqpoint{2.555019in}{0.533676in}}%
\pgfpathlineto{\pgfqpoint{2.557054in}{0.532951in}}%
\pgfpathlineto{\pgfqpoint{2.559089in}{0.532329in}}%
\pgfpathlineto{\pgfqpoint{2.561125in}{0.531797in}}%
\pgfusepath{stroke}%
\end{pgfscope}%
\begin{pgfscope}%
\pgfpathrectangle{\pgfqpoint{2.120000in}{0.529102in}}{\pgfqpoint{1.720000in}{1.224232in}} %
\pgfusepath{clip}%
\pgfsetrectcap%
\pgfsetroundjoin%
\pgfsetlinewidth{1.003750pt}%
\definecolor{currentstroke}{rgb}{0.309804,0.478431,0.682353}%
\pgfsetstrokecolor{currentstroke}%
\pgfsetstrokeopacity{0.500000}%
\pgfsetdash{}{0pt}%
\pgfpathmoveto{\pgfqpoint{2.481996in}{0.534407in}}%
\pgfpathlineto{\pgfqpoint{2.484133in}{0.535453in}}%
\pgfpathlineto{\pgfqpoint{2.486269in}{0.536677in}}%
\pgfpathlineto{\pgfqpoint{2.488406in}{0.538105in}}%
\pgfpathlineto{\pgfqpoint{2.490542in}{0.539762in}}%
\pgfpathlineto{\pgfqpoint{2.492679in}{0.541678in}}%
\pgfpathlineto{\pgfqpoint{2.494815in}{0.543884in}}%
\pgfpathlineto{\pgfqpoint{2.496952in}{0.546413in}}%
\pgfpathlineto{\pgfqpoint{2.499088in}{0.549301in}}%
\pgfpathlineto{\pgfqpoint{2.501225in}{0.552584in}}%
\pgfpathlineto{\pgfqpoint{2.503361in}{0.556301in}}%
\pgfpathlineto{\pgfqpoint{2.505497in}{0.560490in}}%
\pgfpathlineto{\pgfqpoint{2.507634in}{0.565191in}}%
\pgfpathlineto{\pgfqpoint{2.509770in}{0.570445in}}%
\pgfpathlineto{\pgfqpoint{2.511907in}{0.576290in}}%
\pgfpathlineto{\pgfqpoint{2.514043in}{0.582764in}}%
\pgfpathlineto{\pgfqpoint{2.516180in}{0.589902in}}%
\pgfpathlineto{\pgfqpoint{2.518316in}{0.597737in}}%
\pgfpathlineto{\pgfqpoint{2.520453in}{0.606297in}}%
\pgfpathlineto{\pgfqpoint{2.522589in}{0.615607in}}%
\pgfpathlineto{\pgfqpoint{2.524726in}{0.625685in}}%
\pgfpathlineto{\pgfqpoint{2.526862in}{0.636540in}}%
\pgfpathlineto{\pgfqpoint{2.528999in}{0.648178in}}%
\pgfpathlineto{\pgfqpoint{2.531135in}{0.660593in}}%
\pgfpathlineto{\pgfqpoint{2.533272in}{0.673770in}}%
\pgfpathlineto{\pgfqpoint{2.535408in}{0.687683in}}%
\pgfpathlineto{\pgfqpoint{2.537545in}{0.702298in}}%
\pgfpathlineto{\pgfqpoint{2.539681in}{0.717565in}}%
\pgfpathlineto{\pgfqpoint{2.541818in}{0.733427in}}%
\pgfpathlineto{\pgfqpoint{2.543954in}{0.749811in}}%
\pgfpathlineto{\pgfqpoint{2.546091in}{0.766635in}}%
\pgfpathlineto{\pgfqpoint{2.548227in}{0.783805in}}%
\pgfpathlineto{\pgfqpoint{2.550363in}{0.801214in}}%
\pgfpathlineto{\pgfqpoint{2.552500in}{0.818747in}}%
\pgfpathlineto{\pgfqpoint{2.554636in}{0.836279in}}%
\pgfpathlineto{\pgfqpoint{2.556773in}{0.853679in}}%
\pgfpathlineto{\pgfqpoint{2.558909in}{0.870806in}}%
\pgfpathlineto{\pgfqpoint{2.561046in}{0.887518in}}%
\pgfpathlineto{\pgfqpoint{2.563182in}{0.903670in}}%
\pgfpathlineto{\pgfqpoint{2.565319in}{0.919114in}}%
\pgfpathlineto{\pgfqpoint{2.567455in}{0.933706in}}%
\pgfpathlineto{\pgfqpoint{2.569592in}{0.947305in}}%
\pgfpathlineto{\pgfqpoint{2.571728in}{0.959776in}}%
\pgfpathlineto{\pgfqpoint{2.573865in}{0.970993in}}%
\pgfpathlineto{\pgfqpoint{2.576001in}{0.980839in}}%
\pgfpathlineto{\pgfqpoint{2.578138in}{0.989212in}}%
\pgfpathlineto{\pgfqpoint{2.580274in}{0.996022in}}%
\pgfpathlineto{\pgfqpoint{2.582411in}{1.001196in}}%
\pgfpathlineto{\pgfqpoint{2.584547in}{1.004677in}}%
\pgfpathlineto{\pgfqpoint{2.586684in}{1.006427in}}%
\pgfpathlineto{\pgfqpoint{2.588820in}{1.006427in}}%
\pgfpathlineto{\pgfqpoint{2.590956in}{1.004677in}}%
\pgfpathlineto{\pgfqpoint{2.593093in}{1.001196in}}%
\pgfpathlineto{\pgfqpoint{2.595229in}{0.996022in}}%
\pgfpathlineto{\pgfqpoint{2.597366in}{0.989212in}}%
\pgfpathlineto{\pgfqpoint{2.599502in}{0.980839in}}%
\pgfpathlineto{\pgfqpoint{2.601639in}{0.970993in}}%
\pgfpathlineto{\pgfqpoint{2.603775in}{0.959776in}}%
\pgfpathlineto{\pgfqpoint{2.605912in}{0.947305in}}%
\pgfpathlineto{\pgfqpoint{2.608048in}{0.933706in}}%
\pgfpathlineto{\pgfqpoint{2.610185in}{0.919114in}}%
\pgfpathlineto{\pgfqpoint{2.612321in}{0.903670in}}%
\pgfpathlineto{\pgfqpoint{2.614458in}{0.887518in}}%
\pgfpathlineto{\pgfqpoint{2.616594in}{0.870806in}}%
\pgfpathlineto{\pgfqpoint{2.618731in}{0.853679in}}%
\pgfpathlineto{\pgfqpoint{2.620867in}{0.836279in}}%
\pgfpathlineto{\pgfqpoint{2.623004in}{0.818747in}}%
\pgfpathlineto{\pgfqpoint{2.625140in}{0.801214in}}%
\pgfpathlineto{\pgfqpoint{2.627277in}{0.783805in}}%
\pgfpathlineto{\pgfqpoint{2.629413in}{0.766635in}}%
\pgfpathlineto{\pgfqpoint{2.631550in}{0.749811in}}%
\pgfpathlineto{\pgfqpoint{2.633686in}{0.733427in}}%
\pgfpathlineto{\pgfqpoint{2.635822in}{0.717565in}}%
\pgfpathlineto{\pgfqpoint{2.637959in}{0.702298in}}%
\pgfpathlineto{\pgfqpoint{2.640095in}{0.687683in}}%
\pgfpathlineto{\pgfqpoint{2.642232in}{0.673770in}}%
\pgfpathlineto{\pgfqpoint{2.644368in}{0.660593in}}%
\pgfpathlineto{\pgfqpoint{2.646505in}{0.648178in}}%
\pgfpathlineto{\pgfqpoint{2.648641in}{0.636540in}}%
\pgfpathlineto{\pgfqpoint{2.650778in}{0.625685in}}%
\pgfpathlineto{\pgfqpoint{2.652914in}{0.615607in}}%
\pgfpathlineto{\pgfqpoint{2.655051in}{0.606297in}}%
\pgfpathlineto{\pgfqpoint{2.657187in}{0.597737in}}%
\pgfpathlineto{\pgfqpoint{2.659324in}{0.589902in}}%
\pgfpathlineto{\pgfqpoint{2.661460in}{0.582764in}}%
\pgfpathlineto{\pgfqpoint{2.663597in}{0.576290in}}%
\pgfpathlineto{\pgfqpoint{2.665733in}{0.570445in}}%
\pgfpathlineto{\pgfqpoint{2.667870in}{0.565191in}}%
\pgfpathlineto{\pgfqpoint{2.670006in}{0.560490in}}%
\pgfpathlineto{\pgfqpoint{2.672143in}{0.556301in}}%
\pgfpathlineto{\pgfqpoint{2.674279in}{0.552584in}}%
\pgfpathlineto{\pgfqpoint{2.676415in}{0.549301in}}%
\pgfpathlineto{\pgfqpoint{2.678552in}{0.546413in}}%
\pgfpathlineto{\pgfqpoint{2.680688in}{0.543884in}}%
\pgfpathlineto{\pgfqpoint{2.682825in}{0.541678in}}%
\pgfpathlineto{\pgfqpoint{2.684961in}{0.539762in}}%
\pgfpathlineto{\pgfqpoint{2.687098in}{0.538105in}}%
\pgfpathlineto{\pgfqpoint{2.689234in}{0.536677in}}%
\pgfpathlineto{\pgfqpoint{2.691371in}{0.535453in}}%
\pgfpathlineto{\pgfqpoint{2.693507in}{0.534407in}}%
\pgfusepath{stroke}%
\end{pgfscope}%
\begin{pgfscope}%
\pgfpathrectangle{\pgfqpoint{2.120000in}{0.529102in}}{\pgfqpoint{1.720000in}{1.224232in}} %
\pgfusepath{clip}%
\pgfsetrectcap%
\pgfsetroundjoin%
\pgfsetlinewidth{1.003750pt}%
\definecolor{currentstroke}{rgb}{0.309804,0.478431,0.682353}%
\pgfsetstrokecolor{currentstroke}%
\pgfsetstrokeopacity{0.500000}%
\pgfsetdash{}{0pt}%
\pgfpathmoveto{\pgfqpoint{2.611125in}{0.537739in}}%
\pgfpathlineto{\pgfqpoint{2.613301in}{0.539443in}}%
\pgfpathlineto{\pgfqpoint{2.615478in}{0.541437in}}%
\pgfpathlineto{\pgfqpoint{2.617655in}{0.543761in}}%
\pgfpathlineto{\pgfqpoint{2.619831in}{0.546459in}}%
\pgfpathlineto{\pgfqpoint{2.622008in}{0.549579in}}%
\pgfpathlineto{\pgfqpoint{2.624184in}{0.553171in}}%
\pgfpathlineto{\pgfqpoint{2.626361in}{0.557289in}}%
\pgfpathlineto{\pgfqpoint{2.628538in}{0.561991in}}%
\pgfpathlineto{\pgfqpoint{2.630714in}{0.567337in}}%
\pgfpathlineto{\pgfqpoint{2.632891in}{0.573388in}}%
\pgfpathlineto{\pgfqpoint{2.635068in}{0.580209in}}%
\pgfpathlineto{\pgfqpoint{2.637244in}{0.587864in}}%
\pgfpathlineto{\pgfqpoint{2.639421in}{0.596419in}}%
\pgfpathlineto{\pgfqpoint{2.641598in}{0.605935in}}%
\pgfpathlineto{\pgfqpoint{2.643774in}{0.616476in}}%
\pgfpathlineto{\pgfqpoint{2.645951in}{0.628098in}}%
\pgfpathlineto{\pgfqpoint{2.648128in}{0.640855in}}%
\pgfpathlineto{\pgfqpoint{2.650304in}{0.654794in}}%
\pgfpathlineto{\pgfqpoint{2.652481in}{0.669953in}}%
\pgfpathlineto{\pgfqpoint{2.654658in}{0.686361in}}%
\pgfpathlineto{\pgfqpoint{2.656834in}{0.704037in}}%
\pgfpathlineto{\pgfqpoint{2.659011in}{0.722986in}}%
\pgfpathlineto{\pgfqpoint{2.661188in}{0.743200in}}%
\pgfpathlineto{\pgfqpoint{2.663364in}{0.764655in}}%
\pgfpathlineto{\pgfqpoint{2.665541in}{0.787309in}}%
\pgfpathlineto{\pgfqpoint{2.667718in}{0.811105in}}%
\pgfpathlineto{\pgfqpoint{2.669894in}{0.835964in}}%
\pgfpathlineto{\pgfqpoint{2.672071in}{0.861791in}}%
\pgfpathlineto{\pgfqpoint{2.674248in}{0.888468in}}%
\pgfpathlineto{\pgfqpoint{2.676424in}{0.915862in}}%
\pgfpathlineto{\pgfqpoint{2.678601in}{0.943817in}}%
\pgfpathlineto{\pgfqpoint{2.680778in}{0.972163in}}%
\pgfpathlineto{\pgfqpoint{2.682954in}{1.000711in}}%
\pgfpathlineto{\pgfqpoint{2.685131in}{1.029258in}}%
\pgfpathlineto{\pgfqpoint{2.687308in}{1.057588in}}%
\pgfpathlineto{\pgfqpoint{2.689484in}{1.085476in}}%
\pgfpathlineto{\pgfqpoint{2.691661in}{1.112687in}}%
\pgfpathlineto{\pgfqpoint{2.693838in}{1.138985in}}%
\pgfpathlineto{\pgfqpoint{2.696014in}{1.164132in}}%
\pgfpathlineto{\pgfqpoint{2.698191in}{1.187891in}}%
\pgfpathlineto{\pgfqpoint{2.700368in}{1.210033in}}%
\pgfpathlineto{\pgfqpoint{2.702544in}{1.230339in}}%
\pgfpathlineto{\pgfqpoint{2.704721in}{1.248603in}}%
\pgfpathlineto{\pgfqpoint{2.706898in}{1.264635in}}%
\pgfpathlineto{\pgfqpoint{2.709074in}{1.278269in}}%
\pgfpathlineto{\pgfqpoint{2.711251in}{1.289357in}}%
\pgfpathlineto{\pgfqpoint{2.713428in}{1.297781in}}%
\pgfpathlineto{\pgfqpoint{2.715604in}{1.303448in}}%
\pgfpathlineto{\pgfqpoint{2.717781in}{1.306298in}}%
\pgfpathlineto{\pgfqpoint{2.719958in}{1.306298in}}%
\pgfpathlineto{\pgfqpoint{2.722134in}{1.303448in}}%
\pgfpathlineto{\pgfqpoint{2.724311in}{1.297781in}}%
\pgfpathlineto{\pgfqpoint{2.726488in}{1.289357in}}%
\pgfpathlineto{\pgfqpoint{2.728664in}{1.278269in}}%
\pgfpathlineto{\pgfqpoint{2.730841in}{1.264635in}}%
\pgfpathlineto{\pgfqpoint{2.733018in}{1.248603in}}%
\pgfpathlineto{\pgfqpoint{2.735194in}{1.230339in}}%
\pgfpathlineto{\pgfqpoint{2.737371in}{1.210033in}}%
\pgfpathlineto{\pgfqpoint{2.739548in}{1.187891in}}%
\pgfpathlineto{\pgfqpoint{2.741724in}{1.164132in}}%
\pgfpathlineto{\pgfqpoint{2.743901in}{1.138985in}}%
\pgfpathlineto{\pgfqpoint{2.746078in}{1.112687in}}%
\pgfpathlineto{\pgfqpoint{2.748254in}{1.085476in}}%
\pgfpathlineto{\pgfqpoint{2.750431in}{1.057588in}}%
\pgfpathlineto{\pgfqpoint{2.752608in}{1.029258in}}%
\pgfpathlineto{\pgfqpoint{2.754784in}{1.000711in}}%
\pgfpathlineto{\pgfqpoint{2.756961in}{0.972163in}}%
\pgfpathlineto{\pgfqpoint{2.759138in}{0.943817in}}%
\pgfpathlineto{\pgfqpoint{2.761314in}{0.915862in}}%
\pgfpathlineto{\pgfqpoint{2.763491in}{0.888468in}}%
\pgfpathlineto{\pgfqpoint{2.765668in}{0.861791in}}%
\pgfpathlineto{\pgfqpoint{2.767844in}{0.835964in}}%
\pgfpathlineto{\pgfqpoint{2.770021in}{0.811105in}}%
\pgfpathlineto{\pgfqpoint{2.772198in}{0.787309in}}%
\pgfpathlineto{\pgfqpoint{2.774374in}{0.764655in}}%
\pgfpathlineto{\pgfqpoint{2.776551in}{0.743200in}}%
\pgfpathlineto{\pgfqpoint{2.778728in}{0.722986in}}%
\pgfpathlineto{\pgfqpoint{2.780904in}{0.704037in}}%
\pgfpathlineto{\pgfqpoint{2.783081in}{0.686361in}}%
\pgfpathlineto{\pgfqpoint{2.785258in}{0.669953in}}%
\pgfpathlineto{\pgfqpoint{2.787434in}{0.654794in}}%
\pgfpathlineto{\pgfqpoint{2.789611in}{0.640855in}}%
\pgfpathlineto{\pgfqpoint{2.791788in}{0.628098in}}%
\pgfpathlineto{\pgfqpoint{2.793964in}{0.616476in}}%
\pgfpathlineto{\pgfqpoint{2.796141in}{0.605935in}}%
\pgfpathlineto{\pgfqpoint{2.798318in}{0.596419in}}%
\pgfpathlineto{\pgfqpoint{2.800494in}{0.587864in}}%
\pgfpathlineto{\pgfqpoint{2.802671in}{0.580209in}}%
\pgfpathlineto{\pgfqpoint{2.804848in}{0.573388in}}%
\pgfpathlineto{\pgfqpoint{2.807024in}{0.567337in}}%
\pgfpathlineto{\pgfqpoint{2.809201in}{0.561991in}}%
\pgfpathlineto{\pgfqpoint{2.811378in}{0.557289in}}%
\pgfpathlineto{\pgfqpoint{2.813554in}{0.553171in}}%
\pgfpathlineto{\pgfqpoint{2.815731in}{0.549579in}}%
\pgfpathlineto{\pgfqpoint{2.817908in}{0.546459in}}%
\pgfpathlineto{\pgfqpoint{2.820084in}{0.543761in}}%
\pgfpathlineto{\pgfqpoint{2.822261in}{0.541437in}}%
\pgfpathlineto{\pgfqpoint{2.824438in}{0.539443in}}%
\pgfpathlineto{\pgfqpoint{2.826614in}{0.537739in}}%
\pgfusepath{stroke}%
\end{pgfscope}%
\begin{pgfscope}%
\pgfpathrectangle{\pgfqpoint{2.120000in}{0.529102in}}{\pgfqpoint{1.720000in}{1.224232in}} %
\pgfusepath{clip}%
\pgfsetrectcap%
\pgfsetroundjoin%
\pgfsetlinewidth{1.003750pt}%
\definecolor{currentstroke}{rgb}{0.309804,0.478431,0.682353}%
\pgfsetstrokecolor{currentstroke}%
\pgfsetstrokeopacity{0.500000}%
\pgfsetdash{}{0pt}%
\pgfpathmoveto{\pgfqpoint{2.742947in}{0.540696in}}%
\pgfpathlineto{\pgfqpoint{2.745116in}{0.542982in}}%
\pgfpathlineto{\pgfqpoint{2.747284in}{0.545659in}}%
\pgfpathlineto{\pgfqpoint{2.749453in}{0.548778in}}%
\pgfpathlineto{\pgfqpoint{2.751622in}{0.552400in}}%
\pgfpathlineto{\pgfqpoint{2.753791in}{0.556588in}}%
\pgfpathlineto{\pgfqpoint{2.755960in}{0.561409in}}%
\pgfpathlineto{\pgfqpoint{2.758129in}{0.566937in}}%
\pgfpathlineto{\pgfqpoint{2.760297in}{0.573249in}}%
\pgfpathlineto{\pgfqpoint{2.762466in}{0.580424in}}%
\pgfpathlineto{\pgfqpoint{2.764635in}{0.588546in}}%
\pgfpathlineto{\pgfqpoint{2.766804in}{0.597702in}}%
\pgfpathlineto{\pgfqpoint{2.768973in}{0.607977in}}%
\pgfpathlineto{\pgfqpoint{2.771142in}{0.619459in}}%
\pgfpathlineto{\pgfqpoint{2.773310in}{0.632234in}}%
\pgfpathlineto{\pgfqpoint{2.775479in}{0.646382in}}%
\pgfpathlineto{\pgfqpoint{2.777648in}{0.661982in}}%
\pgfpathlineto{\pgfqpoint{2.779817in}{0.679106in}}%
\pgfpathlineto{\pgfqpoint{2.781986in}{0.697815in}}%
\pgfpathlineto{\pgfqpoint{2.784155in}{0.718163in}}%
\pgfpathlineto{\pgfqpoint{2.786323in}{0.740187in}}%
\pgfpathlineto{\pgfqpoint{2.788492in}{0.763913in}}%
\pgfpathlineto{\pgfqpoint{2.790661in}{0.789348in}}%
\pgfpathlineto{\pgfqpoint{2.792830in}{0.816481in}}%
\pgfpathlineto{\pgfqpoint{2.794999in}{0.845279in}}%
\pgfpathlineto{\pgfqpoint{2.797168in}{0.875688in}}%
\pgfpathlineto{\pgfqpoint{2.799337in}{0.907628in}}%
\pgfpathlineto{\pgfqpoint{2.801505in}{0.940996in}}%
\pgfpathlineto{\pgfqpoint{2.803674in}{0.975662in}}%
\pgfpathlineto{\pgfqpoint{2.805843in}{1.011471in}}%
\pgfpathlineto{\pgfqpoint{2.808012in}{1.048240in}}%
\pgfpathlineto{\pgfqpoint{2.810181in}{1.085765in}}%
\pgfpathlineto{\pgfqpoint{2.812350in}{1.123813in}}%
\pgfpathlineto{\pgfqpoint{2.814518in}{1.162132in}}%
\pgfpathlineto{\pgfqpoint{2.816687in}{1.200450in}}%
\pgfpathlineto{\pgfqpoint{2.818856in}{1.238477in}}%
\pgfpathlineto{\pgfqpoint{2.821025in}{1.275909in}}%
\pgfpathlineto{\pgfqpoint{2.823194in}{1.312435in}}%
\pgfpathlineto{\pgfqpoint{2.825363in}{1.347734in}}%
\pgfpathlineto{\pgfqpoint{2.827531in}{1.381487in}}%
\pgfpathlineto{\pgfqpoint{2.829700in}{1.413378in}}%
\pgfpathlineto{\pgfqpoint{2.831869in}{1.443099in}}%
\pgfpathlineto{\pgfqpoint{2.834038in}{1.470355in}}%
\pgfpathlineto{\pgfqpoint{2.836207in}{1.494870in}}%
\pgfpathlineto{\pgfqpoint{2.838376in}{1.516391in}}%
\pgfpathlineto{\pgfqpoint{2.840544in}{1.534690in}}%
\pgfpathlineto{\pgfqpoint{2.842713in}{1.549574in}}%
\pgfpathlineto{\pgfqpoint{2.844882in}{1.560881in}}%
\pgfpathlineto{\pgfqpoint{2.847051in}{1.568488in}}%
\pgfpathlineto{\pgfqpoint{2.849220in}{1.572313in}}%
\pgfpathlineto{\pgfqpoint{2.851389in}{1.572313in}}%
\pgfpathlineto{\pgfqpoint{2.853558in}{1.568488in}}%
\pgfpathlineto{\pgfqpoint{2.855726in}{1.560881in}}%
\pgfpathlineto{\pgfqpoint{2.857895in}{1.549574in}}%
\pgfpathlineto{\pgfqpoint{2.860064in}{1.534690in}}%
\pgfpathlineto{\pgfqpoint{2.862233in}{1.516391in}}%
\pgfpathlineto{\pgfqpoint{2.864402in}{1.494870in}}%
\pgfpathlineto{\pgfqpoint{2.866571in}{1.470355in}}%
\pgfpathlineto{\pgfqpoint{2.868739in}{1.443099in}}%
\pgfpathlineto{\pgfqpoint{2.870908in}{1.413378in}}%
\pgfpathlineto{\pgfqpoint{2.873077in}{1.381487in}}%
\pgfpathlineto{\pgfqpoint{2.875246in}{1.347734in}}%
\pgfpathlineto{\pgfqpoint{2.877415in}{1.312435in}}%
\pgfpathlineto{\pgfqpoint{2.879584in}{1.275909in}}%
\pgfpathlineto{\pgfqpoint{2.881752in}{1.238477in}}%
\pgfpathlineto{\pgfqpoint{2.883921in}{1.200450in}}%
\pgfpathlineto{\pgfqpoint{2.886090in}{1.162132in}}%
\pgfpathlineto{\pgfqpoint{2.888259in}{1.123813in}}%
\pgfpathlineto{\pgfqpoint{2.890428in}{1.085765in}}%
\pgfpathlineto{\pgfqpoint{2.892597in}{1.048240in}}%
\pgfpathlineto{\pgfqpoint{2.894765in}{1.011471in}}%
\pgfpathlineto{\pgfqpoint{2.896934in}{0.975662in}}%
\pgfpathlineto{\pgfqpoint{2.899103in}{0.940996in}}%
\pgfpathlineto{\pgfqpoint{2.901272in}{0.907628in}}%
\pgfpathlineto{\pgfqpoint{2.903441in}{0.875688in}}%
\pgfpathlineto{\pgfqpoint{2.905610in}{0.845279in}}%
\pgfpathlineto{\pgfqpoint{2.907779in}{0.816481in}}%
\pgfpathlineto{\pgfqpoint{2.909947in}{0.789348in}}%
\pgfpathlineto{\pgfqpoint{2.912116in}{0.763913in}}%
\pgfpathlineto{\pgfqpoint{2.914285in}{0.740187in}}%
\pgfpathlineto{\pgfqpoint{2.916454in}{0.718163in}}%
\pgfpathlineto{\pgfqpoint{2.918623in}{0.697815in}}%
\pgfpathlineto{\pgfqpoint{2.920792in}{0.679106in}}%
\pgfpathlineto{\pgfqpoint{2.922960in}{0.661982in}}%
\pgfpathlineto{\pgfqpoint{2.925129in}{0.646382in}}%
\pgfpathlineto{\pgfqpoint{2.927298in}{0.632234in}}%
\pgfpathlineto{\pgfqpoint{2.929467in}{0.619459in}}%
\pgfpathlineto{\pgfqpoint{2.931636in}{0.607977in}}%
\pgfpathlineto{\pgfqpoint{2.933805in}{0.597702in}}%
\pgfpathlineto{\pgfqpoint{2.935973in}{0.588546in}}%
\pgfpathlineto{\pgfqpoint{2.938142in}{0.580424in}}%
\pgfpathlineto{\pgfqpoint{2.940311in}{0.573249in}}%
\pgfpathlineto{\pgfqpoint{2.942480in}{0.566937in}}%
\pgfpathlineto{\pgfqpoint{2.944649in}{0.561409in}}%
\pgfpathlineto{\pgfqpoint{2.946818in}{0.556588in}}%
\pgfpathlineto{\pgfqpoint{2.948986in}{0.552400in}}%
\pgfpathlineto{\pgfqpoint{2.951155in}{0.548778in}}%
\pgfpathlineto{\pgfqpoint{2.953324in}{0.545659in}}%
\pgfpathlineto{\pgfqpoint{2.955493in}{0.542982in}}%
\pgfpathlineto{\pgfqpoint{2.957662in}{0.540696in}}%
\pgfusepath{stroke}%
\end{pgfscope}%
\begin{pgfscope}%
\pgfpathrectangle{\pgfqpoint{2.120000in}{0.529102in}}{\pgfqpoint{1.720000in}{1.224232in}} %
\pgfusepath{clip}%
\pgfsetrectcap%
\pgfsetroundjoin%
\pgfsetlinewidth{1.003750pt}%
\definecolor{currentstroke}{rgb}{0.309804,0.478431,0.682353}%
\pgfsetstrokecolor{currentstroke}%
\pgfsetstrokeopacity{0.500000}%
\pgfsetdash{}{0pt}%
\pgfpathmoveto{\pgfqpoint{2.875770in}{0.541848in}}%
\pgfpathlineto{\pgfqpoint{2.877862in}{0.544362in}}%
\pgfpathlineto{\pgfqpoint{2.879953in}{0.547304in}}%
\pgfpathlineto{\pgfqpoint{2.882044in}{0.550734in}}%
\pgfpathlineto{\pgfqpoint{2.884135in}{0.554716in}}%
\pgfpathlineto{\pgfqpoint{2.886227in}{0.559319in}}%
\pgfpathlineto{\pgfqpoint{2.888318in}{0.564620in}}%
\pgfpathlineto{\pgfqpoint{2.890409in}{0.570697in}}%
\pgfpathlineto{\pgfqpoint{2.892500in}{0.577636in}}%
\pgfpathlineto{\pgfqpoint{2.894592in}{0.585524in}}%
\pgfpathlineto{\pgfqpoint{2.896683in}{0.594453in}}%
\pgfpathlineto{\pgfqpoint{2.898774in}{0.604519in}}%
\pgfpathlineto{\pgfqpoint{2.900865in}{0.615815in}}%
\pgfpathlineto{\pgfqpoint{2.902957in}{0.628438in}}%
\pgfpathlineto{\pgfqpoint{2.905048in}{0.642482in}}%
\pgfpathlineto{\pgfqpoint{2.907139in}{0.658036in}}%
\pgfpathlineto{\pgfqpoint{2.909230in}{0.675187in}}%
\pgfpathlineto{\pgfqpoint{2.911322in}{0.694012in}}%
\pgfpathlineto{\pgfqpoint{2.913413in}{0.714581in}}%
\pgfpathlineto{\pgfqpoint{2.915504in}{0.736950in}}%
\pgfpathlineto{\pgfqpoint{2.917595in}{0.761163in}}%
\pgfpathlineto{\pgfqpoint{2.919687in}{0.787247in}}%
\pgfpathlineto{\pgfqpoint{2.921778in}{0.815209in}}%
\pgfpathlineto{\pgfqpoint{2.923869in}{0.845038in}}%
\pgfpathlineto{\pgfqpoint{2.925960in}{0.876698in}}%
\pgfpathlineto{\pgfqpoint{2.928052in}{0.910128in}}%
\pgfpathlineto{\pgfqpoint{2.930143in}{0.945243in}}%
\pgfpathlineto{\pgfqpoint{2.932234in}{0.981926in}}%
\pgfpathlineto{\pgfqpoint{2.934325in}{1.020037in}}%
\pgfpathlineto{\pgfqpoint{2.936417in}{1.059405in}}%
\pgfpathlineto{\pgfqpoint{2.938508in}{1.099828in}}%
\pgfpathlineto{\pgfqpoint{2.940599in}{1.141081in}}%
\pgfpathlineto{\pgfqpoint{2.942690in}{1.182910in}}%
\pgfpathlineto{\pgfqpoint{2.944782in}{1.225037in}}%
\pgfpathlineto{\pgfqpoint{2.946873in}{1.267163in}}%
\pgfpathlineto{\pgfqpoint{2.948964in}{1.308969in}}%
\pgfpathlineto{\pgfqpoint{2.951055in}{1.350121in}}%
\pgfpathlineto{\pgfqpoint{2.953147in}{1.390276in}}%
\pgfpathlineto{\pgfqpoint{2.955238in}{1.429083in}}%
\pgfpathlineto{\pgfqpoint{2.957329in}{1.466190in}}%
\pgfpathlineto{\pgfqpoint{2.959420in}{1.501251in}}%
\pgfpathlineto{\pgfqpoint{2.961512in}{1.533925in}}%
\pgfpathlineto{\pgfqpoint{2.963603in}{1.563889in}}%
\pgfpathlineto{\pgfqpoint{2.965694in}{1.590841in}}%
\pgfpathlineto{\pgfqpoint{2.967785in}{1.614500in}}%
\pgfpathlineto{\pgfqpoint{2.969877in}{1.634618in}}%
\pgfpathlineto{\pgfqpoint{2.971968in}{1.650980in}}%
\pgfpathlineto{\pgfqpoint{2.974059in}{1.663411in}}%
\pgfpathlineto{\pgfqpoint{2.976150in}{1.671774in}}%
\pgfpathlineto{\pgfqpoint{2.978242in}{1.675979in}}%
\pgfpathlineto{\pgfqpoint{2.980333in}{1.675979in}}%
\pgfpathlineto{\pgfqpoint{2.982424in}{1.671774in}}%
\pgfpathlineto{\pgfqpoint{2.984515in}{1.663411in}}%
\pgfpathlineto{\pgfqpoint{2.986607in}{1.650980in}}%
\pgfpathlineto{\pgfqpoint{2.988698in}{1.634618in}}%
\pgfpathlineto{\pgfqpoint{2.990789in}{1.614500in}}%
\pgfpathlineto{\pgfqpoint{2.992880in}{1.590841in}}%
\pgfpathlineto{\pgfqpoint{2.994972in}{1.563889in}}%
\pgfpathlineto{\pgfqpoint{2.997063in}{1.533925in}}%
\pgfpathlineto{\pgfqpoint{2.999154in}{1.501251in}}%
\pgfpathlineto{\pgfqpoint{3.001245in}{1.466190in}}%
\pgfpathlineto{\pgfqpoint{3.003337in}{1.429083in}}%
\pgfpathlineto{\pgfqpoint{3.005428in}{1.390276in}}%
\pgfpathlineto{\pgfqpoint{3.007519in}{1.350121in}}%
\pgfpathlineto{\pgfqpoint{3.009610in}{1.308969in}}%
\pgfpathlineto{\pgfqpoint{3.011702in}{1.267163in}}%
\pgfpathlineto{\pgfqpoint{3.013793in}{1.225037in}}%
\pgfpathlineto{\pgfqpoint{3.015884in}{1.182910in}}%
\pgfpathlineto{\pgfqpoint{3.017975in}{1.141081in}}%
\pgfpathlineto{\pgfqpoint{3.020067in}{1.099828in}}%
\pgfpathlineto{\pgfqpoint{3.022158in}{1.059405in}}%
\pgfpathlineto{\pgfqpoint{3.024249in}{1.020037in}}%
\pgfpathlineto{\pgfqpoint{3.026340in}{0.981926in}}%
\pgfpathlineto{\pgfqpoint{3.028432in}{0.945243in}}%
\pgfpathlineto{\pgfqpoint{3.030523in}{0.910128in}}%
\pgfpathlineto{\pgfqpoint{3.032614in}{0.876698in}}%
\pgfpathlineto{\pgfqpoint{3.034705in}{0.845038in}}%
\pgfpathlineto{\pgfqpoint{3.036797in}{0.815209in}}%
\pgfpathlineto{\pgfqpoint{3.038888in}{0.787247in}}%
\pgfpathlineto{\pgfqpoint{3.040979in}{0.761163in}}%
\pgfpathlineto{\pgfqpoint{3.043070in}{0.736950in}}%
\pgfpathlineto{\pgfqpoint{3.045162in}{0.714581in}}%
\pgfpathlineto{\pgfqpoint{3.047253in}{0.694012in}}%
\pgfpathlineto{\pgfqpoint{3.049344in}{0.675187in}}%
\pgfpathlineto{\pgfqpoint{3.051435in}{0.658036in}}%
\pgfpathlineto{\pgfqpoint{3.053527in}{0.642482in}}%
\pgfpathlineto{\pgfqpoint{3.055618in}{0.628438in}}%
\pgfpathlineto{\pgfqpoint{3.057709in}{0.615815in}}%
\pgfpathlineto{\pgfqpoint{3.059800in}{0.604519in}}%
\pgfpathlineto{\pgfqpoint{3.061892in}{0.594453in}}%
\pgfpathlineto{\pgfqpoint{3.063983in}{0.585524in}}%
\pgfpathlineto{\pgfqpoint{3.066074in}{0.577636in}}%
\pgfpathlineto{\pgfqpoint{3.068165in}{0.570697in}}%
\pgfpathlineto{\pgfqpoint{3.070257in}{0.564620in}}%
\pgfpathlineto{\pgfqpoint{3.072348in}{0.559319in}}%
\pgfpathlineto{\pgfqpoint{3.074439in}{0.554716in}}%
\pgfpathlineto{\pgfqpoint{3.076530in}{0.550734in}}%
\pgfpathlineto{\pgfqpoint{3.078622in}{0.547304in}}%
\pgfpathlineto{\pgfqpoint{3.080713in}{0.544362in}}%
\pgfpathlineto{\pgfqpoint{3.082804in}{0.541848in}}%
\pgfusepath{stroke}%
\end{pgfscope}%
\begin{pgfscope}%
\pgfpathrectangle{\pgfqpoint{2.120000in}{0.529102in}}{\pgfqpoint{1.720000in}{1.224232in}} %
\pgfusepath{clip}%
\pgfsetrectcap%
\pgfsetroundjoin%
\pgfsetlinewidth{1.003750pt}%
\definecolor{currentstroke}{rgb}{0.309804,0.478431,0.682353}%
\pgfsetstrokecolor{currentstroke}%
\pgfsetstrokeopacity{0.500000}%
\pgfsetdash{}{0pt}%
\pgfpathmoveto{\pgfqpoint{3.000888in}{0.540723in}}%
\pgfpathlineto{\pgfqpoint{3.003059in}{0.543015in}}%
\pgfpathlineto{\pgfqpoint{3.005229in}{0.545697in}}%
\pgfpathlineto{\pgfqpoint{3.007399in}{0.548824in}}%
\pgfpathlineto{\pgfqpoint{3.009570in}{0.552455in}}%
\pgfpathlineto{\pgfqpoint{3.011740in}{0.556652in}}%
\pgfpathlineto{\pgfqpoint{3.013910in}{0.561485in}}%
\pgfpathlineto{\pgfqpoint{3.016081in}{0.567025in}}%
\pgfpathlineto{\pgfqpoint{3.018251in}{0.573351in}}%
\pgfpathlineto{\pgfqpoint{3.020421in}{0.580543in}}%
\pgfpathlineto{\pgfqpoint{3.022592in}{0.588684in}}%
\pgfpathlineto{\pgfqpoint{3.024762in}{0.597861in}}%
\pgfpathlineto{\pgfqpoint{3.026932in}{0.608161in}}%
\pgfpathlineto{\pgfqpoint{3.029103in}{0.619669in}}%
\pgfpathlineto{\pgfqpoint{3.031273in}{0.632473in}}%
\pgfpathlineto{\pgfqpoint{3.033443in}{0.646655in}}%
\pgfpathlineto{\pgfqpoint{3.035613in}{0.662291in}}%
\pgfpathlineto{\pgfqpoint{3.037784in}{0.679455in}}%
\pgfpathlineto{\pgfqpoint{3.039954in}{0.698208in}}%
\pgfpathlineto{\pgfqpoint{3.042124in}{0.718602in}}%
\pgfpathlineto{\pgfqpoint{3.044295in}{0.740678in}}%
\pgfpathlineto{\pgfqpoint{3.046465in}{0.764459in}}%
\pgfpathlineto{\pgfqpoint{3.048635in}{0.789953in}}%
\pgfpathlineto{\pgfqpoint{3.050806in}{0.817149in}}%
\pgfpathlineto{\pgfqpoint{3.052976in}{0.846014in}}%
\pgfpathlineto{\pgfqpoint{3.055146in}{0.876493in}}%
\pgfpathlineto{\pgfqpoint{3.057317in}{0.908507in}}%
\pgfpathlineto{\pgfqpoint{3.059487in}{0.941953in}}%
\pgfpathlineto{\pgfqpoint{3.061657in}{0.976700in}}%
\pgfpathlineto{\pgfqpoint{3.063828in}{1.012592in}}%
\pgfpathlineto{\pgfqpoint{3.065998in}{1.049447in}}%
\pgfpathlineto{\pgfqpoint{3.068168in}{1.087058in}}%
\pgfpathlineto{\pgfqpoint{3.070339in}{1.125195in}}%
\pgfpathlineto{\pgfqpoint{3.072509in}{1.163603in}}%
\pgfpathlineto{\pgfqpoint{3.074679in}{1.202010in}}%
\pgfpathlineto{\pgfqpoint{3.076850in}{1.240125in}}%
\pgfpathlineto{\pgfqpoint{3.079020in}{1.277645in}}%
\pgfpathlineto{\pgfqpoint{3.081190in}{1.314255in}}%
\pgfpathlineto{\pgfqpoint{3.083361in}{1.349636in}}%
\pgfpathlineto{\pgfqpoint{3.085531in}{1.383468in}}%
\pgfpathlineto{\pgfqpoint{3.087701in}{1.415433in}}%
\pgfpathlineto{\pgfqpoint{3.089872in}{1.445223in}}%
\pgfpathlineto{\pgfqpoint{3.092042in}{1.472543in}}%
\pgfpathlineto{\pgfqpoint{3.094212in}{1.497115in}}%
\pgfpathlineto{\pgfqpoint{3.096383in}{1.518685in}}%
\pgfpathlineto{\pgfqpoint{3.098553in}{1.537027in}}%
\pgfpathlineto{\pgfqpoint{3.100723in}{1.551945in}}%
\pgfpathlineto{\pgfqpoint{3.102893in}{1.563279in}}%
\pgfpathlineto{\pgfqpoint{3.105064in}{1.570904in}}%
\pgfpathlineto{\pgfqpoint{3.107234in}{1.574737in}}%
\pgfpathlineto{\pgfqpoint{3.109404in}{1.574737in}}%
\pgfpathlineto{\pgfqpoint{3.111575in}{1.570904in}}%
\pgfpathlineto{\pgfqpoint{3.113745in}{1.563279in}}%
\pgfpathlineto{\pgfqpoint{3.115915in}{1.551945in}}%
\pgfpathlineto{\pgfqpoint{3.118086in}{1.537027in}}%
\pgfpathlineto{\pgfqpoint{3.120256in}{1.518685in}}%
\pgfpathlineto{\pgfqpoint{3.122426in}{1.497115in}}%
\pgfpathlineto{\pgfqpoint{3.124597in}{1.472543in}}%
\pgfpathlineto{\pgfqpoint{3.126767in}{1.445223in}}%
\pgfpathlineto{\pgfqpoint{3.128937in}{1.415433in}}%
\pgfpathlineto{\pgfqpoint{3.131108in}{1.383468in}}%
\pgfpathlineto{\pgfqpoint{3.133278in}{1.349636in}}%
\pgfpathlineto{\pgfqpoint{3.135448in}{1.314255in}}%
\pgfpathlineto{\pgfqpoint{3.137619in}{1.277645in}}%
\pgfpathlineto{\pgfqpoint{3.139789in}{1.240125in}}%
\pgfpathlineto{\pgfqpoint{3.141959in}{1.202010in}}%
\pgfpathlineto{\pgfqpoint{3.144130in}{1.163603in}}%
\pgfpathlineto{\pgfqpoint{3.146300in}{1.125195in}}%
\pgfpathlineto{\pgfqpoint{3.148470in}{1.087058in}}%
\pgfpathlineto{\pgfqpoint{3.150641in}{1.049447in}}%
\pgfpathlineto{\pgfqpoint{3.152811in}{1.012592in}}%
\pgfpathlineto{\pgfqpoint{3.154981in}{0.976700in}}%
\pgfpathlineto{\pgfqpoint{3.157152in}{0.941953in}}%
\pgfpathlineto{\pgfqpoint{3.159322in}{0.908507in}}%
\pgfpathlineto{\pgfqpoint{3.161492in}{0.876493in}}%
\pgfpathlineto{\pgfqpoint{3.163663in}{0.846014in}}%
\pgfpathlineto{\pgfqpoint{3.165833in}{0.817149in}}%
\pgfpathlineto{\pgfqpoint{3.168003in}{0.789953in}}%
\pgfpathlineto{\pgfqpoint{3.170173in}{0.764459in}}%
\pgfpathlineto{\pgfqpoint{3.172344in}{0.740678in}}%
\pgfpathlineto{\pgfqpoint{3.174514in}{0.718602in}}%
\pgfpathlineto{\pgfqpoint{3.176684in}{0.698208in}}%
\pgfpathlineto{\pgfqpoint{3.178855in}{0.679455in}}%
\pgfpathlineto{\pgfqpoint{3.181025in}{0.662291in}}%
\pgfpathlineto{\pgfqpoint{3.183195in}{0.646655in}}%
\pgfpathlineto{\pgfqpoint{3.185366in}{0.632473in}}%
\pgfpathlineto{\pgfqpoint{3.187536in}{0.619669in}}%
\pgfpathlineto{\pgfqpoint{3.189706in}{0.608161in}}%
\pgfpathlineto{\pgfqpoint{3.191877in}{0.597861in}}%
\pgfpathlineto{\pgfqpoint{3.194047in}{0.588684in}}%
\pgfpathlineto{\pgfqpoint{3.196217in}{0.580543in}}%
\pgfpathlineto{\pgfqpoint{3.198388in}{0.573351in}}%
\pgfpathlineto{\pgfqpoint{3.200558in}{0.567025in}}%
\pgfpathlineto{\pgfqpoint{3.202728in}{0.561485in}}%
\pgfpathlineto{\pgfqpoint{3.204899in}{0.556652in}}%
\pgfpathlineto{\pgfqpoint{3.207069in}{0.552455in}}%
\pgfpathlineto{\pgfqpoint{3.209239in}{0.548824in}}%
\pgfpathlineto{\pgfqpoint{3.211410in}{0.545697in}}%
\pgfpathlineto{\pgfqpoint{3.213580in}{0.543015in}}%
\pgfpathlineto{\pgfqpoint{3.215750in}{0.540723in}}%
\pgfusepath{stroke}%
\end{pgfscope}%
\begin{pgfscope}%
\pgfpathrectangle{\pgfqpoint{2.120000in}{0.529102in}}{\pgfqpoint{1.720000in}{1.224232in}} %
\pgfusepath{clip}%
\pgfsetrectcap%
\pgfsetroundjoin%
\pgfsetlinewidth{1.003750pt}%
\definecolor{currentstroke}{rgb}{0.309804,0.478431,0.682353}%
\pgfsetstrokecolor{currentstroke}%
\pgfsetstrokeopacity{0.500000}%
\pgfsetdash{}{0pt}%
\pgfpathmoveto{\pgfqpoint{3.131233in}{0.537774in}}%
\pgfpathlineto{\pgfqpoint{3.133436in}{0.539485in}}%
\pgfpathlineto{\pgfqpoint{3.135639in}{0.541487in}}%
\pgfpathlineto{\pgfqpoint{3.137842in}{0.543820in}}%
\pgfpathlineto{\pgfqpoint{3.140046in}{0.546530in}}%
\pgfpathlineto{\pgfqpoint{3.142249in}{0.549662in}}%
\pgfpathlineto{\pgfqpoint{3.144452in}{0.553269in}}%
\pgfpathlineto{\pgfqpoint{3.146655in}{0.557404in}}%
\pgfpathlineto{\pgfqpoint{3.148858in}{0.562125in}}%
\pgfpathlineto{\pgfqpoint{3.151062in}{0.567492in}}%
\pgfpathlineto{\pgfqpoint{3.153265in}{0.573568in}}%
\pgfpathlineto{\pgfqpoint{3.155468in}{0.580417in}}%
\pgfpathlineto{\pgfqpoint{3.157671in}{0.588103in}}%
\pgfpathlineto{\pgfqpoint{3.159874in}{0.596692in}}%
\pgfpathlineto{\pgfqpoint{3.162078in}{0.606247in}}%
\pgfpathlineto{\pgfqpoint{3.164281in}{0.616831in}}%
\pgfpathlineto{\pgfqpoint{3.166484in}{0.628500in}}%
\pgfpathlineto{\pgfqpoint{3.168687in}{0.641309in}}%
\pgfpathlineto{\pgfqpoint{3.170890in}{0.655304in}}%
\pgfpathlineto{\pgfqpoint{3.173094in}{0.670525in}}%
\pgfpathlineto{\pgfqpoint{3.175297in}{0.687000in}}%
\pgfpathlineto{\pgfqpoint{3.177500in}{0.704747in}}%
\pgfpathlineto{\pgfqpoint{3.179703in}{0.723774in}}%
\pgfpathlineto{\pgfqpoint{3.181906in}{0.744070in}}%
\pgfpathlineto{\pgfqpoint{3.184109in}{0.765611in}}%
\pgfpathlineto{\pgfqpoint{3.186313in}{0.788358in}}%
\pgfpathlineto{\pgfqpoint{3.188516in}{0.812250in}}%
\pgfpathlineto{\pgfqpoint{3.190719in}{0.837210in}}%
\pgfpathlineto{\pgfqpoint{3.192922in}{0.863142in}}%
\pgfpathlineto{\pgfqpoint{3.195125in}{0.889928in}}%
\pgfpathlineto{\pgfqpoint{3.197329in}{0.917432in}}%
\pgfpathlineto{\pgfqpoint{3.199532in}{0.945502in}}%
\pgfpathlineto{\pgfqpoint{3.201735in}{0.973963in}}%
\pgfpathlineto{\pgfqpoint{3.203938in}{1.002627in}}%
\pgfpathlineto{\pgfqpoint{3.206141in}{1.031289in}}%
\pgfpathlineto{\pgfqpoint{3.208345in}{1.059735in}}%
\pgfpathlineto{\pgfqpoint{3.210548in}{1.087735in}}%
\pgfpathlineto{\pgfqpoint{3.212751in}{1.115057in}}%
\pgfpathlineto{\pgfqpoint{3.214954in}{1.141462in}}%
\pgfpathlineto{\pgfqpoint{3.217157in}{1.166711in}}%
\pgfpathlineto{\pgfqpoint{3.219361in}{1.190566in}}%
\pgfpathlineto{\pgfqpoint{3.221564in}{1.212798in}}%
\pgfpathlineto{\pgfqpoint{3.223767in}{1.233187in}}%
\pgfpathlineto{\pgfqpoint{3.225970in}{1.251525in}}%
\pgfpathlineto{\pgfqpoint{3.228173in}{1.267622in}}%
\pgfpathlineto{\pgfqpoint{3.230376in}{1.281311in}}%
\pgfpathlineto{\pgfqpoint{3.232580in}{1.292444in}}%
\pgfpathlineto{\pgfqpoint{3.234783in}{1.300902in}}%
\pgfpathlineto{\pgfqpoint{3.236986in}{1.306593in}}%
\pgfpathlineto{\pgfqpoint{3.239189in}{1.309454in}}%
\pgfpathlineto{\pgfqpoint{3.241392in}{1.309454in}}%
\pgfpathlineto{\pgfqpoint{3.243596in}{1.306593in}}%
\pgfpathlineto{\pgfqpoint{3.245799in}{1.300902in}}%
\pgfpathlineto{\pgfqpoint{3.248002in}{1.292444in}}%
\pgfpathlineto{\pgfqpoint{3.250205in}{1.281311in}}%
\pgfpathlineto{\pgfqpoint{3.252408in}{1.267622in}}%
\pgfpathlineto{\pgfqpoint{3.254612in}{1.251525in}}%
\pgfpathlineto{\pgfqpoint{3.256815in}{1.233187in}}%
\pgfpathlineto{\pgfqpoint{3.259018in}{1.212798in}}%
\pgfpathlineto{\pgfqpoint{3.261221in}{1.190566in}}%
\pgfpathlineto{\pgfqpoint{3.263424in}{1.166711in}}%
\pgfpathlineto{\pgfqpoint{3.265628in}{1.141462in}}%
\pgfpathlineto{\pgfqpoint{3.267831in}{1.115057in}}%
\pgfpathlineto{\pgfqpoint{3.270034in}{1.087735in}}%
\pgfpathlineto{\pgfqpoint{3.272237in}{1.059735in}}%
\pgfpathlineto{\pgfqpoint{3.274440in}{1.031289in}}%
\pgfpathlineto{\pgfqpoint{3.276644in}{1.002627in}}%
\pgfpathlineto{\pgfqpoint{3.278847in}{0.973963in}}%
\pgfpathlineto{\pgfqpoint{3.281050in}{0.945502in}}%
\pgfpathlineto{\pgfqpoint{3.283253in}{0.917432in}}%
\pgfpathlineto{\pgfqpoint{3.285456in}{0.889928in}}%
\pgfpathlineto{\pgfqpoint{3.287659in}{0.863142in}}%
\pgfpathlineto{\pgfqpoint{3.289863in}{0.837210in}}%
\pgfpathlineto{\pgfqpoint{3.292066in}{0.812250in}}%
\pgfpathlineto{\pgfqpoint{3.294269in}{0.788358in}}%
\pgfpathlineto{\pgfqpoint{3.296472in}{0.765611in}}%
\pgfpathlineto{\pgfqpoint{3.298675in}{0.744070in}}%
\pgfpathlineto{\pgfqpoint{3.300879in}{0.723774in}}%
\pgfpathlineto{\pgfqpoint{3.303082in}{0.704747in}}%
\pgfpathlineto{\pgfqpoint{3.305285in}{0.687000in}}%
\pgfpathlineto{\pgfqpoint{3.307488in}{0.670525in}}%
\pgfpathlineto{\pgfqpoint{3.309691in}{0.655304in}}%
\pgfpathlineto{\pgfqpoint{3.311895in}{0.641309in}}%
\pgfpathlineto{\pgfqpoint{3.314098in}{0.628500in}}%
\pgfpathlineto{\pgfqpoint{3.316301in}{0.616831in}}%
\pgfpathlineto{\pgfqpoint{3.318504in}{0.606247in}}%
\pgfpathlineto{\pgfqpoint{3.320707in}{0.596692in}}%
\pgfpathlineto{\pgfqpoint{3.322911in}{0.588103in}}%
\pgfpathlineto{\pgfqpoint{3.325114in}{0.580417in}}%
\pgfpathlineto{\pgfqpoint{3.327317in}{0.573568in}}%
\pgfpathlineto{\pgfqpoint{3.329520in}{0.567492in}}%
\pgfpathlineto{\pgfqpoint{3.331723in}{0.562125in}}%
\pgfpathlineto{\pgfqpoint{3.333926in}{0.557404in}}%
\pgfpathlineto{\pgfqpoint{3.336130in}{0.553269in}}%
\pgfpathlineto{\pgfqpoint{3.338333in}{0.549662in}}%
\pgfpathlineto{\pgfqpoint{3.340536in}{0.546530in}}%
\pgfpathlineto{\pgfqpoint{3.342739in}{0.543820in}}%
\pgfpathlineto{\pgfqpoint{3.344942in}{0.541487in}}%
\pgfpathlineto{\pgfqpoint{3.347146in}{0.539485in}}%
\pgfpathlineto{\pgfqpoint{3.349349in}{0.537774in}}%
\pgfusepath{stroke}%
\end{pgfscope}%
\begin{pgfscope}%
\pgfpathrectangle{\pgfqpoint{2.120000in}{0.529102in}}{\pgfqpoint{1.720000in}{1.224232in}} %
\pgfusepath{clip}%
\pgfsetrectcap%
\pgfsetroundjoin%
\pgfsetlinewidth{1.003750pt}%
\definecolor{currentstroke}{rgb}{0.309804,0.478431,0.682353}%
\pgfsetstrokecolor{currentstroke}%
\pgfsetstrokeopacity{0.500000}%
\pgfsetdash{}{0pt}%
\pgfpathmoveto{\pgfqpoint{3.265723in}{0.534451in}}%
\pgfpathlineto{\pgfqpoint{3.267888in}{0.535506in}}%
\pgfpathlineto{\pgfqpoint{3.270053in}{0.536741in}}%
\pgfpathlineto{\pgfqpoint{3.272218in}{0.538181in}}%
\pgfpathlineto{\pgfqpoint{3.274383in}{0.539852in}}%
\pgfpathlineto{\pgfqpoint{3.276547in}{0.541784in}}%
\pgfpathlineto{\pgfqpoint{3.278712in}{0.544009in}}%
\pgfpathlineto{\pgfqpoint{3.280877in}{0.546559in}}%
\pgfpathlineto{\pgfqpoint{3.283042in}{0.549471in}}%
\pgfpathlineto{\pgfqpoint{3.285207in}{0.552782in}}%
\pgfpathlineto{\pgfqpoint{3.287372in}{0.556530in}}%
\pgfpathlineto{\pgfqpoint{3.289537in}{0.560754in}}%
\pgfpathlineto{\pgfqpoint{3.291702in}{0.565495in}}%
\pgfpathlineto{\pgfqpoint{3.293867in}{0.570793in}}%
\pgfpathlineto{\pgfqpoint{3.296032in}{0.576687in}}%
\pgfpathlineto{\pgfqpoint{3.298197in}{0.583215in}}%
\pgfpathlineto{\pgfqpoint{3.300362in}{0.590413in}}%
\pgfpathlineto{\pgfqpoint{3.302527in}{0.598314in}}%
\pgfpathlineto{\pgfqpoint{3.304692in}{0.606947in}}%
\pgfpathlineto{\pgfqpoint{3.306857in}{0.616335in}}%
\pgfpathlineto{\pgfqpoint{3.309022in}{0.626497in}}%
\pgfpathlineto{\pgfqpoint{3.311187in}{0.637444in}}%
\pgfpathlineto{\pgfqpoint{3.313352in}{0.649180in}}%
\pgfpathlineto{\pgfqpoint{3.315517in}{0.661699in}}%
\pgfpathlineto{\pgfqpoint{3.317682in}{0.674987in}}%
\pgfpathlineto{\pgfqpoint{3.319847in}{0.689017in}}%
\pgfpathlineto{\pgfqpoint{3.322012in}{0.703755in}}%
\pgfpathlineto{\pgfqpoint{3.324177in}{0.719151in}}%
\pgfpathlineto{\pgfqpoint{3.326342in}{0.735146in}}%
\pgfpathlineto{\pgfqpoint{3.328507in}{0.751668in}}%
\pgfpathlineto{\pgfqpoint{3.330672in}{0.768634in}}%
\pgfpathlineto{\pgfqpoint{3.332837in}{0.785948in}}%
\pgfpathlineto{\pgfqpoint{3.335002in}{0.803503in}}%
\pgfpathlineto{\pgfqpoint{3.337167in}{0.821184in}}%
\pgfpathlineto{\pgfqpoint{3.339332in}{0.838864in}}%
\pgfpathlineto{\pgfqpoint{3.341497in}{0.856410in}}%
\pgfpathlineto{\pgfqpoint{3.343662in}{0.873681in}}%
\pgfpathlineto{\pgfqpoint{3.345827in}{0.890534in}}%
\pgfpathlineto{\pgfqpoint{3.347992in}{0.906821in}}%
\pgfpathlineto{\pgfqpoint{3.350157in}{0.922395in}}%
\pgfpathlineto{\pgfqpoint{3.352322in}{0.937110in}}%
\pgfpathlineto{\pgfqpoint{3.354487in}{0.950823in}}%
\pgfpathlineto{\pgfqpoint{3.356652in}{0.963399in}}%
\pgfpathlineto{\pgfqpoint{3.358817in}{0.974710in}}%
\pgfpathlineto{\pgfqpoint{3.360981in}{0.984640in}}%
\pgfpathlineto{\pgfqpoint{3.363146in}{0.993083in}}%
\pgfpathlineto{\pgfqpoint{3.365311in}{0.999951in}}%
\pgfpathlineto{\pgfqpoint{3.367476in}{1.005168in}}%
\pgfpathlineto{\pgfqpoint{3.369641in}{1.008678in}}%
\pgfpathlineto{\pgfqpoint{3.371806in}{1.010443in}}%
\pgfpathlineto{\pgfqpoint{3.373971in}{1.010443in}}%
\pgfpathlineto{\pgfqpoint{3.376136in}{1.008678in}}%
\pgfpathlineto{\pgfqpoint{3.378301in}{1.005168in}}%
\pgfpathlineto{\pgfqpoint{3.380466in}{0.999951in}}%
\pgfpathlineto{\pgfqpoint{3.382631in}{0.993083in}}%
\pgfpathlineto{\pgfqpoint{3.384796in}{0.984640in}}%
\pgfpathlineto{\pgfqpoint{3.386961in}{0.974710in}}%
\pgfpathlineto{\pgfqpoint{3.389126in}{0.963399in}}%
\pgfpathlineto{\pgfqpoint{3.391291in}{0.950823in}}%
\pgfpathlineto{\pgfqpoint{3.393456in}{0.937110in}}%
\pgfpathlineto{\pgfqpoint{3.395621in}{0.922395in}}%
\pgfpathlineto{\pgfqpoint{3.397786in}{0.906821in}}%
\pgfpathlineto{\pgfqpoint{3.399951in}{0.890534in}}%
\pgfpathlineto{\pgfqpoint{3.402116in}{0.873681in}}%
\pgfpathlineto{\pgfqpoint{3.404281in}{0.856410in}}%
\pgfpathlineto{\pgfqpoint{3.406446in}{0.838864in}}%
\pgfpathlineto{\pgfqpoint{3.408611in}{0.821184in}}%
\pgfpathlineto{\pgfqpoint{3.410776in}{0.803503in}}%
\pgfpathlineto{\pgfqpoint{3.412941in}{0.785948in}}%
\pgfpathlineto{\pgfqpoint{3.415106in}{0.768634in}}%
\pgfpathlineto{\pgfqpoint{3.417271in}{0.751668in}}%
\pgfpathlineto{\pgfqpoint{3.419436in}{0.735146in}}%
\pgfpathlineto{\pgfqpoint{3.421601in}{0.719151in}}%
\pgfpathlineto{\pgfqpoint{3.423766in}{0.703755in}}%
\pgfpathlineto{\pgfqpoint{3.425931in}{0.689017in}}%
\pgfpathlineto{\pgfqpoint{3.428096in}{0.674987in}}%
\pgfpathlineto{\pgfqpoint{3.430261in}{0.661699in}}%
\pgfpathlineto{\pgfqpoint{3.432426in}{0.649180in}}%
\pgfpathlineto{\pgfqpoint{3.434591in}{0.637444in}}%
\pgfpathlineto{\pgfqpoint{3.436756in}{0.626497in}}%
\pgfpathlineto{\pgfqpoint{3.438921in}{0.616335in}}%
\pgfpathlineto{\pgfqpoint{3.441086in}{0.606947in}}%
\pgfpathlineto{\pgfqpoint{3.443250in}{0.598314in}}%
\pgfpathlineto{\pgfqpoint{3.445415in}{0.590413in}}%
\pgfpathlineto{\pgfqpoint{3.447580in}{0.583215in}}%
\pgfpathlineto{\pgfqpoint{3.449745in}{0.576687in}}%
\pgfpathlineto{\pgfqpoint{3.451910in}{0.570793in}}%
\pgfpathlineto{\pgfqpoint{3.454075in}{0.565495in}}%
\pgfpathlineto{\pgfqpoint{3.456240in}{0.560754in}}%
\pgfpathlineto{\pgfqpoint{3.458405in}{0.556530in}}%
\pgfpathlineto{\pgfqpoint{3.460570in}{0.552782in}}%
\pgfpathlineto{\pgfqpoint{3.462735in}{0.549471in}}%
\pgfpathlineto{\pgfqpoint{3.464900in}{0.546559in}}%
\pgfpathlineto{\pgfqpoint{3.467065in}{0.544009in}}%
\pgfpathlineto{\pgfqpoint{3.469230in}{0.541784in}}%
\pgfpathlineto{\pgfqpoint{3.471395in}{0.539852in}}%
\pgfpathlineto{\pgfqpoint{3.473560in}{0.538181in}}%
\pgfpathlineto{\pgfqpoint{3.475725in}{0.536741in}}%
\pgfpathlineto{\pgfqpoint{3.477890in}{0.535506in}}%
\pgfpathlineto{\pgfqpoint{3.480055in}{0.534451in}}%
\pgfusepath{stroke}%
\end{pgfscope}%
\begin{pgfscope}%
\pgfpathrectangle{\pgfqpoint{2.120000in}{0.529102in}}{\pgfqpoint{1.720000in}{1.224232in}} %
\pgfusepath{clip}%
\pgfsetrectcap%
\pgfsetroundjoin%
\pgfsetlinewidth{1.003750pt}%
\definecolor{currentstroke}{rgb}{0.309804,0.478431,0.682353}%
\pgfsetstrokecolor{currentstroke}%
\pgfsetstrokeopacity{0.500000}%
\pgfsetdash{}{0pt}%
\pgfpathmoveto{\pgfqpoint{3.401258in}{0.531809in}}%
\pgfpathlineto{\pgfqpoint{3.403266in}{0.532343in}}%
\pgfpathlineto{\pgfqpoint{3.405273in}{0.532968in}}%
\pgfpathlineto{\pgfqpoint{3.407281in}{0.533697in}}%
\pgfpathlineto{\pgfqpoint{3.409288in}{0.534543in}}%
\pgfpathlineto{\pgfqpoint{3.411296in}{0.535521in}}%
\pgfpathlineto{\pgfqpoint{3.413304in}{0.536647in}}%
\pgfpathlineto{\pgfqpoint{3.415311in}{0.537938in}}%
\pgfpathlineto{\pgfqpoint{3.417319in}{0.539412in}}%
\pgfpathlineto{\pgfqpoint{3.419326in}{0.541088in}}%
\pgfpathlineto{\pgfqpoint{3.421334in}{0.542985in}}%
\pgfpathlineto{\pgfqpoint{3.423341in}{0.545123in}}%
\pgfpathlineto{\pgfqpoint{3.425349in}{0.547523in}}%
\pgfpathlineto{\pgfqpoint{3.427357in}{0.550204in}}%
\pgfpathlineto{\pgfqpoint{3.429364in}{0.553188in}}%
\pgfpathlineto{\pgfqpoint{3.431372in}{0.556492in}}%
\pgfpathlineto{\pgfqpoint{3.433379in}{0.560136in}}%
\pgfpathlineto{\pgfqpoint{3.435387in}{0.564135in}}%
\pgfpathlineto{\pgfqpoint{3.437394in}{0.568504in}}%
\pgfpathlineto{\pgfqpoint{3.439402in}{0.573256in}}%
\pgfpathlineto{\pgfqpoint{3.441410in}{0.578400in}}%
\pgfpathlineto{\pgfqpoint{3.443417in}{0.583941in}}%
\pgfpathlineto{\pgfqpoint{3.445425in}{0.589882in}}%
\pgfpathlineto{\pgfqpoint{3.447432in}{0.596218in}}%
\pgfpathlineto{\pgfqpoint{3.449440in}{0.602944in}}%
\pgfpathlineto{\pgfqpoint{3.451448in}{0.610046in}}%
\pgfpathlineto{\pgfqpoint{3.453455in}{0.617505in}}%
\pgfpathlineto{\pgfqpoint{3.455463in}{0.625298in}}%
\pgfpathlineto{\pgfqpoint{3.457470in}{0.633395in}}%
\pgfpathlineto{\pgfqpoint{3.459478in}{0.641758in}}%
\pgfpathlineto{\pgfqpoint{3.461485in}{0.650345in}}%
\pgfpathlineto{\pgfqpoint{3.463493in}{0.659109in}}%
\pgfpathlineto{\pgfqpoint{3.465501in}{0.667995in}}%
\pgfpathlineto{\pgfqpoint{3.467508in}{0.676944in}}%
\pgfpathlineto{\pgfqpoint{3.469516in}{0.685893in}}%
\pgfpathlineto{\pgfqpoint{3.471523in}{0.694774in}}%
\pgfpathlineto{\pgfqpoint{3.473531in}{0.703517in}}%
\pgfpathlineto{\pgfqpoint{3.475538in}{0.712047in}}%
\pgfpathlineto{\pgfqpoint{3.477546in}{0.720291in}}%
\pgfpathlineto{\pgfqpoint{3.479554in}{0.728174in}}%
\pgfpathlineto{\pgfqpoint{3.481561in}{0.735622in}}%
\pgfpathlineto{\pgfqpoint{3.483569in}{0.742563in}}%
\pgfpathlineto{\pgfqpoint{3.485576in}{0.748929in}}%
\pgfpathlineto{\pgfqpoint{3.487584in}{0.754654in}}%
\pgfpathlineto{\pgfqpoint{3.489592in}{0.759680in}}%
\pgfpathlineto{\pgfqpoint{3.491599in}{0.763954in}}%
\pgfpathlineto{\pgfqpoint{3.493607in}{0.767430in}}%
\pgfpathlineto{\pgfqpoint{3.495614in}{0.770071in}}%
\pgfpathlineto{\pgfqpoint{3.497622in}{0.771848in}}%
\pgfpathlineto{\pgfqpoint{3.499629in}{0.772741in}}%
\pgfpathlineto{\pgfqpoint{3.501637in}{0.772741in}}%
\pgfpathlineto{\pgfqpoint{3.503645in}{0.771848in}}%
\pgfpathlineto{\pgfqpoint{3.505652in}{0.770071in}}%
\pgfpathlineto{\pgfqpoint{3.507660in}{0.767430in}}%
\pgfpathlineto{\pgfqpoint{3.509667in}{0.763954in}}%
\pgfpathlineto{\pgfqpoint{3.511675in}{0.759680in}}%
\pgfpathlineto{\pgfqpoint{3.513682in}{0.754654in}}%
\pgfpathlineto{\pgfqpoint{3.515690in}{0.748929in}}%
\pgfpathlineto{\pgfqpoint{3.517698in}{0.742563in}}%
\pgfpathlineto{\pgfqpoint{3.519705in}{0.735622in}}%
\pgfpathlineto{\pgfqpoint{3.521713in}{0.728174in}}%
\pgfpathlineto{\pgfqpoint{3.523720in}{0.720291in}}%
\pgfpathlineto{\pgfqpoint{3.525728in}{0.712047in}}%
\pgfpathlineto{\pgfqpoint{3.527736in}{0.703517in}}%
\pgfpathlineto{\pgfqpoint{3.529743in}{0.694774in}}%
\pgfpathlineto{\pgfqpoint{3.531751in}{0.685893in}}%
\pgfpathlineto{\pgfqpoint{3.533758in}{0.676944in}}%
\pgfpathlineto{\pgfqpoint{3.535766in}{0.667995in}}%
\pgfpathlineto{\pgfqpoint{3.537773in}{0.659109in}}%
\pgfpathlineto{\pgfqpoint{3.539781in}{0.650345in}}%
\pgfpathlineto{\pgfqpoint{3.541789in}{0.641758in}}%
\pgfpathlineto{\pgfqpoint{3.543796in}{0.633395in}}%
\pgfpathlineto{\pgfqpoint{3.545804in}{0.625298in}}%
\pgfpathlineto{\pgfqpoint{3.547811in}{0.617505in}}%
\pgfpathlineto{\pgfqpoint{3.549819in}{0.610046in}}%
\pgfpathlineto{\pgfqpoint{3.551826in}{0.602944in}}%
\pgfpathlineto{\pgfqpoint{3.553834in}{0.596218in}}%
\pgfpathlineto{\pgfqpoint{3.555842in}{0.589882in}}%
\pgfpathlineto{\pgfqpoint{3.557849in}{0.583941in}}%
\pgfpathlineto{\pgfqpoint{3.559857in}{0.578400in}}%
\pgfpathlineto{\pgfqpoint{3.561864in}{0.573256in}}%
\pgfpathlineto{\pgfqpoint{3.563872in}{0.568504in}}%
\pgfpathlineto{\pgfqpoint{3.565880in}{0.564135in}}%
\pgfpathlineto{\pgfqpoint{3.567887in}{0.560136in}}%
\pgfpathlineto{\pgfqpoint{3.569895in}{0.556492in}}%
\pgfpathlineto{\pgfqpoint{3.571902in}{0.553188in}}%
\pgfpathlineto{\pgfqpoint{3.573910in}{0.550204in}}%
\pgfpathlineto{\pgfqpoint{3.575917in}{0.547523in}}%
\pgfpathlineto{\pgfqpoint{3.577925in}{0.545123in}}%
\pgfpathlineto{\pgfqpoint{3.579933in}{0.542985in}}%
\pgfpathlineto{\pgfqpoint{3.581940in}{0.541088in}}%
\pgfpathlineto{\pgfqpoint{3.583948in}{0.539412in}}%
\pgfpathlineto{\pgfqpoint{3.585955in}{0.537938in}}%
\pgfpathlineto{\pgfqpoint{3.587963in}{0.536647in}}%
\pgfpathlineto{\pgfqpoint{3.589970in}{0.535521in}}%
\pgfpathlineto{\pgfqpoint{3.591978in}{0.534543in}}%
\pgfpathlineto{\pgfqpoint{3.593986in}{0.533697in}}%
\pgfpathlineto{\pgfqpoint{3.595993in}{0.532968in}}%
\pgfpathlineto{\pgfqpoint{3.598001in}{0.532343in}}%
\pgfpathlineto{\pgfqpoint{3.600008in}{0.531809in}}%
\pgfusepath{stroke}%
\end{pgfscope}%
\begin{pgfscope}%
\pgfpathrectangle{\pgfqpoint{2.120000in}{0.529102in}}{\pgfqpoint{1.720000in}{1.224232in}} %
\pgfusepath{clip}%
\pgfsetrectcap%
\pgfsetroundjoin%
\pgfsetlinewidth{1.003750pt}%
\definecolor{currentstroke}{rgb}{0.309804,0.478431,0.682353}%
\pgfsetstrokecolor{currentstroke}%
\pgfsetdash{}{0pt}%
\pgfpathmoveto{\pgfqpoint{2.120430in}{0.529102in}}%
\pgfpathlineto{\pgfqpoint{2.350165in}{0.530212in}}%
\pgfpathlineto{\pgfqpoint{2.363072in}{0.532743in}}%
\pgfpathlineto{\pgfqpoint{2.371676in}{0.536507in}}%
\pgfpathlineto{\pgfqpoint{2.379420in}{0.542362in}}%
\pgfpathlineto{\pgfqpoint{2.386303in}{0.550388in}}%
\pgfpathlineto{\pgfqpoint{2.393187in}{0.561864in}}%
\pgfpathlineto{\pgfqpoint{2.400930in}{0.579714in}}%
\pgfpathlineto{\pgfqpoint{2.409535in}{0.606199in}}%
\pgfpathlineto{\pgfqpoint{2.419860in}{0.646251in}}%
\pgfpathlineto{\pgfqpoint{2.444812in}{0.747145in}}%
\pgfpathlineto{\pgfqpoint{2.451696in}{0.764045in}}%
\pgfpathlineto{\pgfqpoint{2.456858in}{0.770907in}}%
\pgfpathlineto{\pgfqpoint{2.460300in}{0.772449in}}%
\pgfpathlineto{\pgfqpoint{2.463742in}{0.771529in}}%
\pgfpathlineto{\pgfqpoint{2.467184in}{0.768212in}}%
\pgfpathlineto{\pgfqpoint{2.471486in}{0.760929in}}%
\pgfpathlineto{\pgfqpoint{2.477509in}{0.745765in}}%
\pgfpathlineto{\pgfqpoint{2.486973in}{0.714484in}}%
\pgfpathlineto{\pgfqpoint{2.500740in}{0.669667in}}%
\pgfpathlineto{\pgfqpoint{2.506763in}{0.656704in}}%
\pgfpathlineto{\pgfqpoint{2.511066in}{0.651605in}}%
\pgfpathlineto{\pgfqpoint{2.514507in}{0.650463in}}%
\pgfpathlineto{\pgfqpoint{2.517949in}{0.652170in}}%
\pgfpathlineto{\pgfqpoint{2.521391in}{0.656881in}}%
\pgfpathlineto{\pgfqpoint{2.525693in}{0.667116in}}%
\pgfpathlineto{\pgfqpoint{2.530855in}{0.685757in}}%
\pgfpathlineto{\pgfqpoint{2.537739in}{0.720750in}}%
\pgfpathlineto{\pgfqpoint{2.546343in}{0.777822in}}%
\pgfpathlineto{\pgfqpoint{2.573877in}{0.972080in}}%
\pgfpathlineto{\pgfqpoint{2.579900in}{0.995813in}}%
\pgfpathlineto{\pgfqpoint{2.584202in}{1.005190in}}%
\pgfpathlineto{\pgfqpoint{2.587644in}{1.007810in}}%
\pgfpathlineto{\pgfqpoint{2.590225in}{1.006882in}}%
\pgfpathlineto{\pgfqpoint{2.592806in}{1.003506in}}%
\pgfpathlineto{\pgfqpoint{2.596248in}{0.995331in}}%
\pgfpathlineto{\pgfqpoint{2.601411in}{0.975847in}}%
\pgfpathlineto{\pgfqpoint{2.608294in}{0.938887in}}%
\pgfpathlineto{\pgfqpoint{2.621201in}{0.852819in}}%
\pgfpathlineto{\pgfqpoint{2.631526in}{0.790401in}}%
\pgfpathlineto{\pgfqpoint{2.637549in}{0.765083in}}%
\pgfpathlineto{\pgfqpoint{2.641851in}{0.754203in}}%
\pgfpathlineto{\pgfqpoint{2.645293in}{0.750491in}}%
\pgfpathlineto{\pgfqpoint{2.647874in}{0.750837in}}%
\pgfpathlineto{\pgfqpoint{2.650455in}{0.753976in}}%
\pgfpathlineto{\pgfqpoint{2.653897in}{0.762611in}}%
\pgfpathlineto{\pgfqpoint{2.658199in}{0.780604in}}%
\pgfpathlineto{\pgfqpoint{2.663362in}{0.812497in}}%
\pgfpathlineto{\pgfqpoint{2.670245in}{0.870960in}}%
\pgfpathlineto{\pgfqpoint{2.679710in}{0.974130in}}%
\pgfpathlineto{\pgfqpoint{2.702081in}{1.228860in}}%
\pgfpathlineto{\pgfqpoint{2.708964in}{1.279361in}}%
\pgfpathlineto{\pgfqpoint{2.714127in}{1.301427in}}%
\pgfpathlineto{\pgfqpoint{2.717569in}{1.307762in}}%
\pgfpathlineto{\pgfqpoint{2.719290in}{1.308337in}}%
\pgfpathlineto{\pgfqpoint{2.721011in}{1.307179in}}%
\pgfpathlineto{\pgfqpoint{2.723592in}{1.302224in}}%
\pgfpathlineto{\pgfqpoint{2.727034in}{1.289784in}}%
\pgfpathlineto{\pgfqpoint{2.731336in}{1.265472in}}%
\pgfpathlineto{\pgfqpoint{2.737359in}{1.217382in}}%
\pgfpathlineto{\pgfqpoint{2.746823in}{1.119427in}}%
\pgfpathlineto{\pgfqpoint{2.764032in}{0.938853in}}%
\pgfpathlineto{\pgfqpoint{2.770915in}{0.890296in}}%
\pgfpathlineto{\pgfqpoint{2.775218in}{0.871696in}}%
\pgfpathlineto{\pgfqpoint{2.778659in}{0.864273in}}%
\pgfpathlineto{\pgfqpoint{2.781241in}{0.863326in}}%
\pgfpathlineto{\pgfqpoint{2.782961in}{0.864961in}}%
\pgfpathlineto{\pgfqpoint{2.785543in}{0.870859in}}%
\pgfpathlineto{\pgfqpoint{2.788984in}{0.885178in}}%
\pgfpathlineto{\pgfqpoint{2.793287in}{0.913282in}}%
\pgfpathlineto{\pgfqpoint{2.799310in}{0.970527in}}%
\pgfpathlineto{\pgfqpoint{2.807054in}{1.070050in}}%
\pgfpathlineto{\pgfqpoint{2.819960in}{1.272453in}}%
\pgfpathlineto{\pgfqpoint{2.832006in}{1.450466in}}%
\pgfpathlineto{\pgfqpoint{2.838889in}{1.524239in}}%
\pgfpathlineto{\pgfqpoint{2.844052in}{1.559303in}}%
\pgfpathlineto{\pgfqpoint{2.847494in}{1.571633in}}%
\pgfpathlineto{\pgfqpoint{2.850075in}{1.574789in}}%
\pgfpathlineto{\pgfqpoint{2.851796in}{1.573956in}}%
\pgfpathlineto{\pgfqpoint{2.854377in}{1.568322in}}%
\pgfpathlineto{\pgfqpoint{2.857819in}{1.552808in}}%
\pgfpathlineto{\pgfqpoint{2.862121in}{1.521285in}}%
\pgfpathlineto{\pgfqpoint{2.868144in}{1.457386in}}%
\pgfpathlineto{\pgfqpoint{2.876748in}{1.337347in}}%
\pgfpathlineto{\pgfqpoint{2.897399in}{1.036806in}}%
\pgfpathlineto{\pgfqpoint{2.904282in}{0.971775in}}%
\pgfpathlineto{\pgfqpoint{2.909445in}{0.943424in}}%
\pgfpathlineto{\pgfqpoint{2.912886in}{0.935427in}}%
\pgfpathlineto{\pgfqpoint{2.914607in}{0.934841in}}%
\pgfpathlineto{\pgfqpoint{2.916328in}{0.936562in}}%
\pgfpathlineto{\pgfqpoint{2.918909in}{0.943483in}}%
\pgfpathlineto{\pgfqpoint{2.922351in}{0.960739in}}%
\pgfpathlineto{\pgfqpoint{2.926653in}{0.994778in}}%
\pgfpathlineto{\pgfqpoint{2.932676in}{1.063680in}}%
\pgfpathlineto{\pgfqpoint{2.940420in}{1.181354in}}%
\pgfpathlineto{\pgfqpoint{2.966233in}{1.603156in}}%
\pgfpathlineto{\pgfqpoint{2.971396in}{1.650994in}}%
\pgfpathlineto{\pgfqpoint{2.975698in}{1.673666in}}%
\pgfpathlineto{\pgfqpoint{2.978279in}{1.679193in}}%
\pgfpathlineto{\pgfqpoint{2.980000in}{1.679431in}}%
\pgfpathlineto{\pgfqpoint{2.981721in}{1.676908in}}%
\pgfpathlineto{\pgfqpoint{2.984302in}{1.667995in}}%
\pgfpathlineto{\pgfqpoint{2.987744in}{1.646825in}}%
\pgfpathlineto{\pgfqpoint{2.992906in}{1.596704in}}%
\pgfpathlineto{\pgfqpoint{2.999790in}{1.501592in}}%
\pgfpathlineto{\pgfqpoint{3.010975in}{1.307777in}}%
\pgfpathlineto{\pgfqpoint{3.023882in}{1.091779in}}%
\pgfpathlineto{\pgfqpoint{3.031626in}{0.997865in}}%
\pgfpathlineto{\pgfqpoint{3.036788in}{0.957611in}}%
\pgfpathlineto{\pgfqpoint{3.041091in}{0.939403in}}%
\pgfpathlineto{\pgfqpoint{3.043672in}{0.935395in}}%
\pgfpathlineto{\pgfqpoint{3.045393in}{0.935613in}}%
\pgfpathlineto{\pgfqpoint{3.047114in}{0.938123in}}%
\pgfpathlineto{\pgfqpoint{3.049695in}{0.946112in}}%
\pgfpathlineto{\pgfqpoint{3.053137in}{0.964379in}}%
\pgfpathlineto{\pgfqpoint{3.058299in}{1.006876in}}%
\pgfpathlineto{\pgfqpoint{3.065183in}{1.087282in}}%
\pgfpathlineto{\pgfqpoint{3.075508in}{1.240143in}}%
\pgfpathlineto{\pgfqpoint{3.090135in}{1.455420in}}%
\pgfpathlineto{\pgfqpoint{3.097019in}{1.528137in}}%
\pgfpathlineto{\pgfqpoint{3.102181in}{1.562555in}}%
\pgfpathlineto{\pgfqpoint{3.105623in}{1.574482in}}%
\pgfpathlineto{\pgfqpoint{3.108204in}{1.577341in}}%
\pgfpathlineto{\pgfqpoint{3.109925in}{1.576311in}}%
\pgfpathlineto{\pgfqpoint{3.112506in}{1.570376in}}%
\pgfpathlineto{\pgfqpoint{3.115948in}{1.554447in}}%
\pgfpathlineto{\pgfqpoint{3.120250in}{1.522360in}}%
\pgfpathlineto{\pgfqpoint{3.126273in}{1.457506in}}%
\pgfpathlineto{\pgfqpoint{3.134877in}{1.335396in}}%
\pgfpathlineto{\pgfqpoint{3.158109in}{0.987665in}}%
\pgfpathlineto{\pgfqpoint{3.164992in}{0.919508in}}%
\pgfpathlineto{\pgfqpoint{3.170155in}{0.886066in}}%
\pgfpathlineto{\pgfqpoint{3.174457in}{0.870631in}}%
\pgfpathlineto{\pgfqpoint{3.177039in}{0.866861in}}%
\pgfpathlineto{\pgfqpoint{3.178759in}{0.866612in}}%
\pgfpathlineto{\pgfqpoint{3.180480in}{0.868141in}}%
\pgfpathlineto{\pgfqpoint{3.183062in}{0.873687in}}%
\pgfpathlineto{\pgfqpoint{3.186503in}{0.886875in}}%
\pgfpathlineto{\pgfqpoint{3.191666in}{0.917952in}}%
\pgfpathlineto{\pgfqpoint{3.198549in}{0.976678in}}%
\pgfpathlineto{\pgfqpoint{3.209735in}{1.096451in}}%
\pgfpathlineto{\pgfqpoint{3.222641in}{1.229488in}}%
\pgfpathlineto{\pgfqpoint{3.229525in}{1.279872in}}%
\pgfpathlineto{\pgfqpoint{3.234687in}{1.302931in}}%
\pgfpathlineto{\pgfqpoint{3.238129in}{1.310291in}}%
\pgfpathlineto{\pgfqpoint{3.240710in}{1.311399in}}%
\pgfpathlineto{\pgfqpoint{3.242431in}{1.310014in}}%
\pgfpathlineto{\pgfqpoint{3.245013in}{1.304766in}}%
\pgfpathlineto{\pgfqpoint{3.248454in}{1.291985in}}%
\pgfpathlineto{\pgfqpoint{3.252756in}{1.267229in}}%
\pgfpathlineto{\pgfqpoint{3.258779in}{1.218183in}}%
\pgfpathlineto{\pgfqpoint{3.267384in}{1.126695in}}%
\pgfpathlineto{\pgfqpoint{3.292336in}{0.847147in}}%
\pgfpathlineto{\pgfqpoint{3.299220in}{0.796367in}}%
\pgfpathlineto{\pgfqpoint{3.305243in}{0.767489in}}%
\pgfpathlineto{\pgfqpoint{3.309545in}{0.756245in}}%
\pgfpathlineto{\pgfqpoint{3.312986in}{0.752857in}}%
\pgfpathlineto{\pgfqpoint{3.315568in}{0.753491in}}%
\pgfpathlineto{\pgfqpoint{3.318149in}{0.756732in}}%
\pgfpathlineto{\pgfqpoint{3.321591in}{0.764868in}}%
\pgfpathlineto{\pgfqpoint{3.326753in}{0.784390in}}%
\pgfpathlineto{\pgfqpoint{3.333637in}{0.821302in}}%
\pgfpathlineto{\pgfqpoint{3.346543in}{0.906900in}}%
\pgfpathlineto{\pgfqpoint{3.356868in}{0.969175in}}%
\pgfpathlineto{\pgfqpoint{3.362891in}{0.994859in}}%
\pgfpathlineto{\pgfqpoint{3.368054in}{1.007977in}}%
\pgfpathlineto{\pgfqpoint{3.371496in}{1.011578in}}%
\pgfpathlineto{\pgfqpoint{3.374077in}{1.011457in}}%
\pgfpathlineto{\pgfqpoint{3.376658in}{1.008898in}}%
\pgfpathlineto{\pgfqpoint{3.380100in}{1.001755in}}%
\pgfpathlineto{\pgfqpoint{3.384402in}{0.987109in}}%
\pgfpathlineto{\pgfqpoint{3.390425in}{0.957113in}}%
\pgfpathlineto{\pgfqpoint{3.399030in}{0.899809in}}%
\pgfpathlineto{\pgfqpoint{3.425703in}{0.709410in}}%
\pgfpathlineto{\pgfqpoint{3.432586in}{0.677836in}}%
\pgfpathlineto{\pgfqpoint{3.438609in}{0.659980in}}%
\pgfpathlineto{\pgfqpoint{3.442911in}{0.653023in}}%
\pgfpathlineto{\pgfqpoint{3.446353in}{0.650873in}}%
\pgfpathlineto{\pgfqpoint{3.449795in}{0.651617in}}%
\pgfpathlineto{\pgfqpoint{3.453237in}{0.655055in}}%
\pgfpathlineto{\pgfqpoint{3.457539in}{0.662735in}}%
\pgfpathlineto{\pgfqpoint{3.463562in}{0.678624in}}%
\pgfpathlineto{\pgfqpoint{3.473887in}{0.713895in}}%
\pgfpathlineto{\pgfqpoint{3.485933in}{0.753216in}}%
\pgfpathlineto{\pgfqpoint{3.491956in}{0.766494in}}%
\pgfpathlineto{\pgfqpoint{3.496258in}{0.771974in}}%
\pgfpathlineto{\pgfqpoint{3.499700in}{0.773640in}}%
\pgfpathlineto{\pgfqpoint{3.503142in}{0.772780in}}%
\pgfpathlineto{\pgfqpoint{3.506583in}{0.769390in}}%
\pgfpathlineto{\pgfqpoint{3.510885in}{0.761730in}}%
\pgfpathlineto{\pgfqpoint{3.516908in}{0.745280in}}%
\pgfpathlineto{\pgfqpoint{3.524652in}{0.716562in}}%
\pgfpathlineto{\pgfqpoint{3.557349in}{0.585386in}}%
\pgfpathlineto{\pgfqpoint{3.565953in}{0.563982in}}%
\pgfpathlineto{\pgfqpoint{3.573697in}{0.550506in}}%
\pgfpathlineto{\pgfqpoint{3.581441in}{0.541538in}}%
\pgfpathlineto{\pgfqpoint{3.589185in}{0.535943in}}%
\pgfpathlineto{\pgfqpoint{3.597789in}{0.532405in}}%
\pgfpathlineto{\pgfqpoint{3.608974in}{0.530260in}}%
\pgfpathlineto{\pgfqpoint{3.627904in}{0.529253in}}%
\pgfpathlineto{\pgfqpoint{3.702761in}{0.529102in}}%
\pgfpathlineto{\pgfqpoint{3.839570in}{0.529102in}}%
\pgfpathlineto{\pgfqpoint{3.839570in}{0.529102in}}%
\pgfusepath{stroke}%
\end{pgfscope}%
\begin{pgfscope}%
\pgfpathrectangle{\pgfqpoint{2.120000in}{0.529102in}}{\pgfqpoint{1.720000in}{1.224232in}} %
\pgfusepath{clip}%
\pgfsetbuttcap%
\pgfsetroundjoin%
\pgfsetlinewidth{1.003750pt}%
\definecolor{currentstroke}{rgb}{1.000000,0.400000,0.200000}%
\pgfsetstrokecolor{currentstroke}%
\pgfsetdash{{1.000000pt}{3.000000pt}}{0.000000pt}%
\pgfpathmoveto{\pgfqpoint{2.120430in}{0.529102in}}%
\pgfpathlineto{\pgfqpoint{2.320050in}{0.530168in}}%
\pgfpathlineto{\pgfqpoint{2.341561in}{0.533303in}}%
\pgfpathlineto{\pgfqpoint{2.349305in}{0.536247in}}%
\pgfpathlineto{\pgfqpoint{2.363932in}{0.545082in}}%
\pgfpathlineto{\pgfqpoint{2.370815in}{0.552284in}}%
\pgfpathlineto{\pgfqpoint{2.378559in}{0.563857in}}%
\pgfpathlineto{\pgfqpoint{2.388884in}{0.583628in}}%
\pgfpathlineto{\pgfqpoint{2.400070in}{0.610129in}}%
\pgfpathlineto{\pgfqpoint{2.409535in}{0.635333in}}%
\pgfpathlineto{\pgfqpoint{2.441371in}{0.735971in}}%
\pgfpathlineto{\pgfqpoint{2.447394in}{0.747433in}}%
\pgfpathlineto{\pgfqpoint{2.462021in}{0.771269in}}%
\pgfpathlineto{\pgfqpoint{2.466323in}{0.771727in}}%
\pgfpathlineto{\pgfqpoint{2.474927in}{0.770369in}}%
\pgfpathlineto{\pgfqpoint{2.492996in}{0.759696in}}%
\pgfpathlineto{\pgfqpoint{2.504182in}{0.749365in}}%
\pgfpathlineto{\pgfqpoint{2.518809in}{0.749621in}}%
\pgfpathlineto{\pgfqpoint{2.522251in}{0.753707in}}%
\pgfpathlineto{\pgfqpoint{2.527414in}{0.764251in}}%
\pgfpathlineto{\pgfqpoint{2.536878in}{0.789774in}}%
\pgfpathlineto{\pgfqpoint{2.542901in}{0.812894in}}%
\pgfpathlineto{\pgfqpoint{2.550645in}{0.853347in}}%
\pgfpathlineto{\pgfqpoint{2.559250in}{0.895130in}}%
\pgfpathlineto{\pgfqpoint{2.579040in}{0.980250in}}%
\pgfpathlineto{\pgfqpoint{2.587644in}{1.003524in}}%
\pgfpathlineto{\pgfqpoint{2.591086in}{1.006578in}}%
\pgfpathlineto{\pgfqpoint{2.594527in}{1.005864in}}%
\pgfpathlineto{\pgfqpoint{2.599690in}{1.004443in}}%
\pgfpathlineto{\pgfqpoint{2.603992in}{1.003753in}}%
\pgfpathlineto{\pgfqpoint{2.606573in}{1.001183in}}%
\pgfpathlineto{\pgfqpoint{2.611736in}{0.991345in}}%
\pgfpathlineto{\pgfqpoint{2.618619in}{0.973794in}}%
\pgfpathlineto{\pgfqpoint{2.628944in}{0.945692in}}%
\pgfpathlineto{\pgfqpoint{2.636688in}{0.928962in}}%
\pgfpathlineto{\pgfqpoint{2.640990in}{0.923810in}}%
\pgfpathlineto{\pgfqpoint{2.643572in}{0.922783in}}%
\pgfpathlineto{\pgfqpoint{2.646153in}{0.924048in}}%
\pgfpathlineto{\pgfqpoint{2.649595in}{0.929752in}}%
\pgfpathlineto{\pgfqpoint{2.656478in}{0.948481in}}%
\pgfpathlineto{\pgfqpoint{2.662501in}{0.969460in}}%
\pgfpathlineto{\pgfqpoint{2.668524in}{1.000676in}}%
\pgfpathlineto{\pgfqpoint{2.684872in}{1.106201in}}%
\pgfpathlineto{\pgfqpoint{2.699500in}{1.221260in}}%
\pgfpathlineto{\pgfqpoint{2.708104in}{1.268824in}}%
\pgfpathlineto{\pgfqpoint{2.715848in}{1.295142in}}%
\pgfpathlineto{\pgfqpoint{2.720150in}{1.304429in}}%
\pgfpathlineto{\pgfqpoint{2.723592in}{1.306563in}}%
\pgfpathlineto{\pgfqpoint{2.727034in}{1.306176in}}%
\pgfpathlineto{\pgfqpoint{2.729615in}{1.304043in}}%
\pgfpathlineto{\pgfqpoint{2.733057in}{1.297699in}}%
\pgfpathlineto{\pgfqpoint{2.737359in}{1.284815in}}%
\pgfpathlineto{\pgfqpoint{2.743382in}{1.258018in}}%
\pgfpathlineto{\pgfqpoint{2.762311in}{1.167810in}}%
\pgfpathlineto{\pgfqpoint{2.771776in}{1.130574in}}%
\pgfpathlineto{\pgfqpoint{2.775218in}{1.122881in}}%
\pgfpathlineto{\pgfqpoint{2.777799in}{1.121208in}}%
\pgfpathlineto{\pgfqpoint{2.780380in}{1.122917in}}%
\pgfpathlineto{\pgfqpoint{2.783822in}{1.129750in}}%
\pgfpathlineto{\pgfqpoint{2.788984in}{1.147299in}}%
\pgfpathlineto{\pgfqpoint{2.795868in}{1.179452in}}%
\pgfpathlineto{\pgfqpoint{2.801891in}{1.217432in}}%
\pgfpathlineto{\pgfqpoint{2.810495in}{1.292792in}}%
\pgfpathlineto{\pgfqpoint{2.821681in}{1.402220in}}%
\pgfpathlineto{\pgfqpoint{2.832866in}{1.501067in}}%
\pgfpathlineto{\pgfqpoint{2.839750in}{1.545347in}}%
\pgfpathlineto{\pgfqpoint{2.844052in}{1.562277in}}%
\pgfpathlineto{\pgfqpoint{2.847494in}{1.568971in}}%
\pgfpathlineto{\pgfqpoint{2.851796in}{1.572392in}}%
\pgfpathlineto{\pgfqpoint{2.854377in}{1.572655in}}%
\pgfpathlineto{\pgfqpoint{2.856958in}{1.570079in}}%
\pgfpathlineto{\pgfqpoint{2.859540in}{1.563068in}}%
\pgfpathlineto{\pgfqpoint{2.863842in}{1.542158in}}%
\pgfpathlineto{\pgfqpoint{2.876748in}{1.462669in}}%
\pgfpathlineto{\pgfqpoint{2.899980in}{1.297022in}}%
\pgfpathlineto{\pgfqpoint{2.907724in}{1.268875in}}%
\pgfpathlineto{\pgfqpoint{2.911166in}{1.262710in}}%
\pgfpathlineto{\pgfqpoint{2.913747in}{1.262294in}}%
\pgfpathlineto{\pgfqpoint{2.916328in}{1.265170in}}%
\pgfpathlineto{\pgfqpoint{2.919770in}{1.273945in}}%
\pgfpathlineto{\pgfqpoint{2.924932in}{1.296256in}}%
\pgfpathlineto{\pgfqpoint{2.930955in}{1.332649in}}%
\pgfpathlineto{\pgfqpoint{2.939560in}{1.400519in}}%
\pgfpathlineto{\pgfqpoint{2.966233in}{1.632248in}}%
\pgfpathlineto{\pgfqpoint{2.972256in}{1.664763in}}%
\pgfpathlineto{\pgfqpoint{2.975698in}{1.673268in}}%
\pgfpathlineto{\pgfqpoint{2.979140in}{1.676411in}}%
\pgfpathlineto{\pgfqpoint{2.981721in}{1.675850in}}%
\pgfpathlineto{\pgfqpoint{2.985163in}{1.671521in}}%
\pgfpathlineto{\pgfqpoint{2.988604in}{1.663252in}}%
\pgfpathlineto{\pgfqpoint{2.992046in}{1.648344in}}%
\pgfpathlineto{\pgfqpoint{2.996348in}{1.618363in}}%
\pgfpathlineto{\pgfqpoint{3.012696in}{1.474052in}}%
\pgfpathlineto{\pgfqpoint{3.026463in}{1.346890in}}%
\pgfpathlineto{\pgfqpoint{3.032486in}{1.303026in}}%
\pgfpathlineto{\pgfqpoint{3.038509in}{1.275528in}}%
\pgfpathlineto{\pgfqpoint{3.042811in}{1.262721in}}%
\pgfpathlineto{\pgfqpoint{3.045393in}{1.260011in}}%
\pgfpathlineto{\pgfqpoint{3.047114in}{1.260555in}}%
\pgfpathlineto{\pgfqpoint{3.049695in}{1.264240in}}%
\pgfpathlineto{\pgfqpoint{3.054857in}{1.277525in}}%
\pgfpathlineto{\pgfqpoint{3.060020in}{1.296590in}}%
\pgfpathlineto{\pgfqpoint{3.065183in}{1.327509in}}%
\pgfpathlineto{\pgfqpoint{3.078089in}{1.421486in}}%
\pgfpathlineto{\pgfqpoint{3.088414in}{1.506247in}}%
\pgfpathlineto{\pgfqpoint{3.094437in}{1.535573in}}%
\pgfpathlineto{\pgfqpoint{3.104762in}{1.573694in}}%
\pgfpathlineto{\pgfqpoint{3.107344in}{1.575132in}}%
\pgfpathlineto{\pgfqpoint{3.109925in}{1.572256in}}%
\pgfpathlineto{\pgfqpoint{3.115088in}{1.561183in}}%
\pgfpathlineto{\pgfqpoint{3.118529in}{1.549725in}}%
\pgfpathlineto{\pgfqpoint{3.122831in}{1.527245in}}%
\pgfpathlineto{\pgfqpoint{3.129715in}{1.477946in}}%
\pgfpathlineto{\pgfqpoint{3.140900in}{1.375020in}}%
\pgfpathlineto{\pgfqpoint{3.163272in}{1.182097in}}%
\pgfpathlineto{\pgfqpoint{3.169295in}{1.153127in}}%
\pgfpathlineto{\pgfqpoint{3.174457in}{1.136709in}}%
\pgfpathlineto{\pgfqpoint{3.177899in}{1.130739in}}%
\pgfpathlineto{\pgfqpoint{3.180480in}{1.129112in}}%
\pgfpathlineto{\pgfqpoint{3.183062in}{1.129692in}}%
\pgfpathlineto{\pgfqpoint{3.186503in}{1.133542in}}%
\pgfpathlineto{\pgfqpoint{3.190805in}{1.142850in}}%
\pgfpathlineto{\pgfqpoint{3.197689in}{1.165037in}}%
\pgfpathlineto{\pgfqpoint{3.206293in}{1.201028in}}%
\pgfpathlineto{\pgfqpoint{3.223502in}{1.283651in}}%
\pgfpathlineto{\pgfqpoint{3.231246in}{1.307000in}}%
\pgfpathlineto{\pgfqpoint{3.233827in}{1.309624in}}%
\pgfpathlineto{\pgfqpoint{3.236408in}{1.309352in}}%
\pgfpathlineto{\pgfqpoint{3.238989in}{1.306820in}}%
\pgfpathlineto{\pgfqpoint{3.242431in}{1.299963in}}%
\pgfpathlineto{\pgfqpoint{3.247594in}{1.283251in}}%
\pgfpathlineto{\pgfqpoint{3.257059in}{1.244110in}}%
\pgfpathlineto{\pgfqpoint{3.263082in}{1.205119in}}%
\pgfpathlineto{\pgfqpoint{3.294917in}{0.978444in}}%
\pgfpathlineto{\pgfqpoint{3.301801in}{0.944943in}}%
\pgfpathlineto{\pgfqpoint{3.306103in}{0.932962in}}%
\pgfpathlineto{\pgfqpoint{3.310405in}{0.927790in}}%
\pgfpathlineto{\pgfqpoint{3.315568in}{0.925167in}}%
\pgfpathlineto{\pgfqpoint{3.319010in}{0.925326in}}%
\pgfpathlineto{\pgfqpoint{3.322451in}{0.928455in}}%
\pgfpathlineto{\pgfqpoint{3.326753in}{0.936444in}}%
\pgfpathlineto{\pgfqpoint{3.339660in}{0.967381in}}%
\pgfpathlineto{\pgfqpoint{3.349125in}{0.991913in}}%
\pgfpathlineto{\pgfqpoint{3.358589in}{1.008500in}}%
\pgfpathlineto{\pgfqpoint{3.362031in}{1.010599in}}%
\pgfpathlineto{\pgfqpoint{3.365473in}{1.009950in}}%
\pgfpathlineto{\pgfqpoint{3.369775in}{1.006462in}}%
\pgfpathlineto{\pgfqpoint{3.374077in}{0.999868in}}%
\pgfpathlineto{\pgfqpoint{3.379240in}{0.986623in}}%
\pgfpathlineto{\pgfqpoint{3.388704in}{0.952755in}}%
\pgfpathlineto{\pgfqpoint{3.395588in}{0.922720in}}%
\pgfpathlineto{\pgfqpoint{3.419680in}{0.806681in}}%
\pgfpathlineto{\pgfqpoint{3.427424in}{0.779339in}}%
\pgfpathlineto{\pgfqpoint{3.434307in}{0.762943in}}%
\pgfpathlineto{\pgfqpoint{3.441191in}{0.752433in}}%
\pgfpathlineto{\pgfqpoint{3.446353in}{0.747568in}}%
\pgfpathlineto{\pgfqpoint{3.451516in}{0.745269in}}%
\pgfpathlineto{\pgfqpoint{3.454957in}{0.745614in}}%
\pgfpathlineto{\pgfqpoint{3.458399in}{0.748127in}}%
\pgfpathlineto{\pgfqpoint{3.471306in}{0.761270in}}%
\pgfpathlineto{\pgfqpoint{3.477329in}{0.763765in}}%
\pgfpathlineto{\pgfqpoint{3.489375in}{0.771904in}}%
\pgfpathlineto{\pgfqpoint{3.498839in}{0.772554in}}%
\pgfpathlineto{\pgfqpoint{3.502281in}{0.770176in}}%
\pgfpathlineto{\pgfqpoint{3.506583in}{0.763585in}}%
\pgfpathlineto{\pgfqpoint{3.516908in}{0.741373in}}%
\pgfpathlineto{\pgfqpoint{3.523792in}{0.721182in}}%
\pgfpathlineto{\pgfqpoint{3.549605in}{0.637833in}}%
\pgfpathlineto{\pgfqpoint{3.562511in}{0.602359in}}%
\pgfpathlineto{\pgfqpoint{3.573697in}{0.577641in}}%
\pgfpathlineto{\pgfqpoint{3.582301in}{0.561980in}}%
\pgfpathlineto{\pgfqpoint{3.592626in}{0.548342in}}%
\pgfpathlineto{\pgfqpoint{3.600370in}{0.541695in}}%
\pgfpathlineto{\pgfqpoint{3.609835in}{0.536578in}}%
\pgfpathlineto{\pgfqpoint{3.620160in}{0.533092in}}%
\pgfpathlineto{\pgfqpoint{3.635648in}{0.530420in}}%
\pgfpathlineto{\pgfqpoint{3.657159in}{0.529341in}}%
\pgfpathlineto{\pgfqpoint{3.799130in}{0.529102in}}%
\pgfpathlineto{\pgfqpoint{3.839570in}{0.529102in}}%
\pgfpathlineto{\pgfqpoint{3.839570in}{0.529102in}}%
\pgfusepath{stroke}%
\end{pgfscope}%
\begin{pgfscope}%
\pgfsetrectcap%
\pgfsetmiterjoin%
\pgfsetlinewidth{1.003750pt}%
\definecolor{currentstroke}{rgb}{0.000000,0.000000,0.000000}%
\pgfsetstrokecolor{currentstroke}%
\pgfsetdash{}{0pt}%
\pgfpathmoveto{\pgfqpoint{3.840000in}{0.529102in}}%
\pgfpathlineto{\pgfqpoint{3.840000in}{1.753333in}}%
\pgfusepath{stroke}%
\end{pgfscope}%
\begin{pgfscope}%
\pgfsetrectcap%
\pgfsetmiterjoin%
\pgfsetlinewidth{1.003750pt}%
\definecolor{currentstroke}{rgb}{0.000000,0.000000,0.000000}%
\pgfsetstrokecolor{currentstroke}%
\pgfsetdash{}{0pt}%
\pgfpathmoveto{\pgfqpoint{2.120000in}{0.529102in}}%
\pgfpathlineto{\pgfqpoint{3.840000in}{0.529102in}}%
\pgfusepath{stroke}%
\end{pgfscope}%
\begin{pgfscope}%
\pgfsetrectcap%
\pgfsetmiterjoin%
\pgfsetlinewidth{1.003750pt}%
\definecolor{currentstroke}{rgb}{0.000000,0.000000,0.000000}%
\pgfsetstrokecolor{currentstroke}%
\pgfsetdash{}{0pt}%
\pgfpathmoveto{\pgfqpoint{2.120000in}{1.753333in}}%
\pgfpathlineto{\pgfqpoint{3.840000in}{1.753333in}}%
\pgfusepath{stroke}%
\end{pgfscope}%
\begin{pgfscope}%
\pgfsetrectcap%
\pgfsetmiterjoin%
\pgfsetlinewidth{1.003750pt}%
\definecolor{currentstroke}{rgb}{0.000000,0.000000,0.000000}%
\pgfsetstrokecolor{currentstroke}%
\pgfsetdash{}{0pt}%
\pgfpathmoveto{\pgfqpoint{2.120000in}{0.529102in}}%
\pgfpathlineto{\pgfqpoint{2.120000in}{1.753333in}}%
\pgfusepath{stroke}%
\end{pgfscope}%
\begin{pgfscope}%
\pgfsetbuttcap%
\pgfsetroundjoin%
\definecolor{currentfill}{rgb}{0.000000,0.000000,0.000000}%
\pgfsetfillcolor{currentfill}%
\pgfsetlinewidth{0.501875pt}%
\definecolor{currentstroke}{rgb}{0.000000,0.000000,0.000000}%
\pgfsetstrokecolor{currentstroke}%
\pgfsetdash{}{0pt}%
\pgfsys@defobject{currentmarker}{\pgfqpoint{0.000000in}{0.000000in}}{\pgfqpoint{0.000000in}{0.055556in}}{%
\pgfpathmoveto{\pgfqpoint{0.000000in}{0.000000in}}%
\pgfpathlineto{\pgfqpoint{0.000000in}{0.055556in}}%
\pgfusepath{stroke,fill}%
}%
\begin{pgfscope}%
\pgfsys@transformshift{2.550000in}{0.529102in}%
\pgfsys@useobject{currentmarker}{}%
\end{pgfscope}%
\end{pgfscope}%
\begin{pgfscope}%
\pgfsetbuttcap%
\pgfsetroundjoin%
\definecolor{currentfill}{rgb}{0.000000,0.000000,0.000000}%
\pgfsetfillcolor{currentfill}%
\pgfsetlinewidth{0.501875pt}%
\definecolor{currentstroke}{rgb}{0.000000,0.000000,0.000000}%
\pgfsetstrokecolor{currentstroke}%
\pgfsetdash{}{0pt}%
\pgfsys@defobject{currentmarker}{\pgfqpoint{0.000000in}{-0.055556in}}{\pgfqpoint{0.000000in}{0.000000in}}{%
\pgfpathmoveto{\pgfqpoint{0.000000in}{0.000000in}}%
\pgfpathlineto{\pgfqpoint{0.000000in}{-0.055556in}}%
\pgfusepath{stroke,fill}%
}%
\begin{pgfscope}%
\pgfsys@transformshift{2.550000in}{1.753333in}%
\pgfsys@useobject{currentmarker}{}%
\end{pgfscope}%
\end{pgfscope}%
\begin{pgfscope}%
\pgftext[x=2.550000in,y=0.473546in,,top]{\fontsize{10.000000}{12.000000}\selectfont -0.5}%
\end{pgfscope}%
\begin{pgfscope}%
\pgfsetbuttcap%
\pgfsetroundjoin%
\definecolor{currentfill}{rgb}{0.000000,0.000000,0.000000}%
\pgfsetfillcolor{currentfill}%
\pgfsetlinewidth{0.501875pt}%
\definecolor{currentstroke}{rgb}{0.000000,0.000000,0.000000}%
\pgfsetstrokecolor{currentstroke}%
\pgfsetdash{}{0pt}%
\pgfsys@defobject{currentmarker}{\pgfqpoint{0.000000in}{0.000000in}}{\pgfqpoint{0.000000in}{0.055556in}}{%
\pgfpathmoveto{\pgfqpoint{0.000000in}{0.000000in}}%
\pgfpathlineto{\pgfqpoint{0.000000in}{0.055556in}}%
\pgfusepath{stroke,fill}%
}%
\begin{pgfscope}%
\pgfsys@transformshift{2.980000in}{0.529102in}%
\pgfsys@useobject{currentmarker}{}%
\end{pgfscope}%
\end{pgfscope}%
\begin{pgfscope}%
\pgfsetbuttcap%
\pgfsetroundjoin%
\definecolor{currentfill}{rgb}{0.000000,0.000000,0.000000}%
\pgfsetfillcolor{currentfill}%
\pgfsetlinewidth{0.501875pt}%
\definecolor{currentstroke}{rgb}{0.000000,0.000000,0.000000}%
\pgfsetstrokecolor{currentstroke}%
\pgfsetdash{}{0pt}%
\pgfsys@defobject{currentmarker}{\pgfqpoint{0.000000in}{-0.055556in}}{\pgfqpoint{0.000000in}{0.000000in}}{%
\pgfpathmoveto{\pgfqpoint{0.000000in}{0.000000in}}%
\pgfpathlineto{\pgfqpoint{0.000000in}{-0.055556in}}%
\pgfusepath{stroke,fill}%
}%
\begin{pgfscope}%
\pgfsys@transformshift{2.980000in}{1.753333in}%
\pgfsys@useobject{currentmarker}{}%
\end{pgfscope}%
\end{pgfscope}%
\begin{pgfscope}%
\pgftext[x=2.980000in,y=0.473546in,,top]{\fontsize{10.000000}{12.000000}\selectfont 0.0}%
\end{pgfscope}%
\begin{pgfscope}%
\pgfsetbuttcap%
\pgfsetroundjoin%
\definecolor{currentfill}{rgb}{0.000000,0.000000,0.000000}%
\pgfsetfillcolor{currentfill}%
\pgfsetlinewidth{0.501875pt}%
\definecolor{currentstroke}{rgb}{0.000000,0.000000,0.000000}%
\pgfsetstrokecolor{currentstroke}%
\pgfsetdash{}{0pt}%
\pgfsys@defobject{currentmarker}{\pgfqpoint{0.000000in}{0.000000in}}{\pgfqpoint{0.000000in}{0.055556in}}{%
\pgfpathmoveto{\pgfqpoint{0.000000in}{0.000000in}}%
\pgfpathlineto{\pgfqpoint{0.000000in}{0.055556in}}%
\pgfusepath{stroke,fill}%
}%
\begin{pgfscope}%
\pgfsys@transformshift{3.410000in}{0.529102in}%
\pgfsys@useobject{currentmarker}{}%
\end{pgfscope}%
\end{pgfscope}%
\begin{pgfscope}%
\pgfsetbuttcap%
\pgfsetroundjoin%
\definecolor{currentfill}{rgb}{0.000000,0.000000,0.000000}%
\pgfsetfillcolor{currentfill}%
\pgfsetlinewidth{0.501875pt}%
\definecolor{currentstroke}{rgb}{0.000000,0.000000,0.000000}%
\pgfsetstrokecolor{currentstroke}%
\pgfsetdash{}{0pt}%
\pgfsys@defobject{currentmarker}{\pgfqpoint{0.000000in}{-0.055556in}}{\pgfqpoint{0.000000in}{0.000000in}}{%
\pgfpathmoveto{\pgfqpoint{0.000000in}{0.000000in}}%
\pgfpathlineto{\pgfqpoint{0.000000in}{-0.055556in}}%
\pgfusepath{stroke,fill}%
}%
\begin{pgfscope}%
\pgfsys@transformshift{3.410000in}{1.753333in}%
\pgfsys@useobject{currentmarker}{}%
\end{pgfscope}%
\end{pgfscope}%
\begin{pgfscope}%
\pgftext[x=3.410000in,y=0.473546in,,top]{\fontsize{10.000000}{12.000000}\selectfont 0.5}%
\end{pgfscope}%
\begin{pgfscope}%
\pgftext[x=2.980000in,y=0.280645in,,top]{\fontsize{10.000000}{12.000000}\selectfont Position (mm)}%
\end{pgfscope}%
\begin{pgfscope}%
\pgfsetbuttcap%
\pgfsetroundjoin%
\definecolor{currentfill}{rgb}{0.000000,0.000000,0.000000}%
\pgfsetfillcolor{currentfill}%
\pgfsetlinewidth{0.501875pt}%
\definecolor{currentstroke}{rgb}{0.000000,0.000000,0.000000}%
\pgfsetstrokecolor{currentstroke}%
\pgfsetdash{}{0pt}%
\pgfsys@defobject{currentmarker}{\pgfqpoint{0.000000in}{0.000000in}}{\pgfqpoint{0.055556in}{0.000000in}}{%
\pgfpathmoveto{\pgfqpoint{0.000000in}{0.000000in}}%
\pgfpathlineto{\pgfqpoint{0.055556in}{0.000000in}}%
\pgfusepath{stroke,fill}%
}%
\begin{pgfscope}%
\pgfsys@transformshift{2.120000in}{0.529102in}%
\pgfsys@useobject{currentmarker}{}%
\end{pgfscope}%
\end{pgfscope}%
\begin{pgfscope}%
\pgfsetbuttcap%
\pgfsetroundjoin%
\definecolor{currentfill}{rgb}{0.000000,0.000000,0.000000}%
\pgfsetfillcolor{currentfill}%
\pgfsetlinewidth{0.501875pt}%
\definecolor{currentstroke}{rgb}{0.000000,0.000000,0.000000}%
\pgfsetstrokecolor{currentstroke}%
\pgfsetdash{}{0pt}%
\pgfsys@defobject{currentmarker}{\pgfqpoint{-0.055556in}{0.000000in}}{\pgfqpoint{0.000000in}{0.000000in}}{%
\pgfpathmoveto{\pgfqpoint{0.000000in}{0.000000in}}%
\pgfpathlineto{\pgfqpoint{-0.055556in}{0.000000in}}%
\pgfusepath{stroke,fill}%
}%
\begin{pgfscope}%
\pgfsys@transformshift{3.840000in}{0.529102in}%
\pgfsys@useobject{currentmarker}{}%
\end{pgfscope}%
\end{pgfscope}%
\begin{pgfscope}%
\pgfsetbuttcap%
\pgfsetroundjoin%
\definecolor{currentfill}{rgb}{0.000000,0.000000,0.000000}%
\pgfsetfillcolor{currentfill}%
\pgfsetlinewidth{0.501875pt}%
\definecolor{currentstroke}{rgb}{0.000000,0.000000,0.000000}%
\pgfsetstrokecolor{currentstroke}%
\pgfsetdash{}{0pt}%
\pgfsys@defobject{currentmarker}{\pgfqpoint{0.000000in}{0.000000in}}{\pgfqpoint{0.055556in}{0.000000in}}{%
\pgfpathmoveto{\pgfqpoint{0.000000in}{0.000000in}}%
\pgfpathlineto{\pgfqpoint{0.055556in}{0.000000in}}%
\pgfusepath{stroke,fill}%
}%
\begin{pgfscope}%
\pgfsys@transformshift{2.120000in}{0.773948in}%
\pgfsys@useobject{currentmarker}{}%
\end{pgfscope}%
\end{pgfscope}%
\begin{pgfscope}%
\pgfsetbuttcap%
\pgfsetroundjoin%
\definecolor{currentfill}{rgb}{0.000000,0.000000,0.000000}%
\pgfsetfillcolor{currentfill}%
\pgfsetlinewidth{0.501875pt}%
\definecolor{currentstroke}{rgb}{0.000000,0.000000,0.000000}%
\pgfsetstrokecolor{currentstroke}%
\pgfsetdash{}{0pt}%
\pgfsys@defobject{currentmarker}{\pgfqpoint{-0.055556in}{0.000000in}}{\pgfqpoint{0.000000in}{0.000000in}}{%
\pgfpathmoveto{\pgfqpoint{0.000000in}{0.000000in}}%
\pgfpathlineto{\pgfqpoint{-0.055556in}{0.000000in}}%
\pgfusepath{stroke,fill}%
}%
\begin{pgfscope}%
\pgfsys@transformshift{3.840000in}{0.773948in}%
\pgfsys@useobject{currentmarker}{}%
\end{pgfscope}%
\end{pgfscope}%
\begin{pgfscope}%
\pgfsetbuttcap%
\pgfsetroundjoin%
\definecolor{currentfill}{rgb}{0.000000,0.000000,0.000000}%
\pgfsetfillcolor{currentfill}%
\pgfsetlinewidth{0.501875pt}%
\definecolor{currentstroke}{rgb}{0.000000,0.000000,0.000000}%
\pgfsetstrokecolor{currentstroke}%
\pgfsetdash{}{0pt}%
\pgfsys@defobject{currentmarker}{\pgfqpoint{0.000000in}{0.000000in}}{\pgfqpoint{0.055556in}{0.000000in}}{%
\pgfpathmoveto{\pgfqpoint{0.000000in}{0.000000in}}%
\pgfpathlineto{\pgfqpoint{0.055556in}{0.000000in}}%
\pgfusepath{stroke,fill}%
}%
\begin{pgfscope}%
\pgfsys@transformshift{2.120000in}{1.018794in}%
\pgfsys@useobject{currentmarker}{}%
\end{pgfscope}%
\end{pgfscope}%
\begin{pgfscope}%
\pgfsetbuttcap%
\pgfsetroundjoin%
\definecolor{currentfill}{rgb}{0.000000,0.000000,0.000000}%
\pgfsetfillcolor{currentfill}%
\pgfsetlinewidth{0.501875pt}%
\definecolor{currentstroke}{rgb}{0.000000,0.000000,0.000000}%
\pgfsetstrokecolor{currentstroke}%
\pgfsetdash{}{0pt}%
\pgfsys@defobject{currentmarker}{\pgfqpoint{-0.055556in}{0.000000in}}{\pgfqpoint{0.000000in}{0.000000in}}{%
\pgfpathmoveto{\pgfqpoint{0.000000in}{0.000000in}}%
\pgfpathlineto{\pgfqpoint{-0.055556in}{0.000000in}}%
\pgfusepath{stroke,fill}%
}%
\begin{pgfscope}%
\pgfsys@transformshift{3.840000in}{1.018794in}%
\pgfsys@useobject{currentmarker}{}%
\end{pgfscope}%
\end{pgfscope}%
\begin{pgfscope}%
\pgfsetbuttcap%
\pgfsetroundjoin%
\definecolor{currentfill}{rgb}{0.000000,0.000000,0.000000}%
\pgfsetfillcolor{currentfill}%
\pgfsetlinewidth{0.501875pt}%
\definecolor{currentstroke}{rgb}{0.000000,0.000000,0.000000}%
\pgfsetstrokecolor{currentstroke}%
\pgfsetdash{}{0pt}%
\pgfsys@defobject{currentmarker}{\pgfqpoint{0.000000in}{0.000000in}}{\pgfqpoint{0.055556in}{0.000000in}}{%
\pgfpathmoveto{\pgfqpoint{0.000000in}{0.000000in}}%
\pgfpathlineto{\pgfqpoint{0.055556in}{0.000000in}}%
\pgfusepath{stroke,fill}%
}%
\begin{pgfscope}%
\pgfsys@transformshift{2.120000in}{1.263641in}%
\pgfsys@useobject{currentmarker}{}%
\end{pgfscope}%
\end{pgfscope}%
\begin{pgfscope}%
\pgfsetbuttcap%
\pgfsetroundjoin%
\definecolor{currentfill}{rgb}{0.000000,0.000000,0.000000}%
\pgfsetfillcolor{currentfill}%
\pgfsetlinewidth{0.501875pt}%
\definecolor{currentstroke}{rgb}{0.000000,0.000000,0.000000}%
\pgfsetstrokecolor{currentstroke}%
\pgfsetdash{}{0pt}%
\pgfsys@defobject{currentmarker}{\pgfqpoint{-0.055556in}{0.000000in}}{\pgfqpoint{0.000000in}{0.000000in}}{%
\pgfpathmoveto{\pgfqpoint{0.000000in}{0.000000in}}%
\pgfpathlineto{\pgfqpoint{-0.055556in}{0.000000in}}%
\pgfusepath{stroke,fill}%
}%
\begin{pgfscope}%
\pgfsys@transformshift{3.840000in}{1.263641in}%
\pgfsys@useobject{currentmarker}{}%
\end{pgfscope}%
\end{pgfscope}%
\begin{pgfscope}%
\pgfsetbuttcap%
\pgfsetroundjoin%
\definecolor{currentfill}{rgb}{0.000000,0.000000,0.000000}%
\pgfsetfillcolor{currentfill}%
\pgfsetlinewidth{0.501875pt}%
\definecolor{currentstroke}{rgb}{0.000000,0.000000,0.000000}%
\pgfsetstrokecolor{currentstroke}%
\pgfsetdash{}{0pt}%
\pgfsys@defobject{currentmarker}{\pgfqpoint{0.000000in}{0.000000in}}{\pgfqpoint{0.055556in}{0.000000in}}{%
\pgfpathmoveto{\pgfqpoint{0.000000in}{0.000000in}}%
\pgfpathlineto{\pgfqpoint{0.055556in}{0.000000in}}%
\pgfusepath{stroke,fill}%
}%
\begin{pgfscope}%
\pgfsys@transformshift{2.120000in}{1.508487in}%
\pgfsys@useobject{currentmarker}{}%
\end{pgfscope}%
\end{pgfscope}%
\begin{pgfscope}%
\pgfsetbuttcap%
\pgfsetroundjoin%
\definecolor{currentfill}{rgb}{0.000000,0.000000,0.000000}%
\pgfsetfillcolor{currentfill}%
\pgfsetlinewidth{0.501875pt}%
\definecolor{currentstroke}{rgb}{0.000000,0.000000,0.000000}%
\pgfsetstrokecolor{currentstroke}%
\pgfsetdash{}{0pt}%
\pgfsys@defobject{currentmarker}{\pgfqpoint{-0.055556in}{0.000000in}}{\pgfqpoint{0.000000in}{0.000000in}}{%
\pgfpathmoveto{\pgfqpoint{0.000000in}{0.000000in}}%
\pgfpathlineto{\pgfqpoint{-0.055556in}{0.000000in}}%
\pgfusepath{stroke,fill}%
}%
\begin{pgfscope}%
\pgfsys@transformshift{3.840000in}{1.508487in}%
\pgfsys@useobject{currentmarker}{}%
\end{pgfscope}%
\end{pgfscope}%
\begin{pgfscope}%
\pgfsetbuttcap%
\pgfsetroundjoin%
\definecolor{currentfill}{rgb}{0.000000,0.000000,0.000000}%
\pgfsetfillcolor{currentfill}%
\pgfsetlinewidth{0.501875pt}%
\definecolor{currentstroke}{rgb}{0.000000,0.000000,0.000000}%
\pgfsetstrokecolor{currentstroke}%
\pgfsetdash{}{0pt}%
\pgfsys@defobject{currentmarker}{\pgfqpoint{0.000000in}{0.000000in}}{\pgfqpoint{0.055556in}{0.000000in}}{%
\pgfpathmoveto{\pgfqpoint{0.000000in}{0.000000in}}%
\pgfpathlineto{\pgfqpoint{0.055556in}{0.000000in}}%
\pgfusepath{stroke,fill}%
}%
\begin{pgfscope}%
\pgfsys@transformshift{2.120000in}{1.753333in}%
\pgfsys@useobject{currentmarker}{}%
\end{pgfscope}%
\end{pgfscope}%
\begin{pgfscope}%
\pgfsetbuttcap%
\pgfsetroundjoin%
\definecolor{currentfill}{rgb}{0.000000,0.000000,0.000000}%
\pgfsetfillcolor{currentfill}%
\pgfsetlinewidth{0.501875pt}%
\definecolor{currentstroke}{rgb}{0.000000,0.000000,0.000000}%
\pgfsetstrokecolor{currentstroke}%
\pgfsetdash{}{0pt}%
\pgfsys@defobject{currentmarker}{\pgfqpoint{-0.055556in}{0.000000in}}{\pgfqpoint{0.000000in}{0.000000in}}{%
\pgfpathmoveto{\pgfqpoint{0.000000in}{0.000000in}}%
\pgfpathlineto{\pgfqpoint{-0.055556in}{0.000000in}}%
\pgfusepath{stroke,fill}%
}%
\begin{pgfscope}%
\pgfsys@transformshift{3.840000in}{1.753333in}%
\pgfsys@useobject{currentmarker}{}%
\end{pgfscope}%
\end{pgfscope}%
\begin{pgfscope}%
\pgfsetbuttcap%
\pgfsetmiterjoin%
\definecolor{currentfill}{rgb}{1.000000,1.000000,1.000000}%
\pgfsetfillcolor{currentfill}%
\pgfsetlinewidth{0.000000pt}%
\definecolor{currentstroke}{rgb}{0.000000,0.000000,0.000000}%
\pgfsetstrokecolor{currentstroke}%
\pgfsetstrokeopacity{0.000000}%
\pgfsetdash{}{0pt}%
\pgfpathmoveto{\pgfqpoint{3.840000in}{0.529102in}}%
\pgfpathlineto{\pgfqpoint{5.560000in}{0.529102in}}%
\pgfpathlineto{\pgfqpoint{5.560000in}{1.753333in}}%
\pgfpathlineto{\pgfqpoint{3.840000in}{1.753333in}}%
\pgfpathclose%
\pgfusepath{fill}%
\end{pgfscope}%
\begin{pgfscope}%
\pgfpathrectangle{\pgfqpoint{3.840000in}{0.529102in}}{\pgfqpoint{1.720000in}{1.224232in}} %
\pgfusepath{clip}%
\pgfsetrectcap%
\pgfsetroundjoin%
\pgfsetlinewidth{1.003750pt}%
\definecolor{currentstroke}{rgb}{0.301961,0.607843,0.301961}%
\pgfsetstrokecolor{currentstroke}%
\pgfsetstrokeopacity{0.500000}%
\pgfsetdash{}{0pt}%
\pgfpathmoveto{\pgfqpoint{4.052082in}{0.531737in}}%
\pgfpathlineto{\pgfqpoint{4.054782in}{0.532257in}}%
\pgfpathlineto{\pgfqpoint{4.057483in}{0.532865in}}%
\pgfpathlineto{\pgfqpoint{4.060183in}{0.533574in}}%
\pgfpathlineto{\pgfqpoint{4.062883in}{0.534398in}}%
\pgfpathlineto{\pgfqpoint{4.065584in}{0.535350in}}%
\pgfpathlineto{\pgfqpoint{4.068284in}{0.536445in}}%
\pgfpathlineto{\pgfqpoint{4.070985in}{0.537702in}}%
\pgfpathlineto{\pgfqpoint{4.073685in}{0.539137in}}%
\pgfpathlineto{\pgfqpoint{4.076386in}{0.540768in}}%
\pgfpathlineto{\pgfqpoint{4.079086in}{0.542614in}}%
\pgfpathlineto{\pgfqpoint{4.081787in}{0.544695in}}%
\pgfpathlineto{\pgfqpoint{4.084487in}{0.547031in}}%
\pgfpathlineto{\pgfqpoint{4.087188in}{0.549641in}}%
\pgfpathlineto{\pgfqpoint{4.089888in}{0.552544in}}%
\pgfpathlineto{\pgfqpoint{4.092589in}{0.555761in}}%
\pgfpathlineto{\pgfqpoint{4.095289in}{0.559307in}}%
\pgfpathlineto{\pgfqpoint{4.097990in}{0.563199in}}%
\pgfpathlineto{\pgfqpoint{4.100690in}{0.567452in}}%
\pgfpathlineto{\pgfqpoint{4.103391in}{0.572077in}}%
\pgfpathlineto{\pgfqpoint{4.106091in}{0.577083in}}%
\pgfpathlineto{\pgfqpoint{4.108792in}{0.582477in}}%
\pgfpathlineto{\pgfqpoint{4.111492in}{0.588258in}}%
\pgfpathlineto{\pgfqpoint{4.114193in}{0.594426in}}%
\pgfpathlineto{\pgfqpoint{4.116893in}{0.600972in}}%
\pgfpathlineto{\pgfqpoint{4.119594in}{0.607884in}}%
\pgfpathlineto{\pgfqpoint{4.122294in}{0.615144in}}%
\pgfpathlineto{\pgfqpoint{4.124995in}{0.622729in}}%
\pgfpathlineto{\pgfqpoint{4.127695in}{0.630609in}}%
\pgfpathlineto{\pgfqpoint{4.130396in}{0.638749in}}%
\pgfpathlineto{\pgfqpoint{4.133096in}{0.647107in}}%
\pgfpathlineto{\pgfqpoint{4.135797in}{0.655637in}}%
\pgfpathlineto{\pgfqpoint{4.138497in}{0.664285in}}%
\pgfpathlineto{\pgfqpoint{4.141198in}{0.672996in}}%
\pgfpathlineto{\pgfqpoint{4.143898in}{0.681706in}}%
\pgfpathlineto{\pgfqpoint{4.146599in}{0.690350in}}%
\pgfpathlineto{\pgfqpoint{4.149299in}{0.698858in}}%
\pgfpathlineto{\pgfqpoint{4.151999in}{0.707161in}}%
\pgfpathlineto{\pgfqpoint{4.154700in}{0.715185in}}%
\pgfpathlineto{\pgfqpoint{4.157400in}{0.722857in}}%
\pgfpathlineto{\pgfqpoint{4.160101in}{0.730106in}}%
\pgfpathlineto{\pgfqpoint{4.162801in}{0.736862in}}%
\pgfpathlineto{\pgfqpoint{4.165502in}{0.743058in}}%
\pgfpathlineto{\pgfqpoint{4.168202in}{0.748630in}}%
\pgfpathlineto{\pgfqpoint{4.170903in}{0.753522in}}%
\pgfpathlineto{\pgfqpoint{4.173603in}{0.757682in}}%
\pgfpathlineto{\pgfqpoint{4.176304in}{0.761065in}}%
\pgfpathlineto{\pgfqpoint{4.179004in}{0.763635in}}%
\pgfpathlineto{\pgfqpoint{4.181705in}{0.765365in}}%
\pgfpathlineto{\pgfqpoint{4.184405in}{0.766234in}}%
\pgfpathlineto{\pgfqpoint{4.187106in}{0.766234in}}%
\pgfpathlineto{\pgfqpoint{4.189806in}{0.765365in}}%
\pgfpathlineto{\pgfqpoint{4.192507in}{0.763635in}}%
\pgfpathlineto{\pgfqpoint{4.195207in}{0.761065in}}%
\pgfpathlineto{\pgfqpoint{4.197908in}{0.757682in}}%
\pgfpathlineto{\pgfqpoint{4.200608in}{0.753522in}}%
\pgfpathlineto{\pgfqpoint{4.203309in}{0.748630in}}%
\pgfpathlineto{\pgfqpoint{4.206009in}{0.743058in}}%
\pgfpathlineto{\pgfqpoint{4.208710in}{0.736862in}}%
\pgfpathlineto{\pgfqpoint{4.211410in}{0.730106in}}%
\pgfpathlineto{\pgfqpoint{4.214111in}{0.722857in}}%
\pgfpathlineto{\pgfqpoint{4.216811in}{0.715185in}}%
\pgfpathlineto{\pgfqpoint{4.219512in}{0.707161in}}%
\pgfpathlineto{\pgfqpoint{4.222212in}{0.698858in}}%
\pgfpathlineto{\pgfqpoint{4.224913in}{0.690350in}}%
\pgfpathlineto{\pgfqpoint{4.227613in}{0.681706in}}%
\pgfpathlineto{\pgfqpoint{4.230314in}{0.672996in}}%
\pgfpathlineto{\pgfqpoint{4.233014in}{0.664285in}}%
\pgfpathlineto{\pgfqpoint{4.235715in}{0.655637in}}%
\pgfpathlineto{\pgfqpoint{4.238415in}{0.647107in}}%
\pgfpathlineto{\pgfqpoint{4.241115in}{0.638749in}}%
\pgfpathlineto{\pgfqpoint{4.243816in}{0.630609in}}%
\pgfpathlineto{\pgfqpoint{4.246516in}{0.622729in}}%
\pgfpathlineto{\pgfqpoint{4.249217in}{0.615144in}}%
\pgfpathlineto{\pgfqpoint{4.251917in}{0.607884in}}%
\pgfpathlineto{\pgfqpoint{4.254618in}{0.600972in}}%
\pgfpathlineto{\pgfqpoint{4.257318in}{0.594426in}}%
\pgfpathlineto{\pgfqpoint{4.260019in}{0.588258in}}%
\pgfpathlineto{\pgfqpoint{4.262719in}{0.582477in}}%
\pgfpathlineto{\pgfqpoint{4.265420in}{0.577083in}}%
\pgfpathlineto{\pgfqpoint{4.268120in}{0.572077in}}%
\pgfpathlineto{\pgfqpoint{4.270821in}{0.567452in}}%
\pgfpathlineto{\pgfqpoint{4.273521in}{0.563199in}}%
\pgfpathlineto{\pgfqpoint{4.276222in}{0.559307in}}%
\pgfpathlineto{\pgfqpoint{4.278922in}{0.555761in}}%
\pgfpathlineto{\pgfqpoint{4.281623in}{0.552544in}}%
\pgfpathlineto{\pgfqpoint{4.284323in}{0.549641in}}%
\pgfpathlineto{\pgfqpoint{4.287024in}{0.547031in}}%
\pgfpathlineto{\pgfqpoint{4.289724in}{0.544695in}}%
\pgfpathlineto{\pgfqpoint{4.292425in}{0.542614in}}%
\pgfpathlineto{\pgfqpoint{4.295125in}{0.540768in}}%
\pgfpathlineto{\pgfqpoint{4.297826in}{0.539137in}}%
\pgfpathlineto{\pgfqpoint{4.300526in}{0.537702in}}%
\pgfpathlineto{\pgfqpoint{4.303227in}{0.536445in}}%
\pgfpathlineto{\pgfqpoint{4.305927in}{0.535350in}}%
\pgfpathlineto{\pgfqpoint{4.308628in}{0.534398in}}%
\pgfpathlineto{\pgfqpoint{4.311328in}{0.533574in}}%
\pgfpathlineto{\pgfqpoint{4.314029in}{0.532865in}}%
\pgfpathlineto{\pgfqpoint{4.316729in}{0.532257in}}%
\pgfpathlineto{\pgfqpoint{4.319430in}{0.531737in}}%
\pgfusepath{stroke}%
\end{pgfscope}%
\begin{pgfscope}%
\pgfpathrectangle{\pgfqpoint{3.840000in}{0.529102in}}{\pgfqpoint{1.720000in}{1.224232in}} %
\pgfusepath{clip}%
\pgfsetrectcap%
\pgfsetroundjoin%
\pgfsetlinewidth{1.003750pt}%
\definecolor{currentstroke}{rgb}{0.301961,0.607843,0.301961}%
\pgfsetstrokecolor{currentstroke}%
\pgfsetstrokeopacity{0.500000}%
\pgfsetdash{}{0pt}%
\pgfpathmoveto{\pgfqpoint{4.183467in}{0.534253in}}%
\pgfpathlineto{\pgfqpoint{4.186098in}{0.535269in}}%
\pgfpathlineto{\pgfqpoint{4.188730in}{0.536458in}}%
\pgfpathlineto{\pgfqpoint{4.191361in}{0.537844in}}%
\pgfpathlineto{\pgfqpoint{4.193992in}{0.539454in}}%
\pgfpathlineto{\pgfqpoint{4.196623in}{0.541314in}}%
\pgfpathlineto{\pgfqpoint{4.199254in}{0.543456in}}%
\pgfpathlineto{\pgfqpoint{4.201885in}{0.545912in}}%
\pgfpathlineto{\pgfqpoint{4.204517in}{0.548717in}}%
\pgfpathlineto{\pgfqpoint{4.207148in}{0.551905in}}%
\pgfpathlineto{\pgfqpoint{4.209779in}{0.555514in}}%
\pgfpathlineto{\pgfqpoint{4.212410in}{0.559581in}}%
\pgfpathlineto{\pgfqpoint{4.215041in}{0.564147in}}%
\pgfpathlineto{\pgfqpoint{4.217672in}{0.569249in}}%
\pgfpathlineto{\pgfqpoint{4.220303in}{0.574924in}}%
\pgfpathlineto{\pgfqpoint{4.222935in}{0.581211in}}%
\pgfpathlineto{\pgfqpoint{4.225566in}{0.588142in}}%
\pgfpathlineto{\pgfqpoint{4.228197in}{0.595750in}}%
\pgfpathlineto{\pgfqpoint{4.230828in}{0.604063in}}%
\pgfpathlineto{\pgfqpoint{4.233459in}{0.613104in}}%
\pgfpathlineto{\pgfqpoint{4.236090in}{0.622889in}}%
\pgfpathlineto{\pgfqpoint{4.238722in}{0.633431in}}%
\pgfpathlineto{\pgfqpoint{4.241353in}{0.644732in}}%
\pgfpathlineto{\pgfqpoint{4.243984in}{0.656788in}}%
\pgfpathlineto{\pgfqpoint{4.246615in}{0.669583in}}%
\pgfpathlineto{\pgfqpoint{4.249246in}{0.683094in}}%
\pgfpathlineto{\pgfqpoint{4.251877in}{0.697285in}}%
\pgfpathlineto{\pgfqpoint{4.254509in}{0.712111in}}%
\pgfpathlineto{\pgfqpoint{4.257140in}{0.727514in}}%
\pgfpathlineto{\pgfqpoint{4.259771in}{0.743424in}}%
\pgfpathlineto{\pgfqpoint{4.262402in}{0.759761in}}%
\pgfpathlineto{\pgfqpoint{4.265033in}{0.776434in}}%
\pgfpathlineto{\pgfqpoint{4.267664in}{0.793339in}}%
\pgfpathlineto{\pgfqpoint{4.270296in}{0.810365in}}%
\pgfpathlineto{\pgfqpoint{4.272927in}{0.827390in}}%
\pgfpathlineto{\pgfqpoint{4.275558in}{0.844285in}}%
\pgfpathlineto{\pgfqpoint{4.278189in}{0.860917in}}%
\pgfpathlineto{\pgfqpoint{4.280820in}{0.877146in}}%
\pgfpathlineto{\pgfqpoint{4.283451in}{0.892830in}}%
\pgfpathlineto{\pgfqpoint{4.286082in}{0.907827in}}%
\pgfpathlineto{\pgfqpoint{4.288714in}{0.921996in}}%
\pgfpathlineto{\pgfqpoint{4.291345in}{0.935202in}}%
\pgfpathlineto{\pgfqpoint{4.293976in}{0.947312in}}%
\pgfpathlineto{\pgfqpoint{4.296607in}{0.958204in}}%
\pgfpathlineto{\pgfqpoint{4.299238in}{0.967766in}}%
\pgfpathlineto{\pgfqpoint{4.301869in}{0.975897in}}%
\pgfpathlineto{\pgfqpoint{4.304501in}{0.982510in}}%
\pgfpathlineto{\pgfqpoint{4.307132in}{0.987533in}}%
\pgfpathlineto{\pgfqpoint{4.309763in}{0.990914in}}%
\pgfpathlineto{\pgfqpoint{4.312394in}{0.992613in}}%
\pgfpathlineto{\pgfqpoint{4.315025in}{0.992613in}}%
\pgfpathlineto{\pgfqpoint{4.317656in}{0.990914in}}%
\pgfpathlineto{\pgfqpoint{4.320288in}{0.987533in}}%
\pgfpathlineto{\pgfqpoint{4.322919in}{0.982510in}}%
\pgfpathlineto{\pgfqpoint{4.325550in}{0.975897in}}%
\pgfpathlineto{\pgfqpoint{4.328181in}{0.967766in}}%
\pgfpathlineto{\pgfqpoint{4.330812in}{0.958204in}}%
\pgfpathlineto{\pgfqpoint{4.333443in}{0.947312in}}%
\pgfpathlineto{\pgfqpoint{4.336074in}{0.935202in}}%
\pgfpathlineto{\pgfqpoint{4.338706in}{0.921996in}}%
\pgfpathlineto{\pgfqpoint{4.341337in}{0.907827in}}%
\pgfpathlineto{\pgfqpoint{4.343968in}{0.892830in}}%
\pgfpathlineto{\pgfqpoint{4.346599in}{0.877146in}}%
\pgfpathlineto{\pgfqpoint{4.349230in}{0.860917in}}%
\pgfpathlineto{\pgfqpoint{4.351861in}{0.844285in}}%
\pgfpathlineto{\pgfqpoint{4.354493in}{0.827390in}}%
\pgfpathlineto{\pgfqpoint{4.357124in}{0.810365in}}%
\pgfpathlineto{\pgfqpoint{4.359755in}{0.793339in}}%
\pgfpathlineto{\pgfqpoint{4.362386in}{0.776434in}}%
\pgfpathlineto{\pgfqpoint{4.365017in}{0.759761in}}%
\pgfpathlineto{\pgfqpoint{4.367648in}{0.743424in}}%
\pgfpathlineto{\pgfqpoint{4.370280in}{0.727514in}}%
\pgfpathlineto{\pgfqpoint{4.372911in}{0.712111in}}%
\pgfpathlineto{\pgfqpoint{4.375542in}{0.697285in}}%
\pgfpathlineto{\pgfqpoint{4.378173in}{0.683094in}}%
\pgfpathlineto{\pgfqpoint{4.380804in}{0.669583in}}%
\pgfpathlineto{\pgfqpoint{4.383435in}{0.656788in}}%
\pgfpathlineto{\pgfqpoint{4.386067in}{0.644732in}}%
\pgfpathlineto{\pgfqpoint{4.388698in}{0.633431in}}%
\pgfpathlineto{\pgfqpoint{4.391329in}{0.622889in}}%
\pgfpathlineto{\pgfqpoint{4.393960in}{0.613104in}}%
\pgfpathlineto{\pgfqpoint{4.396591in}{0.604063in}}%
\pgfpathlineto{\pgfqpoint{4.399222in}{0.595750in}}%
\pgfpathlineto{\pgfqpoint{4.401853in}{0.588142in}}%
\pgfpathlineto{\pgfqpoint{4.404485in}{0.581211in}}%
\pgfpathlineto{\pgfqpoint{4.407116in}{0.574924in}}%
\pgfpathlineto{\pgfqpoint{4.409747in}{0.569249in}}%
\pgfpathlineto{\pgfqpoint{4.412378in}{0.564147in}}%
\pgfpathlineto{\pgfqpoint{4.415009in}{0.559581in}}%
\pgfpathlineto{\pgfqpoint{4.417640in}{0.555514in}}%
\pgfpathlineto{\pgfqpoint{4.420272in}{0.551905in}}%
\pgfpathlineto{\pgfqpoint{4.422903in}{0.548717in}}%
\pgfpathlineto{\pgfqpoint{4.425534in}{0.545912in}}%
\pgfpathlineto{\pgfqpoint{4.428165in}{0.543456in}}%
\pgfpathlineto{\pgfqpoint{4.430796in}{0.541314in}}%
\pgfpathlineto{\pgfqpoint{4.433427in}{0.539454in}}%
\pgfpathlineto{\pgfqpoint{4.436059in}{0.537844in}}%
\pgfpathlineto{\pgfqpoint{4.438690in}{0.536458in}}%
\pgfpathlineto{\pgfqpoint{4.441321in}{0.535269in}}%
\pgfpathlineto{\pgfqpoint{4.443952in}{0.534253in}}%
\pgfusepath{stroke}%
\end{pgfscope}%
\begin{pgfscope}%
\pgfpathrectangle{\pgfqpoint{3.840000in}{0.529102in}}{\pgfqpoint{1.720000in}{1.224232in}} %
\pgfusepath{clip}%
\pgfsetrectcap%
\pgfsetroundjoin%
\pgfsetlinewidth{1.003750pt}%
\definecolor{currentstroke}{rgb}{0.301961,0.607843,0.301961}%
\pgfsetstrokecolor{currentstroke}%
\pgfsetstrokeopacity{0.500000}%
\pgfsetdash{}{0pt}%
\pgfpathmoveto{\pgfqpoint{4.309142in}{0.537540in}}%
\pgfpathlineto{\pgfqpoint{4.311835in}{0.539203in}}%
\pgfpathlineto{\pgfqpoint{4.314528in}{0.541151in}}%
\pgfpathlineto{\pgfqpoint{4.317220in}{0.543422in}}%
\pgfpathlineto{\pgfqpoint{4.319913in}{0.546058in}}%
\pgfpathlineto{\pgfqpoint{4.322606in}{0.549105in}}%
\pgfpathlineto{\pgfqpoint{4.325299in}{0.552614in}}%
\pgfpathlineto{\pgfqpoint{4.327992in}{0.556637in}}%
\pgfpathlineto{\pgfqpoint{4.330685in}{0.561230in}}%
\pgfpathlineto{\pgfqpoint{4.333378in}{0.566452in}}%
\pgfpathlineto{\pgfqpoint{4.336070in}{0.572363in}}%
\pgfpathlineto{\pgfqpoint{4.338763in}{0.579026in}}%
\pgfpathlineto{\pgfqpoint{4.341456in}{0.586504in}}%
\pgfpathlineto{\pgfqpoint{4.344149in}{0.594861in}}%
\pgfpathlineto{\pgfqpoint{4.346842in}{0.604157in}}%
\pgfpathlineto{\pgfqpoint{4.349535in}{0.614454in}}%
\pgfpathlineto{\pgfqpoint{4.352228in}{0.625807in}}%
\pgfpathlineto{\pgfqpoint{4.354920in}{0.638269in}}%
\pgfpathlineto{\pgfqpoint{4.357613in}{0.651885in}}%
\pgfpathlineto{\pgfqpoint{4.360306in}{0.666693in}}%
\pgfpathlineto{\pgfqpoint{4.362999in}{0.682722in}}%
\pgfpathlineto{\pgfqpoint{4.365692in}{0.699989in}}%
\pgfpathlineto{\pgfqpoint{4.368385in}{0.718499in}}%
\pgfpathlineto{\pgfqpoint{4.371078in}{0.738246in}}%
\pgfpathlineto{\pgfqpoint{4.373770in}{0.759204in}}%
\pgfpathlineto{\pgfqpoint{4.376463in}{0.781334in}}%
\pgfpathlineto{\pgfqpoint{4.379156in}{0.804579in}}%
\pgfpathlineto{\pgfqpoint{4.381849in}{0.828863in}}%
\pgfpathlineto{\pgfqpoint{4.384542in}{0.854092in}}%
\pgfpathlineto{\pgfqpoint{4.387235in}{0.880152in}}%
\pgfpathlineto{\pgfqpoint{4.389928in}{0.906912in}}%
\pgfpathlineto{\pgfqpoint{4.392620in}{0.934220in}}%
\pgfpathlineto{\pgfqpoint{4.395313in}{0.961910in}}%
\pgfpathlineto{\pgfqpoint{4.398006in}{0.989798in}}%
\pgfpathlineto{\pgfqpoint{4.400699in}{1.017684in}}%
\pgfpathlineto{\pgfqpoint{4.403392in}{1.045358in}}%
\pgfpathlineto{\pgfqpoint{4.406085in}{1.072601in}}%
\pgfpathlineto{\pgfqpoint{4.408778in}{1.099182in}}%
\pgfpathlineto{\pgfqpoint{4.411470in}{1.124872in}}%
\pgfpathlineto{\pgfqpoint{4.414163in}{1.149436in}}%
\pgfpathlineto{\pgfqpoint{4.416856in}{1.172645in}}%
\pgfpathlineto{\pgfqpoint{4.419549in}{1.194275in}}%
\pgfpathlineto{\pgfqpoint{4.422242in}{1.214111in}}%
\pgfpathlineto{\pgfqpoint{4.424935in}{1.231952in}}%
\pgfpathlineto{\pgfqpoint{4.427628in}{1.247614in}}%
\pgfpathlineto{\pgfqpoint{4.430320in}{1.260932in}}%
\pgfpathlineto{\pgfqpoint{4.433013in}{1.271763in}}%
\pgfpathlineto{\pgfqpoint{4.435706in}{1.279992in}}%
\pgfpathlineto{\pgfqpoint{4.438399in}{1.285529in}}%
\pgfpathlineto{\pgfqpoint{4.441092in}{1.288312in}}%
\pgfpathlineto{\pgfqpoint{4.443785in}{1.288312in}}%
\pgfpathlineto{\pgfqpoint{4.446478in}{1.285529in}}%
\pgfpathlineto{\pgfqpoint{4.449170in}{1.279992in}}%
\pgfpathlineto{\pgfqpoint{4.451863in}{1.271763in}}%
\pgfpathlineto{\pgfqpoint{4.454556in}{1.260932in}}%
\pgfpathlineto{\pgfqpoint{4.457249in}{1.247614in}}%
\pgfpathlineto{\pgfqpoint{4.459942in}{1.231952in}}%
\pgfpathlineto{\pgfqpoint{4.462635in}{1.214111in}}%
\pgfpathlineto{\pgfqpoint{4.465328in}{1.194275in}}%
\pgfpathlineto{\pgfqpoint{4.468020in}{1.172645in}}%
\pgfpathlineto{\pgfqpoint{4.470713in}{1.149436in}}%
\pgfpathlineto{\pgfqpoint{4.473406in}{1.124872in}}%
\pgfpathlineto{\pgfqpoint{4.476099in}{1.099182in}}%
\pgfpathlineto{\pgfqpoint{4.478792in}{1.072601in}}%
\pgfpathlineto{\pgfqpoint{4.481485in}{1.045358in}}%
\pgfpathlineto{\pgfqpoint{4.484178in}{1.017684in}}%
\pgfpathlineto{\pgfqpoint{4.486871in}{0.989798in}}%
\pgfpathlineto{\pgfqpoint{4.489563in}{0.961910in}}%
\pgfpathlineto{\pgfqpoint{4.492256in}{0.934220in}}%
\pgfpathlineto{\pgfqpoint{4.494949in}{0.906912in}}%
\pgfpathlineto{\pgfqpoint{4.497642in}{0.880152in}}%
\pgfpathlineto{\pgfqpoint{4.500335in}{0.854092in}}%
\pgfpathlineto{\pgfqpoint{4.503028in}{0.828863in}}%
\pgfpathlineto{\pgfqpoint{4.505721in}{0.804579in}}%
\pgfpathlineto{\pgfqpoint{4.508413in}{0.781334in}}%
\pgfpathlineto{\pgfqpoint{4.511106in}{0.759204in}}%
\pgfpathlineto{\pgfqpoint{4.513799in}{0.738246in}}%
\pgfpathlineto{\pgfqpoint{4.516492in}{0.718499in}}%
\pgfpathlineto{\pgfqpoint{4.519185in}{0.699989in}}%
\pgfpathlineto{\pgfqpoint{4.521878in}{0.682722in}}%
\pgfpathlineto{\pgfqpoint{4.524571in}{0.666693in}}%
\pgfpathlineto{\pgfqpoint{4.527263in}{0.651885in}}%
\pgfpathlineto{\pgfqpoint{4.529956in}{0.638269in}}%
\pgfpathlineto{\pgfqpoint{4.532649in}{0.625807in}}%
\pgfpathlineto{\pgfqpoint{4.535342in}{0.614454in}}%
\pgfpathlineto{\pgfqpoint{4.538035in}{0.604157in}}%
\pgfpathlineto{\pgfqpoint{4.540728in}{0.594861in}}%
\pgfpathlineto{\pgfqpoint{4.543421in}{0.586504in}}%
\pgfpathlineto{\pgfqpoint{4.546113in}{0.579026in}}%
\pgfpathlineto{\pgfqpoint{4.548806in}{0.572363in}}%
\pgfpathlineto{\pgfqpoint{4.551499in}{0.566452in}}%
\pgfpathlineto{\pgfqpoint{4.554192in}{0.561230in}}%
\pgfpathlineto{\pgfqpoint{4.556885in}{0.556637in}}%
\pgfpathlineto{\pgfqpoint{4.559578in}{0.552614in}}%
\pgfpathlineto{\pgfqpoint{4.562271in}{0.549105in}}%
\pgfpathlineto{\pgfqpoint{4.564963in}{0.546058in}}%
\pgfpathlineto{\pgfqpoint{4.567656in}{0.543422in}}%
\pgfpathlineto{\pgfqpoint{4.570349in}{0.541151in}}%
\pgfpathlineto{\pgfqpoint{4.573042in}{0.539203in}}%
\pgfpathlineto{\pgfqpoint{4.575735in}{0.537540in}}%
\pgfusepath{stroke}%
\end{pgfscope}%
\begin{pgfscope}%
\pgfpathrectangle{\pgfqpoint{3.840000in}{0.529102in}}{\pgfqpoint{1.720000in}{1.224232in}} %
\pgfusepath{clip}%
\pgfsetrectcap%
\pgfsetroundjoin%
\pgfsetlinewidth{1.003750pt}%
\definecolor{currentstroke}{rgb}{0.301961,0.607843,0.301961}%
\pgfsetstrokecolor{currentstroke}%
\pgfsetstrokeopacity{0.500000}%
\pgfsetdash{}{0pt}%
\pgfpathmoveto{\pgfqpoint{4.440169in}{0.540393in}}%
\pgfpathlineto{\pgfqpoint{4.442817in}{0.542620in}}%
\pgfpathlineto{\pgfqpoint{4.445465in}{0.545226in}}%
\pgfpathlineto{\pgfqpoint{4.448114in}{0.548265in}}%
\pgfpathlineto{\pgfqpoint{4.450762in}{0.551792in}}%
\pgfpathlineto{\pgfqpoint{4.453411in}{0.555870in}}%
\pgfpathlineto{\pgfqpoint{4.456059in}{0.560566in}}%
\pgfpathlineto{\pgfqpoint{4.458707in}{0.565949in}}%
\pgfpathlineto{\pgfqpoint{4.461356in}{0.572096in}}%
\pgfpathlineto{\pgfqpoint{4.464004in}{0.579084in}}%
\pgfpathlineto{\pgfqpoint{4.466653in}{0.586994in}}%
\pgfpathlineto{\pgfqpoint{4.469301in}{0.595911in}}%
\pgfpathlineto{\pgfqpoint{4.471950in}{0.605918in}}%
\pgfpathlineto{\pgfqpoint{4.474598in}{0.617100in}}%
\pgfpathlineto{\pgfqpoint{4.477246in}{0.629541in}}%
\pgfpathlineto{\pgfqpoint{4.479895in}{0.643320in}}%
\pgfpathlineto{\pgfqpoint{4.482543in}{0.658513in}}%
\pgfpathlineto{\pgfqpoint{4.485192in}{0.675189in}}%
\pgfpathlineto{\pgfqpoint{4.487840in}{0.693410in}}%
\pgfpathlineto{\pgfqpoint{4.490488in}{0.713226in}}%
\pgfpathlineto{\pgfqpoint{4.493137in}{0.734675in}}%
\pgfpathlineto{\pgfqpoint{4.495785in}{0.757782in}}%
\pgfpathlineto{\pgfqpoint{4.498434in}{0.782553in}}%
\pgfpathlineto{\pgfqpoint{4.501082in}{0.808977in}}%
\pgfpathlineto{\pgfqpoint{4.503730in}{0.837023in}}%
\pgfpathlineto{\pgfqpoint{4.506379in}{0.866638in}}%
\pgfpathlineto{\pgfqpoint{4.509027in}{0.897744in}}%
\pgfpathlineto{\pgfqpoint{4.511676in}{0.930241in}}%
\pgfpathlineto{\pgfqpoint{4.514324in}{0.964002in}}%
\pgfpathlineto{\pgfqpoint{4.516973in}{0.998875in}}%
\pgfpathlineto{\pgfqpoint{4.519621in}{1.034685in}}%
\pgfpathlineto{\pgfqpoint{4.522269in}{1.071230in}}%
\pgfpathlineto{\pgfqpoint{4.524918in}{1.108284in}}%
\pgfpathlineto{\pgfqpoint{4.527566in}{1.145603in}}%
\pgfpathlineto{\pgfqpoint{4.530215in}{1.182920in}}%
\pgfpathlineto{\pgfqpoint{4.532863in}{1.219954in}}%
\pgfpathlineto{\pgfqpoint{4.535511in}{1.256409in}}%
\pgfpathlineto{\pgfqpoint{4.538160in}{1.291981in}}%
\pgfpathlineto{\pgfqpoint{4.540808in}{1.326359in}}%
\pgfpathlineto{\pgfqpoint{4.543457in}{1.359231in}}%
\pgfpathlineto{\pgfqpoint{4.546105in}{1.390289in}}%
\pgfpathlineto{\pgfqpoint{4.548754in}{1.419234in}}%
\pgfpathlineto{\pgfqpoint{4.551402in}{1.445778in}}%
\pgfpathlineto{\pgfqpoint{4.554050in}{1.469653in}}%
\pgfpathlineto{\pgfqpoint{4.556699in}{1.490612in}}%
\pgfpathlineto{\pgfqpoint{4.559347in}{1.508433in}}%
\pgfpathlineto{\pgfqpoint{4.561996in}{1.522928in}}%
\pgfpathlineto{\pgfqpoint{4.564644in}{1.533940in}}%
\pgfpathlineto{\pgfqpoint{4.567292in}{1.541349in}}%
\pgfpathlineto{\pgfqpoint{4.569941in}{1.545074in}}%
\pgfpathlineto{\pgfqpoint{4.572589in}{1.545074in}}%
\pgfpathlineto{\pgfqpoint{4.575238in}{1.541349in}}%
\pgfpathlineto{\pgfqpoint{4.577886in}{1.533940in}}%
\pgfpathlineto{\pgfqpoint{4.580535in}{1.522928in}}%
\pgfpathlineto{\pgfqpoint{4.583183in}{1.508433in}}%
\pgfpathlineto{\pgfqpoint{4.585831in}{1.490612in}}%
\pgfpathlineto{\pgfqpoint{4.588480in}{1.469653in}}%
\pgfpathlineto{\pgfqpoint{4.591128in}{1.445778in}}%
\pgfpathlineto{\pgfqpoint{4.593777in}{1.419234in}}%
\pgfpathlineto{\pgfqpoint{4.596425in}{1.390289in}}%
\pgfpathlineto{\pgfqpoint{4.599073in}{1.359231in}}%
\pgfpathlineto{\pgfqpoint{4.601722in}{1.326359in}}%
\pgfpathlineto{\pgfqpoint{4.604370in}{1.291981in}}%
\pgfpathlineto{\pgfqpoint{4.607019in}{1.256409in}}%
\pgfpathlineto{\pgfqpoint{4.609667in}{1.219954in}}%
\pgfpathlineto{\pgfqpoint{4.612316in}{1.182920in}}%
\pgfpathlineto{\pgfqpoint{4.614964in}{1.145603in}}%
\pgfpathlineto{\pgfqpoint{4.617612in}{1.108284in}}%
\pgfpathlineto{\pgfqpoint{4.620261in}{1.071230in}}%
\pgfpathlineto{\pgfqpoint{4.622909in}{1.034685in}}%
\pgfpathlineto{\pgfqpoint{4.625558in}{0.998875in}}%
\pgfpathlineto{\pgfqpoint{4.628206in}{0.964002in}}%
\pgfpathlineto{\pgfqpoint{4.630854in}{0.930241in}}%
\pgfpathlineto{\pgfqpoint{4.633503in}{0.897744in}}%
\pgfpathlineto{\pgfqpoint{4.636151in}{0.866638in}}%
\pgfpathlineto{\pgfqpoint{4.638800in}{0.837023in}}%
\pgfpathlineto{\pgfqpoint{4.641448in}{0.808977in}}%
\pgfpathlineto{\pgfqpoint{4.644097in}{0.782553in}}%
\pgfpathlineto{\pgfqpoint{4.646745in}{0.757782in}}%
\pgfpathlineto{\pgfqpoint{4.649393in}{0.734675in}}%
\pgfpathlineto{\pgfqpoint{4.652042in}{0.713226in}}%
\pgfpathlineto{\pgfqpoint{4.654690in}{0.693410in}}%
\pgfpathlineto{\pgfqpoint{4.657339in}{0.675189in}}%
\pgfpathlineto{\pgfqpoint{4.659987in}{0.658513in}}%
\pgfpathlineto{\pgfqpoint{4.662635in}{0.643320in}}%
\pgfpathlineto{\pgfqpoint{4.665284in}{0.629541in}}%
\pgfpathlineto{\pgfqpoint{4.667932in}{0.617100in}}%
\pgfpathlineto{\pgfqpoint{4.670581in}{0.605918in}}%
\pgfpathlineto{\pgfqpoint{4.673229in}{0.595911in}}%
\pgfpathlineto{\pgfqpoint{4.675878in}{0.586994in}}%
\pgfpathlineto{\pgfqpoint{4.678526in}{0.579084in}}%
\pgfpathlineto{\pgfqpoint{4.681174in}{0.572096in}}%
\pgfpathlineto{\pgfqpoint{4.683823in}{0.565949in}}%
\pgfpathlineto{\pgfqpoint{4.686471in}{0.560566in}}%
\pgfpathlineto{\pgfqpoint{4.689120in}{0.555870in}}%
\pgfpathlineto{\pgfqpoint{4.691768in}{0.551792in}}%
\pgfpathlineto{\pgfqpoint{4.694416in}{0.548265in}}%
\pgfpathlineto{\pgfqpoint{4.697065in}{0.545226in}}%
\pgfpathlineto{\pgfqpoint{4.699713in}{0.542620in}}%
\pgfpathlineto{\pgfqpoint{4.702362in}{0.540393in}}%
\pgfusepath{stroke}%
\end{pgfscope}%
\begin{pgfscope}%
\pgfpathrectangle{\pgfqpoint{3.840000in}{0.529102in}}{\pgfqpoint{1.720000in}{1.224232in}} %
\pgfusepath{clip}%
\pgfsetrectcap%
\pgfsetroundjoin%
\pgfsetlinewidth{1.003750pt}%
\definecolor{currentstroke}{rgb}{0.301961,0.607843,0.301961}%
\pgfsetstrokecolor{currentstroke}%
\pgfsetstrokeopacity{0.500000}%
\pgfsetdash{}{0pt}%
\pgfpathmoveto{\pgfqpoint{4.567106in}{0.541548in}}%
\pgfpathlineto{\pgfqpoint{4.569788in}{0.544003in}}%
\pgfpathlineto{\pgfqpoint{4.572469in}{0.546876in}}%
\pgfpathlineto{\pgfqpoint{4.575151in}{0.550225in}}%
\pgfpathlineto{\pgfqpoint{4.577833in}{0.554113in}}%
\pgfpathlineto{\pgfqpoint{4.580515in}{0.558609in}}%
\pgfpathlineto{\pgfqpoint{4.583197in}{0.563785in}}%
\pgfpathlineto{\pgfqpoint{4.585879in}{0.569719in}}%
\pgfpathlineto{\pgfqpoint{4.588561in}{0.576494in}}%
\pgfpathlineto{\pgfqpoint{4.591243in}{0.584197in}}%
\pgfpathlineto{\pgfqpoint{4.593925in}{0.592916in}}%
\pgfpathlineto{\pgfqpoint{4.596607in}{0.602745in}}%
\pgfpathlineto{\pgfqpoint{4.599288in}{0.613776in}}%
\pgfpathlineto{\pgfqpoint{4.601970in}{0.626102in}}%
\pgfpathlineto{\pgfqpoint{4.604652in}{0.639815in}}%
\pgfpathlineto{\pgfqpoint{4.607334in}{0.655004in}}%
\pgfpathlineto{\pgfqpoint{4.610016in}{0.671751in}}%
\pgfpathlineto{\pgfqpoint{4.612698in}{0.690133in}}%
\pgfpathlineto{\pgfqpoint{4.615380in}{0.710218in}}%
\pgfpathlineto{\pgfqpoint{4.618062in}{0.732061in}}%
\pgfpathlineto{\pgfqpoint{4.620744in}{0.755705in}}%
\pgfpathlineto{\pgfqpoint{4.623426in}{0.781175in}}%
\pgfpathlineto{\pgfqpoint{4.626107in}{0.808480in}}%
\pgfpathlineto{\pgfqpoint{4.628789in}{0.837607in}}%
\pgfpathlineto{\pgfqpoint{4.631471in}{0.868522in}}%
\pgfpathlineto{\pgfqpoint{4.634153in}{0.901166in}}%
\pgfpathlineto{\pgfqpoint{4.636835in}{0.935455in}}%
\pgfpathlineto{\pgfqpoint{4.639517in}{0.971276in}}%
\pgfpathlineto{\pgfqpoint{4.642199in}{1.008490in}}%
\pgfpathlineto{\pgfqpoint{4.644881in}{1.046931in}}%
\pgfpathlineto{\pgfqpoint{4.647563in}{1.086404in}}%
\pgfpathlineto{\pgfqpoint{4.650245in}{1.126687in}}%
\pgfpathlineto{\pgfqpoint{4.652927in}{1.167532in}}%
\pgfpathlineto{\pgfqpoint{4.655608in}{1.208669in}}%
\pgfpathlineto{\pgfqpoint{4.658290in}{1.249803in}}%
\pgfpathlineto{\pgfqpoint{4.660972in}{1.290626in}}%
\pgfpathlineto{\pgfqpoint{4.663654in}{1.330810in}}%
\pgfpathlineto{\pgfqpoint{4.666336in}{1.370020in}}%
\pgfpathlineto{\pgfqpoint{4.669018in}{1.407915in}}%
\pgfpathlineto{\pgfqpoint{4.671700in}{1.444149in}}%
\pgfpathlineto{\pgfqpoint{4.674382in}{1.478385in}}%
\pgfpathlineto{\pgfqpoint{4.677064in}{1.510291in}}%
\pgfpathlineto{\pgfqpoint{4.679746in}{1.539550in}}%
\pgfpathlineto{\pgfqpoint{4.682427in}{1.565868in}}%
\pgfpathlineto{\pgfqpoint{4.685109in}{1.588970in}}%
\pgfpathlineto{\pgfqpoint{4.687791in}{1.608615in}}%
\pgfpathlineto{\pgfqpoint{4.690473in}{1.624593in}}%
\pgfpathlineto{\pgfqpoint{4.693155in}{1.636731in}}%
\pgfpathlineto{\pgfqpoint{4.695837in}{1.644898in}}%
\pgfpathlineto{\pgfqpoint{4.698519in}{1.649004in}}%
\pgfpathlineto{\pgfqpoint{4.701201in}{1.649004in}}%
\pgfpathlineto{\pgfqpoint{4.703883in}{1.644898in}}%
\pgfpathlineto{\pgfqpoint{4.706565in}{1.636731in}}%
\pgfpathlineto{\pgfqpoint{4.709246in}{1.624593in}}%
\pgfpathlineto{\pgfqpoint{4.711928in}{1.608615in}}%
\pgfpathlineto{\pgfqpoint{4.714610in}{1.588970in}}%
\pgfpathlineto{\pgfqpoint{4.717292in}{1.565868in}}%
\pgfpathlineto{\pgfqpoint{4.719974in}{1.539550in}}%
\pgfpathlineto{\pgfqpoint{4.722656in}{1.510291in}}%
\pgfpathlineto{\pgfqpoint{4.725338in}{1.478385in}}%
\pgfpathlineto{\pgfqpoint{4.728020in}{1.444149in}}%
\pgfpathlineto{\pgfqpoint{4.730702in}{1.407915in}}%
\pgfpathlineto{\pgfqpoint{4.733384in}{1.370020in}}%
\pgfpathlineto{\pgfqpoint{4.736066in}{1.330810in}}%
\pgfpathlineto{\pgfqpoint{4.738747in}{1.290626in}}%
\pgfpathlineto{\pgfqpoint{4.741429in}{1.249803in}}%
\pgfpathlineto{\pgfqpoint{4.744111in}{1.208669in}}%
\pgfpathlineto{\pgfqpoint{4.746793in}{1.167532in}}%
\pgfpathlineto{\pgfqpoint{4.749475in}{1.126687in}}%
\pgfpathlineto{\pgfqpoint{4.752157in}{1.086404in}}%
\pgfpathlineto{\pgfqpoint{4.754839in}{1.046931in}}%
\pgfpathlineto{\pgfqpoint{4.757521in}{1.008490in}}%
\pgfpathlineto{\pgfqpoint{4.760203in}{0.971276in}}%
\pgfpathlineto{\pgfqpoint{4.762885in}{0.935455in}}%
\pgfpathlineto{\pgfqpoint{4.765566in}{0.901166in}}%
\pgfpathlineto{\pgfqpoint{4.768248in}{0.868522in}}%
\pgfpathlineto{\pgfqpoint{4.770930in}{0.837607in}}%
\pgfpathlineto{\pgfqpoint{4.773612in}{0.808480in}}%
\pgfpathlineto{\pgfqpoint{4.776294in}{0.781175in}}%
\pgfpathlineto{\pgfqpoint{4.778976in}{0.755705in}}%
\pgfpathlineto{\pgfqpoint{4.781658in}{0.732061in}}%
\pgfpathlineto{\pgfqpoint{4.784340in}{0.710218in}}%
\pgfpathlineto{\pgfqpoint{4.787022in}{0.690133in}}%
\pgfpathlineto{\pgfqpoint{4.789704in}{0.671751in}}%
\pgfpathlineto{\pgfqpoint{4.792386in}{0.655004in}}%
\pgfpathlineto{\pgfqpoint{4.795067in}{0.639815in}}%
\pgfpathlineto{\pgfqpoint{4.797749in}{0.626102in}}%
\pgfpathlineto{\pgfqpoint{4.800431in}{0.613776in}}%
\pgfpathlineto{\pgfqpoint{4.803113in}{0.602745in}}%
\pgfpathlineto{\pgfqpoint{4.805795in}{0.592916in}}%
\pgfpathlineto{\pgfqpoint{4.808477in}{0.584197in}}%
\pgfpathlineto{\pgfqpoint{4.811159in}{0.576494in}}%
\pgfpathlineto{\pgfqpoint{4.813841in}{0.569719in}}%
\pgfpathlineto{\pgfqpoint{4.816523in}{0.563785in}}%
\pgfpathlineto{\pgfqpoint{4.819205in}{0.558609in}}%
\pgfpathlineto{\pgfqpoint{4.821886in}{0.554113in}}%
\pgfpathlineto{\pgfqpoint{4.824568in}{0.550225in}}%
\pgfpathlineto{\pgfqpoint{4.827250in}{0.546876in}}%
\pgfpathlineto{\pgfqpoint{4.829932in}{0.544003in}}%
\pgfpathlineto{\pgfqpoint{4.832614in}{0.541548in}}%
\pgfusepath{stroke}%
\end{pgfscope}%
\begin{pgfscope}%
\pgfpathrectangle{\pgfqpoint{3.840000in}{0.529102in}}{\pgfqpoint{1.720000in}{1.224232in}} %
\pgfusepath{clip}%
\pgfsetrectcap%
\pgfsetroundjoin%
\pgfsetlinewidth{1.003750pt}%
\definecolor{currentstroke}{rgb}{0.301961,0.607843,0.301961}%
\pgfsetstrokecolor{currentstroke}%
\pgfsetstrokeopacity{0.500000}%
\pgfsetdash{}{0pt}%
\pgfpathmoveto{\pgfqpoint{4.697283in}{0.540393in}}%
\pgfpathlineto{\pgfqpoint{4.699934in}{0.542619in}}%
\pgfpathlineto{\pgfqpoint{4.702585in}{0.545225in}}%
\pgfpathlineto{\pgfqpoint{4.705237in}{0.548264in}}%
\pgfpathlineto{\pgfqpoint{4.707888in}{0.551791in}}%
\pgfpathlineto{\pgfqpoint{4.710539in}{0.555869in}}%
\pgfpathlineto{\pgfqpoint{4.713191in}{0.560564in}}%
\pgfpathlineto{\pgfqpoint{4.715842in}{0.565948in}}%
\pgfpathlineto{\pgfqpoint{4.718494in}{0.572094in}}%
\pgfpathlineto{\pgfqpoint{4.721145in}{0.579081in}}%
\pgfpathlineto{\pgfqpoint{4.723796in}{0.586991in}}%
\pgfpathlineto{\pgfqpoint{4.726448in}{0.595907in}}%
\pgfpathlineto{\pgfqpoint{4.729099in}{0.605914in}}%
\pgfpathlineto{\pgfqpoint{4.731750in}{0.617096in}}%
\pgfpathlineto{\pgfqpoint{4.734402in}{0.629536in}}%
\pgfpathlineto{\pgfqpoint{4.737053in}{0.643314in}}%
\pgfpathlineto{\pgfqpoint{4.739705in}{0.658506in}}%
\pgfpathlineto{\pgfqpoint{4.742356in}{0.675182in}}%
\pgfpathlineto{\pgfqpoint{4.745007in}{0.693402in}}%
\pgfpathlineto{\pgfqpoint{4.747659in}{0.713217in}}%
\pgfpathlineto{\pgfqpoint{4.750310in}{0.734665in}}%
\pgfpathlineto{\pgfqpoint{4.752962in}{0.757770in}}%
\pgfpathlineto{\pgfqpoint{4.755613in}{0.782540in}}%
\pgfpathlineto{\pgfqpoint{4.758264in}{0.808963in}}%
\pgfpathlineto{\pgfqpoint{4.760916in}{0.837008in}}%
\pgfpathlineto{\pgfqpoint{4.763567in}{0.866621in}}%
\pgfpathlineto{\pgfqpoint{4.766218in}{0.897725in}}%
\pgfpathlineto{\pgfqpoint{4.768870in}{0.930220in}}%
\pgfpathlineto{\pgfqpoint{4.771521in}{0.963980in}}%
\pgfpathlineto{\pgfqpoint{4.774173in}{0.998852in}}%
\pgfpathlineto{\pgfqpoint{4.776824in}{1.034659in}}%
\pgfpathlineto{\pgfqpoint{4.779475in}{1.071202in}}%
\pgfpathlineto{\pgfqpoint{4.782127in}{1.108255in}}%
\pgfpathlineto{\pgfqpoint{4.784778in}{1.145571in}}%
\pgfpathlineto{\pgfqpoint{4.787429in}{1.182887in}}%
\pgfpathlineto{\pgfqpoint{4.790081in}{1.219919in}}%
\pgfpathlineto{\pgfqpoint{4.792732in}{1.256372in}}%
\pgfpathlineto{\pgfqpoint{4.795384in}{1.291942in}}%
\pgfpathlineto{\pgfqpoint{4.798035in}{1.326318in}}%
\pgfpathlineto{\pgfqpoint{4.800686in}{1.359188in}}%
\pgfpathlineto{\pgfqpoint{4.803338in}{1.390245in}}%
\pgfpathlineto{\pgfqpoint{4.805989in}{1.419188in}}%
\pgfpathlineto{\pgfqpoint{4.808641in}{1.445731in}}%
\pgfpathlineto{\pgfqpoint{4.811292in}{1.469605in}}%
\pgfpathlineto{\pgfqpoint{4.813943in}{1.490563in}}%
\pgfpathlineto{\pgfqpoint{4.816595in}{1.508383in}}%
\pgfpathlineto{\pgfqpoint{4.819246in}{1.522878in}}%
\pgfpathlineto{\pgfqpoint{4.821897in}{1.533889in}}%
\pgfpathlineto{\pgfqpoint{4.824549in}{1.541297in}}%
\pgfpathlineto{\pgfqpoint{4.827200in}{1.545022in}}%
\pgfpathlineto{\pgfqpoint{4.829852in}{1.545022in}}%
\pgfpathlineto{\pgfqpoint{4.832503in}{1.541297in}}%
\pgfpathlineto{\pgfqpoint{4.835154in}{1.533889in}}%
\pgfpathlineto{\pgfqpoint{4.837806in}{1.522878in}}%
\pgfpathlineto{\pgfqpoint{4.840457in}{1.508383in}}%
\pgfpathlineto{\pgfqpoint{4.843108in}{1.490563in}}%
\pgfpathlineto{\pgfqpoint{4.845760in}{1.469605in}}%
\pgfpathlineto{\pgfqpoint{4.848411in}{1.445731in}}%
\pgfpathlineto{\pgfqpoint{4.851063in}{1.419188in}}%
\pgfpathlineto{\pgfqpoint{4.853714in}{1.390245in}}%
\pgfpathlineto{\pgfqpoint{4.856365in}{1.359188in}}%
\pgfpathlineto{\pgfqpoint{4.859017in}{1.326318in}}%
\pgfpathlineto{\pgfqpoint{4.861668in}{1.291942in}}%
\pgfpathlineto{\pgfqpoint{4.864319in}{1.256372in}}%
\pgfpathlineto{\pgfqpoint{4.866971in}{1.219919in}}%
\pgfpathlineto{\pgfqpoint{4.869622in}{1.182887in}}%
\pgfpathlineto{\pgfqpoint{4.872274in}{1.145571in}}%
\pgfpathlineto{\pgfqpoint{4.874925in}{1.108255in}}%
\pgfpathlineto{\pgfqpoint{4.877576in}{1.071202in}}%
\pgfpathlineto{\pgfqpoint{4.880228in}{1.034659in}}%
\pgfpathlineto{\pgfqpoint{4.882879in}{0.998852in}}%
\pgfpathlineto{\pgfqpoint{4.885531in}{0.963980in}}%
\pgfpathlineto{\pgfqpoint{4.888182in}{0.930220in}}%
\pgfpathlineto{\pgfqpoint{4.890833in}{0.897725in}}%
\pgfpathlineto{\pgfqpoint{4.893485in}{0.866621in}}%
\pgfpathlineto{\pgfqpoint{4.896136in}{0.837008in}}%
\pgfpathlineto{\pgfqpoint{4.898787in}{0.808963in}}%
\pgfpathlineto{\pgfqpoint{4.901439in}{0.782540in}}%
\pgfpathlineto{\pgfqpoint{4.904090in}{0.757770in}}%
\pgfpathlineto{\pgfqpoint{4.906742in}{0.734665in}}%
\pgfpathlineto{\pgfqpoint{4.909393in}{0.713217in}}%
\pgfpathlineto{\pgfqpoint{4.912044in}{0.693402in}}%
\pgfpathlineto{\pgfqpoint{4.914696in}{0.675182in}}%
\pgfpathlineto{\pgfqpoint{4.917347in}{0.658506in}}%
\pgfpathlineto{\pgfqpoint{4.919998in}{0.643314in}}%
\pgfpathlineto{\pgfqpoint{4.922650in}{0.629536in}}%
\pgfpathlineto{\pgfqpoint{4.925301in}{0.617096in}}%
\pgfpathlineto{\pgfqpoint{4.927953in}{0.605914in}}%
\pgfpathlineto{\pgfqpoint{4.930604in}{0.595907in}}%
\pgfpathlineto{\pgfqpoint{4.933255in}{0.586991in}}%
\pgfpathlineto{\pgfqpoint{4.935907in}{0.579081in}}%
\pgfpathlineto{\pgfqpoint{4.938558in}{0.572094in}}%
\pgfpathlineto{\pgfqpoint{4.941210in}{0.565948in}}%
\pgfpathlineto{\pgfqpoint{4.943861in}{0.560564in}}%
\pgfpathlineto{\pgfqpoint{4.946512in}{0.555869in}}%
\pgfpathlineto{\pgfqpoint{4.949164in}{0.551791in}}%
\pgfpathlineto{\pgfqpoint{4.951815in}{0.548264in}}%
\pgfpathlineto{\pgfqpoint{4.954466in}{0.545225in}}%
\pgfpathlineto{\pgfqpoint{4.957118in}{0.542619in}}%
\pgfpathlineto{\pgfqpoint{4.959769in}{0.540393in}}%
\pgfusepath{stroke}%
\end{pgfscope}%
\begin{pgfscope}%
\pgfpathrectangle{\pgfqpoint{3.840000in}{0.529102in}}{\pgfqpoint{1.720000in}{1.224232in}} %
\pgfusepath{clip}%
\pgfsetrectcap%
\pgfsetroundjoin%
\pgfsetlinewidth{1.003750pt}%
\definecolor{currentstroke}{rgb}{0.301961,0.607843,0.301961}%
\pgfsetstrokecolor{currentstroke}%
\pgfsetstrokeopacity{0.500000}%
\pgfsetdash{}{0pt}%
\pgfpathmoveto{\pgfqpoint{4.823650in}{0.537535in}}%
\pgfpathlineto{\pgfqpoint{4.826358in}{0.539198in}}%
\pgfpathlineto{\pgfqpoint{4.829065in}{0.541144in}}%
\pgfpathlineto{\pgfqpoint{4.831772in}{0.543414in}}%
\pgfpathlineto{\pgfqpoint{4.834479in}{0.546048in}}%
\pgfpathlineto{\pgfqpoint{4.837187in}{0.549094in}}%
\pgfpathlineto{\pgfqpoint{4.839894in}{0.552601in}}%
\pgfpathlineto{\pgfqpoint{4.842601in}{0.556621in}}%
\pgfpathlineto{\pgfqpoint{4.845308in}{0.561212in}}%
\pgfpathlineto{\pgfqpoint{4.848016in}{0.566431in}}%
\pgfpathlineto{\pgfqpoint{4.850723in}{0.572339in}}%
\pgfpathlineto{\pgfqpoint{4.853430in}{0.578998in}}%
\pgfpathlineto{\pgfqpoint{4.856137in}{0.586472in}}%
\pgfpathlineto{\pgfqpoint{4.858845in}{0.594823in}}%
\pgfpathlineto{\pgfqpoint{4.861552in}{0.604115in}}%
\pgfpathlineto{\pgfqpoint{4.864259in}{0.614405in}}%
\pgfpathlineto{\pgfqpoint{4.866966in}{0.625752in}}%
\pgfpathlineto{\pgfqpoint{4.869674in}{0.638207in}}%
\pgfpathlineto{\pgfqpoint{4.872381in}{0.651816in}}%
\pgfpathlineto{\pgfqpoint{4.875088in}{0.666615in}}%
\pgfpathlineto{\pgfqpoint{4.877795in}{0.682634in}}%
\pgfpathlineto{\pgfqpoint{4.880503in}{0.699892in}}%
\pgfpathlineto{\pgfqpoint{4.883210in}{0.718392in}}%
\pgfpathlineto{\pgfqpoint{4.885917in}{0.738127in}}%
\pgfpathlineto{\pgfqpoint{4.888625in}{0.759073in}}%
\pgfpathlineto{\pgfqpoint{4.891332in}{0.781191in}}%
\pgfpathlineto{\pgfqpoint{4.894039in}{0.804422in}}%
\pgfpathlineto{\pgfqpoint{4.896746in}{0.828693in}}%
\pgfpathlineto{\pgfqpoint{4.899454in}{0.853907in}}%
\pgfpathlineto{\pgfqpoint{4.902161in}{0.879952in}}%
\pgfpathlineto{\pgfqpoint{4.904868in}{0.906697in}}%
\pgfpathlineto{\pgfqpoint{4.907575in}{0.933990in}}%
\pgfpathlineto{\pgfqpoint{4.910283in}{0.961664in}}%
\pgfpathlineto{\pgfqpoint{4.912990in}{0.989536in}}%
\pgfpathlineto{\pgfqpoint{4.915697in}{1.017406in}}%
\pgfpathlineto{\pgfqpoint{4.918404in}{1.045065in}}%
\pgfpathlineto{\pgfqpoint{4.921112in}{1.072292in}}%
\pgfpathlineto{\pgfqpoint{4.923819in}{1.098859in}}%
\pgfpathlineto{\pgfqpoint{4.926526in}{1.124533in}}%
\pgfpathlineto{\pgfqpoint{4.929233in}{1.149084in}}%
\pgfpathlineto{\pgfqpoint{4.931941in}{1.172280in}}%
\pgfpathlineto{\pgfqpoint{4.934648in}{1.193897in}}%
\pgfpathlineto{\pgfqpoint{4.937355in}{1.213722in}}%
\pgfpathlineto{\pgfqpoint{4.940062in}{1.231553in}}%
\pgfpathlineto{\pgfqpoint{4.942770in}{1.247206in}}%
\pgfpathlineto{\pgfqpoint{4.945477in}{1.260516in}}%
\pgfpathlineto{\pgfqpoint{4.948184in}{1.271342in}}%
\pgfpathlineto{\pgfqpoint{4.950891in}{1.279566in}}%
\pgfpathlineto{\pgfqpoint{4.953599in}{1.285099in}}%
\pgfpathlineto{\pgfqpoint{4.956306in}{1.287881in}}%
\pgfpathlineto{\pgfqpoint{4.959013in}{1.287881in}}%
\pgfpathlineto{\pgfqpoint{4.961720in}{1.285099in}}%
\pgfpathlineto{\pgfqpoint{4.964428in}{1.279566in}}%
\pgfpathlineto{\pgfqpoint{4.967135in}{1.271342in}}%
\pgfpathlineto{\pgfqpoint{4.969842in}{1.260516in}}%
\pgfpathlineto{\pgfqpoint{4.972549in}{1.247206in}}%
\pgfpathlineto{\pgfqpoint{4.975257in}{1.231553in}}%
\pgfpathlineto{\pgfqpoint{4.977964in}{1.213722in}}%
\pgfpathlineto{\pgfqpoint{4.980671in}{1.193897in}}%
\pgfpathlineto{\pgfqpoint{4.983378in}{1.172280in}}%
\pgfpathlineto{\pgfqpoint{4.986086in}{1.149084in}}%
\pgfpathlineto{\pgfqpoint{4.988793in}{1.124533in}}%
\pgfpathlineto{\pgfqpoint{4.991500in}{1.098859in}}%
\pgfpathlineto{\pgfqpoint{4.994207in}{1.072292in}}%
\pgfpathlineto{\pgfqpoint{4.996915in}{1.045065in}}%
\pgfpathlineto{\pgfqpoint{4.999622in}{1.017406in}}%
\pgfpathlineto{\pgfqpoint{5.002329in}{0.989536in}}%
\pgfpathlineto{\pgfqpoint{5.005036in}{0.961664in}}%
\pgfpathlineto{\pgfqpoint{5.007744in}{0.933990in}}%
\pgfpathlineto{\pgfqpoint{5.010451in}{0.906697in}}%
\pgfpathlineto{\pgfqpoint{5.013158in}{0.879952in}}%
\pgfpathlineto{\pgfqpoint{5.015865in}{0.853907in}}%
\pgfpathlineto{\pgfqpoint{5.018573in}{0.828693in}}%
\pgfpathlineto{\pgfqpoint{5.021280in}{0.804422in}}%
\pgfpathlineto{\pgfqpoint{5.023987in}{0.781191in}}%
\pgfpathlineto{\pgfqpoint{5.026694in}{0.759073in}}%
\pgfpathlineto{\pgfqpoint{5.029402in}{0.738127in}}%
\pgfpathlineto{\pgfqpoint{5.032109in}{0.718392in}}%
\pgfpathlineto{\pgfqpoint{5.034816in}{0.699892in}}%
\pgfpathlineto{\pgfqpoint{5.037523in}{0.682634in}}%
\pgfpathlineto{\pgfqpoint{5.040231in}{0.666615in}}%
\pgfpathlineto{\pgfqpoint{5.042938in}{0.651816in}}%
\pgfpathlineto{\pgfqpoint{5.045645in}{0.638207in}}%
\pgfpathlineto{\pgfqpoint{5.048352in}{0.625752in}}%
\pgfpathlineto{\pgfqpoint{5.051060in}{0.614405in}}%
\pgfpathlineto{\pgfqpoint{5.053767in}{0.604115in}}%
\pgfpathlineto{\pgfqpoint{5.056474in}{0.594823in}}%
\pgfpathlineto{\pgfqpoint{5.059181in}{0.586472in}}%
\pgfpathlineto{\pgfqpoint{5.061889in}{0.578998in}}%
\pgfpathlineto{\pgfqpoint{5.064596in}{0.572339in}}%
\pgfpathlineto{\pgfqpoint{5.067303in}{0.566431in}}%
\pgfpathlineto{\pgfqpoint{5.070011in}{0.561212in}}%
\pgfpathlineto{\pgfqpoint{5.072718in}{0.556621in}}%
\pgfpathlineto{\pgfqpoint{5.075425in}{0.552601in}}%
\pgfpathlineto{\pgfqpoint{5.078132in}{0.549094in}}%
\pgfpathlineto{\pgfqpoint{5.080840in}{0.546048in}}%
\pgfpathlineto{\pgfqpoint{5.083547in}{0.543414in}}%
\pgfpathlineto{\pgfqpoint{5.086254in}{0.541144in}}%
\pgfpathlineto{\pgfqpoint{5.088961in}{0.539198in}}%
\pgfpathlineto{\pgfqpoint{5.091669in}{0.537535in}}%
\pgfusepath{stroke}%
\end{pgfscope}%
\begin{pgfscope}%
\pgfpathrectangle{\pgfqpoint{3.840000in}{0.529102in}}{\pgfqpoint{1.720000in}{1.224232in}} %
\pgfusepath{clip}%
\pgfsetrectcap%
\pgfsetroundjoin%
\pgfsetlinewidth{1.003750pt}%
\definecolor{currentstroke}{rgb}{0.301961,0.607843,0.301961}%
\pgfsetstrokecolor{currentstroke}%
\pgfsetstrokeopacity{0.500000}%
\pgfsetdash{}{0pt}%
\pgfpathmoveto{\pgfqpoint{4.957114in}{0.534264in}}%
\pgfpathlineto{\pgfqpoint{4.959736in}{0.535282in}}%
\pgfpathlineto{\pgfqpoint{4.962357in}{0.536473in}}%
\pgfpathlineto{\pgfqpoint{4.964979in}{0.537862in}}%
\pgfpathlineto{\pgfqpoint{4.967601in}{0.539475in}}%
\pgfpathlineto{\pgfqpoint{4.970222in}{0.541340in}}%
\pgfpathlineto{\pgfqpoint{4.972844in}{0.543486in}}%
\pgfpathlineto{\pgfqpoint{4.975465in}{0.545947in}}%
\pgfpathlineto{\pgfqpoint{4.978087in}{0.548757in}}%
\pgfpathlineto{\pgfqpoint{4.980708in}{0.551952in}}%
\pgfpathlineto{\pgfqpoint{4.983330in}{0.555569in}}%
\pgfpathlineto{\pgfqpoint{4.985951in}{0.559645in}}%
\pgfpathlineto{\pgfqpoint{4.988573in}{0.564220in}}%
\pgfpathlineto{\pgfqpoint{4.991195in}{0.569332in}}%
\pgfpathlineto{\pgfqpoint{4.993816in}{0.575020in}}%
\pgfpathlineto{\pgfqpoint{4.996438in}{0.581319in}}%
\pgfpathlineto{\pgfqpoint{4.999059in}{0.588265in}}%
\pgfpathlineto{\pgfqpoint{5.001681in}{0.595889in}}%
\pgfpathlineto{\pgfqpoint{5.004302in}{0.604219in}}%
\pgfpathlineto{\pgfqpoint{5.006924in}{0.613279in}}%
\pgfpathlineto{\pgfqpoint{5.009545in}{0.623085in}}%
\pgfpathlineto{\pgfqpoint{5.012167in}{0.633649in}}%
\pgfpathlineto{\pgfqpoint{5.014789in}{0.644973in}}%
\pgfpathlineto{\pgfqpoint{5.017410in}{0.657054in}}%
\pgfpathlineto{\pgfqpoint{5.020032in}{0.669876in}}%
\pgfpathlineto{\pgfqpoint{5.022653in}{0.683415in}}%
\pgfpathlineto{\pgfqpoint{5.025275in}{0.697636in}}%
\pgfpathlineto{\pgfqpoint{5.027896in}{0.712492in}}%
\pgfpathlineto{\pgfqpoint{5.030518in}{0.727927in}}%
\pgfpathlineto{\pgfqpoint{5.033139in}{0.743871in}}%
\pgfpathlineto{\pgfqpoint{5.035761in}{0.760242in}}%
\pgfpathlineto{\pgfqpoint{5.038383in}{0.776949in}}%
\pgfpathlineto{\pgfqpoint{5.041004in}{0.793890in}}%
\pgfpathlineto{\pgfqpoint{5.043626in}{0.810951in}}%
\pgfpathlineto{\pgfqpoint{5.046247in}{0.828011in}}%
\pgfpathlineto{\pgfqpoint{5.048869in}{0.844942in}}%
\pgfpathlineto{\pgfqpoint{5.051490in}{0.861609in}}%
\pgfpathlineto{\pgfqpoint{5.054112in}{0.877871in}}%
\pgfpathlineto{\pgfqpoint{5.056733in}{0.893588in}}%
\pgfpathlineto{\pgfqpoint{5.059355in}{0.908616in}}%
\pgfpathlineto{\pgfqpoint{5.061977in}{0.922815in}}%
\pgfpathlineto{\pgfqpoint{5.064598in}{0.936048in}}%
\pgfpathlineto{\pgfqpoint{5.067220in}{0.948183in}}%
\pgfpathlineto{\pgfqpoint{5.069841in}{0.959098in}}%
\pgfpathlineto{\pgfqpoint{5.072463in}{0.968680in}}%
\pgfpathlineto{\pgfqpoint{5.075084in}{0.976828in}}%
\pgfpathlineto{\pgfqpoint{5.077706in}{0.983454in}}%
\pgfpathlineto{\pgfqpoint{5.080327in}{0.988489in}}%
\pgfpathlineto{\pgfqpoint{5.082949in}{0.991876in}}%
\pgfpathlineto{\pgfqpoint{5.085571in}{0.993579in}}%
\pgfpathlineto{\pgfqpoint{5.088192in}{0.993579in}}%
\pgfpathlineto{\pgfqpoint{5.090814in}{0.991876in}}%
\pgfpathlineto{\pgfqpoint{5.093435in}{0.988489in}}%
\pgfpathlineto{\pgfqpoint{5.096057in}{0.983454in}}%
\pgfpathlineto{\pgfqpoint{5.098678in}{0.976828in}}%
\pgfpathlineto{\pgfqpoint{5.101300in}{0.968680in}}%
\pgfpathlineto{\pgfqpoint{5.103921in}{0.959098in}}%
\pgfpathlineto{\pgfqpoint{5.106543in}{0.948183in}}%
\pgfpathlineto{\pgfqpoint{5.109165in}{0.936048in}}%
\pgfpathlineto{\pgfqpoint{5.111786in}{0.922815in}}%
\pgfpathlineto{\pgfqpoint{5.114408in}{0.908616in}}%
\pgfpathlineto{\pgfqpoint{5.117029in}{0.893588in}}%
\pgfpathlineto{\pgfqpoint{5.119651in}{0.877871in}}%
\pgfpathlineto{\pgfqpoint{5.122272in}{0.861609in}}%
\pgfpathlineto{\pgfqpoint{5.124894in}{0.844942in}}%
\pgfpathlineto{\pgfqpoint{5.127515in}{0.828011in}}%
\pgfpathlineto{\pgfqpoint{5.130137in}{0.810951in}}%
\pgfpathlineto{\pgfqpoint{5.132759in}{0.793890in}}%
\pgfpathlineto{\pgfqpoint{5.135380in}{0.776949in}}%
\pgfpathlineto{\pgfqpoint{5.138002in}{0.760242in}}%
\pgfpathlineto{\pgfqpoint{5.140623in}{0.743871in}}%
\pgfpathlineto{\pgfqpoint{5.143245in}{0.727927in}}%
\pgfpathlineto{\pgfqpoint{5.145866in}{0.712492in}}%
\pgfpathlineto{\pgfqpoint{5.148488in}{0.697636in}}%
\pgfpathlineto{\pgfqpoint{5.151109in}{0.683415in}}%
\pgfpathlineto{\pgfqpoint{5.153731in}{0.669876in}}%
\pgfpathlineto{\pgfqpoint{5.156353in}{0.657054in}}%
\pgfpathlineto{\pgfqpoint{5.158974in}{0.644973in}}%
\pgfpathlineto{\pgfqpoint{5.161596in}{0.633649in}}%
\pgfpathlineto{\pgfqpoint{5.164217in}{0.623085in}}%
\pgfpathlineto{\pgfqpoint{5.166839in}{0.613279in}}%
\pgfpathlineto{\pgfqpoint{5.169460in}{0.604219in}}%
\pgfpathlineto{\pgfqpoint{5.172082in}{0.595889in}}%
\pgfpathlineto{\pgfqpoint{5.174703in}{0.588265in}}%
\pgfpathlineto{\pgfqpoint{5.177325in}{0.581319in}}%
\pgfpathlineto{\pgfqpoint{5.179947in}{0.575020in}}%
\pgfpathlineto{\pgfqpoint{5.182568in}{0.569332in}}%
\pgfpathlineto{\pgfqpoint{5.185190in}{0.564220in}}%
\pgfpathlineto{\pgfqpoint{5.187811in}{0.559645in}}%
\pgfpathlineto{\pgfqpoint{5.190433in}{0.555569in}}%
\pgfpathlineto{\pgfqpoint{5.193054in}{0.551952in}}%
\pgfpathlineto{\pgfqpoint{5.195676in}{0.548757in}}%
\pgfpathlineto{\pgfqpoint{5.198297in}{0.545947in}}%
\pgfpathlineto{\pgfqpoint{5.200919in}{0.543486in}}%
\pgfpathlineto{\pgfqpoint{5.203541in}{0.541340in}}%
\pgfpathlineto{\pgfqpoint{5.206162in}{0.539475in}}%
\pgfpathlineto{\pgfqpoint{5.208784in}{0.537862in}}%
\pgfpathlineto{\pgfqpoint{5.211405in}{0.536473in}}%
\pgfpathlineto{\pgfqpoint{5.214027in}{0.535282in}}%
\pgfpathlineto{\pgfqpoint{5.216648in}{0.534264in}}%
\pgfusepath{stroke}%
\end{pgfscope}%
\begin{pgfscope}%
\pgfpathrectangle{\pgfqpoint{3.840000in}{0.529102in}}{\pgfqpoint{1.720000in}{1.224232in}} %
\pgfusepath{clip}%
\pgfsetrectcap%
\pgfsetroundjoin%
\pgfsetlinewidth{1.003750pt}%
\definecolor{currentstroke}{rgb}{0.301961,0.607843,0.301961}%
\pgfsetstrokecolor{currentstroke}%
\pgfsetstrokeopacity{0.500000}%
\pgfsetdash{}{0pt}%
\pgfpathmoveto{\pgfqpoint{5.081885in}{0.531745in}}%
\pgfpathlineto{\pgfqpoint{5.084565in}{0.532267in}}%
\pgfpathlineto{\pgfqpoint{5.087245in}{0.532877in}}%
\pgfpathlineto{\pgfqpoint{5.089925in}{0.533588in}}%
\pgfpathlineto{\pgfqpoint{5.092605in}{0.534414in}}%
\pgfpathlineto{\pgfqpoint{5.095284in}{0.535369in}}%
\pgfpathlineto{\pgfqpoint{5.097964in}{0.536469in}}%
\pgfpathlineto{\pgfqpoint{5.100644in}{0.537729in}}%
\pgfpathlineto{\pgfqpoint{5.103324in}{0.539168in}}%
\pgfpathlineto{\pgfqpoint{5.106004in}{0.540804in}}%
\pgfpathlineto{\pgfqpoint{5.108683in}{0.542656in}}%
\pgfpathlineto{\pgfqpoint{5.111363in}{0.544744in}}%
\pgfpathlineto{\pgfqpoint{5.114043in}{0.547087in}}%
\pgfpathlineto{\pgfqpoint{5.116723in}{0.549705in}}%
\pgfpathlineto{\pgfqpoint{5.119403in}{0.552618in}}%
\pgfpathlineto{\pgfqpoint{5.122082in}{0.555844in}}%
\pgfpathlineto{\pgfqpoint{5.124762in}{0.559401in}}%
\pgfpathlineto{\pgfqpoint{5.127442in}{0.563306in}}%
\pgfpathlineto{\pgfqpoint{5.130122in}{0.567572in}}%
\pgfpathlineto{\pgfqpoint{5.132802in}{0.572212in}}%
\pgfpathlineto{\pgfqpoint{5.135481in}{0.577234in}}%
\pgfpathlineto{\pgfqpoint{5.138161in}{0.582644in}}%
\pgfpathlineto{\pgfqpoint{5.140841in}{0.588444in}}%
\pgfpathlineto{\pgfqpoint{5.143521in}{0.594631in}}%
\pgfpathlineto{\pgfqpoint{5.146200in}{0.601197in}}%
\pgfpathlineto{\pgfqpoint{5.148880in}{0.608131in}}%
\pgfpathlineto{\pgfqpoint{5.151560in}{0.615414in}}%
\pgfpathlineto{\pgfqpoint{5.154240in}{0.623023in}}%
\pgfpathlineto{\pgfqpoint{5.156920in}{0.630928in}}%
\pgfpathlineto{\pgfqpoint{5.159599in}{0.639093in}}%
\pgfpathlineto{\pgfqpoint{5.162279in}{0.647477in}}%
\pgfpathlineto{\pgfqpoint{5.164959in}{0.656034in}}%
\pgfpathlineto{\pgfqpoint{5.167639in}{0.664709in}}%
\pgfpathlineto{\pgfqpoint{5.170319in}{0.673447in}}%
\pgfpathlineto{\pgfqpoint{5.172998in}{0.682184in}}%
\pgfpathlineto{\pgfqpoint{5.175678in}{0.690855in}}%
\pgfpathlineto{\pgfqpoint{5.178358in}{0.699391in}}%
\pgfpathlineto{\pgfqpoint{5.181038in}{0.707720in}}%
\pgfpathlineto{\pgfqpoint{5.183718in}{0.715769in}}%
\pgfpathlineto{\pgfqpoint{5.186397in}{0.723465in}}%
\pgfpathlineto{\pgfqpoint{5.189077in}{0.730737in}}%
\pgfpathlineto{\pgfqpoint{5.191757in}{0.737514in}}%
\pgfpathlineto{\pgfqpoint{5.194437in}{0.743729in}}%
\pgfpathlineto{\pgfqpoint{5.197117in}{0.749319in}}%
\pgfpathlineto{\pgfqpoint{5.199796in}{0.754226in}}%
\pgfpathlineto{\pgfqpoint{5.202476in}{0.758399in}}%
\pgfpathlineto{\pgfqpoint{5.205156in}{0.761793in}}%
\pgfpathlineto{\pgfqpoint{5.207836in}{0.764371in}}%
\pgfpathlineto{\pgfqpoint{5.210516in}{0.766106in}}%
\pgfpathlineto{\pgfqpoint{5.213195in}{0.766978in}}%
\pgfpathlineto{\pgfqpoint{5.215875in}{0.766978in}}%
\pgfpathlineto{\pgfqpoint{5.218555in}{0.766106in}}%
\pgfpathlineto{\pgfqpoint{5.221235in}{0.764371in}}%
\pgfpathlineto{\pgfqpoint{5.223915in}{0.761793in}}%
\pgfpathlineto{\pgfqpoint{5.226594in}{0.758399in}}%
\pgfpathlineto{\pgfqpoint{5.229274in}{0.754226in}}%
\pgfpathlineto{\pgfqpoint{5.231954in}{0.749319in}}%
\pgfpathlineto{\pgfqpoint{5.234634in}{0.743729in}}%
\pgfpathlineto{\pgfqpoint{5.237313in}{0.737514in}}%
\pgfpathlineto{\pgfqpoint{5.239993in}{0.730737in}}%
\pgfpathlineto{\pgfqpoint{5.242673in}{0.723465in}}%
\pgfpathlineto{\pgfqpoint{5.245353in}{0.715769in}}%
\pgfpathlineto{\pgfqpoint{5.248033in}{0.707720in}}%
\pgfpathlineto{\pgfqpoint{5.250712in}{0.699391in}}%
\pgfpathlineto{\pgfqpoint{5.253392in}{0.690855in}}%
\pgfpathlineto{\pgfqpoint{5.256072in}{0.682184in}}%
\pgfpathlineto{\pgfqpoint{5.258752in}{0.673447in}}%
\pgfpathlineto{\pgfqpoint{5.261432in}{0.664709in}}%
\pgfpathlineto{\pgfqpoint{5.264111in}{0.656034in}}%
\pgfpathlineto{\pgfqpoint{5.266791in}{0.647477in}}%
\pgfpathlineto{\pgfqpoint{5.269471in}{0.639093in}}%
\pgfpathlineto{\pgfqpoint{5.272151in}{0.630928in}}%
\pgfpathlineto{\pgfqpoint{5.274831in}{0.623023in}}%
\pgfpathlineto{\pgfqpoint{5.277510in}{0.615414in}}%
\pgfpathlineto{\pgfqpoint{5.280190in}{0.608131in}}%
\pgfpathlineto{\pgfqpoint{5.282870in}{0.601197in}}%
\pgfpathlineto{\pgfqpoint{5.285550in}{0.594631in}}%
\pgfpathlineto{\pgfqpoint{5.288230in}{0.588444in}}%
\pgfpathlineto{\pgfqpoint{5.290909in}{0.582644in}}%
\pgfpathlineto{\pgfqpoint{5.293589in}{0.577234in}}%
\pgfpathlineto{\pgfqpoint{5.296269in}{0.572212in}}%
\pgfpathlineto{\pgfqpoint{5.298949in}{0.567572in}}%
\pgfpathlineto{\pgfqpoint{5.301629in}{0.563306in}}%
\pgfpathlineto{\pgfqpoint{5.304308in}{0.559401in}}%
\pgfpathlineto{\pgfqpoint{5.306988in}{0.555844in}}%
\pgfpathlineto{\pgfqpoint{5.309668in}{0.552618in}}%
\pgfpathlineto{\pgfqpoint{5.312348in}{0.549705in}}%
\pgfpathlineto{\pgfqpoint{5.315027in}{0.547087in}}%
\pgfpathlineto{\pgfqpoint{5.317707in}{0.544744in}}%
\pgfpathlineto{\pgfqpoint{5.320387in}{0.542656in}}%
\pgfpathlineto{\pgfqpoint{5.323067in}{0.540804in}}%
\pgfpathlineto{\pgfqpoint{5.325747in}{0.539168in}}%
\pgfpathlineto{\pgfqpoint{5.328426in}{0.537729in}}%
\pgfpathlineto{\pgfqpoint{5.331106in}{0.536469in}}%
\pgfpathlineto{\pgfqpoint{5.333786in}{0.535369in}}%
\pgfpathlineto{\pgfqpoint{5.336466in}{0.534414in}}%
\pgfpathlineto{\pgfqpoint{5.339146in}{0.533588in}}%
\pgfpathlineto{\pgfqpoint{5.341825in}{0.532877in}}%
\pgfpathlineto{\pgfqpoint{5.344505in}{0.532267in}}%
\pgfpathlineto{\pgfqpoint{5.347185in}{0.531745in}}%
\pgfusepath{stroke}%
\end{pgfscope}%
\begin{pgfscope}%
\pgfpathrectangle{\pgfqpoint{3.840000in}{0.529102in}}{\pgfqpoint{1.720000in}{1.224232in}} %
\pgfusepath{clip}%
\pgfsetrectcap%
\pgfsetroundjoin%
\pgfsetlinewidth{1.003750pt}%
\definecolor{currentstroke}{rgb}{0.301961,0.607843,0.301961}%
\pgfsetstrokecolor{currentstroke}%
\pgfsetdash{}{0pt}%
\pgfpathmoveto{\pgfqpoint{3.840430in}{0.529102in}}%
\pgfpathlineto{\pgfqpoint{4.039190in}{0.530163in}}%
\pgfpathlineto{\pgfqpoint{4.056398in}{0.532609in}}%
\pgfpathlineto{\pgfqpoint{4.068444in}{0.536515in}}%
\pgfpathlineto{\pgfqpoint{4.077909in}{0.541781in}}%
\pgfpathlineto{\pgfqpoint{4.086513in}{0.548963in}}%
\pgfpathlineto{\pgfqpoint{4.095118in}{0.559072in}}%
\pgfpathlineto{\pgfqpoint{4.103722in}{0.572674in}}%
\pgfpathlineto{\pgfqpoint{4.113187in}{0.592094in}}%
\pgfpathlineto{\pgfqpoint{4.123512in}{0.618557in}}%
\pgfpathlineto{\pgfqpoint{4.137279in}{0.660492in}}%
\pgfpathlineto{\pgfqpoint{4.160510in}{0.732081in}}%
\pgfpathlineto{\pgfqpoint{4.169975in}{0.753854in}}%
\pgfpathlineto{\pgfqpoint{4.176858in}{0.764886in}}%
\pgfpathlineto{\pgfqpoint{4.182881in}{0.770797in}}%
\pgfpathlineto{\pgfqpoint{4.188044in}{0.773059in}}%
\pgfpathlineto{\pgfqpoint{4.193207in}{0.772896in}}%
\pgfpathlineto{\pgfqpoint{4.199230in}{0.770074in}}%
\pgfpathlineto{\pgfqpoint{4.206973in}{0.763494in}}%
\pgfpathlineto{\pgfqpoint{4.223322in}{0.748475in}}%
\pgfpathlineto{\pgfqpoint{4.229345in}{0.746318in}}%
\pgfpathlineto{\pgfqpoint{4.233647in}{0.746938in}}%
\pgfpathlineto{\pgfqpoint{4.237949in}{0.749742in}}%
\pgfpathlineto{\pgfqpoint{4.243112in}{0.756347in}}%
\pgfpathlineto{\pgfqpoint{4.248274in}{0.766730in}}%
\pgfpathlineto{\pgfqpoint{4.255158in}{0.786497in}}%
\pgfpathlineto{\pgfqpoint{4.262901in}{0.816118in}}%
\pgfpathlineto{\pgfqpoint{4.274087in}{0.868697in}}%
\pgfpathlineto{\pgfqpoint{4.293877in}{0.962207in}}%
\pgfpathlineto{\pgfqpoint{4.301621in}{0.988269in}}%
\pgfpathlineto{\pgfqpoint{4.307644in}{1.001571in}}%
\pgfpathlineto{\pgfqpoint{4.311946in}{1.006921in}}%
\pgfpathlineto{\pgfqpoint{4.315388in}{1.008663in}}%
\pgfpathlineto{\pgfqpoint{4.318829in}{1.008205in}}%
\pgfpathlineto{\pgfqpoint{4.323132in}{1.004730in}}%
\pgfpathlineto{\pgfqpoint{4.328294in}{0.996828in}}%
\pgfpathlineto{\pgfqpoint{4.335178in}{0.981539in}}%
\pgfpathlineto{\pgfqpoint{4.354967in}{0.933898in}}%
\pgfpathlineto{\pgfqpoint{4.360130in}{0.927626in}}%
\pgfpathlineto{\pgfqpoint{4.363572in}{0.926169in}}%
\pgfpathlineto{\pgfqpoint{4.367014in}{0.927226in}}%
\pgfpathlineto{\pgfqpoint{4.370455in}{0.931011in}}%
\pgfpathlineto{\pgfqpoint{4.374757in}{0.939792in}}%
\pgfpathlineto{\pgfqpoint{4.379920in}{0.956396in}}%
\pgfpathlineto{\pgfqpoint{4.385943in}{0.983871in}}%
\pgfpathlineto{\pgfqpoint{4.393687in}{1.030401in}}%
\pgfpathlineto{\pgfqpoint{4.405733in}{1.118901in}}%
\pgfpathlineto{\pgfqpoint{4.422081in}{1.236542in}}%
\pgfpathlineto{\pgfqpoint{4.429825in}{1.277030in}}%
\pgfpathlineto{\pgfqpoint{4.435848in}{1.297567in}}%
\pgfpathlineto{\pgfqpoint{4.440150in}{1.305605in}}%
\pgfpathlineto{\pgfqpoint{4.443592in}{1.307926in}}%
\pgfpathlineto{\pgfqpoint{4.446173in}{1.307287in}}%
\pgfpathlineto{\pgfqpoint{4.449615in}{1.303370in}}%
\pgfpathlineto{\pgfqpoint{4.453917in}{1.293870in}}%
\pgfpathlineto{\pgfqpoint{4.459080in}{1.276582in}}%
\pgfpathlineto{\pgfqpoint{4.466823in}{1.241829in}}%
\pgfpathlineto{\pgfqpoint{4.486613in}{1.148315in}}%
\pgfpathlineto{\pgfqpoint{4.492636in}{1.131847in}}%
\pgfpathlineto{\pgfqpoint{4.496078in}{1.127050in}}%
\pgfpathlineto{\pgfqpoint{4.498659in}{1.125945in}}%
\pgfpathlineto{\pgfqpoint{4.501241in}{1.127104in}}%
\pgfpathlineto{\pgfqpoint{4.504682in}{1.132307in}}%
\pgfpathlineto{\pgfqpoint{4.508984in}{1.144794in}}%
\pgfpathlineto{\pgfqpoint{4.514147in}{1.168410in}}%
\pgfpathlineto{\pgfqpoint{4.520170in}{1.206934in}}%
\pgfpathlineto{\pgfqpoint{4.528774in}{1.278281in}}%
\pgfpathlineto{\pgfqpoint{4.556308in}{1.521992in}}%
\pgfpathlineto{\pgfqpoint{4.562331in}{1.553402in}}%
\pgfpathlineto{\pgfqpoint{4.566633in}{1.567177in}}%
\pgfpathlineto{\pgfqpoint{4.570075in}{1.572617in}}%
\pgfpathlineto{\pgfqpoint{4.572656in}{1.573379in}}%
\pgfpathlineto{\pgfqpoint{4.575238in}{1.571314in}}%
\pgfpathlineto{\pgfqpoint{4.578679in}{1.564278in}}%
\pgfpathlineto{\pgfqpoint{4.582981in}{1.549000in}}%
\pgfpathlineto{\pgfqpoint{4.589005in}{1.517042in}}%
\pgfpathlineto{\pgfqpoint{4.597609in}{1.455978in}}%
\pgfpathlineto{\pgfqpoint{4.617399in}{1.309017in}}%
\pgfpathlineto{\pgfqpoint{4.623422in}{1.279933in}}%
\pgfpathlineto{\pgfqpoint{4.627724in}{1.267144in}}%
\pgfpathlineto{\pgfqpoint{4.631166in}{1.262293in}}%
\pgfpathlineto{\pgfqpoint{4.633747in}{1.261951in}}%
\pgfpathlineto{\pgfqpoint{4.636328in}{1.264484in}}%
\pgfpathlineto{\pgfqpoint{4.639770in}{1.272321in}}%
\pgfpathlineto{\pgfqpoint{4.644072in}{1.289075in}}%
\pgfpathlineto{\pgfqpoint{4.649235in}{1.318593in}}%
\pgfpathlineto{\pgfqpoint{4.656118in}{1.371223in}}%
\pgfpathlineto{\pgfqpoint{4.668164in}{1.484215in}}%
\pgfpathlineto{\pgfqpoint{4.681071in}{1.599658in}}%
\pgfpathlineto{\pgfqpoint{4.687954in}{1.644268in}}%
\pgfpathlineto{\pgfqpoint{4.693117in}{1.665861in}}%
\pgfpathlineto{\pgfqpoint{4.697419in}{1.674964in}}%
\pgfpathlineto{\pgfqpoint{4.700000in}{1.676350in}}%
\pgfpathlineto{\pgfqpoint{4.702581in}{1.674643in}}%
\pgfpathlineto{\pgfqpoint{4.705163in}{1.669871in}}%
\pgfpathlineto{\pgfqpoint{4.708604in}{1.658870in}}%
\pgfpathlineto{\pgfqpoint{4.713767in}{1.633091in}}%
\pgfpathlineto{\pgfqpoint{4.720650in}{1.583995in}}%
\pgfpathlineto{\pgfqpoint{4.730975in}{1.489996in}}%
\pgfpathlineto{\pgfqpoint{4.747324in}{1.341257in}}%
\pgfpathlineto{\pgfqpoint{4.754207in}{1.296425in}}%
\pgfpathlineto{\pgfqpoint{4.759370in}{1.274187in}}%
\pgfpathlineto{\pgfqpoint{4.763672in}{1.264069in}}%
\pgfpathlineto{\pgfqpoint{4.766253in}{1.261815in}}%
\pgfpathlineto{\pgfqpoint{4.768834in}{1.262430in}}%
\pgfpathlineto{\pgfqpoint{4.771416in}{1.265870in}}%
\pgfpathlineto{\pgfqpoint{4.774857in}{1.274686in}}%
\pgfpathlineto{\pgfqpoint{4.779160in}{1.292017in}}%
\pgfpathlineto{\pgfqpoint{4.785183in}{1.326335in}}%
\pgfpathlineto{\pgfqpoint{4.794647in}{1.396140in}}%
\pgfpathlineto{\pgfqpoint{4.810995in}{1.518374in}}%
\pgfpathlineto{\pgfqpoint{4.817879in}{1.553482in}}%
\pgfpathlineto{\pgfqpoint{4.823042in}{1.568895in}}%
\pgfpathlineto{\pgfqpoint{4.826483in}{1.573242in}}%
\pgfpathlineto{\pgfqpoint{4.829065in}{1.573236in}}%
\pgfpathlineto{\pgfqpoint{4.831646in}{1.570393in}}%
\pgfpathlineto{\pgfqpoint{4.835088in}{1.562222in}}%
\pgfpathlineto{\pgfqpoint{4.839390in}{1.545208in}}%
\pgfpathlineto{\pgfqpoint{4.845413in}{1.509832in}}%
\pgfpathlineto{\pgfqpoint{4.853157in}{1.448479in}}%
\pgfpathlineto{\pgfqpoint{4.869505in}{1.293554in}}%
\pgfpathlineto{\pgfqpoint{4.879830in}{1.206778in}}%
\pgfpathlineto{\pgfqpoint{4.886713in}{1.163993in}}%
\pgfpathlineto{\pgfqpoint{4.892736in}{1.139364in}}%
\pgfpathlineto{\pgfqpoint{4.897039in}{1.129637in}}%
\pgfpathlineto{\pgfqpoint{4.900480in}{1.126577in}}%
\pgfpathlineto{\pgfqpoint{4.903062in}{1.126947in}}%
\pgfpathlineto{\pgfqpoint{4.905643in}{1.129490in}}%
\pgfpathlineto{\pgfqpoint{4.909085in}{1.136024in}}%
\pgfpathlineto{\pgfqpoint{4.914247in}{1.151716in}}%
\pgfpathlineto{\pgfqpoint{4.921131in}{1.180947in}}%
\pgfpathlineto{\pgfqpoint{4.944362in}{1.287895in}}%
\pgfpathlineto{\pgfqpoint{4.949525in}{1.300927in}}%
\pgfpathlineto{\pgfqpoint{4.953827in}{1.306354in}}%
\pgfpathlineto{\pgfqpoint{4.956408in}{1.307023in}}%
\pgfpathlineto{\pgfqpoint{4.958989in}{1.305676in}}%
\pgfpathlineto{\pgfqpoint{4.962431in}{1.300708in}}%
\pgfpathlineto{\pgfqpoint{4.966733in}{1.289457in}}%
\pgfpathlineto{\pgfqpoint{4.971896in}{1.268943in}}%
\pgfpathlineto{\pgfqpoint{4.978779in}{1.231202in}}%
\pgfpathlineto{\pgfqpoint{4.988244in}{1.165421in}}%
\pgfpathlineto{\pgfqpoint{5.011476in}{0.998356in}}%
\pgfpathlineto{\pgfqpoint{5.019220in}{0.959618in}}%
\pgfpathlineto{\pgfqpoint{5.025243in}{0.939214in}}%
\pgfpathlineto{\pgfqpoint{5.030405in}{0.928885in}}%
\pgfpathlineto{\pgfqpoint{5.034707in}{0.925202in}}%
\pgfpathlineto{\pgfqpoint{5.038149in}{0.925266in}}%
\pgfpathlineto{\pgfqpoint{5.041591in}{0.927750in}}%
\pgfpathlineto{\pgfqpoint{5.045893in}{0.933799in}}%
\pgfpathlineto{\pgfqpoint{5.051916in}{0.946495in}}%
\pgfpathlineto{\pgfqpoint{5.076868in}{1.004946in}}%
\pgfpathlineto{\pgfqpoint{5.081171in}{1.008882in}}%
\pgfpathlineto{\pgfqpoint{5.084612in}{1.009706in}}%
\pgfpathlineto{\pgfqpoint{5.088054in}{1.008315in}}%
\pgfpathlineto{\pgfqpoint{5.092356in}{1.003367in}}%
\pgfpathlineto{\pgfqpoint{5.097519in}{0.992769in}}%
\pgfpathlineto{\pgfqpoint{5.103542in}{0.974430in}}%
\pgfpathlineto{\pgfqpoint{5.111286in}{0.943049in}}%
\pgfpathlineto{\pgfqpoint{5.125053in}{0.875307in}}%
\pgfpathlineto{\pgfqpoint{5.139680in}{0.806687in}}%
\pgfpathlineto{\pgfqpoint{5.148284in}{0.776309in}}%
\pgfpathlineto{\pgfqpoint{5.155168in}{0.759428in}}%
\pgfpathlineto{\pgfqpoint{5.161191in}{0.750318in}}%
\pgfpathlineto{\pgfqpoint{5.166353in}{0.746486in}}%
\pgfpathlineto{\pgfqpoint{5.170655in}{0.745787in}}%
\pgfpathlineto{\pgfqpoint{5.175818in}{0.747439in}}%
\pgfpathlineto{\pgfqpoint{5.181841in}{0.751916in}}%
\pgfpathlineto{\pgfqpoint{5.205933in}{0.773152in}}%
\pgfpathlineto{\pgfqpoint{5.211096in}{0.773893in}}%
\pgfpathlineto{\pgfqpoint{5.216258in}{0.772210in}}%
\pgfpathlineto{\pgfqpoint{5.221421in}{0.767904in}}%
\pgfpathlineto{\pgfqpoint{5.227444in}{0.759525in}}%
\pgfpathlineto{\pgfqpoint{5.234327in}{0.745795in}}%
\pgfpathlineto{\pgfqpoint{5.242931in}{0.723434in}}%
\pgfpathlineto{\pgfqpoint{5.256698in}{0.680355in}}%
\pgfpathlineto{\pgfqpoint{5.278209in}{0.613509in}}%
\pgfpathlineto{\pgfqpoint{5.289395in}{0.585882in}}%
\pgfpathlineto{\pgfqpoint{5.298859in}{0.567724in}}%
\pgfpathlineto{\pgfqpoint{5.308324in}{0.554196in}}%
\pgfpathlineto{\pgfqpoint{5.317789in}{0.544677in}}%
\pgfpathlineto{\pgfqpoint{5.327254in}{0.538336in}}%
\pgfpathlineto{\pgfqpoint{5.337579in}{0.534056in}}%
\pgfpathlineto{\pgfqpoint{5.350485in}{0.531209in}}%
\pgfpathlineto{\pgfqpoint{5.369415in}{0.529617in}}%
\pgfpathlineto{\pgfqpoint{5.409855in}{0.529115in}}%
\pgfpathlineto{\pgfqpoint{5.559570in}{0.529102in}}%
\pgfpathlineto{\pgfqpoint{5.559570in}{0.529102in}}%
\pgfusepath{stroke}%
\end{pgfscope}%
\begin{pgfscope}%
\pgfpathrectangle{\pgfqpoint{3.840000in}{0.529102in}}{\pgfqpoint{1.720000in}{1.224232in}} %
\pgfusepath{clip}%
\pgfsetbuttcap%
\pgfsetroundjoin%
\pgfsetlinewidth{1.003750pt}%
\definecolor{currentstroke}{rgb}{1.000000,0.400000,0.200000}%
\pgfsetstrokecolor{currentstroke}%
\pgfsetdash{{1.000000pt}{3.000000pt}}{0.000000pt}%
\pgfpathmoveto{\pgfqpoint{3.840430in}{0.529102in}}%
\pgfpathlineto{\pgfqpoint{4.040050in}{0.530168in}}%
\pgfpathlineto{\pgfqpoint{4.061561in}{0.533303in}}%
\pgfpathlineto{\pgfqpoint{4.069305in}{0.536247in}}%
\pgfpathlineto{\pgfqpoint{4.083932in}{0.545082in}}%
\pgfpathlineto{\pgfqpoint{4.090815in}{0.552284in}}%
\pgfpathlineto{\pgfqpoint{4.098559in}{0.563857in}}%
\pgfpathlineto{\pgfqpoint{4.108884in}{0.583628in}}%
\pgfpathlineto{\pgfqpoint{4.120070in}{0.610129in}}%
\pgfpathlineto{\pgfqpoint{4.129535in}{0.635333in}}%
\pgfpathlineto{\pgfqpoint{4.161371in}{0.735971in}}%
\pgfpathlineto{\pgfqpoint{4.167394in}{0.747433in}}%
\pgfpathlineto{\pgfqpoint{4.182021in}{0.771269in}}%
\pgfpathlineto{\pgfqpoint{4.186323in}{0.771727in}}%
\pgfpathlineto{\pgfqpoint{4.194927in}{0.770369in}}%
\pgfpathlineto{\pgfqpoint{4.212996in}{0.759696in}}%
\pgfpathlineto{\pgfqpoint{4.224182in}{0.749365in}}%
\pgfpathlineto{\pgfqpoint{4.238809in}{0.749621in}}%
\pgfpathlineto{\pgfqpoint{4.242251in}{0.753707in}}%
\pgfpathlineto{\pgfqpoint{4.247414in}{0.764251in}}%
\pgfpathlineto{\pgfqpoint{4.256878in}{0.789774in}}%
\pgfpathlineto{\pgfqpoint{4.262901in}{0.812894in}}%
\pgfpathlineto{\pgfqpoint{4.270645in}{0.853347in}}%
\pgfpathlineto{\pgfqpoint{4.279250in}{0.895130in}}%
\pgfpathlineto{\pgfqpoint{4.299040in}{0.980250in}}%
\pgfpathlineto{\pgfqpoint{4.307644in}{1.003524in}}%
\pgfpathlineto{\pgfqpoint{4.311086in}{1.006578in}}%
\pgfpathlineto{\pgfqpoint{4.314527in}{1.005864in}}%
\pgfpathlineto{\pgfqpoint{4.319690in}{1.004443in}}%
\pgfpathlineto{\pgfqpoint{4.323992in}{1.003753in}}%
\pgfpathlineto{\pgfqpoint{4.326573in}{1.001183in}}%
\pgfpathlineto{\pgfqpoint{4.331736in}{0.991345in}}%
\pgfpathlineto{\pgfqpoint{4.338619in}{0.973794in}}%
\pgfpathlineto{\pgfqpoint{4.348944in}{0.945692in}}%
\pgfpathlineto{\pgfqpoint{4.356688in}{0.928962in}}%
\pgfpathlineto{\pgfqpoint{4.360990in}{0.923810in}}%
\pgfpathlineto{\pgfqpoint{4.363572in}{0.922783in}}%
\pgfpathlineto{\pgfqpoint{4.366153in}{0.924048in}}%
\pgfpathlineto{\pgfqpoint{4.369595in}{0.929752in}}%
\pgfpathlineto{\pgfqpoint{4.376478in}{0.948481in}}%
\pgfpathlineto{\pgfqpoint{4.382501in}{0.969460in}}%
\pgfpathlineto{\pgfqpoint{4.388524in}{1.000676in}}%
\pgfpathlineto{\pgfqpoint{4.404872in}{1.106201in}}%
\pgfpathlineto{\pgfqpoint{4.419500in}{1.221260in}}%
\pgfpathlineto{\pgfqpoint{4.428104in}{1.268824in}}%
\pgfpathlineto{\pgfqpoint{4.435848in}{1.295142in}}%
\pgfpathlineto{\pgfqpoint{4.440150in}{1.304429in}}%
\pgfpathlineto{\pgfqpoint{4.443592in}{1.306563in}}%
\pgfpathlineto{\pgfqpoint{4.447034in}{1.306176in}}%
\pgfpathlineto{\pgfqpoint{4.449615in}{1.304043in}}%
\pgfpathlineto{\pgfqpoint{4.453057in}{1.297699in}}%
\pgfpathlineto{\pgfqpoint{4.457359in}{1.284815in}}%
\pgfpathlineto{\pgfqpoint{4.463382in}{1.258018in}}%
\pgfpathlineto{\pgfqpoint{4.482311in}{1.167810in}}%
\pgfpathlineto{\pgfqpoint{4.491776in}{1.130574in}}%
\pgfpathlineto{\pgfqpoint{4.495218in}{1.122881in}}%
\pgfpathlineto{\pgfqpoint{4.497799in}{1.121208in}}%
\pgfpathlineto{\pgfqpoint{4.500380in}{1.122917in}}%
\pgfpathlineto{\pgfqpoint{4.503822in}{1.129750in}}%
\pgfpathlineto{\pgfqpoint{4.508984in}{1.147299in}}%
\pgfpathlineto{\pgfqpoint{4.515868in}{1.179452in}}%
\pgfpathlineto{\pgfqpoint{4.521891in}{1.217432in}}%
\pgfpathlineto{\pgfqpoint{4.530495in}{1.292792in}}%
\pgfpathlineto{\pgfqpoint{4.541681in}{1.402220in}}%
\pgfpathlineto{\pgfqpoint{4.552866in}{1.501067in}}%
\pgfpathlineto{\pgfqpoint{4.559750in}{1.545347in}}%
\pgfpathlineto{\pgfqpoint{4.564052in}{1.562277in}}%
\pgfpathlineto{\pgfqpoint{4.567494in}{1.568971in}}%
\pgfpathlineto{\pgfqpoint{4.571796in}{1.572392in}}%
\pgfpathlineto{\pgfqpoint{4.574377in}{1.572655in}}%
\pgfpathlineto{\pgfqpoint{4.576958in}{1.570079in}}%
\pgfpathlineto{\pgfqpoint{4.579540in}{1.563068in}}%
\pgfpathlineto{\pgfqpoint{4.583842in}{1.542158in}}%
\pgfpathlineto{\pgfqpoint{4.596748in}{1.462669in}}%
\pgfpathlineto{\pgfqpoint{4.619980in}{1.297022in}}%
\pgfpathlineto{\pgfqpoint{4.627724in}{1.268875in}}%
\pgfpathlineto{\pgfqpoint{4.631166in}{1.262710in}}%
\pgfpathlineto{\pgfqpoint{4.633747in}{1.262294in}}%
\pgfpathlineto{\pgfqpoint{4.636328in}{1.265170in}}%
\pgfpathlineto{\pgfqpoint{4.639770in}{1.273945in}}%
\pgfpathlineto{\pgfqpoint{4.644932in}{1.296256in}}%
\pgfpathlineto{\pgfqpoint{4.650955in}{1.332649in}}%
\pgfpathlineto{\pgfqpoint{4.659560in}{1.400519in}}%
\pgfpathlineto{\pgfqpoint{4.686233in}{1.632248in}}%
\pgfpathlineto{\pgfqpoint{4.692256in}{1.664763in}}%
\pgfpathlineto{\pgfqpoint{4.695698in}{1.673268in}}%
\pgfpathlineto{\pgfqpoint{4.699140in}{1.676411in}}%
\pgfpathlineto{\pgfqpoint{4.701721in}{1.675850in}}%
\pgfpathlineto{\pgfqpoint{4.705163in}{1.671521in}}%
\pgfpathlineto{\pgfqpoint{4.708604in}{1.663252in}}%
\pgfpathlineto{\pgfqpoint{4.712046in}{1.648344in}}%
\pgfpathlineto{\pgfqpoint{4.716348in}{1.618363in}}%
\pgfpathlineto{\pgfqpoint{4.732696in}{1.474052in}}%
\pgfpathlineto{\pgfqpoint{4.746463in}{1.346890in}}%
\pgfpathlineto{\pgfqpoint{4.752486in}{1.303026in}}%
\pgfpathlineto{\pgfqpoint{4.758509in}{1.275528in}}%
\pgfpathlineto{\pgfqpoint{4.762811in}{1.262721in}}%
\pgfpathlineto{\pgfqpoint{4.765393in}{1.260011in}}%
\pgfpathlineto{\pgfqpoint{4.767114in}{1.260555in}}%
\pgfpathlineto{\pgfqpoint{4.769695in}{1.264240in}}%
\pgfpathlineto{\pgfqpoint{4.774857in}{1.277525in}}%
\pgfpathlineto{\pgfqpoint{4.780020in}{1.296590in}}%
\pgfpathlineto{\pgfqpoint{4.785183in}{1.327509in}}%
\pgfpathlineto{\pgfqpoint{4.798089in}{1.421486in}}%
\pgfpathlineto{\pgfqpoint{4.808414in}{1.506247in}}%
\pgfpathlineto{\pgfqpoint{4.814437in}{1.535573in}}%
\pgfpathlineto{\pgfqpoint{4.824762in}{1.573694in}}%
\pgfpathlineto{\pgfqpoint{4.827344in}{1.575132in}}%
\pgfpathlineto{\pgfqpoint{4.829925in}{1.572256in}}%
\pgfpathlineto{\pgfqpoint{4.835088in}{1.561183in}}%
\pgfpathlineto{\pgfqpoint{4.838529in}{1.549725in}}%
\pgfpathlineto{\pgfqpoint{4.842831in}{1.527245in}}%
\pgfpathlineto{\pgfqpoint{4.849715in}{1.477946in}}%
\pgfpathlineto{\pgfqpoint{4.860900in}{1.375020in}}%
\pgfpathlineto{\pgfqpoint{4.883272in}{1.182097in}}%
\pgfpathlineto{\pgfqpoint{4.889295in}{1.153127in}}%
\pgfpathlineto{\pgfqpoint{4.894457in}{1.136709in}}%
\pgfpathlineto{\pgfqpoint{4.897899in}{1.130739in}}%
\pgfpathlineto{\pgfqpoint{4.900480in}{1.129112in}}%
\pgfpathlineto{\pgfqpoint{4.903062in}{1.129692in}}%
\pgfpathlineto{\pgfqpoint{4.906503in}{1.133542in}}%
\pgfpathlineto{\pgfqpoint{4.910805in}{1.142850in}}%
\pgfpathlineto{\pgfqpoint{4.917689in}{1.165037in}}%
\pgfpathlineto{\pgfqpoint{4.926293in}{1.201028in}}%
\pgfpathlineto{\pgfqpoint{4.943502in}{1.283651in}}%
\pgfpathlineto{\pgfqpoint{4.951246in}{1.307000in}}%
\pgfpathlineto{\pgfqpoint{4.953827in}{1.309624in}}%
\pgfpathlineto{\pgfqpoint{4.956408in}{1.309352in}}%
\pgfpathlineto{\pgfqpoint{4.958989in}{1.306820in}}%
\pgfpathlineto{\pgfqpoint{4.962431in}{1.299963in}}%
\pgfpathlineto{\pgfqpoint{4.967594in}{1.283251in}}%
\pgfpathlineto{\pgfqpoint{4.977059in}{1.244110in}}%
\pgfpathlineto{\pgfqpoint{4.983082in}{1.205119in}}%
\pgfpathlineto{\pgfqpoint{5.014917in}{0.978444in}}%
\pgfpathlineto{\pgfqpoint{5.021801in}{0.944943in}}%
\pgfpathlineto{\pgfqpoint{5.026103in}{0.932962in}}%
\pgfpathlineto{\pgfqpoint{5.030405in}{0.927790in}}%
\pgfpathlineto{\pgfqpoint{5.035568in}{0.925167in}}%
\pgfpathlineto{\pgfqpoint{5.039010in}{0.925326in}}%
\pgfpathlineto{\pgfqpoint{5.042451in}{0.928455in}}%
\pgfpathlineto{\pgfqpoint{5.046753in}{0.936444in}}%
\pgfpathlineto{\pgfqpoint{5.059660in}{0.967381in}}%
\pgfpathlineto{\pgfqpoint{5.069125in}{0.991913in}}%
\pgfpathlineto{\pgfqpoint{5.078589in}{1.008500in}}%
\pgfpathlineto{\pgfqpoint{5.082031in}{1.010599in}}%
\pgfpathlineto{\pgfqpoint{5.085473in}{1.009950in}}%
\pgfpathlineto{\pgfqpoint{5.089775in}{1.006462in}}%
\pgfpathlineto{\pgfqpoint{5.094077in}{0.999868in}}%
\pgfpathlineto{\pgfqpoint{5.099240in}{0.986623in}}%
\pgfpathlineto{\pgfqpoint{5.108704in}{0.952755in}}%
\pgfpathlineto{\pgfqpoint{5.115588in}{0.922720in}}%
\pgfpathlineto{\pgfqpoint{5.139680in}{0.806681in}}%
\pgfpathlineto{\pgfqpoint{5.147424in}{0.779339in}}%
\pgfpathlineto{\pgfqpoint{5.154307in}{0.762943in}}%
\pgfpathlineto{\pgfqpoint{5.161191in}{0.752433in}}%
\pgfpathlineto{\pgfqpoint{5.166353in}{0.747568in}}%
\pgfpathlineto{\pgfqpoint{5.171516in}{0.745269in}}%
\pgfpathlineto{\pgfqpoint{5.174957in}{0.745614in}}%
\pgfpathlineto{\pgfqpoint{5.178399in}{0.748127in}}%
\pgfpathlineto{\pgfqpoint{5.191306in}{0.761270in}}%
\pgfpathlineto{\pgfqpoint{5.197329in}{0.763765in}}%
\pgfpathlineto{\pgfqpoint{5.209375in}{0.771904in}}%
\pgfpathlineto{\pgfqpoint{5.218839in}{0.772554in}}%
\pgfpathlineto{\pgfqpoint{5.222281in}{0.770176in}}%
\pgfpathlineto{\pgfqpoint{5.226583in}{0.763585in}}%
\pgfpathlineto{\pgfqpoint{5.236908in}{0.741373in}}%
\pgfpathlineto{\pgfqpoint{5.243792in}{0.721182in}}%
\pgfpathlineto{\pgfqpoint{5.269605in}{0.637833in}}%
\pgfpathlineto{\pgfqpoint{5.282511in}{0.602359in}}%
\pgfpathlineto{\pgfqpoint{5.293697in}{0.577641in}}%
\pgfpathlineto{\pgfqpoint{5.302301in}{0.561980in}}%
\pgfpathlineto{\pgfqpoint{5.312626in}{0.548342in}}%
\pgfpathlineto{\pgfqpoint{5.320370in}{0.541695in}}%
\pgfpathlineto{\pgfqpoint{5.329835in}{0.536578in}}%
\pgfpathlineto{\pgfqpoint{5.340160in}{0.533092in}}%
\pgfpathlineto{\pgfqpoint{5.355648in}{0.530420in}}%
\pgfpathlineto{\pgfqpoint{5.377159in}{0.529341in}}%
\pgfpathlineto{\pgfqpoint{5.519130in}{0.529102in}}%
\pgfpathlineto{\pgfqpoint{5.559570in}{0.529102in}}%
\pgfpathlineto{\pgfqpoint{5.559570in}{0.529102in}}%
\pgfusepath{stroke}%
\end{pgfscope}%
\begin{pgfscope}%
\pgfsetrectcap%
\pgfsetmiterjoin%
\pgfsetlinewidth{1.003750pt}%
\definecolor{currentstroke}{rgb}{0.000000,0.000000,0.000000}%
\pgfsetstrokecolor{currentstroke}%
\pgfsetdash{}{0pt}%
\pgfpathmoveto{\pgfqpoint{5.560000in}{0.529102in}}%
\pgfpathlineto{\pgfqpoint{5.560000in}{1.753333in}}%
\pgfusepath{stroke}%
\end{pgfscope}%
\begin{pgfscope}%
\pgfsetrectcap%
\pgfsetmiterjoin%
\pgfsetlinewidth{1.003750pt}%
\definecolor{currentstroke}{rgb}{0.000000,0.000000,0.000000}%
\pgfsetstrokecolor{currentstroke}%
\pgfsetdash{}{0pt}%
\pgfpathmoveto{\pgfqpoint{3.840000in}{0.529102in}}%
\pgfpathlineto{\pgfqpoint{5.560000in}{0.529102in}}%
\pgfusepath{stroke}%
\end{pgfscope}%
\begin{pgfscope}%
\pgfsetrectcap%
\pgfsetmiterjoin%
\pgfsetlinewidth{1.003750pt}%
\definecolor{currentstroke}{rgb}{0.000000,0.000000,0.000000}%
\pgfsetstrokecolor{currentstroke}%
\pgfsetdash{}{0pt}%
\pgfpathmoveto{\pgfqpoint{3.840000in}{1.753333in}}%
\pgfpathlineto{\pgfqpoint{5.560000in}{1.753333in}}%
\pgfusepath{stroke}%
\end{pgfscope}%
\begin{pgfscope}%
\pgfsetrectcap%
\pgfsetmiterjoin%
\pgfsetlinewidth{1.003750pt}%
\definecolor{currentstroke}{rgb}{0.000000,0.000000,0.000000}%
\pgfsetstrokecolor{currentstroke}%
\pgfsetdash{}{0pt}%
\pgfpathmoveto{\pgfqpoint{3.840000in}{0.529102in}}%
\pgfpathlineto{\pgfqpoint{3.840000in}{1.753333in}}%
\pgfusepath{stroke}%
\end{pgfscope}%
\begin{pgfscope}%
\pgfsetbuttcap%
\pgfsetroundjoin%
\definecolor{currentfill}{rgb}{0.000000,0.000000,0.000000}%
\pgfsetfillcolor{currentfill}%
\pgfsetlinewidth{0.501875pt}%
\definecolor{currentstroke}{rgb}{0.000000,0.000000,0.000000}%
\pgfsetstrokecolor{currentstroke}%
\pgfsetdash{}{0pt}%
\pgfsys@defobject{currentmarker}{\pgfqpoint{0.000000in}{0.000000in}}{\pgfqpoint{0.000000in}{0.055556in}}{%
\pgfpathmoveto{\pgfqpoint{0.000000in}{0.000000in}}%
\pgfpathlineto{\pgfqpoint{0.000000in}{0.055556in}}%
\pgfusepath{stroke,fill}%
}%
\begin{pgfscope}%
\pgfsys@transformshift{4.270000in}{0.529102in}%
\pgfsys@useobject{currentmarker}{}%
\end{pgfscope}%
\end{pgfscope}%
\begin{pgfscope}%
\pgfsetbuttcap%
\pgfsetroundjoin%
\definecolor{currentfill}{rgb}{0.000000,0.000000,0.000000}%
\pgfsetfillcolor{currentfill}%
\pgfsetlinewidth{0.501875pt}%
\definecolor{currentstroke}{rgb}{0.000000,0.000000,0.000000}%
\pgfsetstrokecolor{currentstroke}%
\pgfsetdash{}{0pt}%
\pgfsys@defobject{currentmarker}{\pgfqpoint{0.000000in}{-0.055556in}}{\pgfqpoint{0.000000in}{0.000000in}}{%
\pgfpathmoveto{\pgfqpoint{0.000000in}{0.000000in}}%
\pgfpathlineto{\pgfqpoint{0.000000in}{-0.055556in}}%
\pgfusepath{stroke,fill}%
}%
\begin{pgfscope}%
\pgfsys@transformshift{4.270000in}{1.753333in}%
\pgfsys@useobject{currentmarker}{}%
\end{pgfscope}%
\end{pgfscope}%
\begin{pgfscope}%
\pgftext[x=4.270000in,y=0.473546in,,top]{\fontsize{10.000000}{12.000000}\selectfont -0.5}%
\end{pgfscope}%
\begin{pgfscope}%
\pgfsetbuttcap%
\pgfsetroundjoin%
\definecolor{currentfill}{rgb}{0.000000,0.000000,0.000000}%
\pgfsetfillcolor{currentfill}%
\pgfsetlinewidth{0.501875pt}%
\definecolor{currentstroke}{rgb}{0.000000,0.000000,0.000000}%
\pgfsetstrokecolor{currentstroke}%
\pgfsetdash{}{0pt}%
\pgfsys@defobject{currentmarker}{\pgfqpoint{0.000000in}{0.000000in}}{\pgfqpoint{0.000000in}{0.055556in}}{%
\pgfpathmoveto{\pgfqpoint{0.000000in}{0.000000in}}%
\pgfpathlineto{\pgfqpoint{0.000000in}{0.055556in}}%
\pgfusepath{stroke,fill}%
}%
\begin{pgfscope}%
\pgfsys@transformshift{4.700000in}{0.529102in}%
\pgfsys@useobject{currentmarker}{}%
\end{pgfscope}%
\end{pgfscope}%
\begin{pgfscope}%
\pgfsetbuttcap%
\pgfsetroundjoin%
\definecolor{currentfill}{rgb}{0.000000,0.000000,0.000000}%
\pgfsetfillcolor{currentfill}%
\pgfsetlinewidth{0.501875pt}%
\definecolor{currentstroke}{rgb}{0.000000,0.000000,0.000000}%
\pgfsetstrokecolor{currentstroke}%
\pgfsetdash{}{0pt}%
\pgfsys@defobject{currentmarker}{\pgfqpoint{0.000000in}{-0.055556in}}{\pgfqpoint{0.000000in}{0.000000in}}{%
\pgfpathmoveto{\pgfqpoint{0.000000in}{0.000000in}}%
\pgfpathlineto{\pgfqpoint{0.000000in}{-0.055556in}}%
\pgfusepath{stroke,fill}%
}%
\begin{pgfscope}%
\pgfsys@transformshift{4.700000in}{1.753333in}%
\pgfsys@useobject{currentmarker}{}%
\end{pgfscope}%
\end{pgfscope}%
\begin{pgfscope}%
\pgftext[x=4.700000in,y=0.473546in,,top]{\fontsize{10.000000}{12.000000}\selectfont 0.0}%
\end{pgfscope}%
\begin{pgfscope}%
\pgfsetbuttcap%
\pgfsetroundjoin%
\definecolor{currentfill}{rgb}{0.000000,0.000000,0.000000}%
\pgfsetfillcolor{currentfill}%
\pgfsetlinewidth{0.501875pt}%
\definecolor{currentstroke}{rgb}{0.000000,0.000000,0.000000}%
\pgfsetstrokecolor{currentstroke}%
\pgfsetdash{}{0pt}%
\pgfsys@defobject{currentmarker}{\pgfqpoint{0.000000in}{0.000000in}}{\pgfqpoint{0.000000in}{0.055556in}}{%
\pgfpathmoveto{\pgfqpoint{0.000000in}{0.000000in}}%
\pgfpathlineto{\pgfqpoint{0.000000in}{0.055556in}}%
\pgfusepath{stroke,fill}%
}%
\begin{pgfscope}%
\pgfsys@transformshift{5.130000in}{0.529102in}%
\pgfsys@useobject{currentmarker}{}%
\end{pgfscope}%
\end{pgfscope}%
\begin{pgfscope}%
\pgfsetbuttcap%
\pgfsetroundjoin%
\definecolor{currentfill}{rgb}{0.000000,0.000000,0.000000}%
\pgfsetfillcolor{currentfill}%
\pgfsetlinewidth{0.501875pt}%
\definecolor{currentstroke}{rgb}{0.000000,0.000000,0.000000}%
\pgfsetstrokecolor{currentstroke}%
\pgfsetdash{}{0pt}%
\pgfsys@defobject{currentmarker}{\pgfqpoint{0.000000in}{-0.055556in}}{\pgfqpoint{0.000000in}{0.000000in}}{%
\pgfpathmoveto{\pgfqpoint{0.000000in}{0.000000in}}%
\pgfpathlineto{\pgfqpoint{0.000000in}{-0.055556in}}%
\pgfusepath{stroke,fill}%
}%
\begin{pgfscope}%
\pgfsys@transformshift{5.130000in}{1.753333in}%
\pgfsys@useobject{currentmarker}{}%
\end{pgfscope}%
\end{pgfscope}%
\begin{pgfscope}%
\pgftext[x=5.130000in,y=0.473546in,,top]{\fontsize{10.000000}{12.000000}\selectfont 0.5}%
\end{pgfscope}%
\begin{pgfscope}%
\pgfsetbuttcap%
\pgfsetroundjoin%
\definecolor{currentfill}{rgb}{0.000000,0.000000,0.000000}%
\pgfsetfillcolor{currentfill}%
\pgfsetlinewidth{0.501875pt}%
\definecolor{currentstroke}{rgb}{0.000000,0.000000,0.000000}%
\pgfsetstrokecolor{currentstroke}%
\pgfsetdash{}{0pt}%
\pgfsys@defobject{currentmarker}{\pgfqpoint{0.000000in}{0.000000in}}{\pgfqpoint{0.055556in}{0.000000in}}{%
\pgfpathmoveto{\pgfqpoint{0.000000in}{0.000000in}}%
\pgfpathlineto{\pgfqpoint{0.055556in}{0.000000in}}%
\pgfusepath{stroke,fill}%
}%
\begin{pgfscope}%
\pgfsys@transformshift{3.840000in}{0.529102in}%
\pgfsys@useobject{currentmarker}{}%
\end{pgfscope}%
\end{pgfscope}%
\begin{pgfscope}%
\pgfsetbuttcap%
\pgfsetroundjoin%
\definecolor{currentfill}{rgb}{0.000000,0.000000,0.000000}%
\pgfsetfillcolor{currentfill}%
\pgfsetlinewidth{0.501875pt}%
\definecolor{currentstroke}{rgb}{0.000000,0.000000,0.000000}%
\pgfsetstrokecolor{currentstroke}%
\pgfsetdash{}{0pt}%
\pgfsys@defobject{currentmarker}{\pgfqpoint{-0.055556in}{0.000000in}}{\pgfqpoint{0.000000in}{0.000000in}}{%
\pgfpathmoveto{\pgfqpoint{0.000000in}{0.000000in}}%
\pgfpathlineto{\pgfqpoint{-0.055556in}{0.000000in}}%
\pgfusepath{stroke,fill}%
}%
\begin{pgfscope}%
\pgfsys@transformshift{5.560000in}{0.529102in}%
\pgfsys@useobject{currentmarker}{}%
\end{pgfscope}%
\end{pgfscope}%
\begin{pgfscope}%
\pgfsetbuttcap%
\pgfsetroundjoin%
\definecolor{currentfill}{rgb}{0.000000,0.000000,0.000000}%
\pgfsetfillcolor{currentfill}%
\pgfsetlinewidth{0.501875pt}%
\definecolor{currentstroke}{rgb}{0.000000,0.000000,0.000000}%
\pgfsetstrokecolor{currentstroke}%
\pgfsetdash{}{0pt}%
\pgfsys@defobject{currentmarker}{\pgfqpoint{0.000000in}{0.000000in}}{\pgfqpoint{0.055556in}{0.000000in}}{%
\pgfpathmoveto{\pgfqpoint{0.000000in}{0.000000in}}%
\pgfpathlineto{\pgfqpoint{0.055556in}{0.000000in}}%
\pgfusepath{stroke,fill}%
}%
\begin{pgfscope}%
\pgfsys@transformshift{3.840000in}{0.773948in}%
\pgfsys@useobject{currentmarker}{}%
\end{pgfscope}%
\end{pgfscope}%
\begin{pgfscope}%
\pgfsetbuttcap%
\pgfsetroundjoin%
\definecolor{currentfill}{rgb}{0.000000,0.000000,0.000000}%
\pgfsetfillcolor{currentfill}%
\pgfsetlinewidth{0.501875pt}%
\definecolor{currentstroke}{rgb}{0.000000,0.000000,0.000000}%
\pgfsetstrokecolor{currentstroke}%
\pgfsetdash{}{0pt}%
\pgfsys@defobject{currentmarker}{\pgfqpoint{-0.055556in}{0.000000in}}{\pgfqpoint{0.000000in}{0.000000in}}{%
\pgfpathmoveto{\pgfqpoint{0.000000in}{0.000000in}}%
\pgfpathlineto{\pgfqpoint{-0.055556in}{0.000000in}}%
\pgfusepath{stroke,fill}%
}%
\begin{pgfscope}%
\pgfsys@transformshift{5.560000in}{0.773948in}%
\pgfsys@useobject{currentmarker}{}%
\end{pgfscope}%
\end{pgfscope}%
\begin{pgfscope}%
\pgfsetbuttcap%
\pgfsetroundjoin%
\definecolor{currentfill}{rgb}{0.000000,0.000000,0.000000}%
\pgfsetfillcolor{currentfill}%
\pgfsetlinewidth{0.501875pt}%
\definecolor{currentstroke}{rgb}{0.000000,0.000000,0.000000}%
\pgfsetstrokecolor{currentstroke}%
\pgfsetdash{}{0pt}%
\pgfsys@defobject{currentmarker}{\pgfqpoint{0.000000in}{0.000000in}}{\pgfqpoint{0.055556in}{0.000000in}}{%
\pgfpathmoveto{\pgfqpoint{0.000000in}{0.000000in}}%
\pgfpathlineto{\pgfqpoint{0.055556in}{0.000000in}}%
\pgfusepath{stroke,fill}%
}%
\begin{pgfscope}%
\pgfsys@transformshift{3.840000in}{1.018794in}%
\pgfsys@useobject{currentmarker}{}%
\end{pgfscope}%
\end{pgfscope}%
\begin{pgfscope}%
\pgfsetbuttcap%
\pgfsetroundjoin%
\definecolor{currentfill}{rgb}{0.000000,0.000000,0.000000}%
\pgfsetfillcolor{currentfill}%
\pgfsetlinewidth{0.501875pt}%
\definecolor{currentstroke}{rgb}{0.000000,0.000000,0.000000}%
\pgfsetstrokecolor{currentstroke}%
\pgfsetdash{}{0pt}%
\pgfsys@defobject{currentmarker}{\pgfqpoint{-0.055556in}{0.000000in}}{\pgfqpoint{0.000000in}{0.000000in}}{%
\pgfpathmoveto{\pgfqpoint{0.000000in}{0.000000in}}%
\pgfpathlineto{\pgfqpoint{-0.055556in}{0.000000in}}%
\pgfusepath{stroke,fill}%
}%
\begin{pgfscope}%
\pgfsys@transformshift{5.560000in}{1.018794in}%
\pgfsys@useobject{currentmarker}{}%
\end{pgfscope}%
\end{pgfscope}%
\begin{pgfscope}%
\pgfsetbuttcap%
\pgfsetroundjoin%
\definecolor{currentfill}{rgb}{0.000000,0.000000,0.000000}%
\pgfsetfillcolor{currentfill}%
\pgfsetlinewidth{0.501875pt}%
\definecolor{currentstroke}{rgb}{0.000000,0.000000,0.000000}%
\pgfsetstrokecolor{currentstroke}%
\pgfsetdash{}{0pt}%
\pgfsys@defobject{currentmarker}{\pgfqpoint{0.000000in}{0.000000in}}{\pgfqpoint{0.055556in}{0.000000in}}{%
\pgfpathmoveto{\pgfqpoint{0.000000in}{0.000000in}}%
\pgfpathlineto{\pgfqpoint{0.055556in}{0.000000in}}%
\pgfusepath{stroke,fill}%
}%
\begin{pgfscope}%
\pgfsys@transformshift{3.840000in}{1.263641in}%
\pgfsys@useobject{currentmarker}{}%
\end{pgfscope}%
\end{pgfscope}%
\begin{pgfscope}%
\pgfsetbuttcap%
\pgfsetroundjoin%
\definecolor{currentfill}{rgb}{0.000000,0.000000,0.000000}%
\pgfsetfillcolor{currentfill}%
\pgfsetlinewidth{0.501875pt}%
\definecolor{currentstroke}{rgb}{0.000000,0.000000,0.000000}%
\pgfsetstrokecolor{currentstroke}%
\pgfsetdash{}{0pt}%
\pgfsys@defobject{currentmarker}{\pgfqpoint{-0.055556in}{0.000000in}}{\pgfqpoint{0.000000in}{0.000000in}}{%
\pgfpathmoveto{\pgfqpoint{0.000000in}{0.000000in}}%
\pgfpathlineto{\pgfqpoint{-0.055556in}{0.000000in}}%
\pgfusepath{stroke,fill}%
}%
\begin{pgfscope}%
\pgfsys@transformshift{5.560000in}{1.263641in}%
\pgfsys@useobject{currentmarker}{}%
\end{pgfscope}%
\end{pgfscope}%
\begin{pgfscope}%
\pgfsetbuttcap%
\pgfsetroundjoin%
\definecolor{currentfill}{rgb}{0.000000,0.000000,0.000000}%
\pgfsetfillcolor{currentfill}%
\pgfsetlinewidth{0.501875pt}%
\definecolor{currentstroke}{rgb}{0.000000,0.000000,0.000000}%
\pgfsetstrokecolor{currentstroke}%
\pgfsetdash{}{0pt}%
\pgfsys@defobject{currentmarker}{\pgfqpoint{0.000000in}{0.000000in}}{\pgfqpoint{0.055556in}{0.000000in}}{%
\pgfpathmoveto{\pgfqpoint{0.000000in}{0.000000in}}%
\pgfpathlineto{\pgfqpoint{0.055556in}{0.000000in}}%
\pgfusepath{stroke,fill}%
}%
\begin{pgfscope}%
\pgfsys@transformshift{3.840000in}{1.508487in}%
\pgfsys@useobject{currentmarker}{}%
\end{pgfscope}%
\end{pgfscope}%
\begin{pgfscope}%
\pgfsetbuttcap%
\pgfsetroundjoin%
\definecolor{currentfill}{rgb}{0.000000,0.000000,0.000000}%
\pgfsetfillcolor{currentfill}%
\pgfsetlinewidth{0.501875pt}%
\definecolor{currentstroke}{rgb}{0.000000,0.000000,0.000000}%
\pgfsetstrokecolor{currentstroke}%
\pgfsetdash{}{0pt}%
\pgfsys@defobject{currentmarker}{\pgfqpoint{-0.055556in}{0.000000in}}{\pgfqpoint{0.000000in}{0.000000in}}{%
\pgfpathmoveto{\pgfqpoint{0.000000in}{0.000000in}}%
\pgfpathlineto{\pgfqpoint{-0.055556in}{0.000000in}}%
\pgfusepath{stroke,fill}%
}%
\begin{pgfscope}%
\pgfsys@transformshift{5.560000in}{1.508487in}%
\pgfsys@useobject{currentmarker}{}%
\end{pgfscope}%
\end{pgfscope}%
\begin{pgfscope}%
\pgfsetbuttcap%
\pgfsetroundjoin%
\definecolor{currentfill}{rgb}{0.000000,0.000000,0.000000}%
\pgfsetfillcolor{currentfill}%
\pgfsetlinewidth{0.501875pt}%
\definecolor{currentstroke}{rgb}{0.000000,0.000000,0.000000}%
\pgfsetstrokecolor{currentstroke}%
\pgfsetdash{}{0pt}%
\pgfsys@defobject{currentmarker}{\pgfqpoint{0.000000in}{0.000000in}}{\pgfqpoint{0.055556in}{0.000000in}}{%
\pgfpathmoveto{\pgfqpoint{0.000000in}{0.000000in}}%
\pgfpathlineto{\pgfqpoint{0.055556in}{0.000000in}}%
\pgfusepath{stroke,fill}%
}%
\begin{pgfscope}%
\pgfsys@transformshift{3.840000in}{1.753333in}%
\pgfsys@useobject{currentmarker}{}%
\end{pgfscope}%
\end{pgfscope}%
\begin{pgfscope}%
\pgfsetbuttcap%
\pgfsetroundjoin%
\definecolor{currentfill}{rgb}{0.000000,0.000000,0.000000}%
\pgfsetfillcolor{currentfill}%
\pgfsetlinewidth{0.501875pt}%
\definecolor{currentstroke}{rgb}{0.000000,0.000000,0.000000}%
\pgfsetstrokecolor{currentstroke}%
\pgfsetdash{}{0pt}%
\pgfsys@defobject{currentmarker}{\pgfqpoint{-0.055556in}{0.000000in}}{\pgfqpoint{0.000000in}{0.000000in}}{%
\pgfpathmoveto{\pgfqpoint{0.000000in}{0.000000in}}%
\pgfpathlineto{\pgfqpoint{-0.055556in}{0.000000in}}%
\pgfusepath{stroke,fill}%
}%
\begin{pgfscope}%
\pgfsys@transformshift{5.560000in}{1.753333in}%
\pgfsys@useobject{currentmarker}{}%
\end{pgfscope}%
\end{pgfscope}%
\end{pgfpicture}%
\makeatother%
\endgroup%

    \caption[An example of overlapping beamlets.]{An example of overlapping beamlets.
    The left figure shows overlapping beamlets from simulated data (shown as a dotted line in the other figures), the middle figure shows Gaussian parameters for each beamlet derived by considering only the portion of each beamlet between the troughs in the distribution, and the right figure shows Gaussian parameters from fitting the sum of Gaussians to the data.
    The actual emittance of the simulated bunch is listed in the left figure and the emittance derived from the Gaussian sets is listed in the other two figures.
    In the second two figure the pale lines indicate individual Gaussians fitted for each beamlet and solid lines for the sum of the individual Gaussians.}
    \label{figure:overlap_example}
    % Data and code located in Code/Electrons/Simulation/1D Simulation SimScript1D.py and OverlapExample.py
\end{figure}


\subsection{Pepperpot Extent and Beam Coverage}
The third correction that was made was to take into account the proportion of the beam that was covered by the pepperpot mask.
A pepperpot mask that has an extent smaller than the size of the beam is effectively aperturing the beam and apertures reduce the emittance of a beam.
Thus, attempting to measure the emittance with a pepperpot that has an extent smaller the the size of the beam will result in an underestimation of the emittance proportional to coverage of the beam.

The emittance of a beam at a focus is directly proportional to beam size (see Equations~\ref{emittance} and \ref{equation:excess_energy_emittance}) so if a pepperpot measurement of a flat-top beam was to only measure half the beam the measurement would report half the actual emittance of the beam.
The correction is more complicated for a Gaussian profile beam as it is proportional to the integral of the normalised Gaussian covered by the pepperpot.
If the beam shape in the pepperpot plane can be described by a Gaussian function, $f_{\bar{x},\sigma}(x)$, then the true emittance should be
\begin{equation}
\epsilon_t = \epsilon_m \int_{-E}^{E} f_{\bar{x}, \sigma}(x) dx,
\end{equation}
where $\epsilon_m$ is the measured uncorrected emittance and the pepperpot extends from $-E$ to $+E$.
The simulations indicate that the reality is not this simple unfortunately. In order to get the simulated correction to work is it necessary to use:
\begin{equation}
\epsilon_t = \epsilon_m \int_{-0.75*E}^{0.75*E} f_{\bar{x}, \sigma}(x) dx.
\end{equation}
The factor of 0.75 was discovered by fitting to the simulated data.
There is clearly some aspect of this correction that is not yet understood however it performs adequately for use with the measurements in this chapter.
The performance of the correction is shown in Figure~\ref{figure:pepperpot_extent}.
The correction performs well until the pepperpot extent approaches the \gls{rms} width of the beam at which point the beam size estimations become less reliable.

{\color{red}
Try and figure out what's going on with the magic number.
}

\begin{figure}
    \center
    %% Creator: Matplotlib, PGF backend
%%
%% To include the figure in your LaTeX document, write
%%   \input{<filename>.pgf}
%%
%% Make sure the required packages are loaded in your preamble
%%   \usepackage{pgf}
%%
%% Figures using additional raster images can only be included by \input if
%% they are in the same directory as the main LaTeX file. For loading figures
%% from other directories you can use the `import` package
%%   \usepackage{import}
%% and then include the figures with
%%   \import{<path to file>}{<filename>.pgf}
%%
%% Matplotlib used the following preamble
%%
\begingroup%
\makeatletter%
\begin{pgfpicture}%
\pgfpathrectangle{\pgfpointorigin}{\pgfqpoint{4.282500in}{1.903333in}}%
\pgfusepath{use as bounding box, clip}%
\begin{pgfscope}%
\pgfsetbuttcap%
\pgfsetmiterjoin%
\definecolor{currentfill}{rgb}{1.000000,1.000000,1.000000}%
\pgfsetfillcolor{currentfill}%
\pgfsetlinewidth{0.000000pt}%
\definecolor{currentstroke}{rgb}{1.000000,1.000000,1.000000}%
\pgfsetstrokecolor{currentstroke}%
\pgfsetdash{}{0pt}%
\pgfpathmoveto{\pgfqpoint{0.000000in}{0.000000in}}%
\pgfpathlineto{\pgfqpoint{4.282500in}{0.000000in}}%
\pgfpathlineto{\pgfqpoint{4.282500in}{1.903333in}}%
\pgfpathlineto{\pgfqpoint{0.000000in}{1.903333in}}%
\pgfpathclose%
\pgfusepath{fill}%
\end{pgfscope}%
\begin{pgfscope}%
\pgfsetbuttcap%
\pgfsetmiterjoin%
\definecolor{currentfill}{rgb}{1.000000,1.000000,1.000000}%
\pgfsetfillcolor{currentfill}%
\pgfsetlinewidth{0.000000pt}%
\definecolor{currentstroke}{rgb}{0.000000,0.000000,0.000000}%
\pgfsetstrokecolor{currentstroke}%
\pgfsetstrokeopacity{0.000000}%
\pgfsetdash{}{0pt}%
\pgfpathmoveto{\pgfqpoint{0.671094in}{0.532031in}}%
\pgfpathlineto{\pgfqpoint{4.087969in}{0.532031in}}%
\pgfpathlineto{\pgfqpoint{4.087969in}{1.753333in}}%
\pgfpathlineto{\pgfqpoint{0.671094in}{1.753333in}}%
\pgfpathclose%
\pgfusepath{fill}%
\end{pgfscope}%
\begin{pgfscope}%
\pgfpathrectangle{\pgfqpoint{0.671094in}{0.532031in}}{\pgfqpoint{3.416875in}{1.221302in}} %
\pgfusepath{clip}%
\pgfsetrectcap%
\pgfsetroundjoin%
\pgfsetlinewidth{1.003750pt}%
\definecolor{currentstroke}{rgb}{0.309804,0.478431,0.682353}%
\pgfsetstrokecolor{currentstroke}%
\pgfsetdash{}{0pt}%
\pgfpathmoveto{\pgfqpoint{1.012781in}{0.735151in}}%
\pgfpathlineto{\pgfqpoint{1.126677in}{0.811961in}}%
\pgfpathlineto{\pgfqpoint{1.240573in}{0.882966in}}%
\pgfpathlineto{\pgfqpoint{1.354469in}{0.948546in}}%
\pgfpathlineto{\pgfqpoint{1.468365in}{1.016979in}}%
\pgfpathlineto{\pgfqpoint{1.582260in}{1.081004in}}%
\pgfpathlineto{\pgfqpoint{1.696156in}{1.138056in}}%
\pgfpathlineto{\pgfqpoint{1.810052in}{1.198804in}}%
\pgfpathlineto{\pgfqpoint{1.923948in}{1.248067in}}%
\pgfpathlineto{\pgfqpoint{2.037844in}{1.302869in}}%
\pgfpathlineto{\pgfqpoint{2.151740in}{1.342647in}}%
\pgfpathlineto{\pgfqpoint{2.265635in}{1.390572in}}%
\pgfpathlineto{\pgfqpoint{2.379531in}{1.422555in}}%
\pgfpathlineto{\pgfqpoint{2.493427in}{1.463305in}}%
\pgfpathlineto{\pgfqpoint{2.607323in}{1.488589in}}%
\pgfpathlineto{\pgfqpoint{2.721219in}{1.519074in}}%
\pgfpathlineto{\pgfqpoint{2.835115in}{1.535389in}}%
\pgfpathlineto{\pgfqpoint{2.949010in}{1.561191in}}%
\pgfpathlineto{\pgfqpoint{3.062906in}{1.573501in}}%
\pgfpathlineto{\pgfqpoint{3.176802in}{1.590733in}}%
\pgfpathlineto{\pgfqpoint{3.290698in}{1.595325in}}%
\pgfpathlineto{\pgfqpoint{3.404594in}{1.610633in}}%
\pgfpathlineto{\pgfqpoint{3.518490in}{1.612867in}}%
\pgfpathlineto{\pgfqpoint{3.632385in}{1.623346in}}%
\pgfpathlineto{\pgfqpoint{3.746281in}{1.622627in}}%
\pgfpathlineto{\pgfqpoint{3.860177in}{1.630622in}}%
\pgfpathlineto{\pgfqpoint{3.974073in}{1.628043in}}%
\pgfpathlineto{\pgfqpoint{4.087969in}{1.634363in}}%
\pgfusepath{stroke}%
\end{pgfscope}%
\begin{pgfscope}%
\pgfpathrectangle{\pgfqpoint{0.671094in}{0.532031in}}{\pgfqpoint{3.416875in}{1.221302in}} %
\pgfusepath{clip}%
\pgfsetrectcap%
\pgfsetroundjoin%
\pgfsetlinewidth{1.003750pt}%
\definecolor{currentstroke}{rgb}{0.301961,0.607843,0.301961}%
\pgfsetstrokecolor{currentstroke}%
\pgfsetdash{}{0pt}%
\pgfpathmoveto{\pgfqpoint{1.012781in}{1.409438in}}%
\pgfpathlineto{\pgfqpoint{1.126677in}{1.517792in}}%
\pgfpathlineto{\pgfqpoint{1.240573in}{1.558944in}}%
\pgfpathlineto{\pgfqpoint{1.354469in}{1.568895in}}%
\pgfpathlineto{\pgfqpoint{1.468365in}{1.591713in}}%
\pgfpathlineto{\pgfqpoint{1.582260in}{1.607329in}}%
\pgfpathlineto{\pgfqpoint{1.696156in}{1.607011in}}%
\pgfpathlineto{\pgfqpoint{1.810052in}{1.624207in}}%
\pgfpathlineto{\pgfqpoint{1.923948in}{1.620787in}}%
\pgfpathlineto{\pgfqpoint{2.037844in}{1.638037in}}%
\pgfpathlineto{\pgfqpoint{2.151740in}{1.631227in}}%
\pgfpathlineto{\pgfqpoint{2.265635in}{1.646681in}}%
\pgfpathlineto{\pgfqpoint{2.379531in}{1.640469in}}%
\pgfpathlineto{\pgfqpoint{2.493427in}{1.654348in}}%
\pgfpathlineto{\pgfqpoint{2.607323in}{1.648870in}}%
\pgfpathlineto{\pgfqpoint{2.721219in}{1.657425in}}%
\pgfpathlineto{\pgfqpoint{2.835115in}{1.649499in}}%
\pgfpathlineto{\pgfqpoint{2.949010in}{1.658714in}}%
\pgfpathlineto{\pgfqpoint{3.062906in}{1.652836in}}%
\pgfpathlineto{\pgfqpoint{3.176802in}{1.657596in}}%
\pgfpathlineto{\pgfqpoint{3.290698in}{1.648713in}}%
\pgfpathlineto{\pgfqpoint{3.404594in}{1.655256in}}%
\pgfpathlineto{\pgfqpoint{3.518490in}{1.647985in}}%
\pgfpathlineto{\pgfqpoint{3.632385in}{1.652327in}}%
\pgfpathlineto{\pgfqpoint{3.746281in}{1.645082in}}%
\pgfpathlineto{\pgfqpoint{3.860177in}{1.648963in}}%
\pgfpathlineto{\pgfqpoint{3.974073in}{1.642029in}}%
\pgfpathlineto{\pgfqpoint{4.087969in}{1.645666in}}%
\pgfusepath{stroke}%
\end{pgfscope}%
\begin{pgfscope}%
\pgfpathrectangle{\pgfqpoint{0.671094in}{0.532031in}}{\pgfqpoint{3.416875in}{1.221302in}} %
\pgfusepath{clip}%
\pgfsetbuttcap%
\pgfsetroundjoin%
\pgfsetlinewidth{1.003750pt}%
\definecolor{currentstroke}{rgb}{1.000000,0.400000,0.200000}%
\pgfsetstrokecolor{currentstroke}%
\pgfsetdash{{1.000000pt}{3.000000pt}}{0.000000pt}%
\pgfpathmoveto{\pgfqpoint{0.671094in}{1.657867in}}%
\pgfpathlineto{\pgfqpoint{4.087969in}{1.657867in}}%
\pgfusepath{stroke}%
\end{pgfscope}%
\begin{pgfscope}%
\pgfpathrectangle{\pgfqpoint{0.671094in}{0.532031in}}{\pgfqpoint{3.416875in}{1.221302in}} %
\pgfusepath{clip}%
\pgfsetbuttcap%
\pgfsetroundjoin%
\pgfsetlinewidth{1.003750pt}%
\definecolor{currentstroke}{rgb}{0.000000,0.000000,0.000000}%
\pgfsetstrokecolor{currentstroke}%
\pgfsetdash{{1.000000pt}{3.000000pt}}{0.000000pt}%
\pgfpathmoveto{\pgfqpoint{1.167999in}{0.532031in}}%
\pgfpathlineto{\pgfqpoint{1.167999in}{1.753333in}}%
\pgfusepath{stroke}%
\end{pgfscope}%
\begin{pgfscope}%
\pgfsetrectcap%
\pgfsetmiterjoin%
\pgfsetlinewidth{1.003750pt}%
\definecolor{currentstroke}{rgb}{0.000000,0.000000,0.000000}%
\pgfsetstrokecolor{currentstroke}%
\pgfsetdash{}{0pt}%
\pgfpathmoveto{\pgfqpoint{0.671094in}{0.532031in}}%
\pgfpathlineto{\pgfqpoint{0.671094in}{1.753333in}}%
\pgfusepath{stroke}%
\end{pgfscope}%
\begin{pgfscope}%
\pgfsetrectcap%
\pgfsetmiterjoin%
\pgfsetlinewidth{1.003750pt}%
\definecolor{currentstroke}{rgb}{0.000000,0.000000,0.000000}%
\pgfsetstrokecolor{currentstroke}%
\pgfsetdash{}{0pt}%
\pgfpathmoveto{\pgfqpoint{0.671094in}{1.753333in}}%
\pgfpathlineto{\pgfqpoint{4.087969in}{1.753333in}}%
\pgfusepath{stroke}%
\end{pgfscope}%
\begin{pgfscope}%
\pgfsetrectcap%
\pgfsetmiterjoin%
\pgfsetlinewidth{1.003750pt}%
\definecolor{currentstroke}{rgb}{0.000000,0.000000,0.000000}%
\pgfsetstrokecolor{currentstroke}%
\pgfsetdash{}{0pt}%
\pgfpathmoveto{\pgfqpoint{0.671094in}{0.532031in}}%
\pgfpathlineto{\pgfqpoint{4.087969in}{0.532031in}}%
\pgfusepath{stroke}%
\end{pgfscope}%
\begin{pgfscope}%
\pgfsetrectcap%
\pgfsetmiterjoin%
\pgfsetlinewidth{1.003750pt}%
\definecolor{currentstroke}{rgb}{0.000000,0.000000,0.000000}%
\pgfsetstrokecolor{currentstroke}%
\pgfsetdash{}{0pt}%
\pgfpathmoveto{\pgfqpoint{4.087969in}{0.532031in}}%
\pgfpathlineto{\pgfqpoint{4.087969in}{1.753333in}}%
\pgfusepath{stroke}%
\end{pgfscope}%
\begin{pgfscope}%
\pgfsetbuttcap%
\pgfsetroundjoin%
\definecolor{currentfill}{rgb}{0.000000,0.000000,0.000000}%
\pgfsetfillcolor{currentfill}%
\pgfsetlinewidth{0.501875pt}%
\definecolor{currentstroke}{rgb}{0.000000,0.000000,0.000000}%
\pgfsetstrokecolor{currentstroke}%
\pgfsetdash{}{0pt}%
\pgfsys@defobject{currentmarker}{\pgfqpoint{0.000000in}{0.000000in}}{\pgfqpoint{0.000000in}{0.055556in}}{%
\pgfpathmoveto{\pgfqpoint{0.000000in}{0.000000in}}%
\pgfpathlineto{\pgfqpoint{0.000000in}{0.055556in}}%
\pgfusepath{stroke,fill}%
}%
\begin{pgfscope}%
\pgfsys@transformshift{0.671094in}{0.532031in}%
\pgfsys@useobject{currentmarker}{}%
\end{pgfscope}%
\end{pgfscope}%
\begin{pgfscope}%
\pgfsetbuttcap%
\pgfsetroundjoin%
\definecolor{currentfill}{rgb}{0.000000,0.000000,0.000000}%
\pgfsetfillcolor{currentfill}%
\pgfsetlinewidth{0.501875pt}%
\definecolor{currentstroke}{rgb}{0.000000,0.000000,0.000000}%
\pgfsetstrokecolor{currentstroke}%
\pgfsetdash{}{0pt}%
\pgfsys@defobject{currentmarker}{\pgfqpoint{0.000000in}{-0.055556in}}{\pgfqpoint{0.000000in}{0.000000in}}{%
\pgfpathmoveto{\pgfqpoint{0.000000in}{0.000000in}}%
\pgfpathlineto{\pgfqpoint{0.000000in}{-0.055556in}}%
\pgfusepath{stroke,fill}%
}%
\begin{pgfscope}%
\pgfsys@transformshift{0.671094in}{1.753333in}%
\pgfsys@useobject{currentmarker}{}%
\end{pgfscope}%
\end{pgfscope}%
\begin{pgfscope}%
\pgftext[x=0.671094in,y=0.476476in,,top]{\rmfamily\fontsize{10.000000}{12.000000}\selectfont 0}%
\end{pgfscope}%
\begin{pgfscope}%
\pgfsetbuttcap%
\pgfsetroundjoin%
\definecolor{currentfill}{rgb}{0.000000,0.000000,0.000000}%
\pgfsetfillcolor{currentfill}%
\pgfsetlinewidth{0.501875pt}%
\definecolor{currentstroke}{rgb}{0.000000,0.000000,0.000000}%
\pgfsetstrokecolor{currentstroke}%
\pgfsetdash{}{0pt}%
\pgfsys@defobject{currentmarker}{\pgfqpoint{0.000000in}{0.000000in}}{\pgfqpoint{0.000000in}{0.055556in}}{%
\pgfpathmoveto{\pgfqpoint{0.000000in}{0.000000in}}%
\pgfpathlineto{\pgfqpoint{0.000000in}{0.055556in}}%
\pgfusepath{stroke,fill}%
}%
\begin{pgfscope}%
\pgfsys@transformshift{1.240573in}{0.532031in}%
\pgfsys@useobject{currentmarker}{}%
\end{pgfscope}%
\end{pgfscope}%
\begin{pgfscope}%
\pgfsetbuttcap%
\pgfsetroundjoin%
\definecolor{currentfill}{rgb}{0.000000,0.000000,0.000000}%
\pgfsetfillcolor{currentfill}%
\pgfsetlinewidth{0.501875pt}%
\definecolor{currentstroke}{rgb}{0.000000,0.000000,0.000000}%
\pgfsetstrokecolor{currentstroke}%
\pgfsetdash{}{0pt}%
\pgfsys@defobject{currentmarker}{\pgfqpoint{0.000000in}{-0.055556in}}{\pgfqpoint{0.000000in}{0.000000in}}{%
\pgfpathmoveto{\pgfqpoint{0.000000in}{0.000000in}}%
\pgfpathlineto{\pgfqpoint{0.000000in}{-0.055556in}}%
\pgfusepath{stroke,fill}%
}%
\begin{pgfscope}%
\pgfsys@transformshift{1.240573in}{1.753333in}%
\pgfsys@useobject{currentmarker}{}%
\end{pgfscope}%
\end{pgfscope}%
\begin{pgfscope}%
\pgftext[x=1.240573in,y=0.476476in,,top]{\rmfamily\fontsize{10.000000}{12.000000}\selectfont 1}%
\end{pgfscope}%
\begin{pgfscope}%
\pgfsetbuttcap%
\pgfsetroundjoin%
\definecolor{currentfill}{rgb}{0.000000,0.000000,0.000000}%
\pgfsetfillcolor{currentfill}%
\pgfsetlinewidth{0.501875pt}%
\definecolor{currentstroke}{rgb}{0.000000,0.000000,0.000000}%
\pgfsetstrokecolor{currentstroke}%
\pgfsetdash{}{0pt}%
\pgfsys@defobject{currentmarker}{\pgfqpoint{0.000000in}{0.000000in}}{\pgfqpoint{0.000000in}{0.055556in}}{%
\pgfpathmoveto{\pgfqpoint{0.000000in}{0.000000in}}%
\pgfpathlineto{\pgfqpoint{0.000000in}{0.055556in}}%
\pgfusepath{stroke,fill}%
}%
\begin{pgfscope}%
\pgfsys@transformshift{1.810052in}{0.532031in}%
\pgfsys@useobject{currentmarker}{}%
\end{pgfscope}%
\end{pgfscope}%
\begin{pgfscope}%
\pgfsetbuttcap%
\pgfsetroundjoin%
\definecolor{currentfill}{rgb}{0.000000,0.000000,0.000000}%
\pgfsetfillcolor{currentfill}%
\pgfsetlinewidth{0.501875pt}%
\definecolor{currentstroke}{rgb}{0.000000,0.000000,0.000000}%
\pgfsetstrokecolor{currentstroke}%
\pgfsetdash{}{0pt}%
\pgfsys@defobject{currentmarker}{\pgfqpoint{0.000000in}{-0.055556in}}{\pgfqpoint{0.000000in}{0.000000in}}{%
\pgfpathmoveto{\pgfqpoint{0.000000in}{0.000000in}}%
\pgfpathlineto{\pgfqpoint{0.000000in}{-0.055556in}}%
\pgfusepath{stroke,fill}%
}%
\begin{pgfscope}%
\pgfsys@transformshift{1.810052in}{1.753333in}%
\pgfsys@useobject{currentmarker}{}%
\end{pgfscope}%
\end{pgfscope}%
\begin{pgfscope}%
\pgftext[x=1.810052in,y=0.476476in,,top]{\rmfamily\fontsize{10.000000}{12.000000}\selectfont 2}%
\end{pgfscope}%
\begin{pgfscope}%
\pgfsetbuttcap%
\pgfsetroundjoin%
\definecolor{currentfill}{rgb}{0.000000,0.000000,0.000000}%
\pgfsetfillcolor{currentfill}%
\pgfsetlinewidth{0.501875pt}%
\definecolor{currentstroke}{rgb}{0.000000,0.000000,0.000000}%
\pgfsetstrokecolor{currentstroke}%
\pgfsetdash{}{0pt}%
\pgfsys@defobject{currentmarker}{\pgfqpoint{0.000000in}{0.000000in}}{\pgfqpoint{0.000000in}{0.055556in}}{%
\pgfpathmoveto{\pgfqpoint{0.000000in}{0.000000in}}%
\pgfpathlineto{\pgfqpoint{0.000000in}{0.055556in}}%
\pgfusepath{stroke,fill}%
}%
\begin{pgfscope}%
\pgfsys@transformshift{2.379531in}{0.532031in}%
\pgfsys@useobject{currentmarker}{}%
\end{pgfscope}%
\end{pgfscope}%
\begin{pgfscope}%
\pgfsetbuttcap%
\pgfsetroundjoin%
\definecolor{currentfill}{rgb}{0.000000,0.000000,0.000000}%
\pgfsetfillcolor{currentfill}%
\pgfsetlinewidth{0.501875pt}%
\definecolor{currentstroke}{rgb}{0.000000,0.000000,0.000000}%
\pgfsetstrokecolor{currentstroke}%
\pgfsetdash{}{0pt}%
\pgfsys@defobject{currentmarker}{\pgfqpoint{0.000000in}{-0.055556in}}{\pgfqpoint{0.000000in}{0.000000in}}{%
\pgfpathmoveto{\pgfqpoint{0.000000in}{0.000000in}}%
\pgfpathlineto{\pgfqpoint{0.000000in}{-0.055556in}}%
\pgfusepath{stroke,fill}%
}%
\begin{pgfscope}%
\pgfsys@transformshift{2.379531in}{1.753333in}%
\pgfsys@useobject{currentmarker}{}%
\end{pgfscope}%
\end{pgfscope}%
\begin{pgfscope}%
\pgftext[x=2.379531in,y=0.476476in,,top]{\rmfamily\fontsize{10.000000}{12.000000}\selectfont 3}%
\end{pgfscope}%
\begin{pgfscope}%
\pgfsetbuttcap%
\pgfsetroundjoin%
\definecolor{currentfill}{rgb}{0.000000,0.000000,0.000000}%
\pgfsetfillcolor{currentfill}%
\pgfsetlinewidth{0.501875pt}%
\definecolor{currentstroke}{rgb}{0.000000,0.000000,0.000000}%
\pgfsetstrokecolor{currentstroke}%
\pgfsetdash{}{0pt}%
\pgfsys@defobject{currentmarker}{\pgfqpoint{0.000000in}{0.000000in}}{\pgfqpoint{0.000000in}{0.055556in}}{%
\pgfpathmoveto{\pgfqpoint{0.000000in}{0.000000in}}%
\pgfpathlineto{\pgfqpoint{0.000000in}{0.055556in}}%
\pgfusepath{stroke,fill}%
}%
\begin{pgfscope}%
\pgfsys@transformshift{2.949010in}{0.532031in}%
\pgfsys@useobject{currentmarker}{}%
\end{pgfscope}%
\end{pgfscope}%
\begin{pgfscope}%
\pgfsetbuttcap%
\pgfsetroundjoin%
\definecolor{currentfill}{rgb}{0.000000,0.000000,0.000000}%
\pgfsetfillcolor{currentfill}%
\pgfsetlinewidth{0.501875pt}%
\definecolor{currentstroke}{rgb}{0.000000,0.000000,0.000000}%
\pgfsetstrokecolor{currentstroke}%
\pgfsetdash{}{0pt}%
\pgfsys@defobject{currentmarker}{\pgfqpoint{0.000000in}{-0.055556in}}{\pgfqpoint{0.000000in}{0.000000in}}{%
\pgfpathmoveto{\pgfqpoint{0.000000in}{0.000000in}}%
\pgfpathlineto{\pgfqpoint{0.000000in}{-0.055556in}}%
\pgfusepath{stroke,fill}%
}%
\begin{pgfscope}%
\pgfsys@transformshift{2.949010in}{1.753333in}%
\pgfsys@useobject{currentmarker}{}%
\end{pgfscope}%
\end{pgfscope}%
\begin{pgfscope}%
\pgftext[x=2.949010in,y=0.476476in,,top]{\rmfamily\fontsize{10.000000}{12.000000}\selectfont 4}%
\end{pgfscope}%
\begin{pgfscope}%
\pgfsetbuttcap%
\pgfsetroundjoin%
\definecolor{currentfill}{rgb}{0.000000,0.000000,0.000000}%
\pgfsetfillcolor{currentfill}%
\pgfsetlinewidth{0.501875pt}%
\definecolor{currentstroke}{rgb}{0.000000,0.000000,0.000000}%
\pgfsetstrokecolor{currentstroke}%
\pgfsetdash{}{0pt}%
\pgfsys@defobject{currentmarker}{\pgfqpoint{0.000000in}{0.000000in}}{\pgfqpoint{0.000000in}{0.055556in}}{%
\pgfpathmoveto{\pgfqpoint{0.000000in}{0.000000in}}%
\pgfpathlineto{\pgfqpoint{0.000000in}{0.055556in}}%
\pgfusepath{stroke,fill}%
}%
\begin{pgfscope}%
\pgfsys@transformshift{3.518490in}{0.532031in}%
\pgfsys@useobject{currentmarker}{}%
\end{pgfscope}%
\end{pgfscope}%
\begin{pgfscope}%
\pgfsetbuttcap%
\pgfsetroundjoin%
\definecolor{currentfill}{rgb}{0.000000,0.000000,0.000000}%
\pgfsetfillcolor{currentfill}%
\pgfsetlinewidth{0.501875pt}%
\definecolor{currentstroke}{rgb}{0.000000,0.000000,0.000000}%
\pgfsetstrokecolor{currentstroke}%
\pgfsetdash{}{0pt}%
\pgfsys@defobject{currentmarker}{\pgfqpoint{0.000000in}{-0.055556in}}{\pgfqpoint{0.000000in}{0.000000in}}{%
\pgfpathmoveto{\pgfqpoint{0.000000in}{0.000000in}}%
\pgfpathlineto{\pgfqpoint{0.000000in}{-0.055556in}}%
\pgfusepath{stroke,fill}%
}%
\begin{pgfscope}%
\pgfsys@transformshift{3.518490in}{1.753333in}%
\pgfsys@useobject{currentmarker}{}%
\end{pgfscope}%
\end{pgfscope}%
\begin{pgfscope}%
\pgftext[x=3.518490in,y=0.476476in,,top]{\rmfamily\fontsize{10.000000}{12.000000}\selectfont 5}%
\end{pgfscope}%
\begin{pgfscope}%
\pgfsetbuttcap%
\pgfsetroundjoin%
\definecolor{currentfill}{rgb}{0.000000,0.000000,0.000000}%
\pgfsetfillcolor{currentfill}%
\pgfsetlinewidth{0.501875pt}%
\definecolor{currentstroke}{rgb}{0.000000,0.000000,0.000000}%
\pgfsetstrokecolor{currentstroke}%
\pgfsetdash{}{0pt}%
\pgfsys@defobject{currentmarker}{\pgfqpoint{0.000000in}{0.000000in}}{\pgfqpoint{0.000000in}{0.055556in}}{%
\pgfpathmoveto{\pgfqpoint{0.000000in}{0.000000in}}%
\pgfpathlineto{\pgfqpoint{0.000000in}{0.055556in}}%
\pgfusepath{stroke,fill}%
}%
\begin{pgfscope}%
\pgfsys@transformshift{4.087969in}{0.532031in}%
\pgfsys@useobject{currentmarker}{}%
\end{pgfscope}%
\end{pgfscope}%
\begin{pgfscope}%
\pgfsetbuttcap%
\pgfsetroundjoin%
\definecolor{currentfill}{rgb}{0.000000,0.000000,0.000000}%
\pgfsetfillcolor{currentfill}%
\pgfsetlinewidth{0.501875pt}%
\definecolor{currentstroke}{rgb}{0.000000,0.000000,0.000000}%
\pgfsetstrokecolor{currentstroke}%
\pgfsetdash{}{0pt}%
\pgfsys@defobject{currentmarker}{\pgfqpoint{0.000000in}{-0.055556in}}{\pgfqpoint{0.000000in}{0.000000in}}{%
\pgfpathmoveto{\pgfqpoint{0.000000in}{0.000000in}}%
\pgfpathlineto{\pgfqpoint{0.000000in}{-0.055556in}}%
\pgfusepath{stroke,fill}%
}%
\begin{pgfscope}%
\pgfsys@transformshift{4.087969in}{1.753333in}%
\pgfsys@useobject{currentmarker}{}%
\end{pgfscope}%
\end{pgfscope}%
\begin{pgfscope}%
\pgftext[x=4.087969in,y=0.476476in,,top]{\rmfamily\fontsize{10.000000}{12.000000}\selectfont 6}%
\end{pgfscope}%
\begin{pgfscope}%
\pgftext[x=2.379531in,y=0.283575in,,top]{\rmfamily\fontsize{10.000000}{12.000000}\selectfont Pepperpot Extent (mm)}%
\end{pgfscope}%
\begin{pgfscope}%
\pgfsetbuttcap%
\pgfsetroundjoin%
\definecolor{currentfill}{rgb}{0.000000,0.000000,0.000000}%
\pgfsetfillcolor{currentfill}%
\pgfsetlinewidth{0.501875pt}%
\definecolor{currentstroke}{rgb}{0.000000,0.000000,0.000000}%
\pgfsetstrokecolor{currentstroke}%
\pgfsetdash{}{0pt}%
\pgfsys@defobject{currentmarker}{\pgfqpoint{0.000000in}{0.000000in}}{\pgfqpoint{0.055556in}{0.000000in}}{%
\pgfpathmoveto{\pgfqpoint{0.000000in}{0.000000in}}%
\pgfpathlineto{\pgfqpoint{0.055556in}{0.000000in}}%
\pgfusepath{stroke,fill}%
}%
\begin{pgfscope}%
\pgfsys@transformshift{0.671094in}{0.532031in}%
\pgfsys@useobject{currentmarker}{}%
\end{pgfscope}%
\end{pgfscope}%
\begin{pgfscope}%
\pgfsetbuttcap%
\pgfsetroundjoin%
\definecolor{currentfill}{rgb}{0.000000,0.000000,0.000000}%
\pgfsetfillcolor{currentfill}%
\pgfsetlinewidth{0.501875pt}%
\definecolor{currentstroke}{rgb}{0.000000,0.000000,0.000000}%
\pgfsetstrokecolor{currentstroke}%
\pgfsetdash{}{0pt}%
\pgfsys@defobject{currentmarker}{\pgfqpoint{-0.055556in}{0.000000in}}{\pgfqpoint{0.000000in}{0.000000in}}{%
\pgfpathmoveto{\pgfqpoint{0.000000in}{0.000000in}}%
\pgfpathlineto{\pgfqpoint{-0.055556in}{0.000000in}}%
\pgfusepath{stroke,fill}%
}%
\begin{pgfscope}%
\pgfsys@transformshift{4.087969in}{0.532031in}%
\pgfsys@useobject{currentmarker}{}%
\end{pgfscope}%
\end{pgfscope}%
\begin{pgfscope}%
\pgftext[x=0.615538in,y=0.532031in,right,]{\rmfamily\fontsize{10.000000}{12.000000}\selectfont 0}%
\end{pgfscope}%
\begin{pgfscope}%
\pgfsetbuttcap%
\pgfsetroundjoin%
\definecolor{currentfill}{rgb}{0.000000,0.000000,0.000000}%
\pgfsetfillcolor{currentfill}%
\pgfsetlinewidth{0.501875pt}%
\definecolor{currentstroke}{rgb}{0.000000,0.000000,0.000000}%
\pgfsetstrokecolor{currentstroke}%
\pgfsetdash{}{0pt}%
\pgfsys@defobject{currentmarker}{\pgfqpoint{0.000000in}{0.000000in}}{\pgfqpoint{0.055556in}{0.000000in}}{%
\pgfpathmoveto{\pgfqpoint{0.000000in}{0.000000in}}%
\pgfpathlineto{\pgfqpoint{0.055556in}{0.000000in}}%
\pgfusepath{stroke,fill}%
}%
\begin{pgfscope}%
\pgfsys@transformshift{0.671094in}{0.694872in}%
\pgfsys@useobject{currentmarker}{}%
\end{pgfscope}%
\end{pgfscope}%
\begin{pgfscope}%
\pgfsetbuttcap%
\pgfsetroundjoin%
\definecolor{currentfill}{rgb}{0.000000,0.000000,0.000000}%
\pgfsetfillcolor{currentfill}%
\pgfsetlinewidth{0.501875pt}%
\definecolor{currentstroke}{rgb}{0.000000,0.000000,0.000000}%
\pgfsetstrokecolor{currentstroke}%
\pgfsetdash{}{0pt}%
\pgfsys@defobject{currentmarker}{\pgfqpoint{-0.055556in}{0.000000in}}{\pgfqpoint{0.000000in}{0.000000in}}{%
\pgfpathmoveto{\pgfqpoint{0.000000in}{0.000000in}}%
\pgfpathlineto{\pgfqpoint{-0.055556in}{0.000000in}}%
\pgfusepath{stroke,fill}%
}%
\begin{pgfscope}%
\pgfsys@transformshift{4.087969in}{0.694872in}%
\pgfsys@useobject{currentmarker}{}%
\end{pgfscope}%
\end{pgfscope}%
\begin{pgfscope}%
\pgftext[x=0.615538in,y=0.694872in,right,]{\rmfamily\fontsize{10.000000}{12.000000}\selectfont 20}%
\end{pgfscope}%
\begin{pgfscope}%
\pgfsetbuttcap%
\pgfsetroundjoin%
\definecolor{currentfill}{rgb}{0.000000,0.000000,0.000000}%
\pgfsetfillcolor{currentfill}%
\pgfsetlinewidth{0.501875pt}%
\definecolor{currentstroke}{rgb}{0.000000,0.000000,0.000000}%
\pgfsetstrokecolor{currentstroke}%
\pgfsetdash{}{0pt}%
\pgfsys@defobject{currentmarker}{\pgfqpoint{0.000000in}{0.000000in}}{\pgfqpoint{0.055556in}{0.000000in}}{%
\pgfpathmoveto{\pgfqpoint{0.000000in}{0.000000in}}%
\pgfpathlineto{\pgfqpoint{0.055556in}{0.000000in}}%
\pgfusepath{stroke,fill}%
}%
\begin{pgfscope}%
\pgfsys@transformshift{0.671094in}{0.857712in}%
\pgfsys@useobject{currentmarker}{}%
\end{pgfscope}%
\end{pgfscope}%
\begin{pgfscope}%
\pgfsetbuttcap%
\pgfsetroundjoin%
\definecolor{currentfill}{rgb}{0.000000,0.000000,0.000000}%
\pgfsetfillcolor{currentfill}%
\pgfsetlinewidth{0.501875pt}%
\definecolor{currentstroke}{rgb}{0.000000,0.000000,0.000000}%
\pgfsetstrokecolor{currentstroke}%
\pgfsetdash{}{0pt}%
\pgfsys@defobject{currentmarker}{\pgfqpoint{-0.055556in}{0.000000in}}{\pgfqpoint{0.000000in}{0.000000in}}{%
\pgfpathmoveto{\pgfqpoint{0.000000in}{0.000000in}}%
\pgfpathlineto{\pgfqpoint{-0.055556in}{0.000000in}}%
\pgfusepath{stroke,fill}%
}%
\begin{pgfscope}%
\pgfsys@transformshift{4.087969in}{0.857712in}%
\pgfsys@useobject{currentmarker}{}%
\end{pgfscope}%
\end{pgfscope}%
\begin{pgfscope}%
\pgftext[x=0.615538in,y=0.857712in,right,]{\rmfamily\fontsize{10.000000}{12.000000}\selectfont 40}%
\end{pgfscope}%
\begin{pgfscope}%
\pgfsetbuttcap%
\pgfsetroundjoin%
\definecolor{currentfill}{rgb}{0.000000,0.000000,0.000000}%
\pgfsetfillcolor{currentfill}%
\pgfsetlinewidth{0.501875pt}%
\definecolor{currentstroke}{rgb}{0.000000,0.000000,0.000000}%
\pgfsetstrokecolor{currentstroke}%
\pgfsetdash{}{0pt}%
\pgfsys@defobject{currentmarker}{\pgfqpoint{0.000000in}{0.000000in}}{\pgfqpoint{0.055556in}{0.000000in}}{%
\pgfpathmoveto{\pgfqpoint{0.000000in}{0.000000in}}%
\pgfpathlineto{\pgfqpoint{0.055556in}{0.000000in}}%
\pgfusepath{stroke,fill}%
}%
\begin{pgfscope}%
\pgfsys@transformshift{0.671094in}{1.020552in}%
\pgfsys@useobject{currentmarker}{}%
\end{pgfscope}%
\end{pgfscope}%
\begin{pgfscope}%
\pgfsetbuttcap%
\pgfsetroundjoin%
\definecolor{currentfill}{rgb}{0.000000,0.000000,0.000000}%
\pgfsetfillcolor{currentfill}%
\pgfsetlinewidth{0.501875pt}%
\definecolor{currentstroke}{rgb}{0.000000,0.000000,0.000000}%
\pgfsetstrokecolor{currentstroke}%
\pgfsetdash{}{0pt}%
\pgfsys@defobject{currentmarker}{\pgfqpoint{-0.055556in}{0.000000in}}{\pgfqpoint{0.000000in}{0.000000in}}{%
\pgfpathmoveto{\pgfqpoint{0.000000in}{0.000000in}}%
\pgfpathlineto{\pgfqpoint{-0.055556in}{0.000000in}}%
\pgfusepath{stroke,fill}%
}%
\begin{pgfscope}%
\pgfsys@transformshift{4.087969in}{1.020552in}%
\pgfsys@useobject{currentmarker}{}%
\end{pgfscope}%
\end{pgfscope}%
\begin{pgfscope}%
\pgftext[x=0.615538in,y=1.020552in,right,]{\rmfamily\fontsize{10.000000}{12.000000}\selectfont 60}%
\end{pgfscope}%
\begin{pgfscope}%
\pgfsetbuttcap%
\pgfsetroundjoin%
\definecolor{currentfill}{rgb}{0.000000,0.000000,0.000000}%
\pgfsetfillcolor{currentfill}%
\pgfsetlinewidth{0.501875pt}%
\definecolor{currentstroke}{rgb}{0.000000,0.000000,0.000000}%
\pgfsetstrokecolor{currentstroke}%
\pgfsetdash{}{0pt}%
\pgfsys@defobject{currentmarker}{\pgfqpoint{0.000000in}{0.000000in}}{\pgfqpoint{0.055556in}{0.000000in}}{%
\pgfpathmoveto{\pgfqpoint{0.000000in}{0.000000in}}%
\pgfpathlineto{\pgfqpoint{0.055556in}{0.000000in}}%
\pgfusepath{stroke,fill}%
}%
\begin{pgfscope}%
\pgfsys@transformshift{0.671094in}{1.183392in}%
\pgfsys@useobject{currentmarker}{}%
\end{pgfscope}%
\end{pgfscope}%
\begin{pgfscope}%
\pgfsetbuttcap%
\pgfsetroundjoin%
\definecolor{currentfill}{rgb}{0.000000,0.000000,0.000000}%
\pgfsetfillcolor{currentfill}%
\pgfsetlinewidth{0.501875pt}%
\definecolor{currentstroke}{rgb}{0.000000,0.000000,0.000000}%
\pgfsetstrokecolor{currentstroke}%
\pgfsetdash{}{0pt}%
\pgfsys@defobject{currentmarker}{\pgfqpoint{-0.055556in}{0.000000in}}{\pgfqpoint{0.000000in}{0.000000in}}{%
\pgfpathmoveto{\pgfqpoint{0.000000in}{0.000000in}}%
\pgfpathlineto{\pgfqpoint{-0.055556in}{0.000000in}}%
\pgfusepath{stroke,fill}%
}%
\begin{pgfscope}%
\pgfsys@transformshift{4.087969in}{1.183392in}%
\pgfsys@useobject{currentmarker}{}%
\end{pgfscope}%
\end{pgfscope}%
\begin{pgfscope}%
\pgftext[x=0.615538in,y=1.183392in,right,]{\rmfamily\fontsize{10.000000}{12.000000}\selectfont 80}%
\end{pgfscope}%
\begin{pgfscope}%
\pgfsetbuttcap%
\pgfsetroundjoin%
\definecolor{currentfill}{rgb}{0.000000,0.000000,0.000000}%
\pgfsetfillcolor{currentfill}%
\pgfsetlinewidth{0.501875pt}%
\definecolor{currentstroke}{rgb}{0.000000,0.000000,0.000000}%
\pgfsetstrokecolor{currentstroke}%
\pgfsetdash{}{0pt}%
\pgfsys@defobject{currentmarker}{\pgfqpoint{0.000000in}{0.000000in}}{\pgfqpoint{0.055556in}{0.000000in}}{%
\pgfpathmoveto{\pgfqpoint{0.000000in}{0.000000in}}%
\pgfpathlineto{\pgfqpoint{0.055556in}{0.000000in}}%
\pgfusepath{stroke,fill}%
}%
\begin{pgfscope}%
\pgfsys@transformshift{0.671094in}{1.346233in}%
\pgfsys@useobject{currentmarker}{}%
\end{pgfscope}%
\end{pgfscope}%
\begin{pgfscope}%
\pgfsetbuttcap%
\pgfsetroundjoin%
\definecolor{currentfill}{rgb}{0.000000,0.000000,0.000000}%
\pgfsetfillcolor{currentfill}%
\pgfsetlinewidth{0.501875pt}%
\definecolor{currentstroke}{rgb}{0.000000,0.000000,0.000000}%
\pgfsetstrokecolor{currentstroke}%
\pgfsetdash{}{0pt}%
\pgfsys@defobject{currentmarker}{\pgfqpoint{-0.055556in}{0.000000in}}{\pgfqpoint{0.000000in}{0.000000in}}{%
\pgfpathmoveto{\pgfqpoint{0.000000in}{0.000000in}}%
\pgfpathlineto{\pgfqpoint{-0.055556in}{0.000000in}}%
\pgfusepath{stroke,fill}%
}%
\begin{pgfscope}%
\pgfsys@transformshift{4.087969in}{1.346233in}%
\pgfsys@useobject{currentmarker}{}%
\end{pgfscope}%
\end{pgfscope}%
\begin{pgfscope}%
\pgftext[x=0.615538in,y=1.346233in,right,]{\rmfamily\fontsize{10.000000}{12.000000}\selectfont 100}%
\end{pgfscope}%
\begin{pgfscope}%
\pgfsetbuttcap%
\pgfsetroundjoin%
\definecolor{currentfill}{rgb}{0.000000,0.000000,0.000000}%
\pgfsetfillcolor{currentfill}%
\pgfsetlinewidth{0.501875pt}%
\definecolor{currentstroke}{rgb}{0.000000,0.000000,0.000000}%
\pgfsetstrokecolor{currentstroke}%
\pgfsetdash{}{0pt}%
\pgfsys@defobject{currentmarker}{\pgfqpoint{0.000000in}{0.000000in}}{\pgfqpoint{0.055556in}{0.000000in}}{%
\pgfpathmoveto{\pgfqpoint{0.000000in}{0.000000in}}%
\pgfpathlineto{\pgfqpoint{0.055556in}{0.000000in}}%
\pgfusepath{stroke,fill}%
}%
\begin{pgfscope}%
\pgfsys@transformshift{0.671094in}{1.509073in}%
\pgfsys@useobject{currentmarker}{}%
\end{pgfscope}%
\end{pgfscope}%
\begin{pgfscope}%
\pgfsetbuttcap%
\pgfsetroundjoin%
\definecolor{currentfill}{rgb}{0.000000,0.000000,0.000000}%
\pgfsetfillcolor{currentfill}%
\pgfsetlinewidth{0.501875pt}%
\definecolor{currentstroke}{rgb}{0.000000,0.000000,0.000000}%
\pgfsetstrokecolor{currentstroke}%
\pgfsetdash{}{0pt}%
\pgfsys@defobject{currentmarker}{\pgfqpoint{-0.055556in}{0.000000in}}{\pgfqpoint{0.000000in}{0.000000in}}{%
\pgfpathmoveto{\pgfqpoint{0.000000in}{0.000000in}}%
\pgfpathlineto{\pgfqpoint{-0.055556in}{0.000000in}}%
\pgfusepath{stroke,fill}%
}%
\begin{pgfscope}%
\pgfsys@transformshift{4.087969in}{1.509073in}%
\pgfsys@useobject{currentmarker}{}%
\end{pgfscope}%
\end{pgfscope}%
\begin{pgfscope}%
\pgftext[x=0.615538in,y=1.509073in,right,]{\rmfamily\fontsize{10.000000}{12.000000}\selectfont 120}%
\end{pgfscope}%
\begin{pgfscope}%
\pgfsetbuttcap%
\pgfsetroundjoin%
\definecolor{currentfill}{rgb}{0.000000,0.000000,0.000000}%
\pgfsetfillcolor{currentfill}%
\pgfsetlinewidth{0.501875pt}%
\definecolor{currentstroke}{rgb}{0.000000,0.000000,0.000000}%
\pgfsetstrokecolor{currentstroke}%
\pgfsetdash{}{0pt}%
\pgfsys@defobject{currentmarker}{\pgfqpoint{0.000000in}{0.000000in}}{\pgfqpoint{0.055556in}{0.000000in}}{%
\pgfpathmoveto{\pgfqpoint{0.000000in}{0.000000in}}%
\pgfpathlineto{\pgfqpoint{0.055556in}{0.000000in}}%
\pgfusepath{stroke,fill}%
}%
\begin{pgfscope}%
\pgfsys@transformshift{0.671094in}{1.671913in}%
\pgfsys@useobject{currentmarker}{}%
\end{pgfscope}%
\end{pgfscope}%
\begin{pgfscope}%
\pgfsetbuttcap%
\pgfsetroundjoin%
\definecolor{currentfill}{rgb}{0.000000,0.000000,0.000000}%
\pgfsetfillcolor{currentfill}%
\pgfsetlinewidth{0.501875pt}%
\definecolor{currentstroke}{rgb}{0.000000,0.000000,0.000000}%
\pgfsetstrokecolor{currentstroke}%
\pgfsetdash{}{0pt}%
\pgfsys@defobject{currentmarker}{\pgfqpoint{-0.055556in}{0.000000in}}{\pgfqpoint{0.000000in}{0.000000in}}{%
\pgfpathmoveto{\pgfqpoint{0.000000in}{0.000000in}}%
\pgfpathlineto{\pgfqpoint{-0.055556in}{0.000000in}}%
\pgfusepath{stroke,fill}%
}%
\begin{pgfscope}%
\pgfsys@transformshift{4.087969in}{1.671913in}%
\pgfsys@useobject{currentmarker}{}%
\end{pgfscope}%
\end{pgfscope}%
\begin{pgfscope}%
\pgftext[x=0.615538in,y=1.671913in,right,]{\rmfamily\fontsize{10.000000}{12.000000}\selectfont 140}%
\end{pgfscope}%
\begin{pgfscope}%
\pgftext[x=0.337760in,y=1.142682in,,bottom,rotate=90.000000]{\rmfamily\fontsize{10.000000}{12.000000}\selectfont Emittance (nm rad)}%
\end{pgfscope}%
\end{pgfpicture}%
\makeatother%
\endgroup%

    \caption[Results of a simulation investigating the efficacy of the beam coverage correction for various beam sizes.]{The results of analysis of simulations of the apparatus while varying the extent of the pepperpot mask compared to the size of the bunch at the pepperpot plane.
    The red dotted line indicates the true emittance of the bunch, the blue line indicates the measured emittance of the bunch without the beam coverage correction and the green line is the measured emittance with the correction.
    The dotted black line indicates the \gls{rms} width of the beam at the pepperpot plane (\unit[873]{$\muup$m}).}
    \label{figure:pepperpot_extent}
    % Data and code located in Code/Electrons/Simulation/1D Simulation SimScript1D.py
\end{figure}

\section{Results}
In order to demonstrate the feasibility of time-resolved brightness measurements a number of measurements have been made including a comparison of two-dimensional pepperpot emittance measurements with theory and simulation, and one-dimensional pepperpot streaked measurements with long and short duration electron bunches.

\subsection{Analysis}
The analysis procedure developed was based on the theory described in Reference~\cite{zhang_emittance_1996} and Equation~\ref{equation:pepperpot}.
In order to determine the emittance from a one dimensional pepperpot measurement a number of parameters are required:
\begin{itemize}
    \item Size of pepperpot apertures
    \item Position of pepperpot apertures
    \item Size of beamlets on the detector
    \item Position of beamlets on the detector
    \item Total number of electrons in each beamlet
\end{itemize}
The pepperpot parameters are known from the fabrication and the beamlet parameters can be determined from appropriately prepared two-dimensional and streaked one-dimensional pepperpot measurements.
The analysis procedure is roughly as follows:
\begin{enumerate}
    \item Acquire images from camera
    \item Register images (see Section~\ref{section:registration})
    \item Average registered images
    \item Rotate image so dominant features are aligned with the pixel rows
    \item Deskew image so features along the second axis align with the columns
    \begin{itemize}
        \item For two dimensional pepperpots sum the rows and columns to produce two one-dimensional sets of beamlets
        \item For streaked emittance measurements examine each column of pixels independently to produce a number of one-dimensional sets of beamlets.
    \end{itemize}
    \item Separate each beamlet set into individual beamlets
    \item Determine the approximate amplitude, mean position and \gls{rms} width of each beamlet
    \item Refine the amplitude, mean positions and \gls{rms} width of each beamlet by fitting a number of Gaussians equal to the number of beamlets to the beamlet set to account for any overlap
    \item Transform the beamlet position and width measurements from pixel measurement to detector-plane measurements using the camera and \gls{mcp} calibration
    \item Fit a Gaussian convolved with the aperture size to the refined beamlet parameters in order to determine the equivalent beamlet for a infinitesimal aperture
    \item Use Equation~\ref{equation:pepperpot} to determine the emittance
    \item Estimate the full-beam profile of the bunch by fitting a Gaussian to the number of particles transmitted through each aperture (equal to the total particle count for each beamlet)
    \item Correct the emittance by considering the beam coverage of the pepperpot mask given the full-beam profile
    \item Use the beam profile to determine the beam current
    \item Use the beam current and emittance to determine the brightness
\end{enumerate}


\subsection{Calibrations}
A number of experimental parameters require calibration and the procedures for doing so are briefly outlined here.

\subsubsection{Detector-Camera Calibration}
The detector for the \gls{caeis} consists of a phosphor-coupled \gls{mcp} observed with a \gls{ccd} camera.
In order to relate distance in the images, measured in pixels, to distances on the phosphor screen, measured in meters, it is necessary to calibrate the camera.
This is easily achieved by placing a reliable ruler alongside the phosphor screen and taking an image which the number of pixels corresponding to a distance on the ruler, and thus the phosphor screen, to be determined.
The calibration was found to be \unit[24.5]{pixels per mm}.

\subsubsection{Electron-\gls{mcp}-Camera Calibration}
The number of camera counts generated by a single electron is another required calibration.
This is achieved by focusing a beam on to the Faraday cup that can be inserted into the beamline and measuring the number of electrons per bunch followed by removing the cup, unfocusing the beam and recording an image.
The total number of counts on the image can then be compared to the known number of electrons in the bunch.
To reduce the uncertainty in this measurement the number of electrons can be varied by adjusting the power of the red excitation laser.
The calibration was found to be 34.3 counts per electron.

This process was performed with the \gls{mcp} voltage at \unit[1.7]{kV} and the phosphor voltage at \unit[4.0]{kV} and all the measurement shown in this chapter use those parameters.

\subsubsection{Source Size}
Another required parameter is the size of the source which is required to determine the expected emittance using Equation~\ref{equation:excess_energy_emittance}.
The transverse source size was the same size as the excitation laser as it passed through the \gls{mot}.
This was determined by intercepting the excitation laser with a mirror and directing the beam onto a camera placed the same distance away from the intercepting mirror as the centre of the chamber in which the \gls{mot} was located.
Knowing the size of pixels for the camera allowed for the spatial profile of the beam to be determined.
The source size was determined to be \unit[450.8]{$\muup$m} in the vertical direction and \unit[690.6]{$\muup$m} in the horizontal direction.

However by fitting Equation~\ref{equation:excess_energy_emittance} to the data shown in Figure~\ref{figure:emittance_vs_theory} it was apparent that the source size was closer to \unit[340]{$\muup$m} indicating that the \gls{mot} is no longer located in the centre of the chamber and thus that the position at which the laser beam profile was measured was not comparable.
Slow changes in the position of the \gls{mot} were likely due to alignment drift given the age of the apparatus.

\subsection{Two-Dimensional Pepperpots}
The frequency of the ionisation laser could be tuned to control the emittance of electron bunches as it determined the excess energy of bunches.
Varying the wavelength of the ionisation laser and measuring the emittance of the electron bunches using the two dimensional pepperpot allowed for testing of the emittance measurement system and analysis as the results could be compared to the theory described by Equation~\ref{equation:excess_energy_emittance} and the simulations in Section~\ref{section:pepperpot_simulations}.

The results from this test can be seen in Figure~\ref{figure:emittance_vs_theory} and the results are well matched to the theory and previous measurements of the emittance of this \gls{caes}~\cite{mcculloch_high-coherence_2013}.
The data are taken from 100-shot registered averages and agrees with the simulation results.
The lower bound on measurable emittance is apparent in both the measurement and the simulation results.
Due to the aperture size of \unit[50]{$\muup$m} and beam path geometry (propagation distances, focusing parameter, etc.) the measurements were limited to resolving emittance greater than \unit[41]{nm rad}.

\begin{figure}
    \center
    %% Creator: Matplotlib, PGF backend
%%
%% To include the figure in your LaTeX document, write
%%   \input{<filename>.pgf}
%%
%% Make sure the required packages are loaded in your preamble
%%   \usepackage{pgf}
%%
%% Figures using additional raster images can only be included by \input if
%% they are in the same directory as the main LaTeX file. For loading figures
%% from other directories you can use the `import` package
%%   \usepackage{import}
%% and then include the figures with
%%   \import{<path to file>}{<filename>.pgf}
%%
%% Matplotlib used the following preamble
%%
\begingroup%
\makeatletter%
\begin{pgfpicture}%
\pgfpathrectangle{\pgfpointorigin}{\pgfqpoint{5.424500in}{2.603760in}}%
\pgfusepath{use as bounding box, clip}%
\begin{pgfscope}%
\pgfsetbuttcap%
\pgfsetmiterjoin%
\definecolor{currentfill}{rgb}{1.000000,1.000000,1.000000}%
\pgfsetfillcolor{currentfill}%
\pgfsetlinewidth{0.000000pt}%
\definecolor{currentstroke}{rgb}{1.000000,1.000000,1.000000}%
\pgfsetstrokecolor{currentstroke}%
\pgfsetdash{}{0pt}%
\pgfpathmoveto{\pgfqpoint{0.000000in}{0.000000in}}%
\pgfpathlineto{\pgfqpoint{5.424500in}{0.000000in}}%
\pgfpathlineto{\pgfqpoint{5.424500in}{2.603760in}}%
\pgfpathlineto{\pgfqpoint{0.000000in}{2.603760in}}%
\pgfpathclose%
\pgfusepath{fill}%
\end{pgfscope}%
\begin{pgfscope}%
\pgfsetbuttcap%
\pgfsetmiterjoin%
\definecolor{currentfill}{rgb}{1.000000,1.000000,1.000000}%
\pgfsetfillcolor{currentfill}%
\pgfsetlinewidth{0.000000pt}%
\definecolor{currentstroke}{rgb}{0.000000,0.000000,0.000000}%
\pgfsetstrokecolor{currentstroke}%
\pgfsetstrokeopacity{0.000000}%
\pgfsetdash{}{0pt}%
\pgfpathmoveto{\pgfqpoint{0.671094in}{0.532031in}}%
\pgfpathlineto{\pgfqpoint{5.185438in}{0.532031in}}%
\pgfpathlineto{\pgfqpoint{5.185438in}{2.453760in}}%
\pgfpathlineto{\pgfqpoint{0.671094in}{2.453760in}}%
\pgfpathclose%
\pgfusepath{fill}%
\end{pgfscope}%
\begin{pgfscope}%
\pgfpathrectangle{\pgfqpoint{0.671094in}{0.532031in}}{\pgfqpoint{4.514344in}{1.921729in}} %
\pgfusepath{clip}%
\pgfsetbuttcap%
\pgfsetroundjoin%
\pgfsetlinewidth{1.003750pt}%
\definecolor{currentstroke}{rgb}{0.000000,0.000000,1.000000}%
\pgfsetstrokecolor{currentstroke}%
\pgfsetdash{}{0pt}%
\pgfpathmoveto{\pgfqpoint{4.953510in}{1.660023in}}%
\pgfpathlineto{\pgfqpoint{4.953510in}{2.085921in}}%
\pgfusepath{stroke}%
\end{pgfscope}%
\begin{pgfscope}%
\pgfpathrectangle{\pgfqpoint{0.671094in}{0.532031in}}{\pgfqpoint{4.514344in}{1.921729in}} %
\pgfusepath{clip}%
\pgfsetbuttcap%
\pgfsetroundjoin%
\pgfsetlinewidth{1.003750pt}%
\definecolor{currentstroke}{rgb}{0.000000,0.000000,1.000000}%
\pgfsetstrokecolor{currentstroke}%
\pgfsetdash{}{0pt}%
\pgfpathmoveto{\pgfqpoint{4.751996in}{1.290897in}}%
\pgfpathlineto{\pgfqpoint{4.751996in}{2.438584in}}%
\pgfusepath{stroke}%
\end{pgfscope}%
\begin{pgfscope}%
\pgfpathrectangle{\pgfqpoint{0.671094in}{0.532031in}}{\pgfqpoint{4.514344in}{1.921729in}} %
\pgfusepath{clip}%
\pgfsetbuttcap%
\pgfsetroundjoin%
\pgfsetlinewidth{1.003750pt}%
\definecolor{currentstroke}{rgb}{0.000000,0.000000,1.000000}%
\pgfsetstrokecolor{currentstroke}%
\pgfsetdash{}{0pt}%
\pgfpathmoveto{\pgfqpoint{4.551752in}{1.477983in}}%
\pgfpathlineto{\pgfqpoint{4.551752in}{1.871020in}}%
\pgfusepath{stroke}%
\end{pgfscope}%
\begin{pgfscope}%
\pgfpathrectangle{\pgfqpoint{0.671094in}{0.532031in}}{\pgfqpoint{4.514344in}{1.921729in}} %
\pgfusepath{clip}%
\pgfsetbuttcap%
\pgfsetroundjoin%
\pgfsetlinewidth{1.003750pt}%
\definecolor{currentstroke}{rgb}{0.000000,0.000000,1.000000}%
\pgfsetstrokecolor{currentstroke}%
\pgfsetdash{}{0pt}%
\pgfpathmoveto{\pgfqpoint{4.352768in}{1.351668in}}%
\pgfpathlineto{\pgfqpoint{4.352768in}{1.821336in}}%
\pgfusepath{stroke}%
\end{pgfscope}%
\begin{pgfscope}%
\pgfpathrectangle{\pgfqpoint{0.671094in}{0.532031in}}{\pgfqpoint{4.514344in}{1.921729in}} %
\pgfusepath{clip}%
\pgfsetbuttcap%
\pgfsetroundjoin%
\pgfsetlinewidth{1.003750pt}%
\definecolor{currentstroke}{rgb}{0.000000,0.000000,1.000000}%
\pgfsetstrokecolor{currentstroke}%
\pgfsetdash{}{0pt}%
\pgfpathmoveto{\pgfqpoint{4.155422in}{1.616477in}}%
\pgfpathlineto{\pgfqpoint{4.155422in}{1.668251in}}%
\pgfusepath{stroke}%
\end{pgfscope}%
\begin{pgfscope}%
\pgfpathrectangle{\pgfqpoint{0.671094in}{0.532031in}}{\pgfqpoint{4.514344in}{1.921729in}} %
\pgfusepath{clip}%
\pgfsetbuttcap%
\pgfsetroundjoin%
\pgfsetlinewidth{1.003750pt}%
\definecolor{currentstroke}{rgb}{0.000000,0.000000,1.000000}%
\pgfsetstrokecolor{currentstroke}%
\pgfsetdash{}{0pt}%
\pgfpathmoveto{\pgfqpoint{3.959117in}{1.460331in}}%
\pgfpathlineto{\pgfqpoint{3.959117in}{1.621405in}}%
\pgfusepath{stroke}%
\end{pgfscope}%
\begin{pgfscope}%
\pgfpathrectangle{\pgfqpoint{0.671094in}{0.532031in}}{\pgfqpoint{4.514344in}{1.921729in}} %
\pgfusepath{clip}%
\pgfsetbuttcap%
\pgfsetroundjoin%
\pgfsetlinewidth{1.003750pt}%
\definecolor{currentstroke}{rgb}{0.000000,0.000000,1.000000}%
\pgfsetstrokecolor{currentstroke}%
\pgfsetdash{}{0pt}%
\pgfpathmoveto{\pgfqpoint{3.764232in}{1.273912in}}%
\pgfpathlineto{\pgfqpoint{3.764232in}{1.566350in}}%
\pgfusepath{stroke}%
\end{pgfscope}%
\begin{pgfscope}%
\pgfpathrectangle{\pgfqpoint{0.671094in}{0.532031in}}{\pgfqpoint{4.514344in}{1.921729in}} %
\pgfusepath{clip}%
\pgfsetbuttcap%
\pgfsetroundjoin%
\pgfsetlinewidth{1.003750pt}%
\definecolor{currentstroke}{rgb}{0.000000,0.000000,1.000000}%
\pgfsetstrokecolor{currentstroke}%
\pgfsetdash{}{0pt}%
\pgfpathmoveto{\pgfqpoint{3.570368in}{1.322231in}}%
\pgfpathlineto{\pgfqpoint{3.570368in}{1.502017in}}%
\pgfusepath{stroke}%
\end{pgfscope}%
\begin{pgfscope}%
\pgfpathrectangle{\pgfqpoint{0.671094in}{0.532031in}}{\pgfqpoint{4.514344in}{1.921729in}} %
\pgfusepath{clip}%
\pgfsetbuttcap%
\pgfsetroundjoin%
\pgfsetlinewidth{1.003750pt}%
\definecolor{currentstroke}{rgb}{0.000000,0.000000,1.000000}%
\pgfsetstrokecolor{currentstroke}%
\pgfsetdash{}{0pt}%
\pgfpathmoveto{\pgfqpoint{3.378859in}{0.932726in}}%
\pgfpathlineto{\pgfqpoint{3.378859in}{1.494042in}}%
\pgfusepath{stroke}%
\end{pgfscope}%
\begin{pgfscope}%
\pgfpathrectangle{\pgfqpoint{0.671094in}{0.532031in}}{\pgfqpoint{4.514344in}{1.921729in}} %
\pgfusepath{clip}%
\pgfsetbuttcap%
\pgfsetroundjoin%
\pgfsetlinewidth{1.003750pt}%
\definecolor{currentstroke}{rgb}{0.000000,0.000000,1.000000}%
\pgfsetstrokecolor{currentstroke}%
\pgfsetdash{}{0pt}%
\pgfpathmoveto{\pgfqpoint{3.187582in}{1.141421in}}%
\pgfpathlineto{\pgfqpoint{3.187582in}{1.342643in}}%
\pgfusepath{stroke}%
\end{pgfscope}%
\begin{pgfscope}%
\pgfpathrectangle{\pgfqpoint{0.671094in}{0.532031in}}{\pgfqpoint{4.514344in}{1.921729in}} %
\pgfusepath{clip}%
\pgfsetbuttcap%
\pgfsetroundjoin%
\pgfsetlinewidth{1.003750pt}%
\definecolor{currentstroke}{rgb}{0.000000,0.000000,1.000000}%
\pgfsetstrokecolor{currentstroke}%
\pgfsetdash{}{0pt}%
\pgfpathmoveto{\pgfqpoint{2.997489in}{1.221413in}}%
\pgfpathlineto{\pgfqpoint{2.997489in}{1.232215in}}%
\pgfusepath{stroke}%
\end{pgfscope}%
\begin{pgfscope}%
\pgfpathrectangle{\pgfqpoint{0.671094in}{0.532031in}}{\pgfqpoint{4.514344in}{1.921729in}} %
\pgfusepath{clip}%
\pgfsetbuttcap%
\pgfsetroundjoin%
\pgfsetlinewidth{1.003750pt}%
\definecolor{currentstroke}{rgb}{0.000000,0.000000,1.000000}%
\pgfsetstrokecolor{currentstroke}%
\pgfsetdash{}{0pt}%
\pgfpathmoveto{\pgfqpoint{2.808570in}{1.229383in}}%
\pgfpathlineto{\pgfqpoint{2.808570in}{1.265896in}}%
\pgfusepath{stroke}%
\end{pgfscope}%
\begin{pgfscope}%
\pgfpathrectangle{\pgfqpoint{0.671094in}{0.532031in}}{\pgfqpoint{4.514344in}{1.921729in}} %
\pgfusepath{clip}%
\pgfsetbuttcap%
\pgfsetroundjoin%
\pgfsetlinewidth{1.003750pt}%
\definecolor{currentstroke}{rgb}{0.000000,0.000000,1.000000}%
\pgfsetstrokecolor{currentstroke}%
\pgfsetdash{}{0pt}%
\pgfpathmoveto{\pgfqpoint{2.620816in}{0.807523in}}%
\pgfpathlineto{\pgfqpoint{2.620816in}{1.190820in}}%
\pgfusepath{stroke}%
\end{pgfscope}%
\begin{pgfscope}%
\pgfpathrectangle{\pgfqpoint{0.671094in}{0.532031in}}{\pgfqpoint{4.514344in}{1.921729in}} %
\pgfusepath{clip}%
\pgfsetbuttcap%
\pgfsetroundjoin%
\pgfsetlinewidth{1.003750pt}%
\definecolor{currentstroke}{rgb}{0.000000,0.000000,1.000000}%
\pgfsetstrokecolor{currentstroke}%
\pgfsetdash{}{0pt}%
\pgfpathmoveto{\pgfqpoint{2.434404in}{1.126553in}}%
\pgfpathlineto{\pgfqpoint{2.434404in}{1.153984in}}%
\pgfusepath{stroke}%
\end{pgfscope}%
\begin{pgfscope}%
\pgfpathrectangle{\pgfqpoint{0.671094in}{0.532031in}}{\pgfqpoint{4.514344in}{1.921729in}} %
\pgfusepath{clip}%
\pgfsetbuttcap%
\pgfsetroundjoin%
\pgfsetlinewidth{1.003750pt}%
\definecolor{currentstroke}{rgb}{0.000000,0.000000,1.000000}%
\pgfsetstrokecolor{currentstroke}%
\pgfsetdash{}{0pt}%
\pgfpathmoveto{\pgfqpoint{2.249135in}{1.064156in}}%
\pgfpathlineto{\pgfqpoint{2.249135in}{1.257583in}}%
\pgfusepath{stroke}%
\end{pgfscope}%
\begin{pgfscope}%
\pgfpathrectangle{\pgfqpoint{0.671094in}{0.532031in}}{\pgfqpoint{4.514344in}{1.921729in}} %
\pgfusepath{clip}%
\pgfsetbuttcap%
\pgfsetroundjoin%
\pgfsetlinewidth{1.003750pt}%
\definecolor{currentstroke}{rgb}{0.000000,0.000000,1.000000}%
\pgfsetstrokecolor{currentstroke}%
\pgfsetdash{}{0pt}%
\pgfpathmoveto{\pgfqpoint{2.064815in}{1.093212in}}%
\pgfpathlineto{\pgfqpoint{2.064815in}{1.103503in}}%
\pgfusepath{stroke}%
\end{pgfscope}%
\begin{pgfscope}%
\pgfpathrectangle{\pgfqpoint{0.671094in}{0.532031in}}{\pgfqpoint{4.514344in}{1.921729in}} %
\pgfusepath{clip}%
\pgfsetbuttcap%
\pgfsetroundjoin%
\pgfsetlinewidth{1.003750pt}%
\definecolor{currentstroke}{rgb}{0.000000,0.000000,1.000000}%
\pgfsetstrokecolor{currentstroke}%
\pgfsetdash{}{0pt}%
\pgfpathmoveto{\pgfqpoint{1.881622in}{1.033071in}}%
\pgfpathlineto{\pgfqpoint{1.881622in}{1.212178in}}%
\pgfusepath{stroke}%
\end{pgfscope}%
\begin{pgfscope}%
\pgfpathrectangle{\pgfqpoint{0.671094in}{0.532031in}}{\pgfqpoint{4.514344in}{1.921729in}} %
\pgfusepath{clip}%
\pgfsetbuttcap%
\pgfsetroundjoin%
\pgfsetlinewidth{1.003750pt}%
\definecolor{currentstroke}{rgb}{0.000000,0.000000,1.000000}%
\pgfsetstrokecolor{currentstroke}%
\pgfsetdash{}{0pt}%
\pgfpathmoveto{\pgfqpoint{1.699731in}{1.064366in}}%
\pgfpathlineto{\pgfqpoint{1.699731in}{1.091170in}}%
\pgfusepath{stroke}%
\end{pgfscope}%
\begin{pgfscope}%
\pgfpathrectangle{\pgfqpoint{0.671094in}{0.532031in}}{\pgfqpoint{4.514344in}{1.921729in}} %
\pgfusepath{clip}%
\pgfsetbuttcap%
\pgfsetroundjoin%
\pgfsetlinewidth{1.003750pt}%
\definecolor{currentstroke}{rgb}{0.000000,0.000000,1.000000}%
\pgfsetstrokecolor{currentstroke}%
\pgfsetdash{}{0pt}%
\pgfpathmoveto{\pgfqpoint{1.518578in}{0.982975in}}%
\pgfpathlineto{\pgfqpoint{1.518578in}{1.130408in}}%
\pgfusepath{stroke}%
\end{pgfscope}%
\begin{pgfscope}%
\pgfpathrectangle{\pgfqpoint{0.671094in}{0.532031in}}{\pgfqpoint{4.514344in}{1.921729in}} %
\pgfusepath{clip}%
\pgfsetbuttcap%
\pgfsetroundjoin%
\pgfsetlinewidth{1.003750pt}%
\definecolor{currentstroke}{rgb}{0.000000,0.000000,1.000000}%
\pgfsetstrokecolor{currentstroke}%
\pgfsetdash{}{0pt}%
\pgfpathmoveto{\pgfqpoint{1.338710in}{0.896140in}}%
\pgfpathlineto{\pgfqpoint{1.338710in}{1.163839in}}%
\pgfusepath{stroke}%
\end{pgfscope}%
\begin{pgfscope}%
\pgfpathrectangle{\pgfqpoint{0.671094in}{0.532031in}}{\pgfqpoint{4.514344in}{1.921729in}} %
\pgfusepath{clip}%
\pgfsetbuttcap%
\pgfsetroundjoin%
\pgfsetlinewidth{1.003750pt}%
\definecolor{currentstroke}{rgb}{0.000000,0.000000,1.000000}%
\pgfsetstrokecolor{currentstroke}%
\pgfsetdash{}{0pt}%
\pgfpathmoveto{\pgfqpoint{1.163946in}{1.005599in}}%
\pgfpathlineto{\pgfqpoint{1.163946in}{1.054500in}}%
\pgfusepath{stroke}%
\end{pgfscope}%
\begin{pgfscope}%
\pgfpathrectangle{\pgfqpoint{0.671094in}{0.532031in}}{\pgfqpoint{4.514344in}{1.921729in}} %
\pgfusepath{clip}%
\pgfsetbuttcap%
\pgfsetroundjoin%
\pgfsetlinewidth{1.003750pt}%
\definecolor{currentstroke}{rgb}{0.000000,0.000000,1.000000}%
\pgfsetstrokecolor{currentstroke}%
\pgfsetdash{}{0pt}%
\pgfpathmoveto{\pgfqpoint{0.982600in}{0.934763in}}%
\pgfpathlineto{\pgfqpoint{0.982600in}{1.142738in}}%
\pgfusepath{stroke}%
\end{pgfscope}%
\begin{pgfscope}%
\pgfpathrectangle{\pgfqpoint{0.671094in}{0.532031in}}{\pgfqpoint{4.514344in}{1.921729in}} %
\pgfusepath{clip}%
\pgfsetbuttcap%
\pgfsetroundjoin%
\pgfsetlinewidth{1.003750pt}%
\definecolor{currentstroke}{rgb}{0.000000,0.000000,1.000000}%
\pgfsetstrokecolor{currentstroke}%
\pgfsetdash{}{0pt}%
\pgfpathmoveto{\pgfqpoint{0.947183in}{1.030245in}}%
\pgfpathlineto{\pgfqpoint{0.947183in}{1.087798in}}%
\pgfusepath{stroke}%
\end{pgfscope}%
\begin{pgfscope}%
\pgfpathrectangle{\pgfqpoint{0.671094in}{0.532031in}}{\pgfqpoint{4.514344in}{1.921729in}} %
\pgfusepath{clip}%
\pgfsetrectcap%
\pgfsetroundjoin%
\pgfsetlinewidth{1.003750pt}%
\definecolor{currentstroke}{rgb}{1.000000,0.000000,0.000000}%
\pgfsetstrokecolor{currentstroke}%
\pgfsetdash{}{0pt}%
\pgfpathmoveto{\pgfqpoint{5.195438in}{1.897183in}}%
\pgfpathlineto{\pgfqpoint{4.978114in}{1.842899in}}%
\pgfpathlineto{\pgfqpoint{4.766232in}{1.787716in}}%
\pgfpathlineto{\pgfqpoint{4.571508in}{1.734771in}}%
\pgfpathlineto{\pgfqpoint{4.385668in}{1.681971in}}%
\pgfpathlineto{\pgfqpoint{4.208609in}{1.629304in}}%
\pgfpathlineto{\pgfqpoint{4.048238in}{1.579318in}}%
\pgfpathlineto{\pgfqpoint{3.896404in}{1.529688in}}%
\pgfpathlineto{\pgfqpoint{3.753025in}{1.480440in}}%
\pgfpathlineto{\pgfqpoint{3.625954in}{1.434550in}}%
\pgfpathlineto{\pgfqpoint{3.507147in}{1.389425in}}%
\pgfpathlineto{\pgfqpoint{3.396539in}{1.345167in}}%
\pgfpathlineto{\pgfqpoint{3.294072in}{1.301900in}}%
\pgfpathlineto{\pgfqpoint{3.199693in}{1.259775in}}%
\pgfpathlineto{\pgfqpoint{3.113350in}{1.218978in}}%
\pgfpathlineto{\pgfqpoint{3.035000in}{1.179738in}}%
\pgfpathlineto{\pgfqpoint{2.956785in}{1.138036in}}%
\pgfpathlineto{\pgfqpoint{2.886506in}{1.097952in}}%
\pgfpathlineto{\pgfqpoint{2.824129in}{1.059829in}}%
\pgfpathlineto{\pgfqpoint{2.769619in}{1.024102in}}%
\pgfpathlineto{\pgfqpoint{2.715175in}{0.985618in}}%
\pgfpathlineto{\pgfqpoint{2.668560in}{0.949862in}}%
\pgfpathlineto{\pgfqpoint{2.629752in}{0.917572in}}%
\pgfpathlineto{\pgfqpoint{2.590977in}{0.882350in}}%
\pgfpathlineto{\pgfqpoint{2.559981in}{0.851412in}}%
\pgfpathlineto{\pgfqpoint{2.529007in}{0.817162in}}%
\pgfpathlineto{\pgfqpoint{2.505790in}{0.788509in}}%
\pgfpathlineto{\pgfqpoint{2.482585in}{0.756240in}}%
\pgfpathlineto{\pgfqpoint{2.467121in}{0.731865in}}%
\pgfpathlineto{\pgfqpoint{2.451663in}{0.704080in}}%
\pgfpathlineto{\pgfqpoint{2.436211in}{0.670853in}}%
\pgfpathlineto{\pgfqpoint{2.428486in}{0.650809in}}%
\pgfpathlineto{\pgfqpoint{2.420763in}{0.626614in}}%
\pgfpathlineto{\pgfqpoint{2.413041in}{0.593549in}}%
\pgfpathlineto{\pgfqpoint{2.413041in}{0.593549in}}%
\pgfusepath{stroke}%
\end{pgfscope}%
\begin{pgfscope}%
\pgfpathrectangle{\pgfqpoint{0.671094in}{0.532031in}}{\pgfqpoint{4.514344in}{1.921729in}} %
\pgfusepath{clip}%
\pgfsetrectcap%
\pgfsetroundjoin%
\pgfsetlinewidth{1.003750pt}%
\definecolor{currentstroke}{rgb}{0.000000,0.000000,1.000000}%
\pgfsetstrokecolor{currentstroke}%
\pgfsetdash{}{0pt}%
\pgfpathmoveto{\pgfqpoint{4.953510in}{1.872972in}}%
\pgfpathlineto{\pgfqpoint{4.751996in}{1.864740in}}%
\pgfpathlineto{\pgfqpoint{4.551752in}{1.674501in}}%
\pgfpathlineto{\pgfqpoint{4.352768in}{1.586502in}}%
\pgfpathlineto{\pgfqpoint{4.155422in}{1.642364in}}%
\pgfpathlineto{\pgfqpoint{3.959117in}{1.540868in}}%
\pgfpathlineto{\pgfqpoint{3.764232in}{1.420131in}}%
\pgfpathlineto{\pgfqpoint{3.570368in}{1.412124in}}%
\pgfpathlineto{\pgfqpoint{3.378859in}{1.213384in}}%
\pgfpathlineto{\pgfqpoint{3.187582in}{1.242032in}}%
\pgfpathlineto{\pgfqpoint{2.997489in}{1.226814in}}%
\pgfpathlineto{\pgfqpoint{2.808570in}{1.247639in}}%
\pgfpathlineto{\pgfqpoint{2.620816in}{0.999171in}}%
\pgfpathlineto{\pgfqpoint{2.434404in}{1.140269in}}%
\pgfpathlineto{\pgfqpoint{2.249135in}{1.160869in}}%
\pgfpathlineto{\pgfqpoint{2.064815in}{1.098358in}}%
\pgfpathlineto{\pgfqpoint{1.881622in}{1.122625in}}%
\pgfpathlineto{\pgfqpoint{1.699731in}{1.077768in}}%
\pgfpathlineto{\pgfqpoint{1.518578in}{1.056692in}}%
\pgfpathlineto{\pgfqpoint{1.338710in}{1.029990in}}%
\pgfpathlineto{\pgfqpoint{1.163946in}{1.030049in}}%
\pgfpathlineto{\pgfqpoint{0.982600in}{1.038751in}}%
\pgfpathlineto{\pgfqpoint{0.947183in}{1.059022in}}%
\pgfusepath{stroke}%
\end{pgfscope}%
\begin{pgfscope}%
\pgfpathrectangle{\pgfqpoint{0.671094in}{0.532031in}}{\pgfqpoint{4.514344in}{1.921729in}} %
\pgfusepath{clip}%
\pgfsetbuttcap%
\pgfsetroundjoin%
\definecolor{currentfill}{rgb}{0.000000,0.000000,1.000000}%
\pgfsetfillcolor{currentfill}%
\pgfsetlinewidth{0.501875pt}%
\definecolor{currentstroke}{rgb}{0.000000,0.000000,1.000000}%
\pgfsetstrokecolor{currentstroke}%
\pgfsetdash{}{0pt}%
\pgfsys@defobject{currentmarker}{\pgfqpoint{-0.041667in}{-0.000000in}}{\pgfqpoint{0.041667in}{0.000000in}}{%
\pgfpathmoveto{\pgfqpoint{0.041667in}{-0.000000in}}%
\pgfpathlineto{\pgfqpoint{-0.041667in}{0.000000in}}%
\pgfusepath{stroke,fill}%
}%
\begin{pgfscope}%
\pgfsys@transformshift{4.953510in}{1.660023in}%
\pgfsys@useobject{currentmarker}{}%
\end{pgfscope}%
\begin{pgfscope}%
\pgfsys@transformshift{4.751996in}{1.290897in}%
\pgfsys@useobject{currentmarker}{}%
\end{pgfscope}%
\begin{pgfscope}%
\pgfsys@transformshift{4.551752in}{1.477983in}%
\pgfsys@useobject{currentmarker}{}%
\end{pgfscope}%
\begin{pgfscope}%
\pgfsys@transformshift{4.352768in}{1.351668in}%
\pgfsys@useobject{currentmarker}{}%
\end{pgfscope}%
\begin{pgfscope}%
\pgfsys@transformshift{4.155422in}{1.616477in}%
\pgfsys@useobject{currentmarker}{}%
\end{pgfscope}%
\begin{pgfscope}%
\pgfsys@transformshift{3.959117in}{1.460331in}%
\pgfsys@useobject{currentmarker}{}%
\end{pgfscope}%
\begin{pgfscope}%
\pgfsys@transformshift{3.764232in}{1.273912in}%
\pgfsys@useobject{currentmarker}{}%
\end{pgfscope}%
\begin{pgfscope}%
\pgfsys@transformshift{3.570368in}{1.322231in}%
\pgfsys@useobject{currentmarker}{}%
\end{pgfscope}%
\begin{pgfscope}%
\pgfsys@transformshift{3.378859in}{0.932726in}%
\pgfsys@useobject{currentmarker}{}%
\end{pgfscope}%
\begin{pgfscope}%
\pgfsys@transformshift{3.187582in}{1.141421in}%
\pgfsys@useobject{currentmarker}{}%
\end{pgfscope}%
\begin{pgfscope}%
\pgfsys@transformshift{2.997489in}{1.221413in}%
\pgfsys@useobject{currentmarker}{}%
\end{pgfscope}%
\begin{pgfscope}%
\pgfsys@transformshift{2.808570in}{1.229383in}%
\pgfsys@useobject{currentmarker}{}%
\end{pgfscope}%
\begin{pgfscope}%
\pgfsys@transformshift{2.620816in}{0.807523in}%
\pgfsys@useobject{currentmarker}{}%
\end{pgfscope}%
\begin{pgfscope}%
\pgfsys@transformshift{2.434404in}{1.126553in}%
\pgfsys@useobject{currentmarker}{}%
\end{pgfscope}%
\begin{pgfscope}%
\pgfsys@transformshift{2.249135in}{1.064156in}%
\pgfsys@useobject{currentmarker}{}%
\end{pgfscope}%
\begin{pgfscope}%
\pgfsys@transformshift{2.064815in}{1.093212in}%
\pgfsys@useobject{currentmarker}{}%
\end{pgfscope}%
\begin{pgfscope}%
\pgfsys@transformshift{1.881622in}{1.033071in}%
\pgfsys@useobject{currentmarker}{}%
\end{pgfscope}%
\begin{pgfscope}%
\pgfsys@transformshift{1.699731in}{1.064366in}%
\pgfsys@useobject{currentmarker}{}%
\end{pgfscope}%
\begin{pgfscope}%
\pgfsys@transformshift{1.518578in}{0.982975in}%
\pgfsys@useobject{currentmarker}{}%
\end{pgfscope}%
\begin{pgfscope}%
\pgfsys@transformshift{1.338710in}{0.896140in}%
\pgfsys@useobject{currentmarker}{}%
\end{pgfscope}%
\begin{pgfscope}%
\pgfsys@transformshift{1.163946in}{1.005599in}%
\pgfsys@useobject{currentmarker}{}%
\end{pgfscope}%
\begin{pgfscope}%
\pgfsys@transformshift{0.982600in}{0.934763in}%
\pgfsys@useobject{currentmarker}{}%
\end{pgfscope}%
\begin{pgfscope}%
\pgfsys@transformshift{0.947183in}{1.030245in}%
\pgfsys@useobject{currentmarker}{}%
\end{pgfscope}%
\end{pgfscope}%
\begin{pgfscope}%
\pgfpathrectangle{\pgfqpoint{0.671094in}{0.532031in}}{\pgfqpoint{4.514344in}{1.921729in}} %
\pgfusepath{clip}%
\pgfsetbuttcap%
\pgfsetroundjoin%
\definecolor{currentfill}{rgb}{0.000000,0.000000,1.000000}%
\pgfsetfillcolor{currentfill}%
\pgfsetlinewidth{0.501875pt}%
\definecolor{currentstroke}{rgb}{0.000000,0.000000,1.000000}%
\pgfsetstrokecolor{currentstroke}%
\pgfsetdash{}{0pt}%
\pgfsys@defobject{currentmarker}{\pgfqpoint{-0.041667in}{-0.000000in}}{\pgfqpoint{0.041667in}{0.000000in}}{%
\pgfpathmoveto{\pgfqpoint{0.041667in}{-0.000000in}}%
\pgfpathlineto{\pgfqpoint{-0.041667in}{0.000000in}}%
\pgfusepath{stroke,fill}%
}%
\begin{pgfscope}%
\pgfsys@transformshift{4.953510in}{2.085921in}%
\pgfsys@useobject{currentmarker}{}%
\end{pgfscope}%
\begin{pgfscope}%
\pgfsys@transformshift{4.751996in}{2.438584in}%
\pgfsys@useobject{currentmarker}{}%
\end{pgfscope}%
\begin{pgfscope}%
\pgfsys@transformshift{4.551752in}{1.871020in}%
\pgfsys@useobject{currentmarker}{}%
\end{pgfscope}%
\begin{pgfscope}%
\pgfsys@transformshift{4.352768in}{1.821336in}%
\pgfsys@useobject{currentmarker}{}%
\end{pgfscope}%
\begin{pgfscope}%
\pgfsys@transformshift{4.155422in}{1.668251in}%
\pgfsys@useobject{currentmarker}{}%
\end{pgfscope}%
\begin{pgfscope}%
\pgfsys@transformshift{3.959117in}{1.621405in}%
\pgfsys@useobject{currentmarker}{}%
\end{pgfscope}%
\begin{pgfscope}%
\pgfsys@transformshift{3.764232in}{1.566350in}%
\pgfsys@useobject{currentmarker}{}%
\end{pgfscope}%
\begin{pgfscope}%
\pgfsys@transformshift{3.570368in}{1.502017in}%
\pgfsys@useobject{currentmarker}{}%
\end{pgfscope}%
\begin{pgfscope}%
\pgfsys@transformshift{3.378859in}{1.494042in}%
\pgfsys@useobject{currentmarker}{}%
\end{pgfscope}%
\begin{pgfscope}%
\pgfsys@transformshift{3.187582in}{1.342643in}%
\pgfsys@useobject{currentmarker}{}%
\end{pgfscope}%
\begin{pgfscope}%
\pgfsys@transformshift{2.997489in}{1.232215in}%
\pgfsys@useobject{currentmarker}{}%
\end{pgfscope}%
\begin{pgfscope}%
\pgfsys@transformshift{2.808570in}{1.265896in}%
\pgfsys@useobject{currentmarker}{}%
\end{pgfscope}%
\begin{pgfscope}%
\pgfsys@transformshift{2.620816in}{1.190820in}%
\pgfsys@useobject{currentmarker}{}%
\end{pgfscope}%
\begin{pgfscope}%
\pgfsys@transformshift{2.434404in}{1.153984in}%
\pgfsys@useobject{currentmarker}{}%
\end{pgfscope}%
\begin{pgfscope}%
\pgfsys@transformshift{2.249135in}{1.257583in}%
\pgfsys@useobject{currentmarker}{}%
\end{pgfscope}%
\begin{pgfscope}%
\pgfsys@transformshift{2.064815in}{1.103503in}%
\pgfsys@useobject{currentmarker}{}%
\end{pgfscope}%
\begin{pgfscope}%
\pgfsys@transformshift{1.881622in}{1.212178in}%
\pgfsys@useobject{currentmarker}{}%
\end{pgfscope}%
\begin{pgfscope}%
\pgfsys@transformshift{1.699731in}{1.091170in}%
\pgfsys@useobject{currentmarker}{}%
\end{pgfscope}%
\begin{pgfscope}%
\pgfsys@transformshift{1.518578in}{1.130408in}%
\pgfsys@useobject{currentmarker}{}%
\end{pgfscope}%
\begin{pgfscope}%
\pgfsys@transformshift{1.338710in}{1.163839in}%
\pgfsys@useobject{currentmarker}{}%
\end{pgfscope}%
\begin{pgfscope}%
\pgfsys@transformshift{1.163946in}{1.054500in}%
\pgfsys@useobject{currentmarker}{}%
\end{pgfscope}%
\begin{pgfscope}%
\pgfsys@transformshift{0.982600in}{1.142738in}%
\pgfsys@useobject{currentmarker}{}%
\end{pgfscope}%
\begin{pgfscope}%
\pgfsys@transformshift{0.947183in}{1.087798in}%
\pgfsys@useobject{currentmarker}{}%
\end{pgfscope}%
\end{pgfscope}%
\begin{pgfscope}%
\pgfpathrectangle{\pgfqpoint{0.671094in}{0.532031in}}{\pgfqpoint{4.514344in}{1.921729in}} %
\pgfusepath{clip}%
\pgfsetrectcap%
\pgfsetroundjoin%
\pgfsetlinewidth{1.003750pt}%
\definecolor{currentstroke}{rgb}{0.000000,0.000000,1.000000}%
\pgfsetstrokecolor{currentstroke}%
\pgfsetdash{}{0pt}%
\pgfpathmoveto{\pgfqpoint{4.953510in}{1.872972in}}%
\pgfpathlineto{\pgfqpoint{4.751996in}{1.864740in}}%
\pgfpathlineto{\pgfqpoint{4.551752in}{1.674501in}}%
\pgfpathlineto{\pgfqpoint{4.352768in}{1.586502in}}%
\pgfpathlineto{\pgfqpoint{4.155422in}{1.642364in}}%
\pgfpathlineto{\pgfqpoint{3.959117in}{1.540868in}}%
\pgfpathlineto{\pgfqpoint{3.764232in}{1.420131in}}%
\pgfpathlineto{\pgfqpoint{3.570368in}{1.412124in}}%
\pgfpathlineto{\pgfqpoint{3.378859in}{1.213384in}}%
\pgfpathlineto{\pgfqpoint{3.187582in}{1.242032in}}%
\pgfpathlineto{\pgfqpoint{2.997489in}{1.226814in}}%
\pgfpathlineto{\pgfqpoint{2.808570in}{1.247639in}}%
\pgfpathlineto{\pgfqpoint{2.620816in}{0.999171in}}%
\pgfpathlineto{\pgfqpoint{2.434404in}{1.140269in}}%
\pgfpathlineto{\pgfqpoint{2.249135in}{1.160869in}}%
\pgfpathlineto{\pgfqpoint{2.064815in}{1.098358in}}%
\pgfpathlineto{\pgfqpoint{1.881622in}{1.122625in}}%
\pgfpathlineto{\pgfqpoint{1.699731in}{1.077768in}}%
\pgfpathlineto{\pgfqpoint{1.518578in}{1.056692in}}%
\pgfpathlineto{\pgfqpoint{1.338710in}{1.029990in}}%
\pgfpathlineto{\pgfqpoint{1.163946in}{1.030049in}}%
\pgfpathlineto{\pgfqpoint{0.982600in}{1.038751in}}%
\pgfpathlineto{\pgfqpoint{0.947183in}{1.059022in}}%
\pgfusepath{stroke}%
\end{pgfscope}%
\begin{pgfscope}%
\pgfsetrectcap%
\pgfsetmiterjoin%
\pgfsetlinewidth{1.003750pt}%
\definecolor{currentstroke}{rgb}{0.000000,0.000000,0.000000}%
\pgfsetstrokecolor{currentstroke}%
\pgfsetdash{}{0pt}%
\pgfpathmoveto{\pgfqpoint{0.671094in}{0.532031in}}%
\pgfpathlineto{\pgfqpoint{0.671094in}{2.453760in}}%
\pgfusepath{stroke}%
\end{pgfscope}%
\begin{pgfscope}%
\pgfsetrectcap%
\pgfsetmiterjoin%
\pgfsetlinewidth{1.003750pt}%
\definecolor{currentstroke}{rgb}{0.000000,0.000000,0.000000}%
\pgfsetstrokecolor{currentstroke}%
\pgfsetdash{}{0pt}%
\pgfpathmoveto{\pgfqpoint{5.185438in}{0.532031in}}%
\pgfpathlineto{\pgfqpoint{5.185438in}{2.453760in}}%
\pgfusepath{stroke}%
\end{pgfscope}%
\begin{pgfscope}%
\pgfsetrectcap%
\pgfsetmiterjoin%
\pgfsetlinewidth{1.003750pt}%
\definecolor{currentstroke}{rgb}{0.000000,0.000000,0.000000}%
\pgfsetstrokecolor{currentstroke}%
\pgfsetdash{}{0pt}%
\pgfpathmoveto{\pgfqpoint{0.671094in}{0.532031in}}%
\pgfpathlineto{\pgfqpoint{5.185438in}{0.532031in}}%
\pgfusepath{stroke}%
\end{pgfscope}%
\begin{pgfscope}%
\pgfsetrectcap%
\pgfsetmiterjoin%
\pgfsetlinewidth{1.003750pt}%
\definecolor{currentstroke}{rgb}{0.000000,0.000000,0.000000}%
\pgfsetstrokecolor{currentstroke}%
\pgfsetdash{}{0pt}%
\pgfpathmoveto{\pgfqpoint{0.671094in}{2.453760in}}%
\pgfpathlineto{\pgfqpoint{5.185438in}{2.453760in}}%
\pgfusepath{stroke}%
\end{pgfscope}%
\begin{pgfscope}%
\pgfsetbuttcap%
\pgfsetroundjoin%
\definecolor{currentfill}{rgb}{0.000000,0.000000,0.000000}%
\pgfsetfillcolor{currentfill}%
\pgfsetlinewidth{0.501875pt}%
\definecolor{currentstroke}{rgb}{0.000000,0.000000,0.000000}%
\pgfsetstrokecolor{currentstroke}%
\pgfsetdash{}{0pt}%
\pgfsys@defobject{currentmarker}{\pgfqpoint{0.000000in}{0.000000in}}{\pgfqpoint{0.000000in}{0.055556in}}{%
\pgfpathmoveto{\pgfqpoint{0.000000in}{0.000000in}}%
\pgfpathlineto{\pgfqpoint{0.000000in}{0.055556in}}%
\pgfusepath{stroke,fill}%
}%
\begin{pgfscope}%
\pgfsys@transformshift{1.018351in}{0.532031in}%
\pgfsys@useobject{currentmarker}{}%
\end{pgfscope}%
\end{pgfscope}%
\begin{pgfscope}%
\pgfsetbuttcap%
\pgfsetroundjoin%
\definecolor{currentfill}{rgb}{0.000000,0.000000,0.000000}%
\pgfsetfillcolor{currentfill}%
\pgfsetlinewidth{0.501875pt}%
\definecolor{currentstroke}{rgb}{0.000000,0.000000,0.000000}%
\pgfsetstrokecolor{currentstroke}%
\pgfsetdash{}{0pt}%
\pgfsys@defobject{currentmarker}{\pgfqpoint{0.000000in}{-0.055556in}}{\pgfqpoint{0.000000in}{0.000000in}}{%
\pgfpathmoveto{\pgfqpoint{0.000000in}{0.000000in}}%
\pgfpathlineto{\pgfqpoint{0.000000in}{-0.055556in}}%
\pgfusepath{stroke,fill}%
}%
\begin{pgfscope}%
\pgfsys@transformshift{1.018351in}{2.453760in}%
\pgfsys@useobject{currentmarker}{}%
\end{pgfscope}%
\end{pgfscope}%
\begin{pgfscope}%
\pgftext[x=1.018351in,y=0.476476in,,top]{\rmfamily\fontsize{10.000000}{12.000000}\selectfont -40}%
\end{pgfscope}%
\begin{pgfscope}%
\pgfsetbuttcap%
\pgfsetroundjoin%
\definecolor{currentfill}{rgb}{0.000000,0.000000,0.000000}%
\pgfsetfillcolor{currentfill}%
\pgfsetlinewidth{0.501875pt}%
\definecolor{currentstroke}{rgb}{0.000000,0.000000,0.000000}%
\pgfsetstrokecolor{currentstroke}%
\pgfsetdash{}{0pt}%
\pgfsys@defobject{currentmarker}{\pgfqpoint{0.000000in}{0.000000in}}{\pgfqpoint{0.000000in}{0.055556in}}{%
\pgfpathmoveto{\pgfqpoint{0.000000in}{0.000000in}}%
\pgfpathlineto{\pgfqpoint{0.000000in}{0.055556in}}%
\pgfusepath{stroke,fill}%
}%
\begin{pgfscope}%
\pgfsys@transformshift{1.712865in}{0.532031in}%
\pgfsys@useobject{currentmarker}{}%
\end{pgfscope}%
\end{pgfscope}%
\begin{pgfscope}%
\pgfsetbuttcap%
\pgfsetroundjoin%
\definecolor{currentfill}{rgb}{0.000000,0.000000,0.000000}%
\pgfsetfillcolor{currentfill}%
\pgfsetlinewidth{0.501875pt}%
\definecolor{currentstroke}{rgb}{0.000000,0.000000,0.000000}%
\pgfsetstrokecolor{currentstroke}%
\pgfsetdash{}{0pt}%
\pgfsys@defobject{currentmarker}{\pgfqpoint{0.000000in}{-0.055556in}}{\pgfqpoint{0.000000in}{0.000000in}}{%
\pgfpathmoveto{\pgfqpoint{0.000000in}{0.000000in}}%
\pgfpathlineto{\pgfqpoint{0.000000in}{-0.055556in}}%
\pgfusepath{stroke,fill}%
}%
\begin{pgfscope}%
\pgfsys@transformshift{1.712865in}{2.453760in}%
\pgfsys@useobject{currentmarker}{}%
\end{pgfscope}%
\end{pgfscope}%
\begin{pgfscope}%
\pgftext[x=1.712865in,y=0.476476in,,top]{\rmfamily\fontsize{10.000000}{12.000000}\selectfont -20}%
\end{pgfscope}%
\begin{pgfscope}%
\pgfsetbuttcap%
\pgfsetroundjoin%
\definecolor{currentfill}{rgb}{0.000000,0.000000,0.000000}%
\pgfsetfillcolor{currentfill}%
\pgfsetlinewidth{0.501875pt}%
\definecolor{currentstroke}{rgb}{0.000000,0.000000,0.000000}%
\pgfsetstrokecolor{currentstroke}%
\pgfsetdash{}{0pt}%
\pgfsys@defobject{currentmarker}{\pgfqpoint{0.000000in}{0.000000in}}{\pgfqpoint{0.000000in}{0.055556in}}{%
\pgfpathmoveto{\pgfqpoint{0.000000in}{0.000000in}}%
\pgfpathlineto{\pgfqpoint{0.000000in}{0.055556in}}%
\pgfusepath{stroke,fill}%
}%
\begin{pgfscope}%
\pgfsys@transformshift{2.407380in}{0.532031in}%
\pgfsys@useobject{currentmarker}{}%
\end{pgfscope}%
\end{pgfscope}%
\begin{pgfscope}%
\pgfsetbuttcap%
\pgfsetroundjoin%
\definecolor{currentfill}{rgb}{0.000000,0.000000,0.000000}%
\pgfsetfillcolor{currentfill}%
\pgfsetlinewidth{0.501875pt}%
\definecolor{currentstroke}{rgb}{0.000000,0.000000,0.000000}%
\pgfsetstrokecolor{currentstroke}%
\pgfsetdash{}{0pt}%
\pgfsys@defobject{currentmarker}{\pgfqpoint{0.000000in}{-0.055556in}}{\pgfqpoint{0.000000in}{0.000000in}}{%
\pgfpathmoveto{\pgfqpoint{0.000000in}{0.000000in}}%
\pgfpathlineto{\pgfqpoint{0.000000in}{-0.055556in}}%
\pgfusepath{stroke,fill}%
}%
\begin{pgfscope}%
\pgfsys@transformshift{2.407380in}{2.453760in}%
\pgfsys@useobject{currentmarker}{}%
\end{pgfscope}%
\end{pgfscope}%
\begin{pgfscope}%
\pgftext[x=2.407380in,y=0.476476in,,top]{\rmfamily\fontsize{10.000000}{12.000000}\selectfont 0}%
\end{pgfscope}%
\begin{pgfscope}%
\pgfsetbuttcap%
\pgfsetroundjoin%
\definecolor{currentfill}{rgb}{0.000000,0.000000,0.000000}%
\pgfsetfillcolor{currentfill}%
\pgfsetlinewidth{0.501875pt}%
\definecolor{currentstroke}{rgb}{0.000000,0.000000,0.000000}%
\pgfsetstrokecolor{currentstroke}%
\pgfsetdash{}{0pt}%
\pgfsys@defobject{currentmarker}{\pgfqpoint{0.000000in}{0.000000in}}{\pgfqpoint{0.000000in}{0.055556in}}{%
\pgfpathmoveto{\pgfqpoint{0.000000in}{0.000000in}}%
\pgfpathlineto{\pgfqpoint{0.000000in}{0.055556in}}%
\pgfusepath{stroke,fill}%
}%
\begin{pgfscope}%
\pgfsys@transformshift{3.101894in}{0.532031in}%
\pgfsys@useobject{currentmarker}{}%
\end{pgfscope}%
\end{pgfscope}%
\begin{pgfscope}%
\pgfsetbuttcap%
\pgfsetroundjoin%
\definecolor{currentfill}{rgb}{0.000000,0.000000,0.000000}%
\pgfsetfillcolor{currentfill}%
\pgfsetlinewidth{0.501875pt}%
\definecolor{currentstroke}{rgb}{0.000000,0.000000,0.000000}%
\pgfsetstrokecolor{currentstroke}%
\pgfsetdash{}{0pt}%
\pgfsys@defobject{currentmarker}{\pgfqpoint{0.000000in}{-0.055556in}}{\pgfqpoint{0.000000in}{0.000000in}}{%
\pgfpathmoveto{\pgfqpoint{0.000000in}{0.000000in}}%
\pgfpathlineto{\pgfqpoint{0.000000in}{-0.055556in}}%
\pgfusepath{stroke,fill}%
}%
\begin{pgfscope}%
\pgfsys@transformshift{3.101894in}{2.453760in}%
\pgfsys@useobject{currentmarker}{}%
\end{pgfscope}%
\end{pgfscope}%
\begin{pgfscope}%
\pgftext[x=3.101894in,y=0.476476in,,top]{\rmfamily\fontsize{10.000000}{12.000000}\selectfont 20}%
\end{pgfscope}%
\begin{pgfscope}%
\pgfsetbuttcap%
\pgfsetroundjoin%
\definecolor{currentfill}{rgb}{0.000000,0.000000,0.000000}%
\pgfsetfillcolor{currentfill}%
\pgfsetlinewidth{0.501875pt}%
\definecolor{currentstroke}{rgb}{0.000000,0.000000,0.000000}%
\pgfsetstrokecolor{currentstroke}%
\pgfsetdash{}{0pt}%
\pgfsys@defobject{currentmarker}{\pgfqpoint{0.000000in}{0.000000in}}{\pgfqpoint{0.000000in}{0.055556in}}{%
\pgfpathmoveto{\pgfqpoint{0.000000in}{0.000000in}}%
\pgfpathlineto{\pgfqpoint{0.000000in}{0.055556in}}%
\pgfusepath{stroke,fill}%
}%
\begin{pgfscope}%
\pgfsys@transformshift{3.796409in}{0.532031in}%
\pgfsys@useobject{currentmarker}{}%
\end{pgfscope}%
\end{pgfscope}%
\begin{pgfscope}%
\pgfsetbuttcap%
\pgfsetroundjoin%
\definecolor{currentfill}{rgb}{0.000000,0.000000,0.000000}%
\pgfsetfillcolor{currentfill}%
\pgfsetlinewidth{0.501875pt}%
\definecolor{currentstroke}{rgb}{0.000000,0.000000,0.000000}%
\pgfsetstrokecolor{currentstroke}%
\pgfsetdash{}{0pt}%
\pgfsys@defobject{currentmarker}{\pgfqpoint{0.000000in}{-0.055556in}}{\pgfqpoint{0.000000in}{0.000000in}}{%
\pgfpathmoveto{\pgfqpoint{0.000000in}{0.000000in}}%
\pgfpathlineto{\pgfqpoint{0.000000in}{-0.055556in}}%
\pgfusepath{stroke,fill}%
}%
\begin{pgfscope}%
\pgfsys@transformshift{3.796409in}{2.453760in}%
\pgfsys@useobject{currentmarker}{}%
\end{pgfscope}%
\end{pgfscope}%
\begin{pgfscope}%
\pgftext[x=3.796409in,y=0.476476in,,top]{\rmfamily\fontsize{10.000000}{12.000000}\selectfont 40}%
\end{pgfscope}%
\begin{pgfscope}%
\pgfsetbuttcap%
\pgfsetroundjoin%
\definecolor{currentfill}{rgb}{0.000000,0.000000,0.000000}%
\pgfsetfillcolor{currentfill}%
\pgfsetlinewidth{0.501875pt}%
\definecolor{currentstroke}{rgb}{0.000000,0.000000,0.000000}%
\pgfsetstrokecolor{currentstroke}%
\pgfsetdash{}{0pt}%
\pgfsys@defobject{currentmarker}{\pgfqpoint{0.000000in}{0.000000in}}{\pgfqpoint{0.000000in}{0.055556in}}{%
\pgfpathmoveto{\pgfqpoint{0.000000in}{0.000000in}}%
\pgfpathlineto{\pgfqpoint{0.000000in}{0.055556in}}%
\pgfusepath{stroke,fill}%
}%
\begin{pgfscope}%
\pgfsys@transformshift{4.490923in}{0.532031in}%
\pgfsys@useobject{currentmarker}{}%
\end{pgfscope}%
\end{pgfscope}%
\begin{pgfscope}%
\pgfsetbuttcap%
\pgfsetroundjoin%
\definecolor{currentfill}{rgb}{0.000000,0.000000,0.000000}%
\pgfsetfillcolor{currentfill}%
\pgfsetlinewidth{0.501875pt}%
\definecolor{currentstroke}{rgb}{0.000000,0.000000,0.000000}%
\pgfsetstrokecolor{currentstroke}%
\pgfsetdash{}{0pt}%
\pgfsys@defobject{currentmarker}{\pgfqpoint{0.000000in}{-0.055556in}}{\pgfqpoint{0.000000in}{0.000000in}}{%
\pgfpathmoveto{\pgfqpoint{0.000000in}{0.000000in}}%
\pgfpathlineto{\pgfqpoint{0.000000in}{-0.055556in}}%
\pgfusepath{stroke,fill}%
}%
\begin{pgfscope}%
\pgfsys@transformshift{4.490923in}{2.453760in}%
\pgfsys@useobject{currentmarker}{}%
\end{pgfscope}%
\end{pgfscope}%
\begin{pgfscope}%
\pgftext[x=4.490923in,y=0.476476in,,top]{\rmfamily\fontsize{10.000000}{12.000000}\selectfont 60}%
\end{pgfscope}%
\begin{pgfscope}%
\pgfsetbuttcap%
\pgfsetroundjoin%
\definecolor{currentfill}{rgb}{0.000000,0.000000,0.000000}%
\pgfsetfillcolor{currentfill}%
\pgfsetlinewidth{0.501875pt}%
\definecolor{currentstroke}{rgb}{0.000000,0.000000,0.000000}%
\pgfsetstrokecolor{currentstroke}%
\pgfsetdash{}{0pt}%
\pgfsys@defobject{currentmarker}{\pgfqpoint{0.000000in}{0.000000in}}{\pgfqpoint{0.000000in}{0.055556in}}{%
\pgfpathmoveto{\pgfqpoint{0.000000in}{0.000000in}}%
\pgfpathlineto{\pgfqpoint{0.000000in}{0.055556in}}%
\pgfusepath{stroke,fill}%
}%
\begin{pgfscope}%
\pgfsys@transformshift{5.185438in}{0.532031in}%
\pgfsys@useobject{currentmarker}{}%
\end{pgfscope}%
\end{pgfscope}%
\begin{pgfscope}%
\pgfsetbuttcap%
\pgfsetroundjoin%
\definecolor{currentfill}{rgb}{0.000000,0.000000,0.000000}%
\pgfsetfillcolor{currentfill}%
\pgfsetlinewidth{0.501875pt}%
\definecolor{currentstroke}{rgb}{0.000000,0.000000,0.000000}%
\pgfsetstrokecolor{currentstroke}%
\pgfsetdash{}{0pt}%
\pgfsys@defobject{currentmarker}{\pgfqpoint{0.000000in}{-0.055556in}}{\pgfqpoint{0.000000in}{0.000000in}}{%
\pgfpathmoveto{\pgfqpoint{0.000000in}{0.000000in}}%
\pgfpathlineto{\pgfqpoint{0.000000in}{-0.055556in}}%
\pgfusepath{stroke,fill}%
}%
\begin{pgfscope}%
\pgfsys@transformshift{5.185438in}{2.453760in}%
\pgfsys@useobject{currentmarker}{}%
\end{pgfscope}%
\end{pgfscope}%
\begin{pgfscope}%
\pgftext[x=5.185438in,y=0.476476in,,top]{\rmfamily\fontsize{10.000000}{12.000000}\selectfont 80}%
\end{pgfscope}%
\begin{pgfscope}%
\pgftext[x=2.928266in,y=0.283575in,,top]{\rmfamily\fontsize{10.000000}{12.000000}\selectfont Excess Energy (meV)}%
\end{pgfscope}%
\begin{pgfscope}%
\pgfsetbuttcap%
\pgfsetroundjoin%
\definecolor{currentfill}{rgb}{0.000000,0.000000,0.000000}%
\pgfsetfillcolor{currentfill}%
\pgfsetlinewidth{0.501875pt}%
\definecolor{currentstroke}{rgb}{0.000000,0.000000,0.000000}%
\pgfsetstrokecolor{currentstroke}%
\pgfsetdash{}{0pt}%
\pgfsys@defobject{currentmarker}{\pgfqpoint{0.000000in}{0.000000in}}{\pgfqpoint{0.055556in}{0.000000in}}{%
\pgfpathmoveto{\pgfqpoint{0.000000in}{0.000000in}}%
\pgfpathlineto{\pgfqpoint{0.055556in}{0.000000in}}%
\pgfusepath{stroke,fill}%
}%
\begin{pgfscope}%
\pgfsys@transformshift{0.671094in}{0.532031in}%
\pgfsys@useobject{currentmarker}{}%
\end{pgfscope}%
\end{pgfscope}%
\begin{pgfscope}%
\pgfsetbuttcap%
\pgfsetroundjoin%
\definecolor{currentfill}{rgb}{0.000000,0.000000,0.000000}%
\pgfsetfillcolor{currentfill}%
\pgfsetlinewidth{0.501875pt}%
\definecolor{currentstroke}{rgb}{0.000000,0.000000,0.000000}%
\pgfsetstrokecolor{currentstroke}%
\pgfsetdash{}{0pt}%
\pgfsys@defobject{currentmarker}{\pgfqpoint{-0.055556in}{0.000000in}}{\pgfqpoint{0.000000in}{0.000000in}}{%
\pgfpathmoveto{\pgfqpoint{0.000000in}{0.000000in}}%
\pgfpathlineto{\pgfqpoint{-0.055556in}{0.000000in}}%
\pgfusepath{stroke,fill}%
}%
\begin{pgfscope}%
\pgfsys@transformshift{5.185438in}{0.532031in}%
\pgfsys@useobject{currentmarker}{}%
\end{pgfscope}%
\end{pgfscope}%
\begin{pgfscope}%
\pgftext[x=0.615538in,y=0.532031in,right,]{\rmfamily\fontsize{10.000000}{12.000000}\selectfont 0}%
\end{pgfscope}%
\begin{pgfscope}%
\pgfsetbuttcap%
\pgfsetroundjoin%
\definecolor{currentfill}{rgb}{0.000000,0.000000,0.000000}%
\pgfsetfillcolor{currentfill}%
\pgfsetlinewidth{0.501875pt}%
\definecolor{currentstroke}{rgb}{0.000000,0.000000,0.000000}%
\pgfsetstrokecolor{currentstroke}%
\pgfsetdash{}{0pt}%
\pgfsys@defobject{currentmarker}{\pgfqpoint{0.000000in}{0.000000in}}{\pgfqpoint{0.055556in}{0.000000in}}{%
\pgfpathmoveto{\pgfqpoint{0.000000in}{0.000000in}}%
\pgfpathlineto{\pgfqpoint{0.055556in}{0.000000in}}%
\pgfusepath{stroke,fill}%
}%
\begin{pgfscope}%
\pgfsys@transformshift{0.671094in}{0.901594in}%
\pgfsys@useobject{currentmarker}{}%
\end{pgfscope}%
\end{pgfscope}%
\begin{pgfscope}%
\pgfsetbuttcap%
\pgfsetroundjoin%
\definecolor{currentfill}{rgb}{0.000000,0.000000,0.000000}%
\pgfsetfillcolor{currentfill}%
\pgfsetlinewidth{0.501875pt}%
\definecolor{currentstroke}{rgb}{0.000000,0.000000,0.000000}%
\pgfsetstrokecolor{currentstroke}%
\pgfsetdash{}{0pt}%
\pgfsys@defobject{currentmarker}{\pgfqpoint{-0.055556in}{0.000000in}}{\pgfqpoint{0.000000in}{0.000000in}}{%
\pgfpathmoveto{\pgfqpoint{0.000000in}{0.000000in}}%
\pgfpathlineto{\pgfqpoint{-0.055556in}{0.000000in}}%
\pgfusepath{stroke,fill}%
}%
\begin{pgfscope}%
\pgfsys@transformshift{5.185438in}{0.901594in}%
\pgfsys@useobject{currentmarker}{}%
\end{pgfscope}%
\end{pgfscope}%
\begin{pgfscope}%
\pgftext[x=0.615538in,y=0.901594in,right,]{\rmfamily\fontsize{10.000000}{12.000000}\selectfont 50}%
\end{pgfscope}%
\begin{pgfscope}%
\pgfsetbuttcap%
\pgfsetroundjoin%
\definecolor{currentfill}{rgb}{0.000000,0.000000,0.000000}%
\pgfsetfillcolor{currentfill}%
\pgfsetlinewidth{0.501875pt}%
\definecolor{currentstroke}{rgb}{0.000000,0.000000,0.000000}%
\pgfsetstrokecolor{currentstroke}%
\pgfsetdash{}{0pt}%
\pgfsys@defobject{currentmarker}{\pgfqpoint{0.000000in}{0.000000in}}{\pgfqpoint{0.055556in}{0.000000in}}{%
\pgfpathmoveto{\pgfqpoint{0.000000in}{0.000000in}}%
\pgfpathlineto{\pgfqpoint{0.055556in}{0.000000in}}%
\pgfusepath{stroke,fill}%
}%
\begin{pgfscope}%
\pgfsys@transformshift{0.671094in}{1.271158in}%
\pgfsys@useobject{currentmarker}{}%
\end{pgfscope}%
\end{pgfscope}%
\begin{pgfscope}%
\pgfsetbuttcap%
\pgfsetroundjoin%
\definecolor{currentfill}{rgb}{0.000000,0.000000,0.000000}%
\pgfsetfillcolor{currentfill}%
\pgfsetlinewidth{0.501875pt}%
\definecolor{currentstroke}{rgb}{0.000000,0.000000,0.000000}%
\pgfsetstrokecolor{currentstroke}%
\pgfsetdash{}{0pt}%
\pgfsys@defobject{currentmarker}{\pgfqpoint{-0.055556in}{0.000000in}}{\pgfqpoint{0.000000in}{0.000000in}}{%
\pgfpathmoveto{\pgfqpoint{0.000000in}{0.000000in}}%
\pgfpathlineto{\pgfqpoint{-0.055556in}{0.000000in}}%
\pgfusepath{stroke,fill}%
}%
\begin{pgfscope}%
\pgfsys@transformshift{5.185438in}{1.271158in}%
\pgfsys@useobject{currentmarker}{}%
\end{pgfscope}%
\end{pgfscope}%
\begin{pgfscope}%
\pgftext[x=0.615538in,y=1.271158in,right,]{\rmfamily\fontsize{10.000000}{12.000000}\selectfont 100}%
\end{pgfscope}%
\begin{pgfscope}%
\pgfsetbuttcap%
\pgfsetroundjoin%
\definecolor{currentfill}{rgb}{0.000000,0.000000,0.000000}%
\pgfsetfillcolor{currentfill}%
\pgfsetlinewidth{0.501875pt}%
\definecolor{currentstroke}{rgb}{0.000000,0.000000,0.000000}%
\pgfsetstrokecolor{currentstroke}%
\pgfsetdash{}{0pt}%
\pgfsys@defobject{currentmarker}{\pgfqpoint{0.000000in}{0.000000in}}{\pgfqpoint{0.055556in}{0.000000in}}{%
\pgfpathmoveto{\pgfqpoint{0.000000in}{0.000000in}}%
\pgfpathlineto{\pgfqpoint{0.055556in}{0.000000in}}%
\pgfusepath{stroke,fill}%
}%
\begin{pgfscope}%
\pgfsys@transformshift{0.671094in}{1.640721in}%
\pgfsys@useobject{currentmarker}{}%
\end{pgfscope}%
\end{pgfscope}%
\begin{pgfscope}%
\pgfsetbuttcap%
\pgfsetroundjoin%
\definecolor{currentfill}{rgb}{0.000000,0.000000,0.000000}%
\pgfsetfillcolor{currentfill}%
\pgfsetlinewidth{0.501875pt}%
\definecolor{currentstroke}{rgb}{0.000000,0.000000,0.000000}%
\pgfsetstrokecolor{currentstroke}%
\pgfsetdash{}{0pt}%
\pgfsys@defobject{currentmarker}{\pgfqpoint{-0.055556in}{0.000000in}}{\pgfqpoint{0.000000in}{0.000000in}}{%
\pgfpathmoveto{\pgfqpoint{0.000000in}{0.000000in}}%
\pgfpathlineto{\pgfqpoint{-0.055556in}{0.000000in}}%
\pgfusepath{stroke,fill}%
}%
\begin{pgfscope}%
\pgfsys@transformshift{5.185438in}{1.640721in}%
\pgfsys@useobject{currentmarker}{}%
\end{pgfscope}%
\end{pgfscope}%
\begin{pgfscope}%
\pgftext[x=0.615538in,y=1.640721in,right,]{\rmfamily\fontsize{10.000000}{12.000000}\selectfont 150}%
\end{pgfscope}%
\begin{pgfscope}%
\pgfsetbuttcap%
\pgfsetroundjoin%
\definecolor{currentfill}{rgb}{0.000000,0.000000,0.000000}%
\pgfsetfillcolor{currentfill}%
\pgfsetlinewidth{0.501875pt}%
\definecolor{currentstroke}{rgb}{0.000000,0.000000,0.000000}%
\pgfsetstrokecolor{currentstroke}%
\pgfsetdash{}{0pt}%
\pgfsys@defobject{currentmarker}{\pgfqpoint{0.000000in}{0.000000in}}{\pgfqpoint{0.055556in}{0.000000in}}{%
\pgfpathmoveto{\pgfqpoint{0.000000in}{0.000000in}}%
\pgfpathlineto{\pgfqpoint{0.055556in}{0.000000in}}%
\pgfusepath{stroke,fill}%
}%
\begin{pgfscope}%
\pgfsys@transformshift{0.671094in}{2.010284in}%
\pgfsys@useobject{currentmarker}{}%
\end{pgfscope}%
\end{pgfscope}%
\begin{pgfscope}%
\pgfsetbuttcap%
\pgfsetroundjoin%
\definecolor{currentfill}{rgb}{0.000000,0.000000,0.000000}%
\pgfsetfillcolor{currentfill}%
\pgfsetlinewidth{0.501875pt}%
\definecolor{currentstroke}{rgb}{0.000000,0.000000,0.000000}%
\pgfsetstrokecolor{currentstroke}%
\pgfsetdash{}{0pt}%
\pgfsys@defobject{currentmarker}{\pgfqpoint{-0.055556in}{0.000000in}}{\pgfqpoint{0.000000in}{0.000000in}}{%
\pgfpathmoveto{\pgfqpoint{0.000000in}{0.000000in}}%
\pgfpathlineto{\pgfqpoint{-0.055556in}{0.000000in}}%
\pgfusepath{stroke,fill}%
}%
\begin{pgfscope}%
\pgfsys@transformshift{5.185438in}{2.010284in}%
\pgfsys@useobject{currentmarker}{}%
\end{pgfscope}%
\end{pgfscope}%
\begin{pgfscope}%
\pgftext[x=0.615538in,y=2.010284in,right,]{\rmfamily\fontsize{10.000000}{12.000000}\selectfont 200}%
\end{pgfscope}%
\begin{pgfscope}%
\pgfsetbuttcap%
\pgfsetroundjoin%
\definecolor{currentfill}{rgb}{0.000000,0.000000,0.000000}%
\pgfsetfillcolor{currentfill}%
\pgfsetlinewidth{0.501875pt}%
\definecolor{currentstroke}{rgb}{0.000000,0.000000,0.000000}%
\pgfsetstrokecolor{currentstroke}%
\pgfsetdash{}{0pt}%
\pgfsys@defobject{currentmarker}{\pgfqpoint{0.000000in}{0.000000in}}{\pgfqpoint{0.055556in}{0.000000in}}{%
\pgfpathmoveto{\pgfqpoint{0.000000in}{0.000000in}}%
\pgfpathlineto{\pgfqpoint{0.055556in}{0.000000in}}%
\pgfusepath{stroke,fill}%
}%
\begin{pgfscope}%
\pgfsys@transformshift{0.671094in}{2.379847in}%
\pgfsys@useobject{currentmarker}{}%
\end{pgfscope}%
\end{pgfscope}%
\begin{pgfscope}%
\pgfsetbuttcap%
\pgfsetroundjoin%
\definecolor{currentfill}{rgb}{0.000000,0.000000,0.000000}%
\pgfsetfillcolor{currentfill}%
\pgfsetlinewidth{0.501875pt}%
\definecolor{currentstroke}{rgb}{0.000000,0.000000,0.000000}%
\pgfsetstrokecolor{currentstroke}%
\pgfsetdash{}{0pt}%
\pgfsys@defobject{currentmarker}{\pgfqpoint{-0.055556in}{0.000000in}}{\pgfqpoint{0.000000in}{0.000000in}}{%
\pgfpathmoveto{\pgfqpoint{0.000000in}{0.000000in}}%
\pgfpathlineto{\pgfqpoint{-0.055556in}{0.000000in}}%
\pgfusepath{stroke,fill}%
}%
\begin{pgfscope}%
\pgfsys@transformshift{5.185438in}{2.379847in}%
\pgfsys@useobject{currentmarker}{}%
\end{pgfscope}%
\end{pgfscope}%
\begin{pgfscope}%
\pgftext[x=0.615538in,y=2.379847in,right,]{\rmfamily\fontsize{10.000000}{12.000000}\selectfont 250}%
\end{pgfscope}%
\begin{pgfscope}%
\pgftext[x=0.337760in,y=1.492896in,,bottom,rotate=90.000000]{\rmfamily\fontsize{10.000000}{12.000000}\selectfont Emittance (nm rad)}%
\end{pgfscope}%
\end{pgfpicture}%
\makeatother%
\endgroup%

    \caption[Measured emittance compared with theory and simulation.]{Emittance of electron bunches produced by the \gls{caes} (blue) compared to theoretical predictions of Equation~\ref{equation:excess_energy_emittance} (red), and the results of analysis of simulation (green) as the excess energy is varied. The black dotted lines indicate the excess energy of the two streaks shown in Figure~\ref{figure:streaks}.}
    \label{figure:emittance_vs_theory}
    % Data and code located in 2017.06.06, Analyse.py
\end{figure}


\subsection{Streaked One-Dimensional Pepperpots}\label{section:streaked_pepperpot_results}
Shown in Figure~\ref{figure:streaks} are examples of time-resolved brightness measurements from streaked one-dimensional pepperpots.
The streaks were formed from long- and short-duration electron bunches produced from the \gls{caes} with below and above-threshold ionisation and the images created from 1000-shot registered averages.
The long-duration bunch was generated with a blue ionisation laser wavelength of \unit[487.2]{nm} corresponding to an excess ionisation energy of \unit[$-$43.25]{meV}, below-threshold ionisation and a bunch duration of order \unit[10]{$\muup$s}.
The short-duration bunch was generated with a blue wavelength of \unit[475.9]{nm} with an excess energy of \unit[17.18]{meV}, resulting in above-threshold ionisation and a bunch duration of \unit[5]{ns}.
These wavelengths were carefully chosen provide good demonstrations of streaking of long- and short-duration bunches.

\begin{figure}
    \center
    %% Creator: Matplotlib, PGF backend
%%
%% To include the figure in your LaTeX document, write
%%   \input{<filename>.pgf}
%%
%% Make sure the required packages are loaded in your preamble
%%   \usepackage{pgf}
%%
%% Figures using additional raster images can only be included by \input if
%% they are in the same directory as the main LaTeX file. For loading figures
%% from other directories you can use the `import` package
%%   \usepackage{import}
%% and then include the figures with
%%   \import{<path to file>}{<filename>.pgf}
%%
%% Matplotlib used the following preamble
%%
\begingroup%
\makeatletter%
\begin{pgfpicture}%
\pgfpathrectangle{\pgfpointorigin}{\pgfqpoint{5.710000in}{7.613333in}}%
\pgfusepath{use as bounding box, clip}%
\begin{pgfscope}%
\pgfsetbuttcap%
\pgfsetmiterjoin%
\definecolor{currentfill}{rgb}{1.000000,1.000000,1.000000}%
\pgfsetfillcolor{currentfill}%
\pgfsetlinewidth{0.000000pt}%
\definecolor{currentstroke}{rgb}{1.000000,1.000000,1.000000}%
\pgfsetstrokecolor{currentstroke}%
\pgfsetdash{}{0pt}%
\pgfpathmoveto{\pgfqpoint{0.000000in}{0.000000in}}%
\pgfpathlineto{\pgfqpoint{5.710000in}{0.000000in}}%
\pgfpathlineto{\pgfqpoint{5.710000in}{7.613333in}}%
\pgfpathlineto{\pgfqpoint{0.000000in}{7.613333in}}%
\pgfpathclose%
\pgfusepath{fill}%
\end{pgfscope}%
\begin{pgfscope}%
\pgfsetbuttcap%
\pgfsetmiterjoin%
\definecolor{currentfill}{rgb}{1.000000,1.000000,1.000000}%
\pgfsetfillcolor{currentfill}%
\pgfsetlinewidth{0.000000pt}%
\definecolor{currentstroke}{rgb}{0.000000,0.000000,0.000000}%
\pgfsetstrokecolor{currentstroke}%
\pgfsetstrokeopacity{0.000000}%
\pgfsetdash{}{0pt}%
\pgfpathmoveto{\pgfqpoint{0.898557in}{5.749375in}}%
\pgfpathlineto{\pgfqpoint{3.221779in}{5.749375in}}%
\pgfpathlineto{\pgfqpoint{3.221779in}{7.448333in}}%
\pgfpathlineto{\pgfqpoint{0.898557in}{7.448333in}}%
\pgfpathclose%
\pgfusepath{fill}%
\end{pgfscope}%
\begin{pgfscope}%
\pgfpathrectangle{\pgfqpoint{0.898557in}{5.749375in}}{\pgfqpoint{2.323221in}{1.698958in}} %
\pgfusepath{clip}%
\pgftext[at=\pgfqpoint{0.898557in}{5.749375in},left,bottom]{\pgfimage[interpolate=true,width=2.330000in,height=1.710000in]{streaks-img0.png}}%
\end{pgfscope}%
\begin{pgfscope}%
\pgfpathrectangle{\pgfqpoint{0.898557in}{5.749375in}}{\pgfqpoint{2.323221in}{1.698958in}} %
\pgfusepath{clip}%
\pgfsetrectcap%
\pgfsetroundjoin%
\pgfsetlinewidth{0.501875pt}%
\definecolor{currentstroke}{rgb}{1.000000,1.000000,1.000000}%
\pgfsetstrokecolor{currentstroke}%
\pgfsetdash{}{0pt}%
\pgfpathmoveto{\pgfqpoint{2.536793in}{5.804357in}}%
\pgfpathlineto{\pgfqpoint{3.174897in}{5.804357in}}%
\pgfusepath{stroke}%
\end{pgfscope}%
\begin{pgfscope}%
\pgfpathrectangle{\pgfqpoint{0.898557in}{5.749375in}}{\pgfqpoint{2.323221in}{1.698958in}} %
\pgfusepath{clip}%
\pgfsetrectcap%
\pgfsetroundjoin%
\pgfsetlinewidth{0.501875pt}%
\definecolor{currentstroke}{rgb}{1.000000,1.000000,1.000000}%
\pgfsetstrokecolor{currentstroke}%
\pgfsetdash{}{0pt}%
\pgfpathmoveto{\pgfqpoint{2.536793in}{5.826350in}}%
\pgfpathlineto{\pgfqpoint{2.536793in}{5.782364in}}%
\pgfusepath{stroke}%
\end{pgfscope}%
\begin{pgfscope}%
\pgfpathrectangle{\pgfqpoint{0.898557in}{5.749375in}}{\pgfqpoint{2.323221in}{1.698958in}} %
\pgfusepath{clip}%
\pgfsetrectcap%
\pgfsetroundjoin%
\pgfsetlinewidth{0.501875pt}%
\definecolor{currentstroke}{rgb}{1.000000,1.000000,1.000000}%
\pgfsetstrokecolor{currentstroke}%
\pgfsetdash{}{0pt}%
\pgfpathmoveto{\pgfqpoint{3.174897in}{5.826350in}}%
\pgfpathlineto{\pgfqpoint{3.174897in}{5.782364in}}%
\pgfusepath{stroke}%
\end{pgfscope}%
\begin{pgfscope}%
\pgfsetrectcap%
\pgfsetmiterjoin%
\pgfsetlinewidth{1.003750pt}%
\definecolor{currentstroke}{rgb}{0.000000,0.000000,0.000000}%
\pgfsetstrokecolor{currentstroke}%
\pgfsetdash{}{0pt}%
\pgfpathmoveto{\pgfqpoint{3.221779in}{5.749375in}}%
\pgfpathlineto{\pgfqpoint{3.221779in}{7.448333in}}%
\pgfusepath{stroke}%
\end{pgfscope}%
\begin{pgfscope}%
\pgfsetrectcap%
\pgfsetmiterjoin%
\pgfsetlinewidth{1.003750pt}%
\definecolor{currentstroke}{rgb}{0.000000,0.000000,0.000000}%
\pgfsetstrokecolor{currentstroke}%
\pgfsetdash{}{0pt}%
\pgfpathmoveto{\pgfqpoint{0.898557in}{5.749375in}}%
\pgfpathlineto{\pgfqpoint{0.898557in}{7.448333in}}%
\pgfusepath{stroke}%
\end{pgfscope}%
\begin{pgfscope}%
\pgfsetrectcap%
\pgfsetmiterjoin%
\pgfsetlinewidth{1.003750pt}%
\definecolor{currentstroke}{rgb}{0.000000,0.000000,0.000000}%
\pgfsetstrokecolor{currentstroke}%
\pgfsetdash{}{0pt}%
\pgfpathmoveto{\pgfqpoint{0.898557in}{7.448333in}}%
\pgfpathlineto{\pgfqpoint{3.221779in}{7.448333in}}%
\pgfusepath{stroke}%
\end{pgfscope}%
\begin{pgfscope}%
\pgfsetrectcap%
\pgfsetmiterjoin%
\pgfsetlinewidth{1.003750pt}%
\definecolor{currentstroke}{rgb}{0.000000,0.000000,0.000000}%
\pgfsetstrokecolor{currentstroke}%
\pgfsetdash{}{0pt}%
\pgfpathmoveto{\pgfqpoint{0.898557in}{5.749375in}}%
\pgfpathlineto{\pgfqpoint{3.221779in}{5.749375in}}%
\pgfusepath{stroke}%
\end{pgfscope}%
\begin{pgfscope}%
\definecolor{textcolor}{rgb}{1.000000,1.000000,1.000000}%
\pgfsetstrokecolor{textcolor}%
\pgfsetfillcolor{textcolor}%
\pgftext[x=2.855845in,y=5.826350in,,bottom]{\color{textcolor}\fontsize{6.000000}{7.200000}\selectfont 5mm}%
\end{pgfscope}%
\begin{pgfscope}%
\pgfsetbuttcap%
\pgfsetmiterjoin%
\definecolor{currentfill}{rgb}{1.000000,1.000000,1.000000}%
\pgfsetfillcolor{currentfill}%
\pgfsetlinewidth{0.000000pt}%
\definecolor{currentstroke}{rgb}{0.000000,0.000000,0.000000}%
\pgfsetstrokecolor{currentstroke}%
\pgfsetstrokeopacity{0.000000}%
\pgfsetdash{}{0pt}%
\pgfpathmoveto{\pgfqpoint{0.898557in}{4.050417in}}%
\pgfpathlineto{\pgfqpoint{3.221779in}{4.050417in}}%
\pgfpathlineto{\pgfqpoint{3.221779in}{5.749375in}}%
\pgfpathlineto{\pgfqpoint{0.898557in}{5.749375in}}%
\pgfpathclose%
\pgfusepath{fill}%
\end{pgfscope}%
\begin{pgfscope}%
\pgfpathrectangle{\pgfqpoint{0.898557in}{4.050417in}}{\pgfqpoint{2.323221in}{1.698958in}} %
\pgfusepath{clip}%
\pgfsetbuttcap%
\pgfsetroundjoin%
\definecolor{currentfill}{rgb}{1.000000,0.400000,0.200000}%
\pgfsetfillcolor{currentfill}%
\pgfsetfillopacity{0.500000}%
\pgfsetlinewidth{1.003750pt}%
\definecolor{currentstroke}{rgb}{1.000000,0.400000,0.200000}%
\pgfsetstrokecolor{currentstroke}%
\pgfsetstrokeopacity{0.500000}%
\pgfsetdash{}{0pt}%
\pgfpathmoveto{\pgfqpoint{1.195471in}{4.512649in}}%
\pgfpathlineto{\pgfqpoint{1.195471in}{4.512649in}}%
\pgfpathlineto{\pgfqpoint{1.200680in}{4.490993in}}%
\pgfpathlineto{\pgfqpoint{1.205889in}{4.444602in}}%
\pgfpathlineto{\pgfqpoint{1.211098in}{4.379589in}}%
\pgfpathlineto{\pgfqpoint{1.216307in}{4.310386in}}%
\pgfpathlineto{\pgfqpoint{1.221516in}{4.250942in}}%
\pgfpathlineto{\pgfqpoint{1.226725in}{4.207395in}}%
\pgfpathlineto{\pgfqpoint{1.231934in}{4.179188in}}%
\pgfpathlineto{\pgfqpoint{1.237143in}{4.158980in}}%
\pgfpathlineto{\pgfqpoint{1.242352in}{4.145406in}}%
\pgfpathlineto{\pgfqpoint{1.247561in}{4.137583in}}%
\pgfpathlineto{\pgfqpoint{1.252770in}{4.132817in}}%
\pgfpathlineto{\pgfqpoint{1.257979in}{4.130100in}}%
\pgfpathlineto{\pgfqpoint{1.263188in}{4.128289in}}%
\pgfpathlineto{\pgfqpoint{1.268397in}{4.126836in}}%
\pgfpathlineto{\pgfqpoint{1.273606in}{4.125133in}}%
\pgfpathlineto{\pgfqpoint{1.278815in}{4.123857in}}%
\pgfpathlineto{\pgfqpoint{1.284024in}{4.122778in}}%
\pgfpathlineto{\pgfqpoint{1.289233in}{4.122504in}}%
\pgfpathlineto{\pgfqpoint{1.294442in}{4.121866in}}%
\pgfpathlineto{\pgfqpoint{1.299651in}{4.121032in}}%
\pgfpathlineto{\pgfqpoint{1.304860in}{4.120985in}}%
\pgfpathlineto{\pgfqpoint{1.310069in}{4.118930in}}%
\pgfpathlineto{\pgfqpoint{1.315278in}{4.117673in}}%
\pgfpathlineto{\pgfqpoint{1.320487in}{4.117479in}}%
\pgfpathlineto{\pgfqpoint{1.325696in}{4.117476in}}%
\pgfpathlineto{\pgfqpoint{1.330905in}{4.116975in}}%
\pgfpathlineto{\pgfqpoint{1.336114in}{4.116386in}}%
\pgfpathlineto{\pgfqpoint{1.341323in}{4.117721in}}%
\pgfpathlineto{\pgfqpoint{1.346532in}{4.118222in}}%
\pgfpathlineto{\pgfqpoint{1.351741in}{4.117253in}}%
\pgfpathlineto{\pgfqpoint{1.356951in}{4.116300in}}%
\pgfpathlineto{\pgfqpoint{1.362160in}{4.115278in}}%
\pgfpathlineto{\pgfqpoint{1.367369in}{4.115392in}}%
\pgfpathlineto{\pgfqpoint{1.372578in}{4.115178in}}%
\pgfpathlineto{\pgfqpoint{1.377787in}{4.113671in}}%
\pgfpathlineto{\pgfqpoint{1.382996in}{4.112350in}}%
\pgfpathlineto{\pgfqpoint{1.388205in}{4.111110in}}%
\pgfpathlineto{\pgfqpoint{1.393414in}{4.109762in}}%
\pgfpathlineto{\pgfqpoint{1.398623in}{4.109557in}}%
\pgfpathlineto{\pgfqpoint{1.403832in}{4.110326in}}%
\pgfpathlineto{\pgfqpoint{1.409041in}{4.110594in}}%
\pgfpathlineto{\pgfqpoint{1.414250in}{4.110390in}}%
\pgfpathlineto{\pgfqpoint{1.419459in}{4.110305in}}%
\pgfpathlineto{\pgfqpoint{1.424668in}{4.109221in}}%
\pgfpathlineto{\pgfqpoint{1.429877in}{4.108923in}}%
\pgfpathlineto{\pgfqpoint{1.435086in}{4.108205in}}%
\pgfpathlineto{\pgfqpoint{1.440295in}{4.107906in}}%
\pgfpathlineto{\pgfqpoint{1.445504in}{4.107814in}}%
\pgfpathlineto{\pgfqpoint{1.450713in}{4.107262in}}%
\pgfpathlineto{\pgfqpoint{1.455922in}{4.106904in}}%
\pgfpathlineto{\pgfqpoint{1.461131in}{4.105699in}}%
\pgfpathlineto{\pgfqpoint{1.466340in}{4.104571in}}%
\pgfpathlineto{\pgfqpoint{1.471549in}{4.103712in}}%
\pgfpathlineto{\pgfqpoint{1.476758in}{4.103714in}}%
\pgfpathlineto{\pgfqpoint{1.481967in}{4.103592in}}%
\pgfpathlineto{\pgfqpoint{1.487176in}{4.102764in}}%
\pgfpathlineto{\pgfqpoint{1.492385in}{4.101974in}}%
\pgfpathlineto{\pgfqpoint{1.497594in}{4.101225in}}%
\pgfpathlineto{\pgfqpoint{1.502803in}{4.102098in}}%
\pgfpathlineto{\pgfqpoint{1.508012in}{4.101683in}}%
\pgfpathlineto{\pgfqpoint{1.513221in}{4.101554in}}%
\pgfpathlineto{\pgfqpoint{1.518430in}{4.101419in}}%
\pgfpathlineto{\pgfqpoint{1.523639in}{4.100814in}}%
\pgfpathlineto{\pgfqpoint{1.528848in}{4.100081in}}%
\pgfpathlineto{\pgfqpoint{1.534057in}{4.099789in}}%
\pgfpathlineto{\pgfqpoint{1.539266in}{4.099810in}}%
\pgfpathlineto{\pgfqpoint{1.544475in}{4.099169in}}%
\pgfpathlineto{\pgfqpoint{1.549684in}{4.098167in}}%
\pgfpathlineto{\pgfqpoint{1.554893in}{4.098024in}}%
\pgfpathlineto{\pgfqpoint{1.560102in}{4.097753in}}%
\pgfpathlineto{\pgfqpoint{1.565311in}{4.097550in}}%
\pgfpathlineto{\pgfqpoint{1.570520in}{4.096968in}}%
\pgfpathlineto{\pgfqpoint{1.575729in}{4.096434in}}%
\pgfpathlineto{\pgfqpoint{1.580938in}{4.096686in}}%
\pgfpathlineto{\pgfqpoint{1.586147in}{4.097000in}}%
\pgfpathlineto{\pgfqpoint{1.591356in}{4.096757in}}%
\pgfpathlineto{\pgfqpoint{1.596565in}{4.096820in}}%
\pgfpathlineto{\pgfqpoint{1.601774in}{4.097027in}}%
\pgfpathlineto{\pgfqpoint{1.606983in}{4.097046in}}%
\pgfpathlineto{\pgfqpoint{1.612192in}{4.096065in}}%
\pgfpathlineto{\pgfqpoint{1.617401in}{4.095153in}}%
\pgfpathlineto{\pgfqpoint{1.622610in}{4.094503in}}%
\pgfpathlineto{\pgfqpoint{1.627819in}{4.094184in}}%
\pgfpathlineto{\pgfqpoint{1.633028in}{4.094199in}}%
\pgfpathlineto{\pgfqpoint{1.638237in}{4.094511in}}%
\pgfpathlineto{\pgfqpoint{1.643446in}{4.094247in}}%
\pgfpathlineto{\pgfqpoint{1.648655in}{4.095008in}}%
\pgfpathlineto{\pgfqpoint{1.653864in}{4.095344in}}%
\pgfpathlineto{\pgfqpoint{1.659073in}{4.094544in}}%
\pgfpathlineto{\pgfqpoint{1.664282in}{4.093485in}}%
\pgfpathlineto{\pgfqpoint{1.669492in}{4.093181in}}%
\pgfpathlineto{\pgfqpoint{1.674701in}{4.093348in}}%
\pgfpathlineto{\pgfqpoint{1.679910in}{4.092573in}}%
\pgfpathlineto{\pgfqpoint{1.685119in}{4.091602in}}%
\pgfpathlineto{\pgfqpoint{1.690328in}{4.091656in}}%
\pgfpathlineto{\pgfqpoint{1.695537in}{4.091561in}}%
\pgfpathlineto{\pgfqpoint{1.700746in}{4.091656in}}%
\pgfpathlineto{\pgfqpoint{1.705955in}{4.091576in}}%
\pgfpathlineto{\pgfqpoint{1.711164in}{4.091567in}}%
\pgfpathlineto{\pgfqpoint{1.716373in}{4.090867in}}%
\pgfpathlineto{\pgfqpoint{1.721582in}{4.090773in}}%
\pgfpathlineto{\pgfqpoint{1.726791in}{4.090875in}}%
\pgfpathlineto{\pgfqpoint{1.732000in}{4.090361in}}%
\pgfpathlineto{\pgfqpoint{1.737209in}{4.089821in}}%
\pgfpathlineto{\pgfqpoint{1.742418in}{4.089396in}}%
\pgfpathlineto{\pgfqpoint{1.747627in}{4.089844in}}%
\pgfpathlineto{\pgfqpoint{1.752836in}{4.090592in}}%
\pgfpathlineto{\pgfqpoint{1.758045in}{4.090810in}}%
\pgfpathlineto{\pgfqpoint{1.763254in}{4.089858in}}%
\pgfpathlineto{\pgfqpoint{1.768463in}{4.089673in}}%
\pgfpathlineto{\pgfqpoint{1.773672in}{4.088812in}}%
\pgfpathlineto{\pgfqpoint{1.778881in}{4.087944in}}%
\pgfpathlineto{\pgfqpoint{1.784090in}{4.087666in}}%
\pgfpathlineto{\pgfqpoint{1.789299in}{4.088781in}}%
\pgfpathlineto{\pgfqpoint{1.794508in}{4.088998in}}%
\pgfpathlineto{\pgfqpoint{1.799717in}{4.089174in}}%
\pgfpathlineto{\pgfqpoint{1.804926in}{4.089256in}}%
\pgfpathlineto{\pgfqpoint{1.810135in}{4.089378in}}%
\pgfpathlineto{\pgfqpoint{1.815344in}{4.088478in}}%
\pgfpathlineto{\pgfqpoint{1.820553in}{4.087835in}}%
\pgfpathlineto{\pgfqpoint{1.825762in}{4.087537in}}%
\pgfpathlineto{\pgfqpoint{1.830971in}{4.087478in}}%
\pgfpathlineto{\pgfqpoint{1.836180in}{4.088289in}}%
\pgfpathlineto{\pgfqpoint{1.841389in}{4.087835in}}%
\pgfpathlineto{\pgfqpoint{1.846598in}{4.087352in}}%
\pgfpathlineto{\pgfqpoint{1.851807in}{4.086930in}}%
\pgfpathlineto{\pgfqpoint{1.857016in}{4.087140in}}%
\pgfpathlineto{\pgfqpoint{1.862225in}{4.086914in}}%
\pgfpathlineto{\pgfqpoint{1.867434in}{4.086051in}}%
\pgfpathlineto{\pgfqpoint{1.872643in}{4.086322in}}%
\pgfpathlineto{\pgfqpoint{1.877852in}{4.086233in}}%
\pgfpathlineto{\pgfqpoint{1.883061in}{4.085718in}}%
\pgfpathlineto{\pgfqpoint{1.888270in}{4.085315in}}%
\pgfpathlineto{\pgfqpoint{1.893479in}{4.085681in}}%
\pgfpathlineto{\pgfqpoint{1.898688in}{4.084954in}}%
\pgfpathlineto{\pgfqpoint{1.903897in}{4.084727in}}%
\pgfpathlineto{\pgfqpoint{1.909106in}{4.084642in}}%
\pgfpathlineto{\pgfqpoint{1.914315in}{4.085014in}}%
\pgfpathlineto{\pgfqpoint{1.919524in}{4.085439in}}%
\pgfpathlineto{\pgfqpoint{1.924733in}{4.085600in}}%
\pgfpathlineto{\pgfqpoint{1.929942in}{4.085094in}}%
\pgfpathlineto{\pgfqpoint{1.935151in}{4.086287in}}%
\pgfpathlineto{\pgfqpoint{1.940360in}{4.086303in}}%
\pgfpathlineto{\pgfqpoint{1.945569in}{4.085860in}}%
\pgfpathlineto{\pgfqpoint{1.950778in}{4.085460in}}%
\pgfpathlineto{\pgfqpoint{1.955987in}{4.084389in}}%
\pgfpathlineto{\pgfqpoint{1.961196in}{4.083542in}}%
\pgfpathlineto{\pgfqpoint{1.966405in}{4.083370in}}%
\pgfpathlineto{\pgfqpoint{1.971614in}{4.083328in}}%
\pgfpathlineto{\pgfqpoint{1.976824in}{4.083329in}}%
\pgfpathlineto{\pgfqpoint{1.982033in}{4.083000in}}%
\pgfpathlineto{\pgfqpoint{1.987242in}{4.082478in}}%
\pgfpathlineto{\pgfqpoint{1.992451in}{4.082834in}}%
\pgfpathlineto{\pgfqpoint{1.997660in}{4.083100in}}%
\pgfpathlineto{\pgfqpoint{2.002869in}{4.083565in}}%
\pgfpathlineto{\pgfqpoint{2.008078in}{4.083743in}}%
\pgfpathlineto{\pgfqpoint{2.013287in}{4.083749in}}%
\pgfpathlineto{\pgfqpoint{2.018496in}{4.083645in}}%
\pgfpathlineto{\pgfqpoint{2.023705in}{4.082800in}}%
\pgfpathlineto{\pgfqpoint{2.028914in}{4.082885in}}%
\pgfpathlineto{\pgfqpoint{2.034123in}{4.083397in}}%
\pgfpathlineto{\pgfqpoint{2.039332in}{4.083699in}}%
\pgfpathlineto{\pgfqpoint{2.044541in}{4.083936in}}%
\pgfpathlineto{\pgfqpoint{2.049750in}{4.083069in}}%
\pgfpathlineto{\pgfqpoint{2.054959in}{4.082449in}}%
\pgfpathlineto{\pgfqpoint{2.060168in}{4.082556in}}%
\pgfpathlineto{\pgfqpoint{2.065377in}{4.083044in}}%
\pgfpathlineto{\pgfqpoint{2.070586in}{4.083187in}}%
\pgfpathlineto{\pgfqpoint{2.075795in}{4.083187in}}%
\pgfpathlineto{\pgfqpoint{2.081004in}{4.083007in}}%
\pgfpathlineto{\pgfqpoint{2.086213in}{4.083202in}}%
\pgfpathlineto{\pgfqpoint{2.091422in}{4.083291in}}%
\pgfpathlineto{\pgfqpoint{2.096631in}{4.082995in}}%
\pgfpathlineto{\pgfqpoint{2.101840in}{4.082989in}}%
\pgfpathlineto{\pgfqpoint{2.107049in}{4.083601in}}%
\pgfpathlineto{\pgfqpoint{2.112258in}{4.083265in}}%
\pgfpathlineto{\pgfqpoint{2.117467in}{4.083029in}}%
\pgfpathlineto{\pgfqpoint{2.122676in}{4.082787in}}%
\pgfpathlineto{\pgfqpoint{2.127885in}{4.082944in}}%
\pgfpathlineto{\pgfqpoint{2.133094in}{4.082839in}}%
\pgfpathlineto{\pgfqpoint{2.138303in}{4.082705in}}%
\pgfpathlineto{\pgfqpoint{2.143512in}{4.082130in}}%
\pgfpathlineto{\pgfqpoint{2.148721in}{4.082159in}}%
\pgfpathlineto{\pgfqpoint{2.153930in}{4.082373in}}%
\pgfpathlineto{\pgfqpoint{2.159139in}{4.082404in}}%
\pgfpathlineto{\pgfqpoint{2.164348in}{4.082542in}}%
\pgfpathlineto{\pgfqpoint{2.169557in}{4.081919in}}%
\pgfpathlineto{\pgfqpoint{2.174766in}{4.081287in}}%
\pgfpathlineto{\pgfqpoint{2.179975in}{4.081364in}}%
\pgfpathlineto{\pgfqpoint{2.185184in}{4.080959in}}%
\pgfpathlineto{\pgfqpoint{2.190393in}{4.080861in}}%
\pgfpathlineto{\pgfqpoint{2.195602in}{4.081435in}}%
\pgfpathlineto{\pgfqpoint{2.200811in}{4.081223in}}%
\pgfpathlineto{\pgfqpoint{2.206020in}{4.081453in}}%
\pgfpathlineto{\pgfqpoint{2.211229in}{4.081553in}}%
\pgfpathlineto{\pgfqpoint{2.216438in}{4.081331in}}%
\pgfpathlineto{\pgfqpoint{2.221647in}{4.081064in}}%
\pgfpathlineto{\pgfqpoint{2.226856in}{4.080587in}}%
\pgfpathlineto{\pgfqpoint{2.232065in}{4.080394in}}%
\pgfpathlineto{\pgfqpoint{2.237274in}{4.081328in}}%
\pgfpathlineto{\pgfqpoint{2.242483in}{4.081501in}}%
\pgfpathlineto{\pgfqpoint{2.247692in}{4.080882in}}%
\pgfpathlineto{\pgfqpoint{2.252901in}{4.081002in}}%
\pgfpathlineto{\pgfqpoint{2.258110in}{4.081162in}}%
\pgfpathlineto{\pgfqpoint{2.263319in}{4.081332in}}%
\pgfpathlineto{\pgfqpoint{2.268528in}{4.081698in}}%
\pgfpathlineto{\pgfqpoint{2.273737in}{4.081883in}}%
\pgfpathlineto{\pgfqpoint{2.278946in}{4.081745in}}%
\pgfpathlineto{\pgfqpoint{2.284155in}{4.081737in}}%
\pgfpathlineto{\pgfqpoint{2.289365in}{4.080869in}}%
\pgfpathlineto{\pgfqpoint{2.294574in}{4.080612in}}%
\pgfpathlineto{\pgfqpoint{2.299783in}{4.081008in}}%
\pgfpathlineto{\pgfqpoint{2.304992in}{4.080672in}}%
\pgfpathlineto{\pgfqpoint{2.310201in}{4.081017in}}%
\pgfpathlineto{\pgfqpoint{2.315410in}{4.080991in}}%
\pgfpathlineto{\pgfqpoint{2.320619in}{4.081024in}}%
\pgfpathlineto{\pgfqpoint{2.325828in}{4.080874in}}%
\pgfpathlineto{\pgfqpoint{2.331037in}{4.081016in}}%
\pgfpathlineto{\pgfqpoint{2.336246in}{4.080719in}}%
\pgfpathlineto{\pgfqpoint{2.341455in}{4.080220in}}%
\pgfpathlineto{\pgfqpoint{2.346664in}{4.080051in}}%
\pgfpathlineto{\pgfqpoint{2.351873in}{4.079989in}}%
\pgfpathlineto{\pgfqpoint{2.357082in}{4.079474in}}%
\pgfpathlineto{\pgfqpoint{2.362291in}{4.079300in}}%
\pgfpathlineto{\pgfqpoint{2.367500in}{4.080099in}}%
\pgfpathlineto{\pgfqpoint{2.372709in}{4.079551in}}%
\pgfpathlineto{\pgfqpoint{2.377918in}{4.080450in}}%
\pgfpathlineto{\pgfqpoint{2.383127in}{4.081029in}}%
\pgfpathlineto{\pgfqpoint{2.388336in}{4.081587in}}%
\pgfpathlineto{\pgfqpoint{2.393545in}{4.081585in}}%
\pgfpathlineto{\pgfqpoint{2.398754in}{4.081602in}}%
\pgfpathlineto{\pgfqpoint{2.403963in}{4.081819in}}%
\pgfpathlineto{\pgfqpoint{2.409172in}{4.081531in}}%
\pgfpathlineto{\pgfqpoint{2.414381in}{4.080669in}}%
\pgfpathlineto{\pgfqpoint{2.419590in}{4.080749in}}%
\pgfpathlineto{\pgfqpoint{2.424799in}{4.080445in}}%
\pgfpathlineto{\pgfqpoint{2.430008in}{4.080290in}}%
\pgfpathlineto{\pgfqpoint{2.435217in}{4.079601in}}%
\pgfpathlineto{\pgfqpoint{2.440426in}{4.079999in}}%
\pgfpathlineto{\pgfqpoint{2.445635in}{4.079945in}}%
\pgfpathlineto{\pgfqpoint{2.450844in}{4.079440in}}%
\pgfpathlineto{\pgfqpoint{2.456053in}{4.079870in}}%
\pgfpathlineto{\pgfqpoint{2.461262in}{4.080510in}}%
\pgfpathlineto{\pgfqpoint{2.466471in}{4.079963in}}%
\pgfpathlineto{\pgfqpoint{2.471680in}{4.079774in}}%
\pgfpathlineto{\pgfqpoint{2.476889in}{4.079912in}}%
\pgfpathlineto{\pgfqpoint{2.482098in}{4.079742in}}%
\pgfpathlineto{\pgfqpoint{2.487307in}{4.079071in}}%
\pgfpathlineto{\pgfqpoint{2.492516in}{4.079000in}}%
\pgfpathlineto{\pgfqpoint{2.497725in}{4.079381in}}%
\pgfpathlineto{\pgfqpoint{2.502934in}{4.079582in}}%
\pgfpathlineto{\pgfqpoint{2.508143in}{4.079255in}}%
\pgfpathlineto{\pgfqpoint{2.513352in}{4.079923in}}%
\pgfpathlineto{\pgfqpoint{2.518561in}{4.080071in}}%
\pgfpathlineto{\pgfqpoint{2.523770in}{4.079910in}}%
\pgfpathlineto{\pgfqpoint{2.528979in}{4.080496in}}%
\pgfpathlineto{\pgfqpoint{2.534188in}{4.080270in}}%
\pgfpathlineto{\pgfqpoint{2.539397in}{4.079790in}}%
\pgfpathlineto{\pgfqpoint{2.544606in}{4.080146in}}%
\pgfpathlineto{\pgfqpoint{2.549815in}{4.080285in}}%
\pgfpathlineto{\pgfqpoint{2.555024in}{4.080337in}}%
\pgfpathlineto{\pgfqpoint{2.560233in}{4.079928in}}%
\pgfpathlineto{\pgfqpoint{2.565442in}{4.080178in}}%
\pgfpathlineto{\pgfqpoint{2.570651in}{4.080764in}}%
\pgfpathlineto{\pgfqpoint{2.575860in}{4.080425in}}%
\pgfpathlineto{\pgfqpoint{2.581069in}{4.081245in}}%
\pgfpathlineto{\pgfqpoint{2.586278in}{4.081257in}}%
\pgfpathlineto{\pgfqpoint{2.591487in}{4.081219in}}%
\pgfpathlineto{\pgfqpoint{2.596697in}{4.079497in}}%
\pgfpathlineto{\pgfqpoint{2.601906in}{4.078599in}}%
\pgfpathlineto{\pgfqpoint{2.607115in}{4.079474in}}%
\pgfpathlineto{\pgfqpoint{2.612324in}{4.079507in}}%
\pgfpathlineto{\pgfqpoint{2.617533in}{4.078595in}}%
\pgfpathlineto{\pgfqpoint{2.622742in}{4.078732in}}%
\pgfpathlineto{\pgfqpoint{2.627951in}{4.079919in}}%
\pgfpathlineto{\pgfqpoint{2.633160in}{4.080435in}}%
\pgfpathlineto{\pgfqpoint{2.638369in}{4.080447in}}%
\pgfpathlineto{\pgfqpoint{2.643578in}{4.080240in}}%
\pgfpathlineto{\pgfqpoint{2.648787in}{4.079712in}}%
\pgfpathlineto{\pgfqpoint{2.653996in}{4.079479in}}%
\pgfpathlineto{\pgfqpoint{2.659205in}{4.080300in}}%
\pgfpathlineto{\pgfqpoint{2.664414in}{4.079649in}}%
\pgfpathlineto{\pgfqpoint{2.669623in}{4.079187in}}%
\pgfpathlineto{\pgfqpoint{2.674832in}{4.079685in}}%
\pgfpathlineto{\pgfqpoint{2.680041in}{4.079979in}}%
\pgfpathlineto{\pgfqpoint{2.685250in}{4.080050in}}%
\pgfpathlineto{\pgfqpoint{2.690459in}{4.079755in}}%
\pgfpathlineto{\pgfqpoint{2.695668in}{4.080249in}}%
\pgfpathlineto{\pgfqpoint{2.700877in}{4.079559in}}%
\pgfpathlineto{\pgfqpoint{2.706086in}{4.079649in}}%
\pgfpathlineto{\pgfqpoint{2.711295in}{4.079247in}}%
\pgfpathlineto{\pgfqpoint{2.716504in}{4.079263in}}%
\pgfpathlineto{\pgfqpoint{2.721713in}{4.078677in}}%
\pgfpathlineto{\pgfqpoint{2.726922in}{4.078939in}}%
\pgfpathlineto{\pgfqpoint{2.732131in}{4.079160in}}%
\pgfpathlineto{\pgfqpoint{2.737340in}{4.079378in}}%
\pgfpathlineto{\pgfqpoint{2.742549in}{4.079497in}}%
\pgfpathlineto{\pgfqpoint{2.747758in}{4.079805in}}%
\pgfpathlineto{\pgfqpoint{2.752967in}{4.079849in}}%
\pgfpathlineto{\pgfqpoint{2.758176in}{4.079674in}}%
\pgfpathlineto{\pgfqpoint{2.763385in}{4.079974in}}%
\pgfpathlineto{\pgfqpoint{2.768594in}{4.080230in}}%
\pgfpathlineto{\pgfqpoint{2.773803in}{4.079987in}}%
\pgfpathlineto{\pgfqpoint{2.779012in}{4.079515in}}%
\pgfpathlineto{\pgfqpoint{2.784221in}{4.079474in}}%
\pgfpathlineto{\pgfqpoint{2.789430in}{4.079590in}}%
\pgfpathlineto{\pgfqpoint{2.794639in}{4.079456in}}%
\pgfpathlineto{\pgfqpoint{2.799848in}{4.078972in}}%
\pgfpathlineto{\pgfqpoint{2.805057in}{4.078378in}}%
\pgfpathlineto{\pgfqpoint{2.810266in}{4.078916in}}%
\pgfpathlineto{\pgfqpoint{2.815475in}{4.079088in}}%
\pgfpathlineto{\pgfqpoint{2.820684in}{4.079415in}}%
\pgfpathlineto{\pgfqpoint{2.825893in}{4.079742in}}%
\pgfpathlineto{\pgfqpoint{2.831102in}{4.079282in}}%
\pgfpathlineto{\pgfqpoint{2.836311in}{4.079986in}}%
\pgfpathlineto{\pgfqpoint{2.841520in}{4.080119in}}%
\pgfpathlineto{\pgfqpoint{2.846729in}{4.079377in}}%
\pgfpathlineto{\pgfqpoint{2.851938in}{4.079709in}}%
\pgfpathlineto{\pgfqpoint{2.857147in}{4.079609in}}%
\pgfpathlineto{\pgfqpoint{2.862356in}{4.079230in}}%
\pgfpathlineto{\pgfqpoint{2.867565in}{4.078778in}}%
\pgfpathlineto{\pgfqpoint{2.872774in}{4.078793in}}%
\pgfpathlineto{\pgfqpoint{2.877983in}{4.078854in}}%
\pgfpathlineto{\pgfqpoint{2.883192in}{4.079463in}}%
\pgfpathlineto{\pgfqpoint{2.888401in}{4.079318in}}%
\pgfpathlineto{\pgfqpoint{2.893610in}{4.079541in}}%
\pgfpathlineto{\pgfqpoint{2.898819in}{4.080206in}}%
\pgfpathlineto{\pgfqpoint{2.904028in}{4.079937in}}%
\pgfpathlineto{\pgfqpoint{2.909238in}{4.079918in}}%
\pgfpathlineto{\pgfqpoint{2.914447in}{4.080444in}}%
\pgfpathlineto{\pgfqpoint{2.919656in}{4.080968in}}%
\pgfpathlineto{\pgfqpoint{2.924865in}{4.081285in}}%
\pgfpathlineto{\pgfqpoint{2.930074in}{4.081862in}}%
\pgfpathlineto{\pgfqpoint{2.935283in}{4.081909in}}%
\pgfpathlineto{\pgfqpoint{2.940492in}{4.082726in}}%
\pgfpathlineto{\pgfqpoint{2.945701in}{4.083853in}}%
\pgfpathlineto{\pgfqpoint{2.950910in}{4.084609in}}%
\pgfpathlineto{\pgfqpoint{2.956119in}{4.086346in}}%
\pgfpathlineto{\pgfqpoint{2.961328in}{4.088590in}}%
\pgfpathlineto{\pgfqpoint{2.966537in}{4.090602in}}%
\pgfpathlineto{\pgfqpoint{2.971746in}{4.091709in}}%
\pgfpathlineto{\pgfqpoint{2.971746in}{4.091709in}}%
\pgfpathlineto{\pgfqpoint{2.971746in}{4.091709in}}%
\pgfpathlineto{\pgfqpoint{2.966537in}{4.090603in}}%
\pgfpathlineto{\pgfqpoint{2.961328in}{4.088590in}}%
\pgfpathlineto{\pgfqpoint{2.956119in}{4.086347in}}%
\pgfpathlineto{\pgfqpoint{2.950910in}{4.084610in}}%
\pgfpathlineto{\pgfqpoint{2.945701in}{4.083853in}}%
\pgfpathlineto{\pgfqpoint{2.940492in}{4.082727in}}%
\pgfpathlineto{\pgfqpoint{2.935283in}{4.081909in}}%
\pgfpathlineto{\pgfqpoint{2.930074in}{4.081863in}}%
\pgfpathlineto{\pgfqpoint{2.924865in}{4.081285in}}%
\pgfpathlineto{\pgfqpoint{2.919656in}{4.080968in}}%
\pgfpathlineto{\pgfqpoint{2.914447in}{4.080445in}}%
\pgfpathlineto{\pgfqpoint{2.909238in}{4.079918in}}%
\pgfpathlineto{\pgfqpoint{2.904028in}{4.079937in}}%
\pgfpathlineto{\pgfqpoint{2.898819in}{4.080206in}}%
\pgfpathlineto{\pgfqpoint{2.893610in}{4.079541in}}%
\pgfpathlineto{\pgfqpoint{2.888401in}{4.079319in}}%
\pgfpathlineto{\pgfqpoint{2.883192in}{4.079463in}}%
\pgfpathlineto{\pgfqpoint{2.877983in}{4.078854in}}%
\pgfpathlineto{\pgfqpoint{2.872774in}{4.078794in}}%
\pgfpathlineto{\pgfqpoint{2.867565in}{4.078779in}}%
\pgfpathlineto{\pgfqpoint{2.862356in}{4.079231in}}%
\pgfpathlineto{\pgfqpoint{2.857147in}{4.079610in}}%
\pgfpathlineto{\pgfqpoint{2.851938in}{4.079710in}}%
\pgfpathlineto{\pgfqpoint{2.846729in}{4.079378in}}%
\pgfpathlineto{\pgfqpoint{2.841520in}{4.080119in}}%
\pgfpathlineto{\pgfqpoint{2.836311in}{4.079987in}}%
\pgfpathlineto{\pgfqpoint{2.831102in}{4.079282in}}%
\pgfpathlineto{\pgfqpoint{2.825893in}{4.079742in}}%
\pgfpathlineto{\pgfqpoint{2.820684in}{4.079415in}}%
\pgfpathlineto{\pgfqpoint{2.815475in}{4.079089in}}%
\pgfpathlineto{\pgfqpoint{2.810266in}{4.078916in}}%
\pgfpathlineto{\pgfqpoint{2.805057in}{4.078378in}}%
\pgfpathlineto{\pgfqpoint{2.799848in}{4.078972in}}%
\pgfpathlineto{\pgfqpoint{2.794639in}{4.079457in}}%
\pgfpathlineto{\pgfqpoint{2.789430in}{4.079591in}}%
\pgfpathlineto{\pgfqpoint{2.784221in}{4.079475in}}%
\pgfpathlineto{\pgfqpoint{2.779012in}{4.079515in}}%
\pgfpathlineto{\pgfqpoint{2.773803in}{4.079987in}}%
\pgfpathlineto{\pgfqpoint{2.768594in}{4.080231in}}%
\pgfpathlineto{\pgfqpoint{2.763385in}{4.079975in}}%
\pgfpathlineto{\pgfqpoint{2.758176in}{4.079674in}}%
\pgfpathlineto{\pgfqpoint{2.752967in}{4.079849in}}%
\pgfpathlineto{\pgfqpoint{2.747758in}{4.079806in}}%
\pgfpathlineto{\pgfqpoint{2.742549in}{4.079498in}}%
\pgfpathlineto{\pgfqpoint{2.737340in}{4.079378in}}%
\pgfpathlineto{\pgfqpoint{2.732131in}{4.079160in}}%
\pgfpathlineto{\pgfqpoint{2.726922in}{4.078940in}}%
\pgfpathlineto{\pgfqpoint{2.721713in}{4.078678in}}%
\pgfpathlineto{\pgfqpoint{2.716504in}{4.079264in}}%
\pgfpathlineto{\pgfqpoint{2.711295in}{4.079248in}}%
\pgfpathlineto{\pgfqpoint{2.706086in}{4.079649in}}%
\pgfpathlineto{\pgfqpoint{2.700877in}{4.079560in}}%
\pgfpathlineto{\pgfqpoint{2.695668in}{4.080249in}}%
\pgfpathlineto{\pgfqpoint{2.690459in}{4.079755in}}%
\pgfpathlineto{\pgfqpoint{2.685250in}{4.080050in}}%
\pgfpathlineto{\pgfqpoint{2.680041in}{4.079980in}}%
\pgfpathlineto{\pgfqpoint{2.674832in}{4.079686in}}%
\pgfpathlineto{\pgfqpoint{2.669623in}{4.079187in}}%
\pgfpathlineto{\pgfqpoint{2.664414in}{4.079649in}}%
\pgfpathlineto{\pgfqpoint{2.659205in}{4.080300in}}%
\pgfpathlineto{\pgfqpoint{2.653996in}{4.079479in}}%
\pgfpathlineto{\pgfqpoint{2.648787in}{4.079712in}}%
\pgfpathlineto{\pgfqpoint{2.643578in}{4.080240in}}%
\pgfpathlineto{\pgfqpoint{2.638369in}{4.080448in}}%
\pgfpathlineto{\pgfqpoint{2.633160in}{4.080435in}}%
\pgfpathlineto{\pgfqpoint{2.627951in}{4.079920in}}%
\pgfpathlineto{\pgfqpoint{2.622742in}{4.078733in}}%
\pgfpathlineto{\pgfqpoint{2.617533in}{4.078596in}}%
\pgfpathlineto{\pgfqpoint{2.612324in}{4.079507in}}%
\pgfpathlineto{\pgfqpoint{2.607115in}{4.079475in}}%
\pgfpathlineto{\pgfqpoint{2.601906in}{4.078599in}}%
\pgfpathlineto{\pgfqpoint{2.596697in}{4.079497in}}%
\pgfpathlineto{\pgfqpoint{2.591487in}{4.081220in}}%
\pgfpathlineto{\pgfqpoint{2.586278in}{4.081257in}}%
\pgfpathlineto{\pgfqpoint{2.581069in}{4.081245in}}%
\pgfpathlineto{\pgfqpoint{2.575860in}{4.080425in}}%
\pgfpathlineto{\pgfqpoint{2.570651in}{4.080764in}}%
\pgfpathlineto{\pgfqpoint{2.565442in}{4.080178in}}%
\pgfpathlineto{\pgfqpoint{2.560233in}{4.079928in}}%
\pgfpathlineto{\pgfqpoint{2.555024in}{4.080337in}}%
\pgfpathlineto{\pgfqpoint{2.549815in}{4.080285in}}%
\pgfpathlineto{\pgfqpoint{2.544606in}{4.080146in}}%
\pgfpathlineto{\pgfqpoint{2.539397in}{4.079790in}}%
\pgfpathlineto{\pgfqpoint{2.534188in}{4.080270in}}%
\pgfpathlineto{\pgfqpoint{2.528979in}{4.080496in}}%
\pgfpathlineto{\pgfqpoint{2.523770in}{4.079910in}}%
\pgfpathlineto{\pgfqpoint{2.518561in}{4.080071in}}%
\pgfpathlineto{\pgfqpoint{2.513352in}{4.079923in}}%
\pgfpathlineto{\pgfqpoint{2.508143in}{4.079255in}}%
\pgfpathlineto{\pgfqpoint{2.502934in}{4.079582in}}%
\pgfpathlineto{\pgfqpoint{2.497725in}{4.079381in}}%
\pgfpathlineto{\pgfqpoint{2.492516in}{4.079000in}}%
\pgfpathlineto{\pgfqpoint{2.487307in}{4.079071in}}%
\pgfpathlineto{\pgfqpoint{2.482098in}{4.079742in}}%
\pgfpathlineto{\pgfqpoint{2.476889in}{4.079912in}}%
\pgfpathlineto{\pgfqpoint{2.471680in}{4.079774in}}%
\pgfpathlineto{\pgfqpoint{2.466471in}{4.079963in}}%
\pgfpathlineto{\pgfqpoint{2.461262in}{4.080510in}}%
\pgfpathlineto{\pgfqpoint{2.456053in}{4.079870in}}%
\pgfpathlineto{\pgfqpoint{2.450844in}{4.079440in}}%
\pgfpathlineto{\pgfqpoint{2.445635in}{4.079945in}}%
\pgfpathlineto{\pgfqpoint{2.440426in}{4.079999in}}%
\pgfpathlineto{\pgfqpoint{2.435217in}{4.079601in}}%
\pgfpathlineto{\pgfqpoint{2.430008in}{4.080290in}}%
\pgfpathlineto{\pgfqpoint{2.424799in}{4.080445in}}%
\pgfpathlineto{\pgfqpoint{2.419590in}{4.080749in}}%
\pgfpathlineto{\pgfqpoint{2.414381in}{4.080669in}}%
\pgfpathlineto{\pgfqpoint{2.409172in}{4.081531in}}%
\pgfpathlineto{\pgfqpoint{2.403963in}{4.081819in}}%
\pgfpathlineto{\pgfqpoint{2.398754in}{4.081603in}}%
\pgfpathlineto{\pgfqpoint{2.393545in}{4.081585in}}%
\pgfpathlineto{\pgfqpoint{2.388336in}{4.081587in}}%
\pgfpathlineto{\pgfqpoint{2.383127in}{4.081029in}}%
\pgfpathlineto{\pgfqpoint{2.377918in}{4.080450in}}%
\pgfpathlineto{\pgfqpoint{2.372709in}{4.079551in}}%
\pgfpathlineto{\pgfqpoint{2.367500in}{4.080100in}}%
\pgfpathlineto{\pgfqpoint{2.362291in}{4.079300in}}%
\pgfpathlineto{\pgfqpoint{2.357082in}{4.079475in}}%
\pgfpathlineto{\pgfqpoint{2.351873in}{4.079989in}}%
\pgfpathlineto{\pgfqpoint{2.346664in}{4.080052in}}%
\pgfpathlineto{\pgfqpoint{2.341455in}{4.080220in}}%
\pgfpathlineto{\pgfqpoint{2.336246in}{4.080719in}}%
\pgfpathlineto{\pgfqpoint{2.331037in}{4.081016in}}%
\pgfpathlineto{\pgfqpoint{2.325828in}{4.080874in}}%
\pgfpathlineto{\pgfqpoint{2.320619in}{4.081024in}}%
\pgfpathlineto{\pgfqpoint{2.315410in}{4.080991in}}%
\pgfpathlineto{\pgfqpoint{2.310201in}{4.081017in}}%
\pgfpathlineto{\pgfqpoint{2.304992in}{4.080672in}}%
\pgfpathlineto{\pgfqpoint{2.299783in}{4.081008in}}%
\pgfpathlineto{\pgfqpoint{2.294574in}{4.080612in}}%
\pgfpathlineto{\pgfqpoint{2.289365in}{4.080869in}}%
\pgfpathlineto{\pgfqpoint{2.284155in}{4.081737in}}%
\pgfpathlineto{\pgfqpoint{2.278946in}{4.081745in}}%
\pgfpathlineto{\pgfqpoint{2.273737in}{4.081883in}}%
\pgfpathlineto{\pgfqpoint{2.268528in}{4.081698in}}%
\pgfpathlineto{\pgfqpoint{2.263319in}{4.081332in}}%
\pgfpathlineto{\pgfqpoint{2.258110in}{4.081162in}}%
\pgfpathlineto{\pgfqpoint{2.252901in}{4.081003in}}%
\pgfpathlineto{\pgfqpoint{2.247692in}{4.080882in}}%
\pgfpathlineto{\pgfqpoint{2.242483in}{4.081501in}}%
\pgfpathlineto{\pgfqpoint{2.237274in}{4.081328in}}%
\pgfpathlineto{\pgfqpoint{2.232065in}{4.080394in}}%
\pgfpathlineto{\pgfqpoint{2.226856in}{4.080587in}}%
\pgfpathlineto{\pgfqpoint{2.221647in}{4.081064in}}%
\pgfpathlineto{\pgfqpoint{2.216438in}{4.081331in}}%
\pgfpathlineto{\pgfqpoint{2.211229in}{4.081553in}}%
\pgfpathlineto{\pgfqpoint{2.206020in}{4.081453in}}%
\pgfpathlineto{\pgfqpoint{2.200811in}{4.081223in}}%
\pgfpathlineto{\pgfqpoint{2.195602in}{4.081435in}}%
\pgfpathlineto{\pgfqpoint{2.190393in}{4.080861in}}%
\pgfpathlineto{\pgfqpoint{2.185184in}{4.080959in}}%
\pgfpathlineto{\pgfqpoint{2.179975in}{4.081364in}}%
\pgfpathlineto{\pgfqpoint{2.174766in}{4.081287in}}%
\pgfpathlineto{\pgfqpoint{2.169557in}{4.081919in}}%
\pgfpathlineto{\pgfqpoint{2.164348in}{4.082542in}}%
\pgfpathlineto{\pgfqpoint{2.159139in}{4.082405in}}%
\pgfpathlineto{\pgfqpoint{2.153930in}{4.082373in}}%
\pgfpathlineto{\pgfqpoint{2.148721in}{4.082159in}}%
\pgfpathlineto{\pgfqpoint{2.143512in}{4.082130in}}%
\pgfpathlineto{\pgfqpoint{2.138303in}{4.082705in}}%
\pgfpathlineto{\pgfqpoint{2.133094in}{4.082839in}}%
\pgfpathlineto{\pgfqpoint{2.127885in}{4.082944in}}%
\pgfpathlineto{\pgfqpoint{2.122676in}{4.082787in}}%
\pgfpathlineto{\pgfqpoint{2.117467in}{4.083029in}}%
\pgfpathlineto{\pgfqpoint{2.112258in}{4.083265in}}%
\pgfpathlineto{\pgfqpoint{2.107049in}{4.083601in}}%
\pgfpathlineto{\pgfqpoint{2.101840in}{4.082989in}}%
\pgfpathlineto{\pgfqpoint{2.096631in}{4.082995in}}%
\pgfpathlineto{\pgfqpoint{2.091422in}{4.083292in}}%
\pgfpathlineto{\pgfqpoint{2.086213in}{4.083202in}}%
\pgfpathlineto{\pgfqpoint{2.081004in}{4.083007in}}%
\pgfpathlineto{\pgfqpoint{2.075795in}{4.083187in}}%
\pgfpathlineto{\pgfqpoint{2.070586in}{4.083187in}}%
\pgfpathlineto{\pgfqpoint{2.065377in}{4.083044in}}%
\pgfpathlineto{\pgfqpoint{2.060168in}{4.082556in}}%
\pgfpathlineto{\pgfqpoint{2.054959in}{4.082449in}}%
\pgfpathlineto{\pgfqpoint{2.049750in}{4.083069in}}%
\pgfpathlineto{\pgfqpoint{2.044541in}{4.083936in}}%
\pgfpathlineto{\pgfqpoint{2.039332in}{4.083699in}}%
\pgfpathlineto{\pgfqpoint{2.034123in}{4.083397in}}%
\pgfpathlineto{\pgfqpoint{2.028914in}{4.082885in}}%
\pgfpathlineto{\pgfqpoint{2.023705in}{4.082800in}}%
\pgfpathlineto{\pgfqpoint{2.018496in}{4.083645in}}%
\pgfpathlineto{\pgfqpoint{2.013287in}{4.083749in}}%
\pgfpathlineto{\pgfqpoint{2.008078in}{4.083743in}}%
\pgfpathlineto{\pgfqpoint{2.002869in}{4.083565in}}%
\pgfpathlineto{\pgfqpoint{1.997660in}{4.083100in}}%
\pgfpathlineto{\pgfqpoint{1.992451in}{4.082834in}}%
\pgfpathlineto{\pgfqpoint{1.987242in}{4.082478in}}%
\pgfpathlineto{\pgfqpoint{1.982033in}{4.083000in}}%
\pgfpathlineto{\pgfqpoint{1.976824in}{4.083329in}}%
\pgfpathlineto{\pgfqpoint{1.971614in}{4.083328in}}%
\pgfpathlineto{\pgfqpoint{1.966405in}{4.083370in}}%
\pgfpathlineto{\pgfqpoint{1.961196in}{4.083542in}}%
\pgfpathlineto{\pgfqpoint{1.955987in}{4.084389in}}%
\pgfpathlineto{\pgfqpoint{1.950778in}{4.085460in}}%
\pgfpathlineto{\pgfqpoint{1.945569in}{4.085860in}}%
\pgfpathlineto{\pgfqpoint{1.940360in}{4.086303in}}%
\pgfpathlineto{\pgfqpoint{1.935151in}{4.086288in}}%
\pgfpathlineto{\pgfqpoint{1.929942in}{4.085094in}}%
\pgfpathlineto{\pgfqpoint{1.924733in}{4.085600in}}%
\pgfpathlineto{\pgfqpoint{1.919524in}{4.085439in}}%
\pgfpathlineto{\pgfqpoint{1.914315in}{4.085014in}}%
\pgfpathlineto{\pgfqpoint{1.909106in}{4.084642in}}%
\pgfpathlineto{\pgfqpoint{1.903897in}{4.084727in}}%
\pgfpathlineto{\pgfqpoint{1.898688in}{4.084954in}}%
\pgfpathlineto{\pgfqpoint{1.893479in}{4.085681in}}%
\pgfpathlineto{\pgfqpoint{1.888270in}{4.085315in}}%
\pgfpathlineto{\pgfqpoint{1.883061in}{4.085718in}}%
\pgfpathlineto{\pgfqpoint{1.877852in}{4.086233in}}%
\pgfpathlineto{\pgfqpoint{1.872643in}{4.086322in}}%
\pgfpathlineto{\pgfqpoint{1.867434in}{4.086051in}}%
\pgfpathlineto{\pgfqpoint{1.862225in}{4.086914in}}%
\pgfpathlineto{\pgfqpoint{1.857016in}{4.087140in}}%
\pgfpathlineto{\pgfqpoint{1.851807in}{4.086930in}}%
\pgfpathlineto{\pgfqpoint{1.846598in}{4.087352in}}%
\pgfpathlineto{\pgfqpoint{1.841389in}{4.087835in}}%
\pgfpathlineto{\pgfqpoint{1.836180in}{4.088289in}}%
\pgfpathlineto{\pgfqpoint{1.830971in}{4.087478in}}%
\pgfpathlineto{\pgfqpoint{1.825762in}{4.087537in}}%
\pgfpathlineto{\pgfqpoint{1.820553in}{4.087835in}}%
\pgfpathlineto{\pgfqpoint{1.815344in}{4.088478in}}%
\pgfpathlineto{\pgfqpoint{1.810135in}{4.089378in}}%
\pgfpathlineto{\pgfqpoint{1.804926in}{4.089256in}}%
\pgfpathlineto{\pgfqpoint{1.799717in}{4.089174in}}%
\pgfpathlineto{\pgfqpoint{1.794508in}{4.088998in}}%
\pgfpathlineto{\pgfqpoint{1.789299in}{4.088781in}}%
\pgfpathlineto{\pgfqpoint{1.784090in}{4.087666in}}%
\pgfpathlineto{\pgfqpoint{1.778881in}{4.087944in}}%
\pgfpathlineto{\pgfqpoint{1.773672in}{4.088812in}}%
\pgfpathlineto{\pgfqpoint{1.768463in}{4.089673in}}%
\pgfpathlineto{\pgfqpoint{1.763254in}{4.089858in}}%
\pgfpathlineto{\pgfqpoint{1.758045in}{4.090810in}}%
\pgfpathlineto{\pgfqpoint{1.752836in}{4.090592in}}%
\pgfpathlineto{\pgfqpoint{1.747627in}{4.089844in}}%
\pgfpathlineto{\pgfqpoint{1.742418in}{4.089396in}}%
\pgfpathlineto{\pgfqpoint{1.737209in}{4.089821in}}%
\pgfpathlineto{\pgfqpoint{1.732000in}{4.090361in}}%
\pgfpathlineto{\pgfqpoint{1.726791in}{4.090875in}}%
\pgfpathlineto{\pgfqpoint{1.721582in}{4.090773in}}%
\pgfpathlineto{\pgfqpoint{1.716373in}{4.090867in}}%
\pgfpathlineto{\pgfqpoint{1.711164in}{4.091567in}}%
\pgfpathlineto{\pgfqpoint{1.705955in}{4.091576in}}%
\pgfpathlineto{\pgfqpoint{1.700746in}{4.091656in}}%
\pgfpathlineto{\pgfqpoint{1.695537in}{4.091561in}}%
\pgfpathlineto{\pgfqpoint{1.690328in}{4.091656in}}%
\pgfpathlineto{\pgfqpoint{1.685119in}{4.091602in}}%
\pgfpathlineto{\pgfqpoint{1.679910in}{4.092573in}}%
\pgfpathlineto{\pgfqpoint{1.674701in}{4.093348in}}%
\pgfpathlineto{\pgfqpoint{1.669492in}{4.093181in}}%
\pgfpathlineto{\pgfqpoint{1.664282in}{4.093485in}}%
\pgfpathlineto{\pgfqpoint{1.659073in}{4.094544in}}%
\pgfpathlineto{\pgfqpoint{1.653864in}{4.095344in}}%
\pgfpathlineto{\pgfqpoint{1.648655in}{4.095008in}}%
\pgfpathlineto{\pgfqpoint{1.643446in}{4.094247in}}%
\pgfpathlineto{\pgfqpoint{1.638237in}{4.094511in}}%
\pgfpathlineto{\pgfqpoint{1.633028in}{4.094199in}}%
\pgfpathlineto{\pgfqpoint{1.627819in}{4.094184in}}%
\pgfpathlineto{\pgfqpoint{1.622610in}{4.094503in}}%
\pgfpathlineto{\pgfqpoint{1.617401in}{4.095153in}}%
\pgfpathlineto{\pgfqpoint{1.612192in}{4.096065in}}%
\pgfpathlineto{\pgfqpoint{1.606983in}{4.097046in}}%
\pgfpathlineto{\pgfqpoint{1.601774in}{4.097027in}}%
\pgfpathlineto{\pgfqpoint{1.596565in}{4.096820in}}%
\pgfpathlineto{\pgfqpoint{1.591356in}{4.096757in}}%
\pgfpathlineto{\pgfqpoint{1.586147in}{4.097000in}}%
\pgfpathlineto{\pgfqpoint{1.580938in}{4.096686in}}%
\pgfpathlineto{\pgfqpoint{1.575729in}{4.096434in}}%
\pgfpathlineto{\pgfqpoint{1.570520in}{4.096968in}}%
\pgfpathlineto{\pgfqpoint{1.565311in}{4.097550in}}%
\pgfpathlineto{\pgfqpoint{1.560102in}{4.097753in}}%
\pgfpathlineto{\pgfqpoint{1.554893in}{4.098024in}}%
\pgfpathlineto{\pgfqpoint{1.549684in}{4.098167in}}%
\pgfpathlineto{\pgfqpoint{1.544475in}{4.099169in}}%
\pgfpathlineto{\pgfqpoint{1.539266in}{4.099810in}}%
\pgfpathlineto{\pgfqpoint{1.534057in}{4.099789in}}%
\pgfpathlineto{\pgfqpoint{1.528848in}{4.100081in}}%
\pgfpathlineto{\pgfqpoint{1.523639in}{4.100814in}}%
\pgfpathlineto{\pgfqpoint{1.518430in}{4.101419in}}%
\pgfpathlineto{\pgfqpoint{1.513221in}{4.101554in}}%
\pgfpathlineto{\pgfqpoint{1.508012in}{4.101683in}}%
\pgfpathlineto{\pgfqpoint{1.502803in}{4.102098in}}%
\pgfpathlineto{\pgfqpoint{1.497594in}{4.101225in}}%
\pgfpathlineto{\pgfqpoint{1.492385in}{4.101974in}}%
\pgfpathlineto{\pgfqpoint{1.487176in}{4.102765in}}%
\pgfpathlineto{\pgfqpoint{1.481967in}{4.103592in}}%
\pgfpathlineto{\pgfqpoint{1.476758in}{4.103714in}}%
\pgfpathlineto{\pgfqpoint{1.471549in}{4.103712in}}%
\pgfpathlineto{\pgfqpoint{1.466340in}{4.104571in}}%
\pgfpathlineto{\pgfqpoint{1.461131in}{4.105699in}}%
\pgfpathlineto{\pgfqpoint{1.455922in}{4.106904in}}%
\pgfpathlineto{\pgfqpoint{1.450713in}{4.107262in}}%
\pgfpathlineto{\pgfqpoint{1.445504in}{4.107814in}}%
\pgfpathlineto{\pgfqpoint{1.440295in}{4.107906in}}%
\pgfpathlineto{\pgfqpoint{1.435086in}{4.108205in}}%
\pgfpathlineto{\pgfqpoint{1.429877in}{4.108923in}}%
\pgfpathlineto{\pgfqpoint{1.424668in}{4.109221in}}%
\pgfpathlineto{\pgfqpoint{1.419459in}{4.110305in}}%
\pgfpathlineto{\pgfqpoint{1.414250in}{4.110390in}}%
\pgfpathlineto{\pgfqpoint{1.409041in}{4.110594in}}%
\pgfpathlineto{\pgfqpoint{1.403832in}{4.110326in}}%
\pgfpathlineto{\pgfqpoint{1.398623in}{4.109557in}}%
\pgfpathlineto{\pgfqpoint{1.393414in}{4.109762in}}%
\pgfpathlineto{\pgfqpoint{1.388205in}{4.111110in}}%
\pgfpathlineto{\pgfqpoint{1.382996in}{4.112350in}}%
\pgfpathlineto{\pgfqpoint{1.377787in}{4.113671in}}%
\pgfpathlineto{\pgfqpoint{1.372578in}{4.115178in}}%
\pgfpathlineto{\pgfqpoint{1.367369in}{4.115392in}}%
\pgfpathlineto{\pgfqpoint{1.362160in}{4.115278in}}%
\pgfpathlineto{\pgfqpoint{1.356951in}{4.116300in}}%
\pgfpathlineto{\pgfqpoint{1.351741in}{4.117253in}}%
\pgfpathlineto{\pgfqpoint{1.346532in}{4.118222in}}%
\pgfpathlineto{\pgfqpoint{1.341323in}{4.117721in}}%
\pgfpathlineto{\pgfqpoint{1.336114in}{4.116386in}}%
\pgfpathlineto{\pgfqpoint{1.330905in}{4.116975in}}%
\pgfpathlineto{\pgfqpoint{1.325696in}{4.117476in}}%
\pgfpathlineto{\pgfqpoint{1.320487in}{4.117479in}}%
\pgfpathlineto{\pgfqpoint{1.315278in}{4.117673in}}%
\pgfpathlineto{\pgfqpoint{1.310069in}{4.118930in}}%
\pgfpathlineto{\pgfqpoint{1.304860in}{4.120985in}}%
\pgfpathlineto{\pgfqpoint{1.299651in}{4.121032in}}%
\pgfpathlineto{\pgfqpoint{1.294442in}{4.121866in}}%
\pgfpathlineto{\pgfqpoint{1.289233in}{4.122504in}}%
\pgfpathlineto{\pgfqpoint{1.284024in}{4.122778in}}%
\pgfpathlineto{\pgfqpoint{1.278815in}{4.123857in}}%
\pgfpathlineto{\pgfqpoint{1.273606in}{4.125133in}}%
\pgfpathlineto{\pgfqpoint{1.268397in}{4.126836in}}%
\pgfpathlineto{\pgfqpoint{1.263188in}{4.128289in}}%
\pgfpathlineto{\pgfqpoint{1.257979in}{4.130100in}}%
\pgfpathlineto{\pgfqpoint{1.252770in}{4.132817in}}%
\pgfpathlineto{\pgfqpoint{1.247561in}{4.137583in}}%
\pgfpathlineto{\pgfqpoint{1.242352in}{4.145406in}}%
\pgfpathlineto{\pgfqpoint{1.237143in}{4.158980in}}%
\pgfpathlineto{\pgfqpoint{1.231934in}{4.179188in}}%
\pgfpathlineto{\pgfqpoint{1.226725in}{4.207395in}}%
\pgfpathlineto{\pgfqpoint{1.221516in}{4.250942in}}%
\pgfpathlineto{\pgfqpoint{1.216307in}{4.310386in}}%
\pgfpathlineto{\pgfqpoint{1.211098in}{4.379589in}}%
\pgfpathlineto{\pgfqpoint{1.205889in}{4.444602in}}%
\pgfpathlineto{\pgfqpoint{1.200680in}{4.490993in}}%
\pgfpathlineto{\pgfqpoint{1.195471in}{4.512649in}}%
\pgfpathclose%
\pgfusepath{stroke,fill}%
\end{pgfscope}%
\begin{pgfscope}%
\pgfpathrectangle{\pgfqpoint{0.898557in}{4.050417in}}{\pgfqpoint{2.323221in}{1.698958in}} %
\pgfusepath{clip}%
\pgfsetrectcap%
\pgfsetroundjoin%
\pgfsetlinewidth{1.003750pt}%
\definecolor{currentstroke}{rgb}{1.000000,0.400000,0.200000}%
\pgfsetstrokecolor{currentstroke}%
\pgfsetdash{}{0pt}%
\pgfpathmoveto{\pgfqpoint{1.195471in}{4.512649in}}%
\pgfpathlineto{\pgfqpoint{1.200680in}{4.490993in}}%
\pgfpathlineto{\pgfqpoint{1.205889in}{4.444602in}}%
\pgfpathlineto{\pgfqpoint{1.221516in}{4.250942in}}%
\pgfpathlineto{\pgfqpoint{1.226725in}{4.207395in}}%
\pgfpathlineto{\pgfqpoint{1.231934in}{4.179188in}}%
\pgfpathlineto{\pgfqpoint{1.237143in}{4.158980in}}%
\pgfpathlineto{\pgfqpoint{1.242352in}{4.145406in}}%
\pgfpathlineto{\pgfqpoint{1.247561in}{4.137583in}}%
\pgfpathlineto{\pgfqpoint{1.252770in}{4.132817in}}%
\pgfpathlineto{\pgfqpoint{1.263188in}{4.128289in}}%
\pgfpathlineto{\pgfqpoint{1.284024in}{4.122778in}}%
\pgfpathlineto{\pgfqpoint{1.304860in}{4.120985in}}%
\pgfpathlineto{\pgfqpoint{1.315278in}{4.117673in}}%
\pgfpathlineto{\pgfqpoint{1.341323in}{4.117721in}}%
\pgfpathlineto{\pgfqpoint{1.346532in}{4.118222in}}%
\pgfpathlineto{\pgfqpoint{1.367369in}{4.115392in}}%
\pgfpathlineto{\pgfqpoint{1.372578in}{4.115178in}}%
\pgfpathlineto{\pgfqpoint{1.393414in}{4.109762in}}%
\pgfpathlineto{\pgfqpoint{1.403832in}{4.110326in}}%
\pgfpathlineto{\pgfqpoint{1.419459in}{4.110305in}}%
\pgfpathlineto{\pgfqpoint{1.429877in}{4.108923in}}%
\pgfpathlineto{\pgfqpoint{1.445504in}{4.107814in}}%
\pgfpathlineto{\pgfqpoint{1.466340in}{4.104571in}}%
\pgfpathlineto{\pgfqpoint{1.476758in}{4.103714in}}%
\pgfpathlineto{\pgfqpoint{1.487176in}{4.102765in}}%
\pgfpathlineto{\pgfqpoint{1.497594in}{4.101225in}}%
\pgfpathlineto{\pgfqpoint{1.502803in}{4.102098in}}%
\pgfpathlineto{\pgfqpoint{1.580938in}{4.096686in}}%
\pgfpathlineto{\pgfqpoint{1.601774in}{4.097027in}}%
\pgfpathlineto{\pgfqpoint{1.612192in}{4.096065in}}%
\pgfpathlineto{\pgfqpoint{1.627819in}{4.094184in}}%
\pgfpathlineto{\pgfqpoint{1.664282in}{4.093485in}}%
\pgfpathlineto{\pgfqpoint{1.742418in}{4.089396in}}%
\pgfpathlineto{\pgfqpoint{1.758045in}{4.090810in}}%
\pgfpathlineto{\pgfqpoint{1.778881in}{4.087944in}}%
\pgfpathlineto{\pgfqpoint{1.784090in}{4.087666in}}%
\pgfpathlineto{\pgfqpoint{1.794508in}{4.088998in}}%
\pgfpathlineto{\pgfqpoint{1.810135in}{4.089378in}}%
\pgfpathlineto{\pgfqpoint{1.825762in}{4.087537in}}%
\pgfpathlineto{\pgfqpoint{1.877852in}{4.086233in}}%
\pgfpathlineto{\pgfqpoint{1.903897in}{4.084727in}}%
\pgfpathlineto{\pgfqpoint{1.919524in}{4.085439in}}%
\pgfpathlineto{\pgfqpoint{1.950778in}{4.085460in}}%
\pgfpathlineto{\pgfqpoint{1.966405in}{4.083370in}}%
\pgfpathlineto{\pgfqpoint{2.039332in}{4.083699in}}%
\pgfpathlineto{\pgfqpoint{2.049750in}{4.083069in}}%
\pgfpathlineto{\pgfqpoint{2.060168in}{4.082556in}}%
\pgfpathlineto{\pgfqpoint{2.075795in}{4.083187in}}%
\pgfpathlineto{\pgfqpoint{2.101840in}{4.082989in}}%
\pgfpathlineto{\pgfqpoint{2.112258in}{4.083265in}}%
\pgfpathlineto{\pgfqpoint{2.153930in}{4.082373in}}%
\pgfpathlineto{\pgfqpoint{2.169557in}{4.081919in}}%
\pgfpathlineto{\pgfqpoint{2.185184in}{4.080959in}}%
\pgfpathlineto{\pgfqpoint{2.232065in}{4.080394in}}%
\pgfpathlineto{\pgfqpoint{2.242483in}{4.081501in}}%
\pgfpathlineto{\pgfqpoint{2.252901in}{4.081002in}}%
\pgfpathlineto{\pgfqpoint{2.284155in}{4.081737in}}%
\pgfpathlineto{\pgfqpoint{2.294574in}{4.080612in}}%
\pgfpathlineto{\pgfqpoint{2.310201in}{4.081017in}}%
\pgfpathlineto{\pgfqpoint{2.351873in}{4.079989in}}%
\pgfpathlineto{\pgfqpoint{2.362291in}{4.079300in}}%
\pgfpathlineto{\pgfqpoint{2.367500in}{4.080100in}}%
\pgfpathlineto{\pgfqpoint{2.372709in}{4.079551in}}%
\pgfpathlineto{\pgfqpoint{2.388336in}{4.081587in}}%
\pgfpathlineto{\pgfqpoint{2.419590in}{4.080749in}}%
\pgfpathlineto{\pgfqpoint{2.450844in}{4.079440in}}%
\pgfpathlineto{\pgfqpoint{2.466471in}{4.079963in}}%
\pgfpathlineto{\pgfqpoint{2.508143in}{4.079255in}}%
\pgfpathlineto{\pgfqpoint{2.518561in}{4.080071in}}%
\pgfpathlineto{\pgfqpoint{2.544606in}{4.080146in}}%
\pgfpathlineto{\pgfqpoint{2.591487in}{4.081220in}}%
\pgfpathlineto{\pgfqpoint{2.601906in}{4.078599in}}%
\pgfpathlineto{\pgfqpoint{2.612324in}{4.079507in}}%
\pgfpathlineto{\pgfqpoint{2.622742in}{4.078733in}}%
\pgfpathlineto{\pgfqpoint{2.633160in}{4.080435in}}%
\pgfpathlineto{\pgfqpoint{2.664414in}{4.079649in}}%
\pgfpathlineto{\pgfqpoint{2.674832in}{4.079685in}}%
\pgfpathlineto{\pgfqpoint{2.695668in}{4.080249in}}%
\pgfpathlineto{\pgfqpoint{2.711295in}{4.079247in}}%
\pgfpathlineto{\pgfqpoint{2.799848in}{4.078972in}}%
\pgfpathlineto{\pgfqpoint{2.805057in}{4.078378in}}%
\pgfpathlineto{\pgfqpoint{2.825893in}{4.079742in}}%
\pgfpathlineto{\pgfqpoint{2.831102in}{4.079282in}}%
\pgfpathlineto{\pgfqpoint{2.841520in}{4.080119in}}%
\pgfpathlineto{\pgfqpoint{2.851938in}{4.079710in}}%
\pgfpathlineto{\pgfqpoint{2.893610in}{4.079541in}}%
\pgfpathlineto{\pgfqpoint{2.904028in}{4.079937in}}%
\pgfpathlineto{\pgfqpoint{2.919656in}{4.080968in}}%
\pgfpathlineto{\pgfqpoint{2.940492in}{4.082726in}}%
\pgfpathlineto{\pgfqpoint{2.956119in}{4.086346in}}%
\pgfpathlineto{\pgfqpoint{2.966537in}{4.090602in}}%
\pgfpathlineto{\pgfqpoint{2.971746in}{4.091709in}}%
\pgfpathlineto{\pgfqpoint{2.971746in}{4.091709in}}%
\pgfusepath{stroke}%
\end{pgfscope}%
\begin{pgfscope}%
\pgfpathrectangle{\pgfqpoint{0.898557in}{4.050417in}}{\pgfqpoint{2.323221in}{1.698958in}} %
\pgfusepath{clip}%
\pgfsetbuttcap%
\pgfsetroundjoin%
\pgfsetlinewidth{1.003750pt}%
\definecolor{currentstroke}{rgb}{0.000000,0.000000,0.000000}%
\pgfsetstrokecolor{currentstroke}%
\pgfsetdash{{1.000000pt}{3.000000pt}}{0.000000pt}%
\pgfpathmoveto{\pgfqpoint{1.195471in}{4.050417in}}%
\pgfpathlineto{\pgfqpoint{1.195471in}{5.749375in}}%
\pgfusepath{stroke}%
\end{pgfscope}%
\begin{pgfscope}%
\pgfsetrectcap%
\pgfsetmiterjoin%
\pgfsetlinewidth{1.003750pt}%
\definecolor{currentstroke}{rgb}{0.000000,0.000000,0.000000}%
\pgfsetstrokecolor{currentstroke}%
\pgfsetdash{}{0pt}%
\pgfpathmoveto{\pgfqpoint{3.221779in}{4.050417in}}%
\pgfpathlineto{\pgfqpoint{3.221779in}{5.749375in}}%
\pgfusepath{stroke}%
\end{pgfscope}%
\begin{pgfscope}%
\pgfsetrectcap%
\pgfsetmiterjoin%
\pgfsetlinewidth{1.003750pt}%
\definecolor{currentstroke}{rgb}{0.000000,0.000000,0.000000}%
\pgfsetstrokecolor{currentstroke}%
\pgfsetdash{}{0pt}%
\pgfpathmoveto{\pgfqpoint{0.898557in}{4.050417in}}%
\pgfpathlineto{\pgfqpoint{0.898557in}{5.749375in}}%
\pgfusepath{stroke}%
\end{pgfscope}%
\begin{pgfscope}%
\pgfsetrectcap%
\pgfsetmiterjoin%
\pgfsetlinewidth{1.003750pt}%
\definecolor{currentstroke}{rgb}{0.000000,0.000000,0.000000}%
\pgfsetstrokecolor{currentstroke}%
\pgfsetdash{}{0pt}%
\pgfpathmoveto{\pgfqpoint{0.898557in}{5.749375in}}%
\pgfpathlineto{\pgfqpoint{3.221779in}{5.749375in}}%
\pgfusepath{stroke}%
\end{pgfscope}%
\begin{pgfscope}%
\pgfsetrectcap%
\pgfsetmiterjoin%
\pgfsetlinewidth{1.003750pt}%
\definecolor{currentstroke}{rgb}{0.000000,0.000000,0.000000}%
\pgfsetstrokecolor{currentstroke}%
\pgfsetdash{}{0pt}%
\pgfpathmoveto{\pgfqpoint{0.898557in}{4.050417in}}%
\pgfpathlineto{\pgfqpoint{3.221779in}{4.050417in}}%
\pgfusepath{stroke}%
\end{pgfscope}%
\begin{pgfscope}%
\pgfsetbuttcap%
\pgfsetroundjoin%
\definecolor{currentfill}{rgb}{0.000000,0.000000,0.000000}%
\pgfsetfillcolor{currentfill}%
\pgfsetlinewidth{0.501875pt}%
\definecolor{currentstroke}{rgb}{0.000000,0.000000,0.000000}%
\pgfsetstrokecolor{currentstroke}%
\pgfsetdash{}{0pt}%
\pgfsys@defobject{currentmarker}{\pgfqpoint{0.000000in}{0.000000in}}{\pgfqpoint{0.000000in}{0.055556in}}{%
\pgfpathmoveto{\pgfqpoint{0.000000in}{0.000000in}}%
\pgfpathlineto{\pgfqpoint{0.000000in}{0.055556in}}%
\pgfusepath{stroke,fill}%
}%
\begin{pgfscope}%
\pgfsys@transformshift{1.195471in}{4.050417in}%
\pgfsys@useobject{currentmarker}{}%
\end{pgfscope}%
\end{pgfscope}%
\begin{pgfscope}%
\pgfsetbuttcap%
\pgfsetroundjoin%
\definecolor{currentfill}{rgb}{0.000000,0.000000,0.000000}%
\pgfsetfillcolor{currentfill}%
\pgfsetlinewidth{0.501875pt}%
\definecolor{currentstroke}{rgb}{0.000000,0.000000,0.000000}%
\pgfsetstrokecolor{currentstroke}%
\pgfsetdash{}{0pt}%
\pgfsys@defobject{currentmarker}{\pgfqpoint{0.000000in}{-0.055556in}}{\pgfqpoint{0.000000in}{0.000000in}}{%
\pgfpathmoveto{\pgfqpoint{0.000000in}{0.000000in}}%
\pgfpathlineto{\pgfqpoint{0.000000in}{-0.055556in}}%
\pgfusepath{stroke,fill}%
}%
\begin{pgfscope}%
\pgfsys@transformshift{1.195471in}{5.749375in}%
\pgfsys@useobject{currentmarker}{}%
\end{pgfscope}%
\end{pgfscope}%
\begin{pgfscope}%
\pgftext[x=1.195471in,y=3.994861in,,top]{\fontsize{11.000000}{13.200000}\selectfont 0}%
\end{pgfscope}%
\begin{pgfscope}%
\pgfsetbuttcap%
\pgfsetroundjoin%
\definecolor{currentfill}{rgb}{0.000000,0.000000,0.000000}%
\pgfsetfillcolor{currentfill}%
\pgfsetlinewidth{0.501875pt}%
\definecolor{currentstroke}{rgb}{0.000000,0.000000,0.000000}%
\pgfsetstrokecolor{currentstroke}%
\pgfsetdash{}{0pt}%
\pgfsys@defobject{currentmarker}{\pgfqpoint{0.000000in}{0.000000in}}{\pgfqpoint{0.000000in}{0.055556in}}{%
\pgfpathmoveto{\pgfqpoint{0.000000in}{0.000000in}}%
\pgfpathlineto{\pgfqpoint{0.000000in}{0.055556in}}%
\pgfusepath{stroke,fill}%
}%
\begin{pgfscope}%
\pgfsys@transformshift{1.820601in}{4.050417in}%
\pgfsys@useobject{currentmarker}{}%
\end{pgfscope}%
\end{pgfscope}%
\begin{pgfscope}%
\pgfsetbuttcap%
\pgfsetroundjoin%
\definecolor{currentfill}{rgb}{0.000000,0.000000,0.000000}%
\pgfsetfillcolor{currentfill}%
\pgfsetlinewidth{0.501875pt}%
\definecolor{currentstroke}{rgb}{0.000000,0.000000,0.000000}%
\pgfsetstrokecolor{currentstroke}%
\pgfsetdash{}{0pt}%
\pgfsys@defobject{currentmarker}{\pgfqpoint{0.000000in}{-0.055556in}}{\pgfqpoint{0.000000in}{0.000000in}}{%
\pgfpathmoveto{\pgfqpoint{0.000000in}{0.000000in}}%
\pgfpathlineto{\pgfqpoint{0.000000in}{-0.055556in}}%
\pgfusepath{stroke,fill}%
}%
\begin{pgfscope}%
\pgfsys@transformshift{1.820601in}{5.749375in}%
\pgfsys@useobject{currentmarker}{}%
\end{pgfscope}%
\end{pgfscope}%
\begin{pgfscope}%
\pgftext[x=1.820601in,y=3.994861in,,top]{\fontsize{11.000000}{13.200000}\selectfont 5}%
\end{pgfscope}%
\begin{pgfscope}%
\pgfsetbuttcap%
\pgfsetroundjoin%
\definecolor{currentfill}{rgb}{0.000000,0.000000,0.000000}%
\pgfsetfillcolor{currentfill}%
\pgfsetlinewidth{0.501875pt}%
\definecolor{currentstroke}{rgb}{0.000000,0.000000,0.000000}%
\pgfsetstrokecolor{currentstroke}%
\pgfsetdash{}{0pt}%
\pgfsys@defobject{currentmarker}{\pgfqpoint{0.000000in}{0.000000in}}{\pgfqpoint{0.000000in}{0.055556in}}{%
\pgfpathmoveto{\pgfqpoint{0.000000in}{0.000000in}}%
\pgfpathlineto{\pgfqpoint{0.000000in}{0.055556in}}%
\pgfusepath{stroke,fill}%
}%
\begin{pgfscope}%
\pgfsys@transformshift{2.445730in}{4.050417in}%
\pgfsys@useobject{currentmarker}{}%
\end{pgfscope}%
\end{pgfscope}%
\begin{pgfscope}%
\pgfsetbuttcap%
\pgfsetroundjoin%
\definecolor{currentfill}{rgb}{0.000000,0.000000,0.000000}%
\pgfsetfillcolor{currentfill}%
\pgfsetlinewidth{0.501875pt}%
\definecolor{currentstroke}{rgb}{0.000000,0.000000,0.000000}%
\pgfsetstrokecolor{currentstroke}%
\pgfsetdash{}{0pt}%
\pgfsys@defobject{currentmarker}{\pgfqpoint{0.000000in}{-0.055556in}}{\pgfqpoint{0.000000in}{0.000000in}}{%
\pgfpathmoveto{\pgfqpoint{0.000000in}{0.000000in}}%
\pgfpathlineto{\pgfqpoint{0.000000in}{-0.055556in}}%
\pgfusepath{stroke,fill}%
}%
\begin{pgfscope}%
\pgfsys@transformshift{2.445730in}{5.749375in}%
\pgfsys@useobject{currentmarker}{}%
\end{pgfscope}%
\end{pgfscope}%
\begin{pgfscope}%
\pgftext[x=2.445730in,y=3.994861in,,top]{\fontsize{11.000000}{13.200000}\selectfont 10}%
\end{pgfscope}%
\begin{pgfscope}%
\pgfsetbuttcap%
\pgfsetroundjoin%
\definecolor{currentfill}{rgb}{0.000000,0.000000,0.000000}%
\pgfsetfillcolor{currentfill}%
\pgfsetlinewidth{0.501875pt}%
\definecolor{currentstroke}{rgb}{0.000000,0.000000,0.000000}%
\pgfsetstrokecolor{currentstroke}%
\pgfsetdash{}{0pt}%
\pgfsys@defobject{currentmarker}{\pgfqpoint{0.000000in}{0.000000in}}{\pgfqpoint{0.000000in}{0.055556in}}{%
\pgfpathmoveto{\pgfqpoint{0.000000in}{0.000000in}}%
\pgfpathlineto{\pgfqpoint{0.000000in}{0.055556in}}%
\pgfusepath{stroke,fill}%
}%
\begin{pgfscope}%
\pgfsys@transformshift{3.070860in}{4.050417in}%
\pgfsys@useobject{currentmarker}{}%
\end{pgfscope}%
\end{pgfscope}%
\begin{pgfscope}%
\pgfsetbuttcap%
\pgfsetroundjoin%
\definecolor{currentfill}{rgb}{0.000000,0.000000,0.000000}%
\pgfsetfillcolor{currentfill}%
\pgfsetlinewidth{0.501875pt}%
\definecolor{currentstroke}{rgb}{0.000000,0.000000,0.000000}%
\pgfsetstrokecolor{currentstroke}%
\pgfsetdash{}{0pt}%
\pgfsys@defobject{currentmarker}{\pgfqpoint{0.000000in}{-0.055556in}}{\pgfqpoint{0.000000in}{0.000000in}}{%
\pgfpathmoveto{\pgfqpoint{0.000000in}{0.000000in}}%
\pgfpathlineto{\pgfqpoint{0.000000in}{-0.055556in}}%
\pgfusepath{stroke,fill}%
}%
\begin{pgfscope}%
\pgfsys@transformshift{3.070860in}{5.749375in}%
\pgfsys@useobject{currentmarker}{}%
\end{pgfscope}%
\end{pgfscope}%
\begin{pgfscope}%
\pgftext[x=3.070860in,y=3.994861in,,top]{\fontsize{11.000000}{13.200000}\selectfont 15}%
\end{pgfscope}%
\begin{pgfscope}%
\pgfsetbuttcap%
\pgfsetroundjoin%
\definecolor{currentfill}{rgb}{0.000000,0.000000,0.000000}%
\pgfsetfillcolor{currentfill}%
\pgfsetlinewidth{0.501875pt}%
\definecolor{currentstroke}{rgb}{0.000000,0.000000,0.000000}%
\pgfsetstrokecolor{currentstroke}%
\pgfsetdash{}{0pt}%
\pgfsys@defobject{currentmarker}{\pgfqpoint{0.000000in}{0.000000in}}{\pgfqpoint{0.055556in}{0.000000in}}{%
\pgfpathmoveto{\pgfqpoint{0.000000in}{0.000000in}}%
\pgfpathlineto{\pgfqpoint{0.055556in}{0.000000in}}%
\pgfusepath{stroke,fill}%
}%
\begin{pgfscope}%
\pgfsys@transformshift{0.898557in}{4.076158in}%
\pgfsys@useobject{currentmarker}{}%
\end{pgfscope}%
\end{pgfscope}%
\begin{pgfscope}%
\pgfsetbuttcap%
\pgfsetroundjoin%
\definecolor{currentfill}{rgb}{0.000000,0.000000,0.000000}%
\pgfsetfillcolor{currentfill}%
\pgfsetlinewidth{0.501875pt}%
\definecolor{currentstroke}{rgb}{0.000000,0.000000,0.000000}%
\pgfsetstrokecolor{currentstroke}%
\pgfsetdash{}{0pt}%
\pgfsys@defobject{currentmarker}{\pgfqpoint{-0.055556in}{0.000000in}}{\pgfqpoint{0.000000in}{0.000000in}}{%
\pgfpathmoveto{\pgfqpoint{0.000000in}{0.000000in}}%
\pgfpathlineto{\pgfqpoint{-0.055556in}{0.000000in}}%
\pgfusepath{stroke,fill}%
}%
\begin{pgfscope}%
\pgfsys@transformshift{3.221779in}{4.076158in}%
\pgfsys@useobject{currentmarker}{}%
\end{pgfscope}%
\end{pgfscope}%
\begin{pgfscope}%
\pgftext[x=0.843001in,y=4.076158in,right,]{\fontsize{11.000000}{13.200000}\selectfont 0}%
\end{pgfscope}%
\begin{pgfscope}%
\pgfsetbuttcap%
\pgfsetroundjoin%
\definecolor{currentfill}{rgb}{0.000000,0.000000,0.000000}%
\pgfsetfillcolor{currentfill}%
\pgfsetlinewidth{0.501875pt}%
\definecolor{currentstroke}{rgb}{0.000000,0.000000,0.000000}%
\pgfsetstrokecolor{currentstroke}%
\pgfsetdash{}{0pt}%
\pgfsys@defobject{currentmarker}{\pgfqpoint{0.000000in}{0.000000in}}{\pgfqpoint{0.055556in}{0.000000in}}{%
\pgfpathmoveto{\pgfqpoint{0.000000in}{0.000000in}}%
\pgfpathlineto{\pgfqpoint{0.055556in}{0.000000in}}%
\pgfusepath{stroke,fill}%
}%
\begin{pgfscope}%
\pgfsys@transformshift{0.898557in}{4.333576in}%
\pgfsys@useobject{currentmarker}{}%
\end{pgfscope}%
\end{pgfscope}%
\begin{pgfscope}%
\pgfsetbuttcap%
\pgfsetroundjoin%
\definecolor{currentfill}{rgb}{0.000000,0.000000,0.000000}%
\pgfsetfillcolor{currentfill}%
\pgfsetlinewidth{0.501875pt}%
\definecolor{currentstroke}{rgb}{0.000000,0.000000,0.000000}%
\pgfsetstrokecolor{currentstroke}%
\pgfsetdash{}{0pt}%
\pgfsys@defobject{currentmarker}{\pgfqpoint{-0.055556in}{0.000000in}}{\pgfqpoint{0.000000in}{0.000000in}}{%
\pgfpathmoveto{\pgfqpoint{0.000000in}{0.000000in}}%
\pgfpathlineto{\pgfqpoint{-0.055556in}{0.000000in}}%
\pgfusepath{stroke,fill}%
}%
\begin{pgfscope}%
\pgfsys@transformshift{3.221779in}{4.333576in}%
\pgfsys@useobject{currentmarker}{}%
\end{pgfscope}%
\end{pgfscope}%
\begin{pgfscope}%
\pgftext[x=0.843001in,y=4.333576in,right,]{\fontsize{11.000000}{13.200000}\selectfont 10}%
\end{pgfscope}%
\begin{pgfscope}%
\pgfsetbuttcap%
\pgfsetroundjoin%
\definecolor{currentfill}{rgb}{0.000000,0.000000,0.000000}%
\pgfsetfillcolor{currentfill}%
\pgfsetlinewidth{0.501875pt}%
\definecolor{currentstroke}{rgb}{0.000000,0.000000,0.000000}%
\pgfsetstrokecolor{currentstroke}%
\pgfsetdash{}{0pt}%
\pgfsys@defobject{currentmarker}{\pgfqpoint{0.000000in}{0.000000in}}{\pgfqpoint{0.055556in}{0.000000in}}{%
\pgfpathmoveto{\pgfqpoint{0.000000in}{0.000000in}}%
\pgfpathlineto{\pgfqpoint{0.055556in}{0.000000in}}%
\pgfusepath{stroke,fill}%
}%
\begin{pgfscope}%
\pgfsys@transformshift{0.898557in}{4.590994in}%
\pgfsys@useobject{currentmarker}{}%
\end{pgfscope}%
\end{pgfscope}%
\begin{pgfscope}%
\pgfsetbuttcap%
\pgfsetroundjoin%
\definecolor{currentfill}{rgb}{0.000000,0.000000,0.000000}%
\pgfsetfillcolor{currentfill}%
\pgfsetlinewidth{0.501875pt}%
\definecolor{currentstroke}{rgb}{0.000000,0.000000,0.000000}%
\pgfsetstrokecolor{currentstroke}%
\pgfsetdash{}{0pt}%
\pgfsys@defobject{currentmarker}{\pgfqpoint{-0.055556in}{0.000000in}}{\pgfqpoint{0.000000in}{0.000000in}}{%
\pgfpathmoveto{\pgfqpoint{0.000000in}{0.000000in}}%
\pgfpathlineto{\pgfqpoint{-0.055556in}{0.000000in}}%
\pgfusepath{stroke,fill}%
}%
\begin{pgfscope}%
\pgfsys@transformshift{3.221779in}{4.590994in}%
\pgfsys@useobject{currentmarker}{}%
\end{pgfscope}%
\end{pgfscope}%
\begin{pgfscope}%
\pgftext[x=0.843001in,y=4.590994in,right,]{\fontsize{11.000000}{13.200000}\selectfont 20}%
\end{pgfscope}%
\begin{pgfscope}%
\pgfsetbuttcap%
\pgfsetroundjoin%
\definecolor{currentfill}{rgb}{0.000000,0.000000,0.000000}%
\pgfsetfillcolor{currentfill}%
\pgfsetlinewidth{0.501875pt}%
\definecolor{currentstroke}{rgb}{0.000000,0.000000,0.000000}%
\pgfsetstrokecolor{currentstroke}%
\pgfsetdash{}{0pt}%
\pgfsys@defobject{currentmarker}{\pgfqpoint{0.000000in}{0.000000in}}{\pgfqpoint{0.055556in}{0.000000in}}{%
\pgfpathmoveto{\pgfqpoint{0.000000in}{0.000000in}}%
\pgfpathlineto{\pgfqpoint{0.055556in}{0.000000in}}%
\pgfusepath{stroke,fill}%
}%
\begin{pgfscope}%
\pgfsys@transformshift{0.898557in}{4.848412in}%
\pgfsys@useobject{currentmarker}{}%
\end{pgfscope}%
\end{pgfscope}%
\begin{pgfscope}%
\pgfsetbuttcap%
\pgfsetroundjoin%
\definecolor{currentfill}{rgb}{0.000000,0.000000,0.000000}%
\pgfsetfillcolor{currentfill}%
\pgfsetlinewidth{0.501875pt}%
\definecolor{currentstroke}{rgb}{0.000000,0.000000,0.000000}%
\pgfsetstrokecolor{currentstroke}%
\pgfsetdash{}{0pt}%
\pgfsys@defobject{currentmarker}{\pgfqpoint{-0.055556in}{0.000000in}}{\pgfqpoint{0.000000in}{0.000000in}}{%
\pgfpathmoveto{\pgfqpoint{0.000000in}{0.000000in}}%
\pgfpathlineto{\pgfqpoint{-0.055556in}{0.000000in}}%
\pgfusepath{stroke,fill}%
}%
\begin{pgfscope}%
\pgfsys@transformshift{3.221779in}{4.848412in}%
\pgfsys@useobject{currentmarker}{}%
\end{pgfscope}%
\end{pgfscope}%
\begin{pgfscope}%
\pgftext[x=0.843001in,y=4.848412in,right,]{\fontsize{11.000000}{13.200000}\selectfont 30}%
\end{pgfscope}%
\begin{pgfscope}%
\pgfsetbuttcap%
\pgfsetroundjoin%
\definecolor{currentfill}{rgb}{0.000000,0.000000,0.000000}%
\pgfsetfillcolor{currentfill}%
\pgfsetlinewidth{0.501875pt}%
\definecolor{currentstroke}{rgb}{0.000000,0.000000,0.000000}%
\pgfsetstrokecolor{currentstroke}%
\pgfsetdash{}{0pt}%
\pgfsys@defobject{currentmarker}{\pgfqpoint{0.000000in}{0.000000in}}{\pgfqpoint{0.055556in}{0.000000in}}{%
\pgfpathmoveto{\pgfqpoint{0.000000in}{0.000000in}}%
\pgfpathlineto{\pgfqpoint{0.055556in}{0.000000in}}%
\pgfusepath{stroke,fill}%
}%
\begin{pgfscope}%
\pgfsys@transformshift{0.898557in}{5.105830in}%
\pgfsys@useobject{currentmarker}{}%
\end{pgfscope}%
\end{pgfscope}%
\begin{pgfscope}%
\pgfsetbuttcap%
\pgfsetroundjoin%
\definecolor{currentfill}{rgb}{0.000000,0.000000,0.000000}%
\pgfsetfillcolor{currentfill}%
\pgfsetlinewidth{0.501875pt}%
\definecolor{currentstroke}{rgb}{0.000000,0.000000,0.000000}%
\pgfsetstrokecolor{currentstroke}%
\pgfsetdash{}{0pt}%
\pgfsys@defobject{currentmarker}{\pgfqpoint{-0.055556in}{0.000000in}}{\pgfqpoint{0.000000in}{0.000000in}}{%
\pgfpathmoveto{\pgfqpoint{0.000000in}{0.000000in}}%
\pgfpathlineto{\pgfqpoint{-0.055556in}{0.000000in}}%
\pgfusepath{stroke,fill}%
}%
\begin{pgfscope}%
\pgfsys@transformshift{3.221779in}{5.105830in}%
\pgfsys@useobject{currentmarker}{}%
\end{pgfscope}%
\end{pgfscope}%
\begin{pgfscope}%
\pgftext[x=0.843001in,y=5.105830in,right,]{\fontsize{11.000000}{13.200000}\selectfont 40}%
\end{pgfscope}%
\begin{pgfscope}%
\pgfsetbuttcap%
\pgfsetroundjoin%
\definecolor{currentfill}{rgb}{0.000000,0.000000,0.000000}%
\pgfsetfillcolor{currentfill}%
\pgfsetlinewidth{0.501875pt}%
\definecolor{currentstroke}{rgb}{0.000000,0.000000,0.000000}%
\pgfsetstrokecolor{currentstroke}%
\pgfsetdash{}{0pt}%
\pgfsys@defobject{currentmarker}{\pgfqpoint{0.000000in}{0.000000in}}{\pgfqpoint{0.055556in}{0.000000in}}{%
\pgfpathmoveto{\pgfqpoint{0.000000in}{0.000000in}}%
\pgfpathlineto{\pgfqpoint{0.055556in}{0.000000in}}%
\pgfusepath{stroke,fill}%
}%
\begin{pgfscope}%
\pgfsys@transformshift{0.898557in}{5.363248in}%
\pgfsys@useobject{currentmarker}{}%
\end{pgfscope}%
\end{pgfscope}%
\begin{pgfscope}%
\pgfsetbuttcap%
\pgfsetroundjoin%
\definecolor{currentfill}{rgb}{0.000000,0.000000,0.000000}%
\pgfsetfillcolor{currentfill}%
\pgfsetlinewidth{0.501875pt}%
\definecolor{currentstroke}{rgb}{0.000000,0.000000,0.000000}%
\pgfsetstrokecolor{currentstroke}%
\pgfsetdash{}{0pt}%
\pgfsys@defobject{currentmarker}{\pgfqpoint{-0.055556in}{0.000000in}}{\pgfqpoint{0.000000in}{0.000000in}}{%
\pgfpathmoveto{\pgfqpoint{0.000000in}{0.000000in}}%
\pgfpathlineto{\pgfqpoint{-0.055556in}{0.000000in}}%
\pgfusepath{stroke,fill}%
}%
\begin{pgfscope}%
\pgfsys@transformshift{3.221779in}{5.363248in}%
\pgfsys@useobject{currentmarker}{}%
\end{pgfscope}%
\end{pgfscope}%
\begin{pgfscope}%
\pgftext[x=0.843001in,y=5.363248in,right,]{\fontsize{11.000000}{13.200000}\selectfont 50}%
\end{pgfscope}%
\begin{pgfscope}%
\pgfsetbuttcap%
\pgfsetroundjoin%
\definecolor{currentfill}{rgb}{0.000000,0.000000,0.000000}%
\pgfsetfillcolor{currentfill}%
\pgfsetlinewidth{0.501875pt}%
\definecolor{currentstroke}{rgb}{0.000000,0.000000,0.000000}%
\pgfsetstrokecolor{currentstroke}%
\pgfsetdash{}{0pt}%
\pgfsys@defobject{currentmarker}{\pgfqpoint{0.000000in}{0.000000in}}{\pgfqpoint{0.055556in}{0.000000in}}{%
\pgfpathmoveto{\pgfqpoint{0.000000in}{0.000000in}}%
\pgfpathlineto{\pgfqpoint{0.055556in}{0.000000in}}%
\pgfusepath{stroke,fill}%
}%
\begin{pgfscope}%
\pgfsys@transformshift{0.898557in}{5.620666in}%
\pgfsys@useobject{currentmarker}{}%
\end{pgfscope}%
\end{pgfscope}%
\begin{pgfscope}%
\pgfsetbuttcap%
\pgfsetroundjoin%
\definecolor{currentfill}{rgb}{0.000000,0.000000,0.000000}%
\pgfsetfillcolor{currentfill}%
\pgfsetlinewidth{0.501875pt}%
\definecolor{currentstroke}{rgb}{0.000000,0.000000,0.000000}%
\pgfsetstrokecolor{currentstroke}%
\pgfsetdash{}{0pt}%
\pgfsys@defobject{currentmarker}{\pgfqpoint{-0.055556in}{0.000000in}}{\pgfqpoint{0.000000in}{0.000000in}}{%
\pgfpathmoveto{\pgfqpoint{0.000000in}{0.000000in}}%
\pgfpathlineto{\pgfqpoint{-0.055556in}{0.000000in}}%
\pgfusepath{stroke,fill}%
}%
\begin{pgfscope}%
\pgfsys@transformshift{3.221779in}{5.620666in}%
\pgfsys@useobject{currentmarker}{}%
\end{pgfscope}%
\end{pgfscope}%
\begin{pgfscope}%
\pgftext[x=0.843001in,y=5.620666in,right,]{\fontsize{11.000000}{13.200000}\selectfont 60}%
\end{pgfscope}%
\begin{pgfscope}%
\pgftext[x=0.319784in,y=4.639664in,left,base,rotate=90.000000]{\fontsize{11.000000}{13.200000}\selectfont Current}%
\end{pgfscope}%
\begin{pgfscope}%
\pgftext[x=0.486231in,y=4.741475in,left,base,rotate=90.000000]{\fontsize{11.000000}{13.200000}\selectfont (nA)}%
\end{pgfscope}%
\begin{pgfscope}%
\pgfsetbuttcap%
\pgfsetmiterjoin%
\definecolor{currentfill}{rgb}{1.000000,1.000000,1.000000}%
\pgfsetfillcolor{currentfill}%
\pgfsetlinewidth{0.000000pt}%
\definecolor{currentstroke}{rgb}{0.000000,0.000000,0.000000}%
\pgfsetstrokecolor{currentstroke}%
\pgfsetstrokeopacity{0.000000}%
\pgfsetdash{}{0pt}%
\pgfpathmoveto{\pgfqpoint{0.898557in}{2.351458in}}%
\pgfpathlineto{\pgfqpoint{3.221779in}{2.351458in}}%
\pgfpathlineto{\pgfqpoint{3.221779in}{4.050417in}}%
\pgfpathlineto{\pgfqpoint{0.898557in}{4.050417in}}%
\pgfpathclose%
\pgfusepath{fill}%
\end{pgfscope}%
\begin{pgfscope}%
\pgfpathrectangle{\pgfqpoint{0.898557in}{2.351458in}}{\pgfqpoint{2.323221in}{1.698958in}} %
\pgfusepath{clip}%
\pgfsetbuttcap%
\pgfsetroundjoin%
\definecolor{currentfill}{rgb}{0.309804,0.478431,0.682353}%
\pgfsetfillcolor{currentfill}%
\pgfsetfillopacity{0.500000}%
\pgfsetlinewidth{1.003750pt}%
\definecolor{currentstroke}{rgb}{0.309804,0.478431,0.682353}%
\pgfsetstrokecolor{currentstroke}%
\pgfsetstrokeopacity{0.500000}%
\pgfsetdash{}{0pt}%
\pgfpathmoveto{\pgfqpoint{1.195471in}{2.843697in}}%
\pgfpathlineto{\pgfqpoint{1.195471in}{2.800929in}}%
\pgfpathlineto{\pgfqpoint{1.200680in}{2.803827in}}%
\pgfpathlineto{\pgfqpoint{1.205889in}{2.809496in}}%
\pgfpathlineto{\pgfqpoint{1.211098in}{2.817629in}}%
\pgfpathlineto{\pgfqpoint{1.216307in}{2.827688in}}%
\pgfpathlineto{\pgfqpoint{1.221516in}{2.838903in}}%
\pgfpathlineto{\pgfqpoint{1.226725in}{2.850365in}}%
\pgfpathlineto{\pgfqpoint{1.231934in}{2.861171in}}%
\pgfpathlineto{\pgfqpoint{1.237143in}{2.870577in}}%
\pgfpathlineto{\pgfqpoint{1.242352in}{2.878081in}}%
\pgfpathlineto{\pgfqpoint{1.247561in}{2.883434in}}%
\pgfpathlineto{\pgfqpoint{1.252770in}{2.886593in}}%
\pgfpathlineto{\pgfqpoint{1.257979in}{2.887679in}}%
\pgfpathlineto{\pgfqpoint{1.263188in}{2.886973in}}%
\pgfpathlineto{\pgfqpoint{1.268397in}{2.884927in}}%
\pgfpathlineto{\pgfqpoint{1.273606in}{2.882148in}}%
\pgfpathlineto{\pgfqpoint{1.278815in}{2.879320in}}%
\pgfpathlineto{\pgfqpoint{1.284024in}{2.877055in}}%
\pgfpathlineto{\pgfqpoint{1.289233in}{2.875741in}}%
\pgfpathlineto{\pgfqpoint{1.294442in}{2.875438in}}%
\pgfpathlineto{\pgfqpoint{1.299651in}{2.875889in}}%
\pgfpathlineto{\pgfqpoint{1.304860in}{2.876628in}}%
\pgfpathlineto{\pgfqpoint{1.310069in}{2.877152in}}%
\pgfpathlineto{\pgfqpoint{1.315278in}{2.877077in}}%
\pgfpathlineto{\pgfqpoint{1.320487in}{2.876244in}}%
\pgfpathlineto{\pgfqpoint{1.325696in}{2.874715in}}%
\pgfpathlineto{\pgfqpoint{1.330905in}{2.872714in}}%
\pgfpathlineto{\pgfqpoint{1.336114in}{2.870532in}}%
\pgfpathlineto{\pgfqpoint{1.341323in}{2.868445in}}%
\pgfpathlineto{\pgfqpoint{1.346532in}{2.866667in}}%
\pgfpathlineto{\pgfqpoint{1.351741in}{2.865341in}}%
\pgfpathlineto{\pgfqpoint{1.356951in}{2.864527in}}%
\pgfpathlineto{\pgfqpoint{1.362160in}{2.864208in}}%
\pgfpathlineto{\pgfqpoint{1.367369in}{2.864307in}}%
\pgfpathlineto{\pgfqpoint{1.372578in}{2.864732in}}%
\pgfpathlineto{\pgfqpoint{1.377787in}{2.865417in}}%
\pgfpathlineto{\pgfqpoint{1.382996in}{2.866341in}}%
\pgfpathlineto{\pgfqpoint{1.388205in}{2.867492in}}%
\pgfpathlineto{\pgfqpoint{1.393414in}{2.868817in}}%
\pgfpathlineto{\pgfqpoint{1.398623in}{2.870175in}}%
\pgfpathlineto{\pgfqpoint{1.403832in}{2.871355in}}%
\pgfpathlineto{\pgfqpoint{1.409041in}{2.872133in}}%
\pgfpathlineto{\pgfqpoint{1.414250in}{2.872361in}}%
\pgfpathlineto{\pgfqpoint{1.419459in}{2.872034in}}%
\pgfpathlineto{\pgfqpoint{1.424668in}{2.871310in}}%
\pgfpathlineto{\pgfqpoint{1.429877in}{2.870466in}}%
\pgfpathlineto{\pgfqpoint{1.435086in}{2.869814in}}%
\pgfpathlineto{\pgfqpoint{1.440295in}{2.869598in}}%
\pgfpathlineto{\pgfqpoint{1.445504in}{2.869918in}}%
\pgfpathlineto{\pgfqpoint{1.450713in}{2.870699in}}%
\pgfpathlineto{\pgfqpoint{1.455922in}{2.871718in}}%
\pgfpathlineto{\pgfqpoint{1.461131in}{2.872666in}}%
\pgfpathlineto{\pgfqpoint{1.466340in}{2.873246in}}%
\pgfpathlineto{\pgfqpoint{1.471549in}{2.873263in}}%
\pgfpathlineto{\pgfqpoint{1.476758in}{2.872685in}}%
\pgfpathlineto{\pgfqpoint{1.481967in}{2.871647in}}%
\pgfpathlineto{\pgfqpoint{1.487176in}{2.870399in}}%
\pgfpathlineto{\pgfqpoint{1.492385in}{2.869199in}}%
\pgfpathlineto{\pgfqpoint{1.497594in}{2.868230in}}%
\pgfpathlineto{\pgfqpoint{1.502803in}{2.867569in}}%
\pgfpathlineto{\pgfqpoint{1.508012in}{2.867237in}}%
\pgfpathlineto{\pgfqpoint{1.513221in}{2.867276in}}%
\pgfpathlineto{\pgfqpoint{1.518430in}{2.867793in}}%
\pgfpathlineto{\pgfqpoint{1.523639in}{2.868926in}}%
\pgfpathlineto{\pgfqpoint{1.528848in}{2.870730in}}%
\pgfpathlineto{\pgfqpoint{1.534057in}{2.873075in}}%
\pgfpathlineto{\pgfqpoint{1.539266in}{2.875621in}}%
\pgfpathlineto{\pgfqpoint{1.544475in}{2.877912in}}%
\pgfpathlineto{\pgfqpoint{1.549684in}{2.879541in}}%
\pgfpathlineto{\pgfqpoint{1.554893in}{2.880310in}}%
\pgfpathlineto{\pgfqpoint{1.560102in}{2.880323in}}%
\pgfpathlineto{\pgfqpoint{1.565311in}{2.879957in}}%
\pgfpathlineto{\pgfqpoint{1.570520in}{2.879724in}}%
\pgfpathlineto{\pgfqpoint{1.575729in}{2.880082in}}%
\pgfpathlineto{\pgfqpoint{1.580938in}{2.881248in}}%
\pgfpathlineto{\pgfqpoint{1.586147in}{2.883113in}}%
\pgfpathlineto{\pgfqpoint{1.591356in}{2.885285in}}%
\pgfpathlineto{\pgfqpoint{1.596565in}{2.887243in}}%
\pgfpathlineto{\pgfqpoint{1.601774in}{2.888533in}}%
\pgfpathlineto{\pgfqpoint{1.606983in}{2.888930in}}%
\pgfpathlineto{\pgfqpoint{1.612192in}{2.888478in}}%
\pgfpathlineto{\pgfqpoint{1.617401in}{2.887443in}}%
\pgfpathlineto{\pgfqpoint{1.622610in}{2.886180in}}%
\pgfpathlineto{\pgfqpoint{1.627819in}{2.885007in}}%
\pgfpathlineto{\pgfqpoint{1.633028in}{2.884117in}}%
\pgfpathlineto{\pgfqpoint{1.638237in}{2.883558in}}%
\pgfpathlineto{\pgfqpoint{1.643446in}{2.883266in}}%
\pgfpathlineto{\pgfqpoint{1.648655in}{2.883128in}}%
\pgfpathlineto{\pgfqpoint{1.653864in}{2.883050in}}%
\pgfpathlineto{\pgfqpoint{1.659073in}{2.882979in}}%
\pgfpathlineto{\pgfqpoint{1.664282in}{2.882899in}}%
\pgfpathlineto{\pgfqpoint{1.669492in}{2.882781in}}%
\pgfpathlineto{\pgfqpoint{1.674701in}{2.882545in}}%
\pgfpathlineto{\pgfqpoint{1.679910in}{2.882058in}}%
\pgfpathlineto{\pgfqpoint{1.685119in}{2.881194in}}%
\pgfpathlineto{\pgfqpoint{1.690328in}{2.879913in}}%
\pgfpathlineto{\pgfqpoint{1.695537in}{2.878314in}}%
\pgfpathlineto{\pgfqpoint{1.700746in}{2.876626in}}%
\pgfpathlineto{\pgfqpoint{1.705955in}{2.875135in}}%
\pgfpathlineto{\pgfqpoint{1.711164in}{2.874100in}}%
\pgfpathlineto{\pgfqpoint{1.716373in}{2.873695in}}%
\pgfpathlineto{\pgfqpoint{1.721582in}{2.873995in}}%
\pgfpathlineto{\pgfqpoint{1.726791in}{2.874984in}}%
\pgfpathlineto{\pgfqpoint{1.732000in}{2.876566in}}%
\pgfpathlineto{\pgfqpoint{1.737209in}{2.878554in}}%
\pgfpathlineto{\pgfqpoint{1.742418in}{2.880660in}}%
\pgfpathlineto{\pgfqpoint{1.747627in}{2.882523in}}%
\pgfpathlineto{\pgfqpoint{1.752836in}{2.883824in}}%
\pgfpathlineto{\pgfqpoint{1.758045in}{2.884437in}}%
\pgfpathlineto{\pgfqpoint{1.763254in}{2.884557in}}%
\pgfpathlineto{\pgfqpoint{1.768463in}{2.884692in}}%
\pgfpathlineto{\pgfqpoint{1.773672in}{2.885479in}}%
\pgfpathlineto{\pgfqpoint{1.778881in}{2.887382in}}%
\pgfpathlineto{\pgfqpoint{1.784090in}{2.890438in}}%
\pgfpathlineto{\pgfqpoint{1.789299in}{2.894182in}}%
\pgfpathlineto{\pgfqpoint{1.794508in}{2.897808in}}%
\pgfpathlineto{\pgfqpoint{1.799717in}{2.900493in}}%
\pgfpathlineto{\pgfqpoint{1.804926in}{2.901729in}}%
\pgfpathlineto{\pgfqpoint{1.810135in}{2.901501in}}%
\pgfpathlineto{\pgfqpoint{1.815344in}{2.900264in}}%
\pgfpathlineto{\pgfqpoint{1.820553in}{2.898723in}}%
\pgfpathlineto{\pgfqpoint{1.825762in}{2.897555in}}%
\pgfpathlineto{\pgfqpoint{1.830971in}{2.897187in}}%
\pgfpathlineto{\pgfqpoint{1.836180in}{2.897734in}}%
\pgfpathlineto{\pgfqpoint{1.841389in}{2.899051in}}%
\pgfpathlineto{\pgfqpoint{1.846598in}{2.900848in}}%
\pgfpathlineto{\pgfqpoint{1.851807in}{2.902776in}}%
\pgfpathlineto{\pgfqpoint{1.857016in}{2.904452in}}%
\pgfpathlineto{\pgfqpoint{1.862225in}{2.905478in}}%
\pgfpathlineto{\pgfqpoint{1.867434in}{2.905479in}}%
\pgfpathlineto{\pgfqpoint{1.872643in}{2.904181in}}%
\pgfpathlineto{\pgfqpoint{1.877852in}{2.901498in}}%
\pgfpathlineto{\pgfqpoint{1.883061in}{2.897590in}}%
\pgfpathlineto{\pgfqpoint{1.888270in}{2.892881in}}%
\pgfpathlineto{\pgfqpoint{1.893479in}{2.888024in}}%
\pgfpathlineto{\pgfqpoint{1.898688in}{2.883788in}}%
\pgfpathlineto{\pgfqpoint{1.903897in}{2.880886in}}%
\pgfpathlineto{\pgfqpoint{1.909106in}{2.879784in}}%
\pgfpathlineto{\pgfqpoint{1.914315in}{2.880572in}}%
\pgfpathlineto{\pgfqpoint{1.919524in}{2.882956in}}%
\pgfpathlineto{\pgfqpoint{1.924733in}{2.886373in}}%
\pgfpathlineto{\pgfqpoint{1.929942in}{2.890168in}}%
\pgfpathlineto{\pgfqpoint{1.935151in}{2.893738in}}%
\pgfpathlineto{\pgfqpoint{1.940360in}{2.896594in}}%
\pgfpathlineto{\pgfqpoint{1.945569in}{2.898366in}}%
\pgfpathlineto{\pgfqpoint{1.950778in}{2.898811in}}%
\pgfpathlineto{\pgfqpoint{1.955987in}{2.897865in}}%
\pgfpathlineto{\pgfqpoint{1.961196in}{2.895722in}}%
\pgfpathlineto{\pgfqpoint{1.966405in}{2.892848in}}%
\pgfpathlineto{\pgfqpoint{1.971614in}{2.889920in}}%
\pgfpathlineto{\pgfqpoint{1.976824in}{2.887669in}}%
\pgfpathlineto{\pgfqpoint{1.982033in}{2.886735in}}%
\pgfpathlineto{\pgfqpoint{1.987242in}{2.887547in}}%
\pgfpathlineto{\pgfqpoint{1.992451in}{2.890219in}}%
\pgfpathlineto{\pgfqpoint{1.997660in}{2.894501in}}%
\pgfpathlineto{\pgfqpoint{2.002869in}{2.899739in}}%
\pgfpathlineto{\pgfqpoint{2.008078in}{2.904950in}}%
\pgfpathlineto{\pgfqpoint{2.013287in}{2.909015in}}%
\pgfpathlineto{\pgfqpoint{2.018496in}{2.910941in}}%
\pgfpathlineto{\pgfqpoint{2.023705in}{2.910141in}}%
\pgfpathlineto{\pgfqpoint{2.028914in}{2.906621in}}%
\pgfpathlineto{\pgfqpoint{2.034123in}{2.901059in}}%
\pgfpathlineto{\pgfqpoint{2.039332in}{2.894752in}}%
\pgfpathlineto{\pgfqpoint{2.044541in}{2.889391in}}%
\pgfpathlineto{\pgfqpoint{2.049750in}{2.886634in}}%
\pgfpathlineto{\pgfqpoint{2.054959in}{2.887541in}}%
\pgfpathlineto{\pgfqpoint{2.060168in}{2.892095in}}%
\pgfpathlineto{\pgfqpoint{2.065377in}{2.899080in}}%
\pgfpathlineto{\pgfqpoint{2.070586in}{2.906463in}}%
\pgfpathlineto{\pgfqpoint{2.075795in}{2.912133in}}%
\pgfpathlineto{\pgfqpoint{2.081004in}{2.914641in}}%
\pgfpathlineto{\pgfqpoint{2.086213in}{2.913594in}}%
\pgfpathlineto{\pgfqpoint{2.091422in}{2.909566in}}%
\pgfpathlineto{\pgfqpoint{2.096631in}{2.903704in}}%
\pgfpathlineto{\pgfqpoint{2.101840in}{2.897256in}}%
\pgfpathlineto{\pgfqpoint{2.107049in}{2.891216in}}%
\pgfpathlineto{\pgfqpoint{2.112258in}{2.886148in}}%
\pgfpathlineto{\pgfqpoint{2.117467in}{2.882152in}}%
\pgfpathlineto{\pgfqpoint{2.122676in}{2.878944in}}%
\pgfpathlineto{\pgfqpoint{2.127885in}{2.876036in}}%
\pgfpathlineto{\pgfqpoint{2.133094in}{2.872930in}}%
\pgfpathlineto{\pgfqpoint{2.138303in}{2.869236in}}%
\pgfpathlineto{\pgfqpoint{2.143512in}{2.864676in}}%
\pgfpathlineto{\pgfqpoint{2.148721in}{2.859023in}}%
\pgfpathlineto{\pgfqpoint{2.153930in}{2.852082in}}%
\pgfpathlineto{\pgfqpoint{2.159139in}{2.843805in}}%
\pgfpathlineto{\pgfqpoint{2.164348in}{2.834482in}}%
\pgfpathlineto{\pgfqpoint{2.169557in}{2.824903in}}%
\pgfpathlineto{\pgfqpoint{2.174766in}{2.816338in}}%
\pgfpathlineto{\pgfqpoint{2.179975in}{2.810277in}}%
\pgfpathlineto{\pgfqpoint{2.185184in}{2.807978in}}%
\pgfpathlineto{\pgfqpoint{2.190393in}{2.809997in}}%
\pgfpathlineto{\pgfqpoint{2.195602in}{2.815938in}}%
\pgfpathlineto{\pgfqpoint{2.200811in}{2.824543in}}%
\pgfpathlineto{\pgfqpoint{2.206020in}{2.834175in}}%
\pgfpathlineto{\pgfqpoint{2.211229in}{2.843441in}}%
\pgfpathlineto{\pgfqpoint{2.216438in}{2.851673in}}%
\pgfpathlineto{\pgfqpoint{2.221647in}{2.858971in}}%
\pgfpathlineto{\pgfqpoint{2.226856in}{2.865840in}}%
\pgfpathlineto{\pgfqpoint{2.232065in}{2.872625in}}%
\pgfpathlineto{\pgfqpoint{2.237274in}{2.879137in}}%
\pgfpathlineto{\pgfqpoint{2.242483in}{2.884679in}}%
\pgfpathlineto{\pgfqpoint{2.247692in}{2.888399in}}%
\pgfpathlineto{\pgfqpoint{2.252901in}{2.889738in}}%
\pgfpathlineto{\pgfqpoint{2.258110in}{2.888701in}}%
\pgfpathlineto{\pgfqpoint{2.263319in}{2.885854in}}%
\pgfpathlineto{\pgfqpoint{2.268528in}{2.882114in}}%
\pgfpathlineto{\pgfqpoint{2.273737in}{2.878458in}}%
\pgfpathlineto{\pgfqpoint{2.278946in}{2.875693in}}%
\pgfpathlineto{\pgfqpoint{2.284155in}{2.874326in}}%
\pgfpathlineto{\pgfqpoint{2.289365in}{2.874565in}}%
\pgfpathlineto{\pgfqpoint{2.294574in}{2.876406in}}%
\pgfpathlineto{\pgfqpoint{2.299783in}{2.879746in}}%
\pgfpathlineto{\pgfqpoint{2.304992in}{2.884451in}}%
\pgfpathlineto{\pgfqpoint{2.310201in}{2.890346in}}%
\pgfpathlineto{\pgfqpoint{2.315410in}{2.897122in}}%
\pgfpathlineto{\pgfqpoint{2.320619in}{2.904265in}}%
\pgfpathlineto{\pgfqpoint{2.325828in}{2.911080in}}%
\pgfpathlineto{\pgfqpoint{2.331037in}{2.916849in}}%
\pgfpathlineto{\pgfqpoint{2.336246in}{2.921082in}}%
\pgfpathlineto{\pgfqpoint{2.341455in}{2.923761in}}%
\pgfpathlineto{\pgfqpoint{2.346664in}{2.925471in}}%
\pgfpathlineto{\pgfqpoint{2.351873in}{2.927321in}}%
\pgfpathlineto{\pgfqpoint{2.357082in}{2.930671in}}%
\pgfpathlineto{\pgfqpoint{2.362291in}{2.936658in}}%
\pgfpathlineto{\pgfqpoint{2.367500in}{2.945718in}}%
\pgfpathlineto{\pgfqpoint{2.372709in}{2.957256in}}%
\pgfpathlineto{\pgfqpoint{2.377918in}{2.969673in}}%
\pgfpathlineto{\pgfqpoint{2.383127in}{2.980787in}}%
\pgfpathlineto{\pgfqpoint{2.388336in}{2.988508in}}%
\pgfpathlineto{\pgfqpoint{2.393545in}{2.991469in}}%
\pgfpathlineto{\pgfqpoint{2.398754in}{2.989354in}}%
\pgfpathlineto{\pgfqpoint{2.403963in}{2.982803in}}%
\pgfpathlineto{\pgfqpoint{2.409172in}{2.973002in}}%
\pgfpathlineto{\pgfqpoint{2.414381in}{2.961258in}}%
\pgfpathlineto{\pgfqpoint{2.419590in}{2.948731in}}%
\pgfpathlineto{\pgfqpoint{2.424799in}{2.936388in}}%
\pgfpathlineto{\pgfqpoint{2.430008in}{2.925043in}}%
\pgfpathlineto{\pgfqpoint{2.435217in}{2.915349in}}%
\pgfpathlineto{\pgfqpoint{2.440426in}{2.907695in}}%
\pgfpathlineto{\pgfqpoint{2.445635in}{2.902095in}}%
\pgfpathlineto{\pgfqpoint{2.450844in}{2.898183in}}%
\pgfpathlineto{\pgfqpoint{2.456053in}{2.895364in}}%
\pgfpathlineto{\pgfqpoint{2.461262in}{2.893069in}}%
\pgfpathlineto{\pgfqpoint{2.466471in}{2.890973in}}%
\pgfpathlineto{\pgfqpoint{2.471680in}{2.889082in}}%
\pgfpathlineto{\pgfqpoint{2.476889in}{2.887653in}}%
\pgfpathlineto{\pgfqpoint{2.482098in}{2.887005in}}%
\pgfpathlineto{\pgfqpoint{2.487307in}{2.887369in}}%
\pgfpathlineto{\pgfqpoint{2.492516in}{2.888871in}}%
\pgfpathlineto{\pgfqpoint{2.497725in}{2.891652in}}%
\pgfpathlineto{\pgfqpoint{2.502934in}{2.895988in}}%
\pgfpathlineto{\pgfqpoint{2.508143in}{2.902249in}}%
\pgfpathlineto{\pgfqpoint{2.513352in}{2.910707in}}%
\pgfpathlineto{\pgfqpoint{2.518561in}{2.921281in}}%
\pgfpathlineto{\pgfqpoint{2.523770in}{2.933402in}}%
\pgfpathlineto{\pgfqpoint{2.528979in}{2.946106in}}%
\pgfpathlineto{\pgfqpoint{2.534188in}{2.958282in}}%
\pgfpathlineto{\pgfqpoint{2.539397in}{2.969015in}}%
\pgfpathlineto{\pgfqpoint{2.544606in}{2.977878in}}%
\pgfpathlineto{\pgfqpoint{2.549815in}{2.985102in}}%
\pgfpathlineto{\pgfqpoint{2.555024in}{2.991477in}}%
\pgfpathlineto{\pgfqpoint{2.560233in}{2.997943in}}%
\pgfpathlineto{\pgfqpoint{2.565442in}{3.004973in}}%
\pgfpathlineto{\pgfqpoint{2.570651in}{3.012030in}}%
\pgfpathlineto{\pgfqpoint{2.575860in}{3.017482in}}%
\pgfpathlineto{\pgfqpoint{2.581069in}{3.019135in}}%
\pgfpathlineto{\pgfqpoint{2.586278in}{3.015249in}}%
\pgfpathlineto{\pgfqpoint{2.591487in}{3.005571in}}%
\pgfpathlineto{\pgfqpoint{2.596697in}{2.991858in}}%
\pgfpathlineto{\pgfqpoint{2.601906in}{2.977555in}}%
\pgfpathlineto{\pgfqpoint{2.607115in}{2.966668in}}%
\pgfpathlineto{\pgfqpoint{2.612324in}{2.962297in}}%
\pgfpathlineto{\pgfqpoint{2.617533in}{2.965442in}}%
\pgfpathlineto{\pgfqpoint{2.622742in}{2.974665in}}%
\pgfpathlineto{\pgfqpoint{2.627951in}{2.986743in}}%
\pgfpathlineto{\pgfqpoint{2.633160in}{2.998009in}}%
\pgfpathlineto{\pgfqpoint{2.638369in}{3.005714in}}%
\pgfpathlineto{\pgfqpoint{2.643578in}{3.008801in}}%
\pgfpathlineto{\pgfqpoint{2.648787in}{3.007889in}}%
\pgfpathlineto{\pgfqpoint{2.653996in}{3.004651in}}%
\pgfpathlineto{\pgfqpoint{2.659205in}{3.001075in}}%
\pgfpathlineto{\pgfqpoint{2.664414in}{2.998851in}}%
\pgfpathlineto{\pgfqpoint{2.669623in}{2.999064in}}%
\pgfpathlineto{\pgfqpoint{2.674832in}{3.002107in}}%
\pgfpathlineto{\pgfqpoint{2.680041in}{3.007736in}}%
\pgfpathlineto{\pgfqpoint{2.685250in}{3.015235in}}%
\pgfpathlineto{\pgfqpoint{2.690459in}{3.023658in}}%
\pgfpathlineto{\pgfqpoint{2.695668in}{3.032112in}}%
\pgfpathlineto{\pgfqpoint{2.700877in}{3.040019in}}%
\pgfpathlineto{\pgfqpoint{2.706086in}{3.047218in}}%
\pgfpathlineto{\pgfqpoint{2.711295in}{3.053808in}}%
\pgfpathlineto{\pgfqpoint{2.716504in}{3.059776in}}%
\pgfpathlineto{\pgfqpoint{2.721713in}{3.064627in}}%
\pgfpathlineto{\pgfqpoint{2.726922in}{3.067337in}}%
\pgfpathlineto{\pgfqpoint{2.732131in}{3.066792in}}%
\pgfpathlineto{\pgfqpoint{2.737340in}{3.062519in}}%
\pgfpathlineto{\pgfqpoint{2.742549in}{3.055211in}}%
\pgfpathlineto{\pgfqpoint{2.747758in}{3.046620in}}%
\pgfpathlineto{\pgfqpoint{2.752967in}{3.038769in}}%
\pgfpathlineto{\pgfqpoint{2.758176in}{3.032978in}}%
\pgfpathlineto{\pgfqpoint{2.763385in}{3.029241in}}%
\pgfpathlineto{\pgfqpoint{2.768594in}{3.026332in}}%
\pgfpathlineto{\pgfqpoint{2.773803in}{3.022534in}}%
\pgfpathlineto{\pgfqpoint{2.779012in}{3.016551in}}%
\pgfpathlineto{\pgfqpoint{2.784221in}{3.008155in}}%
\pgfpathlineto{\pgfqpoint{2.789430in}{2.998300in}}%
\pgfpathlineto{\pgfqpoint{2.794639in}{2.988749in}}%
\pgfpathlineto{\pgfqpoint{2.799848in}{2.981450in}}%
\pgfpathlineto{\pgfqpoint{2.805057in}{2.977937in}}%
\pgfpathlineto{\pgfqpoint{2.810266in}{2.978880in}}%
\pgfpathlineto{\pgfqpoint{2.815475in}{2.983901in}}%
\pgfpathlineto{\pgfqpoint{2.820684in}{2.991634in}}%
\pgfpathlineto{\pgfqpoint{2.825893in}{3.000066in}}%
\pgfpathlineto{\pgfqpoint{2.831102in}{3.007057in}}%
\pgfpathlineto{\pgfqpoint{2.836311in}{3.010900in}}%
\pgfpathlineto{\pgfqpoint{2.841520in}{3.010712in}}%
\pgfpathlineto{\pgfqpoint{2.846729in}{3.006539in}}%
\pgfpathlineto{\pgfqpoint{2.851938in}{2.999215in}}%
\pgfpathlineto{\pgfqpoint{2.857147in}{2.990141in}}%
\pgfpathlineto{\pgfqpoint{2.862356in}{2.981082in}}%
\pgfpathlineto{\pgfqpoint{2.867565in}{2.973944in}}%
\pgfpathlineto{\pgfqpoint{2.872774in}{2.970403in}}%
\pgfpathlineto{\pgfqpoint{2.877983in}{2.971411in}}%
\pgfpathlineto{\pgfqpoint{2.883192in}{2.976813in}}%
\pgfpathlineto{\pgfqpoint{2.888401in}{2.985343in}}%
\pgfpathlineto{\pgfqpoint{2.893610in}{2.995102in}}%
\pgfpathlineto{\pgfqpoint{2.898819in}{3.004243in}}%
\pgfpathlineto{\pgfqpoint{2.904028in}{3.011459in}}%
\pgfpathlineto{\pgfqpoint{2.909238in}{3.016011in}}%
\pgfpathlineto{\pgfqpoint{2.914447in}{3.017345in}}%
\pgfpathlineto{\pgfqpoint{2.919656in}{3.014678in}}%
\pgfpathlineto{\pgfqpoint{2.924865in}{3.006901in}}%
\pgfpathlineto{\pgfqpoint{2.930074in}{2.992933in}}%
\pgfpathlineto{\pgfqpoint{2.935283in}{2.972306in}}%
\pgfpathlineto{\pgfqpoint{2.940492in}{2.945634in}}%
\pgfpathlineto{\pgfqpoint{2.945701in}{2.914734in}}%
\pgfpathlineto{\pgfqpoint{2.950910in}{2.882338in}}%
\pgfpathlineto{\pgfqpoint{2.956119in}{2.851585in}}%
\pgfpathlineto{\pgfqpoint{2.961328in}{2.825503in}}%
\pgfpathlineto{\pgfqpoint{2.966537in}{2.806624in}}%
\pgfpathlineto{\pgfqpoint{2.971746in}{2.796727in}}%
\pgfpathlineto{\pgfqpoint{2.971746in}{2.956210in}}%
\pgfpathlineto{\pgfqpoint{2.971746in}{2.956210in}}%
\pgfpathlineto{\pgfqpoint{2.966537in}{2.970905in}}%
\pgfpathlineto{\pgfqpoint{2.961328in}{2.999152in}}%
\pgfpathlineto{\pgfqpoint{2.956119in}{3.038701in}}%
\pgfpathlineto{\pgfqpoint{2.950910in}{3.086366in}}%
\pgfpathlineto{\pgfqpoint{2.945701in}{3.138414in}}%
\pgfpathlineto{\pgfqpoint{2.940492in}{3.191167in}}%
\pgfpathlineto{\pgfqpoint{2.935283in}{3.241550in}}%
\pgfpathlineto{\pgfqpoint{2.930074in}{3.287384in}}%
\pgfpathlineto{\pgfqpoint{2.924865in}{3.327287in}}%
\pgfpathlineto{\pgfqpoint{2.919656in}{3.360365in}}%
\pgfpathlineto{\pgfqpoint{2.914447in}{3.385926in}}%
\pgfpathlineto{\pgfqpoint{2.909238in}{3.403443in}}%
\pgfpathlineto{\pgfqpoint{2.904028in}{3.412753in}}%
\pgfpathlineto{\pgfqpoint{2.898819in}{3.414286in}}%
\pgfpathlineto{\pgfqpoint{2.893610in}{3.409167in}}%
\pgfpathlineto{\pgfqpoint{2.888401in}{3.399115in}}%
\pgfpathlineto{\pgfqpoint{2.883192in}{3.386284in}}%
\pgfpathlineto{\pgfqpoint{2.877983in}{3.373074in}}%
\pgfpathlineto{\pgfqpoint{2.872774in}{3.361924in}}%
\pgfpathlineto{\pgfqpoint{2.867565in}{3.354965in}}%
\pgfpathlineto{\pgfqpoint{2.862356in}{3.353533in}}%
\pgfpathlineto{\pgfqpoint{2.857147in}{3.357730in}}%
\pgfpathlineto{\pgfqpoint{2.851938in}{3.366250in}}%
\pgfpathlineto{\pgfqpoint{2.846729in}{3.376624in}}%
\pgfpathlineto{\pgfqpoint{2.841520in}{3.385818in}}%
\pgfpathlineto{\pgfqpoint{2.836311in}{3.391007in}}%
\pgfpathlineto{\pgfqpoint{2.831102in}{3.390277in}}%
\pgfpathlineto{\pgfqpoint{2.825893in}{3.383105in}}%
\pgfpathlineto{\pgfqpoint{2.820684in}{3.370519in}}%
\pgfpathlineto{\pgfqpoint{2.815475in}{3.354879in}}%
\pgfpathlineto{\pgfqpoint{2.810266in}{3.339355in}}%
\pgfpathlineto{\pgfqpoint{2.805057in}{3.327173in}}%
\pgfpathlineto{\pgfqpoint{2.799848in}{3.320850in}}%
\pgfpathlineto{\pgfqpoint{2.794639in}{3.321616in}}%
\pgfpathlineto{\pgfqpoint{2.789430in}{3.329167in}}%
\pgfpathlineto{\pgfqpoint{2.784221in}{3.341763in}}%
\pgfpathlineto{\pgfqpoint{2.779012in}{3.356648in}}%
\pgfpathlineto{\pgfqpoint{2.773803in}{3.370646in}}%
\pgfpathlineto{\pgfqpoint{2.768594in}{3.380811in}}%
\pgfpathlineto{\pgfqpoint{2.763385in}{3.384985in}}%
\pgfpathlineto{\pgfqpoint{2.758176in}{3.382133in}}%
\pgfpathlineto{\pgfqpoint{2.752967in}{3.372398in}}%
\pgfpathlineto{\pgfqpoint{2.747758in}{3.356957in}}%
\pgfpathlineto{\pgfqpoint{2.742549in}{3.337767in}}%
\pgfpathlineto{\pgfqpoint{2.737340in}{3.317313in}}%
\pgfpathlineto{\pgfqpoint{2.732131in}{3.298324in}}%
\pgfpathlineto{\pgfqpoint{2.726922in}{3.283376in}}%
\pgfpathlineto{\pgfqpoint{2.721713in}{3.274392in}}%
\pgfpathlineto{\pgfqpoint{2.716504in}{3.272129in}}%
\pgfpathlineto{\pgfqpoint{2.711295in}{3.275897in}}%
\pgfpathlineto{\pgfqpoint{2.706086in}{3.283756in}}%
\pgfpathlineto{\pgfqpoint{2.700877in}{3.293240in}}%
\pgfpathlineto{\pgfqpoint{2.695668in}{3.302334in}}%
\pgfpathlineto{\pgfqpoint{2.690459in}{3.310234in}}%
\pgfpathlineto{\pgfqpoint{2.685250in}{3.317463in}}%
\pgfpathlineto{\pgfqpoint{2.680041in}{3.325265in}}%
\pgfpathlineto{\pgfqpoint{2.674832in}{3.334619in}}%
\pgfpathlineto{\pgfqpoint{2.669623in}{3.345430in}}%
\pgfpathlineto{\pgfqpoint{2.664414in}{3.356322in}}%
\pgfpathlineto{\pgfqpoint{2.659205in}{3.365105in}}%
\pgfpathlineto{\pgfqpoint{2.653996in}{3.369620in}}%
\pgfpathlineto{\pgfqpoint{2.648787in}{3.368558in}}%
\pgfpathlineto{\pgfqpoint{2.643578in}{3.361914in}}%
\pgfpathlineto{\pgfqpoint{2.638369in}{3.350964in}}%
\pgfpathlineto{\pgfqpoint{2.633160in}{3.337845in}}%
\pgfpathlineto{\pgfqpoint{2.627951in}{3.324961in}}%
\pgfpathlineto{\pgfqpoint{2.622742in}{3.314421in}}%
\pgfpathlineto{\pgfqpoint{2.617533in}{3.307705in}}%
\pgfpathlineto{\pgfqpoint{2.612324in}{3.305550in}}%
\pgfpathlineto{\pgfqpoint{2.607115in}{3.307983in}}%
\pgfpathlineto{\pgfqpoint{2.601906in}{3.314317in}}%
\pgfpathlineto{\pgfqpoint{2.596697in}{3.323131in}}%
\pgfpathlineto{\pgfqpoint{2.591487in}{3.332366in}}%
\pgfpathlineto{\pgfqpoint{2.586278in}{3.339699in}}%
\pgfpathlineto{\pgfqpoint{2.581069in}{3.343107in}}%
\pgfpathlineto{\pgfqpoint{2.575860in}{3.341426in}}%
\pgfpathlineto{\pgfqpoint{2.570651in}{3.334647in}}%
\pgfpathlineto{\pgfqpoint{2.565442in}{3.323834in}}%
\pgfpathlineto{\pgfqpoint{2.560233in}{3.310762in}}%
\pgfpathlineto{\pgfqpoint{2.555024in}{3.297421in}}%
\pgfpathlineto{\pgfqpoint{2.549815in}{3.285552in}}%
\pgfpathlineto{\pgfqpoint{2.544606in}{3.276288in}}%
\pgfpathlineto{\pgfqpoint{2.539397in}{3.270003in}}%
\pgfpathlineto{\pgfqpoint{2.534188in}{3.266384in}}%
\pgfpathlineto{\pgfqpoint{2.528979in}{3.264670in}}%
\pgfpathlineto{\pgfqpoint{2.523770in}{3.263965in}}%
\pgfpathlineto{\pgfqpoint{2.518561in}{3.263520in}}%
\pgfpathlineto{\pgfqpoint{2.513352in}{3.262933in}}%
\pgfpathlineto{\pgfqpoint{2.508143in}{3.262239in}}%
\pgfpathlineto{\pgfqpoint{2.502934in}{3.261883in}}%
\pgfpathlineto{\pgfqpoint{2.497725in}{3.262509in}}%
\pgfpathlineto{\pgfqpoint{2.492516in}{3.264602in}}%
\pgfpathlineto{\pgfqpoint{2.487307in}{3.268145in}}%
\pgfpathlineto{\pgfqpoint{2.482098in}{3.272476in}}%
\pgfpathlineto{\pgfqpoint{2.476889in}{3.276468in}}%
\pgfpathlineto{\pgfqpoint{2.471680in}{3.278964in}}%
\pgfpathlineto{\pgfqpoint{2.466471in}{3.279275in}}%
\pgfpathlineto{\pgfqpoint{2.461262in}{3.277569in}}%
\pgfpathlineto{\pgfqpoint{2.456053in}{3.274951in}}%
\pgfpathlineto{\pgfqpoint{2.450844in}{3.273196in}}%
\pgfpathlineto{\pgfqpoint{2.445635in}{3.274203in}}%
\pgfpathlineto{\pgfqpoint{2.440426in}{3.279368in}}%
\pgfpathlineto{\pgfqpoint{2.435217in}{3.289103in}}%
\pgfpathlineto{\pgfqpoint{2.430008in}{3.302664in}}%
\pgfpathlineto{\pgfqpoint{2.424799in}{3.318322in}}%
\pgfpathlineto{\pgfqpoint{2.419590in}{3.333747in}}%
\pgfpathlineto{\pgfqpoint{2.414381in}{3.346493in}}%
\pgfpathlineto{\pgfqpoint{2.409172in}{3.354427in}}%
\pgfpathlineto{\pgfqpoint{2.403963in}{3.356058in}}%
\pgfpathlineto{\pgfqpoint{2.398754in}{3.350766in}}%
\pgfpathlineto{\pgfqpoint{2.393545in}{3.338920in}}%
\pgfpathlineto{\pgfqpoint{2.388336in}{3.321878in}}%
\pgfpathlineto{\pgfqpoint{2.383127in}{3.301817in}}%
\pgfpathlineto{\pgfqpoint{2.377918in}{3.281337in}}%
\pgfpathlineto{\pgfqpoint{2.372709in}{3.262974in}}%
\pgfpathlineto{\pgfqpoint{2.367500in}{3.248701in}}%
\pgfpathlineto{\pgfqpoint{2.362291in}{3.239598in}}%
\pgfpathlineto{\pgfqpoint{2.357082in}{3.235749in}}%
\pgfpathlineto{\pgfqpoint{2.351873in}{3.236372in}}%
\pgfpathlineto{\pgfqpoint{2.346664in}{3.240100in}}%
\pgfpathlineto{\pgfqpoint{2.341455in}{3.245370in}}%
\pgfpathlineto{\pgfqpoint{2.336246in}{3.250769in}}%
\pgfpathlineto{\pgfqpoint{2.331037in}{3.255258in}}%
\pgfpathlineto{\pgfqpoint{2.325828in}{3.258219in}}%
\pgfpathlineto{\pgfqpoint{2.320619in}{3.259397in}}%
\pgfpathlineto{\pgfqpoint{2.315410in}{3.258847in}}%
\pgfpathlineto{\pgfqpoint{2.310201in}{3.256935in}}%
\pgfpathlineto{\pgfqpoint{2.304992in}{3.254308in}}%
\pgfpathlineto{\pgfqpoint{2.299783in}{3.251725in}}%
\pgfpathlineto{\pgfqpoint{2.294574in}{3.249750in}}%
\pgfpathlineto{\pgfqpoint{2.289365in}{3.248443in}}%
\pgfpathlineto{\pgfqpoint{2.284155in}{3.247289in}}%
\pgfpathlineto{\pgfqpoint{2.278946in}{3.245437in}}%
\pgfpathlineto{\pgfqpoint{2.273737in}{3.242088in}}%
\pgfpathlineto{\pgfqpoint{2.268528in}{3.236804in}}%
\pgfpathlineto{\pgfqpoint{2.263319in}{3.229571in}}%
\pgfpathlineto{\pgfqpoint{2.258110in}{3.220672in}}%
\pgfpathlineto{\pgfqpoint{2.252901in}{3.210567in}}%
\pgfpathlineto{\pgfqpoint{2.247692in}{3.199865in}}%
\pgfpathlineto{\pgfqpoint{2.242483in}{3.189375in}}%
\pgfpathlineto{\pgfqpoint{2.237274in}{3.180103in}}%
\pgfpathlineto{\pgfqpoint{2.232065in}{3.173119in}}%
\pgfpathlineto{\pgfqpoint{2.226856in}{3.169303in}}%
\pgfpathlineto{\pgfqpoint{2.221647in}{3.169074in}}%
\pgfpathlineto{\pgfqpoint{2.216438in}{3.172228in}}%
\pgfpathlineto{\pgfqpoint{2.211229in}{3.177913in}}%
\pgfpathlineto{\pgfqpoint{2.206020in}{3.184798in}}%
\pgfpathlineto{\pgfqpoint{2.200811in}{3.191403in}}%
\pgfpathlineto{\pgfqpoint{2.195602in}{3.196491in}}%
\pgfpathlineto{\pgfqpoint{2.190393in}{3.199405in}}%
\pgfpathlineto{\pgfqpoint{2.185184in}{3.200177in}}%
\pgfpathlineto{\pgfqpoint{2.179975in}{3.199409in}}%
\pgfpathlineto{\pgfqpoint{2.174766in}{3.197947in}}%
\pgfpathlineto{\pgfqpoint{2.169557in}{3.196558in}}%
\pgfpathlineto{\pgfqpoint{2.164348in}{3.195730in}}%
\pgfpathlineto{\pgfqpoint{2.159139in}{3.195633in}}%
\pgfpathlineto{\pgfqpoint{2.153930in}{3.196194in}}%
\pgfpathlineto{\pgfqpoint{2.148721in}{3.197241in}}%
\pgfpathlineto{\pgfqpoint{2.143512in}{3.198623in}}%
\pgfpathlineto{\pgfqpoint{2.138303in}{3.200300in}}%
\pgfpathlineto{\pgfqpoint{2.133094in}{3.202327in}}%
\pgfpathlineto{\pgfqpoint{2.127885in}{3.204729in}}%
\pgfpathlineto{\pgfqpoint{2.122676in}{3.207323in}}%
\pgfpathlineto{\pgfqpoint{2.117467in}{3.209658in}}%
\pgfpathlineto{\pgfqpoint{2.112258in}{3.211124in}}%
\pgfpathlineto{\pgfqpoint{2.107049in}{3.211213in}}%
\pgfpathlineto{\pgfqpoint{2.101840in}{3.209746in}}%
\pgfpathlineto{\pgfqpoint{2.096631in}{3.206924in}}%
\pgfpathlineto{\pgfqpoint{2.091422in}{3.203212in}}%
\pgfpathlineto{\pgfqpoint{2.086213in}{3.199161in}}%
\pgfpathlineto{\pgfqpoint{2.081004in}{3.195226in}}%
\pgfpathlineto{\pgfqpoint{2.075795in}{3.191619in}}%
\pgfpathlineto{\pgfqpoint{2.070586in}{3.188192in}}%
\pgfpathlineto{\pgfqpoint{2.065377in}{3.184403in}}%
\pgfpathlineto{\pgfqpoint{2.060168in}{3.179442in}}%
\pgfpathlineto{\pgfqpoint{2.054959in}{3.172507in}}%
\pgfpathlineto{\pgfqpoint{2.049750in}{3.163158in}}%
\pgfpathlineto{\pgfqpoint{2.044541in}{3.151567in}}%
\pgfpathlineto{\pgfqpoint{2.039332in}{3.138567in}}%
\pgfpathlineto{\pgfqpoint{2.034123in}{3.125471in}}%
\pgfpathlineto{\pgfqpoint{2.028914in}{3.113714in}}%
\pgfpathlineto{\pgfqpoint{2.023705in}{3.104444in}}%
\pgfpathlineto{\pgfqpoint{2.018496in}{3.098178in}}%
\pgfpathlineto{\pgfqpoint{2.013287in}{3.094655in}}%
\pgfpathlineto{\pgfqpoint{2.008078in}{3.092901in}}%
\pgfpathlineto{\pgfqpoint{2.002869in}{3.091542in}}%
\pgfpathlineto{\pgfqpoint{1.997660in}{3.089225in}}%
\pgfpathlineto{\pgfqpoint{1.992451in}{3.085039in}}%
\pgfpathlineto{\pgfqpoint{1.987242in}{3.078806in}}%
\pgfpathlineto{\pgfqpoint{1.982033in}{3.071195in}}%
\pgfpathlineto{\pgfqpoint{1.976824in}{3.063567in}}%
\pgfpathlineto{\pgfqpoint{1.971614in}{3.057619in}}%
\pgfpathlineto{\pgfqpoint{1.966405in}{3.054840in}}%
\pgfpathlineto{\pgfqpoint{1.961196in}{3.055985in}}%
\pgfpathlineto{\pgfqpoint{1.955987in}{3.060789in}}%
\pgfpathlineto{\pgfqpoint{1.950778in}{3.068057in}}%
\pgfpathlineto{\pgfqpoint{1.945569in}{3.076127in}}%
\pgfpathlineto{\pgfqpoint{1.940360in}{3.083470in}}%
\pgfpathlineto{\pgfqpoint{1.935151in}{3.089122in}}%
\pgfpathlineto{\pgfqpoint{1.929942in}{3.092755in}}%
\pgfpathlineto{\pgfqpoint{1.924733in}{3.094441in}}%
\pgfpathlineto{\pgfqpoint{1.919524in}{3.094293in}}%
\pgfpathlineto{\pgfqpoint{1.914315in}{3.092267in}}%
\pgfpathlineto{\pgfqpoint{1.909106in}{3.088226in}}%
\pgfpathlineto{\pgfqpoint{1.903897in}{3.082218in}}%
\pgfpathlineto{\pgfqpoint{1.898688in}{3.074762in}}%
\pgfpathlineto{\pgfqpoint{1.893479in}{3.066941in}}%
\pgfpathlineto{\pgfqpoint{1.888270in}{3.060198in}}%
\pgfpathlineto{\pgfqpoint{1.883061in}{3.055896in}}%
\pgfpathlineto{\pgfqpoint{1.877852in}{3.054833in}}%
\pgfpathlineto{\pgfqpoint{1.872643in}{3.056921in}}%
\pgfpathlineto{\pgfqpoint{1.867434in}{3.061209in}}%
\pgfpathlineto{\pgfqpoint{1.862225in}{3.066210in}}%
\pgfpathlineto{\pgfqpoint{1.857016in}{3.070438in}}%
\pgfpathlineto{\pgfqpoint{1.851807in}{3.072889in}}%
\pgfpathlineto{\pgfqpoint{1.846598in}{3.073289in}}%
\pgfpathlineto{\pgfqpoint{1.841389in}{3.071986in}}%
\pgfpathlineto{\pgfqpoint{1.836180in}{3.069600in}}%
\pgfpathlineto{\pgfqpoint{1.830971in}{3.066604in}}%
\pgfpathlineto{\pgfqpoint{1.825762in}{3.063069in}}%
\pgfpathlineto{\pgfqpoint{1.820553in}{3.058686in}}%
\pgfpathlineto{\pgfqpoint{1.815344in}{3.052992in}}%
\pgfpathlineto{\pgfqpoint{1.810135in}{3.045666in}}%
\pgfpathlineto{\pgfqpoint{1.804926in}{3.036715in}}%
\pgfpathlineto{\pgfqpoint{1.799717in}{3.026499in}}%
\pgfpathlineto{\pgfqpoint{1.794508in}{3.015638in}}%
\pgfpathlineto{\pgfqpoint{1.789299in}{3.004880in}}%
\pgfpathlineto{\pgfqpoint{1.784090in}{2.994996in}}%
\pgfpathlineto{\pgfqpoint{1.778881in}{2.986682in}}%
\pgfpathlineto{\pgfqpoint{1.773672in}{2.980447in}}%
\pgfpathlineto{\pgfqpoint{1.768463in}{2.976513in}}%
\pgfpathlineto{\pgfqpoint{1.763254in}{2.974748in}}%
\pgfpathlineto{\pgfqpoint{1.758045in}{2.974690in}}%
\pgfpathlineto{\pgfqpoint{1.752836in}{2.975666in}}%
\pgfpathlineto{\pgfqpoint{1.747627in}{2.976965in}}%
\pgfpathlineto{\pgfqpoint{1.742418in}{2.978046in}}%
\pgfpathlineto{\pgfqpoint{1.737209in}{2.978675in}}%
\pgfpathlineto{\pgfqpoint{1.732000in}{2.978994in}}%
\pgfpathlineto{\pgfqpoint{1.726791in}{2.979437in}}%
\pgfpathlineto{\pgfqpoint{1.721582in}{2.980545in}}%
\pgfpathlineto{\pgfqpoint{1.716373in}{2.982736in}}%
\pgfpathlineto{\pgfqpoint{1.711164in}{2.986139in}}%
\pgfpathlineto{\pgfqpoint{1.705955in}{2.990546in}}%
\pgfpathlineto{\pgfqpoint{1.700746in}{2.995489in}}%
\pgfpathlineto{\pgfqpoint{1.695537in}{3.000355in}}%
\pgfpathlineto{\pgfqpoint{1.690328in}{3.004504in}}%
\pgfpathlineto{\pgfqpoint{1.685119in}{3.007337in}}%
\pgfpathlineto{\pgfqpoint{1.679910in}{3.008379in}}%
\pgfpathlineto{\pgfqpoint{1.674701in}{3.007370in}}%
\pgfpathlineto{\pgfqpoint{1.669492in}{3.004346in}}%
\pgfpathlineto{\pgfqpoint{1.664282in}{2.999668in}}%
\pgfpathlineto{\pgfqpoint{1.659073in}{2.993956in}}%
\pgfpathlineto{\pgfqpoint{1.653864in}{2.987939in}}%
\pgfpathlineto{\pgfqpoint{1.648655in}{2.982297in}}%
\pgfpathlineto{\pgfqpoint{1.643446in}{2.977534in}}%
\pgfpathlineto{\pgfqpoint{1.638237in}{2.973927in}}%
\pgfpathlineto{\pgfqpoint{1.633028in}{2.971540in}}%
\pgfpathlineto{\pgfqpoint{1.627819in}{2.970253in}}%
\pgfpathlineto{\pgfqpoint{1.622610in}{2.969804in}}%
\pgfpathlineto{\pgfqpoint{1.617401in}{2.969816in}}%
\pgfpathlineto{\pgfqpoint{1.612192in}{2.969855in}}%
\pgfpathlineto{\pgfqpoint{1.606983in}{2.969526in}}%
\pgfpathlineto{\pgfqpoint{1.601774in}{2.968567in}}%
\pgfpathlineto{\pgfqpoint{1.596565in}{2.966926in}}%
\pgfpathlineto{\pgfqpoint{1.591356in}{2.964765in}}%
\pgfpathlineto{\pgfqpoint{1.586147in}{2.962404in}}%
\pgfpathlineto{\pgfqpoint{1.580938in}{2.960217in}}%
\pgfpathlineto{\pgfqpoint{1.575729in}{2.958515in}}%
\pgfpathlineto{\pgfqpoint{1.570520in}{2.957449in}}%
\pgfpathlineto{\pgfqpoint{1.565311in}{2.956957in}}%
\pgfpathlineto{\pgfqpoint{1.560102in}{2.956790in}}%
\pgfpathlineto{\pgfqpoint{1.554893in}{2.956601in}}%
\pgfpathlineto{\pgfqpoint{1.549684in}{2.956063in}}%
\pgfpathlineto{\pgfqpoint{1.544475in}{2.954962in}}%
\pgfpathlineto{\pgfqpoint{1.539266in}{2.953228in}}%
\pgfpathlineto{\pgfqpoint{1.534057in}{2.950914in}}%
\pgfpathlineto{\pgfqpoint{1.528848in}{2.948151in}}%
\pgfpathlineto{\pgfqpoint{1.523639in}{2.945118in}}%
\pgfpathlineto{\pgfqpoint{1.518430in}{2.942026in}}%
\pgfpathlineto{\pgfqpoint{1.513221in}{2.939097in}}%
\pgfpathlineto{\pgfqpoint{1.508012in}{2.936534in}}%
\pgfpathlineto{\pgfqpoint{1.502803in}{2.934482in}}%
\pgfpathlineto{\pgfqpoint{1.497594in}{2.933000in}}%
\pgfpathlineto{\pgfqpoint{1.492385in}{2.932060in}}%
\pgfpathlineto{\pgfqpoint{1.487176in}{2.931576in}}%
\pgfpathlineto{\pgfqpoint{1.481967in}{2.931439in}}%
\pgfpathlineto{\pgfqpoint{1.476758in}{2.931555in}}%
\pgfpathlineto{\pgfqpoint{1.471549in}{2.931875in}}%
\pgfpathlineto{\pgfqpoint{1.466340in}{2.932415in}}%
\pgfpathlineto{\pgfqpoint{1.461131in}{2.933242in}}%
\pgfpathlineto{\pgfqpoint{1.455922in}{2.934443in}}%
\pgfpathlineto{\pgfqpoint{1.450713in}{2.936069in}}%
\pgfpathlineto{\pgfqpoint{1.445504in}{2.938091in}}%
\pgfpathlineto{\pgfqpoint{1.440295in}{2.940387in}}%
\pgfpathlineto{\pgfqpoint{1.435086in}{2.942757in}}%
\pgfpathlineto{\pgfqpoint{1.429877in}{2.944964in}}%
\pgfpathlineto{\pgfqpoint{1.424668in}{2.946752in}}%
\pgfpathlineto{\pgfqpoint{1.419459in}{2.947855in}}%
\pgfpathlineto{\pgfqpoint{1.414250in}{2.948005in}}%
\pgfpathlineto{\pgfqpoint{1.409041in}{2.946987in}}%
\pgfpathlineto{\pgfqpoint{1.403832in}{2.944732in}}%
\pgfpathlineto{\pgfqpoint{1.398623in}{2.941433in}}%
\pgfpathlineto{\pgfqpoint{1.393414in}{2.937576in}}%
\pgfpathlineto{\pgfqpoint{1.388205in}{2.933850in}}%
\pgfpathlineto{\pgfqpoint{1.382996in}{2.930929in}}%
\pgfpathlineto{\pgfqpoint{1.377787in}{2.929246in}}%
\pgfpathlineto{\pgfqpoint{1.372578in}{2.928866in}}%
\pgfpathlineto{\pgfqpoint{1.367369in}{2.929529in}}%
\pgfpathlineto{\pgfqpoint{1.362160in}{2.930814in}}%
\pgfpathlineto{\pgfqpoint{1.356951in}{2.932301in}}%
\pgfpathlineto{\pgfqpoint{1.351741in}{2.933669in}}%
\pgfpathlineto{\pgfqpoint{1.346532in}{2.934713in}}%
\pgfpathlineto{\pgfqpoint{1.341323in}{2.935327in}}%
\pgfpathlineto{\pgfqpoint{1.336114in}{2.935500in}}%
\pgfpathlineto{\pgfqpoint{1.330905in}{2.935318in}}%
\pgfpathlineto{\pgfqpoint{1.325696in}{2.934957in}}%
\pgfpathlineto{\pgfqpoint{1.320487in}{2.934646in}}%
\pgfpathlineto{\pgfqpoint{1.315278in}{2.934602in}}%
\pgfpathlineto{\pgfqpoint{1.310069in}{2.934974in}}%
\pgfpathlineto{\pgfqpoint{1.304860in}{2.935804in}}%
\pgfpathlineto{\pgfqpoint{1.299651in}{2.937036in}}%
\pgfpathlineto{\pgfqpoint{1.294442in}{2.938539in}}%
\pgfpathlineto{\pgfqpoint{1.289233in}{2.940154in}}%
\pgfpathlineto{\pgfqpoint{1.284024in}{2.941717in}}%
\pgfpathlineto{\pgfqpoint{1.278815in}{2.943072in}}%
\pgfpathlineto{\pgfqpoint{1.273606in}{2.944062in}}%
\pgfpathlineto{\pgfqpoint{1.268397in}{2.944517in}}%
\pgfpathlineto{\pgfqpoint{1.263188in}{2.944246in}}%
\pgfpathlineto{\pgfqpoint{1.257979in}{2.943027in}}%
\pgfpathlineto{\pgfqpoint{1.252770in}{2.940593in}}%
\pgfpathlineto{\pgfqpoint{1.247561in}{2.936639in}}%
\pgfpathlineto{\pgfqpoint{1.242352in}{2.930846in}}%
\pgfpathlineto{\pgfqpoint{1.237143in}{2.922981in}}%
\pgfpathlineto{\pgfqpoint{1.231934in}{2.913024in}}%
\pgfpathlineto{\pgfqpoint{1.226725in}{2.901300in}}%
\pgfpathlineto{\pgfqpoint{1.221516in}{2.888513in}}%
\pgfpathlineto{\pgfqpoint{1.216307in}{2.875663in}}%
\pgfpathlineto{\pgfqpoint{1.211098in}{2.863863in}}%
\pgfpathlineto{\pgfqpoint{1.205889in}{2.854134in}}%
\pgfpathlineto{\pgfqpoint{1.200680in}{2.847250in}}%
\pgfpathlineto{\pgfqpoint{1.195471in}{2.843697in}}%
\pgfpathclose%
\pgfusepath{stroke,fill}%
\end{pgfscope}%
\begin{pgfscope}%
\pgfpathrectangle{\pgfqpoint{0.898557in}{2.351458in}}{\pgfqpoint{2.323221in}{1.698958in}} %
\pgfusepath{clip}%
\pgfsetrectcap%
\pgfsetroundjoin%
\pgfsetlinewidth{1.003750pt}%
\definecolor{currentstroke}{rgb}{0.309804,0.478431,0.682353}%
\pgfsetstrokecolor{currentstroke}%
\pgfsetdash{}{0pt}%
\pgfpathmoveto{\pgfqpoint{1.195471in}{2.822313in}}%
\pgfpathlineto{\pgfqpoint{1.200680in}{2.825539in}}%
\pgfpathlineto{\pgfqpoint{1.205889in}{2.831815in}}%
\pgfpathlineto{\pgfqpoint{1.216307in}{2.851675in}}%
\pgfpathlineto{\pgfqpoint{1.231934in}{2.887098in}}%
\pgfpathlineto{\pgfqpoint{1.242352in}{2.904463in}}%
\pgfpathlineto{\pgfqpoint{1.247561in}{2.910037in}}%
\pgfpathlineto{\pgfqpoint{1.252770in}{2.913593in}}%
\pgfpathlineto{\pgfqpoint{1.257979in}{2.915353in}}%
\pgfpathlineto{\pgfqpoint{1.263188in}{2.915610in}}%
\pgfpathlineto{\pgfqpoint{1.273606in}{2.913105in}}%
\pgfpathlineto{\pgfqpoint{1.289233in}{2.907947in}}%
\pgfpathlineto{\pgfqpoint{1.299651in}{2.906463in}}%
\pgfpathlineto{\pgfqpoint{1.330905in}{2.904016in}}%
\pgfpathlineto{\pgfqpoint{1.372578in}{2.896799in}}%
\pgfpathlineto{\pgfqpoint{1.382996in}{2.898635in}}%
\pgfpathlineto{\pgfqpoint{1.398623in}{2.905804in}}%
\pgfpathlineto{\pgfqpoint{1.409041in}{2.909560in}}%
\pgfpathlineto{\pgfqpoint{1.419459in}{2.909945in}}%
\pgfpathlineto{\pgfqpoint{1.435086in}{2.906285in}}%
\pgfpathlineto{\pgfqpoint{1.450713in}{2.903384in}}%
\pgfpathlineto{\pgfqpoint{1.502803in}{2.901026in}}%
\pgfpathlineto{\pgfqpoint{1.513221in}{2.903186in}}%
\pgfpathlineto{\pgfqpoint{1.528848in}{2.909441in}}%
\pgfpathlineto{\pgfqpoint{1.544475in}{2.916437in}}%
\pgfpathlineto{\pgfqpoint{1.554893in}{2.918455in}}%
\pgfpathlineto{\pgfqpoint{1.575729in}{2.919299in}}%
\pgfpathlineto{\pgfqpoint{1.586147in}{2.922759in}}%
\pgfpathlineto{\pgfqpoint{1.601774in}{2.928550in}}%
\pgfpathlineto{\pgfqpoint{1.612192in}{2.929167in}}%
\pgfpathlineto{\pgfqpoint{1.633028in}{2.927829in}}%
\pgfpathlineto{\pgfqpoint{1.643446in}{2.930400in}}%
\pgfpathlineto{\pgfqpoint{1.659073in}{2.938467in}}%
\pgfpathlineto{\pgfqpoint{1.669492in}{2.943564in}}%
\pgfpathlineto{\pgfqpoint{1.679910in}{2.945219in}}%
\pgfpathlineto{\pgfqpoint{1.690328in}{2.942209in}}%
\pgfpathlineto{\pgfqpoint{1.716373in}{2.928216in}}%
\pgfpathlineto{\pgfqpoint{1.726791in}{2.927211in}}%
\pgfpathlineto{\pgfqpoint{1.758045in}{2.929564in}}%
\pgfpathlineto{\pgfqpoint{1.768463in}{2.930603in}}%
\pgfpathlineto{\pgfqpoint{1.773672in}{2.932963in}}%
\pgfpathlineto{\pgfqpoint{1.778881in}{2.937032in}}%
\pgfpathlineto{\pgfqpoint{1.789299in}{2.949531in}}%
\pgfpathlineto{\pgfqpoint{1.804926in}{2.969222in}}%
\pgfpathlineto{\pgfqpoint{1.815344in}{2.976628in}}%
\pgfpathlineto{\pgfqpoint{1.830971in}{2.981896in}}%
\pgfpathlineto{\pgfqpoint{1.851807in}{2.987833in}}%
\pgfpathlineto{\pgfqpoint{1.857016in}{2.987445in}}%
\pgfpathlineto{\pgfqpoint{1.867434in}{2.983344in}}%
\pgfpathlineto{\pgfqpoint{1.877852in}{2.978165in}}%
\pgfpathlineto{\pgfqpoint{1.883061in}{2.976743in}}%
\pgfpathlineto{\pgfqpoint{1.888270in}{2.976539in}}%
\pgfpathlineto{\pgfqpoint{1.898688in}{2.979275in}}%
\pgfpathlineto{\pgfqpoint{1.924733in}{2.990407in}}%
\pgfpathlineto{\pgfqpoint{1.935151in}{2.991430in}}%
\pgfpathlineto{\pgfqpoint{1.940360in}{2.990032in}}%
\pgfpathlineto{\pgfqpoint{1.950778in}{2.983434in}}%
\pgfpathlineto{\pgfqpoint{1.961196in}{2.975853in}}%
\pgfpathlineto{\pgfqpoint{1.966405in}{2.973844in}}%
\pgfpathlineto{\pgfqpoint{1.971614in}{2.973769in}}%
\pgfpathlineto{\pgfqpoint{1.976824in}{2.975618in}}%
\pgfpathlineto{\pgfqpoint{1.987242in}{2.983176in}}%
\pgfpathlineto{\pgfqpoint{2.002869in}{2.995640in}}%
\pgfpathlineto{\pgfqpoint{2.018496in}{3.004560in}}%
\pgfpathlineto{\pgfqpoint{2.039332in}{3.016660in}}%
\pgfpathlineto{\pgfqpoint{2.049750in}{3.024896in}}%
\pgfpathlineto{\pgfqpoint{2.065377in}{3.041742in}}%
\pgfpathlineto{\pgfqpoint{2.075795in}{3.051876in}}%
\pgfpathlineto{\pgfqpoint{2.081004in}{3.054934in}}%
\pgfpathlineto{\pgfqpoint{2.086213in}{3.056377in}}%
\pgfpathlineto{\pgfqpoint{2.096631in}{3.055314in}}%
\pgfpathlineto{\pgfqpoint{2.107049in}{3.051214in}}%
\pgfpathlineto{\pgfqpoint{2.138303in}{3.034768in}}%
\pgfpathlineto{\pgfqpoint{2.153930in}{3.024138in}}%
\pgfpathlineto{\pgfqpoint{2.174766in}{3.007142in}}%
\pgfpathlineto{\pgfqpoint{2.179975in}{3.004843in}}%
\pgfpathlineto{\pgfqpoint{2.185184in}{3.004077in}}%
\pgfpathlineto{\pgfqpoint{2.195602in}{3.006215in}}%
\pgfpathlineto{\pgfqpoint{2.211229in}{3.010677in}}%
\pgfpathlineto{\pgfqpoint{2.221647in}{3.014023in}}%
\pgfpathlineto{\pgfqpoint{2.226856in}{3.017571in}}%
\pgfpathlineto{\pgfqpoint{2.237274in}{3.029620in}}%
\pgfpathlineto{\pgfqpoint{2.252901in}{3.050153in}}%
\pgfpathlineto{\pgfqpoint{2.258110in}{3.054687in}}%
\pgfpathlineto{\pgfqpoint{2.268528in}{3.059459in}}%
\pgfpathlineto{\pgfqpoint{2.284155in}{3.060808in}}%
\pgfpathlineto{\pgfqpoint{2.294574in}{3.063078in}}%
\pgfpathlineto{\pgfqpoint{2.304992in}{3.069380in}}%
\pgfpathlineto{\pgfqpoint{2.320619in}{3.081831in}}%
\pgfpathlineto{\pgfqpoint{2.325828in}{3.084650in}}%
\pgfpathlineto{\pgfqpoint{2.331037in}{3.086054in}}%
\pgfpathlineto{\pgfqpoint{2.336246in}{3.085926in}}%
\pgfpathlineto{\pgfqpoint{2.351873in}{3.081847in}}%
\pgfpathlineto{\pgfqpoint{2.357082in}{3.083210in}}%
\pgfpathlineto{\pgfqpoint{2.362291in}{3.088128in}}%
\pgfpathlineto{\pgfqpoint{2.367500in}{3.097209in}}%
\pgfpathlineto{\pgfqpoint{2.377918in}{3.125505in}}%
\pgfpathlineto{\pgfqpoint{2.388336in}{3.155193in}}%
\pgfpathlineto{\pgfqpoint{2.393545in}{3.165194in}}%
\pgfpathlineto{\pgfqpoint{2.398754in}{3.170060in}}%
\pgfpathlineto{\pgfqpoint{2.403963in}{3.169430in}}%
\pgfpathlineto{\pgfqpoint{2.409172in}{3.163714in}}%
\pgfpathlineto{\pgfqpoint{2.414381in}{3.153876in}}%
\pgfpathlineto{\pgfqpoint{2.440426in}{3.093531in}}%
\pgfpathlineto{\pgfqpoint{2.445635in}{3.088149in}}%
\pgfpathlineto{\pgfqpoint{2.450844in}{3.085690in}}%
\pgfpathlineto{\pgfqpoint{2.476889in}{3.082061in}}%
\pgfpathlineto{\pgfqpoint{2.487307in}{3.077757in}}%
\pgfpathlineto{\pgfqpoint{2.492516in}{3.076737in}}%
\pgfpathlineto{\pgfqpoint{2.497725in}{3.077081in}}%
\pgfpathlineto{\pgfqpoint{2.502934in}{3.078935in}}%
\pgfpathlineto{\pgfqpoint{2.513352in}{3.086820in}}%
\pgfpathlineto{\pgfqpoint{2.523770in}{3.098684in}}%
\pgfpathlineto{\pgfqpoint{2.544606in}{3.127083in}}%
\pgfpathlineto{\pgfqpoint{2.555024in}{3.144449in}}%
\pgfpathlineto{\pgfqpoint{2.570651in}{3.173339in}}%
\pgfpathlineto{\pgfqpoint{2.575860in}{3.179454in}}%
\pgfpathlineto{\pgfqpoint{2.581069in}{3.181121in}}%
\pgfpathlineto{\pgfqpoint{2.586278in}{3.177474in}}%
\pgfpathlineto{\pgfqpoint{2.591487in}{3.168968in}}%
\pgfpathlineto{\pgfqpoint{2.601906in}{3.145936in}}%
\pgfpathlineto{\pgfqpoint{2.607115in}{3.137326in}}%
\pgfpathlineto{\pgfqpoint{2.612324in}{3.133924in}}%
\pgfpathlineto{\pgfqpoint{2.617533in}{3.136574in}}%
\pgfpathlineto{\pgfqpoint{2.622742in}{3.144543in}}%
\pgfpathlineto{\pgfqpoint{2.638369in}{3.178339in}}%
\pgfpathlineto{\pgfqpoint{2.643578in}{3.185358in}}%
\pgfpathlineto{\pgfqpoint{2.648787in}{3.188224in}}%
\pgfpathlineto{\pgfqpoint{2.653996in}{3.187136in}}%
\pgfpathlineto{\pgfqpoint{2.659205in}{3.183090in}}%
\pgfpathlineto{\pgfqpoint{2.669623in}{3.172247in}}%
\pgfpathlineto{\pgfqpoint{2.674832in}{3.168363in}}%
\pgfpathlineto{\pgfqpoint{2.680041in}{3.166500in}}%
\pgfpathlineto{\pgfqpoint{2.690459in}{3.166946in}}%
\pgfpathlineto{\pgfqpoint{2.700877in}{3.166630in}}%
\pgfpathlineto{\pgfqpoint{2.711295in}{3.164852in}}%
\pgfpathlineto{\pgfqpoint{2.716504in}{3.165953in}}%
\pgfpathlineto{\pgfqpoint{2.721713in}{3.169510in}}%
\pgfpathlineto{\pgfqpoint{2.732131in}{3.182558in}}%
\pgfpathlineto{\pgfqpoint{2.742549in}{3.196489in}}%
\pgfpathlineto{\pgfqpoint{2.752967in}{3.205584in}}%
\pgfpathlineto{\pgfqpoint{2.758176in}{3.207555in}}%
\pgfpathlineto{\pgfqpoint{2.763385in}{3.207113in}}%
\pgfpathlineto{\pgfqpoint{2.768594in}{3.203571in}}%
\pgfpathlineto{\pgfqpoint{2.773803in}{3.196590in}}%
\pgfpathlineto{\pgfqpoint{2.784221in}{3.174959in}}%
\pgfpathlineto{\pgfqpoint{2.789430in}{3.163733in}}%
\pgfpathlineto{\pgfqpoint{2.794639in}{3.155183in}}%
\pgfpathlineto{\pgfqpoint{2.799848in}{3.151150in}}%
\pgfpathlineto{\pgfqpoint{2.805057in}{3.152555in}}%
\pgfpathlineto{\pgfqpoint{2.810266in}{3.159118in}}%
\pgfpathlineto{\pgfqpoint{2.831102in}{3.198667in}}%
\pgfpathlineto{\pgfqpoint{2.836311in}{3.200954in}}%
\pgfpathlineto{\pgfqpoint{2.841520in}{3.198265in}}%
\pgfpathlineto{\pgfqpoint{2.846729in}{3.191582in}}%
\pgfpathlineto{\pgfqpoint{2.857147in}{3.173935in}}%
\pgfpathlineto{\pgfqpoint{2.862356in}{3.167307in}}%
\pgfpathlineto{\pgfqpoint{2.867565in}{3.164454in}}%
\pgfpathlineto{\pgfqpoint{2.872774in}{3.166164in}}%
\pgfpathlineto{\pgfqpoint{2.877983in}{3.172243in}}%
\pgfpathlineto{\pgfqpoint{2.898819in}{3.209264in}}%
\pgfpathlineto{\pgfqpoint{2.904028in}{3.212106in}}%
\pgfpathlineto{\pgfqpoint{2.909238in}{3.209727in}}%
\pgfpathlineto{\pgfqpoint{2.914447in}{3.201635in}}%
\pgfpathlineto{\pgfqpoint{2.919656in}{3.187521in}}%
\pgfpathlineto{\pgfqpoint{2.924865in}{3.167094in}}%
\pgfpathlineto{\pgfqpoint{2.930074in}{3.140158in}}%
\pgfpathlineto{\pgfqpoint{2.940492in}{3.068401in}}%
\pgfpathlineto{\pgfqpoint{2.956119in}{2.945143in}}%
\pgfpathlineto{\pgfqpoint{2.961328in}{2.912328in}}%
\pgfpathlineto{\pgfqpoint{2.966537in}{2.888765in}}%
\pgfpathlineto{\pgfqpoint{2.971746in}{2.876468in}}%
\pgfpathlineto{\pgfqpoint{2.971746in}{2.876468in}}%
\pgfusepath{stroke}%
\end{pgfscope}%
\begin{pgfscope}%
\pgfpathrectangle{\pgfqpoint{0.898557in}{2.351458in}}{\pgfqpoint{2.323221in}{1.698958in}} %
\pgfusepath{clip}%
\pgfsetbuttcap%
\pgfsetroundjoin%
\pgfsetlinewidth{1.003750pt}%
\definecolor{currentstroke}{rgb}{0.000000,0.000000,0.000000}%
\pgfsetstrokecolor{currentstroke}%
\pgfsetdash{{1.000000pt}{3.000000pt}}{0.000000pt}%
\pgfpathmoveto{\pgfqpoint{1.195471in}{2.351458in}}%
\pgfpathlineto{\pgfqpoint{1.195471in}{4.050417in}}%
\pgfusepath{stroke}%
\end{pgfscope}%
\begin{pgfscope}%
\pgfpathrectangle{\pgfqpoint{0.898557in}{2.351458in}}{\pgfqpoint{2.323221in}{1.698958in}} %
\pgfusepath{clip}%
\pgfsetbuttcap%
\pgfsetroundjoin%
\pgfsetlinewidth{1.003750pt}%
\definecolor{currentstroke}{rgb}{1.000000,0.400000,0.200000}%
\pgfsetstrokecolor{currentstroke}%
\pgfsetdash{{1.000000pt}{3.000000pt}}{0.000000pt}%
\pgfpathmoveto{\pgfqpoint{0.898557in}{2.957174in}}%
\pgfpathlineto{\pgfqpoint{3.221779in}{2.957174in}}%
\pgfusepath{stroke}%
\end{pgfscope}%
\begin{pgfscope}%
\pgfsetrectcap%
\pgfsetmiterjoin%
\pgfsetlinewidth{1.003750pt}%
\definecolor{currentstroke}{rgb}{0.000000,0.000000,0.000000}%
\pgfsetstrokecolor{currentstroke}%
\pgfsetdash{}{0pt}%
\pgfpathmoveto{\pgfqpoint{3.221779in}{2.351458in}}%
\pgfpathlineto{\pgfqpoint{3.221779in}{4.050417in}}%
\pgfusepath{stroke}%
\end{pgfscope}%
\begin{pgfscope}%
\pgfsetrectcap%
\pgfsetmiterjoin%
\pgfsetlinewidth{1.003750pt}%
\definecolor{currentstroke}{rgb}{0.000000,0.000000,0.000000}%
\pgfsetstrokecolor{currentstroke}%
\pgfsetdash{}{0pt}%
\pgfpathmoveto{\pgfqpoint{0.898557in}{2.351458in}}%
\pgfpathlineto{\pgfqpoint{0.898557in}{4.050417in}}%
\pgfusepath{stroke}%
\end{pgfscope}%
\begin{pgfscope}%
\pgfsetrectcap%
\pgfsetmiterjoin%
\pgfsetlinewidth{1.003750pt}%
\definecolor{currentstroke}{rgb}{0.000000,0.000000,0.000000}%
\pgfsetstrokecolor{currentstroke}%
\pgfsetdash{}{0pt}%
\pgfpathmoveto{\pgfqpoint{0.898557in}{4.050417in}}%
\pgfpathlineto{\pgfqpoint{3.221779in}{4.050417in}}%
\pgfusepath{stroke}%
\end{pgfscope}%
\begin{pgfscope}%
\pgfsetrectcap%
\pgfsetmiterjoin%
\pgfsetlinewidth{1.003750pt}%
\definecolor{currentstroke}{rgb}{0.000000,0.000000,0.000000}%
\pgfsetstrokecolor{currentstroke}%
\pgfsetdash{}{0pt}%
\pgfpathmoveto{\pgfqpoint{0.898557in}{2.351458in}}%
\pgfpathlineto{\pgfqpoint{3.221779in}{2.351458in}}%
\pgfusepath{stroke}%
\end{pgfscope}%
\begin{pgfscope}%
\pgfsetbuttcap%
\pgfsetroundjoin%
\definecolor{currentfill}{rgb}{0.000000,0.000000,0.000000}%
\pgfsetfillcolor{currentfill}%
\pgfsetlinewidth{0.501875pt}%
\definecolor{currentstroke}{rgb}{0.000000,0.000000,0.000000}%
\pgfsetstrokecolor{currentstroke}%
\pgfsetdash{}{0pt}%
\pgfsys@defobject{currentmarker}{\pgfqpoint{0.000000in}{0.000000in}}{\pgfqpoint{0.000000in}{0.055556in}}{%
\pgfpathmoveto{\pgfqpoint{0.000000in}{0.000000in}}%
\pgfpathlineto{\pgfqpoint{0.000000in}{0.055556in}}%
\pgfusepath{stroke,fill}%
}%
\begin{pgfscope}%
\pgfsys@transformshift{1.195471in}{2.351458in}%
\pgfsys@useobject{currentmarker}{}%
\end{pgfscope}%
\end{pgfscope}%
\begin{pgfscope}%
\pgfsetbuttcap%
\pgfsetroundjoin%
\definecolor{currentfill}{rgb}{0.000000,0.000000,0.000000}%
\pgfsetfillcolor{currentfill}%
\pgfsetlinewidth{0.501875pt}%
\definecolor{currentstroke}{rgb}{0.000000,0.000000,0.000000}%
\pgfsetstrokecolor{currentstroke}%
\pgfsetdash{}{0pt}%
\pgfsys@defobject{currentmarker}{\pgfqpoint{0.000000in}{-0.055556in}}{\pgfqpoint{0.000000in}{0.000000in}}{%
\pgfpathmoveto{\pgfqpoint{0.000000in}{0.000000in}}%
\pgfpathlineto{\pgfqpoint{0.000000in}{-0.055556in}}%
\pgfusepath{stroke,fill}%
}%
\begin{pgfscope}%
\pgfsys@transformshift{1.195471in}{4.050417in}%
\pgfsys@useobject{currentmarker}{}%
\end{pgfscope}%
\end{pgfscope}%
\begin{pgfscope}%
\pgftext[x=1.195471in,y=2.295903in,,top]{\fontsize{11.000000}{13.200000}\selectfont 0}%
\end{pgfscope}%
\begin{pgfscope}%
\pgfsetbuttcap%
\pgfsetroundjoin%
\definecolor{currentfill}{rgb}{0.000000,0.000000,0.000000}%
\pgfsetfillcolor{currentfill}%
\pgfsetlinewidth{0.501875pt}%
\definecolor{currentstroke}{rgb}{0.000000,0.000000,0.000000}%
\pgfsetstrokecolor{currentstroke}%
\pgfsetdash{}{0pt}%
\pgfsys@defobject{currentmarker}{\pgfqpoint{0.000000in}{0.000000in}}{\pgfqpoint{0.000000in}{0.055556in}}{%
\pgfpathmoveto{\pgfqpoint{0.000000in}{0.000000in}}%
\pgfpathlineto{\pgfqpoint{0.000000in}{0.055556in}}%
\pgfusepath{stroke,fill}%
}%
\begin{pgfscope}%
\pgfsys@transformshift{1.820601in}{2.351458in}%
\pgfsys@useobject{currentmarker}{}%
\end{pgfscope}%
\end{pgfscope}%
\begin{pgfscope}%
\pgfsetbuttcap%
\pgfsetroundjoin%
\definecolor{currentfill}{rgb}{0.000000,0.000000,0.000000}%
\pgfsetfillcolor{currentfill}%
\pgfsetlinewidth{0.501875pt}%
\definecolor{currentstroke}{rgb}{0.000000,0.000000,0.000000}%
\pgfsetstrokecolor{currentstroke}%
\pgfsetdash{}{0pt}%
\pgfsys@defobject{currentmarker}{\pgfqpoint{0.000000in}{-0.055556in}}{\pgfqpoint{0.000000in}{0.000000in}}{%
\pgfpathmoveto{\pgfqpoint{0.000000in}{0.000000in}}%
\pgfpathlineto{\pgfqpoint{0.000000in}{-0.055556in}}%
\pgfusepath{stroke,fill}%
}%
\begin{pgfscope}%
\pgfsys@transformshift{1.820601in}{4.050417in}%
\pgfsys@useobject{currentmarker}{}%
\end{pgfscope}%
\end{pgfscope}%
\begin{pgfscope}%
\pgftext[x=1.820601in,y=2.295903in,,top]{\fontsize{11.000000}{13.200000}\selectfont 5}%
\end{pgfscope}%
\begin{pgfscope}%
\pgfsetbuttcap%
\pgfsetroundjoin%
\definecolor{currentfill}{rgb}{0.000000,0.000000,0.000000}%
\pgfsetfillcolor{currentfill}%
\pgfsetlinewidth{0.501875pt}%
\definecolor{currentstroke}{rgb}{0.000000,0.000000,0.000000}%
\pgfsetstrokecolor{currentstroke}%
\pgfsetdash{}{0pt}%
\pgfsys@defobject{currentmarker}{\pgfqpoint{0.000000in}{0.000000in}}{\pgfqpoint{0.000000in}{0.055556in}}{%
\pgfpathmoveto{\pgfqpoint{0.000000in}{0.000000in}}%
\pgfpathlineto{\pgfqpoint{0.000000in}{0.055556in}}%
\pgfusepath{stroke,fill}%
}%
\begin{pgfscope}%
\pgfsys@transformshift{2.445730in}{2.351458in}%
\pgfsys@useobject{currentmarker}{}%
\end{pgfscope}%
\end{pgfscope}%
\begin{pgfscope}%
\pgfsetbuttcap%
\pgfsetroundjoin%
\definecolor{currentfill}{rgb}{0.000000,0.000000,0.000000}%
\pgfsetfillcolor{currentfill}%
\pgfsetlinewidth{0.501875pt}%
\definecolor{currentstroke}{rgb}{0.000000,0.000000,0.000000}%
\pgfsetstrokecolor{currentstroke}%
\pgfsetdash{}{0pt}%
\pgfsys@defobject{currentmarker}{\pgfqpoint{0.000000in}{-0.055556in}}{\pgfqpoint{0.000000in}{0.000000in}}{%
\pgfpathmoveto{\pgfqpoint{0.000000in}{0.000000in}}%
\pgfpathlineto{\pgfqpoint{0.000000in}{-0.055556in}}%
\pgfusepath{stroke,fill}%
}%
\begin{pgfscope}%
\pgfsys@transformshift{2.445730in}{4.050417in}%
\pgfsys@useobject{currentmarker}{}%
\end{pgfscope}%
\end{pgfscope}%
\begin{pgfscope}%
\pgftext[x=2.445730in,y=2.295903in,,top]{\fontsize{11.000000}{13.200000}\selectfont 10}%
\end{pgfscope}%
\begin{pgfscope}%
\pgfsetbuttcap%
\pgfsetroundjoin%
\definecolor{currentfill}{rgb}{0.000000,0.000000,0.000000}%
\pgfsetfillcolor{currentfill}%
\pgfsetlinewidth{0.501875pt}%
\definecolor{currentstroke}{rgb}{0.000000,0.000000,0.000000}%
\pgfsetstrokecolor{currentstroke}%
\pgfsetdash{}{0pt}%
\pgfsys@defobject{currentmarker}{\pgfqpoint{0.000000in}{0.000000in}}{\pgfqpoint{0.000000in}{0.055556in}}{%
\pgfpathmoveto{\pgfqpoint{0.000000in}{0.000000in}}%
\pgfpathlineto{\pgfqpoint{0.000000in}{0.055556in}}%
\pgfusepath{stroke,fill}%
}%
\begin{pgfscope}%
\pgfsys@transformshift{3.070860in}{2.351458in}%
\pgfsys@useobject{currentmarker}{}%
\end{pgfscope}%
\end{pgfscope}%
\begin{pgfscope}%
\pgfsetbuttcap%
\pgfsetroundjoin%
\definecolor{currentfill}{rgb}{0.000000,0.000000,0.000000}%
\pgfsetfillcolor{currentfill}%
\pgfsetlinewidth{0.501875pt}%
\definecolor{currentstroke}{rgb}{0.000000,0.000000,0.000000}%
\pgfsetstrokecolor{currentstroke}%
\pgfsetdash{}{0pt}%
\pgfsys@defobject{currentmarker}{\pgfqpoint{0.000000in}{-0.055556in}}{\pgfqpoint{0.000000in}{0.000000in}}{%
\pgfpathmoveto{\pgfqpoint{0.000000in}{0.000000in}}%
\pgfpathlineto{\pgfqpoint{0.000000in}{-0.055556in}}%
\pgfusepath{stroke,fill}%
}%
\begin{pgfscope}%
\pgfsys@transformshift{3.070860in}{4.050417in}%
\pgfsys@useobject{currentmarker}{}%
\end{pgfscope}%
\end{pgfscope}%
\begin{pgfscope}%
\pgftext[x=3.070860in,y=2.295903in,,top]{\fontsize{11.000000}{13.200000}\selectfont 15}%
\end{pgfscope}%
\begin{pgfscope}%
\pgfsetbuttcap%
\pgfsetroundjoin%
\definecolor{currentfill}{rgb}{0.000000,0.000000,0.000000}%
\pgfsetfillcolor{currentfill}%
\pgfsetlinewidth{0.501875pt}%
\definecolor{currentstroke}{rgb}{0.000000,0.000000,0.000000}%
\pgfsetstrokecolor{currentstroke}%
\pgfsetdash{}{0pt}%
\pgfsys@defobject{currentmarker}{\pgfqpoint{0.000000in}{0.000000in}}{\pgfqpoint{0.055556in}{0.000000in}}{%
\pgfpathmoveto{\pgfqpoint{0.000000in}{0.000000in}}%
\pgfpathlineto{\pgfqpoint{0.055556in}{0.000000in}}%
\pgfusepath{stroke,fill}%
}%
\begin{pgfscope}%
\pgfsys@transformshift{0.898557in}{2.351458in}%
\pgfsys@useobject{currentmarker}{}%
\end{pgfscope}%
\end{pgfscope}%
\begin{pgfscope}%
\pgfsetbuttcap%
\pgfsetroundjoin%
\definecolor{currentfill}{rgb}{0.000000,0.000000,0.000000}%
\pgfsetfillcolor{currentfill}%
\pgfsetlinewidth{0.501875pt}%
\definecolor{currentstroke}{rgb}{0.000000,0.000000,0.000000}%
\pgfsetstrokecolor{currentstroke}%
\pgfsetdash{}{0pt}%
\pgfsys@defobject{currentmarker}{\pgfqpoint{-0.055556in}{0.000000in}}{\pgfqpoint{0.000000in}{0.000000in}}{%
\pgfpathmoveto{\pgfqpoint{0.000000in}{0.000000in}}%
\pgfpathlineto{\pgfqpoint{-0.055556in}{0.000000in}}%
\pgfusepath{stroke,fill}%
}%
\begin{pgfscope}%
\pgfsys@transformshift{3.221779in}{2.351458in}%
\pgfsys@useobject{currentmarker}{}%
\end{pgfscope}%
\end{pgfscope}%
\begin{pgfscope}%
\pgftext[x=0.843001in,y=2.351458in,right,]{\fontsize{11.000000}{13.200000}\selectfont 0}%
\end{pgfscope}%
\begin{pgfscope}%
\pgfsetbuttcap%
\pgfsetroundjoin%
\definecolor{currentfill}{rgb}{0.000000,0.000000,0.000000}%
\pgfsetfillcolor{currentfill}%
\pgfsetlinewidth{0.501875pt}%
\definecolor{currentstroke}{rgb}{0.000000,0.000000,0.000000}%
\pgfsetstrokecolor{currentstroke}%
\pgfsetdash{}{0pt}%
\pgfsys@defobject{currentmarker}{\pgfqpoint{0.000000in}{0.000000in}}{\pgfqpoint{0.055556in}{0.000000in}}{%
\pgfpathmoveto{\pgfqpoint{0.000000in}{0.000000in}}%
\pgfpathlineto{\pgfqpoint{0.055556in}{0.000000in}}%
\pgfusepath{stroke,fill}%
}%
\begin{pgfscope}%
\pgfsys@transformshift{0.898557in}{2.646929in}%
\pgfsys@useobject{currentmarker}{}%
\end{pgfscope}%
\end{pgfscope}%
\begin{pgfscope}%
\pgfsetbuttcap%
\pgfsetroundjoin%
\definecolor{currentfill}{rgb}{0.000000,0.000000,0.000000}%
\pgfsetfillcolor{currentfill}%
\pgfsetlinewidth{0.501875pt}%
\definecolor{currentstroke}{rgb}{0.000000,0.000000,0.000000}%
\pgfsetstrokecolor{currentstroke}%
\pgfsetdash{}{0pt}%
\pgfsys@defobject{currentmarker}{\pgfqpoint{-0.055556in}{0.000000in}}{\pgfqpoint{0.000000in}{0.000000in}}{%
\pgfpathmoveto{\pgfqpoint{0.000000in}{0.000000in}}%
\pgfpathlineto{\pgfqpoint{-0.055556in}{0.000000in}}%
\pgfusepath{stroke,fill}%
}%
\begin{pgfscope}%
\pgfsys@transformshift{3.221779in}{2.646929in}%
\pgfsys@useobject{currentmarker}{}%
\end{pgfscope}%
\end{pgfscope}%
\begin{pgfscope}%
\pgftext[x=0.843001in,y=2.646929in,right,]{\fontsize{11.000000}{13.200000}\selectfont 20}%
\end{pgfscope}%
\begin{pgfscope}%
\pgfsetbuttcap%
\pgfsetroundjoin%
\definecolor{currentfill}{rgb}{0.000000,0.000000,0.000000}%
\pgfsetfillcolor{currentfill}%
\pgfsetlinewidth{0.501875pt}%
\definecolor{currentstroke}{rgb}{0.000000,0.000000,0.000000}%
\pgfsetstrokecolor{currentstroke}%
\pgfsetdash{}{0pt}%
\pgfsys@defobject{currentmarker}{\pgfqpoint{0.000000in}{0.000000in}}{\pgfqpoint{0.055556in}{0.000000in}}{%
\pgfpathmoveto{\pgfqpoint{0.000000in}{0.000000in}}%
\pgfpathlineto{\pgfqpoint{0.055556in}{0.000000in}}%
\pgfusepath{stroke,fill}%
}%
\begin{pgfscope}%
\pgfsys@transformshift{0.898557in}{2.942400in}%
\pgfsys@useobject{currentmarker}{}%
\end{pgfscope}%
\end{pgfscope}%
\begin{pgfscope}%
\pgfsetbuttcap%
\pgfsetroundjoin%
\definecolor{currentfill}{rgb}{0.000000,0.000000,0.000000}%
\pgfsetfillcolor{currentfill}%
\pgfsetlinewidth{0.501875pt}%
\definecolor{currentstroke}{rgb}{0.000000,0.000000,0.000000}%
\pgfsetstrokecolor{currentstroke}%
\pgfsetdash{}{0pt}%
\pgfsys@defobject{currentmarker}{\pgfqpoint{-0.055556in}{0.000000in}}{\pgfqpoint{0.000000in}{0.000000in}}{%
\pgfpathmoveto{\pgfqpoint{0.000000in}{0.000000in}}%
\pgfpathlineto{\pgfqpoint{-0.055556in}{0.000000in}}%
\pgfusepath{stroke,fill}%
}%
\begin{pgfscope}%
\pgfsys@transformshift{3.221779in}{2.942400in}%
\pgfsys@useobject{currentmarker}{}%
\end{pgfscope}%
\end{pgfscope}%
\begin{pgfscope}%
\pgftext[x=0.843001in,y=2.942400in,right,]{\fontsize{11.000000}{13.200000}\selectfont 40}%
\end{pgfscope}%
\begin{pgfscope}%
\pgfsetbuttcap%
\pgfsetroundjoin%
\definecolor{currentfill}{rgb}{0.000000,0.000000,0.000000}%
\pgfsetfillcolor{currentfill}%
\pgfsetlinewidth{0.501875pt}%
\definecolor{currentstroke}{rgb}{0.000000,0.000000,0.000000}%
\pgfsetstrokecolor{currentstroke}%
\pgfsetdash{}{0pt}%
\pgfsys@defobject{currentmarker}{\pgfqpoint{0.000000in}{0.000000in}}{\pgfqpoint{0.055556in}{0.000000in}}{%
\pgfpathmoveto{\pgfqpoint{0.000000in}{0.000000in}}%
\pgfpathlineto{\pgfqpoint{0.055556in}{0.000000in}}%
\pgfusepath{stroke,fill}%
}%
\begin{pgfscope}%
\pgfsys@transformshift{0.898557in}{3.237871in}%
\pgfsys@useobject{currentmarker}{}%
\end{pgfscope}%
\end{pgfscope}%
\begin{pgfscope}%
\pgfsetbuttcap%
\pgfsetroundjoin%
\definecolor{currentfill}{rgb}{0.000000,0.000000,0.000000}%
\pgfsetfillcolor{currentfill}%
\pgfsetlinewidth{0.501875pt}%
\definecolor{currentstroke}{rgb}{0.000000,0.000000,0.000000}%
\pgfsetstrokecolor{currentstroke}%
\pgfsetdash{}{0pt}%
\pgfsys@defobject{currentmarker}{\pgfqpoint{-0.055556in}{0.000000in}}{\pgfqpoint{0.000000in}{0.000000in}}{%
\pgfpathmoveto{\pgfqpoint{0.000000in}{0.000000in}}%
\pgfpathlineto{\pgfqpoint{-0.055556in}{0.000000in}}%
\pgfusepath{stroke,fill}%
}%
\begin{pgfscope}%
\pgfsys@transformshift{3.221779in}{3.237871in}%
\pgfsys@useobject{currentmarker}{}%
\end{pgfscope}%
\end{pgfscope}%
\begin{pgfscope}%
\pgftext[x=0.843001in,y=3.237871in,right,]{\fontsize{11.000000}{13.200000}\selectfont 60}%
\end{pgfscope}%
\begin{pgfscope}%
\pgfsetbuttcap%
\pgfsetroundjoin%
\definecolor{currentfill}{rgb}{0.000000,0.000000,0.000000}%
\pgfsetfillcolor{currentfill}%
\pgfsetlinewidth{0.501875pt}%
\definecolor{currentstroke}{rgb}{0.000000,0.000000,0.000000}%
\pgfsetstrokecolor{currentstroke}%
\pgfsetdash{}{0pt}%
\pgfsys@defobject{currentmarker}{\pgfqpoint{0.000000in}{0.000000in}}{\pgfqpoint{0.055556in}{0.000000in}}{%
\pgfpathmoveto{\pgfqpoint{0.000000in}{0.000000in}}%
\pgfpathlineto{\pgfqpoint{0.055556in}{0.000000in}}%
\pgfusepath{stroke,fill}%
}%
\begin{pgfscope}%
\pgfsys@transformshift{0.898557in}{3.533342in}%
\pgfsys@useobject{currentmarker}{}%
\end{pgfscope}%
\end{pgfscope}%
\begin{pgfscope}%
\pgfsetbuttcap%
\pgfsetroundjoin%
\definecolor{currentfill}{rgb}{0.000000,0.000000,0.000000}%
\pgfsetfillcolor{currentfill}%
\pgfsetlinewidth{0.501875pt}%
\definecolor{currentstroke}{rgb}{0.000000,0.000000,0.000000}%
\pgfsetstrokecolor{currentstroke}%
\pgfsetdash{}{0pt}%
\pgfsys@defobject{currentmarker}{\pgfqpoint{-0.055556in}{0.000000in}}{\pgfqpoint{0.000000in}{0.000000in}}{%
\pgfpathmoveto{\pgfqpoint{0.000000in}{0.000000in}}%
\pgfpathlineto{\pgfqpoint{-0.055556in}{0.000000in}}%
\pgfusepath{stroke,fill}%
}%
\begin{pgfscope}%
\pgfsys@transformshift{3.221779in}{3.533342in}%
\pgfsys@useobject{currentmarker}{}%
\end{pgfscope}%
\end{pgfscope}%
\begin{pgfscope}%
\pgftext[x=0.843001in,y=3.533342in,right,]{\fontsize{11.000000}{13.200000}\selectfont 80}%
\end{pgfscope}%
\begin{pgfscope}%
\pgfsetbuttcap%
\pgfsetroundjoin%
\definecolor{currentfill}{rgb}{0.000000,0.000000,0.000000}%
\pgfsetfillcolor{currentfill}%
\pgfsetlinewidth{0.501875pt}%
\definecolor{currentstroke}{rgb}{0.000000,0.000000,0.000000}%
\pgfsetstrokecolor{currentstroke}%
\pgfsetdash{}{0pt}%
\pgfsys@defobject{currentmarker}{\pgfqpoint{0.000000in}{0.000000in}}{\pgfqpoint{0.055556in}{0.000000in}}{%
\pgfpathmoveto{\pgfqpoint{0.000000in}{0.000000in}}%
\pgfpathlineto{\pgfqpoint{0.055556in}{0.000000in}}%
\pgfusepath{stroke,fill}%
}%
\begin{pgfscope}%
\pgfsys@transformshift{0.898557in}{3.828813in}%
\pgfsys@useobject{currentmarker}{}%
\end{pgfscope}%
\end{pgfscope}%
\begin{pgfscope}%
\pgfsetbuttcap%
\pgfsetroundjoin%
\definecolor{currentfill}{rgb}{0.000000,0.000000,0.000000}%
\pgfsetfillcolor{currentfill}%
\pgfsetlinewidth{0.501875pt}%
\definecolor{currentstroke}{rgb}{0.000000,0.000000,0.000000}%
\pgfsetstrokecolor{currentstroke}%
\pgfsetdash{}{0pt}%
\pgfsys@defobject{currentmarker}{\pgfqpoint{-0.055556in}{0.000000in}}{\pgfqpoint{0.000000in}{0.000000in}}{%
\pgfpathmoveto{\pgfqpoint{0.000000in}{0.000000in}}%
\pgfpathlineto{\pgfqpoint{-0.055556in}{0.000000in}}%
\pgfusepath{stroke,fill}%
}%
\begin{pgfscope}%
\pgfsys@transformshift{3.221779in}{3.828813in}%
\pgfsys@useobject{currentmarker}{}%
\end{pgfscope}%
\end{pgfscope}%
\begin{pgfscope}%
\pgftext[x=0.843001in,y=3.828813in,right,]{\fontsize{11.000000}{13.200000}\selectfont 100}%
\end{pgfscope}%
\begin{pgfscope}%
\pgftext[x=0.313186in,y=2.857693in,left,base,rotate=90.000000]{\fontsize{11.000000}{13.200000}\selectfont Emittance}%
\end{pgfscope}%
\begin{pgfscope}%
\pgftext[x=0.479633in,y=2.900784in,left,base,rotate=90.000000]{\fontsize{11.000000}{13.200000}\selectfont (nm rad)}%
\end{pgfscope}%
\begin{pgfscope}%
\pgfsetbuttcap%
\pgfsetmiterjoin%
\definecolor{currentfill}{rgb}{1.000000,1.000000,1.000000}%
\pgfsetfillcolor{currentfill}%
\pgfsetlinewidth{0.000000pt}%
\definecolor{currentstroke}{rgb}{0.000000,0.000000,0.000000}%
\pgfsetstrokecolor{currentstroke}%
\pgfsetstrokeopacity{0.000000}%
\pgfsetdash{}{0pt}%
\pgfpathmoveto{\pgfqpoint{0.898557in}{0.652500in}}%
\pgfpathlineto{\pgfqpoint{3.221779in}{0.652500in}}%
\pgfpathlineto{\pgfqpoint{3.221779in}{2.351458in}}%
\pgfpathlineto{\pgfqpoint{0.898557in}{2.351458in}}%
\pgfpathclose%
\pgfusepath{fill}%
\end{pgfscope}%
\begin{pgfscope}%
\pgfpathrectangle{\pgfqpoint{0.898557in}{0.652500in}}{\pgfqpoint{2.323221in}{1.698958in}} %
\pgfusepath{clip}%
\pgfsetrectcap%
\pgfsetroundjoin%
\pgfsetlinewidth{1.003750pt}%
\definecolor{currentstroke}{rgb}{0.301961,0.607843,0.301961}%
\pgfsetstrokecolor{currentstroke}%
\pgfsetdash{}{0pt}%
\pgfpathmoveto{\pgfqpoint{1.195471in}{1.915476in}}%
\pgfpathlineto{\pgfqpoint{1.200680in}{1.864746in}}%
\pgfpathlineto{\pgfqpoint{1.205889in}{1.769927in}}%
\pgfpathlineto{\pgfqpoint{1.216307in}{1.498876in}}%
\pgfpathlineto{\pgfqpoint{1.226725in}{1.217063in}}%
\pgfpathlineto{\pgfqpoint{1.237143in}{1.006889in}}%
\pgfpathlineto{\pgfqpoint{1.242352in}{0.936090in}}%
\pgfpathlineto{\pgfqpoint{1.247561in}{0.885323in}}%
\pgfpathlineto{\pgfqpoint{1.252770in}{0.850553in}}%
\pgfpathlineto{\pgfqpoint{1.257979in}{0.827635in}}%
\pgfpathlineto{\pgfqpoint{1.263188in}{0.812974in}}%
\pgfpathlineto{\pgfqpoint{1.268397in}{0.803770in}}%
\pgfpathlineto{\pgfqpoint{1.273606in}{0.797992in}}%
\pgfpathlineto{\pgfqpoint{1.284024in}{0.791585in}}%
\pgfpathlineto{\pgfqpoint{1.310069in}{0.782045in}}%
\pgfpathlineto{\pgfqpoint{1.325696in}{0.778516in}}%
\pgfpathlineto{\pgfqpoint{1.367369in}{0.773866in}}%
\pgfpathlineto{\pgfqpoint{1.382996in}{0.768066in}}%
\pgfpathlineto{\pgfqpoint{1.403832in}{0.759307in}}%
\pgfpathlineto{\pgfqpoint{1.419459in}{0.756031in}}%
\pgfpathlineto{\pgfqpoint{1.450713in}{0.751130in}}%
\pgfpathlineto{\pgfqpoint{1.492385in}{0.740955in}}%
\pgfpathlineto{\pgfqpoint{1.523639in}{0.734750in}}%
\pgfpathlineto{\pgfqpoint{1.560102in}{0.725782in}}%
\pgfpathlineto{\pgfqpoint{1.633028in}{0.716869in}}%
\pgfpathlineto{\pgfqpoint{1.659073in}{0.713610in}}%
\pgfpathlineto{\pgfqpoint{1.685119in}{0.709250in}}%
\pgfpathlineto{\pgfqpoint{1.711164in}{0.708396in}}%
\pgfpathlineto{\pgfqpoint{1.747627in}{0.705777in}}%
\pgfpathlineto{\pgfqpoint{1.773672in}{0.703523in}}%
\pgfpathlineto{\pgfqpoint{1.825762in}{0.697009in}}%
\pgfpathlineto{\pgfqpoint{1.888270in}{0.692520in}}%
\pgfpathlineto{\pgfqpoint{1.919524in}{0.691235in}}%
\pgfpathlineto{\pgfqpoint{1.950778in}{0.690534in}}%
\pgfpathlineto{\pgfqpoint{2.013287in}{0.686934in}}%
\pgfpathlineto{\pgfqpoint{2.341455in}{0.679890in}}%
\pgfpathlineto{\pgfqpoint{2.388336in}{0.679547in}}%
\pgfpathlineto{\pgfqpoint{2.482098in}{0.678915in}}%
\pgfpathlineto{\pgfqpoint{2.674832in}{0.678087in}}%
\pgfpathlineto{\pgfqpoint{2.862356in}{0.677559in}}%
\pgfpathlineto{\pgfqpoint{2.904028in}{0.678128in}}%
\pgfpathlineto{\pgfqpoint{2.919656in}{0.680024in}}%
\pgfpathlineto{\pgfqpoint{2.930074in}{0.683165in}}%
\pgfpathlineto{\pgfqpoint{2.940492in}{0.689084in}}%
\pgfpathlineto{\pgfqpoint{2.956119in}{0.702911in}}%
\pgfpathlineto{\pgfqpoint{2.966537in}{0.710974in}}%
\pgfpathlineto{\pgfqpoint{2.971746in}{0.712897in}}%
\pgfpathlineto{\pgfqpoint{2.971746in}{0.712897in}}%
\pgfusepath{stroke}%
\end{pgfscope}%
\begin{pgfscope}%
\pgfpathrectangle{\pgfqpoint{0.898557in}{0.652500in}}{\pgfqpoint{2.323221in}{1.698958in}} %
\pgfusepath{clip}%
\pgfsetbuttcap%
\pgfsetroundjoin%
\pgfsetlinewidth{1.003750pt}%
\definecolor{currentstroke}{rgb}{0.000000,0.000000,0.000000}%
\pgfsetstrokecolor{currentstroke}%
\pgfsetdash{{1.000000pt}{3.000000pt}}{0.000000pt}%
\pgfpathmoveto{\pgfqpoint{1.195471in}{0.652500in}}%
\pgfpathlineto{\pgfqpoint{1.195471in}{2.351458in}}%
\pgfusepath{stroke}%
\end{pgfscope}%
\begin{pgfscope}%
\pgfsetrectcap%
\pgfsetmiterjoin%
\pgfsetlinewidth{1.003750pt}%
\definecolor{currentstroke}{rgb}{0.000000,0.000000,0.000000}%
\pgfsetstrokecolor{currentstroke}%
\pgfsetdash{}{0pt}%
\pgfpathmoveto{\pgfqpoint{3.221779in}{0.652500in}}%
\pgfpathlineto{\pgfqpoint{3.221779in}{2.351458in}}%
\pgfusepath{stroke}%
\end{pgfscope}%
\begin{pgfscope}%
\pgfsetrectcap%
\pgfsetmiterjoin%
\pgfsetlinewidth{1.003750pt}%
\definecolor{currentstroke}{rgb}{0.000000,0.000000,0.000000}%
\pgfsetstrokecolor{currentstroke}%
\pgfsetdash{}{0pt}%
\pgfpathmoveto{\pgfqpoint{0.898557in}{0.652500in}}%
\pgfpathlineto{\pgfqpoint{0.898557in}{2.351458in}}%
\pgfusepath{stroke}%
\end{pgfscope}%
\begin{pgfscope}%
\pgfsetrectcap%
\pgfsetmiterjoin%
\pgfsetlinewidth{1.003750pt}%
\definecolor{currentstroke}{rgb}{0.000000,0.000000,0.000000}%
\pgfsetstrokecolor{currentstroke}%
\pgfsetdash{}{0pt}%
\pgfpathmoveto{\pgfqpoint{0.898557in}{2.351458in}}%
\pgfpathlineto{\pgfqpoint{3.221779in}{2.351458in}}%
\pgfusepath{stroke}%
\end{pgfscope}%
\begin{pgfscope}%
\pgfsetrectcap%
\pgfsetmiterjoin%
\pgfsetlinewidth{1.003750pt}%
\definecolor{currentstroke}{rgb}{0.000000,0.000000,0.000000}%
\pgfsetstrokecolor{currentstroke}%
\pgfsetdash{}{0pt}%
\pgfpathmoveto{\pgfqpoint{0.898557in}{0.652500in}}%
\pgfpathlineto{\pgfqpoint{3.221779in}{0.652500in}}%
\pgfusepath{stroke}%
\end{pgfscope}%
\begin{pgfscope}%
\pgfsetbuttcap%
\pgfsetroundjoin%
\definecolor{currentfill}{rgb}{0.000000,0.000000,0.000000}%
\pgfsetfillcolor{currentfill}%
\pgfsetlinewidth{0.501875pt}%
\definecolor{currentstroke}{rgb}{0.000000,0.000000,0.000000}%
\pgfsetstrokecolor{currentstroke}%
\pgfsetdash{}{0pt}%
\pgfsys@defobject{currentmarker}{\pgfqpoint{0.000000in}{0.000000in}}{\pgfqpoint{0.000000in}{0.055556in}}{%
\pgfpathmoveto{\pgfqpoint{0.000000in}{0.000000in}}%
\pgfpathlineto{\pgfqpoint{0.000000in}{0.055556in}}%
\pgfusepath{stroke,fill}%
}%
\begin{pgfscope}%
\pgfsys@transformshift{1.195471in}{0.652500in}%
\pgfsys@useobject{currentmarker}{}%
\end{pgfscope}%
\end{pgfscope}%
\begin{pgfscope}%
\pgfsetbuttcap%
\pgfsetroundjoin%
\definecolor{currentfill}{rgb}{0.000000,0.000000,0.000000}%
\pgfsetfillcolor{currentfill}%
\pgfsetlinewidth{0.501875pt}%
\definecolor{currentstroke}{rgb}{0.000000,0.000000,0.000000}%
\pgfsetstrokecolor{currentstroke}%
\pgfsetdash{}{0pt}%
\pgfsys@defobject{currentmarker}{\pgfqpoint{0.000000in}{-0.055556in}}{\pgfqpoint{0.000000in}{0.000000in}}{%
\pgfpathmoveto{\pgfqpoint{0.000000in}{0.000000in}}%
\pgfpathlineto{\pgfqpoint{0.000000in}{-0.055556in}}%
\pgfusepath{stroke,fill}%
}%
\begin{pgfscope}%
\pgfsys@transformshift{1.195471in}{2.351458in}%
\pgfsys@useobject{currentmarker}{}%
\end{pgfscope}%
\end{pgfscope}%
\begin{pgfscope}%
\pgftext[x=1.195471in,y=0.596944in,,top]{\fontsize{11.000000}{13.200000}\selectfont 0}%
\end{pgfscope}%
\begin{pgfscope}%
\pgfsetbuttcap%
\pgfsetroundjoin%
\definecolor{currentfill}{rgb}{0.000000,0.000000,0.000000}%
\pgfsetfillcolor{currentfill}%
\pgfsetlinewidth{0.501875pt}%
\definecolor{currentstroke}{rgb}{0.000000,0.000000,0.000000}%
\pgfsetstrokecolor{currentstroke}%
\pgfsetdash{}{0pt}%
\pgfsys@defobject{currentmarker}{\pgfqpoint{0.000000in}{0.000000in}}{\pgfqpoint{0.000000in}{0.055556in}}{%
\pgfpathmoveto{\pgfqpoint{0.000000in}{0.000000in}}%
\pgfpathlineto{\pgfqpoint{0.000000in}{0.055556in}}%
\pgfusepath{stroke,fill}%
}%
\begin{pgfscope}%
\pgfsys@transformshift{1.820601in}{0.652500in}%
\pgfsys@useobject{currentmarker}{}%
\end{pgfscope}%
\end{pgfscope}%
\begin{pgfscope}%
\pgfsetbuttcap%
\pgfsetroundjoin%
\definecolor{currentfill}{rgb}{0.000000,0.000000,0.000000}%
\pgfsetfillcolor{currentfill}%
\pgfsetlinewidth{0.501875pt}%
\definecolor{currentstroke}{rgb}{0.000000,0.000000,0.000000}%
\pgfsetstrokecolor{currentstroke}%
\pgfsetdash{}{0pt}%
\pgfsys@defobject{currentmarker}{\pgfqpoint{0.000000in}{-0.055556in}}{\pgfqpoint{0.000000in}{0.000000in}}{%
\pgfpathmoveto{\pgfqpoint{0.000000in}{0.000000in}}%
\pgfpathlineto{\pgfqpoint{0.000000in}{-0.055556in}}%
\pgfusepath{stroke,fill}%
}%
\begin{pgfscope}%
\pgfsys@transformshift{1.820601in}{2.351458in}%
\pgfsys@useobject{currentmarker}{}%
\end{pgfscope}%
\end{pgfscope}%
\begin{pgfscope}%
\pgftext[x=1.820601in,y=0.596944in,,top]{\fontsize{11.000000}{13.200000}\selectfont 5}%
\end{pgfscope}%
\begin{pgfscope}%
\pgfsetbuttcap%
\pgfsetroundjoin%
\definecolor{currentfill}{rgb}{0.000000,0.000000,0.000000}%
\pgfsetfillcolor{currentfill}%
\pgfsetlinewidth{0.501875pt}%
\definecolor{currentstroke}{rgb}{0.000000,0.000000,0.000000}%
\pgfsetstrokecolor{currentstroke}%
\pgfsetdash{}{0pt}%
\pgfsys@defobject{currentmarker}{\pgfqpoint{0.000000in}{0.000000in}}{\pgfqpoint{0.000000in}{0.055556in}}{%
\pgfpathmoveto{\pgfqpoint{0.000000in}{0.000000in}}%
\pgfpathlineto{\pgfqpoint{0.000000in}{0.055556in}}%
\pgfusepath{stroke,fill}%
}%
\begin{pgfscope}%
\pgfsys@transformshift{2.445730in}{0.652500in}%
\pgfsys@useobject{currentmarker}{}%
\end{pgfscope}%
\end{pgfscope}%
\begin{pgfscope}%
\pgfsetbuttcap%
\pgfsetroundjoin%
\definecolor{currentfill}{rgb}{0.000000,0.000000,0.000000}%
\pgfsetfillcolor{currentfill}%
\pgfsetlinewidth{0.501875pt}%
\definecolor{currentstroke}{rgb}{0.000000,0.000000,0.000000}%
\pgfsetstrokecolor{currentstroke}%
\pgfsetdash{}{0pt}%
\pgfsys@defobject{currentmarker}{\pgfqpoint{0.000000in}{-0.055556in}}{\pgfqpoint{0.000000in}{0.000000in}}{%
\pgfpathmoveto{\pgfqpoint{0.000000in}{0.000000in}}%
\pgfpathlineto{\pgfqpoint{0.000000in}{-0.055556in}}%
\pgfusepath{stroke,fill}%
}%
\begin{pgfscope}%
\pgfsys@transformshift{2.445730in}{2.351458in}%
\pgfsys@useobject{currentmarker}{}%
\end{pgfscope}%
\end{pgfscope}%
\begin{pgfscope}%
\pgftext[x=2.445730in,y=0.596944in,,top]{\fontsize{11.000000}{13.200000}\selectfont 10}%
\end{pgfscope}%
\begin{pgfscope}%
\pgfsetbuttcap%
\pgfsetroundjoin%
\definecolor{currentfill}{rgb}{0.000000,0.000000,0.000000}%
\pgfsetfillcolor{currentfill}%
\pgfsetlinewidth{0.501875pt}%
\definecolor{currentstroke}{rgb}{0.000000,0.000000,0.000000}%
\pgfsetstrokecolor{currentstroke}%
\pgfsetdash{}{0pt}%
\pgfsys@defobject{currentmarker}{\pgfqpoint{0.000000in}{0.000000in}}{\pgfqpoint{0.000000in}{0.055556in}}{%
\pgfpathmoveto{\pgfqpoint{0.000000in}{0.000000in}}%
\pgfpathlineto{\pgfqpoint{0.000000in}{0.055556in}}%
\pgfusepath{stroke,fill}%
}%
\begin{pgfscope}%
\pgfsys@transformshift{3.070860in}{0.652500in}%
\pgfsys@useobject{currentmarker}{}%
\end{pgfscope}%
\end{pgfscope}%
\begin{pgfscope}%
\pgfsetbuttcap%
\pgfsetroundjoin%
\definecolor{currentfill}{rgb}{0.000000,0.000000,0.000000}%
\pgfsetfillcolor{currentfill}%
\pgfsetlinewidth{0.501875pt}%
\definecolor{currentstroke}{rgb}{0.000000,0.000000,0.000000}%
\pgfsetstrokecolor{currentstroke}%
\pgfsetdash{}{0pt}%
\pgfsys@defobject{currentmarker}{\pgfqpoint{0.000000in}{-0.055556in}}{\pgfqpoint{0.000000in}{0.000000in}}{%
\pgfpathmoveto{\pgfqpoint{0.000000in}{0.000000in}}%
\pgfpathlineto{\pgfqpoint{0.000000in}{-0.055556in}}%
\pgfusepath{stroke,fill}%
}%
\begin{pgfscope}%
\pgfsys@transformshift{3.070860in}{2.351458in}%
\pgfsys@useobject{currentmarker}{}%
\end{pgfscope}%
\end{pgfscope}%
\begin{pgfscope}%
\pgftext[x=3.070860in,y=0.596944in,,top]{\fontsize{11.000000}{13.200000}\selectfont 15}%
\end{pgfscope}%
\begin{pgfscope}%
\pgftext[x=2.060168in,y=0.392315in,,top]{\fontsize{11.000000}{13.200000}\selectfont Time (\(\displaystyle \muup\)s)}%
\end{pgfscope}%
\begin{pgfscope}%
\pgfsetbuttcap%
\pgfsetroundjoin%
\definecolor{currentfill}{rgb}{0.000000,0.000000,0.000000}%
\pgfsetfillcolor{currentfill}%
\pgfsetlinewidth{0.501875pt}%
\definecolor{currentstroke}{rgb}{0.000000,0.000000,0.000000}%
\pgfsetstrokecolor{currentstroke}%
\pgfsetdash{}{0pt}%
\pgfsys@defobject{currentmarker}{\pgfqpoint{0.000000in}{0.000000in}}{\pgfqpoint{0.055556in}{0.000000in}}{%
\pgfpathmoveto{\pgfqpoint{0.000000in}{0.000000in}}%
\pgfpathlineto{\pgfqpoint{0.055556in}{0.000000in}}%
\pgfusepath{stroke,fill}%
}%
\begin{pgfscope}%
\pgfsys@transformshift{0.898557in}{0.673915in}%
\pgfsys@useobject{currentmarker}{}%
\end{pgfscope}%
\end{pgfscope}%
\begin{pgfscope}%
\pgfsetbuttcap%
\pgfsetroundjoin%
\definecolor{currentfill}{rgb}{0.000000,0.000000,0.000000}%
\pgfsetfillcolor{currentfill}%
\pgfsetlinewidth{0.501875pt}%
\definecolor{currentstroke}{rgb}{0.000000,0.000000,0.000000}%
\pgfsetstrokecolor{currentstroke}%
\pgfsetdash{}{0pt}%
\pgfsys@defobject{currentmarker}{\pgfqpoint{-0.055556in}{0.000000in}}{\pgfqpoint{0.000000in}{0.000000in}}{%
\pgfpathmoveto{\pgfqpoint{0.000000in}{0.000000in}}%
\pgfpathlineto{\pgfqpoint{-0.055556in}{0.000000in}}%
\pgfusepath{stroke,fill}%
}%
\begin{pgfscope}%
\pgfsys@transformshift{3.221779in}{0.673915in}%
\pgfsys@useobject{currentmarker}{}%
\end{pgfscope}%
\end{pgfscope}%
\begin{pgfscope}%
\pgftext[x=0.843001in,y=0.673915in,right,]{\fontsize{11.000000}{13.200000}\selectfont 0}%
\end{pgfscope}%
\begin{pgfscope}%
\pgfsetbuttcap%
\pgfsetroundjoin%
\definecolor{currentfill}{rgb}{0.000000,0.000000,0.000000}%
\pgfsetfillcolor{currentfill}%
\pgfsetlinewidth{0.501875pt}%
\definecolor{currentstroke}{rgb}{0.000000,0.000000,0.000000}%
\pgfsetstrokecolor{currentstroke}%
\pgfsetdash{}{0pt}%
\pgfsys@defobject{currentmarker}{\pgfqpoint{0.000000in}{0.000000in}}{\pgfqpoint{0.055556in}{0.000000in}}{%
\pgfpathmoveto{\pgfqpoint{0.000000in}{0.000000in}}%
\pgfpathlineto{\pgfqpoint{0.055556in}{0.000000in}}%
\pgfusepath{stroke,fill}%
}%
\begin{pgfscope}%
\pgfsys@transformshift{0.898557in}{1.030839in}%
\pgfsys@useobject{currentmarker}{}%
\end{pgfscope}%
\end{pgfscope}%
\begin{pgfscope}%
\pgfsetbuttcap%
\pgfsetroundjoin%
\definecolor{currentfill}{rgb}{0.000000,0.000000,0.000000}%
\pgfsetfillcolor{currentfill}%
\pgfsetlinewidth{0.501875pt}%
\definecolor{currentstroke}{rgb}{0.000000,0.000000,0.000000}%
\pgfsetstrokecolor{currentstroke}%
\pgfsetdash{}{0pt}%
\pgfsys@defobject{currentmarker}{\pgfqpoint{-0.055556in}{0.000000in}}{\pgfqpoint{0.000000in}{0.000000in}}{%
\pgfpathmoveto{\pgfqpoint{0.000000in}{0.000000in}}%
\pgfpathlineto{\pgfqpoint{-0.055556in}{0.000000in}}%
\pgfusepath{stroke,fill}%
}%
\begin{pgfscope}%
\pgfsys@transformshift{3.221779in}{1.030839in}%
\pgfsys@useobject{currentmarker}{}%
\end{pgfscope}%
\end{pgfscope}%
\begin{pgfscope}%
\pgftext[x=0.843001in,y=1.030839in,right,]{\fontsize{11.000000}{13.200000}\selectfont 50}%
\end{pgfscope}%
\begin{pgfscope}%
\pgfsetbuttcap%
\pgfsetroundjoin%
\definecolor{currentfill}{rgb}{0.000000,0.000000,0.000000}%
\pgfsetfillcolor{currentfill}%
\pgfsetlinewidth{0.501875pt}%
\definecolor{currentstroke}{rgb}{0.000000,0.000000,0.000000}%
\pgfsetstrokecolor{currentstroke}%
\pgfsetdash{}{0pt}%
\pgfsys@defobject{currentmarker}{\pgfqpoint{0.000000in}{0.000000in}}{\pgfqpoint{0.055556in}{0.000000in}}{%
\pgfpathmoveto{\pgfqpoint{0.000000in}{0.000000in}}%
\pgfpathlineto{\pgfqpoint{0.055556in}{0.000000in}}%
\pgfusepath{stroke,fill}%
}%
\begin{pgfscope}%
\pgfsys@transformshift{0.898557in}{1.387763in}%
\pgfsys@useobject{currentmarker}{}%
\end{pgfscope}%
\end{pgfscope}%
\begin{pgfscope}%
\pgfsetbuttcap%
\pgfsetroundjoin%
\definecolor{currentfill}{rgb}{0.000000,0.000000,0.000000}%
\pgfsetfillcolor{currentfill}%
\pgfsetlinewidth{0.501875pt}%
\definecolor{currentstroke}{rgb}{0.000000,0.000000,0.000000}%
\pgfsetstrokecolor{currentstroke}%
\pgfsetdash{}{0pt}%
\pgfsys@defobject{currentmarker}{\pgfqpoint{-0.055556in}{0.000000in}}{\pgfqpoint{0.000000in}{0.000000in}}{%
\pgfpathmoveto{\pgfqpoint{0.000000in}{0.000000in}}%
\pgfpathlineto{\pgfqpoint{-0.055556in}{0.000000in}}%
\pgfusepath{stroke,fill}%
}%
\begin{pgfscope}%
\pgfsys@transformshift{3.221779in}{1.387763in}%
\pgfsys@useobject{currentmarker}{}%
\end{pgfscope}%
\end{pgfscope}%
\begin{pgfscope}%
\pgftext[x=0.843001in,y=1.387763in,right,]{\fontsize{11.000000}{13.200000}\selectfont 100}%
\end{pgfscope}%
\begin{pgfscope}%
\pgfsetbuttcap%
\pgfsetroundjoin%
\definecolor{currentfill}{rgb}{0.000000,0.000000,0.000000}%
\pgfsetfillcolor{currentfill}%
\pgfsetlinewidth{0.501875pt}%
\definecolor{currentstroke}{rgb}{0.000000,0.000000,0.000000}%
\pgfsetstrokecolor{currentstroke}%
\pgfsetdash{}{0pt}%
\pgfsys@defobject{currentmarker}{\pgfqpoint{0.000000in}{0.000000in}}{\pgfqpoint{0.055556in}{0.000000in}}{%
\pgfpathmoveto{\pgfqpoint{0.000000in}{0.000000in}}%
\pgfpathlineto{\pgfqpoint{0.055556in}{0.000000in}}%
\pgfusepath{stroke,fill}%
}%
\begin{pgfscope}%
\pgfsys@transformshift{0.898557in}{1.744687in}%
\pgfsys@useobject{currentmarker}{}%
\end{pgfscope}%
\end{pgfscope}%
\begin{pgfscope}%
\pgfsetbuttcap%
\pgfsetroundjoin%
\definecolor{currentfill}{rgb}{0.000000,0.000000,0.000000}%
\pgfsetfillcolor{currentfill}%
\pgfsetlinewidth{0.501875pt}%
\definecolor{currentstroke}{rgb}{0.000000,0.000000,0.000000}%
\pgfsetstrokecolor{currentstroke}%
\pgfsetdash{}{0pt}%
\pgfsys@defobject{currentmarker}{\pgfqpoint{-0.055556in}{0.000000in}}{\pgfqpoint{0.000000in}{0.000000in}}{%
\pgfpathmoveto{\pgfqpoint{0.000000in}{0.000000in}}%
\pgfpathlineto{\pgfqpoint{-0.055556in}{0.000000in}}%
\pgfusepath{stroke,fill}%
}%
\begin{pgfscope}%
\pgfsys@transformshift{3.221779in}{1.744687in}%
\pgfsys@useobject{currentmarker}{}%
\end{pgfscope}%
\end{pgfscope}%
\begin{pgfscope}%
\pgftext[x=0.843001in,y=1.744687in,right,]{\fontsize{11.000000}{13.200000}\selectfont 150}%
\end{pgfscope}%
\begin{pgfscope}%
\pgfsetbuttcap%
\pgfsetroundjoin%
\definecolor{currentfill}{rgb}{0.000000,0.000000,0.000000}%
\pgfsetfillcolor{currentfill}%
\pgfsetlinewidth{0.501875pt}%
\definecolor{currentstroke}{rgb}{0.000000,0.000000,0.000000}%
\pgfsetstrokecolor{currentstroke}%
\pgfsetdash{}{0pt}%
\pgfsys@defobject{currentmarker}{\pgfqpoint{0.000000in}{0.000000in}}{\pgfqpoint{0.055556in}{0.000000in}}{%
\pgfpathmoveto{\pgfqpoint{0.000000in}{0.000000in}}%
\pgfpathlineto{\pgfqpoint{0.055556in}{0.000000in}}%
\pgfusepath{stroke,fill}%
}%
\begin{pgfscope}%
\pgfsys@transformshift{0.898557in}{2.101612in}%
\pgfsys@useobject{currentmarker}{}%
\end{pgfscope}%
\end{pgfscope}%
\begin{pgfscope}%
\pgfsetbuttcap%
\pgfsetroundjoin%
\definecolor{currentfill}{rgb}{0.000000,0.000000,0.000000}%
\pgfsetfillcolor{currentfill}%
\pgfsetlinewidth{0.501875pt}%
\definecolor{currentstroke}{rgb}{0.000000,0.000000,0.000000}%
\pgfsetstrokecolor{currentstroke}%
\pgfsetdash{}{0pt}%
\pgfsys@defobject{currentmarker}{\pgfqpoint{-0.055556in}{0.000000in}}{\pgfqpoint{0.000000in}{0.000000in}}{%
\pgfpathmoveto{\pgfqpoint{0.000000in}{0.000000in}}%
\pgfpathlineto{\pgfqpoint{-0.055556in}{0.000000in}}%
\pgfusepath{stroke,fill}%
}%
\begin{pgfscope}%
\pgfsys@transformshift{3.221779in}{2.101612in}%
\pgfsys@useobject{currentmarker}{}%
\end{pgfscope}%
\end{pgfscope}%
\begin{pgfscope}%
\pgftext[x=0.843001in,y=2.101612in,right,]{\fontsize{11.000000}{13.200000}\selectfont 200}%
\end{pgfscope}%
\begin{pgfscope}%
\pgftext[x=0.316560in,y=1.153454in,left,base,rotate=90.000000]{\fontsize{11.000000}{13.200000}\selectfont Brightness}%
\end{pgfscope}%
\begin{pgfscope}%
\pgftext[x=0.507411in,y=0.963764in,left,base,rotate=90.000000]{\fontsize{11.000000}{13.200000}\selectfont (kA m\(\displaystyle ^{-2}\) rad\(\displaystyle ^{-2}\))}%
\end{pgfscope}%
\begin{pgfscope}%
\pgfsetbuttcap%
\pgfsetmiterjoin%
\definecolor{currentfill}{rgb}{1.000000,1.000000,1.000000}%
\pgfsetfillcolor{currentfill}%
\pgfsetlinewidth{0.000000pt}%
\definecolor{currentstroke}{rgb}{0.000000,0.000000,0.000000}%
\pgfsetstrokecolor{currentstroke}%
\pgfsetstrokeopacity{0.000000}%
\pgfsetdash{}{0pt}%
\pgfpathmoveto{\pgfqpoint{3.221779in}{5.749375in}}%
\pgfpathlineto{\pgfqpoint{5.545000in}{5.749375in}}%
\pgfpathlineto{\pgfqpoint{5.545000in}{7.448333in}}%
\pgfpathlineto{\pgfqpoint{3.221779in}{7.448333in}}%
\pgfpathclose%
\pgfusepath{fill}%
\end{pgfscope}%
\begin{pgfscope}%
\pgfpathrectangle{\pgfqpoint{3.221779in}{5.749375in}}{\pgfqpoint{2.323221in}{1.698958in}} %
\pgfusepath{clip}%
\pgftext[at=\pgfqpoint{3.221779in}{5.749375in},left,bottom]{\pgfimage[interpolate=true,width=2.330000in,height=1.710000in]{streaks-img1.png}}%
\end{pgfscope}%
\begin{pgfscope}%
\pgfpathrectangle{\pgfqpoint{3.221779in}{5.749375in}}{\pgfqpoint{2.323221in}{1.698958in}} %
\pgfusepath{clip}%
\pgfsetrectcap%
\pgfsetroundjoin%
\pgfsetlinewidth{0.501875pt}%
\definecolor{currentstroke}{rgb}{1.000000,1.000000,1.000000}%
\pgfsetstrokecolor{currentstroke}%
\pgfsetdash{}{0pt}%
\pgfpathmoveto{\pgfqpoint{4.798047in}{5.809197in}}%
\pgfpathlineto{\pgfqpoint{5.493878in}{5.809197in}}%
\pgfusepath{stroke}%
\end{pgfscope}%
\begin{pgfscope}%
\pgfpathrectangle{\pgfqpoint{3.221779in}{5.749375in}}{\pgfqpoint{2.323221in}{1.698958in}} %
\pgfusepath{clip}%
\pgfsetrectcap%
\pgfsetroundjoin%
\pgfsetlinewidth{0.501875pt}%
\definecolor{currentstroke}{rgb}{1.000000,1.000000,1.000000}%
\pgfsetstrokecolor{currentstroke}%
\pgfsetdash{}{0pt}%
\pgfpathmoveto{\pgfqpoint{4.798047in}{5.833126in}}%
\pgfpathlineto{\pgfqpoint{4.798047in}{5.785268in}}%
\pgfusepath{stroke}%
\end{pgfscope}%
\begin{pgfscope}%
\pgfpathrectangle{\pgfqpoint{3.221779in}{5.749375in}}{\pgfqpoint{2.323221in}{1.698958in}} %
\pgfusepath{clip}%
\pgfsetrectcap%
\pgfsetroundjoin%
\pgfsetlinewidth{0.501875pt}%
\definecolor{currentstroke}{rgb}{1.000000,1.000000,1.000000}%
\pgfsetstrokecolor{currentstroke}%
\pgfsetdash{}{0pt}%
\pgfpathmoveto{\pgfqpoint{5.493878in}{5.833126in}}%
\pgfpathlineto{\pgfqpoint{5.493878in}{5.785268in}}%
\pgfusepath{stroke}%
\end{pgfscope}%
\begin{pgfscope}%
\pgfsetrectcap%
\pgfsetmiterjoin%
\pgfsetlinewidth{1.003750pt}%
\definecolor{currentstroke}{rgb}{0.000000,0.000000,0.000000}%
\pgfsetstrokecolor{currentstroke}%
\pgfsetdash{}{0pt}%
\pgfpathmoveto{\pgfqpoint{5.545000in}{5.749375in}}%
\pgfpathlineto{\pgfqpoint{5.545000in}{7.448333in}}%
\pgfusepath{stroke}%
\end{pgfscope}%
\begin{pgfscope}%
\pgfsetrectcap%
\pgfsetmiterjoin%
\pgfsetlinewidth{1.003750pt}%
\definecolor{currentstroke}{rgb}{0.000000,0.000000,0.000000}%
\pgfsetstrokecolor{currentstroke}%
\pgfsetdash{}{0pt}%
\pgfpathmoveto{\pgfqpoint{3.221779in}{5.749375in}}%
\pgfpathlineto{\pgfqpoint{3.221779in}{7.448333in}}%
\pgfusepath{stroke}%
\end{pgfscope}%
\begin{pgfscope}%
\pgfsetrectcap%
\pgfsetmiterjoin%
\pgfsetlinewidth{1.003750pt}%
\definecolor{currentstroke}{rgb}{0.000000,0.000000,0.000000}%
\pgfsetstrokecolor{currentstroke}%
\pgfsetdash{}{0pt}%
\pgfpathmoveto{\pgfqpoint{3.221779in}{7.448333in}}%
\pgfpathlineto{\pgfqpoint{5.545000in}{7.448333in}}%
\pgfusepath{stroke}%
\end{pgfscope}%
\begin{pgfscope}%
\pgfsetrectcap%
\pgfsetmiterjoin%
\pgfsetlinewidth{1.003750pt}%
\definecolor{currentstroke}{rgb}{0.000000,0.000000,0.000000}%
\pgfsetstrokecolor{currentstroke}%
\pgfsetdash{}{0pt}%
\pgfpathmoveto{\pgfqpoint{3.221779in}{5.749375in}}%
\pgfpathlineto{\pgfqpoint{5.545000in}{5.749375in}}%
\pgfusepath{stroke}%
\end{pgfscope}%
\begin{pgfscope}%
\definecolor{textcolor}{rgb}{1.000000,1.000000,1.000000}%
\pgfsetstrokecolor{textcolor}%
\pgfsetfillcolor{textcolor}%
\pgftext[x=5.145963in,y=5.833126in,,bottom]{\color{textcolor}\fontsize{6.000000}{7.200000}\selectfont 5mm}%
\end{pgfscope}%
\begin{pgfscope}%
\pgfsetbuttcap%
\pgfsetmiterjoin%
\definecolor{currentfill}{rgb}{1.000000,1.000000,1.000000}%
\pgfsetfillcolor{currentfill}%
\pgfsetlinewidth{0.000000pt}%
\definecolor{currentstroke}{rgb}{0.000000,0.000000,0.000000}%
\pgfsetstrokecolor{currentstroke}%
\pgfsetstrokeopacity{0.000000}%
\pgfsetdash{}{0pt}%
\pgfpathmoveto{\pgfqpoint{3.221779in}{4.050417in}}%
\pgfpathlineto{\pgfqpoint{5.545000in}{4.050417in}}%
\pgfpathlineto{\pgfqpoint{5.545000in}{5.749375in}}%
\pgfpathlineto{\pgfqpoint{3.221779in}{5.749375in}}%
\pgfpathclose%
\pgfusepath{fill}%
\end{pgfscope}%
\begin{pgfscope}%
\pgfpathrectangle{\pgfqpoint{3.221779in}{4.050417in}}{\pgfqpoint{2.323221in}{1.698958in}} %
\pgfusepath{clip}%
\pgfsetbuttcap%
\pgfsetroundjoin%
\definecolor{currentfill}{rgb}{1.000000,0.400000,0.200000}%
\pgfsetfillcolor{currentfill}%
\pgfsetfillopacity{0.500000}%
\pgfsetlinewidth{1.003750pt}%
\definecolor{currentstroke}{rgb}{1.000000,0.400000,0.200000}%
\pgfsetstrokecolor{currentstroke}%
\pgfsetstrokeopacity{0.500000}%
\pgfsetdash{}{0pt}%
\pgfpathmoveto{\pgfqpoint{3.437628in}{4.380086in}}%
\pgfpathlineto{\pgfqpoint{3.437628in}{4.380070in}}%
\pgfpathlineto{\pgfqpoint{3.443308in}{4.385948in}}%
\pgfpathlineto{\pgfqpoint{3.448988in}{4.397235in}}%
\pgfpathlineto{\pgfqpoint{3.454669in}{4.413127in}}%
\pgfpathlineto{\pgfqpoint{3.460349in}{4.432685in}}%
\pgfpathlineto{\pgfqpoint{3.466029in}{4.455038in}}%
\pgfpathlineto{\pgfqpoint{3.471709in}{4.479521in}}%
\pgfpathlineto{\pgfqpoint{3.477390in}{4.505718in}}%
\pgfpathlineto{\pgfqpoint{3.483070in}{4.533440in}}%
\pgfpathlineto{\pgfqpoint{3.488750in}{4.562655in}}%
\pgfpathlineto{\pgfqpoint{3.494430in}{4.593404in}}%
\pgfpathlineto{\pgfqpoint{3.500111in}{4.625732in}}%
\pgfpathlineto{\pgfqpoint{3.505791in}{4.659627in}}%
\pgfpathlineto{\pgfqpoint{3.511471in}{4.694991in}}%
\pgfpathlineto{\pgfqpoint{3.517151in}{4.731633in}}%
\pgfpathlineto{\pgfqpoint{3.522832in}{4.769285in}}%
\pgfpathlineto{\pgfqpoint{3.528512in}{4.807636in}}%
\pgfpathlineto{\pgfqpoint{3.534192in}{4.846370in}}%
\pgfpathlineto{\pgfqpoint{3.539872in}{4.885212in}}%
\pgfpathlineto{\pgfqpoint{3.545553in}{4.923954in}}%
\pgfpathlineto{\pgfqpoint{3.551233in}{4.962480in}}%
\pgfpathlineto{\pgfqpoint{3.556913in}{5.000759in}}%
\pgfpathlineto{\pgfqpoint{3.562593in}{5.038822in}}%
\pgfpathlineto{\pgfqpoint{3.568274in}{5.076733in}}%
\pgfpathlineto{\pgfqpoint{3.573954in}{5.114534in}}%
\pgfpathlineto{\pgfqpoint{3.579634in}{5.152205in}}%
\pgfpathlineto{\pgfqpoint{3.585314in}{5.189617in}}%
\pgfpathlineto{\pgfqpoint{3.590995in}{5.226519in}}%
\pgfpathlineto{\pgfqpoint{3.596675in}{5.262538in}}%
\pgfpathlineto{\pgfqpoint{3.602355in}{5.297204in}}%
\pgfpathlineto{\pgfqpoint{3.608035in}{5.330008in}}%
\pgfpathlineto{\pgfqpoint{3.613716in}{5.360453in}}%
\pgfpathlineto{\pgfqpoint{3.619396in}{5.388120in}}%
\pgfpathlineto{\pgfqpoint{3.625076in}{5.412706in}}%
\pgfpathlineto{\pgfqpoint{3.630756in}{5.434055in}}%
\pgfpathlineto{\pgfqpoint{3.636437in}{5.452156in}}%
\pgfpathlineto{\pgfqpoint{3.642117in}{5.467131in}}%
\pgfpathlineto{\pgfqpoint{3.647797in}{5.479199in}}%
\pgfpathlineto{\pgfqpoint{3.653477in}{5.488645in}}%
\pgfpathlineto{\pgfqpoint{3.659158in}{5.495781in}}%
\pgfpathlineto{\pgfqpoint{3.664838in}{5.500917in}}%
\pgfpathlineto{\pgfqpoint{3.670518in}{5.504341in}}%
\pgfpathlineto{\pgfqpoint{3.676198in}{5.506310in}}%
\pgfpathlineto{\pgfqpoint{3.681879in}{5.507052in}}%
\pgfpathlineto{\pgfqpoint{3.687559in}{5.506778in}}%
\pgfpathlineto{\pgfqpoint{3.693239in}{5.505702in}}%
\pgfpathlineto{\pgfqpoint{3.698919in}{5.504049in}}%
\pgfpathlineto{\pgfqpoint{3.704600in}{5.502073in}}%
\pgfpathlineto{\pgfqpoint{3.710280in}{5.500045in}}%
\pgfpathlineto{\pgfqpoint{3.715960in}{5.498229in}}%
\pgfpathlineto{\pgfqpoint{3.721640in}{5.496837in}}%
\pgfpathlineto{\pgfqpoint{3.727321in}{5.495969in}}%
\pgfpathlineto{\pgfqpoint{3.733001in}{5.495570in}}%
\pgfpathlineto{\pgfqpoint{3.738681in}{5.495393in}}%
\pgfpathlineto{\pgfqpoint{3.744361in}{5.495006in}}%
\pgfpathlineto{\pgfqpoint{3.750042in}{5.493842in}}%
\pgfpathlineto{\pgfqpoint{3.755722in}{5.491278in}}%
\pgfpathlineto{\pgfqpoint{3.761402in}{5.486746in}}%
\pgfpathlineto{\pgfqpoint{3.767082in}{5.479833in}}%
\pgfpathlineto{\pgfqpoint{3.772763in}{5.470361in}}%
\pgfpathlineto{\pgfqpoint{3.778443in}{5.458426in}}%
\pgfpathlineto{\pgfqpoint{3.784123in}{5.444377in}}%
\pgfpathlineto{\pgfqpoint{3.789803in}{5.428758in}}%
\pgfpathlineto{\pgfqpoint{3.795484in}{5.412219in}}%
\pgfpathlineto{\pgfqpoint{3.801164in}{5.395408in}}%
\pgfpathlineto{\pgfqpoint{3.806844in}{5.378886in}}%
\pgfpathlineto{\pgfqpoint{3.812524in}{5.363059in}}%
\pgfpathlineto{\pgfqpoint{3.818205in}{5.348141in}}%
\pgfpathlineto{\pgfqpoint{3.823885in}{5.334153in}}%
\pgfpathlineto{\pgfqpoint{3.829565in}{5.320943in}}%
\pgfpathlineto{\pgfqpoint{3.835245in}{5.308235in}}%
\pgfpathlineto{\pgfqpoint{3.840926in}{5.295676in}}%
\pgfpathlineto{\pgfqpoint{3.846606in}{5.282903in}}%
\pgfpathlineto{\pgfqpoint{3.852286in}{5.269604in}}%
\pgfpathlineto{\pgfqpoint{3.857966in}{5.255562in}}%
\pgfpathlineto{\pgfqpoint{3.863647in}{5.240700in}}%
\pgfpathlineto{\pgfqpoint{3.869327in}{5.225089in}}%
\pgfpathlineto{\pgfqpoint{3.875007in}{5.208935in}}%
\pgfpathlineto{\pgfqpoint{3.880687in}{5.192548in}}%
\pgfpathlineto{\pgfqpoint{3.886368in}{5.176285in}}%
\pgfpathlineto{\pgfqpoint{3.892048in}{5.160494in}}%
\pgfpathlineto{\pgfqpoint{3.897728in}{5.145452in}}%
\pgfpathlineto{\pgfqpoint{3.903408in}{5.131324in}}%
\pgfpathlineto{\pgfqpoint{3.909089in}{5.118142in}}%
\pgfpathlineto{\pgfqpoint{3.914769in}{5.105811in}}%
\pgfpathlineto{\pgfqpoint{3.920449in}{5.094135in}}%
\pgfpathlineto{\pgfqpoint{3.926129in}{5.082863in}}%
\pgfpathlineto{\pgfqpoint{3.931810in}{5.071743in}}%
\pgfpathlineto{\pgfqpoint{3.937490in}{5.060567in}}%
\pgfpathlineto{\pgfqpoint{3.943170in}{5.049202in}}%
\pgfpathlineto{\pgfqpoint{3.948850in}{5.037602in}}%
\pgfpathlineto{\pgfqpoint{3.954531in}{5.025794in}}%
\pgfpathlineto{\pgfqpoint{3.960211in}{5.013853in}}%
\pgfpathlineto{\pgfqpoint{3.965891in}{5.001870in}}%
\pgfpathlineto{\pgfqpoint{3.971571in}{4.989925in}}%
\pgfpathlineto{\pgfqpoint{3.977252in}{4.978068in}}%
\pgfpathlineto{\pgfqpoint{3.982932in}{4.966320in}}%
\pgfpathlineto{\pgfqpoint{3.988612in}{4.954678in}}%
\pgfpathlineto{\pgfqpoint{3.994292in}{4.943130in}}%
\pgfpathlineto{\pgfqpoint{3.999973in}{4.931665in}}%
\pgfpathlineto{\pgfqpoint{4.005653in}{4.920286in}}%
\pgfpathlineto{\pgfqpoint{4.011333in}{4.909008in}}%
\pgfpathlineto{\pgfqpoint{4.017013in}{4.897848in}}%
\pgfpathlineto{\pgfqpoint{4.022694in}{4.886820in}}%
\pgfpathlineto{\pgfqpoint{4.028374in}{4.875921in}}%
\pgfpathlineto{\pgfqpoint{4.034054in}{4.865133in}}%
\pgfpathlineto{\pgfqpoint{4.039734in}{4.854433in}}%
\pgfpathlineto{\pgfqpoint{4.045414in}{4.843809in}}%
\pgfpathlineto{\pgfqpoint{4.051095in}{4.833278in}}%
\pgfpathlineto{\pgfqpoint{4.056775in}{4.822895in}}%
\pgfpathlineto{\pgfqpoint{4.062455in}{4.812748in}}%
\pgfpathlineto{\pgfqpoint{4.068135in}{4.802930in}}%
\pgfpathlineto{\pgfqpoint{4.073816in}{4.793511in}}%
\pgfpathlineto{\pgfqpoint{4.079496in}{4.784502in}}%
\pgfpathlineto{\pgfqpoint{4.085176in}{4.775836in}}%
\pgfpathlineto{\pgfqpoint{4.090856in}{4.767384in}}%
\pgfpathlineto{\pgfqpoint{4.096537in}{4.758980in}}%
\pgfpathlineto{\pgfqpoint{4.102217in}{4.750474in}}%
\pgfpathlineto{\pgfqpoint{4.107897in}{4.741778in}}%
\pgfpathlineto{\pgfqpoint{4.113577in}{4.732898in}}%
\pgfpathlineto{\pgfqpoint{4.119258in}{4.723932in}}%
\pgfpathlineto{\pgfqpoint{4.124938in}{4.715040in}}%
\pgfpathlineto{\pgfqpoint{4.130618in}{4.706403in}}%
\pgfpathlineto{\pgfqpoint{4.136298in}{4.698185in}}%
\pgfpathlineto{\pgfqpoint{4.141979in}{4.690516in}}%
\pgfpathlineto{\pgfqpoint{4.147659in}{4.683485in}}%
\pgfpathlineto{\pgfqpoint{4.153339in}{4.677136in}}%
\pgfpathlineto{\pgfqpoint{4.159019in}{4.671449in}}%
\pgfpathlineto{\pgfqpoint{4.164700in}{4.666323in}}%
\pgfpathlineto{\pgfqpoint{4.170380in}{4.661576in}}%
\pgfpathlineto{\pgfqpoint{4.176060in}{4.656980in}}%
\pgfpathlineto{\pgfqpoint{4.181740in}{4.652318in}}%
\pgfpathlineto{\pgfqpoint{4.187421in}{4.647439in}}%
\pgfpathlineto{\pgfqpoint{4.193101in}{4.642292in}}%
\pgfpathlineto{\pgfqpoint{4.198781in}{4.636915in}}%
\pgfpathlineto{\pgfqpoint{4.204461in}{4.631405in}}%
\pgfpathlineto{\pgfqpoint{4.210142in}{4.625873in}}%
\pgfpathlineto{\pgfqpoint{4.215822in}{4.620416in}}%
\pgfpathlineto{\pgfqpoint{4.221502in}{4.615090in}}%
\pgfpathlineto{\pgfqpoint{4.227182in}{4.609918in}}%
\pgfpathlineto{\pgfqpoint{4.232863in}{4.604887in}}%
\pgfpathlineto{\pgfqpoint{4.238543in}{4.599967in}}%
\pgfpathlineto{\pgfqpoint{4.244223in}{4.595126in}}%
\pgfpathlineto{\pgfqpoint{4.249903in}{4.590339in}}%
\pgfpathlineto{\pgfqpoint{4.255584in}{4.585594in}}%
\pgfpathlineto{\pgfqpoint{4.261264in}{4.580886in}}%
\pgfpathlineto{\pgfqpoint{4.266944in}{4.576210in}}%
\pgfpathlineto{\pgfqpoint{4.272624in}{4.571545in}}%
\pgfpathlineto{\pgfqpoint{4.278305in}{4.566852in}}%
\pgfpathlineto{\pgfqpoint{4.283985in}{4.562075in}}%
\pgfpathlineto{\pgfqpoint{4.289665in}{4.557150in}}%
\pgfpathlineto{\pgfqpoint{4.295345in}{4.552019in}}%
\pgfpathlineto{\pgfqpoint{4.301026in}{4.546645in}}%
\pgfpathlineto{\pgfqpoint{4.306706in}{4.541013in}}%
\pgfpathlineto{\pgfqpoint{4.312386in}{4.535133in}}%
\pgfpathlineto{\pgfqpoint{4.318066in}{4.529036in}}%
\pgfpathlineto{\pgfqpoint{4.323747in}{4.522763in}}%
\pgfpathlineto{\pgfqpoint{4.329427in}{4.516360in}}%
\pgfpathlineto{\pgfqpoint{4.335107in}{4.509877in}}%
\pgfpathlineto{\pgfqpoint{4.340787in}{4.503359in}}%
\pgfpathlineto{\pgfqpoint{4.346468in}{4.496849in}}%
\pgfpathlineto{\pgfqpoint{4.352148in}{4.490389in}}%
\pgfpathlineto{\pgfqpoint{4.357828in}{4.484020in}}%
\pgfpathlineto{\pgfqpoint{4.363508in}{4.477794in}}%
\pgfpathlineto{\pgfqpoint{4.369189in}{4.471776in}}%
\pgfpathlineto{\pgfqpoint{4.374869in}{4.466055in}}%
\pgfpathlineto{\pgfqpoint{4.380549in}{4.460737in}}%
\pgfpathlineto{\pgfqpoint{4.386229in}{4.455939in}}%
\pgfpathlineto{\pgfqpoint{4.391910in}{4.451766in}}%
\pgfpathlineto{\pgfqpoint{4.397590in}{4.448285in}}%
\pgfpathlineto{\pgfqpoint{4.403270in}{4.445507in}}%
\pgfpathlineto{\pgfqpoint{4.408950in}{4.443380in}}%
\pgfpathlineto{\pgfqpoint{4.414631in}{4.441796in}}%
\pgfpathlineto{\pgfqpoint{4.420311in}{4.440611in}}%
\pgfpathlineto{\pgfqpoint{4.425991in}{4.439668in}}%
\pgfpathlineto{\pgfqpoint{4.431671in}{4.438823in}}%
\pgfpathlineto{\pgfqpoint{4.437352in}{4.437952in}}%
\pgfpathlineto{\pgfqpoint{4.443032in}{4.436948in}}%
\pgfpathlineto{\pgfqpoint{4.448712in}{4.435717in}}%
\pgfpathlineto{\pgfqpoint{4.454392in}{4.434166in}}%
\pgfpathlineto{\pgfqpoint{4.460073in}{4.432199in}}%
\pgfpathlineto{\pgfqpoint{4.465753in}{4.429726in}}%
\pgfpathlineto{\pgfqpoint{4.471433in}{4.426678in}}%
\pgfpathlineto{\pgfqpoint{4.477113in}{4.423017in}}%
\pgfpathlineto{\pgfqpoint{4.482794in}{4.418751in}}%
\pgfpathlineto{\pgfqpoint{4.488474in}{4.413937in}}%
\pgfpathlineto{\pgfqpoint{4.494154in}{4.408670in}}%
\pgfpathlineto{\pgfqpoint{4.499834in}{4.403070in}}%
\pgfpathlineto{\pgfqpoint{4.505515in}{4.397261in}}%
\pgfpathlineto{\pgfqpoint{4.511195in}{4.391352in}}%
\pgfpathlineto{\pgfqpoint{4.516875in}{4.385429in}}%
\pgfpathlineto{\pgfqpoint{4.522555in}{4.379550in}}%
\pgfpathlineto{\pgfqpoint{4.528236in}{4.373762in}}%
\pgfpathlineto{\pgfqpoint{4.533916in}{4.368109in}}%
\pgfpathlineto{\pgfqpoint{4.539596in}{4.362650in}}%
\pgfpathlineto{\pgfqpoint{4.545276in}{4.357462in}}%
\pgfpathlineto{\pgfqpoint{4.550957in}{4.352632in}}%
\pgfpathlineto{\pgfqpoint{4.556637in}{4.348234in}}%
\pgfpathlineto{\pgfqpoint{4.562317in}{4.344309in}}%
\pgfpathlineto{\pgfqpoint{4.567997in}{4.340843in}}%
\pgfpathlineto{\pgfqpoint{4.573678in}{4.337757in}}%
\pgfpathlineto{\pgfqpoint{4.579358in}{4.334918in}}%
\pgfpathlineto{\pgfqpoint{4.585038in}{4.332158in}}%
\pgfpathlineto{\pgfqpoint{4.590718in}{4.329308in}}%
\pgfpathlineto{\pgfqpoint{4.596399in}{4.326233in}}%
\pgfpathlineto{\pgfqpoint{4.602079in}{4.322847in}}%
\pgfpathlineto{\pgfqpoint{4.607759in}{4.319135in}}%
\pgfpathlineto{\pgfqpoint{4.613439in}{4.315144in}}%
\pgfpathlineto{\pgfqpoint{4.619120in}{4.310971in}}%
\pgfpathlineto{\pgfqpoint{4.624800in}{4.306742in}}%
\pgfpathlineto{\pgfqpoint{4.630480in}{4.302588in}}%
\pgfpathlineto{\pgfqpoint{4.636160in}{4.298622in}}%
\pgfpathlineto{\pgfqpoint{4.641841in}{4.294924in}}%
\pgfpathlineto{\pgfqpoint{4.647521in}{4.291527in}}%
\pgfpathlineto{\pgfqpoint{4.653201in}{4.288417in}}%
\pgfpathlineto{\pgfqpoint{4.658881in}{4.285536in}}%
\pgfpathlineto{\pgfqpoint{4.664562in}{4.282799in}}%
\pgfpathlineto{\pgfqpoint{4.670242in}{4.280113in}}%
\pgfpathlineto{\pgfqpoint{4.675922in}{4.277398in}}%
\pgfpathlineto{\pgfqpoint{4.681602in}{4.274600in}}%
\pgfpathlineto{\pgfqpoint{4.687283in}{4.271701in}}%
\pgfpathlineto{\pgfqpoint{4.692963in}{4.268716in}}%
\pgfpathlineto{\pgfqpoint{4.698643in}{4.265681in}}%
\pgfpathlineto{\pgfqpoint{4.704323in}{4.262645in}}%
\pgfpathlineto{\pgfqpoint{4.710004in}{4.259659in}}%
\pgfpathlineto{\pgfqpoint{4.715684in}{4.256769in}}%
\pgfpathlineto{\pgfqpoint{4.721364in}{4.254020in}}%
\pgfpathlineto{\pgfqpoint{4.727044in}{4.251450in}}%
\pgfpathlineto{\pgfqpoint{4.732725in}{4.249096in}}%
\pgfpathlineto{\pgfqpoint{4.738405in}{4.246989in}}%
\pgfpathlineto{\pgfqpoint{4.744085in}{4.245142in}}%
\pgfpathlineto{\pgfqpoint{4.749765in}{4.243547in}}%
\pgfpathlineto{\pgfqpoint{4.755446in}{4.242165in}}%
\pgfpathlineto{\pgfqpoint{4.761126in}{4.240931in}}%
\pgfpathlineto{\pgfqpoint{4.766806in}{4.239751in}}%
\pgfpathlineto{\pgfqpoint{4.772486in}{4.238526in}}%
\pgfpathlineto{\pgfqpoint{4.778167in}{4.237159in}}%
\pgfpathlineto{\pgfqpoint{4.783847in}{4.235574in}}%
\pgfpathlineto{\pgfqpoint{4.789527in}{4.233724in}}%
\pgfpathlineto{\pgfqpoint{4.795207in}{4.231586in}}%
\pgfpathlineto{\pgfqpoint{4.800887in}{4.229163in}}%
\pgfpathlineto{\pgfqpoint{4.806568in}{4.226467in}}%
\pgfpathlineto{\pgfqpoint{4.812248in}{4.223519in}}%
\pgfpathlineto{\pgfqpoint{4.817928in}{4.220344in}}%
\pgfpathlineto{\pgfqpoint{4.823608in}{4.216971in}}%
\pgfpathlineto{\pgfqpoint{4.829289in}{4.213446in}}%
\pgfpathlineto{\pgfqpoint{4.834969in}{4.209829in}}%
\pgfpathlineto{\pgfqpoint{4.840649in}{4.206194in}}%
\pgfpathlineto{\pgfqpoint{4.846329in}{4.202621in}}%
\pgfpathlineto{\pgfqpoint{4.852010in}{4.199185in}}%
\pgfpathlineto{\pgfqpoint{4.857690in}{4.195946in}}%
\pgfpathlineto{\pgfqpoint{4.863370in}{4.192933in}}%
\pgfpathlineto{\pgfqpoint{4.869050in}{4.190147in}}%
\pgfpathlineto{\pgfqpoint{4.874731in}{4.187565in}}%
\pgfpathlineto{\pgfqpoint{4.880411in}{4.185144in}}%
\pgfpathlineto{\pgfqpoint{4.886091in}{4.182838in}}%
\pgfpathlineto{\pgfqpoint{4.891771in}{4.180608in}}%
\pgfpathlineto{\pgfqpoint{4.897452in}{4.178430in}}%
\pgfpathlineto{\pgfqpoint{4.903132in}{4.176298in}}%
\pgfpathlineto{\pgfqpoint{4.908812in}{4.174221in}}%
\pgfpathlineto{\pgfqpoint{4.914492in}{4.172222in}}%
\pgfpathlineto{\pgfqpoint{4.920173in}{4.170327in}}%
\pgfpathlineto{\pgfqpoint{4.925853in}{4.168561in}}%
\pgfpathlineto{\pgfqpoint{4.931533in}{4.166941in}}%
\pgfpathlineto{\pgfqpoint{4.937213in}{4.165475in}}%
\pgfpathlineto{\pgfqpoint{4.942894in}{4.164161in}}%
\pgfpathlineto{\pgfqpoint{4.948574in}{4.162986in}}%
\pgfpathlineto{\pgfqpoint{4.954254in}{4.161933in}}%
\pgfpathlineto{\pgfqpoint{4.959934in}{4.160981in}}%
\pgfpathlineto{\pgfqpoint{4.965615in}{4.160120in}}%
\pgfpathlineto{\pgfqpoint{4.971295in}{4.159353in}}%
\pgfpathlineto{\pgfqpoint{4.976975in}{4.158708in}}%
\pgfpathlineto{\pgfqpoint{4.982655in}{4.158237in}}%
\pgfpathlineto{\pgfqpoint{4.988336in}{4.158004in}}%
\pgfpathlineto{\pgfqpoint{4.994016in}{4.158070in}}%
\pgfpathlineto{\pgfqpoint{4.999696in}{4.158468in}}%
\pgfpathlineto{\pgfqpoint{5.005376in}{4.159186in}}%
\pgfpathlineto{\pgfqpoint{5.011057in}{4.160165in}}%
\pgfpathlineto{\pgfqpoint{5.016737in}{4.161306in}}%
\pgfpathlineto{\pgfqpoint{5.022417in}{4.162495in}}%
\pgfpathlineto{\pgfqpoint{5.028097in}{4.163624in}}%
\pgfpathlineto{\pgfqpoint{5.033778in}{4.164612in}}%
\pgfpathlineto{\pgfqpoint{5.039458in}{4.165421in}}%
\pgfpathlineto{\pgfqpoint{5.045138in}{4.166061in}}%
\pgfpathlineto{\pgfqpoint{5.050818in}{4.166573in}}%
\pgfpathlineto{\pgfqpoint{5.056499in}{4.167013in}}%
\pgfpathlineto{\pgfqpoint{5.062179in}{4.167415in}}%
\pgfpathlineto{\pgfqpoint{5.067859in}{4.167782in}}%
\pgfpathlineto{\pgfqpoint{5.073539in}{4.168075in}}%
\pgfpathlineto{\pgfqpoint{5.079220in}{4.168240in}}%
\pgfpathlineto{\pgfqpoint{5.079220in}{4.168252in}}%
\pgfpathlineto{\pgfqpoint{5.079220in}{4.168252in}}%
\pgfpathlineto{\pgfqpoint{5.073539in}{4.168087in}}%
\pgfpathlineto{\pgfqpoint{5.067859in}{4.167794in}}%
\pgfpathlineto{\pgfqpoint{5.062179in}{4.167429in}}%
\pgfpathlineto{\pgfqpoint{5.056499in}{4.167028in}}%
\pgfpathlineto{\pgfqpoint{5.050818in}{4.166591in}}%
\pgfpathlineto{\pgfqpoint{5.045138in}{4.166081in}}%
\pgfpathlineto{\pgfqpoint{5.039458in}{4.165444in}}%
\pgfpathlineto{\pgfqpoint{5.033778in}{4.164637in}}%
\pgfpathlineto{\pgfqpoint{5.028097in}{4.163651in}}%
\pgfpathlineto{\pgfqpoint{5.022417in}{4.162524in}}%
\pgfpathlineto{\pgfqpoint{5.016737in}{4.161335in}}%
\pgfpathlineto{\pgfqpoint{5.011057in}{4.160193in}}%
\pgfpathlineto{\pgfqpoint{5.005376in}{4.159213in}}%
\pgfpathlineto{\pgfqpoint{4.999696in}{4.158492in}}%
\pgfpathlineto{\pgfqpoint{4.994016in}{4.158091in}}%
\pgfpathlineto{\pgfqpoint{4.988336in}{4.158021in}}%
\pgfpathlineto{\pgfqpoint{4.982655in}{4.158251in}}%
\pgfpathlineto{\pgfqpoint{4.976975in}{4.158720in}}%
\pgfpathlineto{\pgfqpoint{4.971295in}{4.159363in}}%
\pgfpathlineto{\pgfqpoint{4.965615in}{4.160128in}}%
\pgfpathlineto{\pgfqpoint{4.959934in}{4.160989in}}%
\pgfpathlineto{\pgfqpoint{4.954254in}{4.161940in}}%
\pgfpathlineto{\pgfqpoint{4.948574in}{4.162994in}}%
\pgfpathlineto{\pgfqpoint{4.942894in}{4.164169in}}%
\pgfpathlineto{\pgfqpoint{4.937213in}{4.165483in}}%
\pgfpathlineto{\pgfqpoint{4.931533in}{4.166949in}}%
\pgfpathlineto{\pgfqpoint{4.925853in}{4.168569in}}%
\pgfpathlineto{\pgfqpoint{4.920173in}{4.170336in}}%
\pgfpathlineto{\pgfqpoint{4.914492in}{4.172231in}}%
\pgfpathlineto{\pgfqpoint{4.908812in}{4.174230in}}%
\pgfpathlineto{\pgfqpoint{4.903132in}{4.176307in}}%
\pgfpathlineto{\pgfqpoint{4.897452in}{4.178439in}}%
\pgfpathlineto{\pgfqpoint{4.891771in}{4.180617in}}%
\pgfpathlineto{\pgfqpoint{4.886091in}{4.182847in}}%
\pgfpathlineto{\pgfqpoint{4.880411in}{4.185152in}}%
\pgfpathlineto{\pgfqpoint{4.874731in}{4.187572in}}%
\pgfpathlineto{\pgfqpoint{4.869050in}{4.190153in}}%
\pgfpathlineto{\pgfqpoint{4.863370in}{4.192937in}}%
\pgfpathlineto{\pgfqpoint{4.857690in}{4.195949in}}%
\pgfpathlineto{\pgfqpoint{4.852010in}{4.199189in}}%
\pgfpathlineto{\pgfqpoint{4.846329in}{4.202624in}}%
\pgfpathlineto{\pgfqpoint{4.840649in}{4.206197in}}%
\pgfpathlineto{\pgfqpoint{4.834969in}{4.209833in}}%
\pgfpathlineto{\pgfqpoint{4.829289in}{4.213451in}}%
\pgfpathlineto{\pgfqpoint{4.823608in}{4.216976in}}%
\pgfpathlineto{\pgfqpoint{4.817928in}{4.220349in}}%
\pgfpathlineto{\pgfqpoint{4.812248in}{4.223525in}}%
\pgfpathlineto{\pgfqpoint{4.806568in}{4.226473in}}%
\pgfpathlineto{\pgfqpoint{4.800887in}{4.229170in}}%
\pgfpathlineto{\pgfqpoint{4.795207in}{4.231595in}}%
\pgfpathlineto{\pgfqpoint{4.789527in}{4.233734in}}%
\pgfpathlineto{\pgfqpoint{4.783847in}{4.235587in}}%
\pgfpathlineto{\pgfqpoint{4.778167in}{4.237173in}}%
\pgfpathlineto{\pgfqpoint{4.772486in}{4.238541in}}%
\pgfpathlineto{\pgfqpoint{4.766806in}{4.239767in}}%
\pgfpathlineto{\pgfqpoint{4.761126in}{4.240945in}}%
\pgfpathlineto{\pgfqpoint{4.755446in}{4.242179in}}%
\pgfpathlineto{\pgfqpoint{4.749765in}{4.243558in}}%
\pgfpathlineto{\pgfqpoint{4.744085in}{4.245152in}}%
\pgfpathlineto{\pgfqpoint{4.738405in}{4.246997in}}%
\pgfpathlineto{\pgfqpoint{4.732725in}{4.249103in}}%
\pgfpathlineto{\pgfqpoint{4.727044in}{4.251455in}}%
\pgfpathlineto{\pgfqpoint{4.721364in}{4.254024in}}%
\pgfpathlineto{\pgfqpoint{4.715684in}{4.256772in}}%
\pgfpathlineto{\pgfqpoint{4.710004in}{4.259661in}}%
\pgfpathlineto{\pgfqpoint{4.704323in}{4.262647in}}%
\pgfpathlineto{\pgfqpoint{4.698643in}{4.265683in}}%
\pgfpathlineto{\pgfqpoint{4.692963in}{4.268718in}}%
\pgfpathlineto{\pgfqpoint{4.687283in}{4.271703in}}%
\pgfpathlineto{\pgfqpoint{4.681602in}{4.274601in}}%
\pgfpathlineto{\pgfqpoint{4.675922in}{4.277400in}}%
\pgfpathlineto{\pgfqpoint{4.670242in}{4.280116in}}%
\pgfpathlineto{\pgfqpoint{4.664562in}{4.282802in}}%
\pgfpathlineto{\pgfqpoint{4.658881in}{4.285539in}}%
\pgfpathlineto{\pgfqpoint{4.653201in}{4.288420in}}%
\pgfpathlineto{\pgfqpoint{4.647521in}{4.291530in}}%
\pgfpathlineto{\pgfqpoint{4.641841in}{4.294926in}}%
\pgfpathlineto{\pgfqpoint{4.636160in}{4.298624in}}%
\pgfpathlineto{\pgfqpoint{4.630480in}{4.302589in}}%
\pgfpathlineto{\pgfqpoint{4.624800in}{4.306744in}}%
\pgfpathlineto{\pgfqpoint{4.619120in}{4.310973in}}%
\pgfpathlineto{\pgfqpoint{4.613439in}{4.315145in}}%
\pgfpathlineto{\pgfqpoint{4.607759in}{4.319136in}}%
\pgfpathlineto{\pgfqpoint{4.602079in}{4.322848in}}%
\pgfpathlineto{\pgfqpoint{4.596399in}{4.326233in}}%
\pgfpathlineto{\pgfqpoint{4.590718in}{4.329309in}}%
\pgfpathlineto{\pgfqpoint{4.585038in}{4.332158in}}%
\pgfpathlineto{\pgfqpoint{4.579358in}{4.334918in}}%
\pgfpathlineto{\pgfqpoint{4.573678in}{4.337758in}}%
\pgfpathlineto{\pgfqpoint{4.567997in}{4.340844in}}%
\pgfpathlineto{\pgfqpoint{4.562317in}{4.344310in}}%
\pgfpathlineto{\pgfqpoint{4.556637in}{4.348235in}}%
\pgfpathlineto{\pgfqpoint{4.550957in}{4.352633in}}%
\pgfpathlineto{\pgfqpoint{4.545276in}{4.357464in}}%
\pgfpathlineto{\pgfqpoint{4.539596in}{4.362651in}}%
\pgfpathlineto{\pgfqpoint{4.533916in}{4.368110in}}%
\pgfpathlineto{\pgfqpoint{4.528236in}{4.373763in}}%
\pgfpathlineto{\pgfqpoint{4.522555in}{4.379552in}}%
\pgfpathlineto{\pgfqpoint{4.516875in}{4.385430in}}%
\pgfpathlineto{\pgfqpoint{4.511195in}{4.391354in}}%
\pgfpathlineto{\pgfqpoint{4.505515in}{4.397263in}}%
\pgfpathlineto{\pgfqpoint{4.499834in}{4.403071in}}%
\pgfpathlineto{\pgfqpoint{4.494154in}{4.408671in}}%
\pgfpathlineto{\pgfqpoint{4.488474in}{4.413938in}}%
\pgfpathlineto{\pgfqpoint{4.482794in}{4.418752in}}%
\pgfpathlineto{\pgfqpoint{4.477113in}{4.423017in}}%
\pgfpathlineto{\pgfqpoint{4.471433in}{4.426678in}}%
\pgfpathlineto{\pgfqpoint{4.465753in}{4.429726in}}%
\pgfpathlineto{\pgfqpoint{4.460073in}{4.432199in}}%
\pgfpathlineto{\pgfqpoint{4.454392in}{4.434166in}}%
\pgfpathlineto{\pgfqpoint{4.448712in}{4.435718in}}%
\pgfpathlineto{\pgfqpoint{4.443032in}{4.436948in}}%
\pgfpathlineto{\pgfqpoint{4.437352in}{4.437952in}}%
\pgfpathlineto{\pgfqpoint{4.431671in}{4.438823in}}%
\pgfpathlineto{\pgfqpoint{4.425991in}{4.439668in}}%
\pgfpathlineto{\pgfqpoint{4.420311in}{4.440611in}}%
\pgfpathlineto{\pgfqpoint{4.414631in}{4.441797in}}%
\pgfpathlineto{\pgfqpoint{4.408950in}{4.443381in}}%
\pgfpathlineto{\pgfqpoint{4.403270in}{4.445507in}}%
\pgfpathlineto{\pgfqpoint{4.397590in}{4.448285in}}%
\pgfpathlineto{\pgfqpoint{4.391910in}{4.451766in}}%
\pgfpathlineto{\pgfqpoint{4.386229in}{4.455939in}}%
\pgfpathlineto{\pgfqpoint{4.380549in}{4.460737in}}%
\pgfpathlineto{\pgfqpoint{4.374869in}{4.466055in}}%
\pgfpathlineto{\pgfqpoint{4.369189in}{4.471776in}}%
\pgfpathlineto{\pgfqpoint{4.363508in}{4.477794in}}%
\pgfpathlineto{\pgfqpoint{4.357828in}{4.484020in}}%
\pgfpathlineto{\pgfqpoint{4.352148in}{4.490389in}}%
\pgfpathlineto{\pgfqpoint{4.346468in}{4.496849in}}%
\pgfpathlineto{\pgfqpoint{4.340787in}{4.503359in}}%
\pgfpathlineto{\pgfqpoint{4.335107in}{4.509877in}}%
\pgfpathlineto{\pgfqpoint{4.329427in}{4.516361in}}%
\pgfpathlineto{\pgfqpoint{4.323747in}{4.522763in}}%
\pgfpathlineto{\pgfqpoint{4.318066in}{4.529036in}}%
\pgfpathlineto{\pgfqpoint{4.312386in}{4.535133in}}%
\pgfpathlineto{\pgfqpoint{4.306706in}{4.541013in}}%
\pgfpathlineto{\pgfqpoint{4.301026in}{4.546645in}}%
\pgfpathlineto{\pgfqpoint{4.295345in}{4.552020in}}%
\pgfpathlineto{\pgfqpoint{4.289665in}{4.557150in}}%
\pgfpathlineto{\pgfqpoint{4.283985in}{4.562075in}}%
\pgfpathlineto{\pgfqpoint{4.278305in}{4.566852in}}%
\pgfpathlineto{\pgfqpoint{4.272624in}{4.571545in}}%
\pgfpathlineto{\pgfqpoint{4.266944in}{4.576210in}}%
\pgfpathlineto{\pgfqpoint{4.261264in}{4.580886in}}%
\pgfpathlineto{\pgfqpoint{4.255584in}{4.585594in}}%
\pgfpathlineto{\pgfqpoint{4.249903in}{4.590339in}}%
\pgfpathlineto{\pgfqpoint{4.244223in}{4.595126in}}%
\pgfpathlineto{\pgfqpoint{4.238543in}{4.599968in}}%
\pgfpathlineto{\pgfqpoint{4.232863in}{4.604887in}}%
\pgfpathlineto{\pgfqpoint{4.227182in}{4.609918in}}%
\pgfpathlineto{\pgfqpoint{4.221502in}{4.615091in}}%
\pgfpathlineto{\pgfqpoint{4.215822in}{4.620416in}}%
\pgfpathlineto{\pgfqpoint{4.210142in}{4.625874in}}%
\pgfpathlineto{\pgfqpoint{4.204461in}{4.631405in}}%
\pgfpathlineto{\pgfqpoint{4.198781in}{4.636915in}}%
\pgfpathlineto{\pgfqpoint{4.193101in}{4.642292in}}%
\pgfpathlineto{\pgfqpoint{4.187421in}{4.647439in}}%
\pgfpathlineto{\pgfqpoint{4.181740in}{4.652318in}}%
\pgfpathlineto{\pgfqpoint{4.176060in}{4.656980in}}%
\pgfpathlineto{\pgfqpoint{4.170380in}{4.661576in}}%
\pgfpathlineto{\pgfqpoint{4.164700in}{4.666323in}}%
\pgfpathlineto{\pgfqpoint{4.159019in}{4.671449in}}%
\pgfpathlineto{\pgfqpoint{4.153339in}{4.677136in}}%
\pgfpathlineto{\pgfqpoint{4.147659in}{4.683485in}}%
\pgfpathlineto{\pgfqpoint{4.141979in}{4.690516in}}%
\pgfpathlineto{\pgfqpoint{4.136298in}{4.698185in}}%
\pgfpathlineto{\pgfqpoint{4.130618in}{4.706403in}}%
\pgfpathlineto{\pgfqpoint{4.124938in}{4.715040in}}%
\pgfpathlineto{\pgfqpoint{4.119258in}{4.723932in}}%
\pgfpathlineto{\pgfqpoint{4.113577in}{4.732898in}}%
\pgfpathlineto{\pgfqpoint{4.107897in}{4.741778in}}%
\pgfpathlineto{\pgfqpoint{4.102217in}{4.750474in}}%
\pgfpathlineto{\pgfqpoint{4.096537in}{4.758980in}}%
\pgfpathlineto{\pgfqpoint{4.090856in}{4.767384in}}%
\pgfpathlineto{\pgfqpoint{4.085176in}{4.775836in}}%
\pgfpathlineto{\pgfqpoint{4.079496in}{4.784502in}}%
\pgfpathlineto{\pgfqpoint{4.073816in}{4.793512in}}%
\pgfpathlineto{\pgfqpoint{4.068135in}{4.802930in}}%
\pgfpathlineto{\pgfqpoint{4.062455in}{4.812748in}}%
\pgfpathlineto{\pgfqpoint{4.056775in}{4.822895in}}%
\pgfpathlineto{\pgfqpoint{4.051095in}{4.833278in}}%
\pgfpathlineto{\pgfqpoint{4.045414in}{4.843809in}}%
\pgfpathlineto{\pgfqpoint{4.039734in}{4.854434in}}%
\pgfpathlineto{\pgfqpoint{4.034054in}{4.865134in}}%
\pgfpathlineto{\pgfqpoint{4.028374in}{4.875921in}}%
\pgfpathlineto{\pgfqpoint{4.022694in}{4.886820in}}%
\pgfpathlineto{\pgfqpoint{4.017013in}{4.897848in}}%
\pgfpathlineto{\pgfqpoint{4.011333in}{4.909008in}}%
\pgfpathlineto{\pgfqpoint{4.005653in}{4.920286in}}%
\pgfpathlineto{\pgfqpoint{3.999973in}{4.931665in}}%
\pgfpathlineto{\pgfqpoint{3.994292in}{4.943130in}}%
\pgfpathlineto{\pgfqpoint{3.988612in}{4.954679in}}%
\pgfpathlineto{\pgfqpoint{3.982932in}{4.966320in}}%
\pgfpathlineto{\pgfqpoint{3.977252in}{4.978068in}}%
\pgfpathlineto{\pgfqpoint{3.971571in}{4.989925in}}%
\pgfpathlineto{\pgfqpoint{3.965891in}{5.001870in}}%
\pgfpathlineto{\pgfqpoint{3.960211in}{5.013853in}}%
\pgfpathlineto{\pgfqpoint{3.954531in}{5.025794in}}%
\pgfpathlineto{\pgfqpoint{3.948850in}{5.037602in}}%
\pgfpathlineto{\pgfqpoint{3.943170in}{5.049202in}}%
\pgfpathlineto{\pgfqpoint{3.937490in}{5.060567in}}%
\pgfpathlineto{\pgfqpoint{3.931810in}{5.071743in}}%
\pgfpathlineto{\pgfqpoint{3.926129in}{5.082863in}}%
\pgfpathlineto{\pgfqpoint{3.920449in}{5.094135in}}%
\pgfpathlineto{\pgfqpoint{3.914769in}{5.105811in}}%
\pgfpathlineto{\pgfqpoint{3.909089in}{5.118142in}}%
\pgfpathlineto{\pgfqpoint{3.903408in}{5.131324in}}%
\pgfpathlineto{\pgfqpoint{3.897728in}{5.145452in}}%
\pgfpathlineto{\pgfqpoint{3.892048in}{5.160494in}}%
\pgfpathlineto{\pgfqpoint{3.886368in}{5.176285in}}%
\pgfpathlineto{\pgfqpoint{3.880687in}{5.192548in}}%
\pgfpathlineto{\pgfqpoint{3.875007in}{5.208935in}}%
\pgfpathlineto{\pgfqpoint{3.869327in}{5.225089in}}%
\pgfpathlineto{\pgfqpoint{3.863647in}{5.240701in}}%
\pgfpathlineto{\pgfqpoint{3.857966in}{5.255562in}}%
\pgfpathlineto{\pgfqpoint{3.852286in}{5.269604in}}%
\pgfpathlineto{\pgfqpoint{3.846606in}{5.282903in}}%
\pgfpathlineto{\pgfqpoint{3.840926in}{5.295676in}}%
\pgfpathlineto{\pgfqpoint{3.835245in}{5.308235in}}%
\pgfpathlineto{\pgfqpoint{3.829565in}{5.320944in}}%
\pgfpathlineto{\pgfqpoint{3.823885in}{5.334153in}}%
\pgfpathlineto{\pgfqpoint{3.818205in}{5.348142in}}%
\pgfpathlineto{\pgfqpoint{3.812524in}{5.363060in}}%
\pgfpathlineto{\pgfqpoint{3.806844in}{5.378886in}}%
\pgfpathlineto{\pgfqpoint{3.801164in}{5.395408in}}%
\pgfpathlineto{\pgfqpoint{3.795484in}{5.412219in}}%
\pgfpathlineto{\pgfqpoint{3.789803in}{5.428759in}}%
\pgfpathlineto{\pgfqpoint{3.784123in}{5.444377in}}%
\pgfpathlineto{\pgfqpoint{3.778443in}{5.458426in}}%
\pgfpathlineto{\pgfqpoint{3.772763in}{5.470362in}}%
\pgfpathlineto{\pgfqpoint{3.767082in}{5.479833in}}%
\pgfpathlineto{\pgfqpoint{3.761402in}{5.486746in}}%
\pgfpathlineto{\pgfqpoint{3.755722in}{5.491278in}}%
\pgfpathlineto{\pgfqpoint{3.750042in}{5.493842in}}%
\pgfpathlineto{\pgfqpoint{3.744361in}{5.495006in}}%
\pgfpathlineto{\pgfqpoint{3.738681in}{5.495393in}}%
\pgfpathlineto{\pgfqpoint{3.733001in}{5.495571in}}%
\pgfpathlineto{\pgfqpoint{3.727321in}{5.495970in}}%
\pgfpathlineto{\pgfqpoint{3.721640in}{5.496837in}}%
\pgfpathlineto{\pgfqpoint{3.715960in}{5.498230in}}%
\pgfpathlineto{\pgfqpoint{3.710280in}{5.500045in}}%
\pgfpathlineto{\pgfqpoint{3.704600in}{5.502073in}}%
\pgfpathlineto{\pgfqpoint{3.698919in}{5.504049in}}%
\pgfpathlineto{\pgfqpoint{3.693239in}{5.505702in}}%
\pgfpathlineto{\pgfqpoint{3.687559in}{5.506779in}}%
\pgfpathlineto{\pgfqpoint{3.681879in}{5.507053in}}%
\pgfpathlineto{\pgfqpoint{3.676198in}{5.506312in}}%
\pgfpathlineto{\pgfqpoint{3.670518in}{5.504344in}}%
\pgfpathlineto{\pgfqpoint{3.664838in}{5.500921in}}%
\pgfpathlineto{\pgfqpoint{3.659158in}{5.495786in}}%
\pgfpathlineto{\pgfqpoint{3.653477in}{5.488651in}}%
\pgfpathlineto{\pgfqpoint{3.647797in}{5.479206in}}%
\pgfpathlineto{\pgfqpoint{3.642117in}{5.467138in}}%
\pgfpathlineto{\pgfqpoint{3.636437in}{5.452164in}}%
\pgfpathlineto{\pgfqpoint{3.630756in}{5.434063in}}%
\pgfpathlineto{\pgfqpoint{3.625076in}{5.412715in}}%
\pgfpathlineto{\pgfqpoint{3.619396in}{5.388128in}}%
\pgfpathlineto{\pgfqpoint{3.613716in}{5.360461in}}%
\pgfpathlineto{\pgfqpoint{3.608035in}{5.330016in}}%
\pgfpathlineto{\pgfqpoint{3.602355in}{5.297212in}}%
\pgfpathlineto{\pgfqpoint{3.596675in}{5.262546in}}%
\pgfpathlineto{\pgfqpoint{3.590995in}{5.226528in}}%
\pgfpathlineto{\pgfqpoint{3.585314in}{5.189626in}}%
\pgfpathlineto{\pgfqpoint{3.579634in}{5.152215in}}%
\pgfpathlineto{\pgfqpoint{3.573954in}{5.114546in}}%
\pgfpathlineto{\pgfqpoint{3.568274in}{5.076746in}}%
\pgfpathlineto{\pgfqpoint{3.562593in}{5.038838in}}%
\pgfpathlineto{\pgfqpoint{3.556913in}{5.000777in}}%
\pgfpathlineto{\pgfqpoint{3.551233in}{4.962501in}}%
\pgfpathlineto{\pgfqpoint{3.545553in}{4.923976in}}%
\pgfpathlineto{\pgfqpoint{3.539872in}{4.885233in}}%
\pgfpathlineto{\pgfqpoint{3.534192in}{4.846391in}}%
\pgfpathlineto{\pgfqpoint{3.528512in}{4.807655in}}%
\pgfpathlineto{\pgfqpoint{3.522832in}{4.769303in}}%
\pgfpathlineto{\pgfqpoint{3.517151in}{4.731648in}}%
\pgfpathlineto{\pgfqpoint{3.511471in}{4.695005in}}%
\pgfpathlineto{\pgfqpoint{3.505791in}{4.659640in}}%
\pgfpathlineto{\pgfqpoint{3.500111in}{4.625744in}}%
\pgfpathlineto{\pgfqpoint{3.494430in}{4.593415in}}%
\pgfpathlineto{\pgfqpoint{3.488750in}{4.562665in}}%
\pgfpathlineto{\pgfqpoint{3.483070in}{4.533450in}}%
\pgfpathlineto{\pgfqpoint{3.477390in}{4.505729in}}%
\pgfpathlineto{\pgfqpoint{3.471709in}{4.479533in}}%
\pgfpathlineto{\pgfqpoint{3.466029in}{4.455051in}}%
\pgfpathlineto{\pgfqpoint{3.460349in}{4.432699in}}%
\pgfpathlineto{\pgfqpoint{3.454669in}{4.413142in}}%
\pgfpathlineto{\pgfqpoint{3.448988in}{4.397251in}}%
\pgfpathlineto{\pgfqpoint{3.443308in}{4.385964in}}%
\pgfpathlineto{\pgfqpoint{3.437628in}{4.380086in}}%
\pgfpathclose%
\pgfusepath{stroke,fill}%
\end{pgfscope}%
\begin{pgfscope}%
\pgfpathrectangle{\pgfqpoint{3.221779in}{4.050417in}}{\pgfqpoint{2.323221in}{1.698958in}} %
\pgfusepath{clip}%
\pgfsetrectcap%
\pgfsetroundjoin%
\pgfsetlinewidth{1.003750pt}%
\definecolor{currentstroke}{rgb}{1.000000,0.400000,0.200000}%
\pgfsetstrokecolor{currentstroke}%
\pgfsetdash{}{0pt}%
\pgfpathmoveto{\pgfqpoint{3.437628in}{4.380078in}}%
\pgfpathlineto{\pgfqpoint{3.443308in}{4.385956in}}%
\pgfpathlineto{\pgfqpoint{3.448988in}{4.397243in}}%
\pgfpathlineto{\pgfqpoint{3.454669in}{4.413134in}}%
\pgfpathlineto{\pgfqpoint{3.466029in}{4.455045in}}%
\pgfpathlineto{\pgfqpoint{3.477390in}{4.505723in}}%
\pgfpathlineto{\pgfqpoint{3.494430in}{4.593410in}}%
\pgfpathlineto{\pgfqpoint{3.511471in}{4.694998in}}%
\pgfpathlineto{\pgfqpoint{3.539872in}{4.885223in}}%
\pgfpathlineto{\pgfqpoint{3.590995in}{5.226524in}}%
\pgfpathlineto{\pgfqpoint{3.608035in}{5.330012in}}%
\pgfpathlineto{\pgfqpoint{3.619396in}{5.388124in}}%
\pgfpathlineto{\pgfqpoint{3.630756in}{5.434059in}}%
\pgfpathlineto{\pgfqpoint{3.642117in}{5.467135in}}%
\pgfpathlineto{\pgfqpoint{3.647797in}{5.479202in}}%
\pgfpathlineto{\pgfqpoint{3.653477in}{5.488648in}}%
\pgfpathlineto{\pgfqpoint{3.659158in}{5.495784in}}%
\pgfpathlineto{\pgfqpoint{3.664838in}{5.500919in}}%
\pgfpathlineto{\pgfqpoint{3.670518in}{5.504342in}}%
\pgfpathlineto{\pgfqpoint{3.676198in}{5.506311in}}%
\pgfpathlineto{\pgfqpoint{3.687559in}{5.506779in}}%
\pgfpathlineto{\pgfqpoint{3.698919in}{5.504049in}}%
\pgfpathlineto{\pgfqpoint{3.721640in}{5.496837in}}%
\pgfpathlineto{\pgfqpoint{3.733001in}{5.495571in}}%
\pgfpathlineto{\pgfqpoint{3.744361in}{5.495006in}}%
\pgfpathlineto{\pgfqpoint{3.750042in}{5.493842in}}%
\pgfpathlineto{\pgfqpoint{3.755722in}{5.491278in}}%
\pgfpathlineto{\pgfqpoint{3.761402in}{5.486746in}}%
\pgfpathlineto{\pgfqpoint{3.767082in}{5.479833in}}%
\pgfpathlineto{\pgfqpoint{3.772763in}{5.470361in}}%
\pgfpathlineto{\pgfqpoint{3.784123in}{5.444377in}}%
\pgfpathlineto{\pgfqpoint{3.829565in}{5.320943in}}%
\pgfpathlineto{\pgfqpoint{3.857966in}{5.255562in}}%
\pgfpathlineto{\pgfqpoint{3.875007in}{5.208935in}}%
\pgfpathlineto{\pgfqpoint{3.897728in}{5.145452in}}%
\pgfpathlineto{\pgfqpoint{3.914769in}{5.105811in}}%
\pgfpathlineto{\pgfqpoint{4.017013in}{4.897848in}}%
\pgfpathlineto{\pgfqpoint{4.056775in}{4.822895in}}%
\pgfpathlineto{\pgfqpoint{4.079496in}{4.784502in}}%
\pgfpathlineto{\pgfqpoint{4.141979in}{4.690516in}}%
\pgfpathlineto{\pgfqpoint{4.153339in}{4.677136in}}%
\pgfpathlineto{\pgfqpoint{4.170380in}{4.661576in}}%
\pgfpathlineto{\pgfqpoint{4.193101in}{4.642292in}}%
\pgfpathlineto{\pgfqpoint{4.244223in}{4.595126in}}%
\pgfpathlineto{\pgfqpoint{4.301026in}{4.546645in}}%
\pgfpathlineto{\pgfqpoint{4.323747in}{4.522763in}}%
\pgfpathlineto{\pgfqpoint{4.374869in}{4.466055in}}%
\pgfpathlineto{\pgfqpoint{4.386229in}{4.455939in}}%
\pgfpathlineto{\pgfqpoint{4.397590in}{4.448285in}}%
\pgfpathlineto{\pgfqpoint{4.408950in}{4.443380in}}%
\pgfpathlineto{\pgfqpoint{4.425991in}{4.439668in}}%
\pgfpathlineto{\pgfqpoint{4.448712in}{4.435718in}}%
\pgfpathlineto{\pgfqpoint{4.460073in}{4.432199in}}%
\pgfpathlineto{\pgfqpoint{4.471433in}{4.426678in}}%
\pgfpathlineto{\pgfqpoint{4.482794in}{4.418752in}}%
\pgfpathlineto{\pgfqpoint{4.499834in}{4.403071in}}%
\pgfpathlineto{\pgfqpoint{4.545276in}{4.357463in}}%
\pgfpathlineto{\pgfqpoint{4.562317in}{4.344310in}}%
\pgfpathlineto{\pgfqpoint{4.579358in}{4.334918in}}%
\pgfpathlineto{\pgfqpoint{4.602079in}{4.322847in}}%
\pgfpathlineto{\pgfqpoint{4.624800in}{4.306743in}}%
\pgfpathlineto{\pgfqpoint{4.641841in}{4.294925in}}%
\pgfpathlineto{\pgfqpoint{4.658881in}{4.285538in}}%
\pgfpathlineto{\pgfqpoint{4.704323in}{4.262646in}}%
\pgfpathlineto{\pgfqpoint{4.727044in}{4.251452in}}%
\pgfpathlineto{\pgfqpoint{4.744085in}{4.245147in}}%
\pgfpathlineto{\pgfqpoint{4.766806in}{4.239759in}}%
\pgfpathlineto{\pgfqpoint{4.789527in}{4.233729in}}%
\pgfpathlineto{\pgfqpoint{4.806568in}{4.226470in}}%
\pgfpathlineto{\pgfqpoint{4.829289in}{4.213449in}}%
\pgfpathlineto{\pgfqpoint{4.863370in}{4.192935in}}%
\pgfpathlineto{\pgfqpoint{4.886091in}{4.182842in}}%
\pgfpathlineto{\pgfqpoint{4.920173in}{4.170331in}}%
\pgfpathlineto{\pgfqpoint{4.942894in}{4.164165in}}%
\pgfpathlineto{\pgfqpoint{4.971295in}{4.159358in}}%
\pgfpathlineto{\pgfqpoint{4.988336in}{4.158013in}}%
\pgfpathlineto{\pgfqpoint{5.005376in}{4.159200in}}%
\pgfpathlineto{\pgfqpoint{5.056499in}{4.167020in}}%
\pgfpathlineto{\pgfqpoint{5.079220in}{4.168246in}}%
\pgfpathlineto{\pgfqpoint{5.079220in}{4.168246in}}%
\pgfusepath{stroke}%
\end{pgfscope}%
\begin{pgfscope}%
\pgfpathrectangle{\pgfqpoint{3.221779in}{4.050417in}}{\pgfqpoint{2.323221in}{1.698958in}} %
\pgfusepath{clip}%
\pgfsetbuttcap%
\pgfsetroundjoin%
\pgfsetlinewidth{1.003750pt}%
\definecolor{currentstroke}{rgb}{0.000000,0.000000,0.000000}%
\pgfsetstrokecolor{currentstroke}%
\pgfsetdash{{1.000000pt}{3.000000pt}}{0.000000pt}%
\pgfpathmoveto{\pgfqpoint{3.437628in}{4.050417in}}%
\pgfpathlineto{\pgfqpoint{3.437628in}{5.749375in}}%
\pgfusepath{stroke}%
\end{pgfscope}%
\begin{pgfscope}%
\pgfsetrectcap%
\pgfsetmiterjoin%
\pgfsetlinewidth{1.003750pt}%
\definecolor{currentstroke}{rgb}{0.000000,0.000000,0.000000}%
\pgfsetstrokecolor{currentstroke}%
\pgfsetdash{}{0pt}%
\pgfpathmoveto{\pgfqpoint{5.545000in}{4.050417in}}%
\pgfpathlineto{\pgfqpoint{5.545000in}{5.749375in}}%
\pgfusepath{stroke}%
\end{pgfscope}%
\begin{pgfscope}%
\pgfsetrectcap%
\pgfsetmiterjoin%
\pgfsetlinewidth{1.003750pt}%
\definecolor{currentstroke}{rgb}{0.000000,0.000000,0.000000}%
\pgfsetstrokecolor{currentstroke}%
\pgfsetdash{}{0pt}%
\pgfpathmoveto{\pgfqpoint{3.221779in}{4.050417in}}%
\pgfpathlineto{\pgfqpoint{3.221779in}{5.749375in}}%
\pgfusepath{stroke}%
\end{pgfscope}%
\begin{pgfscope}%
\pgfsetrectcap%
\pgfsetmiterjoin%
\pgfsetlinewidth{1.003750pt}%
\definecolor{currentstroke}{rgb}{0.000000,0.000000,0.000000}%
\pgfsetstrokecolor{currentstroke}%
\pgfsetdash{}{0pt}%
\pgfpathmoveto{\pgfqpoint{3.221779in}{5.749375in}}%
\pgfpathlineto{\pgfqpoint{5.545000in}{5.749375in}}%
\pgfusepath{stroke}%
\end{pgfscope}%
\begin{pgfscope}%
\pgfsetrectcap%
\pgfsetmiterjoin%
\pgfsetlinewidth{1.003750pt}%
\definecolor{currentstroke}{rgb}{0.000000,0.000000,0.000000}%
\pgfsetstrokecolor{currentstroke}%
\pgfsetdash{}{0pt}%
\pgfpathmoveto{\pgfqpoint{3.221779in}{4.050417in}}%
\pgfpathlineto{\pgfqpoint{5.545000in}{4.050417in}}%
\pgfusepath{stroke}%
\end{pgfscope}%
\begin{pgfscope}%
\pgfsetbuttcap%
\pgfsetroundjoin%
\definecolor{currentfill}{rgb}{0.000000,0.000000,0.000000}%
\pgfsetfillcolor{currentfill}%
\pgfsetlinewidth{0.501875pt}%
\definecolor{currentstroke}{rgb}{0.000000,0.000000,0.000000}%
\pgfsetstrokecolor{currentstroke}%
\pgfsetdash{}{0pt}%
\pgfsys@defobject{currentmarker}{\pgfqpoint{0.000000in}{0.000000in}}{\pgfqpoint{0.000000in}{0.055556in}}{%
\pgfpathmoveto{\pgfqpoint{0.000000in}{0.000000in}}%
\pgfpathlineto{\pgfqpoint{0.000000in}{0.055556in}}%
\pgfusepath{stroke,fill}%
}%
\begin{pgfscope}%
\pgfsys@transformshift{3.437628in}{4.050417in}%
\pgfsys@useobject{currentmarker}{}%
\end{pgfscope}%
\end{pgfscope}%
\begin{pgfscope}%
\pgfsetbuttcap%
\pgfsetroundjoin%
\definecolor{currentfill}{rgb}{0.000000,0.000000,0.000000}%
\pgfsetfillcolor{currentfill}%
\pgfsetlinewidth{0.501875pt}%
\definecolor{currentstroke}{rgb}{0.000000,0.000000,0.000000}%
\pgfsetstrokecolor{currentstroke}%
\pgfsetdash{}{0pt}%
\pgfsys@defobject{currentmarker}{\pgfqpoint{0.000000in}{-0.055556in}}{\pgfqpoint{0.000000in}{0.000000in}}{%
\pgfpathmoveto{\pgfqpoint{0.000000in}{0.000000in}}%
\pgfpathlineto{\pgfqpoint{0.000000in}{-0.055556in}}%
\pgfusepath{stroke,fill}%
}%
\begin{pgfscope}%
\pgfsys@transformshift{3.437628in}{5.749375in}%
\pgfsys@useobject{currentmarker}{}%
\end{pgfscope}%
\end{pgfscope}%
\begin{pgfscope}%
\pgftext[x=3.437628in,y=3.994861in,,top]{\fontsize{11.000000}{13.200000}\selectfont 0}%
\end{pgfscope}%
\begin{pgfscope}%
\pgfsetbuttcap%
\pgfsetroundjoin%
\definecolor{currentfill}{rgb}{0.000000,0.000000,0.000000}%
\pgfsetfillcolor{currentfill}%
\pgfsetlinewidth{0.501875pt}%
\definecolor{currentstroke}{rgb}{0.000000,0.000000,0.000000}%
\pgfsetstrokecolor{currentstroke}%
\pgfsetdash{}{0pt}%
\pgfsys@defobject{currentmarker}{\pgfqpoint{0.000000in}{0.000000in}}{\pgfqpoint{0.000000in}{0.055556in}}{%
\pgfpathmoveto{\pgfqpoint{0.000000in}{0.000000in}}%
\pgfpathlineto{\pgfqpoint{0.000000in}{0.055556in}}%
\pgfusepath{stroke,fill}%
}%
\begin{pgfscope}%
\pgfsys@transformshift{3.726995in}{4.050417in}%
\pgfsys@useobject{currentmarker}{}%
\end{pgfscope}%
\end{pgfscope}%
\begin{pgfscope}%
\pgfsetbuttcap%
\pgfsetroundjoin%
\definecolor{currentfill}{rgb}{0.000000,0.000000,0.000000}%
\pgfsetfillcolor{currentfill}%
\pgfsetlinewidth{0.501875pt}%
\definecolor{currentstroke}{rgb}{0.000000,0.000000,0.000000}%
\pgfsetstrokecolor{currentstroke}%
\pgfsetdash{}{0pt}%
\pgfsys@defobject{currentmarker}{\pgfqpoint{0.000000in}{-0.055556in}}{\pgfqpoint{0.000000in}{0.000000in}}{%
\pgfpathmoveto{\pgfqpoint{0.000000in}{0.000000in}}%
\pgfpathlineto{\pgfqpoint{0.000000in}{-0.055556in}}%
\pgfusepath{stroke,fill}%
}%
\begin{pgfscope}%
\pgfsys@transformshift{3.726995in}{5.749375in}%
\pgfsys@useobject{currentmarker}{}%
\end{pgfscope}%
\end{pgfscope}%
\begin{pgfscope}%
\pgftext[x=3.726995in,y=3.994861in,,top]{\fontsize{11.000000}{13.200000}\selectfont 1}%
\end{pgfscope}%
\begin{pgfscope}%
\pgfsetbuttcap%
\pgfsetroundjoin%
\definecolor{currentfill}{rgb}{0.000000,0.000000,0.000000}%
\pgfsetfillcolor{currentfill}%
\pgfsetlinewidth{0.501875pt}%
\definecolor{currentstroke}{rgb}{0.000000,0.000000,0.000000}%
\pgfsetstrokecolor{currentstroke}%
\pgfsetdash{}{0pt}%
\pgfsys@defobject{currentmarker}{\pgfqpoint{0.000000in}{0.000000in}}{\pgfqpoint{0.000000in}{0.055556in}}{%
\pgfpathmoveto{\pgfqpoint{0.000000in}{0.000000in}}%
\pgfpathlineto{\pgfqpoint{0.000000in}{0.055556in}}%
\pgfusepath{stroke,fill}%
}%
\begin{pgfscope}%
\pgfsys@transformshift{4.016362in}{4.050417in}%
\pgfsys@useobject{currentmarker}{}%
\end{pgfscope}%
\end{pgfscope}%
\begin{pgfscope}%
\pgfsetbuttcap%
\pgfsetroundjoin%
\definecolor{currentfill}{rgb}{0.000000,0.000000,0.000000}%
\pgfsetfillcolor{currentfill}%
\pgfsetlinewidth{0.501875pt}%
\definecolor{currentstroke}{rgb}{0.000000,0.000000,0.000000}%
\pgfsetstrokecolor{currentstroke}%
\pgfsetdash{}{0pt}%
\pgfsys@defobject{currentmarker}{\pgfqpoint{0.000000in}{-0.055556in}}{\pgfqpoint{0.000000in}{0.000000in}}{%
\pgfpathmoveto{\pgfqpoint{0.000000in}{0.000000in}}%
\pgfpathlineto{\pgfqpoint{0.000000in}{-0.055556in}}%
\pgfusepath{stroke,fill}%
}%
\begin{pgfscope}%
\pgfsys@transformshift{4.016362in}{5.749375in}%
\pgfsys@useobject{currentmarker}{}%
\end{pgfscope}%
\end{pgfscope}%
\begin{pgfscope}%
\pgftext[x=4.016362in,y=3.994861in,,top]{\fontsize{11.000000}{13.200000}\selectfont 2}%
\end{pgfscope}%
\begin{pgfscope}%
\pgfsetbuttcap%
\pgfsetroundjoin%
\definecolor{currentfill}{rgb}{0.000000,0.000000,0.000000}%
\pgfsetfillcolor{currentfill}%
\pgfsetlinewidth{0.501875pt}%
\definecolor{currentstroke}{rgb}{0.000000,0.000000,0.000000}%
\pgfsetstrokecolor{currentstroke}%
\pgfsetdash{}{0pt}%
\pgfsys@defobject{currentmarker}{\pgfqpoint{0.000000in}{0.000000in}}{\pgfqpoint{0.000000in}{0.055556in}}{%
\pgfpathmoveto{\pgfqpoint{0.000000in}{0.000000in}}%
\pgfpathlineto{\pgfqpoint{0.000000in}{0.055556in}}%
\pgfusepath{stroke,fill}%
}%
\begin{pgfscope}%
\pgfsys@transformshift{4.305728in}{4.050417in}%
\pgfsys@useobject{currentmarker}{}%
\end{pgfscope}%
\end{pgfscope}%
\begin{pgfscope}%
\pgfsetbuttcap%
\pgfsetroundjoin%
\definecolor{currentfill}{rgb}{0.000000,0.000000,0.000000}%
\pgfsetfillcolor{currentfill}%
\pgfsetlinewidth{0.501875pt}%
\definecolor{currentstroke}{rgb}{0.000000,0.000000,0.000000}%
\pgfsetstrokecolor{currentstroke}%
\pgfsetdash{}{0pt}%
\pgfsys@defobject{currentmarker}{\pgfqpoint{0.000000in}{-0.055556in}}{\pgfqpoint{0.000000in}{0.000000in}}{%
\pgfpathmoveto{\pgfqpoint{0.000000in}{0.000000in}}%
\pgfpathlineto{\pgfqpoint{0.000000in}{-0.055556in}}%
\pgfusepath{stroke,fill}%
}%
\begin{pgfscope}%
\pgfsys@transformshift{4.305728in}{5.749375in}%
\pgfsys@useobject{currentmarker}{}%
\end{pgfscope}%
\end{pgfscope}%
\begin{pgfscope}%
\pgftext[x=4.305728in,y=3.994861in,,top]{\fontsize{11.000000}{13.200000}\selectfont 3}%
\end{pgfscope}%
\begin{pgfscope}%
\pgfsetbuttcap%
\pgfsetroundjoin%
\definecolor{currentfill}{rgb}{0.000000,0.000000,0.000000}%
\pgfsetfillcolor{currentfill}%
\pgfsetlinewidth{0.501875pt}%
\definecolor{currentstroke}{rgb}{0.000000,0.000000,0.000000}%
\pgfsetstrokecolor{currentstroke}%
\pgfsetdash{}{0pt}%
\pgfsys@defobject{currentmarker}{\pgfqpoint{0.000000in}{0.000000in}}{\pgfqpoint{0.000000in}{0.055556in}}{%
\pgfpathmoveto{\pgfqpoint{0.000000in}{0.000000in}}%
\pgfpathlineto{\pgfqpoint{0.000000in}{0.055556in}}%
\pgfusepath{stroke,fill}%
}%
\begin{pgfscope}%
\pgfsys@transformshift{4.595095in}{4.050417in}%
\pgfsys@useobject{currentmarker}{}%
\end{pgfscope}%
\end{pgfscope}%
\begin{pgfscope}%
\pgfsetbuttcap%
\pgfsetroundjoin%
\definecolor{currentfill}{rgb}{0.000000,0.000000,0.000000}%
\pgfsetfillcolor{currentfill}%
\pgfsetlinewidth{0.501875pt}%
\definecolor{currentstroke}{rgb}{0.000000,0.000000,0.000000}%
\pgfsetstrokecolor{currentstroke}%
\pgfsetdash{}{0pt}%
\pgfsys@defobject{currentmarker}{\pgfqpoint{0.000000in}{-0.055556in}}{\pgfqpoint{0.000000in}{0.000000in}}{%
\pgfpathmoveto{\pgfqpoint{0.000000in}{0.000000in}}%
\pgfpathlineto{\pgfqpoint{0.000000in}{-0.055556in}}%
\pgfusepath{stroke,fill}%
}%
\begin{pgfscope}%
\pgfsys@transformshift{4.595095in}{5.749375in}%
\pgfsys@useobject{currentmarker}{}%
\end{pgfscope}%
\end{pgfscope}%
\begin{pgfscope}%
\pgftext[x=4.595095in,y=3.994861in,,top]{\fontsize{11.000000}{13.200000}\selectfont 4}%
\end{pgfscope}%
\begin{pgfscope}%
\pgfsetbuttcap%
\pgfsetroundjoin%
\definecolor{currentfill}{rgb}{0.000000,0.000000,0.000000}%
\pgfsetfillcolor{currentfill}%
\pgfsetlinewidth{0.501875pt}%
\definecolor{currentstroke}{rgb}{0.000000,0.000000,0.000000}%
\pgfsetstrokecolor{currentstroke}%
\pgfsetdash{}{0pt}%
\pgfsys@defobject{currentmarker}{\pgfqpoint{0.000000in}{0.000000in}}{\pgfqpoint{0.000000in}{0.055556in}}{%
\pgfpathmoveto{\pgfqpoint{0.000000in}{0.000000in}}%
\pgfpathlineto{\pgfqpoint{0.000000in}{0.055556in}}%
\pgfusepath{stroke,fill}%
}%
\begin{pgfscope}%
\pgfsys@transformshift{4.884462in}{4.050417in}%
\pgfsys@useobject{currentmarker}{}%
\end{pgfscope}%
\end{pgfscope}%
\begin{pgfscope}%
\pgfsetbuttcap%
\pgfsetroundjoin%
\definecolor{currentfill}{rgb}{0.000000,0.000000,0.000000}%
\pgfsetfillcolor{currentfill}%
\pgfsetlinewidth{0.501875pt}%
\definecolor{currentstroke}{rgb}{0.000000,0.000000,0.000000}%
\pgfsetstrokecolor{currentstroke}%
\pgfsetdash{}{0pt}%
\pgfsys@defobject{currentmarker}{\pgfqpoint{0.000000in}{-0.055556in}}{\pgfqpoint{0.000000in}{0.000000in}}{%
\pgfpathmoveto{\pgfqpoint{0.000000in}{0.000000in}}%
\pgfpathlineto{\pgfqpoint{0.000000in}{-0.055556in}}%
\pgfusepath{stroke,fill}%
}%
\begin{pgfscope}%
\pgfsys@transformshift{4.884462in}{5.749375in}%
\pgfsys@useobject{currentmarker}{}%
\end{pgfscope}%
\end{pgfscope}%
\begin{pgfscope}%
\pgftext[x=4.884462in,y=3.994861in,,top]{\fontsize{11.000000}{13.200000}\selectfont 5}%
\end{pgfscope}%
\begin{pgfscope}%
\pgfsetbuttcap%
\pgfsetroundjoin%
\definecolor{currentfill}{rgb}{0.000000,0.000000,0.000000}%
\pgfsetfillcolor{currentfill}%
\pgfsetlinewidth{0.501875pt}%
\definecolor{currentstroke}{rgb}{0.000000,0.000000,0.000000}%
\pgfsetstrokecolor{currentstroke}%
\pgfsetdash{}{0pt}%
\pgfsys@defobject{currentmarker}{\pgfqpoint{0.000000in}{0.000000in}}{\pgfqpoint{0.000000in}{0.055556in}}{%
\pgfpathmoveto{\pgfqpoint{0.000000in}{0.000000in}}%
\pgfpathlineto{\pgfqpoint{0.000000in}{0.055556in}}%
\pgfusepath{stroke,fill}%
}%
\begin{pgfscope}%
\pgfsys@transformshift{5.173829in}{4.050417in}%
\pgfsys@useobject{currentmarker}{}%
\end{pgfscope}%
\end{pgfscope}%
\begin{pgfscope}%
\pgfsetbuttcap%
\pgfsetroundjoin%
\definecolor{currentfill}{rgb}{0.000000,0.000000,0.000000}%
\pgfsetfillcolor{currentfill}%
\pgfsetlinewidth{0.501875pt}%
\definecolor{currentstroke}{rgb}{0.000000,0.000000,0.000000}%
\pgfsetstrokecolor{currentstroke}%
\pgfsetdash{}{0pt}%
\pgfsys@defobject{currentmarker}{\pgfqpoint{0.000000in}{-0.055556in}}{\pgfqpoint{0.000000in}{0.000000in}}{%
\pgfpathmoveto{\pgfqpoint{0.000000in}{0.000000in}}%
\pgfpathlineto{\pgfqpoint{0.000000in}{-0.055556in}}%
\pgfusepath{stroke,fill}%
}%
\begin{pgfscope}%
\pgfsys@transformshift{5.173829in}{5.749375in}%
\pgfsys@useobject{currentmarker}{}%
\end{pgfscope}%
\end{pgfscope}%
\begin{pgfscope}%
\pgftext[x=5.173829in,y=3.994861in,,top]{\fontsize{11.000000}{13.200000}\selectfont 6}%
\end{pgfscope}%
\begin{pgfscope}%
\pgfsetbuttcap%
\pgfsetroundjoin%
\definecolor{currentfill}{rgb}{0.000000,0.000000,0.000000}%
\pgfsetfillcolor{currentfill}%
\pgfsetlinewidth{0.501875pt}%
\definecolor{currentstroke}{rgb}{0.000000,0.000000,0.000000}%
\pgfsetstrokecolor{currentstroke}%
\pgfsetdash{}{0pt}%
\pgfsys@defobject{currentmarker}{\pgfqpoint{0.000000in}{0.000000in}}{\pgfqpoint{0.000000in}{0.055556in}}{%
\pgfpathmoveto{\pgfqpoint{0.000000in}{0.000000in}}%
\pgfpathlineto{\pgfqpoint{0.000000in}{0.055556in}}%
\pgfusepath{stroke,fill}%
}%
\begin{pgfscope}%
\pgfsys@transformshift{5.463196in}{4.050417in}%
\pgfsys@useobject{currentmarker}{}%
\end{pgfscope}%
\end{pgfscope}%
\begin{pgfscope}%
\pgfsetbuttcap%
\pgfsetroundjoin%
\definecolor{currentfill}{rgb}{0.000000,0.000000,0.000000}%
\pgfsetfillcolor{currentfill}%
\pgfsetlinewidth{0.501875pt}%
\definecolor{currentstroke}{rgb}{0.000000,0.000000,0.000000}%
\pgfsetstrokecolor{currentstroke}%
\pgfsetdash{}{0pt}%
\pgfsys@defobject{currentmarker}{\pgfqpoint{0.000000in}{-0.055556in}}{\pgfqpoint{0.000000in}{0.000000in}}{%
\pgfpathmoveto{\pgfqpoint{0.000000in}{0.000000in}}%
\pgfpathlineto{\pgfqpoint{0.000000in}{-0.055556in}}%
\pgfusepath{stroke,fill}%
}%
\begin{pgfscope}%
\pgfsys@transformshift{5.463196in}{5.749375in}%
\pgfsys@useobject{currentmarker}{}%
\end{pgfscope}%
\end{pgfscope}%
\begin{pgfscope}%
\pgftext[x=5.463196in,y=3.994861in,,top]{\fontsize{11.000000}{13.200000}\selectfont 7}%
\end{pgfscope}%
\begin{pgfscope}%
\pgfsetbuttcap%
\pgfsetroundjoin%
\definecolor{currentfill}{rgb}{0.000000,0.000000,0.000000}%
\pgfsetfillcolor{currentfill}%
\pgfsetlinewidth{0.501875pt}%
\definecolor{currentstroke}{rgb}{0.000000,0.000000,0.000000}%
\pgfsetstrokecolor{currentstroke}%
\pgfsetdash{}{0pt}%
\pgfsys@defobject{currentmarker}{\pgfqpoint{0.000000in}{0.000000in}}{\pgfqpoint{0.055556in}{0.000000in}}{%
\pgfpathmoveto{\pgfqpoint{0.000000in}{0.000000in}}%
\pgfpathlineto{\pgfqpoint{0.055556in}{0.000000in}}%
\pgfusepath{stroke,fill}%
}%
\begin{pgfscope}%
\pgfsys@transformshift{3.221779in}{4.076158in}%
\pgfsys@useobject{currentmarker}{}%
\end{pgfscope}%
\end{pgfscope}%
\begin{pgfscope}%
\pgfsetbuttcap%
\pgfsetroundjoin%
\definecolor{currentfill}{rgb}{0.000000,0.000000,0.000000}%
\pgfsetfillcolor{currentfill}%
\pgfsetlinewidth{0.501875pt}%
\definecolor{currentstroke}{rgb}{0.000000,0.000000,0.000000}%
\pgfsetstrokecolor{currentstroke}%
\pgfsetdash{}{0pt}%
\pgfsys@defobject{currentmarker}{\pgfqpoint{-0.055556in}{0.000000in}}{\pgfqpoint{0.000000in}{0.000000in}}{%
\pgfpathmoveto{\pgfqpoint{0.000000in}{0.000000in}}%
\pgfpathlineto{\pgfqpoint{-0.055556in}{0.000000in}}%
\pgfusepath{stroke,fill}%
}%
\begin{pgfscope}%
\pgfsys@transformshift{5.545000in}{4.076158in}%
\pgfsys@useobject{currentmarker}{}%
\end{pgfscope}%
\end{pgfscope}%
\begin{pgfscope}%
\pgfsetbuttcap%
\pgfsetroundjoin%
\definecolor{currentfill}{rgb}{0.000000,0.000000,0.000000}%
\pgfsetfillcolor{currentfill}%
\pgfsetlinewidth{0.501875pt}%
\definecolor{currentstroke}{rgb}{0.000000,0.000000,0.000000}%
\pgfsetstrokecolor{currentstroke}%
\pgfsetdash{}{0pt}%
\pgfsys@defobject{currentmarker}{\pgfqpoint{0.000000in}{0.000000in}}{\pgfqpoint{0.055556in}{0.000000in}}{%
\pgfpathmoveto{\pgfqpoint{0.000000in}{0.000000in}}%
\pgfpathlineto{\pgfqpoint{0.055556in}{0.000000in}}%
\pgfusepath{stroke,fill}%
}%
\begin{pgfscope}%
\pgfsys@transformshift{3.221779in}{4.333576in}%
\pgfsys@useobject{currentmarker}{}%
\end{pgfscope}%
\end{pgfscope}%
\begin{pgfscope}%
\pgfsetbuttcap%
\pgfsetroundjoin%
\definecolor{currentfill}{rgb}{0.000000,0.000000,0.000000}%
\pgfsetfillcolor{currentfill}%
\pgfsetlinewidth{0.501875pt}%
\definecolor{currentstroke}{rgb}{0.000000,0.000000,0.000000}%
\pgfsetstrokecolor{currentstroke}%
\pgfsetdash{}{0pt}%
\pgfsys@defobject{currentmarker}{\pgfqpoint{-0.055556in}{0.000000in}}{\pgfqpoint{0.000000in}{0.000000in}}{%
\pgfpathmoveto{\pgfqpoint{0.000000in}{0.000000in}}%
\pgfpathlineto{\pgfqpoint{-0.055556in}{0.000000in}}%
\pgfusepath{stroke,fill}%
}%
\begin{pgfscope}%
\pgfsys@transformshift{5.545000in}{4.333576in}%
\pgfsys@useobject{currentmarker}{}%
\end{pgfscope}%
\end{pgfscope}%
\begin{pgfscope}%
\pgfsetbuttcap%
\pgfsetroundjoin%
\definecolor{currentfill}{rgb}{0.000000,0.000000,0.000000}%
\pgfsetfillcolor{currentfill}%
\pgfsetlinewidth{0.501875pt}%
\definecolor{currentstroke}{rgb}{0.000000,0.000000,0.000000}%
\pgfsetstrokecolor{currentstroke}%
\pgfsetdash{}{0pt}%
\pgfsys@defobject{currentmarker}{\pgfqpoint{0.000000in}{0.000000in}}{\pgfqpoint{0.055556in}{0.000000in}}{%
\pgfpathmoveto{\pgfqpoint{0.000000in}{0.000000in}}%
\pgfpathlineto{\pgfqpoint{0.055556in}{0.000000in}}%
\pgfusepath{stroke,fill}%
}%
\begin{pgfscope}%
\pgfsys@transformshift{3.221779in}{4.590994in}%
\pgfsys@useobject{currentmarker}{}%
\end{pgfscope}%
\end{pgfscope}%
\begin{pgfscope}%
\pgfsetbuttcap%
\pgfsetroundjoin%
\definecolor{currentfill}{rgb}{0.000000,0.000000,0.000000}%
\pgfsetfillcolor{currentfill}%
\pgfsetlinewidth{0.501875pt}%
\definecolor{currentstroke}{rgb}{0.000000,0.000000,0.000000}%
\pgfsetstrokecolor{currentstroke}%
\pgfsetdash{}{0pt}%
\pgfsys@defobject{currentmarker}{\pgfqpoint{-0.055556in}{0.000000in}}{\pgfqpoint{0.000000in}{0.000000in}}{%
\pgfpathmoveto{\pgfqpoint{0.000000in}{0.000000in}}%
\pgfpathlineto{\pgfqpoint{-0.055556in}{0.000000in}}%
\pgfusepath{stroke,fill}%
}%
\begin{pgfscope}%
\pgfsys@transformshift{5.545000in}{4.590994in}%
\pgfsys@useobject{currentmarker}{}%
\end{pgfscope}%
\end{pgfscope}%
\begin{pgfscope}%
\pgfsetbuttcap%
\pgfsetroundjoin%
\definecolor{currentfill}{rgb}{0.000000,0.000000,0.000000}%
\pgfsetfillcolor{currentfill}%
\pgfsetlinewidth{0.501875pt}%
\definecolor{currentstroke}{rgb}{0.000000,0.000000,0.000000}%
\pgfsetstrokecolor{currentstroke}%
\pgfsetdash{}{0pt}%
\pgfsys@defobject{currentmarker}{\pgfqpoint{0.000000in}{0.000000in}}{\pgfqpoint{0.055556in}{0.000000in}}{%
\pgfpathmoveto{\pgfqpoint{0.000000in}{0.000000in}}%
\pgfpathlineto{\pgfqpoint{0.055556in}{0.000000in}}%
\pgfusepath{stroke,fill}%
}%
\begin{pgfscope}%
\pgfsys@transformshift{3.221779in}{4.848412in}%
\pgfsys@useobject{currentmarker}{}%
\end{pgfscope}%
\end{pgfscope}%
\begin{pgfscope}%
\pgfsetbuttcap%
\pgfsetroundjoin%
\definecolor{currentfill}{rgb}{0.000000,0.000000,0.000000}%
\pgfsetfillcolor{currentfill}%
\pgfsetlinewidth{0.501875pt}%
\definecolor{currentstroke}{rgb}{0.000000,0.000000,0.000000}%
\pgfsetstrokecolor{currentstroke}%
\pgfsetdash{}{0pt}%
\pgfsys@defobject{currentmarker}{\pgfqpoint{-0.055556in}{0.000000in}}{\pgfqpoint{0.000000in}{0.000000in}}{%
\pgfpathmoveto{\pgfqpoint{0.000000in}{0.000000in}}%
\pgfpathlineto{\pgfqpoint{-0.055556in}{0.000000in}}%
\pgfusepath{stroke,fill}%
}%
\begin{pgfscope}%
\pgfsys@transformshift{5.545000in}{4.848412in}%
\pgfsys@useobject{currentmarker}{}%
\end{pgfscope}%
\end{pgfscope}%
\begin{pgfscope}%
\pgfsetbuttcap%
\pgfsetroundjoin%
\definecolor{currentfill}{rgb}{0.000000,0.000000,0.000000}%
\pgfsetfillcolor{currentfill}%
\pgfsetlinewidth{0.501875pt}%
\definecolor{currentstroke}{rgb}{0.000000,0.000000,0.000000}%
\pgfsetstrokecolor{currentstroke}%
\pgfsetdash{}{0pt}%
\pgfsys@defobject{currentmarker}{\pgfqpoint{0.000000in}{0.000000in}}{\pgfqpoint{0.055556in}{0.000000in}}{%
\pgfpathmoveto{\pgfqpoint{0.000000in}{0.000000in}}%
\pgfpathlineto{\pgfqpoint{0.055556in}{0.000000in}}%
\pgfusepath{stroke,fill}%
}%
\begin{pgfscope}%
\pgfsys@transformshift{3.221779in}{5.105830in}%
\pgfsys@useobject{currentmarker}{}%
\end{pgfscope}%
\end{pgfscope}%
\begin{pgfscope}%
\pgfsetbuttcap%
\pgfsetroundjoin%
\definecolor{currentfill}{rgb}{0.000000,0.000000,0.000000}%
\pgfsetfillcolor{currentfill}%
\pgfsetlinewidth{0.501875pt}%
\definecolor{currentstroke}{rgb}{0.000000,0.000000,0.000000}%
\pgfsetstrokecolor{currentstroke}%
\pgfsetdash{}{0pt}%
\pgfsys@defobject{currentmarker}{\pgfqpoint{-0.055556in}{0.000000in}}{\pgfqpoint{0.000000in}{0.000000in}}{%
\pgfpathmoveto{\pgfqpoint{0.000000in}{0.000000in}}%
\pgfpathlineto{\pgfqpoint{-0.055556in}{0.000000in}}%
\pgfusepath{stroke,fill}%
}%
\begin{pgfscope}%
\pgfsys@transformshift{5.545000in}{5.105830in}%
\pgfsys@useobject{currentmarker}{}%
\end{pgfscope}%
\end{pgfscope}%
\begin{pgfscope}%
\pgfsetbuttcap%
\pgfsetroundjoin%
\definecolor{currentfill}{rgb}{0.000000,0.000000,0.000000}%
\pgfsetfillcolor{currentfill}%
\pgfsetlinewidth{0.501875pt}%
\definecolor{currentstroke}{rgb}{0.000000,0.000000,0.000000}%
\pgfsetstrokecolor{currentstroke}%
\pgfsetdash{}{0pt}%
\pgfsys@defobject{currentmarker}{\pgfqpoint{0.000000in}{0.000000in}}{\pgfqpoint{0.055556in}{0.000000in}}{%
\pgfpathmoveto{\pgfqpoint{0.000000in}{0.000000in}}%
\pgfpathlineto{\pgfqpoint{0.055556in}{0.000000in}}%
\pgfusepath{stroke,fill}%
}%
\begin{pgfscope}%
\pgfsys@transformshift{3.221779in}{5.363248in}%
\pgfsys@useobject{currentmarker}{}%
\end{pgfscope}%
\end{pgfscope}%
\begin{pgfscope}%
\pgfsetbuttcap%
\pgfsetroundjoin%
\definecolor{currentfill}{rgb}{0.000000,0.000000,0.000000}%
\pgfsetfillcolor{currentfill}%
\pgfsetlinewidth{0.501875pt}%
\definecolor{currentstroke}{rgb}{0.000000,0.000000,0.000000}%
\pgfsetstrokecolor{currentstroke}%
\pgfsetdash{}{0pt}%
\pgfsys@defobject{currentmarker}{\pgfqpoint{-0.055556in}{0.000000in}}{\pgfqpoint{0.000000in}{0.000000in}}{%
\pgfpathmoveto{\pgfqpoint{0.000000in}{0.000000in}}%
\pgfpathlineto{\pgfqpoint{-0.055556in}{0.000000in}}%
\pgfusepath{stroke,fill}%
}%
\begin{pgfscope}%
\pgfsys@transformshift{5.545000in}{5.363248in}%
\pgfsys@useobject{currentmarker}{}%
\end{pgfscope}%
\end{pgfscope}%
\begin{pgfscope}%
\pgfsetbuttcap%
\pgfsetroundjoin%
\definecolor{currentfill}{rgb}{0.000000,0.000000,0.000000}%
\pgfsetfillcolor{currentfill}%
\pgfsetlinewidth{0.501875pt}%
\definecolor{currentstroke}{rgb}{0.000000,0.000000,0.000000}%
\pgfsetstrokecolor{currentstroke}%
\pgfsetdash{}{0pt}%
\pgfsys@defobject{currentmarker}{\pgfqpoint{0.000000in}{0.000000in}}{\pgfqpoint{0.055556in}{0.000000in}}{%
\pgfpathmoveto{\pgfqpoint{0.000000in}{0.000000in}}%
\pgfpathlineto{\pgfqpoint{0.055556in}{0.000000in}}%
\pgfusepath{stroke,fill}%
}%
\begin{pgfscope}%
\pgfsys@transformshift{3.221779in}{5.620666in}%
\pgfsys@useobject{currentmarker}{}%
\end{pgfscope}%
\end{pgfscope}%
\begin{pgfscope}%
\pgfsetbuttcap%
\pgfsetroundjoin%
\definecolor{currentfill}{rgb}{0.000000,0.000000,0.000000}%
\pgfsetfillcolor{currentfill}%
\pgfsetlinewidth{0.501875pt}%
\definecolor{currentstroke}{rgb}{0.000000,0.000000,0.000000}%
\pgfsetstrokecolor{currentstroke}%
\pgfsetdash{}{0pt}%
\pgfsys@defobject{currentmarker}{\pgfqpoint{-0.055556in}{0.000000in}}{\pgfqpoint{0.000000in}{0.000000in}}{%
\pgfpathmoveto{\pgfqpoint{0.000000in}{0.000000in}}%
\pgfpathlineto{\pgfqpoint{-0.055556in}{0.000000in}}%
\pgfusepath{stroke,fill}%
}%
\begin{pgfscope}%
\pgfsys@transformshift{5.545000in}{5.620666in}%
\pgfsys@useobject{currentmarker}{}%
\end{pgfscope}%
\end{pgfscope}%
\begin{pgfscope}%
\pgftext[x=4.537222in,y=4.462285in,left,base]{\fontsize{11.000000}{13.200000}\selectfont \(\displaystyle \times100\)}%
\end{pgfscope}%
\begin{pgfscope}%
\pgfsetbuttcap%
\pgfsetmiterjoin%
\definecolor{currentfill}{rgb}{1.000000,1.000000,1.000000}%
\pgfsetfillcolor{currentfill}%
\pgfsetlinewidth{0.000000pt}%
\definecolor{currentstroke}{rgb}{0.000000,0.000000,0.000000}%
\pgfsetstrokecolor{currentstroke}%
\pgfsetstrokeopacity{0.000000}%
\pgfsetdash{}{0pt}%
\pgfpathmoveto{\pgfqpoint{3.221779in}{2.351458in}}%
\pgfpathlineto{\pgfqpoint{5.545000in}{2.351458in}}%
\pgfpathlineto{\pgfqpoint{5.545000in}{4.050417in}}%
\pgfpathlineto{\pgfqpoint{3.221779in}{4.050417in}}%
\pgfpathclose%
\pgfusepath{fill}%
\end{pgfscope}%
\begin{pgfscope}%
\pgfpathrectangle{\pgfqpoint{3.221779in}{2.351458in}}{\pgfqpoint{2.323221in}{1.698958in}} %
\pgfusepath{clip}%
\pgfsetbuttcap%
\pgfsetroundjoin%
\definecolor{currentfill}{rgb}{0.309804,0.478431,0.682353}%
\pgfsetfillcolor{currentfill}%
\pgfsetfillopacity{0.500000}%
\pgfsetlinewidth{1.003750pt}%
\definecolor{currentstroke}{rgb}{0.309804,0.478431,0.682353}%
\pgfsetstrokecolor{currentstroke}%
\pgfsetstrokeopacity{0.500000}%
\pgfsetdash{}{0pt}%
\pgfpathmoveto{\pgfqpoint{3.437628in}{3.602065in}}%
\pgfpathlineto{\pgfqpoint{3.437628in}{3.346578in}}%
\pgfpathlineto{\pgfqpoint{3.443308in}{3.346282in}}%
\pgfpathlineto{\pgfqpoint{3.448988in}{3.345715in}}%
\pgfpathlineto{\pgfqpoint{3.454669in}{3.344961in}}%
\pgfpathlineto{\pgfqpoint{3.460349in}{3.344187in}}%
\pgfpathlineto{\pgfqpoint{3.466029in}{3.343597in}}%
\pgfpathlineto{\pgfqpoint{3.471709in}{3.343339in}}%
\pgfpathlineto{\pgfqpoint{3.477390in}{3.343417in}}%
\pgfpathlineto{\pgfqpoint{3.483070in}{3.343648in}}%
\pgfpathlineto{\pgfqpoint{3.488750in}{3.343692in}}%
\pgfpathlineto{\pgfqpoint{3.494430in}{3.343133in}}%
\pgfpathlineto{\pgfqpoint{3.500111in}{3.341590in}}%
\pgfpathlineto{\pgfqpoint{3.505791in}{3.338842in}}%
\pgfpathlineto{\pgfqpoint{3.511471in}{3.334910in}}%
\pgfpathlineto{\pgfqpoint{3.517151in}{3.330060in}}%
\pgfpathlineto{\pgfqpoint{3.522832in}{3.324699in}}%
\pgfpathlineto{\pgfqpoint{3.528512in}{3.319207in}}%
\pgfpathlineto{\pgfqpoint{3.534192in}{3.313791in}}%
\pgfpathlineto{\pgfqpoint{3.539872in}{3.308438in}}%
\pgfpathlineto{\pgfqpoint{3.545553in}{3.302984in}}%
\pgfpathlineto{\pgfqpoint{3.551233in}{3.297248in}}%
\pgfpathlineto{\pgfqpoint{3.556913in}{3.291163in}}%
\pgfpathlineto{\pgfqpoint{3.562593in}{3.284863in}}%
\pgfpathlineto{\pgfqpoint{3.568274in}{3.278695in}}%
\pgfpathlineto{\pgfqpoint{3.573954in}{3.273152in}}%
\pgfpathlineto{\pgfqpoint{3.579634in}{3.268736in}}%
\pgfpathlineto{\pgfqpoint{3.585314in}{3.265778in}}%
\pgfpathlineto{\pgfqpoint{3.590995in}{3.264286in}}%
\pgfpathlineto{\pgfqpoint{3.596675in}{3.263912in}}%
\pgfpathlineto{\pgfqpoint{3.602355in}{3.264070in}}%
\pgfpathlineto{\pgfqpoint{3.608035in}{3.264154in}}%
\pgfpathlineto{\pgfqpoint{3.613716in}{3.263766in}}%
\pgfpathlineto{\pgfqpoint{3.619396in}{3.262841in}}%
\pgfpathlineto{\pgfqpoint{3.625076in}{3.261592in}}%
\pgfpathlineto{\pgfqpoint{3.630756in}{3.260350in}}%
\pgfpathlineto{\pgfqpoint{3.636437in}{3.259367in}}%
\pgfpathlineto{\pgfqpoint{3.642117in}{3.258708in}}%
\pgfpathlineto{\pgfqpoint{3.647797in}{3.258278in}}%
\pgfpathlineto{\pgfqpoint{3.653477in}{3.257933in}}%
\pgfpathlineto{\pgfqpoint{3.659158in}{3.257591in}}%
\pgfpathlineto{\pgfqpoint{3.664838in}{3.257270in}}%
\pgfpathlineto{\pgfqpoint{3.670518in}{3.257034in}}%
\pgfpathlineto{\pgfqpoint{3.676198in}{3.256914in}}%
\pgfpathlineto{\pgfqpoint{3.681879in}{3.256860in}}%
\pgfpathlineto{\pgfqpoint{3.687559in}{3.256767in}}%
\pgfpathlineto{\pgfqpoint{3.693239in}{3.256530in}}%
\pgfpathlineto{\pgfqpoint{3.698919in}{3.256120in}}%
\pgfpathlineto{\pgfqpoint{3.704600in}{3.255620in}}%
\pgfpathlineto{\pgfqpoint{3.710280in}{3.255240in}}%
\pgfpathlineto{\pgfqpoint{3.715960in}{3.255281in}}%
\pgfpathlineto{\pgfqpoint{3.721640in}{3.256061in}}%
\pgfpathlineto{\pgfqpoint{3.727321in}{3.257808in}}%
\pgfpathlineto{\pgfqpoint{3.733001in}{3.260566in}}%
\pgfpathlineto{\pgfqpoint{3.738681in}{3.264145in}}%
\pgfpathlineto{\pgfqpoint{3.744361in}{3.268167in}}%
\pgfpathlineto{\pgfqpoint{3.750042in}{3.272175in}}%
\pgfpathlineto{\pgfqpoint{3.755722in}{3.275774in}}%
\pgfpathlineto{\pgfqpoint{3.761402in}{3.278713in}}%
\pgfpathlineto{\pgfqpoint{3.767082in}{3.280917in}}%
\pgfpathlineto{\pgfqpoint{3.772763in}{3.282440in}}%
\pgfpathlineto{\pgfqpoint{3.778443in}{3.283394in}}%
\pgfpathlineto{\pgfqpoint{3.784123in}{3.283885in}}%
\pgfpathlineto{\pgfqpoint{3.789803in}{3.283982in}}%
\pgfpathlineto{\pgfqpoint{3.795484in}{3.283710in}}%
\pgfpathlineto{\pgfqpoint{3.801164in}{3.283058in}}%
\pgfpathlineto{\pgfqpoint{3.806844in}{3.281993in}}%
\pgfpathlineto{\pgfqpoint{3.812524in}{3.280462in}}%
\pgfpathlineto{\pgfqpoint{3.818205in}{3.278415in}}%
\pgfpathlineto{\pgfqpoint{3.823885in}{3.275840in}}%
\pgfpathlineto{\pgfqpoint{3.829565in}{3.272806in}}%
\pgfpathlineto{\pgfqpoint{3.835245in}{3.269480in}}%
\pgfpathlineto{\pgfqpoint{3.840926in}{3.266087in}}%
\pgfpathlineto{\pgfqpoint{3.846606in}{3.262838in}}%
\pgfpathlineto{\pgfqpoint{3.852286in}{3.259851in}}%
\pgfpathlineto{\pgfqpoint{3.857966in}{3.257122in}}%
\pgfpathlineto{\pgfqpoint{3.863647in}{3.254563in}}%
\pgfpathlineto{\pgfqpoint{3.869327in}{3.252089in}}%
\pgfpathlineto{\pgfqpoint{3.875007in}{3.249699in}}%
\pgfpathlineto{\pgfqpoint{3.880687in}{3.247539in}}%
\pgfpathlineto{\pgfqpoint{3.886368in}{3.245903in}}%
\pgfpathlineto{\pgfqpoint{3.892048in}{3.245195in}}%
\pgfpathlineto{\pgfqpoint{3.897728in}{3.245834in}}%
\pgfpathlineto{\pgfqpoint{3.903408in}{3.248116in}}%
\pgfpathlineto{\pgfqpoint{3.909089in}{3.252080in}}%
\pgfpathlineto{\pgfqpoint{3.914769in}{3.257426in}}%
\pgfpathlineto{\pgfqpoint{3.920449in}{3.263535in}}%
\pgfpathlineto{\pgfqpoint{3.926129in}{3.269606in}}%
\pgfpathlineto{\pgfqpoint{3.931810in}{3.274838in}}%
\pgfpathlineto{\pgfqpoint{3.937490in}{3.278585in}}%
\pgfpathlineto{\pgfqpoint{3.943170in}{3.280426in}}%
\pgfpathlineto{\pgfqpoint{3.948850in}{3.280148in}}%
\pgfpathlineto{\pgfqpoint{3.954531in}{3.277740in}}%
\pgfpathlineto{\pgfqpoint{3.960211in}{3.273421in}}%
\pgfpathlineto{\pgfqpoint{3.965891in}{3.267707in}}%
\pgfpathlineto{\pgfqpoint{3.971571in}{3.261432in}}%
\pgfpathlineto{\pgfqpoint{3.977252in}{3.255627in}}%
\pgfpathlineto{\pgfqpoint{3.982932in}{3.251246in}}%
\pgfpathlineto{\pgfqpoint{3.988612in}{3.248882in}}%
\pgfpathlineto{\pgfqpoint{3.994292in}{3.248610in}}%
\pgfpathlineto{\pgfqpoint{3.999973in}{3.250075in}}%
\pgfpathlineto{\pgfqpoint{4.005653in}{3.252740in}}%
\pgfpathlineto{\pgfqpoint{4.011333in}{3.256143in}}%
\pgfpathlineto{\pgfqpoint{4.017013in}{3.260016in}}%
\pgfpathlineto{\pgfqpoint{4.022694in}{3.264222in}}%
\pgfpathlineto{\pgfqpoint{4.028374in}{3.268631in}}%
\pgfpathlineto{\pgfqpoint{4.034054in}{3.273022in}}%
\pgfpathlineto{\pgfqpoint{4.039734in}{3.277100in}}%
\pgfpathlineto{\pgfqpoint{4.045414in}{3.280601in}}%
\pgfpathlineto{\pgfqpoint{4.051095in}{3.283401in}}%
\pgfpathlineto{\pgfqpoint{4.056775in}{3.285579in}}%
\pgfpathlineto{\pgfqpoint{4.062455in}{3.287391in}}%
\pgfpathlineto{\pgfqpoint{4.068135in}{3.289164in}}%
\pgfpathlineto{\pgfqpoint{4.073816in}{3.291166in}}%
\pgfpathlineto{\pgfqpoint{4.079496in}{3.293460in}}%
\pgfpathlineto{\pgfqpoint{4.085176in}{3.295831in}}%
\pgfpathlineto{\pgfqpoint{4.090856in}{3.297802in}}%
\pgfpathlineto{\pgfqpoint{4.096537in}{3.298758in}}%
\pgfpathlineto{\pgfqpoint{4.102217in}{3.298144in}}%
\pgfpathlineto{\pgfqpoint{4.107897in}{3.295670in}}%
\pgfpathlineto{\pgfqpoint{4.113577in}{3.291460in}}%
\pgfpathlineto{\pgfqpoint{4.119258in}{3.286054in}}%
\pgfpathlineto{\pgfqpoint{4.124938in}{3.280274in}}%
\pgfpathlineto{\pgfqpoint{4.130618in}{3.274980in}}%
\pgfpathlineto{\pgfqpoint{4.136298in}{3.270839in}}%
\pgfpathlineto{\pgfqpoint{4.141979in}{3.268184in}}%
\pgfpathlineto{\pgfqpoint{4.147659in}{3.267010in}}%
\pgfpathlineto{\pgfqpoint{4.153339in}{3.267076in}}%
\pgfpathlineto{\pgfqpoint{4.159019in}{3.268007in}}%
\pgfpathlineto{\pgfqpoint{4.164700in}{3.269389in}}%
\pgfpathlineto{\pgfqpoint{4.170380in}{3.270819in}}%
\pgfpathlineto{\pgfqpoint{4.176060in}{3.271941in}}%
\pgfpathlineto{\pgfqpoint{4.181740in}{3.272493in}}%
\pgfpathlineto{\pgfqpoint{4.187421in}{3.272351in}}%
\pgfpathlineto{\pgfqpoint{4.193101in}{3.271549in}}%
\pgfpathlineto{\pgfqpoint{4.198781in}{3.270255in}}%
\pgfpathlineto{\pgfqpoint{4.204461in}{3.268692in}}%
\pgfpathlineto{\pgfqpoint{4.210142in}{3.267038in}}%
\pgfpathlineto{\pgfqpoint{4.215822in}{3.265323in}}%
\pgfpathlineto{\pgfqpoint{4.221502in}{3.263385in}}%
\pgfpathlineto{\pgfqpoint{4.227182in}{3.260930in}}%
\pgfpathlineto{\pgfqpoint{4.232863in}{3.257690in}}%
\pgfpathlineto{\pgfqpoint{4.238543in}{3.253637in}}%
\pgfpathlineto{\pgfqpoint{4.244223in}{3.249120in}}%
\pgfpathlineto{\pgfqpoint{4.249903in}{3.244848in}}%
\pgfpathlineto{\pgfqpoint{4.255584in}{3.241683in}}%
\pgfpathlineto{\pgfqpoint{4.261264in}{3.240347in}}%
\pgfpathlineto{\pgfqpoint{4.266944in}{3.241146in}}%
\pgfpathlineto{\pgfqpoint{4.272624in}{3.243853in}}%
\pgfpathlineto{\pgfqpoint{4.278305in}{3.247748in}}%
\pgfpathlineto{\pgfqpoint{4.283985in}{3.251817in}}%
\pgfpathlineto{\pgfqpoint{4.289665in}{3.255012in}}%
\pgfpathlineto{\pgfqpoint{4.295345in}{3.256470in}}%
\pgfpathlineto{\pgfqpoint{4.301026in}{3.255691in}}%
\pgfpathlineto{\pgfqpoint{4.306706in}{3.252541in}}%
\pgfpathlineto{\pgfqpoint{4.312386in}{3.247162in}}%
\pgfpathlineto{\pgfqpoint{4.318066in}{3.239767in}}%
\pgfpathlineto{\pgfqpoint{4.323747in}{3.230514in}}%
\pgfpathlineto{\pgfqpoint{4.329427in}{3.219520in}}%
\pgfpathlineto{\pgfqpoint{4.335107in}{3.207078in}}%
\pgfpathlineto{\pgfqpoint{4.340787in}{3.193934in}}%
\pgfpathlineto{\pgfqpoint{4.346468in}{3.181409in}}%
\pgfpathlineto{\pgfqpoint{4.352148in}{3.171241in}}%
\pgfpathlineto{\pgfqpoint{4.357828in}{3.165151in}}%
\pgfpathlineto{\pgfqpoint{4.363508in}{3.164323in}}%
\pgfpathlineto{\pgfqpoint{4.369189in}{3.169011in}}%
\pgfpathlineto{\pgfqpoint{4.374869in}{3.178392in}}%
\pgfpathlineto{\pgfqpoint{4.380549in}{3.190728in}}%
\pgfpathlineto{\pgfqpoint{4.386229in}{3.203804in}}%
\pgfpathlineto{\pgfqpoint{4.391910in}{3.215512in}}%
\pgfpathlineto{\pgfqpoint{4.397590in}{3.224388in}}%
\pgfpathlineto{\pgfqpoint{4.403270in}{3.229873in}}%
\pgfpathlineto{\pgfqpoint{4.408950in}{3.232162in}}%
\pgfpathlineto{\pgfqpoint{4.414631in}{3.231765in}}%
\pgfpathlineto{\pgfqpoint{4.420311in}{3.229040in}}%
\pgfpathlineto{\pgfqpoint{4.425991in}{3.223989in}}%
\pgfpathlineto{\pgfqpoint{4.431671in}{3.216471in}}%
\pgfpathlineto{\pgfqpoint{4.437352in}{3.206649in}}%
\pgfpathlineto{\pgfqpoint{4.443032in}{3.195421in}}%
\pgfpathlineto{\pgfqpoint{4.448712in}{3.184510in}}%
\pgfpathlineto{\pgfqpoint{4.454392in}{3.176046in}}%
\pgfpathlineto{\pgfqpoint{4.460073in}{3.171835in}}%
\pgfpathlineto{\pgfqpoint{4.465753in}{3.172588in}}%
\pgfpathlineto{\pgfqpoint{4.471433in}{3.177579in}}%
\pgfpathlineto{\pgfqpoint{4.477113in}{3.184912in}}%
\pgfpathlineto{\pgfqpoint{4.482794in}{3.192309in}}%
\pgfpathlineto{\pgfqpoint{4.488474in}{3.197988in}}%
\pgfpathlineto{\pgfqpoint{4.494154in}{3.201190in}}%
\pgfpathlineto{\pgfqpoint{4.499834in}{3.202134in}}%
\pgfpathlineto{\pgfqpoint{4.505515in}{3.201542in}}%
\pgfpathlineto{\pgfqpoint{4.511195in}{3.200056in}}%
\pgfpathlineto{\pgfqpoint{4.516875in}{3.197903in}}%
\pgfpathlineto{\pgfqpoint{4.522555in}{3.194886in}}%
\pgfpathlineto{\pgfqpoint{4.528236in}{3.190673in}}%
\pgfpathlineto{\pgfqpoint{4.533916in}{3.185118in}}%
\pgfpathlineto{\pgfqpoint{4.539596in}{3.178486in}}%
\pgfpathlineto{\pgfqpoint{4.545276in}{3.171436in}}%
\pgfpathlineto{\pgfqpoint{4.550957in}{3.164809in}}%
\pgfpathlineto{\pgfqpoint{4.556637in}{3.159262in}}%
\pgfpathlineto{\pgfqpoint{4.562317in}{3.155002in}}%
\pgfpathlineto{\pgfqpoint{4.567997in}{3.151741in}}%
\pgfpathlineto{\pgfqpoint{4.573678in}{3.148929in}}%
\pgfpathlineto{\pgfqpoint{4.579358in}{3.146110in}}%
\pgfpathlineto{\pgfqpoint{4.585038in}{3.143195in}}%
\pgfpathlineto{\pgfqpoint{4.590718in}{3.140509in}}%
\pgfpathlineto{\pgfqpoint{4.596399in}{3.138548in}}%
\pgfpathlineto{\pgfqpoint{4.602079in}{3.137684in}}%
\pgfpathlineto{\pgfqpoint{4.607759in}{3.137941in}}%
\pgfpathlineto{\pgfqpoint{4.613439in}{3.139001in}}%
\pgfpathlineto{\pgfqpoint{4.619120in}{3.140357in}}%
\pgfpathlineto{\pgfqpoint{4.624800in}{3.141549in}}%
\pgfpathlineto{\pgfqpoint{4.630480in}{3.142294in}}%
\pgfpathlineto{\pgfqpoint{4.636160in}{3.142518in}}%
\pgfpathlineto{\pgfqpoint{4.641841in}{3.142306in}}%
\pgfpathlineto{\pgfqpoint{4.647521in}{3.141881in}}%
\pgfpathlineto{\pgfqpoint{4.653201in}{3.141561in}}%
\pgfpathlineto{\pgfqpoint{4.658881in}{3.141666in}}%
\pgfpathlineto{\pgfqpoint{4.664562in}{3.142350in}}%
\pgfpathlineto{\pgfqpoint{4.670242in}{3.143444in}}%
\pgfpathlineto{\pgfqpoint{4.675922in}{3.144421in}}%
\pgfpathlineto{\pgfqpoint{4.681602in}{3.144562in}}%
\pgfpathlineto{\pgfqpoint{4.687283in}{3.143227in}}%
\pgfpathlineto{\pgfqpoint{4.692963in}{3.140122in}}%
\pgfpathlineto{\pgfqpoint{4.698643in}{3.135419in}}%
\pgfpathlineto{\pgfqpoint{4.704323in}{3.129683in}}%
\pgfpathlineto{\pgfqpoint{4.710004in}{3.123699in}}%
\pgfpathlineto{\pgfqpoint{4.715684in}{3.118233in}}%
\pgfpathlineto{\pgfqpoint{4.721364in}{3.113819in}}%
\pgfpathlineto{\pgfqpoint{4.727044in}{3.110681in}}%
\pgfpathlineto{\pgfqpoint{4.732725in}{3.108716in}}%
\pgfpathlineto{\pgfqpoint{4.738405in}{3.107630in}}%
\pgfpathlineto{\pgfqpoint{4.744085in}{3.107073in}}%
\pgfpathlineto{\pgfqpoint{4.749765in}{3.106733in}}%
\pgfpathlineto{\pgfqpoint{4.755446in}{3.106297in}}%
\pgfpathlineto{\pgfqpoint{4.761126in}{3.105348in}}%
\pgfpathlineto{\pgfqpoint{4.766806in}{3.103310in}}%
\pgfpathlineto{\pgfqpoint{4.772486in}{3.099593in}}%
\pgfpathlineto{\pgfqpoint{4.778167in}{3.093851in}}%
\pgfpathlineto{\pgfqpoint{4.783847in}{3.086205in}}%
\pgfpathlineto{\pgfqpoint{4.789527in}{3.077235in}}%
\pgfpathlineto{\pgfqpoint{4.795207in}{3.067781in}}%
\pgfpathlineto{\pgfqpoint{4.800887in}{3.058659in}}%
\pgfpathlineto{\pgfqpoint{4.806568in}{3.050453in}}%
\pgfpathlineto{\pgfqpoint{4.812248in}{3.043415in}}%
\pgfpathlineto{\pgfqpoint{4.817928in}{3.037411in}}%
\pgfpathlineto{\pgfqpoint{4.823608in}{3.031935in}}%
\pgfpathlineto{\pgfqpoint{4.829289in}{3.026268in}}%
\pgfpathlineto{\pgfqpoint{4.834969in}{3.019783in}}%
\pgfpathlineto{\pgfqpoint{4.840649in}{3.012291in}}%
\pgfpathlineto{\pgfqpoint{4.846329in}{3.004201in}}%
\pgfpathlineto{\pgfqpoint{4.852010in}{2.996336in}}%
\pgfpathlineto{\pgfqpoint{4.857690in}{2.989511in}}%
\pgfpathlineto{\pgfqpoint{4.863370in}{2.984118in}}%
\pgfpathlineto{\pgfqpoint{4.869050in}{2.980017in}}%
\pgfpathlineto{\pgfqpoint{4.874731in}{2.976838in}}%
\pgfpathlineto{\pgfqpoint{4.880411in}{2.974445in}}%
\pgfpathlineto{\pgfqpoint{4.886091in}{2.973261in}}%
\pgfpathlineto{\pgfqpoint{4.891771in}{2.974140in}}%
\pgfpathlineto{\pgfqpoint{4.897452in}{2.977836in}}%
\pgfpathlineto{\pgfqpoint{4.903132in}{2.984353in}}%
\pgfpathlineto{\pgfqpoint{4.908812in}{2.992611in}}%
\pgfpathlineto{\pgfqpoint{4.914492in}{3.000754in}}%
\pgfpathlineto{\pgfqpoint{4.920173in}{3.006876in}}%
\pgfpathlineto{\pgfqpoint{4.925853in}{3.009809in}}%
\pgfpathlineto{\pgfqpoint{4.931533in}{3.009517in}}%
\pgfpathlineto{\pgfqpoint{4.937213in}{3.006993in}}%
\pgfpathlineto{\pgfqpoint{4.942894in}{3.003796in}}%
\pgfpathlineto{\pgfqpoint{4.948574in}{3.001527in}}%
\pgfpathlineto{\pgfqpoint{4.954254in}{3.001425in}}%
\pgfpathlineto{\pgfqpoint{4.959934in}{3.004120in}}%
\pgfpathlineto{\pgfqpoint{4.965615in}{3.009608in}}%
\pgfpathlineto{\pgfqpoint{4.971295in}{3.017441in}}%
\pgfpathlineto{\pgfqpoint{4.976975in}{3.027102in}}%
\pgfpathlineto{\pgfqpoint{4.982655in}{3.038355in}}%
\pgfpathlineto{\pgfqpoint{4.988336in}{3.051412in}}%
\pgfpathlineto{\pgfqpoint{4.994016in}{3.066810in}}%
\pgfpathlineto{\pgfqpoint{4.999696in}{3.085121in}}%
\pgfpathlineto{\pgfqpoint{5.005376in}{3.106590in}}%
\pgfpathlineto{\pgfqpoint{5.011057in}{3.130827in}}%
\pgfpathlineto{\pgfqpoint{5.016737in}{3.156590in}}%
\pgfpathlineto{\pgfqpoint{5.022417in}{3.181780in}}%
\pgfpathlineto{\pgfqpoint{5.028097in}{3.203733in}}%
\pgfpathlineto{\pgfqpoint{5.033778in}{3.219857in}}%
\pgfpathlineto{\pgfqpoint{5.039458in}{3.228424in}}%
\pgfpathlineto{\pgfqpoint{5.045138in}{3.229175in}}%
\pgfpathlineto{\pgfqpoint{5.050818in}{3.223410in}}%
\pgfpathlineto{\pgfqpoint{5.056499in}{3.213528in}}%
\pgfpathlineto{\pgfqpoint{5.062179in}{3.202273in}}%
\pgfpathlineto{\pgfqpoint{5.067859in}{3.192008in}}%
\pgfpathlineto{\pgfqpoint{5.073539in}{3.184404in}}%
\pgfpathlineto{\pgfqpoint{5.079220in}{3.180403in}}%
\pgfpathlineto{\pgfqpoint{5.079220in}{3.723778in}}%
\pgfpathlineto{\pgfqpoint{5.079220in}{3.723778in}}%
\pgfpathlineto{\pgfqpoint{5.073539in}{3.712866in}}%
\pgfpathlineto{\pgfqpoint{5.067859in}{3.692382in}}%
\pgfpathlineto{\pgfqpoint{5.062179in}{3.664723in}}%
\pgfpathlineto{\pgfqpoint{5.056499in}{3.632936in}}%
\pgfpathlineto{\pgfqpoint{5.050818in}{3.600266in}}%
\pgfpathlineto{\pgfqpoint{5.045138in}{3.569837in}}%
\pgfpathlineto{\pgfqpoint{5.039458in}{3.544277in}}%
\pgfpathlineto{\pgfqpoint{5.033778in}{3.525454in}}%
\pgfpathlineto{\pgfqpoint{5.028097in}{3.514326in}}%
\pgfpathlineto{\pgfqpoint{5.022417in}{3.510962in}}%
\pgfpathlineto{\pgfqpoint{5.016737in}{3.514661in}}%
\pgfpathlineto{\pgfqpoint{5.011057in}{3.524067in}}%
\pgfpathlineto{\pgfqpoint{5.005376in}{3.537252in}}%
\pgfpathlineto{\pgfqpoint{4.999696in}{3.551880in}}%
\pgfpathlineto{\pgfqpoint{4.994016in}{3.565575in}}%
\pgfpathlineto{\pgfqpoint{4.988336in}{3.576418in}}%
\pgfpathlineto{\pgfqpoint{4.982655in}{3.583313in}}%
\pgfpathlineto{\pgfqpoint{4.976975in}{3.586025in}}%
\pgfpathlineto{\pgfqpoint{4.971295in}{3.584865in}}%
\pgfpathlineto{\pgfqpoint{4.965615in}{3.580299in}}%
\pgfpathlineto{\pgfqpoint{4.959934in}{3.572764in}}%
\pgfpathlineto{\pgfqpoint{4.954254in}{3.562739in}}%
\pgfpathlineto{\pgfqpoint{4.948574in}{3.550991in}}%
\pgfpathlineto{\pgfqpoint{4.942894in}{3.538724in}}%
\pgfpathlineto{\pgfqpoint{4.937213in}{3.527578in}}%
\pgfpathlineto{\pgfqpoint{4.931533in}{3.519406in}}%
\pgfpathlineto{\pgfqpoint{4.925853in}{3.515889in}}%
\pgfpathlineto{\pgfqpoint{4.920173in}{3.518007in}}%
\pgfpathlineto{\pgfqpoint{4.914492in}{3.525578in}}%
\pgfpathlineto{\pgfqpoint{4.908812in}{3.537149in}}%
\pgfpathlineto{\pgfqpoint{4.903132in}{3.550424in}}%
\pgfpathlineto{\pgfqpoint{4.897452in}{3.563113in}}%
\pgfpathlineto{\pgfqpoint{4.891771in}{3.573737in}}%
\pgfpathlineto{\pgfqpoint{4.886091in}{3.582016in}}%
\pgfpathlineto{\pgfqpoint{4.880411in}{3.588596in}}%
\pgfpathlineto{\pgfqpoint{4.874731in}{3.594406in}}%
\pgfpathlineto{\pgfqpoint{4.869050in}{3.600074in}}%
\pgfpathlineto{\pgfqpoint{4.863370in}{3.605705in}}%
\pgfpathlineto{\pgfqpoint{4.857690in}{3.611098in}}%
\pgfpathlineto{\pgfqpoint{4.852010in}{3.616116in}}%
\pgfpathlineto{\pgfqpoint{4.846329in}{3.620949in}}%
\pgfpathlineto{\pgfqpoint{4.840649in}{3.626027in}}%
\pgfpathlineto{\pgfqpoint{4.834969in}{3.631688in}}%
\pgfpathlineto{\pgfqpoint{4.829289in}{3.637866in}}%
\pgfpathlineto{\pgfqpoint{4.823608in}{3.644002in}}%
\pgfpathlineto{\pgfqpoint{4.817928in}{3.649256in}}%
\pgfpathlineto{\pgfqpoint{4.812248in}{3.652844in}}%
\pgfpathlineto{\pgfqpoint{4.806568in}{3.654277in}}%
\pgfpathlineto{\pgfqpoint{4.800887in}{3.653450in}}%
\pgfpathlineto{\pgfqpoint{4.795207in}{3.650581in}}%
\pgfpathlineto{\pgfqpoint{4.789527in}{3.646153in}}%
\pgfpathlineto{\pgfqpoint{4.783847in}{3.640838in}}%
\pgfpathlineto{\pgfqpoint{4.778167in}{3.635361in}}%
\pgfpathlineto{\pgfqpoint{4.772486in}{3.630299in}}%
\pgfpathlineto{\pgfqpoint{4.766806in}{3.625929in}}%
\pgfpathlineto{\pgfqpoint{4.761126in}{3.622257in}}%
\pgfpathlineto{\pgfqpoint{4.755446in}{3.619202in}}%
\pgfpathlineto{\pgfqpoint{4.749765in}{3.616789in}}%
\pgfpathlineto{\pgfqpoint{4.744085in}{3.615205in}}%
\pgfpathlineto{\pgfqpoint{4.738405in}{3.614651in}}%
\pgfpathlineto{\pgfqpoint{4.732725in}{3.615168in}}%
\pgfpathlineto{\pgfqpoint{4.727044in}{3.616556in}}%
\pgfpathlineto{\pgfqpoint{4.721364in}{3.618471in}}%
\pgfpathlineto{\pgfqpoint{4.715684in}{3.620577in}}%
\pgfpathlineto{\pgfqpoint{4.710004in}{3.622715in}}%
\pgfpathlineto{\pgfqpoint{4.704323in}{3.624926in}}%
\pgfpathlineto{\pgfqpoint{4.698643in}{3.627403in}}%
\pgfpathlineto{\pgfqpoint{4.692963in}{3.630341in}}%
\pgfpathlineto{\pgfqpoint{4.687283in}{3.633778in}}%
\pgfpathlineto{\pgfqpoint{4.681602in}{3.637517in}}%
\pgfpathlineto{\pgfqpoint{4.675922in}{3.641142in}}%
\pgfpathlineto{\pgfqpoint{4.670242in}{3.644194in}}%
\pgfpathlineto{\pgfqpoint{4.664562in}{3.646413in}}%
\pgfpathlineto{\pgfqpoint{4.658881in}{3.647920in}}%
\pgfpathlineto{\pgfqpoint{4.653201in}{3.649228in}}%
\pgfpathlineto{\pgfqpoint{4.647521in}{3.651054in}}%
\pgfpathlineto{\pgfqpoint{4.641841in}{3.654021in}}%
\pgfpathlineto{\pgfqpoint{4.636160in}{3.658427in}}%
\pgfpathlineto{\pgfqpoint{4.630480in}{3.664138in}}%
\pgfpathlineto{\pgfqpoint{4.624800in}{3.670620in}}%
\pgfpathlineto{\pgfqpoint{4.619120in}{3.677002in}}%
\pgfpathlineto{\pgfqpoint{4.613439in}{3.682168in}}%
\pgfpathlineto{\pgfqpoint{4.607759in}{3.684907in}}%
\pgfpathlineto{\pgfqpoint{4.602079in}{3.684168in}}%
\pgfpathlineto{\pgfqpoint{4.596399in}{3.679392in}}%
\pgfpathlineto{\pgfqpoint{4.590718in}{3.670724in}}%
\pgfpathlineto{\pgfqpoint{4.585038in}{3.659017in}}%
\pgfpathlineto{\pgfqpoint{4.579358in}{3.645557in}}%
\pgfpathlineto{\pgfqpoint{4.573678in}{3.631640in}}%
\pgfpathlineto{\pgfqpoint{4.567997in}{3.618169in}}%
\pgfpathlineto{\pgfqpoint{4.562317in}{3.605535in}}%
\pgfpathlineto{\pgfqpoint{4.556637in}{3.593757in}}%
\pgfpathlineto{\pgfqpoint{4.550957in}{3.582791in}}%
\pgfpathlineto{\pgfqpoint{4.545276in}{3.572788in}}%
\pgfpathlineto{\pgfqpoint{4.539596in}{3.564150in}}%
\pgfpathlineto{\pgfqpoint{4.533916in}{3.557386in}}%
\pgfpathlineto{\pgfqpoint{4.528236in}{3.552845in}}%
\pgfpathlineto{\pgfqpoint{4.522555in}{3.550562in}}%
\pgfpathlineto{\pgfqpoint{4.516875in}{3.550232in}}%
\pgfpathlineto{\pgfqpoint{4.511195in}{3.551388in}}%
\pgfpathlineto{\pgfqpoint{4.505515in}{3.553671in}}%
\pgfpathlineto{\pgfqpoint{4.499834in}{3.557072in}}%
\pgfpathlineto{\pgfqpoint{4.494154in}{3.561929in}}%
\pgfpathlineto{\pgfqpoint{4.488474in}{3.568619in}}%
\pgfpathlineto{\pgfqpoint{4.482794in}{3.577047in}}%
\pgfpathlineto{\pgfqpoint{4.477113in}{3.586225in}}%
\pgfpathlineto{\pgfqpoint{4.471433in}{3.594273in}}%
\pgfpathlineto{\pgfqpoint{4.465753in}{3.598957in}}%
\pgfpathlineto{\pgfqpoint{4.460073in}{3.598535in}}%
\pgfpathlineto{\pgfqpoint{4.454392in}{3.592492in}}%
\pgfpathlineto{\pgfqpoint{4.448712in}{3.581738in}}%
\pgfpathlineto{\pgfqpoint{4.443032in}{3.568206in}}%
\pgfpathlineto{\pgfqpoint{4.437352in}{3.554066in}}%
\pgfpathlineto{\pgfqpoint{4.431671in}{3.541009in}}%
\pgfpathlineto{\pgfqpoint{4.425991in}{3.529895in}}%
\pgfpathlineto{\pgfqpoint{4.420311in}{3.520909in}}%
\pgfpathlineto{\pgfqpoint{4.414631in}{3.514051in}}%
\pgfpathlineto{\pgfqpoint{4.408950in}{3.509585in}}%
\pgfpathlineto{\pgfqpoint{4.403270in}{3.508198in}}%
\pgfpathlineto{\pgfqpoint{4.397590in}{3.510721in}}%
\pgfpathlineto{\pgfqpoint{4.391910in}{3.517567in}}%
\pgfpathlineto{\pgfqpoint{4.386229in}{3.528230in}}%
\pgfpathlineto{\pgfqpoint{4.380549in}{3.541145in}}%
\pgfpathlineto{\pgfqpoint{4.374869in}{3.553986in}}%
\pgfpathlineto{\pgfqpoint{4.369189in}{3.564276in}}%
\pgfpathlineto{\pgfqpoint{4.363508in}{3.570055in}}%
\pgfpathlineto{\pgfqpoint{4.357828in}{3.570351in}}%
\pgfpathlineto{\pgfqpoint{4.352148in}{3.565353in}}%
\pgfpathlineto{\pgfqpoint{4.346468in}{3.556229in}}%
\pgfpathlineto{\pgfqpoint{4.340787in}{3.544693in}}%
\pgfpathlineto{\pgfqpoint{4.335107in}{3.532453in}}%
\pgfpathlineto{\pgfqpoint{4.329427in}{3.520762in}}%
\pgfpathlineto{\pgfqpoint{4.323747in}{3.510272in}}%
\pgfpathlineto{\pgfqpoint{4.318066in}{3.501179in}}%
\pgfpathlineto{\pgfqpoint{4.312386in}{3.493517in}}%
\pgfpathlineto{\pgfqpoint{4.306706in}{3.487382in}}%
\pgfpathlineto{\pgfqpoint{4.301026in}{3.482961in}}%
\pgfpathlineto{\pgfqpoint{4.295345in}{3.480408in}}%
\pgfpathlineto{\pgfqpoint{4.289665in}{3.479647in}}%
\pgfpathlineto{\pgfqpoint{4.283985in}{3.480292in}}%
\pgfpathlineto{\pgfqpoint{4.278305in}{3.481643in}}%
\pgfpathlineto{\pgfqpoint{4.272624in}{3.482840in}}%
\pgfpathlineto{\pgfqpoint{4.266944in}{3.483048in}}%
\pgfpathlineto{\pgfqpoint{4.261264in}{3.481676in}}%
\pgfpathlineto{\pgfqpoint{4.255584in}{3.478542in}}%
\pgfpathlineto{\pgfqpoint{4.249903in}{3.473913in}}%
\pgfpathlineto{\pgfqpoint{4.244223in}{3.468390in}}%
\pgfpathlineto{\pgfqpoint{4.238543in}{3.462676in}}%
\pgfpathlineto{\pgfqpoint{4.232863in}{3.457317in}}%
\pgfpathlineto{\pgfqpoint{4.227182in}{3.452535in}}%
\pgfpathlineto{\pgfqpoint{4.221502in}{3.448233in}}%
\pgfpathlineto{\pgfqpoint{4.215822in}{3.444140in}}%
\pgfpathlineto{\pgfqpoint{4.210142in}{3.440011in}}%
\pgfpathlineto{\pgfqpoint{4.204461in}{3.435772in}}%
\pgfpathlineto{\pgfqpoint{4.198781in}{3.431573in}}%
\pgfpathlineto{\pgfqpoint{4.193101in}{3.427718in}}%
\pgfpathlineto{\pgfqpoint{4.187421in}{3.424563in}}%
\pgfpathlineto{\pgfqpoint{4.181740in}{3.422386in}}%
\pgfpathlineto{\pgfqpoint{4.176060in}{3.421305in}}%
\pgfpathlineto{\pgfqpoint{4.170380in}{3.421264in}}%
\pgfpathlineto{\pgfqpoint{4.164700in}{3.422082in}}%
\pgfpathlineto{\pgfqpoint{4.159019in}{3.423530in}}%
\pgfpathlineto{\pgfqpoint{4.153339in}{3.425373in}}%
\pgfpathlineto{\pgfqpoint{4.147659in}{3.427354in}}%
\pgfpathlineto{\pgfqpoint{4.141979in}{3.429149in}}%
\pgfpathlineto{\pgfqpoint{4.136298in}{3.430364in}}%
\pgfpathlineto{\pgfqpoint{4.130618in}{3.430624in}}%
\pgfpathlineto{\pgfqpoint{4.124938in}{3.429720in}}%
\pgfpathlineto{\pgfqpoint{4.119258in}{3.427730in}}%
\pgfpathlineto{\pgfqpoint{4.113577in}{3.425001in}}%
\pgfpathlineto{\pgfqpoint{4.107897in}{3.422020in}}%
\pgfpathlineto{\pgfqpoint{4.102217in}{3.419224in}}%
\pgfpathlineto{\pgfqpoint{4.096537in}{3.416856in}}%
\pgfpathlineto{\pgfqpoint{4.090856in}{3.414930in}}%
\pgfpathlineto{\pgfqpoint{4.085176in}{3.413290in}}%
\pgfpathlineto{\pgfqpoint{4.079496in}{3.411719in}}%
\pgfpathlineto{\pgfqpoint{4.073816in}{3.410045in}}%
\pgfpathlineto{\pgfqpoint{4.068135in}{3.408206in}}%
\pgfpathlineto{\pgfqpoint{4.062455in}{3.406267in}}%
\pgfpathlineto{\pgfqpoint{4.056775in}{3.404397in}}%
\pgfpathlineto{\pgfqpoint{4.051095in}{3.402805in}}%
\pgfpathlineto{\pgfqpoint{4.045414in}{3.401680in}}%
\pgfpathlineto{\pgfqpoint{4.039734in}{3.401123in}}%
\pgfpathlineto{\pgfqpoint{4.034054in}{3.401129in}}%
\pgfpathlineto{\pgfqpoint{4.028374in}{3.401617in}}%
\pgfpathlineto{\pgfqpoint{4.022694in}{3.402508in}}%
\pgfpathlineto{\pgfqpoint{4.017013in}{3.403791in}}%
\pgfpathlineto{\pgfqpoint{4.011333in}{3.405510in}}%
\pgfpathlineto{\pgfqpoint{4.005653in}{3.407655in}}%
\pgfpathlineto{\pgfqpoint{3.999973in}{3.410006in}}%
\pgfpathlineto{\pgfqpoint{3.994292in}{3.412066in}}%
\pgfpathlineto{\pgfqpoint{3.988612in}{3.413172in}}%
\pgfpathlineto{\pgfqpoint{3.982932in}{3.412753in}}%
\pgfpathlineto{\pgfqpoint{3.977252in}{3.410603in}}%
\pgfpathlineto{\pgfqpoint{3.971571in}{3.406992in}}%
\pgfpathlineto{\pgfqpoint{3.965891in}{3.402549in}}%
\pgfpathlineto{\pgfqpoint{3.960211in}{3.398005in}}%
\pgfpathlineto{\pgfqpoint{3.954531in}{3.393948in}}%
\pgfpathlineto{\pgfqpoint{3.948850in}{3.390718in}}%
\pgfpathlineto{\pgfqpoint{3.943170in}{3.388442in}}%
\pgfpathlineto{\pgfqpoint{3.937490in}{3.387116in}}%
\pgfpathlineto{\pgfqpoint{3.931810in}{3.386651in}}%
\pgfpathlineto{\pgfqpoint{3.926129in}{3.386854in}}%
\pgfpathlineto{\pgfqpoint{3.920449in}{3.387414in}}%
\pgfpathlineto{\pgfqpoint{3.914769in}{3.387927in}}%
\pgfpathlineto{\pgfqpoint{3.909089in}{3.388006in}}%
\pgfpathlineto{\pgfqpoint{3.903408in}{3.387400in}}%
\pgfpathlineto{\pgfqpoint{3.897728in}{3.386070in}}%
\pgfpathlineto{\pgfqpoint{3.892048in}{3.384174in}}%
\pgfpathlineto{\pgfqpoint{3.886368in}{3.381967in}}%
\pgfpathlineto{\pgfqpoint{3.880687in}{3.379679in}}%
\pgfpathlineto{\pgfqpoint{3.875007in}{3.377432in}}%
\pgfpathlineto{\pgfqpoint{3.869327in}{3.375210in}}%
\pgfpathlineto{\pgfqpoint{3.863647in}{3.372891in}}%
\pgfpathlineto{\pgfqpoint{3.857966in}{3.370312in}}%
\pgfpathlineto{\pgfqpoint{3.852286in}{3.367353in}}%
\pgfpathlineto{\pgfqpoint{3.846606in}{3.364014in}}%
\pgfpathlineto{\pgfqpoint{3.840926in}{3.360453in}}%
\pgfpathlineto{\pgfqpoint{3.835245in}{3.356965in}}%
\pgfpathlineto{\pgfqpoint{3.829565in}{3.353907in}}%
\pgfpathlineto{\pgfqpoint{3.823885in}{3.351593in}}%
\pgfpathlineto{\pgfqpoint{3.818205in}{3.350213in}}%
\pgfpathlineto{\pgfqpoint{3.812524in}{3.349807in}}%
\pgfpathlineto{\pgfqpoint{3.806844in}{3.350272in}}%
\pgfpathlineto{\pgfqpoint{3.801164in}{3.351403in}}%
\pgfpathlineto{\pgfqpoint{3.795484in}{3.352926in}}%
\pgfpathlineto{\pgfqpoint{3.789803in}{3.354527in}}%
\pgfpathlineto{\pgfqpoint{3.784123in}{3.355899in}}%
\pgfpathlineto{\pgfqpoint{3.778443in}{3.356793in}}%
\pgfpathlineto{\pgfqpoint{3.772763in}{3.357086in}}%
\pgfpathlineto{\pgfqpoint{3.767082in}{3.356811in}}%
\pgfpathlineto{\pgfqpoint{3.761402in}{3.356151in}}%
\pgfpathlineto{\pgfqpoint{3.755722in}{3.355370in}}%
\pgfpathlineto{\pgfqpoint{3.750042in}{3.354718in}}%
\pgfpathlineto{\pgfqpoint{3.744361in}{3.354335in}}%
\pgfpathlineto{\pgfqpoint{3.738681in}{3.354197in}}%
\pgfpathlineto{\pgfqpoint{3.733001in}{3.354133in}}%
\pgfpathlineto{\pgfqpoint{3.727321in}{3.353911in}}%
\pgfpathlineto{\pgfqpoint{3.721640in}{3.353329in}}%
\pgfpathlineto{\pgfqpoint{3.715960in}{3.352300in}}%
\pgfpathlineto{\pgfqpoint{3.710280in}{3.350866in}}%
\pgfpathlineto{\pgfqpoint{3.704600in}{3.349176in}}%
\pgfpathlineto{\pgfqpoint{3.698919in}{3.347438in}}%
\pgfpathlineto{\pgfqpoint{3.693239in}{3.345864in}}%
\pgfpathlineto{\pgfqpoint{3.687559in}{3.344627in}}%
\pgfpathlineto{\pgfqpoint{3.681879in}{3.343838in}}%
\pgfpathlineto{\pgfqpoint{3.676198in}{3.343543in}}%
\pgfpathlineto{\pgfqpoint{3.670518in}{3.343753in}}%
\pgfpathlineto{\pgfqpoint{3.664838in}{3.344474in}}%
\pgfpathlineto{\pgfqpoint{3.659158in}{3.345707in}}%
\pgfpathlineto{\pgfqpoint{3.653477in}{3.347408in}}%
\pgfpathlineto{\pgfqpoint{3.647797in}{3.349422in}}%
\pgfpathlineto{\pgfqpoint{3.642117in}{3.351460in}}%
\pgfpathlineto{\pgfqpoint{3.636437in}{3.353173in}}%
\pgfpathlineto{\pgfqpoint{3.630756in}{3.354297in}}%
\pgfpathlineto{\pgfqpoint{3.625076in}{3.354803in}}%
\pgfpathlineto{\pgfqpoint{3.619396in}{3.354945in}}%
\pgfpathlineto{\pgfqpoint{3.613716in}{3.355161in}}%
\pgfpathlineto{\pgfqpoint{3.608035in}{3.355880in}}%
\pgfpathlineto{\pgfqpoint{3.602355in}{3.357323in}}%
\pgfpathlineto{\pgfqpoint{3.596675in}{3.359423in}}%
\pgfpathlineto{\pgfqpoint{3.590995in}{3.361891in}}%
\pgfpathlineto{\pgfqpoint{3.585314in}{3.364410in}}%
\pgfpathlineto{\pgfqpoint{3.579634in}{3.366830in}}%
\pgfpathlineto{\pgfqpoint{3.573954in}{3.369279in}}%
\pgfpathlineto{\pgfqpoint{3.568274in}{3.372131in}}%
\pgfpathlineto{\pgfqpoint{3.562593in}{3.375863in}}%
\pgfpathlineto{\pgfqpoint{3.556913in}{3.380879in}}%
\pgfpathlineto{\pgfqpoint{3.551233in}{3.387372in}}%
\pgfpathlineto{\pgfqpoint{3.545553in}{3.395257in}}%
\pgfpathlineto{\pgfqpoint{3.539872in}{3.404189in}}%
\pgfpathlineto{\pgfqpoint{3.534192in}{3.413673in}}%
\pgfpathlineto{\pgfqpoint{3.528512in}{3.423229in}}%
\pgfpathlineto{\pgfqpoint{3.522832in}{3.432567in}}%
\pgfpathlineto{\pgfqpoint{3.517151in}{3.441706in}}%
\pgfpathlineto{\pgfqpoint{3.511471in}{3.450957in}}%
\pgfpathlineto{\pgfqpoint{3.505791in}{3.460784in}}%
\pgfpathlineto{\pgfqpoint{3.500111in}{3.471616in}}%
\pgfpathlineto{\pgfqpoint{3.494430in}{3.483696in}}%
\pgfpathlineto{\pgfqpoint{3.488750in}{3.497016in}}%
\pgfpathlineto{\pgfqpoint{3.483070in}{3.511332in}}%
\pgfpathlineto{\pgfqpoint{3.477390in}{3.526236in}}%
\pgfpathlineto{\pgfqpoint{3.471709in}{3.541222in}}%
\pgfpathlineto{\pgfqpoint{3.466029in}{3.555761in}}%
\pgfpathlineto{\pgfqpoint{3.460349in}{3.569319in}}%
\pgfpathlineto{\pgfqpoint{3.454669in}{3.581345in}}%
\pgfpathlineto{\pgfqpoint{3.448988in}{3.591233in}}%
\pgfpathlineto{\pgfqpoint{3.443308in}{3.598335in}}%
\pgfpathlineto{\pgfqpoint{3.437628in}{3.602065in}}%
\pgfpathclose%
\pgfusepath{stroke,fill}%
\end{pgfscope}%
\begin{pgfscope}%
\pgfpathrectangle{\pgfqpoint{3.221779in}{2.351458in}}{\pgfqpoint{2.323221in}{1.698958in}} %
\pgfusepath{clip}%
\pgfsetrectcap%
\pgfsetroundjoin%
\pgfsetlinewidth{1.003750pt}%
\definecolor{currentstroke}{rgb}{0.309804,0.478431,0.682353}%
\pgfsetstrokecolor{currentstroke}%
\pgfsetdash{}{0pt}%
\pgfpathmoveto{\pgfqpoint{3.437628in}{3.474321in}}%
\pgfpathlineto{\pgfqpoint{3.443308in}{3.472309in}}%
\pgfpathlineto{\pgfqpoint{3.448988in}{3.468474in}}%
\pgfpathlineto{\pgfqpoint{3.460349in}{3.456753in}}%
\pgfpathlineto{\pgfqpoint{3.562593in}{3.330363in}}%
\pgfpathlineto{\pgfqpoint{3.573954in}{3.321216in}}%
\pgfpathlineto{\pgfqpoint{3.585314in}{3.315094in}}%
\pgfpathlineto{\pgfqpoint{3.596675in}{3.311668in}}%
\pgfpathlineto{\pgfqpoint{3.619396in}{3.308893in}}%
\pgfpathlineto{\pgfqpoint{3.642117in}{3.305084in}}%
\pgfpathlineto{\pgfqpoint{3.664838in}{3.300872in}}%
\pgfpathlineto{\pgfqpoint{3.681879in}{3.300349in}}%
\pgfpathlineto{\pgfqpoint{3.710280in}{3.303053in}}%
\pgfpathlineto{\pgfqpoint{3.727321in}{3.305860in}}%
\pgfpathlineto{\pgfqpoint{3.744361in}{3.311251in}}%
\pgfpathlineto{\pgfqpoint{3.767082in}{3.318864in}}%
\pgfpathlineto{\pgfqpoint{3.778443in}{3.320093in}}%
\pgfpathlineto{\pgfqpoint{3.789803in}{3.319255in}}%
\pgfpathlineto{\pgfqpoint{3.829565in}{3.313357in}}%
\pgfpathlineto{\pgfqpoint{3.857966in}{3.313717in}}%
\pgfpathlineto{\pgfqpoint{3.892048in}{3.314685in}}%
\pgfpathlineto{\pgfqpoint{3.903408in}{3.317758in}}%
\pgfpathlineto{\pgfqpoint{3.920449in}{3.325474in}}%
\pgfpathlineto{\pgfqpoint{3.937490in}{3.332851in}}%
\pgfpathlineto{\pgfqpoint{3.948850in}{3.335433in}}%
\pgfpathlineto{\pgfqpoint{3.960211in}{3.335713in}}%
\pgfpathlineto{\pgfqpoint{3.977252in}{3.333115in}}%
\pgfpathlineto{\pgfqpoint{3.994292in}{3.330338in}}%
\pgfpathlineto{\pgfqpoint{4.005653in}{3.330197in}}%
\pgfpathlineto{\pgfqpoint{4.017013in}{3.331903in}}%
\pgfpathlineto{\pgfqpoint{4.034054in}{3.337075in}}%
\pgfpathlineto{\pgfqpoint{4.090856in}{3.356366in}}%
\pgfpathlineto{\pgfqpoint{4.102217in}{3.358684in}}%
\pgfpathlineto{\pgfqpoint{4.113577in}{3.358230in}}%
\pgfpathlineto{\pgfqpoint{4.124938in}{3.354997in}}%
\pgfpathlineto{\pgfqpoint{4.147659in}{3.347182in}}%
\pgfpathlineto{\pgfqpoint{4.159019in}{3.345768in}}%
\pgfpathlineto{\pgfqpoint{4.176060in}{3.346623in}}%
\pgfpathlineto{\pgfqpoint{4.198781in}{3.350914in}}%
\pgfpathlineto{\pgfqpoint{4.227182in}{3.356732in}}%
\pgfpathlineto{\pgfqpoint{4.278305in}{3.364696in}}%
\pgfpathlineto{\pgfqpoint{4.301026in}{3.369326in}}%
\pgfpathlineto{\pgfqpoint{4.318066in}{3.370473in}}%
\pgfpathlineto{\pgfqpoint{4.340787in}{3.369314in}}%
\pgfpathlineto{\pgfqpoint{4.386229in}{3.366017in}}%
\pgfpathlineto{\pgfqpoint{4.397590in}{3.367555in}}%
\pgfpathlineto{\pgfqpoint{4.414631in}{3.372908in}}%
\pgfpathlineto{\pgfqpoint{4.437352in}{3.380358in}}%
\pgfpathlineto{\pgfqpoint{4.460073in}{3.385185in}}%
\pgfpathlineto{\pgfqpoint{4.471433in}{3.385926in}}%
\pgfpathlineto{\pgfqpoint{4.482794in}{3.384678in}}%
\pgfpathlineto{\pgfqpoint{4.499834in}{3.379603in}}%
\pgfpathlineto{\pgfqpoint{4.522555in}{3.372724in}}%
\pgfpathlineto{\pgfqpoint{4.533916in}{3.371252in}}%
\pgfpathlineto{\pgfqpoint{4.545276in}{3.372112in}}%
\pgfpathlineto{\pgfqpoint{4.556637in}{3.376509in}}%
\pgfpathlineto{\pgfqpoint{4.567997in}{3.384955in}}%
\pgfpathlineto{\pgfqpoint{4.590718in}{3.405616in}}%
\pgfpathlineto{\pgfqpoint{4.596399in}{3.408970in}}%
\pgfpathlineto{\pgfqpoint{4.602079in}{3.410926in}}%
\pgfpathlineto{\pgfqpoint{4.607759in}{3.411424in}}%
\pgfpathlineto{\pgfqpoint{4.619120in}{3.408680in}}%
\pgfpathlineto{\pgfqpoint{4.647521in}{3.396467in}}%
\pgfpathlineto{\pgfqpoint{4.658881in}{3.394793in}}%
\pgfpathlineto{\pgfqpoint{4.675922in}{3.392782in}}%
\pgfpathlineto{\pgfqpoint{4.687283in}{3.388503in}}%
\pgfpathlineto{\pgfqpoint{4.698643in}{3.381411in}}%
\pgfpathlineto{\pgfqpoint{4.721364in}{3.366145in}}%
\pgfpathlineto{\pgfqpoint{4.732725in}{3.361942in}}%
\pgfpathlineto{\pgfqpoint{4.744085in}{3.361139in}}%
\pgfpathlineto{\pgfqpoint{4.778167in}{3.364606in}}%
\pgfpathlineto{\pgfqpoint{4.789527in}{3.361694in}}%
\pgfpathlineto{\pgfqpoint{4.800887in}{3.356054in}}%
\pgfpathlineto{\pgfqpoint{4.812248in}{3.348129in}}%
\pgfpathlineto{\pgfqpoint{4.823608in}{3.337968in}}%
\pgfpathlineto{\pgfqpoint{4.840649in}{3.319159in}}%
\pgfpathlineto{\pgfqpoint{4.857690in}{3.300304in}}%
\pgfpathlineto{\pgfqpoint{4.874731in}{3.285622in}}%
\pgfpathlineto{\pgfqpoint{4.897452in}{3.270475in}}%
\pgfpathlineto{\pgfqpoint{4.908812in}{3.264880in}}%
\pgfpathlineto{\pgfqpoint{4.920173in}{3.262442in}}%
\pgfpathlineto{\pgfqpoint{4.925853in}{3.262849in}}%
\pgfpathlineto{\pgfqpoint{4.931533in}{3.264461in}}%
\pgfpathlineto{\pgfqpoint{4.942894in}{3.271260in}}%
\pgfpathlineto{\pgfqpoint{4.954254in}{3.282082in}}%
\pgfpathlineto{\pgfqpoint{4.976975in}{3.306563in}}%
\pgfpathlineto{\pgfqpoint{4.988336in}{3.313915in}}%
\pgfpathlineto{\pgfqpoint{5.005376in}{3.321921in}}%
\pgfpathlineto{\pgfqpoint{5.011057in}{3.327447in}}%
\pgfpathlineto{\pgfqpoint{5.016737in}{3.335625in}}%
\pgfpathlineto{\pgfqpoint{5.028097in}{3.359030in}}%
\pgfpathlineto{\pgfqpoint{5.050818in}{3.411838in}}%
\pgfpathlineto{\pgfqpoint{5.062179in}{3.433498in}}%
\pgfpathlineto{\pgfqpoint{5.067859in}{3.442195in}}%
\pgfpathlineto{\pgfqpoint{5.073539in}{3.448635in}}%
\pgfpathlineto{\pgfqpoint{5.079220in}{3.452091in}}%
\pgfpathlineto{\pgfqpoint{5.079220in}{3.452091in}}%
\pgfusepath{stroke}%
\end{pgfscope}%
\begin{pgfscope}%
\pgfpathrectangle{\pgfqpoint{3.221779in}{2.351458in}}{\pgfqpoint{2.323221in}{1.698958in}} %
\pgfusepath{clip}%
\pgfsetbuttcap%
\pgfsetroundjoin%
\pgfsetlinewidth{1.003750pt}%
\definecolor{currentstroke}{rgb}{0.000000,0.000000,0.000000}%
\pgfsetstrokecolor{currentstroke}%
\pgfsetdash{{1.000000pt}{3.000000pt}}{0.000000pt}%
\pgfpathmoveto{\pgfqpoint{3.437628in}{2.351458in}}%
\pgfpathlineto{\pgfqpoint{3.437628in}{4.050417in}}%
\pgfusepath{stroke}%
\end{pgfscope}%
\begin{pgfscope}%
\pgfpathrectangle{\pgfqpoint{3.221779in}{2.351458in}}{\pgfqpoint{2.323221in}{1.698958in}} %
\pgfusepath{clip}%
\pgfsetbuttcap%
\pgfsetroundjoin%
\pgfsetlinewidth{1.003750pt}%
\definecolor{currentstroke}{rgb}{1.000000,0.400000,0.200000}%
\pgfsetstrokecolor{currentstroke}%
\pgfsetdash{{1.000000pt}{3.000000pt}}{0.000000pt}%
\pgfpathmoveto{\pgfqpoint{3.221779in}{3.434360in}}%
\pgfpathlineto{\pgfqpoint{5.545000in}{3.434360in}}%
\pgfusepath{stroke}%
\end{pgfscope}%
\begin{pgfscope}%
\pgfsetrectcap%
\pgfsetmiterjoin%
\pgfsetlinewidth{1.003750pt}%
\definecolor{currentstroke}{rgb}{0.000000,0.000000,0.000000}%
\pgfsetstrokecolor{currentstroke}%
\pgfsetdash{}{0pt}%
\pgfpathmoveto{\pgfqpoint{5.545000in}{2.351458in}}%
\pgfpathlineto{\pgfqpoint{5.545000in}{4.050417in}}%
\pgfusepath{stroke}%
\end{pgfscope}%
\begin{pgfscope}%
\pgfsetrectcap%
\pgfsetmiterjoin%
\pgfsetlinewidth{1.003750pt}%
\definecolor{currentstroke}{rgb}{0.000000,0.000000,0.000000}%
\pgfsetstrokecolor{currentstroke}%
\pgfsetdash{}{0pt}%
\pgfpathmoveto{\pgfqpoint{3.221779in}{2.351458in}}%
\pgfpathlineto{\pgfqpoint{3.221779in}{4.050417in}}%
\pgfusepath{stroke}%
\end{pgfscope}%
\begin{pgfscope}%
\pgfsetrectcap%
\pgfsetmiterjoin%
\pgfsetlinewidth{1.003750pt}%
\definecolor{currentstroke}{rgb}{0.000000,0.000000,0.000000}%
\pgfsetstrokecolor{currentstroke}%
\pgfsetdash{}{0pt}%
\pgfpathmoveto{\pgfqpoint{3.221779in}{4.050417in}}%
\pgfpathlineto{\pgfqpoint{5.545000in}{4.050417in}}%
\pgfusepath{stroke}%
\end{pgfscope}%
\begin{pgfscope}%
\pgfsetrectcap%
\pgfsetmiterjoin%
\pgfsetlinewidth{1.003750pt}%
\definecolor{currentstroke}{rgb}{0.000000,0.000000,0.000000}%
\pgfsetstrokecolor{currentstroke}%
\pgfsetdash{}{0pt}%
\pgfpathmoveto{\pgfqpoint{3.221779in}{2.351458in}}%
\pgfpathlineto{\pgfqpoint{5.545000in}{2.351458in}}%
\pgfusepath{stroke}%
\end{pgfscope}%
\begin{pgfscope}%
\pgfsetbuttcap%
\pgfsetroundjoin%
\definecolor{currentfill}{rgb}{0.000000,0.000000,0.000000}%
\pgfsetfillcolor{currentfill}%
\pgfsetlinewidth{0.501875pt}%
\definecolor{currentstroke}{rgb}{0.000000,0.000000,0.000000}%
\pgfsetstrokecolor{currentstroke}%
\pgfsetdash{}{0pt}%
\pgfsys@defobject{currentmarker}{\pgfqpoint{0.000000in}{0.000000in}}{\pgfqpoint{0.000000in}{0.055556in}}{%
\pgfpathmoveto{\pgfqpoint{0.000000in}{0.000000in}}%
\pgfpathlineto{\pgfqpoint{0.000000in}{0.055556in}}%
\pgfusepath{stroke,fill}%
}%
\begin{pgfscope}%
\pgfsys@transformshift{3.437628in}{2.351458in}%
\pgfsys@useobject{currentmarker}{}%
\end{pgfscope}%
\end{pgfscope}%
\begin{pgfscope}%
\pgfsetbuttcap%
\pgfsetroundjoin%
\definecolor{currentfill}{rgb}{0.000000,0.000000,0.000000}%
\pgfsetfillcolor{currentfill}%
\pgfsetlinewidth{0.501875pt}%
\definecolor{currentstroke}{rgb}{0.000000,0.000000,0.000000}%
\pgfsetstrokecolor{currentstroke}%
\pgfsetdash{}{0pt}%
\pgfsys@defobject{currentmarker}{\pgfqpoint{0.000000in}{-0.055556in}}{\pgfqpoint{0.000000in}{0.000000in}}{%
\pgfpathmoveto{\pgfqpoint{0.000000in}{0.000000in}}%
\pgfpathlineto{\pgfqpoint{0.000000in}{-0.055556in}}%
\pgfusepath{stroke,fill}%
}%
\begin{pgfscope}%
\pgfsys@transformshift{3.437628in}{4.050417in}%
\pgfsys@useobject{currentmarker}{}%
\end{pgfscope}%
\end{pgfscope}%
\begin{pgfscope}%
\pgftext[x=3.437628in,y=2.295903in,,top]{\fontsize{11.000000}{13.200000}\selectfont 0}%
\end{pgfscope}%
\begin{pgfscope}%
\pgfsetbuttcap%
\pgfsetroundjoin%
\definecolor{currentfill}{rgb}{0.000000,0.000000,0.000000}%
\pgfsetfillcolor{currentfill}%
\pgfsetlinewidth{0.501875pt}%
\definecolor{currentstroke}{rgb}{0.000000,0.000000,0.000000}%
\pgfsetstrokecolor{currentstroke}%
\pgfsetdash{}{0pt}%
\pgfsys@defobject{currentmarker}{\pgfqpoint{0.000000in}{0.000000in}}{\pgfqpoint{0.000000in}{0.055556in}}{%
\pgfpathmoveto{\pgfqpoint{0.000000in}{0.000000in}}%
\pgfpathlineto{\pgfqpoint{0.000000in}{0.055556in}}%
\pgfusepath{stroke,fill}%
}%
\begin{pgfscope}%
\pgfsys@transformshift{3.726995in}{2.351458in}%
\pgfsys@useobject{currentmarker}{}%
\end{pgfscope}%
\end{pgfscope}%
\begin{pgfscope}%
\pgfsetbuttcap%
\pgfsetroundjoin%
\definecolor{currentfill}{rgb}{0.000000,0.000000,0.000000}%
\pgfsetfillcolor{currentfill}%
\pgfsetlinewidth{0.501875pt}%
\definecolor{currentstroke}{rgb}{0.000000,0.000000,0.000000}%
\pgfsetstrokecolor{currentstroke}%
\pgfsetdash{}{0pt}%
\pgfsys@defobject{currentmarker}{\pgfqpoint{0.000000in}{-0.055556in}}{\pgfqpoint{0.000000in}{0.000000in}}{%
\pgfpathmoveto{\pgfqpoint{0.000000in}{0.000000in}}%
\pgfpathlineto{\pgfqpoint{0.000000in}{-0.055556in}}%
\pgfusepath{stroke,fill}%
}%
\begin{pgfscope}%
\pgfsys@transformshift{3.726995in}{4.050417in}%
\pgfsys@useobject{currentmarker}{}%
\end{pgfscope}%
\end{pgfscope}%
\begin{pgfscope}%
\pgftext[x=3.726995in,y=2.295903in,,top]{\fontsize{11.000000}{13.200000}\selectfont 1}%
\end{pgfscope}%
\begin{pgfscope}%
\pgfsetbuttcap%
\pgfsetroundjoin%
\definecolor{currentfill}{rgb}{0.000000,0.000000,0.000000}%
\pgfsetfillcolor{currentfill}%
\pgfsetlinewidth{0.501875pt}%
\definecolor{currentstroke}{rgb}{0.000000,0.000000,0.000000}%
\pgfsetstrokecolor{currentstroke}%
\pgfsetdash{}{0pt}%
\pgfsys@defobject{currentmarker}{\pgfqpoint{0.000000in}{0.000000in}}{\pgfqpoint{0.000000in}{0.055556in}}{%
\pgfpathmoveto{\pgfqpoint{0.000000in}{0.000000in}}%
\pgfpathlineto{\pgfqpoint{0.000000in}{0.055556in}}%
\pgfusepath{stroke,fill}%
}%
\begin{pgfscope}%
\pgfsys@transformshift{4.016362in}{2.351458in}%
\pgfsys@useobject{currentmarker}{}%
\end{pgfscope}%
\end{pgfscope}%
\begin{pgfscope}%
\pgfsetbuttcap%
\pgfsetroundjoin%
\definecolor{currentfill}{rgb}{0.000000,0.000000,0.000000}%
\pgfsetfillcolor{currentfill}%
\pgfsetlinewidth{0.501875pt}%
\definecolor{currentstroke}{rgb}{0.000000,0.000000,0.000000}%
\pgfsetstrokecolor{currentstroke}%
\pgfsetdash{}{0pt}%
\pgfsys@defobject{currentmarker}{\pgfqpoint{0.000000in}{-0.055556in}}{\pgfqpoint{0.000000in}{0.000000in}}{%
\pgfpathmoveto{\pgfqpoint{0.000000in}{0.000000in}}%
\pgfpathlineto{\pgfqpoint{0.000000in}{-0.055556in}}%
\pgfusepath{stroke,fill}%
}%
\begin{pgfscope}%
\pgfsys@transformshift{4.016362in}{4.050417in}%
\pgfsys@useobject{currentmarker}{}%
\end{pgfscope}%
\end{pgfscope}%
\begin{pgfscope}%
\pgftext[x=4.016362in,y=2.295903in,,top]{\fontsize{11.000000}{13.200000}\selectfont 2}%
\end{pgfscope}%
\begin{pgfscope}%
\pgfsetbuttcap%
\pgfsetroundjoin%
\definecolor{currentfill}{rgb}{0.000000,0.000000,0.000000}%
\pgfsetfillcolor{currentfill}%
\pgfsetlinewidth{0.501875pt}%
\definecolor{currentstroke}{rgb}{0.000000,0.000000,0.000000}%
\pgfsetstrokecolor{currentstroke}%
\pgfsetdash{}{0pt}%
\pgfsys@defobject{currentmarker}{\pgfqpoint{0.000000in}{0.000000in}}{\pgfqpoint{0.000000in}{0.055556in}}{%
\pgfpathmoveto{\pgfqpoint{0.000000in}{0.000000in}}%
\pgfpathlineto{\pgfqpoint{0.000000in}{0.055556in}}%
\pgfusepath{stroke,fill}%
}%
\begin{pgfscope}%
\pgfsys@transformshift{4.305728in}{2.351458in}%
\pgfsys@useobject{currentmarker}{}%
\end{pgfscope}%
\end{pgfscope}%
\begin{pgfscope}%
\pgfsetbuttcap%
\pgfsetroundjoin%
\definecolor{currentfill}{rgb}{0.000000,0.000000,0.000000}%
\pgfsetfillcolor{currentfill}%
\pgfsetlinewidth{0.501875pt}%
\definecolor{currentstroke}{rgb}{0.000000,0.000000,0.000000}%
\pgfsetstrokecolor{currentstroke}%
\pgfsetdash{}{0pt}%
\pgfsys@defobject{currentmarker}{\pgfqpoint{0.000000in}{-0.055556in}}{\pgfqpoint{0.000000in}{0.000000in}}{%
\pgfpathmoveto{\pgfqpoint{0.000000in}{0.000000in}}%
\pgfpathlineto{\pgfqpoint{0.000000in}{-0.055556in}}%
\pgfusepath{stroke,fill}%
}%
\begin{pgfscope}%
\pgfsys@transformshift{4.305728in}{4.050417in}%
\pgfsys@useobject{currentmarker}{}%
\end{pgfscope}%
\end{pgfscope}%
\begin{pgfscope}%
\pgftext[x=4.305728in,y=2.295903in,,top]{\fontsize{11.000000}{13.200000}\selectfont 3}%
\end{pgfscope}%
\begin{pgfscope}%
\pgfsetbuttcap%
\pgfsetroundjoin%
\definecolor{currentfill}{rgb}{0.000000,0.000000,0.000000}%
\pgfsetfillcolor{currentfill}%
\pgfsetlinewidth{0.501875pt}%
\definecolor{currentstroke}{rgb}{0.000000,0.000000,0.000000}%
\pgfsetstrokecolor{currentstroke}%
\pgfsetdash{}{0pt}%
\pgfsys@defobject{currentmarker}{\pgfqpoint{0.000000in}{0.000000in}}{\pgfqpoint{0.000000in}{0.055556in}}{%
\pgfpathmoveto{\pgfqpoint{0.000000in}{0.000000in}}%
\pgfpathlineto{\pgfqpoint{0.000000in}{0.055556in}}%
\pgfusepath{stroke,fill}%
}%
\begin{pgfscope}%
\pgfsys@transformshift{4.595095in}{2.351458in}%
\pgfsys@useobject{currentmarker}{}%
\end{pgfscope}%
\end{pgfscope}%
\begin{pgfscope}%
\pgfsetbuttcap%
\pgfsetroundjoin%
\definecolor{currentfill}{rgb}{0.000000,0.000000,0.000000}%
\pgfsetfillcolor{currentfill}%
\pgfsetlinewidth{0.501875pt}%
\definecolor{currentstroke}{rgb}{0.000000,0.000000,0.000000}%
\pgfsetstrokecolor{currentstroke}%
\pgfsetdash{}{0pt}%
\pgfsys@defobject{currentmarker}{\pgfqpoint{0.000000in}{-0.055556in}}{\pgfqpoint{0.000000in}{0.000000in}}{%
\pgfpathmoveto{\pgfqpoint{0.000000in}{0.000000in}}%
\pgfpathlineto{\pgfqpoint{0.000000in}{-0.055556in}}%
\pgfusepath{stroke,fill}%
}%
\begin{pgfscope}%
\pgfsys@transformshift{4.595095in}{4.050417in}%
\pgfsys@useobject{currentmarker}{}%
\end{pgfscope}%
\end{pgfscope}%
\begin{pgfscope}%
\pgftext[x=4.595095in,y=2.295903in,,top]{\fontsize{11.000000}{13.200000}\selectfont 4}%
\end{pgfscope}%
\begin{pgfscope}%
\pgfsetbuttcap%
\pgfsetroundjoin%
\definecolor{currentfill}{rgb}{0.000000,0.000000,0.000000}%
\pgfsetfillcolor{currentfill}%
\pgfsetlinewidth{0.501875pt}%
\definecolor{currentstroke}{rgb}{0.000000,0.000000,0.000000}%
\pgfsetstrokecolor{currentstroke}%
\pgfsetdash{}{0pt}%
\pgfsys@defobject{currentmarker}{\pgfqpoint{0.000000in}{0.000000in}}{\pgfqpoint{0.000000in}{0.055556in}}{%
\pgfpathmoveto{\pgfqpoint{0.000000in}{0.000000in}}%
\pgfpathlineto{\pgfqpoint{0.000000in}{0.055556in}}%
\pgfusepath{stroke,fill}%
}%
\begin{pgfscope}%
\pgfsys@transformshift{4.884462in}{2.351458in}%
\pgfsys@useobject{currentmarker}{}%
\end{pgfscope}%
\end{pgfscope}%
\begin{pgfscope}%
\pgfsetbuttcap%
\pgfsetroundjoin%
\definecolor{currentfill}{rgb}{0.000000,0.000000,0.000000}%
\pgfsetfillcolor{currentfill}%
\pgfsetlinewidth{0.501875pt}%
\definecolor{currentstroke}{rgb}{0.000000,0.000000,0.000000}%
\pgfsetstrokecolor{currentstroke}%
\pgfsetdash{}{0pt}%
\pgfsys@defobject{currentmarker}{\pgfqpoint{0.000000in}{-0.055556in}}{\pgfqpoint{0.000000in}{0.000000in}}{%
\pgfpathmoveto{\pgfqpoint{0.000000in}{0.000000in}}%
\pgfpathlineto{\pgfqpoint{0.000000in}{-0.055556in}}%
\pgfusepath{stroke,fill}%
}%
\begin{pgfscope}%
\pgfsys@transformshift{4.884462in}{4.050417in}%
\pgfsys@useobject{currentmarker}{}%
\end{pgfscope}%
\end{pgfscope}%
\begin{pgfscope}%
\pgftext[x=4.884462in,y=2.295903in,,top]{\fontsize{11.000000}{13.200000}\selectfont 5}%
\end{pgfscope}%
\begin{pgfscope}%
\pgfsetbuttcap%
\pgfsetroundjoin%
\definecolor{currentfill}{rgb}{0.000000,0.000000,0.000000}%
\pgfsetfillcolor{currentfill}%
\pgfsetlinewidth{0.501875pt}%
\definecolor{currentstroke}{rgb}{0.000000,0.000000,0.000000}%
\pgfsetstrokecolor{currentstroke}%
\pgfsetdash{}{0pt}%
\pgfsys@defobject{currentmarker}{\pgfqpoint{0.000000in}{0.000000in}}{\pgfqpoint{0.000000in}{0.055556in}}{%
\pgfpathmoveto{\pgfqpoint{0.000000in}{0.000000in}}%
\pgfpathlineto{\pgfqpoint{0.000000in}{0.055556in}}%
\pgfusepath{stroke,fill}%
}%
\begin{pgfscope}%
\pgfsys@transformshift{5.173829in}{2.351458in}%
\pgfsys@useobject{currentmarker}{}%
\end{pgfscope}%
\end{pgfscope}%
\begin{pgfscope}%
\pgfsetbuttcap%
\pgfsetroundjoin%
\definecolor{currentfill}{rgb}{0.000000,0.000000,0.000000}%
\pgfsetfillcolor{currentfill}%
\pgfsetlinewidth{0.501875pt}%
\definecolor{currentstroke}{rgb}{0.000000,0.000000,0.000000}%
\pgfsetstrokecolor{currentstroke}%
\pgfsetdash{}{0pt}%
\pgfsys@defobject{currentmarker}{\pgfqpoint{0.000000in}{-0.055556in}}{\pgfqpoint{0.000000in}{0.000000in}}{%
\pgfpathmoveto{\pgfqpoint{0.000000in}{0.000000in}}%
\pgfpathlineto{\pgfqpoint{0.000000in}{-0.055556in}}%
\pgfusepath{stroke,fill}%
}%
\begin{pgfscope}%
\pgfsys@transformshift{5.173829in}{4.050417in}%
\pgfsys@useobject{currentmarker}{}%
\end{pgfscope}%
\end{pgfscope}%
\begin{pgfscope}%
\pgftext[x=5.173829in,y=2.295903in,,top]{\fontsize{11.000000}{13.200000}\selectfont 6}%
\end{pgfscope}%
\begin{pgfscope}%
\pgfsetbuttcap%
\pgfsetroundjoin%
\definecolor{currentfill}{rgb}{0.000000,0.000000,0.000000}%
\pgfsetfillcolor{currentfill}%
\pgfsetlinewidth{0.501875pt}%
\definecolor{currentstroke}{rgb}{0.000000,0.000000,0.000000}%
\pgfsetstrokecolor{currentstroke}%
\pgfsetdash{}{0pt}%
\pgfsys@defobject{currentmarker}{\pgfqpoint{0.000000in}{0.000000in}}{\pgfqpoint{0.000000in}{0.055556in}}{%
\pgfpathmoveto{\pgfqpoint{0.000000in}{0.000000in}}%
\pgfpathlineto{\pgfqpoint{0.000000in}{0.055556in}}%
\pgfusepath{stroke,fill}%
}%
\begin{pgfscope}%
\pgfsys@transformshift{5.463196in}{2.351458in}%
\pgfsys@useobject{currentmarker}{}%
\end{pgfscope}%
\end{pgfscope}%
\begin{pgfscope}%
\pgfsetbuttcap%
\pgfsetroundjoin%
\definecolor{currentfill}{rgb}{0.000000,0.000000,0.000000}%
\pgfsetfillcolor{currentfill}%
\pgfsetlinewidth{0.501875pt}%
\definecolor{currentstroke}{rgb}{0.000000,0.000000,0.000000}%
\pgfsetstrokecolor{currentstroke}%
\pgfsetdash{}{0pt}%
\pgfsys@defobject{currentmarker}{\pgfqpoint{0.000000in}{-0.055556in}}{\pgfqpoint{0.000000in}{0.000000in}}{%
\pgfpathmoveto{\pgfqpoint{0.000000in}{0.000000in}}%
\pgfpathlineto{\pgfqpoint{0.000000in}{-0.055556in}}%
\pgfusepath{stroke,fill}%
}%
\begin{pgfscope}%
\pgfsys@transformshift{5.463196in}{4.050417in}%
\pgfsys@useobject{currentmarker}{}%
\end{pgfscope}%
\end{pgfscope}%
\begin{pgfscope}%
\pgftext[x=5.463196in,y=2.295903in,,top]{\fontsize{11.000000}{13.200000}\selectfont 7}%
\end{pgfscope}%
\begin{pgfscope}%
\pgfsetbuttcap%
\pgfsetroundjoin%
\definecolor{currentfill}{rgb}{0.000000,0.000000,0.000000}%
\pgfsetfillcolor{currentfill}%
\pgfsetlinewidth{0.501875pt}%
\definecolor{currentstroke}{rgb}{0.000000,0.000000,0.000000}%
\pgfsetstrokecolor{currentstroke}%
\pgfsetdash{}{0pt}%
\pgfsys@defobject{currentmarker}{\pgfqpoint{0.000000in}{0.000000in}}{\pgfqpoint{0.055556in}{0.000000in}}{%
\pgfpathmoveto{\pgfqpoint{0.000000in}{0.000000in}}%
\pgfpathlineto{\pgfqpoint{0.055556in}{0.000000in}}%
\pgfusepath{stroke,fill}%
}%
\begin{pgfscope}%
\pgfsys@transformshift{3.221779in}{2.351458in}%
\pgfsys@useobject{currentmarker}{}%
\end{pgfscope}%
\end{pgfscope}%
\begin{pgfscope}%
\pgfsetbuttcap%
\pgfsetroundjoin%
\definecolor{currentfill}{rgb}{0.000000,0.000000,0.000000}%
\pgfsetfillcolor{currentfill}%
\pgfsetlinewidth{0.501875pt}%
\definecolor{currentstroke}{rgb}{0.000000,0.000000,0.000000}%
\pgfsetstrokecolor{currentstroke}%
\pgfsetdash{}{0pt}%
\pgfsys@defobject{currentmarker}{\pgfqpoint{-0.055556in}{0.000000in}}{\pgfqpoint{0.000000in}{0.000000in}}{%
\pgfpathmoveto{\pgfqpoint{0.000000in}{0.000000in}}%
\pgfpathlineto{\pgfqpoint{-0.055556in}{0.000000in}}%
\pgfusepath{stroke,fill}%
}%
\begin{pgfscope}%
\pgfsys@transformshift{5.545000in}{2.351458in}%
\pgfsys@useobject{currentmarker}{}%
\end{pgfscope}%
\end{pgfscope}%
\begin{pgfscope}%
\pgfsetbuttcap%
\pgfsetroundjoin%
\definecolor{currentfill}{rgb}{0.000000,0.000000,0.000000}%
\pgfsetfillcolor{currentfill}%
\pgfsetlinewidth{0.501875pt}%
\definecolor{currentstroke}{rgb}{0.000000,0.000000,0.000000}%
\pgfsetstrokecolor{currentstroke}%
\pgfsetdash{}{0pt}%
\pgfsys@defobject{currentmarker}{\pgfqpoint{0.000000in}{0.000000in}}{\pgfqpoint{0.055556in}{0.000000in}}{%
\pgfpathmoveto{\pgfqpoint{0.000000in}{0.000000in}}%
\pgfpathlineto{\pgfqpoint{0.055556in}{0.000000in}}%
\pgfusepath{stroke,fill}%
}%
\begin{pgfscope}%
\pgfsys@transformshift{3.221779in}{2.646929in}%
\pgfsys@useobject{currentmarker}{}%
\end{pgfscope}%
\end{pgfscope}%
\begin{pgfscope}%
\pgfsetbuttcap%
\pgfsetroundjoin%
\definecolor{currentfill}{rgb}{0.000000,0.000000,0.000000}%
\pgfsetfillcolor{currentfill}%
\pgfsetlinewidth{0.501875pt}%
\definecolor{currentstroke}{rgb}{0.000000,0.000000,0.000000}%
\pgfsetstrokecolor{currentstroke}%
\pgfsetdash{}{0pt}%
\pgfsys@defobject{currentmarker}{\pgfqpoint{-0.055556in}{0.000000in}}{\pgfqpoint{0.000000in}{0.000000in}}{%
\pgfpathmoveto{\pgfqpoint{0.000000in}{0.000000in}}%
\pgfpathlineto{\pgfqpoint{-0.055556in}{0.000000in}}%
\pgfusepath{stroke,fill}%
}%
\begin{pgfscope}%
\pgfsys@transformshift{5.545000in}{2.646929in}%
\pgfsys@useobject{currentmarker}{}%
\end{pgfscope}%
\end{pgfscope}%
\begin{pgfscope}%
\pgfsetbuttcap%
\pgfsetroundjoin%
\definecolor{currentfill}{rgb}{0.000000,0.000000,0.000000}%
\pgfsetfillcolor{currentfill}%
\pgfsetlinewidth{0.501875pt}%
\definecolor{currentstroke}{rgb}{0.000000,0.000000,0.000000}%
\pgfsetstrokecolor{currentstroke}%
\pgfsetdash{}{0pt}%
\pgfsys@defobject{currentmarker}{\pgfqpoint{0.000000in}{0.000000in}}{\pgfqpoint{0.055556in}{0.000000in}}{%
\pgfpathmoveto{\pgfqpoint{0.000000in}{0.000000in}}%
\pgfpathlineto{\pgfqpoint{0.055556in}{0.000000in}}%
\pgfusepath{stroke,fill}%
}%
\begin{pgfscope}%
\pgfsys@transformshift{3.221779in}{2.942400in}%
\pgfsys@useobject{currentmarker}{}%
\end{pgfscope}%
\end{pgfscope}%
\begin{pgfscope}%
\pgfsetbuttcap%
\pgfsetroundjoin%
\definecolor{currentfill}{rgb}{0.000000,0.000000,0.000000}%
\pgfsetfillcolor{currentfill}%
\pgfsetlinewidth{0.501875pt}%
\definecolor{currentstroke}{rgb}{0.000000,0.000000,0.000000}%
\pgfsetstrokecolor{currentstroke}%
\pgfsetdash{}{0pt}%
\pgfsys@defobject{currentmarker}{\pgfqpoint{-0.055556in}{0.000000in}}{\pgfqpoint{0.000000in}{0.000000in}}{%
\pgfpathmoveto{\pgfqpoint{0.000000in}{0.000000in}}%
\pgfpathlineto{\pgfqpoint{-0.055556in}{0.000000in}}%
\pgfusepath{stroke,fill}%
}%
\begin{pgfscope}%
\pgfsys@transformshift{5.545000in}{2.942400in}%
\pgfsys@useobject{currentmarker}{}%
\end{pgfscope}%
\end{pgfscope}%
\begin{pgfscope}%
\pgfsetbuttcap%
\pgfsetroundjoin%
\definecolor{currentfill}{rgb}{0.000000,0.000000,0.000000}%
\pgfsetfillcolor{currentfill}%
\pgfsetlinewidth{0.501875pt}%
\definecolor{currentstroke}{rgb}{0.000000,0.000000,0.000000}%
\pgfsetstrokecolor{currentstroke}%
\pgfsetdash{}{0pt}%
\pgfsys@defobject{currentmarker}{\pgfqpoint{0.000000in}{0.000000in}}{\pgfqpoint{0.055556in}{0.000000in}}{%
\pgfpathmoveto{\pgfqpoint{0.000000in}{0.000000in}}%
\pgfpathlineto{\pgfqpoint{0.055556in}{0.000000in}}%
\pgfusepath{stroke,fill}%
}%
\begin{pgfscope}%
\pgfsys@transformshift{3.221779in}{3.237871in}%
\pgfsys@useobject{currentmarker}{}%
\end{pgfscope}%
\end{pgfscope}%
\begin{pgfscope}%
\pgfsetbuttcap%
\pgfsetroundjoin%
\definecolor{currentfill}{rgb}{0.000000,0.000000,0.000000}%
\pgfsetfillcolor{currentfill}%
\pgfsetlinewidth{0.501875pt}%
\definecolor{currentstroke}{rgb}{0.000000,0.000000,0.000000}%
\pgfsetstrokecolor{currentstroke}%
\pgfsetdash{}{0pt}%
\pgfsys@defobject{currentmarker}{\pgfqpoint{-0.055556in}{0.000000in}}{\pgfqpoint{0.000000in}{0.000000in}}{%
\pgfpathmoveto{\pgfqpoint{0.000000in}{0.000000in}}%
\pgfpathlineto{\pgfqpoint{-0.055556in}{0.000000in}}%
\pgfusepath{stroke,fill}%
}%
\begin{pgfscope}%
\pgfsys@transformshift{5.545000in}{3.237871in}%
\pgfsys@useobject{currentmarker}{}%
\end{pgfscope}%
\end{pgfscope}%
\begin{pgfscope}%
\pgfsetbuttcap%
\pgfsetroundjoin%
\definecolor{currentfill}{rgb}{0.000000,0.000000,0.000000}%
\pgfsetfillcolor{currentfill}%
\pgfsetlinewidth{0.501875pt}%
\definecolor{currentstroke}{rgb}{0.000000,0.000000,0.000000}%
\pgfsetstrokecolor{currentstroke}%
\pgfsetdash{}{0pt}%
\pgfsys@defobject{currentmarker}{\pgfqpoint{0.000000in}{0.000000in}}{\pgfqpoint{0.055556in}{0.000000in}}{%
\pgfpathmoveto{\pgfqpoint{0.000000in}{0.000000in}}%
\pgfpathlineto{\pgfqpoint{0.055556in}{0.000000in}}%
\pgfusepath{stroke,fill}%
}%
\begin{pgfscope}%
\pgfsys@transformshift{3.221779in}{3.533342in}%
\pgfsys@useobject{currentmarker}{}%
\end{pgfscope}%
\end{pgfscope}%
\begin{pgfscope}%
\pgfsetbuttcap%
\pgfsetroundjoin%
\definecolor{currentfill}{rgb}{0.000000,0.000000,0.000000}%
\pgfsetfillcolor{currentfill}%
\pgfsetlinewidth{0.501875pt}%
\definecolor{currentstroke}{rgb}{0.000000,0.000000,0.000000}%
\pgfsetstrokecolor{currentstroke}%
\pgfsetdash{}{0pt}%
\pgfsys@defobject{currentmarker}{\pgfqpoint{-0.055556in}{0.000000in}}{\pgfqpoint{0.000000in}{0.000000in}}{%
\pgfpathmoveto{\pgfqpoint{0.000000in}{0.000000in}}%
\pgfpathlineto{\pgfqpoint{-0.055556in}{0.000000in}}%
\pgfusepath{stroke,fill}%
}%
\begin{pgfscope}%
\pgfsys@transformshift{5.545000in}{3.533342in}%
\pgfsys@useobject{currentmarker}{}%
\end{pgfscope}%
\end{pgfscope}%
\begin{pgfscope}%
\pgfsetbuttcap%
\pgfsetroundjoin%
\definecolor{currentfill}{rgb}{0.000000,0.000000,0.000000}%
\pgfsetfillcolor{currentfill}%
\pgfsetlinewidth{0.501875pt}%
\definecolor{currentstroke}{rgb}{0.000000,0.000000,0.000000}%
\pgfsetstrokecolor{currentstroke}%
\pgfsetdash{}{0pt}%
\pgfsys@defobject{currentmarker}{\pgfqpoint{0.000000in}{0.000000in}}{\pgfqpoint{0.055556in}{0.000000in}}{%
\pgfpathmoveto{\pgfqpoint{0.000000in}{0.000000in}}%
\pgfpathlineto{\pgfqpoint{0.055556in}{0.000000in}}%
\pgfusepath{stroke,fill}%
}%
\begin{pgfscope}%
\pgfsys@transformshift{3.221779in}{3.828813in}%
\pgfsys@useobject{currentmarker}{}%
\end{pgfscope}%
\end{pgfscope}%
\begin{pgfscope}%
\pgfsetbuttcap%
\pgfsetroundjoin%
\definecolor{currentfill}{rgb}{0.000000,0.000000,0.000000}%
\pgfsetfillcolor{currentfill}%
\pgfsetlinewidth{0.501875pt}%
\definecolor{currentstroke}{rgb}{0.000000,0.000000,0.000000}%
\pgfsetstrokecolor{currentstroke}%
\pgfsetdash{}{0pt}%
\pgfsys@defobject{currentmarker}{\pgfqpoint{-0.055556in}{0.000000in}}{\pgfqpoint{0.000000in}{0.000000in}}{%
\pgfpathmoveto{\pgfqpoint{0.000000in}{0.000000in}}%
\pgfpathlineto{\pgfqpoint{-0.055556in}{0.000000in}}%
\pgfusepath{stroke,fill}%
}%
\begin{pgfscope}%
\pgfsys@transformshift{5.545000in}{3.828813in}%
\pgfsys@useobject{currentmarker}{}%
\end{pgfscope}%
\end{pgfscope}%
\begin{pgfscope}%
\pgfsetbuttcap%
\pgfsetmiterjoin%
\definecolor{currentfill}{rgb}{1.000000,1.000000,1.000000}%
\pgfsetfillcolor{currentfill}%
\pgfsetlinewidth{0.000000pt}%
\definecolor{currentstroke}{rgb}{0.000000,0.000000,0.000000}%
\pgfsetstrokecolor{currentstroke}%
\pgfsetstrokeopacity{0.000000}%
\pgfsetdash{}{0pt}%
\pgfpathmoveto{\pgfqpoint{3.221779in}{0.652500in}}%
\pgfpathlineto{\pgfqpoint{5.545000in}{0.652500in}}%
\pgfpathlineto{\pgfqpoint{5.545000in}{2.351458in}}%
\pgfpathlineto{\pgfqpoint{3.221779in}{2.351458in}}%
\pgfpathclose%
\pgfusepath{fill}%
\end{pgfscope}%
\begin{pgfscope}%
\pgfpathrectangle{\pgfqpoint{3.221779in}{0.652500in}}{\pgfqpoint{2.323221in}{1.698958in}} %
\pgfusepath{clip}%
\pgfsetrectcap%
\pgfsetroundjoin%
\pgfsetlinewidth{1.003750pt}%
\definecolor{currentstroke}{rgb}{0.301961,0.607843,0.301961}%
\pgfsetstrokecolor{currentstroke}%
\pgfsetdash{}{0pt}%
\pgfpathmoveto{\pgfqpoint{3.437628in}{0.859561in}}%
\pgfpathlineto{\pgfqpoint{3.443308in}{0.863994in}}%
\pgfpathlineto{\pgfqpoint{3.448988in}{0.872544in}}%
\pgfpathlineto{\pgfqpoint{3.454669in}{0.884668in}}%
\pgfpathlineto{\pgfqpoint{3.466029in}{0.917127in}}%
\pgfpathlineto{\pgfqpoint{3.477390in}{0.957259in}}%
\pgfpathlineto{\pgfqpoint{3.494430in}{1.028680in}}%
\pgfpathlineto{\pgfqpoint{3.505791in}{1.084541in}}%
\pgfpathlineto{\pgfqpoint{3.522832in}{1.182240in}}%
\pgfpathlineto{\pgfqpoint{3.545553in}{1.331954in}}%
\pgfpathlineto{\pgfqpoint{3.579634in}{1.559493in}}%
\pgfpathlineto{\pgfqpoint{3.596675in}{1.661095in}}%
\pgfpathlineto{\pgfqpoint{3.613716in}{1.746883in}}%
\pgfpathlineto{\pgfqpoint{3.625076in}{1.793500in}}%
\pgfpathlineto{\pgfqpoint{3.636437in}{1.831248in}}%
\pgfpathlineto{\pgfqpoint{3.647797in}{1.859948in}}%
\pgfpathlineto{\pgfqpoint{3.659158in}{1.879411in}}%
\pgfpathlineto{\pgfqpoint{3.664838in}{1.885695in}}%
\pgfpathlineto{\pgfqpoint{3.670518in}{1.889786in}}%
\pgfpathlineto{\pgfqpoint{3.676198in}{1.891858in}}%
\pgfpathlineto{\pgfqpoint{3.681879in}{1.892167in}}%
\pgfpathlineto{\pgfqpoint{3.693239in}{1.888844in}}%
\pgfpathlineto{\pgfqpoint{3.710280in}{1.879327in}}%
\pgfpathlineto{\pgfqpoint{3.733001in}{1.864840in}}%
\pgfpathlineto{\pgfqpoint{3.744361in}{1.854788in}}%
\pgfpathlineto{\pgfqpoint{3.755722in}{1.841185in}}%
\pgfpathlineto{\pgfqpoint{3.767082in}{1.823795in}}%
\pgfpathlineto{\pgfqpoint{3.784123in}{1.792089in}}%
\pgfpathlineto{\pgfqpoint{3.818205in}{1.725485in}}%
\pgfpathlineto{\pgfqpoint{3.852286in}{1.662206in}}%
\pgfpathlineto{\pgfqpoint{3.869327in}{1.625188in}}%
\pgfpathlineto{\pgfqpoint{3.892048in}{1.570093in}}%
\pgfpathlineto{\pgfqpoint{3.931810in}{1.470299in}}%
\pgfpathlineto{\pgfqpoint{3.948850in}{1.435300in}}%
\pgfpathlineto{\pgfqpoint{3.965891in}{1.407173in}}%
\pgfpathlineto{\pgfqpoint{4.005653in}{1.349506in}}%
\pgfpathlineto{\pgfqpoint{4.022694in}{1.318795in}}%
\pgfpathlineto{\pgfqpoint{4.062455in}{1.244162in}}%
\pgfpathlineto{\pgfqpoint{4.085176in}{1.207261in}}%
\pgfpathlineto{\pgfqpoint{4.102217in}{1.183535in}}%
\pgfpathlineto{\pgfqpoint{4.113577in}{1.170520in}}%
\pgfpathlineto{\pgfqpoint{4.124938in}{1.160033in}}%
\pgfpathlineto{\pgfqpoint{4.147659in}{1.143406in}}%
\pgfpathlineto{\pgfqpoint{4.187421in}{1.114594in}}%
\pgfpathlineto{\pgfqpoint{4.244223in}{1.066042in}}%
\pgfpathlineto{\pgfqpoint{4.278305in}{1.040413in}}%
\pgfpathlineto{\pgfqpoint{4.374869in}{0.964146in}}%
\pgfpathlineto{\pgfqpoint{4.397590in}{0.950260in}}%
\pgfpathlineto{\pgfqpoint{4.414631in}{0.942622in}}%
\pgfpathlineto{\pgfqpoint{4.437352in}{0.935929in}}%
\pgfpathlineto{\pgfqpoint{4.460073in}{0.929343in}}%
\pgfpathlineto{\pgfqpoint{4.477113in}{0.922508in}}%
\pgfpathlineto{\pgfqpoint{4.499834in}{0.910846in}}%
\pgfpathlineto{\pgfqpoint{4.522555in}{0.896814in}}%
\pgfpathlineto{\pgfqpoint{4.556637in}{0.872648in}}%
\pgfpathlineto{\pgfqpoint{4.585038in}{0.852344in}}%
\pgfpathlineto{\pgfqpoint{4.602079in}{0.842467in}}%
\pgfpathlineto{\pgfqpoint{4.619120in}{0.834862in}}%
\pgfpathlineto{\pgfqpoint{4.653201in}{0.823188in}}%
\pgfpathlineto{\pgfqpoint{4.687283in}{0.813216in}}%
\pgfpathlineto{\pgfqpoint{4.772486in}{0.795094in}}%
\pgfpathlineto{\pgfqpoint{4.812248in}{0.787484in}}%
\pgfpathlineto{\pgfqpoint{4.937213in}{0.755675in}}%
\pgfpathlineto{\pgfqpoint{4.976975in}{0.743490in}}%
\pgfpathlineto{\pgfqpoint{4.988336in}{0.741833in}}%
\pgfpathlineto{\pgfqpoint{5.033778in}{0.739174in}}%
\pgfpathlineto{\pgfqpoint{5.067859in}{0.733054in}}%
\pgfpathlineto{\pgfqpoint{5.079220in}{0.732256in}}%
\pgfpathlineto{\pgfqpoint{5.079220in}{0.732256in}}%
\pgfusepath{stroke}%
\end{pgfscope}%
\begin{pgfscope}%
\pgfpathrectangle{\pgfqpoint{3.221779in}{0.652500in}}{\pgfqpoint{2.323221in}{1.698958in}} %
\pgfusepath{clip}%
\pgfsetbuttcap%
\pgfsetroundjoin%
\pgfsetlinewidth{1.003750pt}%
\definecolor{currentstroke}{rgb}{0.000000,0.000000,0.000000}%
\pgfsetstrokecolor{currentstroke}%
\pgfsetdash{{1.000000pt}{3.000000pt}}{0.000000pt}%
\pgfpathmoveto{\pgfqpoint{3.437628in}{0.652500in}}%
\pgfpathlineto{\pgfqpoint{3.437628in}{2.351458in}}%
\pgfusepath{stroke}%
\end{pgfscope}%
\begin{pgfscope}%
\pgfsetrectcap%
\pgfsetmiterjoin%
\pgfsetlinewidth{1.003750pt}%
\definecolor{currentstroke}{rgb}{0.000000,0.000000,0.000000}%
\pgfsetstrokecolor{currentstroke}%
\pgfsetdash{}{0pt}%
\pgfpathmoveto{\pgfqpoint{5.545000in}{0.652500in}}%
\pgfpathlineto{\pgfqpoint{5.545000in}{2.351458in}}%
\pgfusepath{stroke}%
\end{pgfscope}%
\begin{pgfscope}%
\pgfsetrectcap%
\pgfsetmiterjoin%
\pgfsetlinewidth{1.003750pt}%
\definecolor{currentstroke}{rgb}{0.000000,0.000000,0.000000}%
\pgfsetstrokecolor{currentstroke}%
\pgfsetdash{}{0pt}%
\pgfpathmoveto{\pgfqpoint{3.221779in}{0.652500in}}%
\pgfpathlineto{\pgfqpoint{3.221779in}{2.351458in}}%
\pgfusepath{stroke}%
\end{pgfscope}%
\begin{pgfscope}%
\pgfsetrectcap%
\pgfsetmiterjoin%
\pgfsetlinewidth{1.003750pt}%
\definecolor{currentstroke}{rgb}{0.000000,0.000000,0.000000}%
\pgfsetstrokecolor{currentstroke}%
\pgfsetdash{}{0pt}%
\pgfpathmoveto{\pgfqpoint{3.221779in}{2.351458in}}%
\pgfpathlineto{\pgfqpoint{5.545000in}{2.351458in}}%
\pgfusepath{stroke}%
\end{pgfscope}%
\begin{pgfscope}%
\pgfsetrectcap%
\pgfsetmiterjoin%
\pgfsetlinewidth{1.003750pt}%
\definecolor{currentstroke}{rgb}{0.000000,0.000000,0.000000}%
\pgfsetstrokecolor{currentstroke}%
\pgfsetdash{}{0pt}%
\pgfpathmoveto{\pgfqpoint{3.221779in}{0.652500in}}%
\pgfpathlineto{\pgfqpoint{5.545000in}{0.652500in}}%
\pgfusepath{stroke}%
\end{pgfscope}%
\begin{pgfscope}%
\pgfsetbuttcap%
\pgfsetroundjoin%
\definecolor{currentfill}{rgb}{0.000000,0.000000,0.000000}%
\pgfsetfillcolor{currentfill}%
\pgfsetlinewidth{0.501875pt}%
\definecolor{currentstroke}{rgb}{0.000000,0.000000,0.000000}%
\pgfsetstrokecolor{currentstroke}%
\pgfsetdash{}{0pt}%
\pgfsys@defobject{currentmarker}{\pgfqpoint{0.000000in}{0.000000in}}{\pgfqpoint{0.000000in}{0.055556in}}{%
\pgfpathmoveto{\pgfqpoint{0.000000in}{0.000000in}}%
\pgfpathlineto{\pgfqpoint{0.000000in}{0.055556in}}%
\pgfusepath{stroke,fill}%
}%
\begin{pgfscope}%
\pgfsys@transformshift{3.437628in}{0.652500in}%
\pgfsys@useobject{currentmarker}{}%
\end{pgfscope}%
\end{pgfscope}%
\begin{pgfscope}%
\pgfsetbuttcap%
\pgfsetroundjoin%
\definecolor{currentfill}{rgb}{0.000000,0.000000,0.000000}%
\pgfsetfillcolor{currentfill}%
\pgfsetlinewidth{0.501875pt}%
\definecolor{currentstroke}{rgb}{0.000000,0.000000,0.000000}%
\pgfsetstrokecolor{currentstroke}%
\pgfsetdash{}{0pt}%
\pgfsys@defobject{currentmarker}{\pgfqpoint{0.000000in}{-0.055556in}}{\pgfqpoint{0.000000in}{0.000000in}}{%
\pgfpathmoveto{\pgfqpoint{0.000000in}{0.000000in}}%
\pgfpathlineto{\pgfqpoint{0.000000in}{-0.055556in}}%
\pgfusepath{stroke,fill}%
}%
\begin{pgfscope}%
\pgfsys@transformshift{3.437628in}{2.351458in}%
\pgfsys@useobject{currentmarker}{}%
\end{pgfscope}%
\end{pgfscope}%
\begin{pgfscope}%
\pgftext[x=3.437628in,y=0.596944in,,top]{\fontsize{11.000000}{13.200000}\selectfont 0}%
\end{pgfscope}%
\begin{pgfscope}%
\pgfsetbuttcap%
\pgfsetroundjoin%
\definecolor{currentfill}{rgb}{0.000000,0.000000,0.000000}%
\pgfsetfillcolor{currentfill}%
\pgfsetlinewidth{0.501875pt}%
\definecolor{currentstroke}{rgb}{0.000000,0.000000,0.000000}%
\pgfsetstrokecolor{currentstroke}%
\pgfsetdash{}{0pt}%
\pgfsys@defobject{currentmarker}{\pgfqpoint{0.000000in}{0.000000in}}{\pgfqpoint{0.000000in}{0.055556in}}{%
\pgfpathmoveto{\pgfqpoint{0.000000in}{0.000000in}}%
\pgfpathlineto{\pgfqpoint{0.000000in}{0.055556in}}%
\pgfusepath{stroke,fill}%
}%
\begin{pgfscope}%
\pgfsys@transformshift{3.726995in}{0.652500in}%
\pgfsys@useobject{currentmarker}{}%
\end{pgfscope}%
\end{pgfscope}%
\begin{pgfscope}%
\pgfsetbuttcap%
\pgfsetroundjoin%
\definecolor{currentfill}{rgb}{0.000000,0.000000,0.000000}%
\pgfsetfillcolor{currentfill}%
\pgfsetlinewidth{0.501875pt}%
\definecolor{currentstroke}{rgb}{0.000000,0.000000,0.000000}%
\pgfsetstrokecolor{currentstroke}%
\pgfsetdash{}{0pt}%
\pgfsys@defobject{currentmarker}{\pgfqpoint{0.000000in}{-0.055556in}}{\pgfqpoint{0.000000in}{0.000000in}}{%
\pgfpathmoveto{\pgfqpoint{0.000000in}{0.000000in}}%
\pgfpathlineto{\pgfqpoint{0.000000in}{-0.055556in}}%
\pgfusepath{stroke,fill}%
}%
\begin{pgfscope}%
\pgfsys@transformshift{3.726995in}{2.351458in}%
\pgfsys@useobject{currentmarker}{}%
\end{pgfscope}%
\end{pgfscope}%
\begin{pgfscope}%
\pgftext[x=3.726995in,y=0.596944in,,top]{\fontsize{11.000000}{13.200000}\selectfont 1}%
\end{pgfscope}%
\begin{pgfscope}%
\pgfsetbuttcap%
\pgfsetroundjoin%
\definecolor{currentfill}{rgb}{0.000000,0.000000,0.000000}%
\pgfsetfillcolor{currentfill}%
\pgfsetlinewidth{0.501875pt}%
\definecolor{currentstroke}{rgb}{0.000000,0.000000,0.000000}%
\pgfsetstrokecolor{currentstroke}%
\pgfsetdash{}{0pt}%
\pgfsys@defobject{currentmarker}{\pgfqpoint{0.000000in}{0.000000in}}{\pgfqpoint{0.000000in}{0.055556in}}{%
\pgfpathmoveto{\pgfqpoint{0.000000in}{0.000000in}}%
\pgfpathlineto{\pgfqpoint{0.000000in}{0.055556in}}%
\pgfusepath{stroke,fill}%
}%
\begin{pgfscope}%
\pgfsys@transformshift{4.016362in}{0.652500in}%
\pgfsys@useobject{currentmarker}{}%
\end{pgfscope}%
\end{pgfscope}%
\begin{pgfscope}%
\pgfsetbuttcap%
\pgfsetroundjoin%
\definecolor{currentfill}{rgb}{0.000000,0.000000,0.000000}%
\pgfsetfillcolor{currentfill}%
\pgfsetlinewidth{0.501875pt}%
\definecolor{currentstroke}{rgb}{0.000000,0.000000,0.000000}%
\pgfsetstrokecolor{currentstroke}%
\pgfsetdash{}{0pt}%
\pgfsys@defobject{currentmarker}{\pgfqpoint{0.000000in}{-0.055556in}}{\pgfqpoint{0.000000in}{0.000000in}}{%
\pgfpathmoveto{\pgfqpoint{0.000000in}{0.000000in}}%
\pgfpathlineto{\pgfqpoint{0.000000in}{-0.055556in}}%
\pgfusepath{stroke,fill}%
}%
\begin{pgfscope}%
\pgfsys@transformshift{4.016362in}{2.351458in}%
\pgfsys@useobject{currentmarker}{}%
\end{pgfscope}%
\end{pgfscope}%
\begin{pgfscope}%
\pgftext[x=4.016362in,y=0.596944in,,top]{\fontsize{11.000000}{13.200000}\selectfont 2}%
\end{pgfscope}%
\begin{pgfscope}%
\pgfsetbuttcap%
\pgfsetroundjoin%
\definecolor{currentfill}{rgb}{0.000000,0.000000,0.000000}%
\pgfsetfillcolor{currentfill}%
\pgfsetlinewidth{0.501875pt}%
\definecolor{currentstroke}{rgb}{0.000000,0.000000,0.000000}%
\pgfsetstrokecolor{currentstroke}%
\pgfsetdash{}{0pt}%
\pgfsys@defobject{currentmarker}{\pgfqpoint{0.000000in}{0.000000in}}{\pgfqpoint{0.000000in}{0.055556in}}{%
\pgfpathmoveto{\pgfqpoint{0.000000in}{0.000000in}}%
\pgfpathlineto{\pgfqpoint{0.000000in}{0.055556in}}%
\pgfusepath{stroke,fill}%
}%
\begin{pgfscope}%
\pgfsys@transformshift{4.305728in}{0.652500in}%
\pgfsys@useobject{currentmarker}{}%
\end{pgfscope}%
\end{pgfscope}%
\begin{pgfscope}%
\pgfsetbuttcap%
\pgfsetroundjoin%
\definecolor{currentfill}{rgb}{0.000000,0.000000,0.000000}%
\pgfsetfillcolor{currentfill}%
\pgfsetlinewidth{0.501875pt}%
\definecolor{currentstroke}{rgb}{0.000000,0.000000,0.000000}%
\pgfsetstrokecolor{currentstroke}%
\pgfsetdash{}{0pt}%
\pgfsys@defobject{currentmarker}{\pgfqpoint{0.000000in}{-0.055556in}}{\pgfqpoint{0.000000in}{0.000000in}}{%
\pgfpathmoveto{\pgfqpoint{0.000000in}{0.000000in}}%
\pgfpathlineto{\pgfqpoint{0.000000in}{-0.055556in}}%
\pgfusepath{stroke,fill}%
}%
\begin{pgfscope}%
\pgfsys@transformshift{4.305728in}{2.351458in}%
\pgfsys@useobject{currentmarker}{}%
\end{pgfscope}%
\end{pgfscope}%
\begin{pgfscope}%
\pgftext[x=4.305728in,y=0.596944in,,top]{\fontsize{11.000000}{13.200000}\selectfont 3}%
\end{pgfscope}%
\begin{pgfscope}%
\pgfsetbuttcap%
\pgfsetroundjoin%
\definecolor{currentfill}{rgb}{0.000000,0.000000,0.000000}%
\pgfsetfillcolor{currentfill}%
\pgfsetlinewidth{0.501875pt}%
\definecolor{currentstroke}{rgb}{0.000000,0.000000,0.000000}%
\pgfsetstrokecolor{currentstroke}%
\pgfsetdash{}{0pt}%
\pgfsys@defobject{currentmarker}{\pgfqpoint{0.000000in}{0.000000in}}{\pgfqpoint{0.000000in}{0.055556in}}{%
\pgfpathmoveto{\pgfqpoint{0.000000in}{0.000000in}}%
\pgfpathlineto{\pgfqpoint{0.000000in}{0.055556in}}%
\pgfusepath{stroke,fill}%
}%
\begin{pgfscope}%
\pgfsys@transformshift{4.595095in}{0.652500in}%
\pgfsys@useobject{currentmarker}{}%
\end{pgfscope}%
\end{pgfscope}%
\begin{pgfscope}%
\pgfsetbuttcap%
\pgfsetroundjoin%
\definecolor{currentfill}{rgb}{0.000000,0.000000,0.000000}%
\pgfsetfillcolor{currentfill}%
\pgfsetlinewidth{0.501875pt}%
\definecolor{currentstroke}{rgb}{0.000000,0.000000,0.000000}%
\pgfsetstrokecolor{currentstroke}%
\pgfsetdash{}{0pt}%
\pgfsys@defobject{currentmarker}{\pgfqpoint{0.000000in}{-0.055556in}}{\pgfqpoint{0.000000in}{0.000000in}}{%
\pgfpathmoveto{\pgfqpoint{0.000000in}{0.000000in}}%
\pgfpathlineto{\pgfqpoint{0.000000in}{-0.055556in}}%
\pgfusepath{stroke,fill}%
}%
\begin{pgfscope}%
\pgfsys@transformshift{4.595095in}{2.351458in}%
\pgfsys@useobject{currentmarker}{}%
\end{pgfscope}%
\end{pgfscope}%
\begin{pgfscope}%
\pgftext[x=4.595095in,y=0.596944in,,top]{\fontsize{11.000000}{13.200000}\selectfont 4}%
\end{pgfscope}%
\begin{pgfscope}%
\pgfsetbuttcap%
\pgfsetroundjoin%
\definecolor{currentfill}{rgb}{0.000000,0.000000,0.000000}%
\pgfsetfillcolor{currentfill}%
\pgfsetlinewidth{0.501875pt}%
\definecolor{currentstroke}{rgb}{0.000000,0.000000,0.000000}%
\pgfsetstrokecolor{currentstroke}%
\pgfsetdash{}{0pt}%
\pgfsys@defobject{currentmarker}{\pgfqpoint{0.000000in}{0.000000in}}{\pgfqpoint{0.000000in}{0.055556in}}{%
\pgfpathmoveto{\pgfqpoint{0.000000in}{0.000000in}}%
\pgfpathlineto{\pgfqpoint{0.000000in}{0.055556in}}%
\pgfusepath{stroke,fill}%
}%
\begin{pgfscope}%
\pgfsys@transformshift{4.884462in}{0.652500in}%
\pgfsys@useobject{currentmarker}{}%
\end{pgfscope}%
\end{pgfscope}%
\begin{pgfscope}%
\pgfsetbuttcap%
\pgfsetroundjoin%
\definecolor{currentfill}{rgb}{0.000000,0.000000,0.000000}%
\pgfsetfillcolor{currentfill}%
\pgfsetlinewidth{0.501875pt}%
\definecolor{currentstroke}{rgb}{0.000000,0.000000,0.000000}%
\pgfsetstrokecolor{currentstroke}%
\pgfsetdash{}{0pt}%
\pgfsys@defobject{currentmarker}{\pgfqpoint{0.000000in}{-0.055556in}}{\pgfqpoint{0.000000in}{0.000000in}}{%
\pgfpathmoveto{\pgfqpoint{0.000000in}{0.000000in}}%
\pgfpathlineto{\pgfqpoint{0.000000in}{-0.055556in}}%
\pgfusepath{stroke,fill}%
}%
\begin{pgfscope}%
\pgfsys@transformshift{4.884462in}{2.351458in}%
\pgfsys@useobject{currentmarker}{}%
\end{pgfscope}%
\end{pgfscope}%
\begin{pgfscope}%
\pgftext[x=4.884462in,y=0.596944in,,top]{\fontsize{11.000000}{13.200000}\selectfont 5}%
\end{pgfscope}%
\begin{pgfscope}%
\pgfsetbuttcap%
\pgfsetroundjoin%
\definecolor{currentfill}{rgb}{0.000000,0.000000,0.000000}%
\pgfsetfillcolor{currentfill}%
\pgfsetlinewidth{0.501875pt}%
\definecolor{currentstroke}{rgb}{0.000000,0.000000,0.000000}%
\pgfsetstrokecolor{currentstroke}%
\pgfsetdash{}{0pt}%
\pgfsys@defobject{currentmarker}{\pgfqpoint{0.000000in}{0.000000in}}{\pgfqpoint{0.000000in}{0.055556in}}{%
\pgfpathmoveto{\pgfqpoint{0.000000in}{0.000000in}}%
\pgfpathlineto{\pgfqpoint{0.000000in}{0.055556in}}%
\pgfusepath{stroke,fill}%
}%
\begin{pgfscope}%
\pgfsys@transformshift{5.173829in}{0.652500in}%
\pgfsys@useobject{currentmarker}{}%
\end{pgfscope}%
\end{pgfscope}%
\begin{pgfscope}%
\pgfsetbuttcap%
\pgfsetroundjoin%
\definecolor{currentfill}{rgb}{0.000000,0.000000,0.000000}%
\pgfsetfillcolor{currentfill}%
\pgfsetlinewidth{0.501875pt}%
\definecolor{currentstroke}{rgb}{0.000000,0.000000,0.000000}%
\pgfsetstrokecolor{currentstroke}%
\pgfsetdash{}{0pt}%
\pgfsys@defobject{currentmarker}{\pgfqpoint{0.000000in}{-0.055556in}}{\pgfqpoint{0.000000in}{0.000000in}}{%
\pgfpathmoveto{\pgfqpoint{0.000000in}{0.000000in}}%
\pgfpathlineto{\pgfqpoint{0.000000in}{-0.055556in}}%
\pgfusepath{stroke,fill}%
}%
\begin{pgfscope}%
\pgfsys@transformshift{5.173829in}{2.351458in}%
\pgfsys@useobject{currentmarker}{}%
\end{pgfscope}%
\end{pgfscope}%
\begin{pgfscope}%
\pgftext[x=5.173829in,y=0.596944in,,top]{\fontsize{11.000000}{13.200000}\selectfont 6}%
\end{pgfscope}%
\begin{pgfscope}%
\pgfsetbuttcap%
\pgfsetroundjoin%
\definecolor{currentfill}{rgb}{0.000000,0.000000,0.000000}%
\pgfsetfillcolor{currentfill}%
\pgfsetlinewidth{0.501875pt}%
\definecolor{currentstroke}{rgb}{0.000000,0.000000,0.000000}%
\pgfsetstrokecolor{currentstroke}%
\pgfsetdash{}{0pt}%
\pgfsys@defobject{currentmarker}{\pgfqpoint{0.000000in}{0.000000in}}{\pgfqpoint{0.000000in}{0.055556in}}{%
\pgfpathmoveto{\pgfqpoint{0.000000in}{0.000000in}}%
\pgfpathlineto{\pgfqpoint{0.000000in}{0.055556in}}%
\pgfusepath{stroke,fill}%
}%
\begin{pgfscope}%
\pgfsys@transformshift{5.463196in}{0.652500in}%
\pgfsys@useobject{currentmarker}{}%
\end{pgfscope}%
\end{pgfscope}%
\begin{pgfscope}%
\pgfsetbuttcap%
\pgfsetroundjoin%
\definecolor{currentfill}{rgb}{0.000000,0.000000,0.000000}%
\pgfsetfillcolor{currentfill}%
\pgfsetlinewidth{0.501875pt}%
\definecolor{currentstroke}{rgb}{0.000000,0.000000,0.000000}%
\pgfsetstrokecolor{currentstroke}%
\pgfsetdash{}{0pt}%
\pgfsys@defobject{currentmarker}{\pgfqpoint{0.000000in}{-0.055556in}}{\pgfqpoint{0.000000in}{0.000000in}}{%
\pgfpathmoveto{\pgfqpoint{0.000000in}{0.000000in}}%
\pgfpathlineto{\pgfqpoint{0.000000in}{-0.055556in}}%
\pgfusepath{stroke,fill}%
}%
\begin{pgfscope}%
\pgfsys@transformshift{5.463196in}{2.351458in}%
\pgfsys@useobject{currentmarker}{}%
\end{pgfscope}%
\end{pgfscope}%
\begin{pgfscope}%
\pgftext[x=5.463196in,y=0.596944in,,top]{\fontsize{11.000000}{13.200000}\selectfont 7}%
\end{pgfscope}%
\begin{pgfscope}%
\pgftext[x=4.383389in,y=0.392315in,,top]{\fontsize{11.000000}{13.200000}\selectfont Time (ns)}%
\end{pgfscope}%
\begin{pgfscope}%
\pgfsetbuttcap%
\pgfsetroundjoin%
\definecolor{currentfill}{rgb}{0.000000,0.000000,0.000000}%
\pgfsetfillcolor{currentfill}%
\pgfsetlinewidth{0.501875pt}%
\definecolor{currentstroke}{rgb}{0.000000,0.000000,0.000000}%
\pgfsetstrokecolor{currentstroke}%
\pgfsetdash{}{0pt}%
\pgfsys@defobject{currentmarker}{\pgfqpoint{0.000000in}{0.000000in}}{\pgfqpoint{0.055556in}{0.000000in}}{%
\pgfpathmoveto{\pgfqpoint{0.000000in}{0.000000in}}%
\pgfpathlineto{\pgfqpoint{0.055556in}{0.000000in}}%
\pgfusepath{stroke,fill}%
}%
\begin{pgfscope}%
\pgfsys@transformshift{3.221779in}{0.673915in}%
\pgfsys@useobject{currentmarker}{}%
\end{pgfscope}%
\end{pgfscope}%
\begin{pgfscope}%
\pgfsetbuttcap%
\pgfsetroundjoin%
\definecolor{currentfill}{rgb}{0.000000,0.000000,0.000000}%
\pgfsetfillcolor{currentfill}%
\pgfsetlinewidth{0.501875pt}%
\definecolor{currentstroke}{rgb}{0.000000,0.000000,0.000000}%
\pgfsetstrokecolor{currentstroke}%
\pgfsetdash{}{0pt}%
\pgfsys@defobject{currentmarker}{\pgfqpoint{-0.055556in}{0.000000in}}{\pgfqpoint{0.000000in}{0.000000in}}{%
\pgfpathmoveto{\pgfqpoint{0.000000in}{0.000000in}}%
\pgfpathlineto{\pgfqpoint{-0.055556in}{0.000000in}}%
\pgfusepath{stroke,fill}%
}%
\begin{pgfscope}%
\pgfsys@transformshift{5.545000in}{0.673915in}%
\pgfsys@useobject{currentmarker}{}%
\end{pgfscope}%
\end{pgfscope}%
\begin{pgfscope}%
\pgfsetbuttcap%
\pgfsetroundjoin%
\definecolor{currentfill}{rgb}{0.000000,0.000000,0.000000}%
\pgfsetfillcolor{currentfill}%
\pgfsetlinewidth{0.501875pt}%
\definecolor{currentstroke}{rgb}{0.000000,0.000000,0.000000}%
\pgfsetstrokecolor{currentstroke}%
\pgfsetdash{}{0pt}%
\pgfsys@defobject{currentmarker}{\pgfqpoint{0.000000in}{0.000000in}}{\pgfqpoint{0.055556in}{0.000000in}}{%
\pgfpathmoveto{\pgfqpoint{0.000000in}{0.000000in}}%
\pgfpathlineto{\pgfqpoint{0.055556in}{0.000000in}}%
\pgfusepath{stroke,fill}%
}%
\begin{pgfscope}%
\pgfsys@transformshift{3.221779in}{1.030839in}%
\pgfsys@useobject{currentmarker}{}%
\end{pgfscope}%
\end{pgfscope}%
\begin{pgfscope}%
\pgfsetbuttcap%
\pgfsetroundjoin%
\definecolor{currentfill}{rgb}{0.000000,0.000000,0.000000}%
\pgfsetfillcolor{currentfill}%
\pgfsetlinewidth{0.501875pt}%
\definecolor{currentstroke}{rgb}{0.000000,0.000000,0.000000}%
\pgfsetstrokecolor{currentstroke}%
\pgfsetdash{}{0pt}%
\pgfsys@defobject{currentmarker}{\pgfqpoint{-0.055556in}{0.000000in}}{\pgfqpoint{0.000000in}{0.000000in}}{%
\pgfpathmoveto{\pgfqpoint{0.000000in}{0.000000in}}%
\pgfpathlineto{\pgfqpoint{-0.055556in}{0.000000in}}%
\pgfusepath{stroke,fill}%
}%
\begin{pgfscope}%
\pgfsys@transformshift{5.545000in}{1.030839in}%
\pgfsys@useobject{currentmarker}{}%
\end{pgfscope}%
\end{pgfscope}%
\begin{pgfscope}%
\pgfsetbuttcap%
\pgfsetroundjoin%
\definecolor{currentfill}{rgb}{0.000000,0.000000,0.000000}%
\pgfsetfillcolor{currentfill}%
\pgfsetlinewidth{0.501875pt}%
\definecolor{currentstroke}{rgb}{0.000000,0.000000,0.000000}%
\pgfsetstrokecolor{currentstroke}%
\pgfsetdash{}{0pt}%
\pgfsys@defobject{currentmarker}{\pgfqpoint{0.000000in}{0.000000in}}{\pgfqpoint{0.055556in}{0.000000in}}{%
\pgfpathmoveto{\pgfqpoint{0.000000in}{0.000000in}}%
\pgfpathlineto{\pgfqpoint{0.055556in}{0.000000in}}%
\pgfusepath{stroke,fill}%
}%
\begin{pgfscope}%
\pgfsys@transformshift{3.221779in}{1.387763in}%
\pgfsys@useobject{currentmarker}{}%
\end{pgfscope}%
\end{pgfscope}%
\begin{pgfscope}%
\pgfsetbuttcap%
\pgfsetroundjoin%
\definecolor{currentfill}{rgb}{0.000000,0.000000,0.000000}%
\pgfsetfillcolor{currentfill}%
\pgfsetlinewidth{0.501875pt}%
\definecolor{currentstroke}{rgb}{0.000000,0.000000,0.000000}%
\pgfsetstrokecolor{currentstroke}%
\pgfsetdash{}{0pt}%
\pgfsys@defobject{currentmarker}{\pgfqpoint{-0.055556in}{0.000000in}}{\pgfqpoint{0.000000in}{0.000000in}}{%
\pgfpathmoveto{\pgfqpoint{0.000000in}{0.000000in}}%
\pgfpathlineto{\pgfqpoint{-0.055556in}{0.000000in}}%
\pgfusepath{stroke,fill}%
}%
\begin{pgfscope}%
\pgfsys@transformshift{5.545000in}{1.387763in}%
\pgfsys@useobject{currentmarker}{}%
\end{pgfscope}%
\end{pgfscope}%
\begin{pgfscope}%
\pgfsetbuttcap%
\pgfsetroundjoin%
\definecolor{currentfill}{rgb}{0.000000,0.000000,0.000000}%
\pgfsetfillcolor{currentfill}%
\pgfsetlinewidth{0.501875pt}%
\definecolor{currentstroke}{rgb}{0.000000,0.000000,0.000000}%
\pgfsetstrokecolor{currentstroke}%
\pgfsetdash{}{0pt}%
\pgfsys@defobject{currentmarker}{\pgfqpoint{0.000000in}{0.000000in}}{\pgfqpoint{0.055556in}{0.000000in}}{%
\pgfpathmoveto{\pgfqpoint{0.000000in}{0.000000in}}%
\pgfpathlineto{\pgfqpoint{0.055556in}{0.000000in}}%
\pgfusepath{stroke,fill}%
}%
\begin{pgfscope}%
\pgfsys@transformshift{3.221779in}{1.744687in}%
\pgfsys@useobject{currentmarker}{}%
\end{pgfscope}%
\end{pgfscope}%
\begin{pgfscope}%
\pgfsetbuttcap%
\pgfsetroundjoin%
\definecolor{currentfill}{rgb}{0.000000,0.000000,0.000000}%
\pgfsetfillcolor{currentfill}%
\pgfsetlinewidth{0.501875pt}%
\definecolor{currentstroke}{rgb}{0.000000,0.000000,0.000000}%
\pgfsetstrokecolor{currentstroke}%
\pgfsetdash{}{0pt}%
\pgfsys@defobject{currentmarker}{\pgfqpoint{-0.055556in}{0.000000in}}{\pgfqpoint{0.000000in}{0.000000in}}{%
\pgfpathmoveto{\pgfqpoint{0.000000in}{0.000000in}}%
\pgfpathlineto{\pgfqpoint{-0.055556in}{0.000000in}}%
\pgfusepath{stroke,fill}%
}%
\begin{pgfscope}%
\pgfsys@transformshift{5.545000in}{1.744687in}%
\pgfsys@useobject{currentmarker}{}%
\end{pgfscope}%
\end{pgfscope}%
\begin{pgfscope}%
\pgfsetbuttcap%
\pgfsetroundjoin%
\definecolor{currentfill}{rgb}{0.000000,0.000000,0.000000}%
\pgfsetfillcolor{currentfill}%
\pgfsetlinewidth{0.501875pt}%
\definecolor{currentstroke}{rgb}{0.000000,0.000000,0.000000}%
\pgfsetstrokecolor{currentstroke}%
\pgfsetdash{}{0pt}%
\pgfsys@defobject{currentmarker}{\pgfqpoint{0.000000in}{0.000000in}}{\pgfqpoint{0.055556in}{0.000000in}}{%
\pgfpathmoveto{\pgfqpoint{0.000000in}{0.000000in}}%
\pgfpathlineto{\pgfqpoint{0.055556in}{0.000000in}}%
\pgfusepath{stroke,fill}%
}%
\begin{pgfscope}%
\pgfsys@transformshift{3.221779in}{2.101612in}%
\pgfsys@useobject{currentmarker}{}%
\end{pgfscope}%
\end{pgfscope}%
\begin{pgfscope}%
\pgfsetbuttcap%
\pgfsetroundjoin%
\definecolor{currentfill}{rgb}{0.000000,0.000000,0.000000}%
\pgfsetfillcolor{currentfill}%
\pgfsetlinewidth{0.501875pt}%
\definecolor{currentstroke}{rgb}{0.000000,0.000000,0.000000}%
\pgfsetstrokecolor{currentstroke}%
\pgfsetdash{}{0pt}%
\pgfsys@defobject{currentmarker}{\pgfqpoint{-0.055556in}{0.000000in}}{\pgfqpoint{0.000000in}{0.000000in}}{%
\pgfpathmoveto{\pgfqpoint{0.000000in}{0.000000in}}%
\pgfpathlineto{\pgfqpoint{-0.055556in}{0.000000in}}%
\pgfusepath{stroke,fill}%
}%
\begin{pgfscope}%
\pgfsys@transformshift{5.545000in}{2.101612in}%
\pgfsys@useobject{currentmarker}{}%
\end{pgfscope}%
\end{pgfscope}%
\begin{pgfscope}%
\pgftext[x=4.537222in,y=1.009424in,left,base]{\fontsize{11.000000}{13.200000}\selectfont \(\displaystyle \times100\)}%
\end{pgfscope}%
\end{pgfpicture}%
\makeatother%
\endgroup%

    \caption[Streaked brightness measurements for $\muup$ and ns timescales.]{Long (left) and short (right) duration electron bunch streaked pepperpot measurements showing, from top to bottom, the false-colour streak image, full-beam current, normalised \gls{rms} emittance, and normalised brightness. The dotted black lines indicate the start of the electron bunches. The red dotted lines in the emittance measurement indicate the expected emittance from the simulations discussed in Section~\ref{section:pepperpot_simulations}. The shaded regions indicate the standard deviation of ten 100-shot measurements.}
    \label{figure:streaks}
\end{figure}

\subsubsection{Measurement Resolution}
The temporal resolution of these measurements in dictated by a number of factors ;the finite \gls{psf} of the detector, the gradient or `slew' of the deflector voltages and the size of the beamlets on the detector.
The \gls{ccd} camera and lens used to take images of the detector is assumed to have a resolution much better than the point spread function of the phosphor screen.

The \gls{psf} of the \gls{mcp} and phosphor screen is approximately Gaussian with a standard deviation of \unit[35]{$\muup$m} and is measured by examining single electron events on the detector~\cite{speirs_electron_2017}.
The streak calibration images (such as Figure~\ref{figure:example_calibration}) can be used as a simple measure for the size of the detected beamlets, which are a convolution of the beamlets size at the detector and the \gls{psf} of the detector.
The detected beamlets tend to have a size with an \gls{rms} around \unit[183]{$\muup$m} although this does vary with the emittance of the beam.
The majority of the signal from an electron is therefore contained within an area of four times the standard deviation or \unit[732]{$\muup$m} and thus the long- and short-duration streaks shown in Figure~\ref{figure:streaks} can be sliced into 27 and 18 temporal slices which correspond to time resolutions of \unit[524]{ns} and \unit[247]{ps} respectively.

The temporal resolution can be enhanced by carefully maximising the size of the streak on the detector, by using faster electronics, or by using alternative methods such as rf-cavities or photoactivated switches which are able to achieve \unit[100]{fs} resolution~\cite{li_note:_2010,kassier_compact_2010,van_rens_theory_2018}.
Ensuring that the streak takes up the maximum available extent on the detector minimises the effect of the \gls{psf} on the temporal resolution with the downside that signal is now spread over a wider area thus reducing the \gls{snr}.

\subsection{Electron Flux}
The current electron flux of the \gls{caes} is not sufficient to perform the streaked emittance measurement in a single-shot and in some scenarios (such as ionisation pathways with poor coupling) the signal is so low that registration of measurements cannot be performed making those measurements impossible.

In order to perform these measurement with the same \gls{snr} would require 1000 times as many electrons per bunch, or $10^8$ electrons, since these measurements are composed of 100-shot averages.
Certain photocathode sources are able to achieve electron counts of order $10^8$ electrons per bunch and thus these techniques could perform equivalent measurements in a single shot and thus prove useful for beam diagnostics for these sources~\cite{li_note:_2010,musumeci_high_2010}.
There are a number of strategies available to the \gls{caes} to improve the number of electron per shot such as increasing the \gls{mot} density, optimising the ionisation pathways, and using a higher intensity blue ionisation laser.

\section{Conclusion}
This chapter has presented the first demonstration of time-resolved brightness measurements using streaked pepperpots.
The technique was developed to observe time-varying effects during the ionisation and extraction of electrons from a \gls{caes} but the results show that the emittance over the duration of the bunches is constant, indicating that there are no time-dependent effects on source emittance, such as hotter electrons produced through alternative ionisation pathways like multi-photon ionisation~\cite{speirs_identification_2017}.
The measured emittance of the bunches correlates well with the results of simulation and theory with the below-threshold bunch measurements being limited by the minimum resolvable emittance of the apparatus.

This technique has promise for use in a wide range of charged particle beam sources and is widely applicable to a range of scenarios.
With a sufficiently high-current source, such as photocathode electron sources, this technique can also be used to perform measurements in a single-shot.
With the next generation of \gls{caeis} currently under development this technique could be refined with less constrained apparatus, provide interesting insights into the ionisation processes involved and enhance work on achieving \gls{ued}.
Time-resolved brightness measurements through the streaking of one-dimensional pepperpots have the potential to provide information on the behaviour of many charged particle sources, allow for a better understanding of the physics involved and suggest pathways towards improvement to these sources.
