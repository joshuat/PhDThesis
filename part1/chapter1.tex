\chapter{Introduction}

Laser frequency stabilisation is an essential tool for atomic physics experiments, without it experiments involving \glspl{bec}, atomic clocks and many more would not be possible~\cite{anderson_observation_1995,ye_quantum_2008}.
There are a plethora of techniques available for laser frequency stabilisation each with numerous advantages and disadvantages.

\Gls{ps} is one such technique that will be discussed in detail here~\cite{demtroder_laser_2003}.
\Gls{ps} was first described by Wieman and H\"anch in 1976 as, ``...a sensitive new method of Doppler-free spectroscopy, monitoring the nonlinear interaction of two monochromatic laser beams in an absorbing gas via changes in light polarisation."~\cite{wieman_doppler-free_1976}
This chapter provides an overview of laser frequency stabilisation, a detailed discussion of the physics of \gls{ps} followed by details on the implementation of measurement of high bandwidth frequency stabilisation using \gls{ps}.

\section{Laser Frequency Stabilisation}

Laser frequency stabilisation describes a number of techniques that are used to reduce the temporal frequency spread of a laser's frequency.
Typically these techniques use some reference to measure frequency deviation from a given frequency and provide negative feedback to the laser, using a servo system, to keep it at the target frequency.

The desirable traits of a laser frequency stabilization scheme include the ability to stabilize to an absolute atomic reference, absence of frequency or amplitude modulation, high bandwidth to achieve low spectral linewidth, low complexity and low cost.
Techniques for stabilization include \gls*{sa}~\cite{maguire_theoretical_2006, haroche_theory_1972, preston_doppler-free_1996}, which can reduce \gls*{ecdl} linewidth below \unit[100]{kHz}~\cite{cuneo_optically_1994, saliba_linewidths_2009}, through to elaborate experiments involving extremely high-finesse optical cavities that are able to achieve sub-hertz linewidth using the \gls*{pdh} technique \cite{ludlow_compact_2007}.
Saturated absorption spectroscopy, \gls*{davll} \cite{corwin_frequency-stabilized_1998,millett-sikking_davll_2007}, \gls*{mts} \cite{shirley_modulation_1982, mccarron_modulation_2008,xiang-hui_ultra-stable_2009} and Sagnac interferometry \cite{robins_Interferometric_2002,jundt_non-linear_2003} lock to atomic references.
Optical cavity techniques such as \gls*{pdh} \cite{drever_laser_1983} may require a secondary lock to maintain absolute frequency stability.
Some locking techniques are limited by the rate of evolution of the atomic states, which are constrained by the lifetime of the excited state.
\Glspl{ecdl} subjected to mechanical vibrations will experience frequency noise as the alignment of the light coupled back into the diode off the grating varies.
Vibrations can also affect the alignment, and thus transmitted power, of laser through optical fibres, optical isolators and through apertures.

\section{Feedback Methods}

\section{Noise}
Thermal noise can affect the alignment of optics, the efficiency and polarisation of light transmitted through fibres, atomic vapour cell opacity and the intensity and frequency of light emitted from laser diodes, not to mention mode hops.

Noise in the electronic environment can also cause frequency instability. Noise on the power supply to the laser diode affects the intensity and frequency of the light emitted.

In certain applications {\color{red}(such as?? imaging?)} intensity noise can be as much as a drawback as frequency noise and with some frequency discrimination methods intensity noise is interpreted as frequency noise which the servo would then attempt to correct for thus inducing frequency noise.

Feedback to laser systems typically takes the form of modulation of either the current supply to the diode or the voltage to a piezo that controls the angle of the grating in an \gls{ecdl}.
Temperature feedback has also been used to maintain frequency stability {\color{red}[find that paper again]}.
The bandwidth of current feedback tends to range from \unit[0]{Hz} to MHz~\cite{ludlow_compact_2007}{\color{red}[cite my paper here?]}.
Piezo response is significantly slower ranging from \unit[0]{Hz} to \unit[100]{kHz}.

{\color{red}Frequency ranges of noise.}
{\color{red}More detail? More References on noise in general.}

\section{Stabilisation Techniques}
\subsection{Saturated Absorption Spectroscopy}
\Gls{sa} is a simple and common technique for laser frequency stabilisation~\cite{demtroder_laser_2003}.
 
\begin{figure}
\includegraphics[width=\linewidth]{part1/Figs/SatAbs.pdf}
\caption{Saturated absorption spectroscopy.}
\label{figure:satabs}
\end{figure}

\subsection{Polarisation Spectroscopy}

Summary of PS.

Pol Spec developments

It has been shown previously that \gls*{ps} can be used to reduce the linewidth of a distributed feedback diode from \unit[2]{MHz} to \unit[20]{kHz}~\cite{torii_laser-phase_2012} and of \glspl*{ecdl} to \unit[65]{kHz}
~\cite{yoshikawa_frequency_2003}.

Balanced polarimeter.\cite{pearman_polarization_2002,yoshikawa_frequency_2003}

Bi-polarisation spectroscopy.\cite{tiwari_laser_2006}


\subsection{Pound Drever Hall}

The \gls{pdh} technique is the standard for laser frequency linewidth reduction.\cite{drever_laser_1983}
\gls{pdh} uses an optical cavity as a frequency reference and an \gls{eom} to modulate the the light incident to the cavity.

More detail. Some maths. What linewidth can it achieve?\cite{ludlow_compact_2007}

\begin{figure}
\centering
\includegraphics[width=\linewidth]{part1/Figs/pdh.jpg}
\caption{A beautiful \gls{pdh} schematic.}
\end{figure}
\subsubsection{Dichroic Atomic Vapour Laser Lock}
\Gls{davll} works by....

It can achieve linewidths of....

\subsection{Modulation Transfer Spectroscopy}
\Gls{mts}...

It can achieve linewidths of...\cite{negnevitsky_wideband_2013}

\subsection{Other Techniques}

Any others?


