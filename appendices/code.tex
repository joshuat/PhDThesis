\chapter{Code}\label{appendix:code}

The code presented in this appendix was written for use with Python 3.

\section{One-Dimensional Pepperpot Simulation}\label{section:1d_code}

This code was used to simulate the behaviour of the electron beam as the emittance is measured with a pepperpot mask.
The results of the simulations performed with this code confirmed the accuracy of the corrections applied to the emittance measurements.
Bunch1d.py contains the core simulator code, Mask1D.py contains functions for applying pepperpot masks to the beam, and SimScript1D.py contains simulations of a number of different scenarios.
These simulations are discussed in Section~\ref{section:pepperpot_simulations}.

\subsection{Bunch1D.py}
\lstinputlisting[language=Python,basicstyle=\tiny\ttfamily]{appendices/code/Bunch1D.py}

\subsection{Mask1D.py}
\lstinputlisting[language=Python,basicstyle=\tiny\ttfamily]{appendices/code/Mask1D.py}

\subsection{SimScript1D.py}
\lstinputlisting[language=Python,basicstyle=\tiny\ttfamily]{appendices/code/SimScript1D.py}

\newpage
\section{Two-Dimensional Quadrupole and Pepperpot Simulations}

This code was used in Sections~\ref{section:quadrupole} and \ref{section:pepperpot_simulations}.
This code simulates charged particle beams, performing particle tracing, magnetic field interactions and optional particle-particle interactions.
ElectronLens.py contains the class ElectronBunch which performs the beam simulation.
MagneticFields.py contains a number of functions for the construction of magnetic fields for use with the simulation.
EmittanceSim.py simulates the emittance measurements for Section~\ref{section:pepperpot_simulations} but the code shown in Section~\ref{section:1d_code} was preferable simply due to the time taken to perform equivalent simulations.

\subsection{ElectronLens.py}
\lstinputlisting[language=Python,basicstyle=\tiny\ttfamily]{appendices/code/ElectronLens.py}

\subsection{MagneticFields.py}
\lstinputlisting[language=Python,basicstyle=\tiny\ttfamily]{appendices/code/MagneticFields.py}

\subsection{EmittanceSim.py}
\lstinputlisting[language=Python,basicstyle=\tiny\ttfamily]{appendices/code/EmittanceSim.py}


\newpage
\section{Image Registration}

This code was used to perform the image processing and registration for the diffraction measurements in Chapter~\ref{chapter:diffraction} and the brightness measurements in Chapter~\ref{chapter:brightness}.

\subsection{ImageSet.py}
\lstinputlisting[language=Python,basicstyle=\tiny\ttfamily]{appendices/code/ImageSet.py}

\newpage
\section{Utility Code}

Emittance.py was used for both the brightness measurements and simulations for Chapter~\ref{chapter:brightness}.
Fitting.py was used by a number of the codes in this appendix to fit functions to data.

\subsection{Emittance.py}
\lstinputlisting[language=Python,basicstyle=\tiny\ttfamily]{appendices/code/Emittance.py}

\subsection{Fitting.py}
\lstinputlisting[language=Python,basicstyle=\tiny\ttfamily]{appendices/code/Fitting.py}