\chapter{Heterodyne Beatnote of Two Lasers}\label{appendix:heterodyne}

In Chapter~\ref{chapter:polspec}, the heterodyne beatnote between two independent lasers was used to determine the laser linewidth as a diagnostic of frequency stabilisation using polarisation spectroscopy.
Although the technique is commonly used for similar applications, a detailed analysis has been missing.
This appendix analyses the heterodyne beatnote spectrum in detail and shows how the spectral width is related to the individual laser linewidths.

Consider two independent lasers with Gaussian spectral distributions such that the Fourier transform of the electric field is
\begin{align}
\mathcal{F}\{E_i\} = \exp\left(-\frac{(f_i-f_0)^2}{2\sigma_i^2}\right),
\end{align}
where the lasers have a spectral width of $\sigma_i$ centred at frequency $f_i$.

Ignoring the centre frequency $f_0$ for now,
\begin{align}
E_i &= \int_{-\infty}^{\infty} \exp\left(-\frac{f_i^2}{2\sigma_i^2}\right) \exp(2\pi i f_i t) \,df_i\notag\\
&= \int_{-\infty}^{\infty} \exp\left(-\frac{f_i^2}{2\sigma_i^2}\right) \big(\cos(2\pi f_i t) + i\sin(2\pi f_i t)\big) \,df_i
\end{align}
The integrand with the sine is odd and will evaluate to zero, so we can write
\begin{align}
E_i &= \int_{-\infty}^{\infty} \exp\left(-\frac{f^2}{2\sigma_i^2}\right)\cos(2\pi f t) \,df\notag\\
&= \sigma_i\sqrt{2 \pi} e^{-2 \pi^2t^2 \sigma_i^2}\notag\\
&= \sigma_i\sqrt{2 \pi} e^{-2 \pi^2t^2 \sigma_i^2}\,e^{-2\pi i t f_0},
\end{align}
using the Fourier transform shift property.
%According to Reference~\cite{abramowitz_handbook_2012} (page 302, equation 7.4.6),
%\begin{align}
%\int_{-\infty}^{\infty}e^{-ax^2}\cos(2\pi k x)\,dx = \sqrt{\frac{\pi}{a}} e^{-\pi^2k^2/a}
%\end{align}

%So we get,
%\begin{align}
%E_i &= \sigma_i\sqrt{2 \pi} e^{-2 \pi^2t^2 \sigma_i^2}
%\end{align}

%Using the Fourier transform shift property we can put $f_0$ back in.
%\begin{align}
%E_i &= \sigma_i\sqrt{2 \pi} e^{-2 \pi^2t^2 \sigma_i^2}\,e^{-2\pi i t f_0}
%\end{align}

The detector measures the power of the lasers which is given by
\begin{align}
P_{detector}(t) &= I(t) \, A = \frac{cA\epsilon_0}{2}|E|^2
\end{align}
where $A$ is the area of the detector.

The photodetector is a square-law detector; that is the output signal is proportional to the square of the input signal.
Hence the power spectral density (PSD) displayed on a conventional RF spectrum analyser will be
\begin{align}
\mathcal{F}\{P_{elec}\} &\propto \mathcal{F}\{\left[P_{detector}\right]^2\}\notag\\
&= \mathcal{F}\left\{\left[\frac{cA\epsilon_0}{2}|E|^2\right]^2\right\}\notag\\
&= \mathcal{F}\left\{\left[\frac{cA\epsilon_0}{2}|\vec{E_1}+\vec{E_2}|^2\right]^2\right\}
\end{align}
\begin{align}
|\vec{E_1}+\vec{E_2}|^2 &= (\vec{E_1}+\vec{E_2})(\vec{E_1}+\vec{E_2})^*\notag\\
&= \sqrt{2\pi} \Bigg(\sigma_1 e^{-2\pi^2t^2\sigma_1^2-2\pi i t f_1} + \sigma_2 e^{-2\pi^2t^2\sigma_2^2-2\pi i t f_2}\Bigg)\Bigg(\sigma_1 e^{-2\pi^2t^2\sigma_1^2+2\pi i t f_1} + \sigma_2 e^{-2\pi^2t^2\sigma_2^2+2\pi i t f_2}\Bigg)\notag\\
&= \sqrt{2\pi}\Bigg(\sigma_1^2 e^{-4\pi^2t^2\sigma_1^2} + \sigma_2^2 e^{-4\pi^2t^2\sigma_2^2} + \sigma_1\sigma_2 e^{-2\pi^2t^2(\sigma_1^2+\sigma_2^2)}\left\{e^{-2\pi i t (f_1-f_2)} + e^{+2\pi i t (f_1-f_2)} \right\}\Bigg)\notag\\
&= \sqrt{2\pi}\Bigg(\sigma_1^2 e^{-4\pi^2t^2\sigma_1^2} + \sigma_2^2 e^{-4\pi^2t^2\sigma_2^2} + 2\sigma_1\sigma_2 e^{-2\pi^2t^2(\sigma_1^2+\sigma_2^2)}\cos\big[2\pi t (f_1-f_2)\big] \Bigg)
\end{align}
\begin{align}
\mathcal{F}\{\left[P_{detector}\right]^2\} &= \frac{c^2A^2\epsilon_0^2 \pi}{2}\mathcal{F}\left\{\left[\sigma_1^2 e^{-4\pi^2t^2\sigma_1^2} + \sigma_2^2 e^{-4\pi^2t^2\sigma_2^2} + 2\sigma_1\sigma_2 e^{-2\pi^2t^2(\sigma_1^2+\sigma_2^2)}\cos\big[2\pi t (f_1-f_2)\big] \right]^2\right\}\notag\\
&= \frac{c^2A^2\epsilon_0^2 \pi}{2}\mathcal{F}\Bigg\{\sigma_1^4 e^{-8\pi^2t^2\sigma_1^2} + \sigma_2^4 e^{-8\pi^2t^2\sigma_2^2} + 2\sigma_1^2\sigma_2^2 e^{-4\pi^2t^2(\sigma_1^2+\sigma_2^2)}\left(\cos\big[4\pi t (f_1-f_2)\big]+1\right)\notag\\
&\quad\quad\quad\quad\quad\quad + 2\sigma_1^2\sigma_2^2 e^{-4\pi^2t^2(\sigma_1^2+\sigma_2^2)} + 4\sigma_1^3\sigma_2 e^{-2\pi^2t^2(3\sigma_1^2+\sigma_2^2)}\cos\big[2\pi t (f_1-f_2)\big] \notag\\
&\quad\quad\quad\quad\quad\quad + 4\sigma_1\sigma_2^3e^{-2\pi^2t^2(\sigma_1^2+3\sigma_2^2)}\cos\big[2\pi t (f_1-f_2)\big] \Bigg\}\notag\\
&= \frac{c^2A^2\epsilon_0^2 \pi}{2}\Bigg[\frac{\sigma_1^3}{\sqrt{8\pi}} e^{-f^2/8\sigma_1^2} + \frac{\sigma_2^3}{\sqrt{8\pi}} e^{-f^2/8\sigma_2^2} + \notag\\
&\quad\quad\quad\quad\quad\quad \frac{\sigma_1^2\sigma_2^2}{\sqrt{4\pi(\sigma_1^2+\sigma_2^2)}}\Bigg(e^{-(f-2(f_1-f_2))^2/4(\sigma_1^2+\sigma_2^2)} + e^{-(f+2(f_1-f_2))^2/4(\sigma_1^2+\sigma_2^2)}+\notag\\
&\quad\quad\quad\quad\quad\quad  \quad\quad\quad\quad\quad\quad\quad\quad\quad 4e^{-f^2/4(\sigma_1^2+\sigma_2^2)} \Bigg) +\notag\\
&\quad\quad\quad\quad\quad\quad  \frac{2\sigma_1^3\sigma_2}{\sqrt{2\pi(3\sigma_1^2+\sigma_2^2)}}\left(e^{-(f-(f_1-f_2))^2/2(3\sigma_1^2+\sigma_2^2)}+e^{-(f+(f_1-f_2))^2/2(3\sigma_1^2+\sigma_2^2)}\right)\notag\\
&\quad\quad\quad\quad\quad\quad  \frac{2\sigma_1\sigma_2^3}{\sqrt{2\pi(\sigma_1^2+3\sigma_2^2)}}\left(e^{-(f-(f_1-f_2))^2/2(\sigma_1^2+3\sigma_2^2)}+e^{-(f+(f_1-f_2))^2/2(\sigma_1^2+3\sigma_2^2)}\right)\Bigg]
\end{align}

Now if we let $\Delta=f_1-f_2$ and since the spectrum analyser doesn't know the difference between $\pm f$;
\begin{align}
\mathcal{F}\{P_{elec}\} &= \frac{c^2A^2\epsilon_0^2 \pi}{2}\Bigg[\frac{\sigma_1^3}{\sqrt{8\pi}} e^{-f^2/8\sigma_1^2} + \frac{\sigma_2^3}{\sqrt{8\pi}} e^{-f^2/8\sigma_2^2} + \notag\\
&\quad\quad\quad\quad\quad\quad \frac{\sigma_1^2\sigma_2^2}{\sqrt{\pi(\sigma_1^2+\sigma_2^2)}}\left(e^{-(f-2\Delta)^2/4(\sigma_1^2+\sigma_2^2)} + 2e^{-f^2/4(\sigma_1^2+\sigma_2^2)} \right) +\notag\\
&\quad\quad\quad\quad\quad\quad  \frac{4\sigma_1^3\sigma_2}{\sqrt{2\pi(3\sigma_1^2+\sigma_2^2)}}e^{-(f-\Delta)^2/2(3\sigma_1^2+\sigma_2^2)}+\frac{4\sigma_1\sigma_2^3}{\sqrt{2\pi(\sigma_1^2+3\sigma_2^2)}}e^{-(f-\Delta)^2/2(\sigma_1^2+3\sigma_2^2)}\Bigg]
\end{align}

If we have two identical but uncorrelated lasers then $\sigma=\sigma_1=\sigma_2$.
\begin{align}
\mathcal{F}\{P_{elec}\} &= \frac{c^2A^2\epsilon_0^2 \pi}{2}\Bigg[\frac{\sigma^3}{\sqrt{2\pi}} e^{-f^2/8\sigma^2} + \frac{\sigma^3}{\sqrt{2\pi}}\left(e^{-(f-2\Delta)^2/8\sigma^2} + 2e^{-f^2/8\sigma^2} \right) + \frac{4\sigma^3}{\sqrt{2\pi}}e^{-(f-\Delta)^2/8\sigma^2}\Bigg]\notag\\
&= \frac{c^2A^2\epsilon_0^2 \pi}{2}\frac{\sigma^3}{\sqrt{2\pi}}\Bigg[3e^{-f^2/8\sigma^2} + e^{-(f-2\Delta)^2/8\sigma^2} + 4e^{-(f-\Delta)^2/8\sigma^2}\Bigg]\notag\\
&= \frac{c^2A^2\epsilon_0^2 \pi}{2}\frac{\sigma^3}{\sqrt{2\pi}}\Bigg[3e^{-f^2/2(2\sigma)^2} + e^{-(f-2\Delta)^2/2(2\sigma)^2} + 4e^{-(f-\Delta)^2/2(2\sigma)^2}\Bigg]
\end{align}

Thus the spectrum analyser displays a beatnote located at $f=\Delta$ with a RMS width $2\sigma$ and FWHM of $2\sigma2\sqrt{2\log_e{2}}=2\sigma\times2.35$.
