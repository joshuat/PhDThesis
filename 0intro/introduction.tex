 \chapter{Introduction}
 
\pagenumbering{arabic}
\setcounter{page}{1}

Molecular imaging has provided science with great advances, such as the determination of the helical structure of DNA in 1953~\cite{watson_molecular_1953}, and the discovery of the structure of myoglobin and haemoglobin in 1962~\cite{kendrew_x-ray_1957}.
The \gls{caeis} at the University of Melbourne is an ongoing project that aims to develop a source capable of achieving the `holy grail' of scientific imaging, the \emph{molecular movie}~\cite{dwyer_femtosecond_2006,sciaini_femtosecond_2011}, see below.
A \gls{caeis} also has enormous potential for ion beam technologies such \glspl{fib}~\cite{mcclelland_bright_2016}, ion microscopes~\cite{knuffman_cold_2013}, and as a source for accelerators such as synchrotrons and \glspl{xfel}~\cite{van_oudheusden_electron_2007,zhu_future_2015,mcculloch_cold_2016}.
Depending on the context a \gls{caeis} is also sometimes referred to as a \gls{caes} or \gls{cais}.

\section{Making the `Molecular Movie'}

The ability to observe molecular dynamics at an atomic spatial and temporal scale could provide great insight into some of the basic processes of biology and chemistry.
The fabled `molecular movie' refers to capturing the atomic motions of molecular systems with atomic-scale spatial resolution as they undergo some transition such as a chemical reaction, protein folding, melting, or even just atomic vibrations.
Molecular movies could provide greater understanding of important biological reactions such as photosynthesis or oxygen transport by haemoglobin.

One of the stepping stones towards molecular movies is single-shot imaging of non-crystalline molecules which would allow structural determination of membrane proteins, an essential step in rational drug design~\cite{hardy_atomic_1987, barrett_discovering_1999, pinto_influenza_1992}.
The majority of imaging performed to date has been on crystalline targets and the related imaging techniques are much better developed than those for non-crystalline targets.
One of the main techniques in structural determination of crystalline targets is electron diffraction which, along with X-ray and neutron methods~\cite{cullity_elements_2001,bacon_x-ray_2013}, has been utilised in structural determination of many materials with atomic resolution.
Continuing research into real-space and diffraction imaging techniques is aiding in the understanding of a growing range of samples as well as providing new types of information on the samples under investigation.

In the past the greatest success with structural determination has been achieved with crystallography, limited to samples that can be crystallised.
There are a large number of biological proteins of interest, particularly membrane proteins, that cannot be crystallised and developments in ultrafast imaging techniques are providing a route towards structural determination without crystallisation~\cite{dauter_current_2006,levitt_nature_2009}.
Cryogenic electron microscopy is also making headway in this area but is unable to measure dynamics~\cite{henderson_model_1990,zhou_towards_2008}.
Ultrafast imaging techniques are a relatively recent development and they provide the opportunity to study electronic and atomic dynamics.
Ultrafast techniques allow for imaging to be completed before damage to the sample from the illumination makes imaging impossible~\cite{gaffney_imaging_2007,barty_ultrafast_2008,miao_beyond_2015}.
Both X-ray and electron imaging techniques are limited by the capabilities of their sources which are undergoing continual development.
The techniques using X-rays and electrons each have different advantages and disadvantages with regards to dynamic imaging and structural determination and they often give complementary information.

One proposed method for generating molecular movies involves using a high-brightness ultrashort duration beam source, such as an \gls{xfel} or future \gls{caes}, to perform \gls{cdi} on individual molecules~\cite{chapman_femtosecond_2006,dwyer_femtosecond_2006,gaffney_imaging_2007}.
The individual molecules would be dropped one-by-one in front of the illuminating bunches which would be short enough that imaging is completed before damage from the illumination affects the diffraction pattern (see Figure~\ref{figure:molecule_cdi}).
Dynamic processes, such as phase transitions, melting, and pumped vibrational modes, can be studied with this technique if the process can be triggered, say with a precisely timed laser pulse.
By varying the delay between triggering the process and imaging a `movie' can be constructed.
The random orientation of molecules as they are imaged can be managed algorithmically as long as there is sufficient brightness with each shot~\cite{yefanov_orientation_2013}.

\begin{figure}
    \center
    \includegraphics[width=0.65\linewidth]{0intro/Figs/single_molecule_cdi.jpg}
    \caption[Structural determination of single molecules.]{Structural determination of single molecules should be possible if a sufficiently bright and short pulse of X-rays is used image the molecule before damage affects the diffraction pattern produced. Electrons could be used in place of X-rays if a suitable source can be developed. Image adapted from Reference~\cite{gaffney_imaging_2007}}
    \label{figure:molecule_cdi}
\end{figure}

\Glspl{xfel} have been able to produce molecular movies~\cite{kupitz_structural_2016,pande_femtosecond_2016,nango_three-dimensional_2016} but there is still motivation for attempting to develop electron source alternatives.
Electrons interact much more strongly with matter than X-rays, with interactions having cross-sections $10^5$ to $10^6$ larger~\cite{sciaini_femtosecond_2011} and thus much fewer electrons are required per bunch than X-ray photons.
While \glspl{xfel} require a several kilometre long facility with a billion dollar price tag a hypothetical electron source with similar capabilities would likely fit in a room with a much lower cost: current \gls{caes} implementations easily fit on a single optical bench.

\subsection{Ultrafast Coherent Diffractive Imaging}

To resolve the structure of single molecules requires sufficient signal such that the orientation problem can be solved and averaging performed, and the imaging needs to be completed before the molecule is substantially damaged by the beam~\cite{huldt_diffraction_2003}.
Thus a single pulse must be extremely intense, in one scenario requiring $10^{12}$ \unit[8]{kV} X-ray pulses or $10^6$ \unit[3]{MeV} electrons, and extremely short in duration, 10s of femtoseconds~\cite{chapman_femtosecond_2006,spence_outrunning_2017}.
Damage to molecules can occur via numerous mechanisms and are different for X-rays and electrons but in both cases occur in approximately 10s of femtosecond timescales~\cite{spence_outrunning_2017}.

The coherence of the source is also an important consideration for diffraction imaging as the beam must have a coherence length as large as that of the molecule under consideration so that portions of the illuminating wave diffracted from different positions on the molecule interfere coherently.
When performing \gls{cdi} the diffraction pattern detected is the square of the Fourier coefficients that represents the molecule under examination.
Due to the loss of the complex phase components when the Fourier coefficients are squared it is not possible to directly invert to recover the structure of the molecules, this is known as the \emph{phase problem}~\cite{rodenburg_phase_1989}.
The most common solution to the phase problem is iterative computational phase-retrieval~\cite{chapman_coherent_2010}.

Ultrafast \gls{cdi} has been demonstrated has been demonstrated previously on micron-scale objects using an \gls{xfel}~\cite{chapman_femtosecond_2006}.

\subsection{Imaging Targets}

Crystallography has been of enormous benefit to biology as the structure of biological molecules plays a vital role in their function, with proteins interacting with sub-structures of other proteins to mediate biological processes, somewhat analogous to a lock and key.
Knowing the structure of biological proteins is essential to fully understanding how the mechanism that protein is involved in functions, and can provide vital information to researchers on how to produce drugs to manipulate that mechanism~\cite{aloy_structural_2006,almen_mapping_2009}.
The design and creation of drugs based on knowledge of the structure of the proteins involved in a mechanism is sometimes referred to as \emph{direct drug design} and has had some success to date~\cite{klebe_recent_2000,jhoti_structure-based_2007,mauser_recent_2008}.
The first example of structurally informed direct drug design was the drug dorzolamide which was released to the market in 1995 to reduce intraocular pressure in certain circumstances~\cite{greer_application_1994}.

One of the obvious restrictions on crystallography is that the target sample must be crystallisable in order for it to be imaged and unfortunately there are large numbers of important biological proteins that scientists have been unable to crystallise, notably a large portion of membrane proteins which mediate interactions on the surfaces of cells~\cite{geerlof_impact_2006}.
Alternative structure determination techniques such as ultrafast \gls{cdi} with an \gls{xfel} or \gls{caes} would be able to bypass the crystallisation requirement and thus provide researchers with a wealth of useful information.

Knowledge of the changes in molecular structures during many complex and interesting dynamics (such as melting, photosynthesis, molecular phase transitions, or chemical reactions) would prove invaluable to understanding the physics, chemistry and biology of many areas.

\section{Cold-atom electron and ion sources}

The research described by this thesis is part of an ongoing effort to develop an alternative source of electrons and ions that extracts the charged particles from a laser cooled atomic cloud.
Initially the aim of the project was to create ultrafast coherent electron bunches for use in diffraction imaging in a similar way to ultrafast X-ray pulses however it was later realised that \gls{caes} could also serve as an injection source for particle accelerators.
It has also become apparent that by simply reversing the polarity of the accelerator the source can generate high quality ion bunches with similar properties to the electron bunches thus providing a new source for use in ion microscopy and nano-fabrication.

Cold-atom sources operate by carefully ionising atoms in a \gls{mot} such that the resulting ions and electrons are cold and thus the bunches accelerated from those clouds have low emittance and high coherence.
A brief schematic of \gls{caeis} operation is shown in Figure~\ref{figure:simple_caeis_schem}.
The low transverse temperature of ions and electrons produced by a \gls{caeis} is one of the main advantages of this source.
The source also allows for the production of ultrafast, and arbitrarily shaped electron and ion bunches.
These techniques are applicable to any of the many atomic species that can be laser cooled and trapped.

\begin{figure}
    \center
    \includegraphics{0intro/Figs/simple_caeis_schem.pdf}
    \caption[Simplified cold atom ion and electron source schematic.]{Rubidium atoms are trapped and cooled in a \gls{mot} before being ionised with a red and blue laser. The electrons or ions are then accelerated by two electrodes forming the electron or ion bunches.}
    \label{figure:simple_caeis_schem}
\end{figure}

A \gls{caes} can be thought of as a photocathode electron source with the solid cathode replaced with an ultracold gas which provides the \gls{caes} with a few advantages such as the high quantum efficiency achievable, and near-threshold ionisation producing colder electrons than those from other photocathode sources~\cite{engelen_effective_2014}.
The simplicity of the interactions between photons and atoms allows for the high quantum efficiency to be achieved~\cite{baranov_field_1994}.

Another advantage of gas photocathodes is the lack of optically induced damage from high-intensity laser-fields.
Solid cathodes undergo constant degradation and require regular replacement~\cite{dowell_results_1995} whereas the gaseous target in a \gls{caes} is renewed with every bunch produced.

A major advantage of \glspl{caeis} as a electron or ion source is the low temperature of the source.
The low source temperature is due to the careful ionisation of the atoms trapped within the \gls{mot}.
The ultracold atoms trapped in the \gls{mot} have temperatures around \unit[100]{$\muup$K} but after ionisation the electrons have temperatures determined by the excess energy from the ionisation process~\cite{engelen_high-coherence_2013,engelen_analytical_2014,sparkes_high-coherence_2014,speirs_identification_2017}.
Precise control over the ionisation is possible to minimise the excess energy from the ionisation process and electrons produced can have transverse temperature as low as \unit[$\sim$10]{K}~\cite{saliba_spatial_2012}.
Unlike electrons, ions produced by the source have their initial temperature determined predominantly by the temperature of the trapped atoms before ionisation, \unit[100]{$\muup$K}, with an ion temperature around \unit[1]{mK}~\cite{debernardi_measurement_2011,murphy_detailed_2014}.

One of the most important figures of metrit for charged particle beam is beam emittance which can be directly correlated with the temperature of the particle source.
Emittance decribes the angular spread of particles within a beam or bunch and can be thought of as the ``focusability of the beam'' with low-emittance being preferable to high-emittance.
Due to the extremely low temperature \glspl{caeis} have enormous potential as a source for low-emittance electron and ion bunches.
Low-emittance bunches is also related to high-coherence which is an important parameter of sources for imaging.
The coherence length of a source must be at least as large as that of the sample under consideration for techniques such as \gls{cdi} or at least as large as the unit cell length for crystallographic diffraction techniques.
\Glspl{caeis} have already demonstrated impressive coherence lengths as large as those of some small biomolecules with further improvements expected~\cite{saliba_spatial_2012} .

The ionisation methods used in \glspl{caeis} involve a two-step ionisation process that first excites the atoms to an intermediate state with one laser followed by near-threshold ionisation of the excited atoms with a second laser frequency-tuned to minimise the excess ionisation energy.
This sophisticated ionisation scheme has a number usual benefits ontop of the low temperature ions and electrons produced, of particular interest are the ability to produce ultrashort duration bunches and the arbitrary bunch shaping.
\Glspl{caeis} have been shown to be able to produce electron bunches with duration less than \unit[130]{ps} (with expected bunch duration of a few tens of picoseconds)~\cite{speirs_identification_2017} and should be able to provide bunches suitable for compresssion to bunch durations of order \unit[100]{fs} suitable for ultrafast imaging~\cite{van_oudheusden_compression_2010}.

\Glspl{caeis} also have the ability to produce arbitrarily shaped electron or ion bunches by manipulting the profiles of the laser beams used to ionise the atoms~\cite{mcculloch_arbitrarily_2011}.
This capability is useful in a large number of ways but has particular potential as an avenue for production of uniform ellipsoidal to allow for reversal of beam-quality degradation by space-charge expansion~\cite{luiten_how_2004}.

{\color{red}
Thus, CDI, molecular movie.

Currently CAEIS can....

Next it needs.... I've done X, Y Z.}


\subsection{Ion Source}

By reversing the polarity of the acceleration stage of beam production in a \gls{caeis} a beam of ions can be produced instead of an electron beam.
Ion beams share the same advantages as the electron beams with bunches being shapeable, coherent and low-emittance~\cite{knuffman_cold_2013}.

\glsreset{fib}
\Glspl{fib} have a wide variety of applications such as high-resolution imaging~\cite{scipioni_helium_2008}, sample analysis and nanofabrication~\cite{khizroev_focused-ion-beam-based_2004}.
Liquid metal ion sources are the most commonly used \gls{fib} sources, usually using gallium ions due to it's simplicity and robustness despite gallium having a tendency to contaminate or destroy samples.
\Glspl{cais} promise to provide an attractive alternative to conventional sources as they are high brightness, low-emittance and are able to operate with a large range of ions, which could be selected to avoid sample contamination.

\Glspl{cais} have been used to demonstrate microscopy with lithium ions~\cite{knuffman_nanoscale_2011} and chromium ions~\cite{steele_focused_2010} and rubidium ion beams have been characterised and used to demonstrate the suppression of space-charge expansion~\cite{murphy_detailed_2014,thompson_suppression_2016}.


{\color{red}Introduce the rest of the thesis. Remind the reader of what they've read.}

