 \chapter{Introduction}
 
\pagenumbering{arabic}
\setcounter{page}{1}

The \gls{caes} at the University of Melbourne is an ongoing project that aims to develop a source capable of achieving the `holy grail' of scientific imaging, the molecular movie\cite{dwyer_femtosecond_2006}.
A \gls{caes} also has potential applications in ion microscopes, \glspl{fib} and as particle source for accelerators such as synchrotrons and \glspl{xfel}.

{\color{red}
Some kind of introduction for everything will go here.

Motivate CAES. Imaging of things is important. Membrane proteins, cystallisation. What else can we image with it? What else can we study with it? Ion source.

Briefly motivate part 1 and part 2.}

\section{Making the `Molecular Movie'}

\subsection{Imaging Targets}

Membrane proteins.
Chemical reactions.


\subsection{Requirements}

Ultra-fast, single-shot, orientation, flux, coherence.

\subsection{X-rays}

XFELs

\subsection{Electrons}

\subsection{Pump-Probe}

\section{Cold-atom electron sources}

\subsection{Developments}

Arbitrary shaping, charged partical dynamics, diffraction.
\subsection{Ion Source}

\section{Thesis Outline}

\subsection{Part I}

\subsection{Part II}
