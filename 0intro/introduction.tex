 \chapter{Introduction}
 
\pagenumbering{arabic}
\setcounter{page}{1}

Molecular imaging has provided science with great advances during it's history, such as X-ray crystallography determining the helical structure of DNA in 1953~\cite{watson_molecular_1953} and structure of myoglobin and haemoglobin in 1962~\cite{kendrew_x-ray_1957}.
The \gls{caeis} at the University of Melbourne is an ongoing project that aims to develop a source capable of achieving the `holy grail' of scientific imaging, the `molecular movie'~\cite{dwyer_femtosecond_2006,sciaini_femtosecond_2011}.
A \gls{caeis} also has potential applications with \gls{fib} techniques~\cite{mcclelland_bright_2016}, and as particle source for accelerators such as synchrotrons and \glspl{xfel} for \gls{cdi} of non-crystalline targets~\cite{van_oudheusden_electron_2007, zhu_future_2015, mcculloch_cold_2016}.



{\color{red}
Some kind of introduction for everything will go here.

Motivate CAES. Imaging of things is important. Membrane proteins, cystallisation. What else can we image with it? What else can we study with it? Ion source.

Briefly motivate part 1 and part 2.}

\section{Making the `Molecular Movie'}

Being able to observe molecular dynamics at an atomic spatial and temporal scale could provide great insight into some of the basic processes of biology and chemistry.
The fabled `molecular movie' refers to capturing the atomic motions of molecular systems as they undergo some transition such as a chemical reaction, protein folding or melting.
Molecular movies could provide greater understanding of important biological reactions such as photosynthesis or oxygen transport by haemoglobin.

One of the stepping stones towards molecular movies is single-shot imaging of non-crystalline molecules which would allow structural determination of membrane proteins, an essential step in rational drug design~\cite{hardy_atomic_1987, barrett_discovering_1999, pinto_influenza_1992}.
One of the main techniques in stuctural determination of crystalline targets is electron diffraction which, along with X-ray and neutron methods~\cite{cullity_elements_2001,bacon_x-ray_2013}, has be utilised in structural determination of many materials with atomic resolution.
Continuing research into real-space and diffraction imaging techniques is aiding in the understanding of a growing range of samples as well as providing new types of information on the sample under investigation.

Ultrafast imaging techniques are a relatively recent development and they provide the opportunity to study electronic and atomic dynamics.
Ultrafast techniques also provide an avenue for the imaging of structures that, to date, cannot be imaged by cyrstallographic techniques as some of these techniques should be applicable to uncrystallised samples with imaging being faster than damage processes~\cite{gaffney_imaging_2007,barty_ultrafast_2008,miao_beyond_2015}.
In the past the greatest success with structural determination has been acheived with crystallography which is limited by to samples that can be crystallised.
There are a large number of biological proteins of interest, particularly membrane proteins, that cannot be crystallised and developements in ultrafast imaging techniques is providing a route towards structural determination without crystallisation~\cite{dauter_current_2006,levitt_nature_2009}, cryogenic electron microscopy is also making headway in this area~\cite{henderson_model_1990,zhou_towards_2008}.
Both X-ray and electron imaging techniques are limited by the capabilities of their sources which are undergoing continual developement, in the realms of fundamental physics and engineering.
The techniques using X-rays and electrons each have different advantages and disadvantages with regards to dynamic imaging and structural determination and they often give complementary information.

Some recent results from X-ray sources are well on the way to producing molecular movies~\cite{pande_femtosecond_2016,nango_three-dimensional_2016}.

\subsection{Imaging Targets}

Membrane proteins.
Chemical reactions.


\subsection{Requirements}

Ultra-fast, single-shot, orientation, flux, coherence.

\subsection{X-rays}

XFELs

\subsection{Electrons}

\subsection{Pump-Probe}

\section{Cold-atom electron sources}

\subsection{Developments}

Arbitrary shaping, charged partical dynamics, diffraction.
\subsection{Ion Source}

\section{Thesis Outline}

\subsection{Part I}

\subsection{Part II}
