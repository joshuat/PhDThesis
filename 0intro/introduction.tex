 \chapter{Introduction}
 
\pagenumbering{arabic}
\setcounter{page}{1}

The \gls{caes} at the University of Melbourne is an ongoing project that aims to develop a source capable of achieving the `holy grail' of scientific imaging, the `molecular movie'~\cite{dwyer_femtosecond_2006}.
A \gls{caes} also has potential applications with \gls{fib} techniques~\cite{mcclelland_bright_2016} and as particle source for accelerators such as synchrotrons and \glspl{xfel} for \gls{cdi} of non-crystalline targets~\cite{van_oudheusden_electron_2007, zhu_future_2015, mcculloch_cold_2016}.

Molecular imaging has provided science with great advances during it's history, such as X-ray crystallography determining the helical structure of DNA in 1953~\cite{watson_molecular_1953} and structure of myoglobin and haemoglobin in 1962~\cite{kendrew_x-ray_1957}.

{\color{red}
Some kind of introduction for everything will go here.

Motivate CAES. Imaging of things is important. Membrane proteins, cystallisation. What else can we image with it? What else can we study with it? Ion source.

Briefly motivate part 1 and part 2.}

\section{Making the `Molecular Movie'}

Being able to observe molecular dynamics at an atomic spatial and temporal scale could provide great insight into some of the basic processes of biology and chemistry.
The fabled `molecular movie' refers to capturing the atomic motions of molecular systems as they undergo some transition such as a chemical reaction, protein folding or melting.
Molecular movies could provide greater understanding of important biological reactions such as photosynthesis or oxygen transport by haemoglobin.

One of the stepping stones towards molecular movies is single-shot imaging of non-crystalline molecules which would allow structural determination of membrane proteins, an essential step in rational drug design~\cite{hardy_atomic_1987, barrett_discovering_1999, pinto_influenza_1992}.

\subsection{Imaging Targets}

Membrane proteins.
Chemical reactions.


\subsection{Requirements}

Ultra-fast, single-shot, orientation, flux, coherence.

\subsection{X-rays}

XFELs

\subsection{Electrons}

\subsection{Pump-Probe}

\section{Cold-atom electron sources}

\subsection{Developments}

Arbitrary shaping, charged partical dynamics, diffraction.
\subsection{Ion Source}

\section{Thesis Outline}

\subsection{Part I}

\subsection{Part II}
